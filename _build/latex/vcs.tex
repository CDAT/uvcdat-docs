% Generated by Sphinx.
\def\sphinxdocclass{report}
\newif\ifsphinxKeepOldNames \sphinxKeepOldNamestrue
\documentclass[letterpaper,10pt,oneside]{sphinxmanual}
\usepackage{iftex}

\ifPDFTeX
  \usepackage[utf8]{inputenc}
\fi
\ifdefined\DeclareUnicodeCharacter
  \DeclareUnicodeCharacter{00A0}{\nobreakspace}
\fi
\usepackage{cmap}
\usepackage[T1]{fontenc}
\usepackage{amsmath,amssymb,amstext}
\usepackage{babel}
\usepackage{times}
\usepackage[Bjarne]{fncychap}
\usepackage{longtable}
\usepackage{sphinx}
\usepackage{multirow}
\usepackage{eqparbox}


\addto\captionsenglish{\renewcommand{\figurename}{Fig.\@ }}
\addto\captionsenglish{\renewcommand{\tablename}{Table }}
\SetupFloatingEnvironment{literal-block}{name=Listing }

\addto\extrasenglish{\def\pageautorefname{page}}

\setcounter{tocdepth}{2}


\title{VCS Documentation}
\date{Sep 26, 2016}
\release{2.4.1}
\author{AIMS Team}
\newcommand{\sphinxlogo}{\sphinxincludegraphics{uvcdat.png}\par}
\renewcommand{\releasename}{Release}
\makeindex

\makeatletter
\def\PYG@reset{\let\PYG@it=\relax \let\PYG@bf=\relax%
    \let\PYG@ul=\relax \let\PYG@tc=\relax%
    \let\PYG@bc=\relax \let\PYG@ff=\relax}
\def\PYG@tok#1{\csname PYG@tok@#1\endcsname}
\def\PYG@toks#1+{\ifx\relax#1\empty\else%
    \PYG@tok{#1}\expandafter\PYG@toks\fi}
\def\PYG@do#1{\PYG@bc{\PYG@tc{\PYG@ul{%
    \PYG@it{\PYG@bf{\PYG@ff{#1}}}}}}}
\def\PYG#1#2{\PYG@reset\PYG@toks#1+\relax+\PYG@do{#2}}

\expandafter\def\csname PYG@tok@gd\endcsname{\def\PYG@tc##1{\textcolor[rgb]{0.63,0.00,0.00}{##1}}}
\expandafter\def\csname PYG@tok@gu\endcsname{\let\PYG@bf=\textbf\def\PYG@tc##1{\textcolor[rgb]{0.50,0.00,0.50}{##1}}}
\expandafter\def\csname PYG@tok@gt\endcsname{\def\PYG@tc##1{\textcolor[rgb]{0.00,0.27,0.87}{##1}}}
\expandafter\def\csname PYG@tok@gs\endcsname{\let\PYG@bf=\textbf}
\expandafter\def\csname PYG@tok@gr\endcsname{\def\PYG@tc##1{\textcolor[rgb]{1.00,0.00,0.00}{##1}}}
\expandafter\def\csname PYG@tok@cm\endcsname{\let\PYG@it=\textit\def\PYG@tc##1{\textcolor[rgb]{0.25,0.50,0.56}{##1}}}
\expandafter\def\csname PYG@tok@vg\endcsname{\def\PYG@tc##1{\textcolor[rgb]{0.73,0.38,0.84}{##1}}}
\expandafter\def\csname PYG@tok@vi\endcsname{\def\PYG@tc##1{\textcolor[rgb]{0.73,0.38,0.84}{##1}}}
\expandafter\def\csname PYG@tok@mh\endcsname{\def\PYG@tc##1{\textcolor[rgb]{0.13,0.50,0.31}{##1}}}
\expandafter\def\csname PYG@tok@cs\endcsname{\def\PYG@tc##1{\textcolor[rgb]{0.25,0.50,0.56}{##1}}\def\PYG@bc##1{\setlength{\fboxsep}{0pt}\colorbox[rgb]{1.00,0.94,0.94}{\strut ##1}}}
\expandafter\def\csname PYG@tok@ge\endcsname{\let\PYG@it=\textit}
\expandafter\def\csname PYG@tok@vc\endcsname{\def\PYG@tc##1{\textcolor[rgb]{0.73,0.38,0.84}{##1}}}
\expandafter\def\csname PYG@tok@il\endcsname{\def\PYG@tc##1{\textcolor[rgb]{0.13,0.50,0.31}{##1}}}
\expandafter\def\csname PYG@tok@go\endcsname{\def\PYG@tc##1{\textcolor[rgb]{0.20,0.20,0.20}{##1}}}
\expandafter\def\csname PYG@tok@cp\endcsname{\def\PYG@tc##1{\textcolor[rgb]{0.00,0.44,0.13}{##1}}}
\expandafter\def\csname PYG@tok@gi\endcsname{\def\PYG@tc##1{\textcolor[rgb]{0.00,0.63,0.00}{##1}}}
\expandafter\def\csname PYG@tok@gh\endcsname{\let\PYG@bf=\textbf\def\PYG@tc##1{\textcolor[rgb]{0.00,0.00,0.50}{##1}}}
\expandafter\def\csname PYG@tok@ni\endcsname{\let\PYG@bf=\textbf\def\PYG@tc##1{\textcolor[rgb]{0.84,0.33,0.22}{##1}}}
\expandafter\def\csname PYG@tok@nl\endcsname{\let\PYG@bf=\textbf\def\PYG@tc##1{\textcolor[rgb]{0.00,0.13,0.44}{##1}}}
\expandafter\def\csname PYG@tok@nn\endcsname{\let\PYG@bf=\textbf\def\PYG@tc##1{\textcolor[rgb]{0.05,0.52,0.71}{##1}}}
\expandafter\def\csname PYG@tok@no\endcsname{\def\PYG@tc##1{\textcolor[rgb]{0.38,0.68,0.84}{##1}}}
\expandafter\def\csname PYG@tok@na\endcsname{\def\PYG@tc##1{\textcolor[rgb]{0.25,0.44,0.63}{##1}}}
\expandafter\def\csname PYG@tok@nb\endcsname{\def\PYG@tc##1{\textcolor[rgb]{0.00,0.44,0.13}{##1}}}
\expandafter\def\csname PYG@tok@nc\endcsname{\let\PYG@bf=\textbf\def\PYG@tc##1{\textcolor[rgb]{0.05,0.52,0.71}{##1}}}
\expandafter\def\csname PYG@tok@nd\endcsname{\let\PYG@bf=\textbf\def\PYG@tc##1{\textcolor[rgb]{0.33,0.33,0.33}{##1}}}
\expandafter\def\csname PYG@tok@ne\endcsname{\def\PYG@tc##1{\textcolor[rgb]{0.00,0.44,0.13}{##1}}}
\expandafter\def\csname PYG@tok@nf\endcsname{\def\PYG@tc##1{\textcolor[rgb]{0.02,0.16,0.49}{##1}}}
\expandafter\def\csname PYG@tok@si\endcsname{\let\PYG@it=\textit\def\PYG@tc##1{\textcolor[rgb]{0.44,0.63,0.82}{##1}}}
\expandafter\def\csname PYG@tok@s2\endcsname{\def\PYG@tc##1{\textcolor[rgb]{0.25,0.44,0.63}{##1}}}
\expandafter\def\csname PYG@tok@nt\endcsname{\let\PYG@bf=\textbf\def\PYG@tc##1{\textcolor[rgb]{0.02,0.16,0.45}{##1}}}
\expandafter\def\csname PYG@tok@nv\endcsname{\def\PYG@tc##1{\textcolor[rgb]{0.73,0.38,0.84}{##1}}}
\expandafter\def\csname PYG@tok@s1\endcsname{\def\PYG@tc##1{\textcolor[rgb]{0.25,0.44,0.63}{##1}}}
\expandafter\def\csname PYG@tok@ch\endcsname{\let\PYG@it=\textit\def\PYG@tc##1{\textcolor[rgb]{0.25,0.50,0.56}{##1}}}
\expandafter\def\csname PYG@tok@m\endcsname{\def\PYG@tc##1{\textcolor[rgb]{0.13,0.50,0.31}{##1}}}
\expandafter\def\csname PYG@tok@gp\endcsname{\let\PYG@bf=\textbf\def\PYG@tc##1{\textcolor[rgb]{0.78,0.36,0.04}{##1}}}
\expandafter\def\csname PYG@tok@sh\endcsname{\def\PYG@tc##1{\textcolor[rgb]{0.25,0.44,0.63}{##1}}}
\expandafter\def\csname PYG@tok@ow\endcsname{\let\PYG@bf=\textbf\def\PYG@tc##1{\textcolor[rgb]{0.00,0.44,0.13}{##1}}}
\expandafter\def\csname PYG@tok@sx\endcsname{\def\PYG@tc##1{\textcolor[rgb]{0.78,0.36,0.04}{##1}}}
\expandafter\def\csname PYG@tok@bp\endcsname{\def\PYG@tc##1{\textcolor[rgb]{0.00,0.44,0.13}{##1}}}
\expandafter\def\csname PYG@tok@c1\endcsname{\let\PYG@it=\textit\def\PYG@tc##1{\textcolor[rgb]{0.25,0.50,0.56}{##1}}}
\expandafter\def\csname PYG@tok@o\endcsname{\def\PYG@tc##1{\textcolor[rgb]{0.40,0.40,0.40}{##1}}}
\expandafter\def\csname PYG@tok@kc\endcsname{\let\PYG@bf=\textbf\def\PYG@tc##1{\textcolor[rgb]{0.00,0.44,0.13}{##1}}}
\expandafter\def\csname PYG@tok@c\endcsname{\let\PYG@it=\textit\def\PYG@tc##1{\textcolor[rgb]{0.25,0.50,0.56}{##1}}}
\expandafter\def\csname PYG@tok@mf\endcsname{\def\PYG@tc##1{\textcolor[rgb]{0.13,0.50,0.31}{##1}}}
\expandafter\def\csname PYG@tok@err\endcsname{\def\PYG@bc##1{\setlength{\fboxsep}{0pt}\fcolorbox[rgb]{1.00,0.00,0.00}{1,1,1}{\strut ##1}}}
\expandafter\def\csname PYG@tok@mb\endcsname{\def\PYG@tc##1{\textcolor[rgb]{0.13,0.50,0.31}{##1}}}
\expandafter\def\csname PYG@tok@ss\endcsname{\def\PYG@tc##1{\textcolor[rgb]{0.32,0.47,0.09}{##1}}}
\expandafter\def\csname PYG@tok@sr\endcsname{\def\PYG@tc##1{\textcolor[rgb]{0.14,0.33,0.53}{##1}}}
\expandafter\def\csname PYG@tok@mo\endcsname{\def\PYG@tc##1{\textcolor[rgb]{0.13,0.50,0.31}{##1}}}
\expandafter\def\csname PYG@tok@kd\endcsname{\let\PYG@bf=\textbf\def\PYG@tc##1{\textcolor[rgb]{0.00,0.44,0.13}{##1}}}
\expandafter\def\csname PYG@tok@mi\endcsname{\def\PYG@tc##1{\textcolor[rgb]{0.13,0.50,0.31}{##1}}}
\expandafter\def\csname PYG@tok@kn\endcsname{\let\PYG@bf=\textbf\def\PYG@tc##1{\textcolor[rgb]{0.00,0.44,0.13}{##1}}}
\expandafter\def\csname PYG@tok@cpf\endcsname{\let\PYG@it=\textit\def\PYG@tc##1{\textcolor[rgb]{0.25,0.50,0.56}{##1}}}
\expandafter\def\csname PYG@tok@kr\endcsname{\let\PYG@bf=\textbf\def\PYG@tc##1{\textcolor[rgb]{0.00,0.44,0.13}{##1}}}
\expandafter\def\csname PYG@tok@s\endcsname{\def\PYG@tc##1{\textcolor[rgb]{0.25,0.44,0.63}{##1}}}
\expandafter\def\csname PYG@tok@kp\endcsname{\def\PYG@tc##1{\textcolor[rgb]{0.00,0.44,0.13}{##1}}}
\expandafter\def\csname PYG@tok@w\endcsname{\def\PYG@tc##1{\textcolor[rgb]{0.73,0.73,0.73}{##1}}}
\expandafter\def\csname PYG@tok@kt\endcsname{\def\PYG@tc##1{\textcolor[rgb]{0.56,0.13,0.00}{##1}}}
\expandafter\def\csname PYG@tok@sc\endcsname{\def\PYG@tc##1{\textcolor[rgb]{0.25,0.44,0.63}{##1}}}
\expandafter\def\csname PYG@tok@sb\endcsname{\def\PYG@tc##1{\textcolor[rgb]{0.25,0.44,0.63}{##1}}}
\expandafter\def\csname PYG@tok@k\endcsname{\let\PYG@bf=\textbf\def\PYG@tc##1{\textcolor[rgb]{0.00,0.44,0.13}{##1}}}
\expandafter\def\csname PYG@tok@se\endcsname{\let\PYG@bf=\textbf\def\PYG@tc##1{\textcolor[rgb]{0.25,0.44,0.63}{##1}}}
\expandafter\def\csname PYG@tok@sd\endcsname{\let\PYG@it=\textit\def\PYG@tc##1{\textcolor[rgb]{0.25,0.44,0.63}{##1}}}

\def\PYGZbs{\char`\\}
\def\PYGZus{\char`\_}
\def\PYGZob{\char`\{}
\def\PYGZcb{\char`\}}
\def\PYGZca{\char`\^}
\def\PYGZam{\char`\&}
\def\PYGZlt{\char`\<}
\def\PYGZgt{\char`\>}
\def\PYGZsh{\char`\#}
\def\PYGZpc{\char`\%}
\def\PYGZdl{\char`\$}
\def\PYGZhy{\char`\-}
\def\PYGZsq{\char`\'}
\def\PYGZdq{\char`\"}
\def\PYGZti{\char`\~}
% for compatibility with earlier versions
\def\PYGZat{@}
\def\PYGZlb{[}
\def\PYGZrb{]}
\makeatother

\renewcommand\PYGZsq{\textquotesingle}

\begin{document}

\maketitle
\tableofcontents
\phantomsection\label{index::doc}



\chapter{VCS}
\label{vcs/vcs:module-vcs}\label{vcs/vcs:vcs}\label{vcs/vcs::doc}\label{vcs/vcs:cdat-api-documentation}\index{vcs (module)}

\section{VCS: Visualization and Control System}
\label{vcs/vcs:vcs-visualization-and-control-system}

\subsection{Authors}
\label{vcs/vcs:authors}
Creator: \href{http://computation.llnl.gov/about/our-people/highlights/dean-williams}{Dean Williams} (LLNL, AIMS Team)

Lead Developer: \href{https://github.com/doutriaux1}{Charles Doutriaux} (LLNL, AIMS Team)

Contributors: \url{https://github.com/UV-CDAT/uvcdat/graphs/contributors}

Support Email: \href{mailto:uvcdat-support@llnl.gov}{uvcdat-support@llnl.gov}

Project Site: \url{http://uvcdat.llnl.gov/}

Project Repo: \url{https://github.com/UV-CDAT/uvcdat/graphs/contributors}


\subsection{Description}
\label{vcs/vcs:charles-doutriaux}\label{vcs/vcs:description}
VCS is a visualization library for scientific data. It has a simple
model for defining a plot, that is decomposed into three parts:
\begin{enumerate}
\item {} 
\textbf{Data}: If it's iterable, we'll plot it... or at least try!
Currently we support numpy arrays, lists (nested and not),
and CDMS2 variables (there's some special support for metadata
from CDMS2 that gives some niceties in your plot, but it's not
mandatory).

\item {} 
\textbf{Graphics Method}: We have a variety of plot types that we
support out-of-the box; you can easily customize every aspect
of them to create the effect that you're looking for. If you can't,
we also support defining your own graphics methods, which you can
share with other users using standard python infrastructure (conda, pip).

\item {} 
\textbf{Template}: Templates control the appearance of everything that
\emph{isn't} your data. They position labels, control fonts, adjust borders,
place legends, and more. They're very flexible, and give the fine-grained
control of your plot that is needed for the truly perfect plot. Once you've
customized them, you can also save them out for later use, and distribute
them to other users.

\end{enumerate}
\index{init() (in module vcs)}

\begin{fulllineitems}
\phantomsection\label{vcs/vcs:vcs.init}\pysiglinewithargsret{\sphinxcode{vcs.}\sphinxbfcode{init}}{\emph{mode=1}, \emph{pause\_time=0}, \emph{call\_from\_gui=0}, \emph{size=None}, \emph{backend='vtk'}, \emph{geometry=None}, \emph{bg=None}}{}
Initialize and construct a VCS Canvas object.
\begin{quote}\begin{description}
\item[{Example}] \leavevmode
\end{description}\end{quote}

\begin{Verbatim}[commandchars=\\\{\}]
\PYG{k+kn}{import} \PYG{n+nn}{vcs}

\PYG{c+c1}{\PYGZsh{} Portrait orientation of 1 width per 2 height}
\PYG{n}{portrait} \PYG{o}{=} \PYG{n}{vcs}\PYG{o}{.}\PYG{n}{init}\PYG{p}{(}\PYG{n}{size}\PYG{o}{=}\PYG{o}{.}\PYG{l+m+mi}{5}\PYG{p}{)}
\PYG{c+c1}{\PYGZsh{} also accepts \PYGZdq{}usletter\PYGZdq{}}
\PYG{n}{letter} \PYG{o}{=} \PYG{n}{vcs}\PYG{o}{.}\PYG{n}{init}\PYG{p}{(}\PYG{n}{size}\PYG{o}{=}\PYG{l+s+s2}{\PYGZdq{}}\PYG{l+s+s2}{letter}\PYG{l+s+s2}{\PYGZdq{}}\PYG{p}{)}
\PYG{n}{a4} \PYG{o}{=} \PYG{n}{vcs}\PYG{o}{.}\PYG{n}{init}\PYG{p}{(}\PYG{n}{size}\PYG{o}{=}\PYG{l+s+s2}{\PYGZdq{}}\PYG{l+s+s2}{a4}\PYG{l+s+s2}{\PYGZdq{}}\PYG{p}{)}

\PYG{k+kn}{import} \PYG{n+nn}{vtk}
\PYG{c+c1}{\PYGZsh{} Useful for embedding VCS inside another application}
\PYG{n}{my\PYGZus{}win} \PYG{o}{=} \PYG{n}{vtk}\PYG{o}{.}\PYG{n}{vtkRenderWindow}\PYG{p}{(}\PYG{p}{)}
\PYG{n}{embedded} \PYG{o}{=} \PYG{n}{vcs}\PYG{o}{.}\PYG{n}{init}\PYG{p}{(}\PYG{n}{backend}\PYG{o}{=}\PYG{n}{my\PYGZus{}win}\PYG{p}{)}

\PYG{n}{dict\PYGZus{}init} \PYG{o}{=} \PYG{n}{vcs}\PYG{o}{.}\PYG{n}{init}\PYG{p}{(}\PYG{n}{geometry}\PYG{o}{=}\PYG{p}{\PYGZob{}}\PYG{l+s+s2}{\PYGZdq{}}\PYG{l+s+s2}{width}\PYG{l+s+s2}{\PYGZdq{}}\PYG{p}{:} \PYG{l+m+mi}{1200}\PYG{p}{,} \PYG{l+s+s2}{\PYGZdq{}}\PYG{l+s+s2}{height}\PYG{l+s+s2}{\PYGZdq{}}\PYG{p}{:} \PYG{l+m+mi}{600}\PYG{p}{\PYGZcb{}}\PYG{p}{)}
\PYG{n}{tuple\PYGZus{}init} \PYG{o}{=} \PYG{n}{vcs}\PYG{o}{.}\PYG{n}{init}\PYG{p}{(}\PYG{n}{geometry}\PYG{o}{=}\PYG{p}{(}\PYG{l+m+mi}{1200}\PYG{p}{,} \PYG{l+m+mi}{600}\PYG{p}{)}\PYG{p}{)}

\PYG{n}{bg\PYGZus{}canvas} \PYG{o}{=} \PYG{n}{vcs}\PYG{o}{.}\PYG{n}{init}\PYG{p}{(}\PYG{n}{bg}\PYG{o}{=}\PYG{k+kc}{True}\PYG{p}{)}
\end{Verbatim}
\begin{quote}\begin{description}
\item[{Parameters}] \leavevmode\begin{itemize}
\item {} 
\textbf{\texttt{size}} (\emph{\texttt{float or case-insensitive str}}) -- Aspect ratio for canvas (width / height)

\item {} 
\textbf{\texttt{backend}} (str, \sphinxtitleref{vtk.vtkRenderWindow}) -- Which VCS backend to use

\item {} 
\textbf{\texttt{geometry}} (\emph{\texttt{dict or tuple}}) -- Size (in pixels) you want the canvas to be.

\item {} 
\textbf{\texttt{bg}} (\href{https://docs.python.org/2/library/functions.html\#bool}{\emph{\texttt{bool}}}) -- Initialize a canvas to render in ``background'' mode (without displaying a window)

\end{itemize}

\item[{Returns}] \leavevmode
an initialized canvas

\item[{Return type}] \leavevmode
{\hyperref[vcs/Canvas:vcs.Canvas.Canvas]{\sphinxcrossref{vcs.Canvas.Canvas}}}

\end{description}\end{quote}

\end{fulllineitems}



\section{Canvas}
\label{vcs/Canvas:module-vcs.Canvas}\label{vcs/Canvas::doc}\label{vcs/Canvas:canvas}\index{vcs.Canvas (module)}\begin{description}
\item[{Canvas}] \leavevmode
The object onto which all plots are drawn.

Usually created using {\hyperref[vcs/vcs:vcs.init]{\sphinxcrossref{\sphinxcode{vcs.init()}}}}, this object provides easy access
to the functionality of the entire VCS module.

\end{description}
\index{Canvas (class in vcs.Canvas)}

\begin{fulllineitems}
\phantomsection\label{vcs/Canvas:vcs.Canvas.Canvas}\pysiglinewithargsret{\sphinxstrong{class }\sphinxcode{vcs.Canvas.}\sphinxbfcode{Canvas}}{\emph{mode=1}, \emph{pause\_time=0}, \emph{call\_from\_gui=0}, \emph{size=None}, \emph{backend='vtk'}, \emph{geometry=None}, \emph{bg=None}}{}
The object onto which all plots are drawn.

Usually created using {\hyperref[vcs/vcs:vcs.init]{\sphinxcrossref{\sphinxcode{vcs.init()}}}}, this object provides easy access
to the functionality of the entire VCS module.
\index{addfont() (vcs.Canvas.Canvas method)}

\begin{fulllineitems}
\phantomsection\label{vcs/Canvas:vcs.Canvas.Canvas.addfont}\pysiglinewithargsret{\sphinxbfcode{addfont}}{\emph{path}, \emph{name='`}}{}
Add a font to VCS.
\begin{quote}\begin{description}
\item[{Parameters}] \leavevmode\begin{itemize}
\item {} 
\textbf{\texttt{path}} (\href{https://docs.python.org/2/library/functions.html\#str}{\emph{\texttt{str}}}) -- Path to the font file you wish to add (must be .ttf)

\item {} 
\textbf{\texttt{name}} (\href{https://docs.python.org/2/library/functions.html\#str}{\emph{\texttt{str}}}) -- Name to use to represent the font.

\end{itemize}

\end{description}\end{quote}

\end{fulllineitems}

\index{boxfill() (vcs.Canvas.Canvas method)}

\begin{fulllineitems}
\phantomsection\label{vcs/Canvas:vcs.Canvas.Canvas.boxfill}\pysiglinewithargsret{\sphinxbfcode{boxfill}}{\emph{*args}, \emph{**parms}}{}~\begin{quote}

Generate a boxfill plot given the data, boxfill graphics method, and
template. If no boxfill class object is given, then the `default' boxfill
graphics method is used. Similarly, if no template class object is given,
then the `default' template is used.
\begin{quote}\begin{description}
\item[{Example}] \leavevmode
\begin{Verbatim}[commandchars=\\\{\}]
\PYG{g+gp}{\PYGZgt{}\PYGZgt{}\PYGZgt{} }\PYG{n}{a}\PYG{o}{=}\PYG{n}{vcs}\PYG{o}{.}\PYG{n}{init}\PYG{p}{(}\PYG{p}{)}
\PYG{g+gp}{\PYGZgt{}\PYGZgt{}\PYGZgt{} }\PYG{n}{a}\PYG{o}{.}\PYG{n}{show}\PYG{p}{(}\PYG{l+s+s1}{\PYGZsq{}}\PYG{l+s+s1}{boxfill}\PYG{l+s+s1}{\PYGZsq{}}\PYG{p}{)} \PYG{c+c1}{\PYGZsh{} Show all the existing boxfill graphics methods}
\PYG{g+go}{*******************Boxfill Names List**********************}
\PYG{g+gp}{...}
\PYG{g+go}{*******************End Boxfill Names List**********************}
\PYG{g+gp}{\PYGZgt{}\PYGZgt{}\PYGZgt{} }\PYG{n}{box}\PYG{o}{=}\PYG{n}{a}\PYG{o}{.}\PYG{n}{getboxfill}\PYG{p}{(}\PYG{l+s+s1}{\PYGZsq{}}\PYG{l+s+s1}{quick}\PYG{l+s+s1}{\PYGZsq{}}\PYG{p}{)} \PYG{c+c1}{\PYGZsh{} Create instance of \PYGZsq{}quick\PYGZsq{}}
\PYG{g+gp}{\PYGZgt{}\PYGZgt{}\PYGZgt{} }\PYG{n}{array}\PYG{o}{=}\PYG{p}{[}\PYG{n+nb}{range}\PYG{p}{(}\PYG{l+m+mi}{10}\PYG{p}{)} \PYG{k}{for} \PYG{n}{\PYGZus{}} \PYG{o+ow}{in} \PYG{n+nb}{range}\PYG{p}{(}\PYG{l+m+mi}{10}\PYG{p}{)}\PYG{p}{]}
\PYG{g+gp}{\PYGZgt{}\PYGZgt{}\PYGZgt{} }\PYG{n}{a}\PYG{o}{.}\PYG{n}{boxfill}\PYG{p}{(}\PYG{n}{array}\PYG{p}{,} \PYG{n}{box}\PYG{p}{)} \PYG{c+c1}{\PYGZsh{} Plot array using specified box and default template}
\PYG{g+go}{\PYGZlt{}vcs.displayplot.Dp ...\PYGZgt{}}
\PYG{g+gp}{\PYGZgt{}\PYGZgt{}\PYGZgt{} }\PYG{n}{template} \PYG{o}{=} \PYG{n}{a}\PYG{o}{.}\PYG{n}{gettemplate}\PYG{p}{(}\PYG{l+s+s1}{\PYGZsq{}}\PYG{l+s+s1}{quick}\PYG{l+s+s1}{\PYGZsq{}}\PYG{p}{)} \PYG{c+c1}{\PYGZsh{} get quick template}
\PYG{g+gp}{\PYGZgt{}\PYGZgt{}\PYGZgt{} }\PYG{n}{a}\PYG{o}{.}\PYG{n}{clear}\PYG{p}{(}\PYG{p}{)} \PYG{c+c1}{\PYGZsh{} Clear VCS canvas}
\PYG{g+gp}{\PYGZgt{}\PYGZgt{}\PYGZgt{} }\PYG{n}{a}\PYG{o}{.}\PYG{n}{boxfill}\PYG{p}{(}\PYG{n}{array}\PYG{p}{,} \PYG{n}{box}\PYG{p}{,} \PYG{n}{template}\PYG{p}{)} \PYG{c+c1}{\PYGZsh{} Plot array using specified box and template}
\PYG{g+go}{\PYGZlt{}vcs.displayplot.Dp ...\PYGZgt{}}
\PYG{g+gp}{\PYGZgt{}\PYGZgt{}\PYGZgt{} }\PYG{n}{a}\PYG{o}{.}\PYG{n}{boxfill}\PYG{p}{(}\PYG{n}{box}\PYG{p}{,} \PYG{n}{array}\PYG{p}{,} \PYG{n}{template}\PYG{p}{)} \PYG{c+c1}{\PYGZsh{} Plot array using specified box and template}
\PYG{g+go}{\PYGZlt{}vcs.displayplot.Dp ...\PYGZgt{}}
\PYG{g+gp}{\PYGZgt{}\PYGZgt{}\PYGZgt{} }\PYG{n}{a}\PYG{o}{.}\PYG{n}{boxfill}\PYG{p}{(}\PYG{n}{template}\PYG{p}{,} \PYG{n}{array}\PYG{p}{,} \PYG{n}{box}\PYG{p}{)} \PYG{c+c1}{\PYGZsh{} Plot array using specified box and template}
\PYG{g+go}{\PYGZlt{}vcs.displayplot.Dp ...\PYGZgt{}}
\PYG{g+gp}{\PYGZgt{}\PYGZgt{}\PYGZgt{} }\PYG{n}{a}\PYG{o}{.}\PYG{n}{boxfill}\PYG{p}{(}\PYG{n}{template}\PYG{p}{,} \PYG{n}{box}\PYG{p}{,} \PYG{n}{array}\PYG{p}{)} \PYG{c+c1}{\PYGZsh{} Plot array using specified box and template}
\PYG{g+go}{\PYGZlt{}vcs.displayplot.Dp ...\PYGZgt{}}
\PYG{g+gp}{\PYGZgt{}\PYGZgt{}\PYGZgt{} }\PYG{n}{a}\PYG{o}{.}\PYG{n}{boxfill}\PYG{p}{(}\PYG{n}{array}\PYG{p}{,} \PYG{l+s+s1}{\PYGZsq{}}\PYG{l+s+s1}{hovmuller}\PYG{l+s+s1}{\PYGZsq{}}\PYG{p}{,} \PYG{l+s+s1}{\PYGZsq{}}\PYG{l+s+s1}{quick}\PYG{l+s+s1}{\PYGZsq{}}\PYG{p}{)} \PYG{c+c1}{\PYGZsh{} Use \PYGZsq{}hovmuller\PYGZsq{} template and \PYGZsq{}quick\PYGZsq{} boxfill}
\PYG{g+go}{\PYGZlt{}vcs.displayplot.Dp ...\PYGZgt{}}
\PYG{g+gp}{\PYGZgt{}\PYGZgt{}\PYGZgt{} }\PYG{n}{a}\PYG{o}{.}\PYG{n}{boxfill}\PYG{p}{(}\PYG{l+s+s1}{\PYGZsq{}}\PYG{l+s+s1}{hovmuller}\PYG{l+s+s1}{\PYGZsq{}}\PYG{p}{,} \PYG{n}{array}\PYG{p}{,} \PYG{l+s+s1}{\PYGZsq{}}\PYG{l+s+s1}{quick}\PYG{l+s+s1}{\PYGZsq{}}\PYG{p}{)} \PYG{c+c1}{\PYGZsh{} Use \PYGZsq{}hovmuller\PYGZsq{} template and \PYGZsq{}quick\PYGZsq{} boxfill}
\PYG{g+go}{\PYGZlt{}vcs.displayplot.Dp ...\PYGZgt{}}
\PYG{g+gp}{\PYGZgt{}\PYGZgt{}\PYGZgt{} }\PYG{n}{a}\PYG{o}{.}\PYG{n}{boxfill}\PYG{p}{(}\PYG{l+s+s1}{\PYGZsq{}}\PYG{l+s+s1}{hovmuller}\PYG{l+s+s1}{\PYGZsq{}}\PYG{p}{,} \PYG{l+s+s1}{\PYGZsq{}}\PYG{l+s+s1}{quick}\PYG{l+s+s1}{\PYGZsq{}}\PYG{p}{,} \PYG{n}{array}\PYG{p}{)} \PYG{c+c1}{\PYGZsh{} Use \PYGZsq{}hovmuller template and \PYGZsq{}quick\PYGZsq{} boxfill}
\PYG{g+go}{\PYGZlt{}vcs.displayplot.Dp ...\PYGZgt{}}
\end{Verbatim}

\begin{notice}{note}{Note:}
As shown above, the array, `template', and `box' parameters can be provided in any order.
The `template' and `box' parameters can either be VCS template and boxfill objects,
or string names of template and boxfill objects.
\end{notice}

\end{description}\end{quote}
\end{quote}
\begin{quote}\begin{description}
\item[{Parameters}] \leavevmode\begin{itemize}
\item {} 
\textbf{\texttt{xaxis}} (\emph{\texttt{cdms2.axis.TransientAxis}}) -- Axis object to replace the slab -1 dim axis

\item {} 
\textbf{\texttt{yaxis}} (\emph{\texttt{cdms2.axis.TransientAxis}}) -- Axis object to replace the slab -2 dim axis, only if slab has more than 1D

\item {} 
\textbf{\texttt{zaxis}} (\emph{\texttt{cdms2.axis.TransientAxis}}) -- Axis object to replace the slab -3 dim axis, only if slab has more than 2D

\item {} 
\textbf{\texttt{taxis}} (\emph{\texttt{cdms2.axis.TransientAxis}}) -- Axis object to replace the slab -4 dim axis, only if slab has more than 3D

\item {} 
\textbf{\texttt{waxis}} (\emph{\texttt{cdms2.axis.TransientAxis}}) -- Axis object to replace the slab -5 dim axis, only if slab has more than 4D

\item {} 
\textbf{\texttt{xrev}} (\href{https://docs.python.org/2/library/functions.html\#bool}{\emph{\texttt{bool}}}) -- reverse x axis

\item {} 
\textbf{\texttt{yrev}} (\href{https://docs.python.org/2/library/functions.html\#bool}{\emph{\texttt{bool}}}) -- reverse y axis, only if slab has more than 1D

\item {} 
\textbf{\texttt{xarray}} (\href{https://docs.python.org/2/library/array.html\#module-array}{\emph{\texttt{array}}}) -- Values to use instead of x axis

\item {} 
\textbf{\texttt{yarray}} (\href{https://docs.python.org/2/library/array.html\#module-array}{\emph{\texttt{array}}}) -- Values to use instead of y axis, only if var has more than 1D

\item {} 
\textbf{\texttt{zarray}} (\href{https://docs.python.org/2/library/array.html\#module-array}{\emph{\texttt{array}}}) -- Values to use instead of z axis, only if var has more than 2D

\item {} 
\textbf{\texttt{tarray}} (\href{https://docs.python.org/2/library/array.html\#module-array}{\emph{\texttt{array}}}) -- Values to use instead of t axis, only if var has more than 3D

\item {} 
\textbf{\texttt{warray}} (\href{https://docs.python.org/2/library/array.html\#module-array}{\emph{\texttt{array}}}) -- Values to use instead of w axis, only if var has more than 4D

\item {} 
\textbf{\texttt{continents}} (\href{https://docs.python.org/2/library/functions.html\#int}{\emph{\texttt{int}}}) -- continents type number

\item {} 
\textbf{\texttt{name}} (\href{https://docs.python.org/2/library/functions.html\#str}{\emph{\texttt{str}}}) -- replaces variable name on plot

\item {} 
\textbf{\texttt{time}} (\emph{\texttt{A cdtime object}}) -- replaces time name on plot

\item {} 
\textbf{\texttt{units}} (\href{https://docs.python.org/2/library/functions.html\#str}{\emph{\texttt{str}}}) -- replaces units value on plot

\item {} 
\textbf{\texttt{ymd}} (\href{https://docs.python.org/2/library/functions.html\#str}{\emph{\texttt{str}}}) -- replaces year/month/day on plot

\item {} 
\textbf{\texttt{hms}} (\href{https://docs.python.org/2/library/functions.html\#str}{\emph{\texttt{str}}}) -- replaces hh/mm/ss on plot

\item {} 
\textbf{\texttt{file\_comment}} (\href{https://docs.python.org/2/library/functions.html\#str}{\emph{\texttt{str}}}) -- replaces file\_comment on plot

\item {} 
\textbf{\texttt{xbounds}} (\href{https://docs.python.org/2/library/array.html\#module-array}{\emph{\texttt{array}}}) -- Values to use instead of x axis bounds values

\item {} 
\textbf{\texttt{ybounds}} (\href{https://docs.python.org/2/library/array.html\#module-array}{\emph{\texttt{array}}}) -- Values to use instead of y axis bounds values (if exist)

\item {} 
\textbf{\texttt{xname}} (\href{https://docs.python.org/2/library/functions.html\#str}{\emph{\texttt{str}}}) -- replace xaxis name on plot

\item {} 
\textbf{\texttt{yname}} (\href{https://docs.python.org/2/library/functions.html\#str}{\emph{\texttt{str}}}) -- replace yaxis name on plot (if exists)

\item {} 
\textbf{\texttt{zname}} (\href{https://docs.python.org/2/library/functions.html\#str}{\emph{\texttt{str}}}) -- replace zaxis name on plot (if exists)

\item {} 
\textbf{\texttt{tname}} (\href{https://docs.python.org/2/library/functions.html\#str}{\emph{\texttt{str}}}) -- replace taxis name on plot (if exists)

\item {} 
\textbf{\texttt{wname}} (\href{https://docs.python.org/2/library/functions.html\#str}{\emph{\texttt{str}}}) -- replace waxis name on plot (if exists)

\item {} 
\textbf{\texttt{xunits}} (\href{https://docs.python.org/2/library/functions.html\#str}{\emph{\texttt{str}}}) -- replace xaxis units on plot

\item {} 
\textbf{\texttt{yunits}} (\href{https://docs.python.org/2/library/functions.html\#str}{\emph{\texttt{str}}}) -- replace yaxis units on plot (if exists)

\item {} 
\textbf{\texttt{zunits}} (\href{https://docs.python.org/2/library/functions.html\#str}{\emph{\texttt{str}}}) -- replace zaxis units on plot (if exists)

\item {} 
\textbf{\texttt{tunits}} (\href{https://docs.python.org/2/library/functions.html\#str}{\emph{\texttt{str}}}) -- replace taxis units on plot (if exists)

\item {} 
\textbf{\texttt{wunits}} (\href{https://docs.python.org/2/library/functions.html\#str}{\emph{\texttt{str}}}) -- replace waxis units on plot (if exists)

\item {} 
\textbf{\texttt{xweights}} (\href{https://docs.python.org/2/library/array.html\#module-array}{\emph{\texttt{array}}}) -- replace xaxis weights used for computing mean

\item {} 
\textbf{\texttt{yweights}} (\href{https://docs.python.org/2/library/array.html\#module-array}{\emph{\texttt{array}}}) -- replace xaxis weights used for computing mean

\item {} 
\textbf{\texttt{comment1}} (\href{https://docs.python.org/2/library/functions.html\#str}{\emph{\texttt{str}}}) -- replaces comment1 on plot

\item {} 
\textbf{\texttt{comment2}} (\href{https://docs.python.org/2/library/functions.html\#str}{\emph{\texttt{str}}}) -- replaces comment2 on plot

\item {} 
\textbf{\texttt{comment3}} (\href{https://docs.python.org/2/library/functions.html\#str}{\emph{\texttt{str}}}) -- replaces comment3 on plot

\item {} 
\textbf{\texttt{comment4}} (\href{https://docs.python.org/2/library/functions.html\#str}{\emph{\texttt{str}}}) -- replaces comment4 on plot

\item {} 
\textbf{\texttt{long\_name}} (\href{https://docs.python.org/2/library/functions.html\#str}{\emph{\texttt{str}}}) -- replaces long\_name on plot

\item {} 
\textbf{\texttt{grid}} (\emph{\texttt{cdms2.grid.TransientRectGrid}}) -- replaces array grid (if exists)

\item {} 
\textbf{\texttt{bg}} (\emph{\texttt{bool/int}}) -- plots in background mode

\item {} 
\textbf{\texttt{ratio}} (\index{xmtics1 (vcs.Canvas.Canvas attribute)}\index{xmtics2 (vcs.Canvas.Canvas attribute)}\index{ymtics1 (vcs.Canvas.Canvas attribute)}\index{ymtics2 (vcs.Canvas.Canvas attribute)}\index{xticlabels1 (vcs.Canvas.Canvas attribute)}\index{xticlabels2 (vcs.Canvas.Canvas attribute)}\index{yticlabels1 (vcs.Canvas.Canvas attribute)}\index{yticlabels2 (vcs.Canvas.Canvas attribute)}\index{projection (vcs.Canvas.Canvas attribute)}\index{datawc\_x1 (vcs.Canvas.Canvas attribute)}\index{datawc\_x2 (vcs.Canvas.Canvas attribute)}\index{datawc\_y1 (vcs.Canvas.Canvas attribute)}\index{datawc\_y2 (vcs.Canvas.Canvas attribute)}\index{datawc\_timeunits (vcs.Canvas.Canvas attribute)}\index{datawc\_calendar (vcs.Canvas.Canvas attribute)}) -- sets the y/x ratio ,if passed as a string with `t' at the end, will aslo moves the ticks

\item {} 
\textbf{\texttt{xaxisconvert}} (\href{https://docs.python.org/2/library/functions.html\#str}{\emph{\texttt{str}}}) -- (Ex: `linear') converting xaxis linear/log/log10/ln/exp/area\_wt

\item {} 
\textbf{\texttt{yaxisconvert}} (\href{https://docs.python.org/2/library/functions.html\#str}{\emph{\texttt{str}}}) -- (Ex: `linear') converting yaxis linear/log/log10/ln/exp/area\_wt

\item {} 
\textbf{\texttt{slab}} (\href{https://docs.python.org/2/library/array.html\#module-array}{\emph{\texttt{array}}}) -- (Ex: {[}{[}0, 1{]}{]}) Data at least 2D, last 2 dimensions will be plotted

\end{itemize}

\item[{Returns}] \leavevmode
Display Plot object representing the plot.

\item[{Return type}] \leavevmode
{\hyperref[vcs/misc/displayplot:vcs.displayplot.Dp]{\sphinxcrossref{vcs.displayplot.Dp}}}

\end{description}\end{quote}

\end{fulllineitems}

\index{canvasid() (vcs.Canvas.Canvas method)}

\begin{fulllineitems}
\phantomsection\label{vcs/Canvas:vcs.Canvas.Canvas.canvasid}\pysiglinewithargsret{\sphinxbfcode{canvasid}}{\emph{*args}}{}
Get the ID of this canvas.

This ID number is found at the top of the VCS Canvas, as part of its title.

\end{fulllineitems}

\index{canvasinfo() (vcs.Canvas.Canvas method)}

\begin{fulllineitems}
\phantomsection\label{vcs/Canvas:vcs.Canvas.Canvas.canvasinfo}\pysiglinewithargsret{\sphinxbfcode{canvasinfo}}{\emph{*args}, \emph{**kargs}}{}
Obtain the current attributes of the VCS Canvas window.
\begin{quote}\begin{description}
\item[{Returns}] \leavevmode
Dictionary with keys: ``mapstate'' (whether the canvas is opened), ``height'', ``width'', ``depth'', ``x'', ``y''

\item[{Return type}] \leavevmode
\href{https://docs.python.org/2/library/stdtypes.html\#dict}{dict}

\end{description}\end{quote}

\end{fulllineitems}

\index{cgm() (vcs.Canvas.Canvas method)}

\begin{fulllineitems}
\phantomsection\label{vcs/Canvas:vcs.Canvas.Canvas.cgm}\pysiglinewithargsret{\sphinxbfcode{cgm}}{\emph{file}, \emph{mode='w'}}{}
Export an image in CGM format.
\begin{quote}\begin{description}
\item[{Parameters}] \leavevmode\begin{itemize}
\item {} 
\textbf{\texttt{file}} -- Filename to save

\item {} 
\textbf{\texttt{mode}} -- Ignored.

\end{itemize}

\end{description}\end{quote}

\end{fulllineitems}

\index{change\_display\_graphic\_method() (vcs.Canvas.Canvas method)}

\begin{fulllineitems}
\phantomsection\label{vcs/Canvas:vcs.Canvas.Canvas.change_display_graphic_method}\pysiglinewithargsret{\sphinxbfcode{change\_display\_graphic\_method}}{\emph{display}, \emph{type}, \emph{name}}{}
Changes the type and graphic method of a plot.
\begin{quote}\begin{description}
\item[{Parameters}] \leavevmode\begin{itemize}
\item {} 
\textbf{\texttt{display}} (\emph{\texttt{str or vcs.displayplot.Dp}}) -- Display to change.

\item {} 
\textbf{\texttt{type}} (\href{https://docs.python.org/2/library/functions.html\#str}{\emph{\texttt{str}}}) -- New graphics method type.

\item {} 
\textbf{\texttt{name}} (\href{https://docs.python.org/2/library/functions.html\#str}{\emph{\texttt{str}}}) -- Name of new graphics method.

\end{itemize}

\end{description}\end{quote}

\end{fulllineitems}

\index{check\_name\_source() (vcs.Canvas.Canvas method)}

\begin{fulllineitems}
\phantomsection\label{vcs/Canvas:vcs.Canvas.Canvas.check_name_source}\pysiglinewithargsret{\sphinxbfcode{check\_name\_source}}{\emph{name}, \emph{source}, \emph{typ}}{}
make sure it is a unique name for this type or generates a name for user

\end{fulllineitems}

\index{clean\_auto\_generated\_objects() (vcs.Canvas.Canvas method)}

\begin{fulllineitems}
\phantomsection\label{vcs/Canvas:vcs.Canvas.Canvas.clean_auto_generated_objects}\pysiglinewithargsret{\sphinxbfcode{clean\_auto\_generated\_objects}}{\emph{type=None}}{}
Cleans up all automatically generated VCS objects.

This function will delete all references to objects that
VCS created automatically in response to user actions but are
no longer in use. This shouldn't be necessary most of the time,
but if you're running into performance/memory issues, calling it
periodically may help.
\begin{quote}\begin{description}
\item[{Parameters}] \leavevmode
\textbf{\texttt{type}} (\emph{\texttt{None, str, list/tuple (of str)}}) -- Type of objects to remove. By default, will remove everything.

\end{description}\end{quote}

\end{fulllineitems}

\index{clear() (vcs.Canvas.Canvas method)}

\begin{fulllineitems}
\phantomsection\label{vcs/Canvas:vcs.Canvas.Canvas.clear}\pysiglinewithargsret{\sphinxbfcode{clear}}{\emph{*args}, \emph{**kargs}}{}
Clears all the VCS displays on a page (i.e., the VCS Canvas object).
\begin{quote}\begin{description}
\item[{Example}] \leavevmode
\begin{Verbatim}[commandchars=\\\{\}]
\PYG{g+gp}{\PYGZgt{}\PYGZgt{}\PYGZgt{} }\PYG{n}{a}\PYG{o}{=}\PYG{n}{vcs}\PYG{o}{.}\PYG{n}{init}\PYG{p}{(}\PYG{p}{)}
\PYG{g+gp}{\PYGZgt{}\PYGZgt{}\PYGZgt{} }\PYG{n}{array} \PYG{o}{=} \PYG{p}{[}\PYG{n+nb}{range}\PYG{p}{(}\PYG{l+m+mi}{1}\PYG{p}{,} \PYG{l+m+mi}{11}\PYG{p}{)} \PYG{k}{for} \PYG{n}{\PYGZus{}} \PYG{o+ow}{in} \PYG{n+nb}{range}\PYG{p}{(}\PYG{l+m+mi}{1}\PYG{p}{,} \PYG{l+m+mi}{11}\PYG{p}{)}\PYG{p}{]}
\PYG{g+gp}{\PYGZgt{}\PYGZgt{}\PYGZgt{} }\PYG{n}{a}\PYG{o}{.}\PYG{n}{plot}\PYG{p}{(}\PYG{n}{array}\PYG{p}{,}\PYG{l+s+s1}{\PYGZsq{}}\PYG{l+s+s1}{default}\PYG{l+s+s1}{\PYGZsq{}}\PYG{p}{,}\PYG{l+s+s1}{\PYGZsq{}}\PYG{l+s+s1}{isofill}\PYG{l+s+s1}{\PYGZsq{}}\PYG{p}{,}\PYG{l+s+s1}{\PYGZsq{}}\PYG{l+s+s1}{quick}\PYG{l+s+s1}{\PYGZsq{}}\PYG{p}{)}
\PYG{g+go}{\PYGZlt{}vcs.displayplot.Dp ...\PYGZgt{}}
\PYG{g+gp}{\PYGZgt{}\PYGZgt{}\PYGZgt{} }\PYG{n}{a}\PYG{o}{.}\PYG{n}{clear}\PYG{p}{(}\PYG{p}{)} \PYG{c+c1}{\PYGZsh{} clear VCS displays from the page}
\end{Verbatim}

\end{description}\end{quote}

\end{fulllineitems}

\index{close() (vcs.Canvas.Canvas method)}

\begin{fulllineitems}
\phantomsection\label{vcs/Canvas:vcs.Canvas.Canvas.close}\pysiglinewithargsret{\sphinxbfcode{close}}{\emph{*args}, \emph{**kargs}}{}
Close the VCS Canvas. It will not deallocate the VCS Canvas object.
To deallocate the VCS Canvas, use the destroy method.
\begin{quote}\begin{description}
\item[{Example}] \leavevmode
\begin{Verbatim}[commandchars=\\\{\}]
\PYG{g+gp}{\PYGZgt{}\PYGZgt{}\PYGZgt{} }\PYG{n}{a}\PYG{o}{=}\PYG{n}{vcs}\PYG{o}{.}\PYG{n}{init}\PYG{p}{(}\PYG{p}{)}
\PYG{g+gp}{\PYGZgt{}\PYGZgt{}\PYGZgt{} }\PYG{n}{array} \PYG{o}{=} \PYG{p}{[}\PYG{n+nb}{range}\PYG{p}{(}\PYG{l+m+mi}{1}\PYG{p}{,} \PYG{l+m+mi}{11}\PYG{p}{)} \PYG{k}{for} \PYG{n}{\PYGZus{}} \PYG{o+ow}{in} \PYG{n+nb}{range}\PYG{p}{(}\PYG{l+m+mi}{1}\PYG{p}{,} \PYG{l+m+mi}{11}\PYG{p}{)}\PYG{p}{]}
\PYG{g+gp}{\PYGZgt{}\PYGZgt{}\PYGZgt{} }\PYG{n}{a}\PYG{o}{.}\PYG{n}{plot}\PYG{p}{(}\PYG{n}{array}\PYG{p}{,}\PYG{l+s+s1}{\PYGZsq{}}\PYG{l+s+s1}{default}\PYG{l+s+s1}{\PYGZsq{}}\PYG{p}{,}\PYG{l+s+s1}{\PYGZsq{}}\PYG{l+s+s1}{isofill}\PYG{l+s+s1}{\PYGZsq{}}\PYG{p}{,}\PYG{l+s+s1}{\PYGZsq{}}\PYG{l+s+s1}{quick}\PYG{l+s+s1}{\PYGZsq{}}\PYG{p}{)}
\PYG{g+go}{\PYGZlt{}vcs.displayplot.Dp ...\PYGZgt{}}
\PYG{g+gp}{\PYGZgt{}\PYGZgt{}\PYGZgt{} }\PYG{n}{a}\PYG{o}{.}\PYG{n}{close}\PYG{p}{(}\PYG{p}{)} \PYG{c+c1}{\PYGZsh{}close the vcs canvas}
\end{Verbatim}

\end{description}\end{quote}

\end{fulllineitems}

\index{copyfontto() (vcs.Canvas.Canvas method)}

\begin{fulllineitems}
\phantomsection\label{vcs/Canvas:vcs.Canvas.Canvas.copyfontto}\pysiglinewithargsret{\sphinxbfcode{copyfontto}}{\emph{font1}, \emph{font2}}{}
Copy `font1' into `font2'.
\begin{quote}\begin{description}
\item[{Parameters}] \leavevmode\begin{itemize}
\item {} 
\textbf{\texttt{font1}} (\emph{\texttt{str or int}}) -- Name/number of font to copy

\item {} 
\textbf{\texttt{font2}} (\emph{\texttt{str or int}}) -- Name/number of destination

\end{itemize}

\end{description}\end{quote}

\end{fulllineitems}

\index{create3d\_dual\_scalar() (vcs.Canvas.Canvas method)}

\begin{fulllineitems}
\phantomsection\label{vcs/Canvas:vcs.Canvas.Canvas.create3d_dual_scalar}\pysiglinewithargsret{\sphinxbfcode{create3d\_dual\_scalar}}{\emph{name=None}, \emph{source='default'}}{}
Create a new dv3d graphics method given the the name and the existing
dv3d graphics method to copy the attributes from. If no existing
dv3d graphics method is given, then the default dv3d graphics method will be used as the graphics method
to which the attributes will be copied from.

\begin{notice}{note}{Note:}
If the name provided already exists, then an error will be returned. graphics method
names must be unique.
\end{notice}
\begin{quote}\begin{description}
\item[{Example}] \leavevmode
\begin{Verbatim}[commandchars=\\\{\}]
\PYG{g+gp}{\PYGZgt{}\PYGZgt{}\PYGZgt{} }\PYG{n}{vcs}\PYG{o}{.}\PYG{n}{show}\PYG{p}{(}\PYG{l+s+s1}{\PYGZsq{}}\PYG{l+s+s1}{3d\PYGZus{}dual\PYGZus{}scalar}\PYG{l+s+s1}{\PYGZsq{}}\PYG{p}{)} \PYG{c+c1}{\PYGZsh{} show all available 3d\PYGZus{}dual\PYGZus{}scalar}
\PYG{g+go}{*******************3d\PYGZus{}dual\PYGZus{}scalar Names List**********************}
\PYG{g+gp}{...}
\PYG{g+go}{*******************End 3d\PYGZus{}dual\PYGZus{}scalar Names List**********************}
\PYG{g+gp}{\PYGZgt{}\PYGZgt{}\PYGZgt{} }\PYG{n}{ex}\PYG{o}{=}\PYG{n}{vcs}\PYG{o}{.}\PYG{n}{create3d\PYGZus{}dual\PYGZus{}scalar}\PYG{p}{(}\PYG{l+s+s1}{\PYGZsq{}}\PYG{l+s+s1}{3d\PYGZus{}dual\PYGZus{}scalar\PYGZus{}ex1}\PYG{l+s+s1}{\PYGZsq{}}\PYG{p}{)} \PYG{c+c1}{\PYGZsh{} Create 3d\PYGZus{}dual\PYGZus{}scalar \PYGZsq{}3d\PYGZus{}dual\PYGZus{}scalar\PYGZus{}ex1\PYGZsq{} that inherits from \PYGZsq{}default\PYGZsq{}}
\PYG{g+gp}{\PYGZgt{}\PYGZgt{}\PYGZgt{} }\PYG{n}{vcs}\PYG{o}{.}\PYG{n}{listelements}\PYG{p}{(}\PYG{l+s+s1}{\PYGZsq{}}\PYG{l+s+s1}{3d\PYGZus{}dual\PYGZus{}scalar}\PYG{l+s+s1}{\PYGZsq{}}\PYG{p}{)} \PYG{c+c1}{\PYGZsh{} should now contain the \PYGZsq{}3d\PYGZus{}dual\PYGZus{}scalar\PYGZus{}ex1\PYGZsq{} 3d\PYGZus{}dual\PYGZus{}scalar}
\PYG{g+go}{[...\PYGZsq{}3d\PYGZus{}dual\PYGZus{}scalar\PYGZus{}ex1\PYGZsq{}...]}
\end{Verbatim}

\item[{Parameters}] \leavevmode\begin{itemize}
\item {} 
\textbf{\texttt{name}} (\href{https://docs.python.org/2/library/functions.html\#str}{\emph{\texttt{str}}}) -- The name of the created object

\item {} 
\textbf{\texttt{source}} (\emph{\texttt{a 3d\_dual\_scalar or a string name of a 3d\_dual\_scalar}}) -- The object to inherit from

\end{itemize}

\item[{Returns}] \leavevmode
A 3d\_dual\_scalar graphics method object

\item[{Return type}] \leavevmode
vcs.dv3d.Gf3DDualScalar

\end{description}\end{quote}

\end{fulllineitems}

\index{create3d\_scalar() (vcs.Canvas.Canvas method)}

\begin{fulllineitems}
\phantomsection\label{vcs/Canvas:vcs.Canvas.Canvas.create3d_scalar}\pysiglinewithargsret{\sphinxbfcode{create3d\_scalar}}{\emph{name=None}, \emph{source='default'}}{}
Create a new dv3d graphics method given the the name and the existing
dv3d graphics method to copy the attributes from. If no existing
dv3d graphics method is given, then the default dv3d graphics method will be used as the graphics method
to which the attributes will be copied from.

\begin{notice}{note}{Note:}
If the name provided already exists, then an error will be returned. graphics method
names must be unique.
\end{notice}
\begin{quote}\begin{description}
\item[{Example}] \leavevmode
\begin{Verbatim}[commandchars=\\\{\}]
\PYG{g+gp}{\PYGZgt{}\PYGZgt{}\PYGZgt{} }\PYG{n}{vcs}\PYG{o}{.}\PYG{n}{show}\PYG{p}{(}\PYG{l+s+s1}{\PYGZsq{}}\PYG{l+s+s1}{3d\PYGZus{}scalar}\PYG{l+s+s1}{\PYGZsq{}}\PYG{p}{)} \PYG{c+c1}{\PYGZsh{} show all available 3d\PYGZus{}scalar}
\PYG{g+go}{*******************3d\PYGZus{}scalar Names List**********************}
\PYG{g+gp}{...}
\PYG{g+go}{*******************End 3d\PYGZus{}scalar Names List**********************}
\PYG{g+gp}{\PYGZgt{}\PYGZgt{}\PYGZgt{} }\PYG{n}{ex}\PYG{o}{=}\PYG{n}{vcs}\PYG{o}{.}\PYG{n}{create3d\PYGZus{}scalar}\PYG{p}{(}\PYG{l+s+s1}{\PYGZsq{}}\PYG{l+s+s1}{3d\PYGZus{}scalar\PYGZus{}ex1}\PYG{l+s+s1}{\PYGZsq{}}\PYG{p}{)} \PYG{c+c1}{\PYGZsh{} Create 3d\PYGZus{}scalar \PYGZsq{}3d\PYGZus{}scalar\PYGZus{}ex1\PYGZsq{} that inherits from \PYGZsq{}default\PYGZsq{}}
\PYG{g+gp}{\PYGZgt{}\PYGZgt{}\PYGZgt{} }\PYG{n}{vcs}\PYG{o}{.}\PYG{n}{listelements}\PYG{p}{(}\PYG{l+s+s1}{\PYGZsq{}}\PYG{l+s+s1}{3d\PYGZus{}scalar}\PYG{l+s+s1}{\PYGZsq{}}\PYG{p}{)} \PYG{c+c1}{\PYGZsh{} should now contain the \PYGZsq{}3d\PYGZus{}scalar\PYGZus{}ex1\PYGZsq{} 3d\PYGZus{}scalar}
\PYG{g+go}{[...\PYGZsq{}3d\PYGZus{}scalar\PYGZus{}ex1\PYGZsq{}...]}
\end{Verbatim}

\item[{Parameters}] \leavevmode\begin{itemize}
\item {} 
\textbf{\texttt{name}} (\href{https://docs.python.org/2/library/functions.html\#str}{\emph{\texttt{str}}}) -- The name of the created object

\item {} 
\textbf{\texttt{source}} (\emph{\texttt{a 3d\_scalar or a string name of a 3d\_scalar}}) -- The object to inherit from

\end{itemize}

\item[{Returns}] \leavevmode
A 3d\_scalar graphics method object

\item[{Return type}] \leavevmode
vcs.dv3d.Gf3Dscalar

\end{description}\end{quote}

\end{fulllineitems}

\index{create3d\_vector() (vcs.Canvas.Canvas method)}

\begin{fulllineitems}
\phantomsection\label{vcs/Canvas:vcs.Canvas.Canvas.create3d_vector}\pysiglinewithargsret{\sphinxbfcode{create3d\_vector}}{\emph{name=None}, \emph{source='default'}}{}
Create a new dv3d graphics method given the the name and the existing
dv3d graphics method to copy the attributes from. If no existing
dv3d graphics method is given, then the default dv3d graphics method will be used as the graphics method
to which the attributes will be copied from.

\begin{notice}{note}{Note:}
If the name provided already exists, then an error will be returned. graphics method
names must be unique.
\end{notice}
\begin{quote}\begin{description}
\item[{Example}] \leavevmode
\begin{Verbatim}[commandchars=\\\{\}]
\PYG{g+gp}{\PYGZgt{}\PYGZgt{}\PYGZgt{} }\PYG{n}{vcs}\PYG{o}{.}\PYG{n}{show}\PYG{p}{(}\PYG{l+s+s1}{\PYGZsq{}}\PYG{l+s+s1}{3d\PYGZus{}vector}\PYG{l+s+s1}{\PYGZsq{}}\PYG{p}{)} \PYG{c+c1}{\PYGZsh{} show all available 3d\PYGZus{}vector}
\PYG{g+go}{*******************3d\PYGZus{}vector Names List**********************}
\PYG{g+gp}{...}
\PYG{g+go}{*******************End 3d\PYGZus{}vector Names List**********************}
\PYG{g+gp}{\PYGZgt{}\PYGZgt{}\PYGZgt{} }\PYG{n}{ex}\PYG{o}{=}\PYG{n}{vcs}\PYG{o}{.}\PYG{n}{create3d\PYGZus{}vector}\PYG{p}{(}\PYG{l+s+s1}{\PYGZsq{}}\PYG{l+s+s1}{3d\PYGZus{}vector\PYGZus{}ex1}\PYG{l+s+s1}{\PYGZsq{}}\PYG{p}{)} \PYG{c+c1}{\PYGZsh{} Create 3d\PYGZus{}vector \PYGZsq{}3d\PYGZus{}vector\PYGZus{}ex1\PYGZsq{} that inherits from \PYGZsq{}default\PYGZsq{}}
\PYG{g+gp}{\PYGZgt{}\PYGZgt{}\PYGZgt{} }\PYG{n}{vcs}\PYG{o}{.}\PYG{n}{listelements}\PYG{p}{(}\PYG{l+s+s1}{\PYGZsq{}}\PYG{l+s+s1}{3d\PYGZus{}vector}\PYG{l+s+s1}{\PYGZsq{}}\PYG{p}{)} \PYG{c+c1}{\PYGZsh{} should now contain the \PYGZsq{}3d\PYGZus{}vector\PYGZus{}ex1\PYGZsq{} 3d\PYGZus{}vector}
\PYG{g+go}{[...\PYGZsq{}3d\PYGZus{}vector\PYGZus{}ex1\PYGZsq{}...]}
\end{Verbatim}

\item[{Parameters}] \leavevmode\begin{itemize}
\item {} 
\textbf{\texttt{name}} (\href{https://docs.python.org/2/library/functions.html\#str}{\emph{\texttt{str}}}) -- The name of the created object

\item {} 
\textbf{\texttt{source}} (\emph{\texttt{a 3d\_vector or a string name of a 3d\_vector}}) -- The object to inherit from

\end{itemize}

\item[{Returns}] \leavevmode
A 3d\_vector graphics method object

\item[{Return type}] \leavevmode
vcs.dv3d.Gf3Dvector

\end{description}\end{quote}

\end{fulllineitems}

\index{createboxfill() (vcs.Canvas.Canvas method)}

\begin{fulllineitems}
\phantomsection\label{vcs/Canvas:vcs.Canvas.Canvas.createboxfill}\pysiglinewithargsret{\sphinxbfcode{createboxfill}}{\emph{name=None}, \emph{source='default'}}{}
Create a new boxfill graphics method given the the name and the existing
boxfill graphics method to copy the attributes from. If no existing
boxfill graphics method is given, then the default boxfill graphics method will be used as the graphics method
to which the attributes will be copied from.

\begin{notice}{note}{Note:}
If the name provided already exists, then an error will be returned. graphics method
names must be unique.
\end{notice}
\begin{quote}\begin{description}
\item[{Example}] \leavevmode
\begin{Verbatim}[commandchars=\\\{\}]
\PYG{g+gp}{\PYGZgt{}\PYGZgt{}\PYGZgt{} }\PYG{n}{vcs}\PYG{o}{.}\PYG{n}{show}\PYG{p}{(}\PYG{l+s+s1}{\PYGZsq{}}\PYG{l+s+s1}{boxfill}\PYG{l+s+s1}{\PYGZsq{}}\PYG{p}{)} \PYG{c+c1}{\PYGZsh{} show all available boxfill}
\PYG{g+go}{*******************Boxfill Names List**********************}
\PYG{g+gp}{...}
\PYG{g+go}{*******************End Boxfill Names List**********************}
\PYG{g+gp}{\PYGZgt{}\PYGZgt{}\PYGZgt{} }\PYG{n}{ex}\PYG{o}{=}\PYG{n}{vcs}\PYG{o}{.}\PYG{n}{createboxfill}\PYG{p}{(}\PYG{l+s+s1}{\PYGZsq{}}\PYG{l+s+s1}{boxfill\PYGZus{}ex1}\PYG{l+s+s1}{\PYGZsq{}}\PYG{p}{)} \PYG{c+c1}{\PYGZsh{} Create boxfill \PYGZsq{}boxfill\PYGZus{}ex1\PYGZsq{} that inherits from \PYGZsq{}default\PYGZsq{}}
\PYG{g+gp}{\PYGZgt{}\PYGZgt{}\PYGZgt{} }\PYG{n}{vcs}\PYG{o}{.}\PYG{n}{listelements}\PYG{p}{(}\PYG{l+s+s1}{\PYGZsq{}}\PYG{l+s+s1}{boxfill}\PYG{l+s+s1}{\PYGZsq{}}\PYG{p}{)} \PYG{c+c1}{\PYGZsh{} should now contain the \PYGZsq{}boxfill\PYGZus{}ex1\PYGZsq{} boxfill}
\PYG{g+go}{[...\PYGZsq{}boxfill\PYGZus{}ex1\PYGZsq{}...]}
\PYG{g+gp}{\PYGZgt{}\PYGZgt{}\PYGZgt{} }\PYG{n}{ex2}\PYG{o}{=}\PYG{n}{vcs}\PYG{o}{.}\PYG{n}{createboxfill}\PYG{p}{(}\PYG{l+s+s1}{\PYGZsq{}}\PYG{l+s+s1}{boxfill\PYGZus{}ex2}\PYG{l+s+s1}{\PYGZsq{}}\PYG{p}{,}\PYG{l+s+s1}{\PYGZsq{}}\PYG{l+s+s1}{polar}\PYG{l+s+s1}{\PYGZsq{}}\PYG{p}{)} \PYG{c+c1}{\PYGZsh{} create \PYGZsq{}boxfill\PYGZus{}ex2\PYGZsq{} from \PYGZsq{}polar\PYGZsq{} template}
\PYG{g+gp}{\PYGZgt{}\PYGZgt{}\PYGZgt{} }\PYG{n}{vcs}\PYG{o}{.}\PYG{n}{listelements}\PYG{p}{(}\PYG{l+s+s1}{\PYGZsq{}}\PYG{l+s+s1}{boxfill}\PYG{l+s+s1}{\PYGZsq{}}\PYG{p}{)} \PYG{c+c1}{\PYGZsh{} should now contain the \PYGZsq{}boxfill\PYGZus{}ex2\PYGZsq{} boxfill}
\PYG{g+go}{[...\PYGZsq{}boxfill\PYGZus{}ex2\PYGZsq{}...]}
\end{Verbatim}

\item[{Parameters}] \leavevmode\begin{itemize}
\item {} 
\textbf{\texttt{name}} (\href{https://docs.python.org/2/library/functions.html\#str}{\emph{\texttt{str}}}) -- The name of the created object

\item {} 
\textbf{\texttt{source}} (\emph{\texttt{a boxfill or a string name of a boxfill}}) -- The object to inherit from

\item {} 
\textbf{\texttt{xaxis}} (\emph{\texttt{cdms2.axis.TransientAxis}}) -- Axis object to replace the slab -1 dim axis

\item {} 
\textbf{\texttt{yaxis}} (\emph{\texttt{cdms2.axis.TransientAxis}}) -- Axis object to replace the slab -2 dim axis, only if slab has more than 1D

\item {} 
\textbf{\texttt{zaxis}} (\emph{\texttt{cdms2.axis.TransientAxis}}) -- Axis object to replace the slab -3 dim axis, only if slab has more than 2D

\item {} 
\textbf{\texttt{taxis}} (\emph{\texttt{cdms2.axis.TransientAxis}}) -- Axis object to replace the slab -4 dim axis, only if slab has more than 3D

\item {} 
\textbf{\texttt{waxis}} (\emph{\texttt{cdms2.axis.TransientAxis}}) -- Axis object to replace the slab -5 dim axis, only if slab has more than 4D

\item {} 
\textbf{\texttt{xrev}} (\href{https://docs.python.org/2/library/functions.html\#bool}{\emph{\texttt{bool}}}) -- reverse x axis

\item {} 
\textbf{\texttt{yrev}} (\href{https://docs.python.org/2/library/functions.html\#bool}{\emph{\texttt{bool}}}) -- reverse y axis, only if slab has more than 1D

\item {} 
\textbf{\texttt{xarray}} (\href{https://docs.python.org/2/library/array.html\#module-array}{\emph{\texttt{array}}}) -- Values to use instead of x axis

\item {} 
\textbf{\texttt{yarray}} (\href{https://docs.python.org/2/library/array.html\#module-array}{\emph{\texttt{array}}}) -- Values to use instead of y axis, only if var has more than 1D

\item {} 
\textbf{\texttt{zarray}} (\href{https://docs.python.org/2/library/array.html\#module-array}{\emph{\texttt{array}}}) -- Values to use instead of z axis, only if var has more than 2D

\item {} 
\textbf{\texttt{tarray}} (\href{https://docs.python.org/2/library/array.html\#module-array}{\emph{\texttt{array}}}) -- Values to use instead of t axis, only if var has more than 3D

\item {} 
\textbf{\texttt{warray}} (\href{https://docs.python.org/2/library/array.html\#module-array}{\emph{\texttt{array}}}) -- Values to use instead of w axis, only if var has more than 4D

\item {} 
\textbf{\texttt{continents}} (\href{https://docs.python.org/2/library/functions.html\#int}{\emph{\texttt{int}}}) -- continents type number

\item {} 
\textbf{\texttt{name}} -- replaces variable name on plot

\item {} 
\textbf{\texttt{time}} (\emph{\texttt{A cdtime object}}) -- replaces time name on plot

\item {} 
\textbf{\texttt{units}} (\href{https://docs.python.org/2/library/functions.html\#str}{\emph{\texttt{str}}}) -- replaces units value on plot

\item {} 
\textbf{\texttt{ymd}} (\href{https://docs.python.org/2/library/functions.html\#str}{\emph{\texttt{str}}}) -- replaces year/month/day on plot

\item {} 
\textbf{\texttt{hms}} (\href{https://docs.python.org/2/library/functions.html\#str}{\emph{\texttt{str}}}) -- replaces hh/mm/ss on plot

\item {} 
\textbf{\texttt{file\_comment}} (\href{https://docs.python.org/2/library/functions.html\#str}{\emph{\texttt{str}}}) -- replaces file\_comment on plot

\item {} 
\textbf{\texttt{xbounds}} (\href{https://docs.python.org/2/library/array.html\#module-array}{\emph{\texttt{array}}}) -- Values to use instead of x axis bounds values

\item {} 
\textbf{\texttt{ybounds}} (\href{https://docs.python.org/2/library/array.html\#module-array}{\emph{\texttt{array}}}) -- Values to use instead of y axis bounds values (if exist)

\item {} 
\textbf{\texttt{xname}} (\href{https://docs.python.org/2/library/functions.html\#str}{\emph{\texttt{str}}}) -- replace xaxis name on plot

\item {} 
\textbf{\texttt{yname}} (\href{https://docs.python.org/2/library/functions.html\#str}{\emph{\texttt{str}}}) -- replace yaxis name on plot (if exists)

\item {} 
\textbf{\texttt{zname}} (\href{https://docs.python.org/2/library/functions.html\#str}{\emph{\texttt{str}}}) -- replace zaxis name on plot (if exists)

\item {} 
\textbf{\texttt{tname}} (\href{https://docs.python.org/2/library/functions.html\#str}{\emph{\texttt{str}}}) -- replace taxis name on plot (if exists)

\item {} 
\textbf{\texttt{wname}} (\href{https://docs.python.org/2/library/functions.html\#str}{\emph{\texttt{str}}}) -- replace waxis name on plot (if exists)

\item {} 
\textbf{\texttt{xunits}} (\href{https://docs.python.org/2/library/functions.html\#str}{\emph{\texttt{str}}}) -- replace xaxis units on plot

\item {} 
\textbf{\texttt{yunits}} (\href{https://docs.python.org/2/library/functions.html\#str}{\emph{\texttt{str}}}) -- replace yaxis units on plot (if exists)

\item {} 
\textbf{\texttt{zunits}} (\href{https://docs.python.org/2/library/functions.html\#str}{\emph{\texttt{str}}}) -- replace zaxis units on plot (if exists)

\item {} 
\textbf{\texttt{tunits}} (\href{https://docs.python.org/2/library/functions.html\#str}{\emph{\texttt{str}}}) -- replace taxis units on plot (if exists)

\item {} 
\textbf{\texttt{wunits}} (\href{https://docs.python.org/2/library/functions.html\#str}{\emph{\texttt{str}}}) -- replace waxis units on plot (if exists)

\item {} 
\textbf{\texttt{xweights}} (\href{https://docs.python.org/2/library/array.html\#module-array}{\emph{\texttt{array}}}) -- replace xaxis weights used for computing mean

\item {} 
\textbf{\texttt{yweights}} (\href{https://docs.python.org/2/library/array.html\#module-array}{\emph{\texttt{array}}}) -- replace xaxis weights used for computing mean

\item {} 
\textbf{\texttt{comment1}} (\href{https://docs.python.org/2/library/functions.html\#str}{\emph{\texttt{str}}}) -- replaces comment1 on plot

\item {} 
\textbf{\texttt{comment2}} (\href{https://docs.python.org/2/library/functions.html\#str}{\emph{\texttt{str}}}) -- replaces comment2 on plot

\item {} 
\textbf{\texttt{comment3}} (\href{https://docs.python.org/2/library/functions.html\#str}{\emph{\texttt{str}}}) -- replaces comment3 on plot

\item {} 
\textbf{\texttt{comment4}} (\href{https://docs.python.org/2/library/functions.html\#str}{\emph{\texttt{str}}}) -- replaces comment4 on plot

\item {} 
\textbf{\texttt{long\_name}} (\href{https://docs.python.org/2/library/functions.html\#str}{\emph{\texttt{str}}}) -- replaces long\_name on plot

\item {} 
\textbf{\texttt{grid}} (\emph{\texttt{cdms2.grid.TransientRectGrid}}) -- replaces array grid (if exists)

\item {} 
\textbf{\texttt{bg}} (\emph{\texttt{bool/int}}) -- plots in background mode

\item {} 
\textbf{\texttt{ratio}} (\index{xmtics1 (vcs.Canvas.Canvas attribute)}\index{xmtics2 (vcs.Canvas.Canvas attribute)}\index{ymtics1 (vcs.Canvas.Canvas attribute)}\index{ymtics2 (vcs.Canvas.Canvas attribute)}\index{xticlabels1 (vcs.Canvas.Canvas attribute)}\index{xticlabels2 (vcs.Canvas.Canvas attribute)}\index{yticlabels1 (vcs.Canvas.Canvas attribute)}\index{yticlabels2 (vcs.Canvas.Canvas attribute)}\index{projection (vcs.Canvas.Canvas attribute)}\index{datawc\_x1 (vcs.Canvas.Canvas attribute)}\index{datawc\_x2 (vcs.Canvas.Canvas attribute)}\index{datawc\_y1 (vcs.Canvas.Canvas attribute)}\index{datawc\_y2 (vcs.Canvas.Canvas attribute)}\index{datawc\_timeunits (vcs.Canvas.Canvas attribute)}\index{datawc\_calendar (vcs.Canvas.Canvas attribute)}) -- sets the y/x ratio ,if passed as a string with `t' at the end, will aslo moves the ticks

\item {} 
\textbf{\texttt{xaxisconvert}} (\href{https://docs.python.org/2/library/functions.html\#str}{\emph{\texttt{str}}}) -- (Ex: `linear') converting xaxis linear/log/log10/ln/exp/area\_wt

\item {} 
\textbf{\texttt{yaxisconvert}} (\href{https://docs.python.org/2/library/functions.html\#str}{\emph{\texttt{str}}}) -- (Ex: `linear') converting yaxis linear/log/log10/ln/exp/area\_wt

\item {} 
\textbf{\texttt{new\_GM\_name}} (\href{https://docs.python.org/2/library/functions.html\#str}{\emph{\texttt{str}}}) -- (Ex: `my\_awesome\_gm') name of the new graphics method object. If no name is given, then one will be created for use.

\item {} 
\textbf{\texttt{source\_GM\_name}} -- (Ex: `default') copy the contents of the source object to the newly created one. If no name is given, then the `default' graphics methond contents is copied over to the new object.

\end{itemize}

\item[{Returns}] \leavevmode
A boxfill graphics method object

\item[{Return type}] \leavevmode
{\hyperref[vcs/graphics/boxfill:vcs.boxfill.Gfb]{\sphinxcrossref{vcs.boxfill.Gfb}}}

\end{description}\end{quote}

\end{fulllineitems}

\index{createcolormap() (vcs.Canvas.Canvas method)}

\begin{fulllineitems}
\phantomsection\label{vcs/Canvas:vcs.Canvas.Canvas.createcolormap}\pysiglinewithargsret{\sphinxbfcode{createcolormap}}{\emph{Cp\_name=None}, \emph{Cp\_name\_src='default'}}{}
Create a new colormap secondary method given the the name and the existing
colormap secondary method to copy the attributes from. If no existing
colormap secondary method is given, then the default colormap secondary method will be used as the graphics method
to which the attributes will be copied from.

\begin{notice}{note}{Note:}
If the name provided already exists, then an error will be returned. secondary method
names must be unique.
\end{notice}
\begin{quote}\begin{description}
\item[{Example}] \leavevmode
\begin{Verbatim}[commandchars=\\\{\}]
\PYG{g+gp}{\PYGZgt{}\PYGZgt{}\PYGZgt{} }\PYG{n}{vcs}\PYG{o}{.}\PYG{n}{show}\PYG{p}{(}\PYG{l+s+s1}{\PYGZsq{}}\PYG{l+s+s1}{colormap}\PYG{l+s+s1}{\PYGZsq{}}\PYG{p}{)} \PYG{c+c1}{\PYGZsh{} show all available colormap}
\PYG{g+go}{*******************Colormap Names List**********************}
\PYG{g+gp}{...}
\PYG{g+go}{*******************End Colormap Names List**********************}
\PYG{g+gp}{\PYGZgt{}\PYGZgt{}\PYGZgt{} }\PYG{n}{ex}\PYG{o}{=}\PYG{n}{vcs}\PYG{o}{.}\PYG{n}{createcolormap}\PYG{p}{(}\PYG{l+s+s1}{\PYGZsq{}}\PYG{l+s+s1}{colormap\PYGZus{}ex1}\PYG{l+s+s1}{\PYGZsq{}}\PYG{p}{)} \PYG{c+c1}{\PYGZsh{} Create colormap \PYGZsq{}colormap\PYGZus{}ex1\PYGZsq{} that inherits from \PYGZsq{}default\PYGZsq{}}
\PYG{g+gp}{\PYGZgt{}\PYGZgt{}\PYGZgt{} }\PYG{n}{vcs}\PYG{o}{.}\PYG{n}{listelements}\PYG{p}{(}\PYG{l+s+s1}{\PYGZsq{}}\PYG{l+s+s1}{colormap}\PYG{l+s+s1}{\PYGZsq{}}\PYG{p}{)} \PYG{c+c1}{\PYGZsh{} should now contain the \PYGZsq{}colormap\PYGZus{}ex1\PYGZsq{} colormap}
\PYG{g+go}{[...\PYGZsq{}colormap\PYGZus{}ex1\PYGZsq{}...]}
\PYG{g+gp}{\PYGZgt{}\PYGZgt{}\PYGZgt{} }\PYG{n}{ex2}\PYG{o}{=}\PYG{n}{vcs}\PYG{o}{.}\PYG{n}{createcolormap}\PYG{p}{(}\PYG{l+s+s1}{\PYGZsq{}}\PYG{l+s+s1}{colormap\PYGZus{}ex2}\PYG{l+s+s1}{\PYGZsq{}}\PYG{p}{,}\PYG{l+s+s1}{\PYGZsq{}}\PYG{l+s+s1}{rainbow}\PYG{l+s+s1}{\PYGZsq{}}\PYG{p}{)} \PYG{c+c1}{\PYGZsh{} create \PYGZsq{}colormap\PYGZus{}ex2\PYGZsq{} from \PYGZsq{}rainbow\PYGZsq{} template}
\PYG{g+gp}{\PYGZgt{}\PYGZgt{}\PYGZgt{} }\PYG{n}{vcs}\PYG{o}{.}\PYG{n}{listelements}\PYG{p}{(}\PYG{l+s+s1}{\PYGZsq{}}\PYG{l+s+s1}{colormap}\PYG{l+s+s1}{\PYGZsq{}}\PYG{p}{)} \PYG{c+c1}{\PYGZsh{} should now contain the \PYGZsq{}colormap\PYGZus{}ex2\PYGZsq{} colormap}
\PYG{g+go}{[...\PYGZsq{}colormap\PYGZus{}ex2\PYGZsq{}...]}
\end{Verbatim}

\item[{Parameters}] \leavevmode\begin{itemize}
\item {} 
\textbf{\texttt{Cp\_name}} (\href{https://docs.python.org/2/library/functions.html\#str}{\emph{\texttt{str}}}) -- The name of the created object

\item {} 
\textbf{\texttt{Cp\_name\_src}} (\emph{\texttt{a colormap or a string name of a colormap}}) -- The object to inherit

\end{itemize}

\item[{Returns}] \leavevmode
A VCS colormap object

\item[{Return type}] \leavevmode
{\hyperref[vcs/misc/colormap:vcs.colormap.Cp]{\sphinxcrossref{vcs.colormap.Cp}}}

\end{description}\end{quote}

\end{fulllineitems}

\index{createfillarea() (vcs.Canvas.Canvas method)}

\begin{fulllineitems}
\phantomsection\label{vcs/Canvas:vcs.Canvas.Canvas.createfillarea}\pysiglinewithargsret{\sphinxbfcode{createfillarea}}{\emph{name=None}, \emph{source='default'}, \emph{style=None}, \emph{index=None}, \emph{color=None}, \emph{priority=1}, \emph{viewport=None}, \emph{worldcoordinate=None}, \emph{x=None}, \emph{y=None}}{}
Create a new fillarea secondary method given the the name and the existing
fillarea secondary method to copy the attributes from. If no existing
fillarea secondary method is given, then the default fillarea secondary method will be used as the graphics method
to which the attributes will be copied from.

\begin{notice}{note}{Note:}
If the name provided already exists, then an error will be returned. secondary method
names must be unique.
\end{notice}
\begin{quote}\begin{description}
\item[{Example}] \leavevmode
\begin{Verbatim}[commandchars=\\\{\}]
\PYG{g+gp}{\PYGZgt{}\PYGZgt{}\PYGZgt{} }\PYG{n}{vcs}\PYG{o}{.}\PYG{n}{show}\PYG{p}{(}\PYG{l+s+s1}{\PYGZsq{}}\PYG{l+s+s1}{fillarea}\PYG{l+s+s1}{\PYGZsq{}}\PYG{p}{)} \PYG{c+c1}{\PYGZsh{} show all available fillarea}
\PYG{g+go}{*******************Fillarea Names List**********************}
\PYG{g+gp}{...}
\PYG{g+go}{*******************End Fillarea Names List**********************}
\PYG{g+gp}{\PYGZgt{}\PYGZgt{}\PYGZgt{} }\PYG{n}{ex}\PYG{o}{=}\PYG{n}{vcs}\PYG{o}{.}\PYG{n}{createfillarea}\PYG{p}{(}\PYG{l+s+s1}{\PYGZsq{}}\PYG{l+s+s1}{fillarea\PYGZus{}ex1}\PYG{l+s+s1}{\PYGZsq{}}\PYG{p}{)} \PYG{c+c1}{\PYGZsh{} Create fillarea \PYGZsq{}fillarea\PYGZus{}ex1\PYGZsq{} that inherits from \PYGZsq{}default\PYGZsq{}}
\PYG{g+gp}{\PYGZgt{}\PYGZgt{}\PYGZgt{} }\PYG{n}{vcs}\PYG{o}{.}\PYG{n}{listelements}\PYG{p}{(}\PYG{l+s+s1}{\PYGZsq{}}\PYG{l+s+s1}{fillarea}\PYG{l+s+s1}{\PYGZsq{}}\PYG{p}{)} \PYG{c+c1}{\PYGZsh{} should now contain the \PYGZsq{}fillarea\PYGZus{}ex1\PYGZsq{} fillarea}
\PYG{g+go}{[...\PYGZsq{}fillarea\PYGZus{}ex1\PYGZsq{}...]}
\end{Verbatim}

\item[{Parameters}] \leavevmode\begin{itemize}
\item {} 
\textbf{\texttt{name}} (\href{https://docs.python.org/2/library/functions.html\#str}{\emph{\texttt{str}}}) -- Name of created object

\item {} 
\textbf{\texttt{source}} (\href{https://docs.python.org/2/library/functions.html\#str}{\emph{\texttt{str}}}) -- a fillarea, or string name of a fillarea

\item {} 
\textbf{\texttt{style}} (\href{https://docs.python.org/2/library/functions.html\#str}{\emph{\texttt{str}}}) -- One of ``hatch'', ``solid'', or ``pattern''.

\item {} 
\textbf{\texttt{index}} -- Specifies which \href{http://uvcdat.llnl.gov/gallery/fullsize/pattern\_chart.png}{pattern} to fill with.

\end{itemize}

\end{description}\end{quote}

Accepts ints from 1-20.
\begin{quote}\begin{description}
\item[{Parameters}] \leavevmode
\textbf{\texttt{color}} -- A color name from the \href{https://en.wikipedia.org/wiki/X11\_color\_names}{X11 Color Names list},

\end{description}\end{quote}

or an integer value from 0-255, or an RGB/RGBA tuple/list (e.g. (0,100,0), (100,100,0,50))
\begin{quote}\begin{description}
\item[{Parameters}] \leavevmode\begin{itemize}
\item {} 
\textbf{\texttt{priority}} (\href{https://docs.python.org/2/library/functions.html\#int}{\emph{\texttt{int}}}) -- The layer on which the fillarea will be drawn.

\item {} 
\textbf{\texttt{viewport}} (\emph{\texttt{list of floats}}) -- 4 floats between 0 and 1. These specify the area that the X/Y values are mapped to inside of the canvas

\item {} 
\textbf{\texttt{worldcoordinate}} (\emph{\texttt{list of floats}}) -- List of 4 floats (xmin, xmax, ymin, ymax)

\item {} 
\textbf{\texttt{x}} (\emph{\texttt{list of floats}}) -- List of lists of x coordinates. Values must be between worldcoordinate{[}0{]} and worldcoordinate{[}1{]}.

\item {} 
\textbf{\texttt{y}} (\emph{\texttt{list of floats}}) -- List of lists of y coordinates. Values must be between worldcoordinate{[}2{]} and worldcoordinate{[}3{]}.

\end{itemize}

\item[{Returns}] \leavevmode
A fillarea object

\item[{Return type}] \leavevmode
{\hyperref[vcs/secondary/fillarea:vcs.fillarea.Tf]{\sphinxcrossref{vcs.fillarea.Tf}}}

\end{description}\end{quote}

\end{fulllineitems}

\index{createisofill() (vcs.Canvas.Canvas method)}

\begin{fulllineitems}
\phantomsection\label{vcs/Canvas:vcs.Canvas.Canvas.createisofill}\pysiglinewithargsret{\sphinxbfcode{createisofill}}{\emph{name=None}, \emph{source='default'}}{}
Create a new isofill graphics method given the the name and the existing
isofill graphics method to copy the attributes from. If no existing
isofill graphics method is given, then the default isofill graphics method will be used as the graphics method
to which the attributes will be copied from.

\begin{notice}{note}{Note:}
If the name provided already exists, then an error will be returned. graphics method
names must be unique.
\end{notice}
\begin{quote}\begin{description}
\item[{Example}] \leavevmode
\begin{Verbatim}[commandchars=\\\{\}]
\PYG{g+gp}{\PYGZgt{}\PYGZgt{}\PYGZgt{} }\PYG{n}{vcs}\PYG{o}{.}\PYG{n}{show}\PYG{p}{(}\PYG{l+s+s1}{\PYGZsq{}}\PYG{l+s+s1}{isofill}\PYG{l+s+s1}{\PYGZsq{}}\PYG{p}{)} \PYG{c+c1}{\PYGZsh{} show all available isofill}
\PYG{g+go}{*******************Isofill Names List**********************}
\PYG{g+gp}{...}
\PYG{g+go}{*******************End Isofill Names List**********************}
\PYG{g+gp}{\PYGZgt{}\PYGZgt{}\PYGZgt{} }\PYG{n}{ex}\PYG{o}{=}\PYG{n}{vcs}\PYG{o}{.}\PYG{n}{createisofill}\PYG{p}{(}\PYG{l+s+s1}{\PYGZsq{}}\PYG{l+s+s1}{isofill\PYGZus{}ex1}\PYG{l+s+s1}{\PYGZsq{}}\PYG{p}{)} \PYG{c+c1}{\PYGZsh{} Create isofill \PYGZsq{}isofill\PYGZus{}ex1\PYGZsq{} that inherits from \PYGZsq{}default\PYGZsq{}}
\PYG{g+gp}{\PYGZgt{}\PYGZgt{}\PYGZgt{} }\PYG{n}{vcs}\PYG{o}{.}\PYG{n}{listelements}\PYG{p}{(}\PYG{l+s+s1}{\PYGZsq{}}\PYG{l+s+s1}{isofill}\PYG{l+s+s1}{\PYGZsq{}}\PYG{p}{)} \PYG{c+c1}{\PYGZsh{} should now contain the \PYGZsq{}isofill\PYGZus{}ex1\PYGZsq{} isofill}
\PYG{g+go}{[...\PYGZsq{}isofill\PYGZus{}ex1\PYGZsq{}...]}
\PYG{g+gp}{\PYGZgt{}\PYGZgt{}\PYGZgt{} }\PYG{n}{ex2}\PYG{o}{=}\PYG{n}{vcs}\PYG{o}{.}\PYG{n}{createisofill}\PYG{p}{(}\PYG{l+s+s1}{\PYGZsq{}}\PYG{l+s+s1}{isofill\PYGZus{}ex2}\PYG{l+s+s1}{\PYGZsq{}}\PYG{p}{,}\PYG{l+s+s1}{\PYGZsq{}}\PYG{l+s+s1}{polar}\PYG{l+s+s1}{\PYGZsq{}}\PYG{p}{)} \PYG{c+c1}{\PYGZsh{} create \PYGZsq{}isofill\PYGZus{}ex2\PYGZsq{} from \PYGZsq{}polar\PYGZsq{} template}
\PYG{g+gp}{\PYGZgt{}\PYGZgt{}\PYGZgt{} }\PYG{n}{vcs}\PYG{o}{.}\PYG{n}{listelements}\PYG{p}{(}\PYG{l+s+s1}{\PYGZsq{}}\PYG{l+s+s1}{isofill}\PYG{l+s+s1}{\PYGZsq{}}\PYG{p}{)} \PYG{c+c1}{\PYGZsh{} should now contain the \PYGZsq{}isofill\PYGZus{}ex2\PYGZsq{} isofill}
\PYG{g+go}{[...\PYGZsq{}isofill\PYGZus{}ex2\PYGZsq{}...]}
\end{Verbatim}

\item[{Parameters}] \leavevmode\begin{itemize}
\item {} 
\textbf{\texttt{name}} (\href{https://docs.python.org/2/library/functions.html\#str}{\emph{\texttt{str}}}) -- The name of the created object

\item {} 
\textbf{\texttt{source}} (\emph{\texttt{an isofill object, or string name of an isofill object}}) -- The object to inherit from

\item {} 
\textbf{\texttt{xaxis}} (\emph{\texttt{cdms2.axis.TransientAxis}}) -- Axis object to replace the slab -1 dim axis

\item {} 
\textbf{\texttt{yaxis}} (\emph{\texttt{cdms2.axis.TransientAxis}}) -- Axis object to replace the slab -2 dim axis, only if slab has more than 1D

\item {} 
\textbf{\texttt{zaxis}} (\emph{\texttt{cdms2.axis.TransientAxis}}) -- Axis object to replace the slab -3 dim axis, only if slab has more than 2D

\item {} 
\textbf{\texttt{taxis}} (\emph{\texttt{cdms2.axis.TransientAxis}}) -- Axis object to replace the slab -4 dim axis, only if slab has more than 3D

\item {} 
\textbf{\texttt{waxis}} (\emph{\texttt{cdms2.axis.TransientAxis}}) -- Axis object to replace the slab -5 dim axis, only if slab has more than 4D

\item {} 
\textbf{\texttt{xrev}} (\href{https://docs.python.org/2/library/functions.html\#bool}{\emph{\texttt{bool}}}) -- reverse x axis

\item {} 
\textbf{\texttt{yrev}} (\href{https://docs.python.org/2/library/functions.html\#bool}{\emph{\texttt{bool}}}) -- reverse y axis, only if slab has more than 1D

\item {} 
\textbf{\texttt{xarray}} (\href{https://docs.python.org/2/library/array.html\#module-array}{\emph{\texttt{array}}}) -- Values to use instead of x axis

\item {} 
\textbf{\texttt{yarray}} (\href{https://docs.python.org/2/library/array.html\#module-array}{\emph{\texttt{array}}}) -- Values to use instead of y axis, only if var has more than 1D

\item {} 
\textbf{\texttt{zarray}} (\href{https://docs.python.org/2/library/array.html\#module-array}{\emph{\texttt{array}}}) -- Values to use instead of z axis, only if var has more than 2D

\item {} 
\textbf{\texttt{tarray}} (\href{https://docs.python.org/2/library/array.html\#module-array}{\emph{\texttt{array}}}) -- Values to use instead of t axis, only if var has more than 3D

\item {} 
\textbf{\texttt{warray}} (\href{https://docs.python.org/2/library/array.html\#module-array}{\emph{\texttt{array}}}) -- Values to use instead of w axis, only if var has more than 4D

\item {} 
\textbf{\texttt{continents}} (\href{https://docs.python.org/2/library/functions.html\#int}{\emph{\texttt{int}}}) -- continents type number

\item {} 
\textbf{\texttt{name}} -- replaces variable name on plot

\item {} 
\textbf{\texttt{time}} (\emph{\texttt{A cdtime object}}) -- replaces time name on plot

\item {} 
\textbf{\texttt{units}} (\href{https://docs.python.org/2/library/functions.html\#str}{\emph{\texttt{str}}}) -- replaces units value on plot

\item {} 
\textbf{\texttt{ymd}} (\href{https://docs.python.org/2/library/functions.html\#str}{\emph{\texttt{str}}}) -- replaces year/month/day on plot

\item {} 
\textbf{\texttt{hms}} (\href{https://docs.python.org/2/library/functions.html\#str}{\emph{\texttt{str}}}) -- replaces hh/mm/ss on plot

\item {} 
\textbf{\texttt{file\_comment}} (\href{https://docs.python.org/2/library/functions.html\#str}{\emph{\texttt{str}}}) -- replaces file\_comment on plot

\item {} 
\textbf{\texttt{xbounds}} (\href{https://docs.python.org/2/library/array.html\#module-array}{\emph{\texttt{array}}}) -- Values to use instead of x axis bounds values

\item {} 
\textbf{\texttt{ybounds}} (\href{https://docs.python.org/2/library/array.html\#module-array}{\emph{\texttt{array}}}) -- Values to use instead of y axis bounds values (if exist)

\item {} 
\textbf{\texttt{xname}} (\href{https://docs.python.org/2/library/functions.html\#str}{\emph{\texttt{str}}}) -- replace xaxis name on plot

\item {} 
\textbf{\texttt{yname}} (\href{https://docs.python.org/2/library/functions.html\#str}{\emph{\texttt{str}}}) -- replace yaxis name on plot (if exists)

\item {} 
\textbf{\texttt{zname}} (\href{https://docs.python.org/2/library/functions.html\#str}{\emph{\texttt{str}}}) -- replace zaxis name on plot (if exists)

\item {} 
\textbf{\texttt{tname}} (\href{https://docs.python.org/2/library/functions.html\#str}{\emph{\texttt{str}}}) -- replace taxis name on plot (if exists)

\item {} 
\textbf{\texttt{wname}} (\href{https://docs.python.org/2/library/functions.html\#str}{\emph{\texttt{str}}}) -- replace waxis name on plot (if exists)

\item {} 
\textbf{\texttt{xunits}} (\href{https://docs.python.org/2/library/functions.html\#str}{\emph{\texttt{str}}}) -- replace xaxis units on plot

\item {} 
\textbf{\texttt{yunits}} (\href{https://docs.python.org/2/library/functions.html\#str}{\emph{\texttt{str}}}) -- replace yaxis units on plot (if exists)

\item {} 
\textbf{\texttt{zunits}} (\href{https://docs.python.org/2/library/functions.html\#str}{\emph{\texttt{str}}}) -- replace zaxis units on plot (if exists)

\item {} 
\textbf{\texttt{tunits}} (\href{https://docs.python.org/2/library/functions.html\#str}{\emph{\texttt{str}}}) -- replace taxis units on plot (if exists)

\item {} 
\textbf{\texttt{wunits}} (\href{https://docs.python.org/2/library/functions.html\#str}{\emph{\texttt{str}}}) -- replace waxis units on plot (if exists)

\item {} 
\textbf{\texttt{xweights}} (\href{https://docs.python.org/2/library/array.html\#module-array}{\emph{\texttt{array}}}) -- replace xaxis weights used for computing mean

\item {} 
\textbf{\texttt{yweights}} (\href{https://docs.python.org/2/library/array.html\#module-array}{\emph{\texttt{array}}}) -- replace xaxis weights used for computing mean

\item {} 
\textbf{\texttt{comment1}} (\href{https://docs.python.org/2/library/functions.html\#str}{\emph{\texttt{str}}}) -- replaces comment1 on plot

\item {} 
\textbf{\texttt{comment2}} (\href{https://docs.python.org/2/library/functions.html\#str}{\emph{\texttt{str}}}) -- replaces comment2 on plot

\item {} 
\textbf{\texttt{comment3}} (\href{https://docs.python.org/2/library/functions.html\#str}{\emph{\texttt{str}}}) -- replaces comment3 on plot

\item {} 
\textbf{\texttt{comment4}} (\href{https://docs.python.org/2/library/functions.html\#str}{\emph{\texttt{str}}}) -- replaces comment4 on plot

\item {} 
\textbf{\texttt{long\_name}} (\href{https://docs.python.org/2/library/functions.html\#str}{\emph{\texttt{str}}}) -- replaces long\_name on plot

\item {} 
\textbf{\texttt{grid}} (\emph{\texttt{cdms2.grid.TransientRectGrid}}) -- replaces array grid (if exists)

\item {} 
\textbf{\texttt{bg}} (\emph{\texttt{bool/int}}) -- plots in background mode

\item {} 
\textbf{\texttt{ratio}} (\index{xmtics1 (vcs.Canvas.Canvas attribute)}\index{xmtics2 (vcs.Canvas.Canvas attribute)}\index{ymtics1 (vcs.Canvas.Canvas attribute)}\index{ymtics2 (vcs.Canvas.Canvas attribute)}\index{xticlabels1 (vcs.Canvas.Canvas attribute)}\index{xticlabels2 (vcs.Canvas.Canvas attribute)}\index{yticlabels1 (vcs.Canvas.Canvas attribute)}\index{yticlabels2 (vcs.Canvas.Canvas attribute)}\index{projection (vcs.Canvas.Canvas attribute)}\index{datawc\_x1 (vcs.Canvas.Canvas attribute)}\index{datawc\_x2 (vcs.Canvas.Canvas attribute)}\index{datawc\_y1 (vcs.Canvas.Canvas attribute)}\index{datawc\_y2 (vcs.Canvas.Canvas attribute)}\index{datawc\_timeunits (vcs.Canvas.Canvas attribute)}\index{datawc\_calendar (vcs.Canvas.Canvas attribute)}) -- sets the y/x ratio ,if passed as a string with `t' at the end, will aslo moves the ticks

\item {} 
\textbf{\texttt{xaxisconvert}} (\href{https://docs.python.org/2/library/functions.html\#str}{\emph{\texttt{str}}}) -- (Ex: `linear') converting xaxis linear/log/log10/ln/exp/area\_wt

\item {} 
\textbf{\texttt{yaxisconvert}} (\href{https://docs.python.org/2/library/functions.html\#str}{\emph{\texttt{str}}}) -- (Ex: `linear') converting yaxis linear/log/log10/ln/exp/area\_wt

\item {} 
\textbf{\texttt{new\_GM\_name}} (\href{https://docs.python.org/2/library/functions.html\#str}{\emph{\texttt{str}}}) -- (Ex: `my\_awesome\_gm') name of the new graphics method object. If no name is given, then one will be created for use.

\item {} 
\textbf{\texttt{source\_GM\_name}} -- (Ex: `default') copy the contents of the source object to the newly created one. If no name is given, then the `default' graphics methond contents is copied over to the new object.

\end{itemize}

\item[{Returns}] \leavevmode
An isofill graphics method

\item[{Return type}] \leavevmode
{\hyperref[vcs/graphics/isofill:vcs.isofill.Gfi]{\sphinxcrossref{vcs.isofill.Gfi}}}

\end{description}\end{quote}

\end{fulllineitems}

\index{createisoline() (vcs.Canvas.Canvas method)}

\begin{fulllineitems}
\phantomsection\label{vcs/Canvas:vcs.Canvas.Canvas.createisoline}\pysiglinewithargsret{\sphinxbfcode{createisoline}}{\emph{name=None}, \emph{source='default'}}{}
Create a new isoline graphics method given the the name and the existing
isoline graphics method to copy the attributes from. If no existing
isoline graphics method is given, then the default isoline graphics method will be used as the graphics method
to which the attributes will be copied from.

\begin{notice}{note}{Note:}
If the name provided already exists, then an error will be returned. graphics method
names must be unique.
\end{notice}
\begin{quote}\begin{description}
\item[{Example}] \leavevmode
\begin{Verbatim}[commandchars=\\\{\}]
\PYG{g+gp}{\PYGZgt{}\PYGZgt{}\PYGZgt{} }\PYG{n}{vcs}\PYG{o}{.}\PYG{n}{show}\PYG{p}{(}\PYG{l+s+s1}{\PYGZsq{}}\PYG{l+s+s1}{isoline}\PYG{l+s+s1}{\PYGZsq{}}\PYG{p}{)} \PYG{c+c1}{\PYGZsh{} show all available isoline}
\PYG{g+go}{*******************Isoline Names List**********************}
\PYG{g+gp}{...}
\PYG{g+go}{*******************End Isoline Names List**********************}
\PYG{g+gp}{\PYGZgt{}\PYGZgt{}\PYGZgt{} }\PYG{n}{ex}\PYG{o}{=}\PYG{n}{vcs}\PYG{o}{.}\PYG{n}{createisoline}\PYG{p}{(}\PYG{l+s+s1}{\PYGZsq{}}\PYG{l+s+s1}{isoline\PYGZus{}ex1}\PYG{l+s+s1}{\PYGZsq{}}\PYG{p}{)} \PYG{c+c1}{\PYGZsh{} Create isoline \PYGZsq{}isoline\PYGZus{}ex1\PYGZsq{} that inherits from \PYGZsq{}default\PYGZsq{}}
\PYG{g+gp}{\PYGZgt{}\PYGZgt{}\PYGZgt{} }\PYG{n}{vcs}\PYG{o}{.}\PYG{n}{listelements}\PYG{p}{(}\PYG{l+s+s1}{\PYGZsq{}}\PYG{l+s+s1}{isoline}\PYG{l+s+s1}{\PYGZsq{}}\PYG{p}{)} \PYG{c+c1}{\PYGZsh{} should now contain the \PYGZsq{}isoline\PYGZus{}ex1\PYGZsq{} isoline}
\PYG{g+go}{[...\PYGZsq{}isoline\PYGZus{}ex1\PYGZsq{}...]}
\PYG{g+gp}{\PYGZgt{}\PYGZgt{}\PYGZgt{} }\PYG{n}{ex2}\PYG{o}{=}\PYG{n}{vcs}\PYG{o}{.}\PYG{n}{createisoline}\PYG{p}{(}\PYG{l+s+s1}{\PYGZsq{}}\PYG{l+s+s1}{isoline\PYGZus{}ex2}\PYG{l+s+s1}{\PYGZsq{}}\PYG{p}{,}\PYG{l+s+s1}{\PYGZsq{}}\PYG{l+s+s1}{polar}\PYG{l+s+s1}{\PYGZsq{}}\PYG{p}{)} \PYG{c+c1}{\PYGZsh{} create \PYGZsq{}isoline\PYGZus{}ex2\PYGZsq{} from \PYGZsq{}polar\PYGZsq{} template}
\PYG{g+gp}{\PYGZgt{}\PYGZgt{}\PYGZgt{} }\PYG{n}{vcs}\PYG{o}{.}\PYG{n}{listelements}\PYG{p}{(}\PYG{l+s+s1}{\PYGZsq{}}\PYG{l+s+s1}{isoline}\PYG{l+s+s1}{\PYGZsq{}}\PYG{p}{)} \PYG{c+c1}{\PYGZsh{} should now contain the \PYGZsq{}isoline\PYGZus{}ex2\PYGZsq{} isoline}
\PYG{g+go}{[...\PYGZsq{}isoline\PYGZus{}ex2\PYGZsq{}...]}
\end{Verbatim}

\item[{Parameters}] \leavevmode\begin{itemize}
\item {} 
\textbf{\texttt{name}} (\href{https://docs.python.org/2/library/functions.html\#str}{\emph{\texttt{str}}}) -- The name of the created object

\item {} 
\textbf{\texttt{source}} (\emph{\texttt{an isoline object, or string name of an isoline object}}) -- The object to inherit from

\item {} 
\textbf{\texttt{xaxis}} (\emph{\texttt{cdms2.axis.TransientAxis}}) -- Axis object to replace the slab -1 dim axis

\item {} 
\textbf{\texttt{yaxis}} (\emph{\texttt{cdms2.axis.TransientAxis}}) -- Axis object to replace the slab -2 dim axis, only if slab has more than 1D

\item {} 
\textbf{\texttt{zaxis}} (\emph{\texttt{cdms2.axis.TransientAxis}}) -- Axis object to replace the slab -3 dim axis, only if slab has more than 2D

\item {} 
\textbf{\texttt{taxis}} (\emph{\texttt{cdms2.axis.TransientAxis}}) -- Axis object to replace the slab -4 dim axis, only if slab has more than 3D

\item {} 
\textbf{\texttt{waxis}} (\emph{\texttt{cdms2.axis.TransientAxis}}) -- Axis object to replace the slab -5 dim axis, only if slab has more than 4D

\item {} 
\textbf{\texttt{xrev}} (\href{https://docs.python.org/2/library/functions.html\#bool}{\emph{\texttt{bool}}}) -- reverse x axis

\item {} 
\textbf{\texttt{yrev}} (\href{https://docs.python.org/2/library/functions.html\#bool}{\emph{\texttt{bool}}}) -- reverse y axis, only if slab has more than 1D

\item {} 
\textbf{\texttt{xarray}} (\href{https://docs.python.org/2/library/array.html\#module-array}{\emph{\texttt{array}}}) -- Values to use instead of x axis

\item {} 
\textbf{\texttt{yarray}} (\href{https://docs.python.org/2/library/array.html\#module-array}{\emph{\texttt{array}}}) -- Values to use instead of y axis, only if var has more than 1D

\item {} 
\textbf{\texttt{zarray}} (\href{https://docs.python.org/2/library/array.html\#module-array}{\emph{\texttt{array}}}) -- Values to use instead of z axis, only if var has more than 2D

\item {} 
\textbf{\texttt{tarray}} (\href{https://docs.python.org/2/library/array.html\#module-array}{\emph{\texttt{array}}}) -- Values to use instead of t axis, only if var has more than 3D

\item {} 
\textbf{\texttt{warray}} (\href{https://docs.python.org/2/library/array.html\#module-array}{\emph{\texttt{array}}}) -- Values to use instead of w axis, only if var has more than 4D

\item {} 
\textbf{\texttt{continents}} (\href{https://docs.python.org/2/library/functions.html\#int}{\emph{\texttt{int}}}) -- continents type number

\item {} 
\textbf{\texttt{name}} -- replaces variable name on plot

\item {} 
\textbf{\texttt{time}} (\emph{\texttt{A cdtime object}}) -- replaces time name on plot

\item {} 
\textbf{\texttt{units}} (\href{https://docs.python.org/2/library/functions.html\#str}{\emph{\texttt{str}}}) -- replaces units value on plot

\item {} 
\textbf{\texttt{ymd}} (\href{https://docs.python.org/2/library/functions.html\#str}{\emph{\texttt{str}}}) -- replaces year/month/day on plot

\item {} 
\textbf{\texttt{hms}} (\href{https://docs.python.org/2/library/functions.html\#str}{\emph{\texttt{str}}}) -- replaces hh/mm/ss on plot

\item {} 
\textbf{\texttt{file\_comment}} (\href{https://docs.python.org/2/library/functions.html\#str}{\emph{\texttt{str}}}) -- replaces file\_comment on plot

\item {} 
\textbf{\texttt{xbounds}} (\href{https://docs.python.org/2/library/array.html\#module-array}{\emph{\texttt{array}}}) -- Values to use instead of x axis bounds values

\item {} 
\textbf{\texttt{ybounds}} (\href{https://docs.python.org/2/library/array.html\#module-array}{\emph{\texttt{array}}}) -- Values to use instead of y axis bounds values (if exist)

\item {} 
\textbf{\texttt{xname}} (\href{https://docs.python.org/2/library/functions.html\#str}{\emph{\texttt{str}}}) -- replace xaxis name on plot

\item {} 
\textbf{\texttt{yname}} (\href{https://docs.python.org/2/library/functions.html\#str}{\emph{\texttt{str}}}) -- replace yaxis name on plot (if exists)

\item {} 
\textbf{\texttt{zname}} (\href{https://docs.python.org/2/library/functions.html\#str}{\emph{\texttt{str}}}) -- replace zaxis name on plot (if exists)

\item {} 
\textbf{\texttt{tname}} (\href{https://docs.python.org/2/library/functions.html\#str}{\emph{\texttt{str}}}) -- replace taxis name on plot (if exists)

\item {} 
\textbf{\texttt{wname}} (\href{https://docs.python.org/2/library/functions.html\#str}{\emph{\texttt{str}}}) -- replace waxis name on plot (if exists)

\item {} 
\textbf{\texttt{xunits}} (\href{https://docs.python.org/2/library/functions.html\#str}{\emph{\texttt{str}}}) -- replace xaxis units on plot

\item {} 
\textbf{\texttt{yunits}} (\href{https://docs.python.org/2/library/functions.html\#str}{\emph{\texttt{str}}}) -- replace yaxis units on plot (if exists)

\item {} 
\textbf{\texttt{zunits}} (\href{https://docs.python.org/2/library/functions.html\#str}{\emph{\texttt{str}}}) -- replace zaxis units on plot (if exists)

\item {} 
\textbf{\texttt{tunits}} (\href{https://docs.python.org/2/library/functions.html\#str}{\emph{\texttt{str}}}) -- replace taxis units on plot (if exists)

\item {} 
\textbf{\texttt{wunits}} (\href{https://docs.python.org/2/library/functions.html\#str}{\emph{\texttt{str}}}) -- replace waxis units on plot (if exists)

\item {} 
\textbf{\texttt{xweights}} (\href{https://docs.python.org/2/library/array.html\#module-array}{\emph{\texttt{array}}}) -- replace xaxis weights used for computing mean

\item {} 
\textbf{\texttt{yweights}} (\href{https://docs.python.org/2/library/array.html\#module-array}{\emph{\texttt{array}}}) -- replace xaxis weights used for computing mean

\item {} 
\textbf{\texttt{comment1}} (\href{https://docs.python.org/2/library/functions.html\#str}{\emph{\texttt{str}}}) -- replaces comment1 on plot

\item {} 
\textbf{\texttt{comment2}} (\href{https://docs.python.org/2/library/functions.html\#str}{\emph{\texttt{str}}}) -- replaces comment2 on plot

\item {} 
\textbf{\texttt{comment3}} (\href{https://docs.python.org/2/library/functions.html\#str}{\emph{\texttt{str}}}) -- replaces comment3 on plot

\item {} 
\textbf{\texttt{comment4}} (\href{https://docs.python.org/2/library/functions.html\#str}{\emph{\texttt{str}}}) -- replaces comment4 on plot

\item {} 
\textbf{\texttt{long\_name}} (\href{https://docs.python.org/2/library/functions.html\#str}{\emph{\texttt{str}}}) -- replaces long\_name on plot

\item {} 
\textbf{\texttt{grid}} (\emph{\texttt{cdms2.grid.TransientRectGrid}}) -- replaces array grid (if exists)

\item {} 
\textbf{\texttt{bg}} (\emph{\texttt{bool/int}}) -- plots in background mode

\item {} 
\textbf{\texttt{ratio}} (\index{xmtics1 (vcs.Canvas.Canvas attribute)}\index{xmtics2 (vcs.Canvas.Canvas attribute)}\index{ymtics1 (vcs.Canvas.Canvas attribute)}\index{ymtics2 (vcs.Canvas.Canvas attribute)}\index{xticlabels1 (vcs.Canvas.Canvas attribute)}\index{xticlabels2 (vcs.Canvas.Canvas attribute)}\index{yticlabels1 (vcs.Canvas.Canvas attribute)}\index{yticlabels2 (vcs.Canvas.Canvas attribute)}\index{projection (vcs.Canvas.Canvas attribute)}\index{datawc\_x1 (vcs.Canvas.Canvas attribute)}\index{datawc\_x2 (vcs.Canvas.Canvas attribute)}\index{datawc\_y1 (vcs.Canvas.Canvas attribute)}\index{datawc\_y2 (vcs.Canvas.Canvas attribute)}\index{datawc\_timeunits (vcs.Canvas.Canvas attribute)}\index{datawc\_calendar (vcs.Canvas.Canvas attribute)}) -- sets the y/x ratio ,if passed as a string with `t' at the end, will aslo moves the ticks

\item {} 
\textbf{\texttt{xaxisconvert}} (\href{https://docs.python.org/2/library/functions.html\#str}{\emph{\texttt{str}}}) -- (Ex: `linear') converting xaxis linear/log/log10/ln/exp/area\_wt

\item {} 
\textbf{\texttt{yaxisconvert}} (\href{https://docs.python.org/2/library/functions.html\#str}{\emph{\texttt{str}}}) -- (Ex: `linear') converting yaxis linear/log/log10/ln/exp/area\_wt

\item {} 
\textbf{\texttt{new\_GM\_name}} (\href{https://docs.python.org/2/library/functions.html\#str}{\emph{\texttt{str}}}) -- (Ex: `my\_awesome\_gm') name of the new graphics method object. If no name is given, then one will be created for use.

\item {} 
\textbf{\texttt{source\_GM\_name}} -- (Ex: `default') copy the contents of the source object to the newly created one. If no name is given, then the `default' graphics methond contents is copied over to the new object.

\end{itemize}

\item[{Returns}] \leavevmode
An isoline graphics method object

\item[{Return type}] \leavevmode
{\hyperref[vcs/graphics/isoline:vcs.isoline.Gi]{\sphinxcrossref{vcs.isoline.Gi}}}

\end{description}\end{quote}

\end{fulllineitems}

\index{createline() (vcs.Canvas.Canvas method)}

\begin{fulllineitems}
\phantomsection\label{vcs/Canvas:vcs.Canvas.Canvas.createline}\pysiglinewithargsret{\sphinxbfcode{createline}}{\emph{name=None}, \emph{source='default'}, \emph{ltype=None}, \emph{width=None}, \emph{color=None}, \emph{priority=None}, \emph{viewport=None}, \emph{worldcoordinate=None}, \emph{x=None}, \emph{y=None}, \emph{projection=None}}{}
Create a new line secondary method given the the name and the existing
line secondary method to copy the attributes from. If no existing
line secondary method is given, then the default line secondary method will be used as the graphics method
to which the attributes will be copied from.

\begin{notice}{note}{Note:}
If the name provided already exists, then an error will be returned. secondary method
names must be unique.
\end{notice}
\begin{quote}\begin{description}
\item[{Example}] \leavevmode
\begin{Verbatim}[commandchars=\\\{\}]
\PYG{g+gp}{\PYGZgt{}\PYGZgt{}\PYGZgt{} }\PYG{n}{vcs}\PYG{o}{.}\PYG{n}{show}\PYG{p}{(}\PYG{l+s+s1}{\PYGZsq{}}\PYG{l+s+s1}{line}\PYG{l+s+s1}{\PYGZsq{}}\PYG{p}{)} \PYG{c+c1}{\PYGZsh{} show all available line}
\PYG{g+go}{*******************Line Names List**********************}
\PYG{g+gp}{...}
\PYG{g+go}{*******************End Line Names List**********************}
\PYG{g+gp}{\PYGZgt{}\PYGZgt{}\PYGZgt{} }\PYG{n}{ex}\PYG{o}{=}\PYG{n}{vcs}\PYG{o}{.}\PYG{n}{createline}\PYG{p}{(}\PYG{l+s+s1}{\PYGZsq{}}\PYG{l+s+s1}{line\PYGZus{}ex1}\PYG{l+s+s1}{\PYGZsq{}}\PYG{p}{)} \PYG{c+c1}{\PYGZsh{} Create line \PYGZsq{}line\PYGZus{}ex1\PYGZsq{} that inherits from \PYGZsq{}default\PYGZsq{}}
\PYG{g+gp}{\PYGZgt{}\PYGZgt{}\PYGZgt{} }\PYG{n}{vcs}\PYG{o}{.}\PYG{n}{listelements}\PYG{p}{(}\PYG{l+s+s1}{\PYGZsq{}}\PYG{l+s+s1}{line}\PYG{l+s+s1}{\PYGZsq{}}\PYG{p}{)} \PYG{c+c1}{\PYGZsh{} should now contain the \PYGZsq{}line\PYGZus{}ex1\PYGZsq{} line}
\PYG{g+go}{[...\PYGZsq{}line\PYGZus{}ex1\PYGZsq{}...]}
\PYG{g+gp}{\PYGZgt{}\PYGZgt{}\PYGZgt{} }\PYG{n}{ex2}\PYG{o}{=}\PYG{n}{vcs}\PYG{o}{.}\PYG{n}{createline}\PYG{p}{(}\PYG{l+s+s1}{\PYGZsq{}}\PYG{l+s+s1}{line\PYGZus{}ex2}\PYG{l+s+s1}{\PYGZsq{}}\PYG{p}{,}\PYG{l+s+s1}{\PYGZsq{}}\PYG{l+s+s1}{red}\PYG{l+s+s1}{\PYGZsq{}}\PYG{p}{)} \PYG{c+c1}{\PYGZsh{} create \PYGZsq{}line\PYGZus{}ex2\PYGZsq{} from \PYGZsq{}red\PYGZsq{} template}
\PYG{g+gp}{\PYGZgt{}\PYGZgt{}\PYGZgt{} }\PYG{n}{vcs}\PYG{o}{.}\PYG{n}{listelements}\PYG{p}{(}\PYG{l+s+s1}{\PYGZsq{}}\PYG{l+s+s1}{line}\PYG{l+s+s1}{\PYGZsq{}}\PYG{p}{)} \PYG{c+c1}{\PYGZsh{} should now contain the \PYGZsq{}line\PYGZus{}ex2\PYGZsq{} line}
\PYG{g+go}{[...\PYGZsq{}line\PYGZus{}ex2\PYGZsq{}...]}
\end{Verbatim}

\item[{Parameters}] \leavevmode\begin{itemize}
\item {} 
\textbf{\texttt{name}} (\href{https://docs.python.org/2/library/functions.html\#str}{\emph{\texttt{str}}}) -- Name of created object

\item {} 
\textbf{\texttt{source}} (\href{https://docs.python.org/2/library/functions.html\#str}{\emph{\texttt{str}}}) -- a line, or string name of a line

\item {} 
\textbf{\texttt{ltype}} (\href{https://docs.python.org/2/library/functions.html\#str}{\emph{\texttt{str}}}) -- One of ``dash'', ``dash-dot'', ``solid'', ``dot'', or ``long-dash''.

\item {} 
\textbf{\texttt{width}} (\href{https://docs.python.org/2/library/functions.html\#int}{\emph{\texttt{int}}}) -- Thickness of the line to be created

\item {} 
\textbf{\texttt{color}} (\emph{\texttt{str or int}}) -- 
A color name from the \href{https://en.wikipedia.org/wiki/X11\_color\_names}{X11 Color Names list},
or an integer value from 0-255, or an RGB/RGBA tuple/list (e.g. (0,100,0), (100,100,0,50))


\item {} 
\textbf{\texttt{priority}} (\href{https://docs.python.org/2/library/functions.html\#int}{\emph{\texttt{int}}}) -- The layer on which the line will be drawn.

\item {} 
\textbf{\texttt{viewport}} (\emph{\texttt{list of floats}}) -- 4 floats between 0 and 1. These specify the area that the X/Y values are mapped to inside of the canvas

\item {} 
\textbf{\texttt{worldcoordinate}} (\emph{\texttt{list of floats}}) -- List of 4 floats (xmin, xmax, ymin, ymax)

\item {} 
\textbf{\texttt{x}} (\emph{\texttt{list of floats}}) -- List of lists of x coordinates. Values must be between worldcoordinate{[}0{]} and worldcoordinate{[}1{]}.

\item {} 
\textbf{\texttt{y}} (\emph{\texttt{list of floats}}) -- List of lists of y coordinates. Values must be between worldcoordinate{[}2{]} and worldcoordinate{[}3{]}.

\item {} 
\textbf{\texttt{projection}} (\emph{\texttt{str or projection object}}) -- Specify a geographic projection used to convert x/y from spherical coordinates into 2D coordinates.

\end{itemize}

\item[{Returns}] \leavevmode
A VCS line secondary method object

\item[{Return type}] \leavevmode
{\hyperref[vcs/secondary/line:vcs.line.Tl]{\sphinxcrossref{vcs.line.Tl}}}

\end{description}\end{quote}

\end{fulllineitems}

\index{createmarker() (vcs.Canvas.Canvas method)}

\begin{fulllineitems}
\phantomsection\label{vcs/Canvas:vcs.Canvas.Canvas.createmarker}\pysiglinewithargsret{\sphinxbfcode{createmarker}}{\emph{name=None}, \emph{source='default'}, \emph{mtype=None}, \emph{size=None}, \emph{color=None}, \emph{priority=1}, \emph{viewport=None}, \emph{worldcoordinate=None}, \emph{x=None}, \emph{y=None}, \emph{projection=None}}{}
Create a new marker secondary method given the the name and the existing
marker secondary method to copy the attributes from. If no existing
marker secondary method is given, then the default marker secondary method will be used as the graphics method
to which the attributes will be copied from.

\begin{notice}{note}{Note:}
If the name provided already exists, then an error will be returned. secondary method
names must be unique.
\end{notice}
\begin{quote}\begin{description}
\item[{Example}] \leavevmode
\begin{Verbatim}[commandchars=\\\{\}]
\PYG{g+gp}{\PYGZgt{}\PYGZgt{}\PYGZgt{} }\PYG{n}{vcs}\PYG{o}{.}\PYG{n}{show}\PYG{p}{(}\PYG{l+s+s1}{\PYGZsq{}}\PYG{l+s+s1}{marker}\PYG{l+s+s1}{\PYGZsq{}}\PYG{p}{)} \PYG{c+c1}{\PYGZsh{} show all available marker}
\PYG{g+go}{*******************Marker Names List**********************}
\PYG{g+gp}{...}
\PYG{g+go}{*******************End Marker Names List**********************}
\PYG{g+gp}{\PYGZgt{}\PYGZgt{}\PYGZgt{} }\PYG{n}{ex}\PYG{o}{=}\PYG{n}{vcs}\PYG{o}{.}\PYG{n}{createmarker}\PYG{p}{(}\PYG{l+s+s1}{\PYGZsq{}}\PYG{l+s+s1}{marker\PYGZus{}ex1}\PYG{l+s+s1}{\PYGZsq{}}\PYG{p}{)} \PYG{c+c1}{\PYGZsh{} Create marker \PYGZsq{}marker\PYGZus{}ex1\PYGZsq{} that inherits from \PYGZsq{}default\PYGZsq{}}
\PYG{g+gp}{\PYGZgt{}\PYGZgt{}\PYGZgt{} }\PYG{n}{vcs}\PYG{o}{.}\PYG{n}{listelements}\PYG{p}{(}\PYG{l+s+s1}{\PYGZsq{}}\PYG{l+s+s1}{marker}\PYG{l+s+s1}{\PYGZsq{}}\PYG{p}{)} \PYG{c+c1}{\PYGZsh{} should now contain the \PYGZsq{}marker\PYGZus{}ex1\PYGZsq{} marker}
\PYG{g+go}{[...\PYGZsq{}marker\PYGZus{}ex1\PYGZsq{}...]}
\PYG{g+gp}{\PYGZgt{}\PYGZgt{}\PYGZgt{} }\PYG{n}{ex2}\PYG{o}{=}\PYG{n}{vcs}\PYG{o}{.}\PYG{n}{createmarker}\PYG{p}{(}\PYG{l+s+s1}{\PYGZsq{}}\PYG{l+s+s1}{marker\PYGZus{}ex2}\PYG{l+s+s1}{\PYGZsq{}}\PYG{p}{,}\PYG{l+s+s1}{\PYGZsq{}}\PYG{l+s+s1}{red}\PYG{l+s+s1}{\PYGZsq{}}\PYG{p}{)} \PYG{c+c1}{\PYGZsh{} create \PYGZsq{}marker\PYGZus{}ex2\PYGZsq{} from \PYGZsq{}red\PYGZsq{} template}
\PYG{g+gp}{\PYGZgt{}\PYGZgt{}\PYGZgt{} }\PYG{n}{vcs}\PYG{o}{.}\PYG{n}{listelements}\PYG{p}{(}\PYG{l+s+s1}{\PYGZsq{}}\PYG{l+s+s1}{marker}\PYG{l+s+s1}{\PYGZsq{}}\PYG{p}{)} \PYG{c+c1}{\PYGZsh{} should now contain the \PYGZsq{}marker\PYGZus{}ex2\PYGZsq{} marker}
\PYG{g+go}{[...\PYGZsq{}marker\PYGZus{}ex2\PYGZsq{}...]}
\end{Verbatim}

\item[{Parameters}] \leavevmode\begin{itemize}
\item {} 
\textbf{\texttt{name}} (\href{https://docs.python.org/2/library/functions.html\#str}{\emph{\texttt{str}}}) -- Name of created object

\item {} 
\textbf{\texttt{source}} (\href{https://docs.python.org/2/library/functions.html\#str}{\emph{\texttt{str}}}) -- A marker, or string name of a marker

\item {} 
\textbf{\texttt{mtype}} (\href{https://docs.python.org/2/library/functions.html\#str}{\emph{\texttt{str}}}) -- Specifies the type of marker, i.e. ``dot'', ``circle''

\item {} 
\textbf{\texttt{size}} (\href{https://docs.python.org/2/library/functions.html\#int}{\emph{\texttt{int}}}) -- 

\item {} 
\textbf{\texttt{color}} (\emph{\texttt{str or int}}) -- 
A color name from the \href{https://en.wikipedia.org/wiki/X11\_color\_names}{X11 Color Names list},
or an integer value from 0-255, or an RGB/RGBA tuple/list (e.g. (0,100,0), (100,100,0,50))


\item {} 
\textbf{\texttt{priority}} (\href{https://docs.python.org/2/library/functions.html\#int}{\emph{\texttt{int}}}) -- The layer on which the marker will be drawn.

\item {} 
\textbf{\texttt{viewport}} (\emph{\texttt{list of floats}}) -- 4 floats between 0 and 1. These specify the area that the X/Y values are mapped to inside of the canvas

\item {} 
\textbf{\texttt{worldcoordinate}} (\emph{\texttt{list of floats}}) -- List of 4 floats (xmin, xmax, ymin, ymax)

\item {} 
\textbf{\texttt{x}} (\emph{\texttt{list of floats}}) -- List of lists of x coordinates. Values must be between worldcoordinate{[}0{]} and worldcoordinate{[}1{]}.

\item {} 
\textbf{\texttt{y}} (\emph{\texttt{list of floats}}) -- List of lists of y coordinates. Values must be between worldcoordinate{[}2{]} and worldcoordinate{[}3{]}.

\end{itemize}

\item[{Returns}] \leavevmode
A secondary marker method

\item[{Return type}] \leavevmode
{\hyperref[vcs/secondary/marker:vcs.marker.Tm]{\sphinxcrossref{vcs.marker.Tm}}}

\end{description}\end{quote}

\end{fulllineitems}

\index{createmeshfill() (vcs.Canvas.Canvas method)}

\begin{fulllineitems}
\phantomsection\label{vcs/Canvas:vcs.Canvas.Canvas.createmeshfill}\pysiglinewithargsret{\sphinxbfcode{createmeshfill}}{\emph{name=None}, \emph{source='default'}}{}
Create a new meshfill graphics method given the the name and the existing
meshfill graphics method to copy the attributes from. If no existing
meshfill graphics method is given, then the default meshfill graphics method will be used as the graphics method
to which the attributes will be copied from.

\begin{notice}{note}{Note:}
If the name provided already exists, then an error will be returned. graphics method
names must be unique.
\end{notice}
\begin{quote}\begin{description}
\item[{Example}] \leavevmode
\begin{Verbatim}[commandchars=\\\{\}]
\PYG{g+gp}{\PYGZgt{}\PYGZgt{}\PYGZgt{} }\PYG{n}{vcs}\PYG{o}{.}\PYG{n}{show}\PYG{p}{(}\PYG{l+s+s1}{\PYGZsq{}}\PYG{l+s+s1}{meshfill}\PYG{l+s+s1}{\PYGZsq{}}\PYG{p}{)} \PYG{c+c1}{\PYGZsh{} show all available meshfill}
\PYG{g+go}{*******************Meshfill Names List**********************}
\PYG{g+gp}{...}
\PYG{g+go}{*******************End Meshfill Names List**********************}
\PYG{g+gp}{\PYGZgt{}\PYGZgt{}\PYGZgt{} }\PYG{n}{ex}\PYG{o}{=}\PYG{n}{vcs}\PYG{o}{.}\PYG{n}{createmeshfill}\PYG{p}{(}\PYG{l+s+s1}{\PYGZsq{}}\PYG{l+s+s1}{meshfill\PYGZus{}ex1}\PYG{l+s+s1}{\PYGZsq{}}\PYG{p}{)} \PYG{c+c1}{\PYGZsh{} Create meshfill \PYGZsq{}meshfill\PYGZus{}ex1\PYGZsq{} that inherits from \PYGZsq{}default\PYGZsq{}}
\PYG{g+gp}{\PYGZgt{}\PYGZgt{}\PYGZgt{} }\PYG{n}{vcs}\PYG{o}{.}\PYG{n}{listelements}\PYG{p}{(}\PYG{l+s+s1}{\PYGZsq{}}\PYG{l+s+s1}{meshfill}\PYG{l+s+s1}{\PYGZsq{}}\PYG{p}{)} \PYG{c+c1}{\PYGZsh{} should now contain the \PYGZsq{}meshfill\PYGZus{}ex1\PYGZsq{} meshfill}
\PYG{g+go}{[...\PYGZsq{}meshfill\PYGZus{}ex1\PYGZsq{}...]}
\PYG{g+gp}{\PYGZgt{}\PYGZgt{}\PYGZgt{} }\PYG{n}{ex2}\PYG{o}{=}\PYG{n}{vcs}\PYG{o}{.}\PYG{n}{createmeshfill}\PYG{p}{(}\PYG{l+s+s1}{\PYGZsq{}}\PYG{l+s+s1}{meshfill\PYGZus{}ex2}\PYG{l+s+s1}{\PYGZsq{}}\PYG{p}{,}\PYG{l+s+s1}{\PYGZsq{}}\PYG{l+s+s1}{a\PYGZus{}polar\PYGZus{}meshfill}\PYG{l+s+s1}{\PYGZsq{}}\PYG{p}{)} \PYG{c+c1}{\PYGZsh{} create \PYGZsq{}meshfill\PYGZus{}ex2\PYGZsq{} from \PYGZsq{}a\PYGZus{}polar\PYGZus{}meshfill\PYGZsq{} template}
\PYG{g+gp}{\PYGZgt{}\PYGZgt{}\PYGZgt{} }\PYG{n}{vcs}\PYG{o}{.}\PYG{n}{listelements}\PYG{p}{(}\PYG{l+s+s1}{\PYGZsq{}}\PYG{l+s+s1}{meshfill}\PYG{l+s+s1}{\PYGZsq{}}\PYG{p}{)} \PYG{c+c1}{\PYGZsh{} should now contain the \PYGZsq{}meshfill\PYGZus{}ex2\PYGZsq{} meshfill}
\PYG{g+go}{[...\PYGZsq{}meshfill\PYGZus{}ex2\PYGZsq{}...]}
\end{Verbatim}

\item[{Parameters}] \leavevmode\begin{itemize}
\item {} 
\textbf{\texttt{name}} (\href{https://docs.python.org/2/library/functions.html\#str}{\emph{\texttt{str}}}) -- The name of the created object

\item {} 
\textbf{\texttt{source}} (\emph{\texttt{a meshfill or a string name of a meshfill}}) -- The object to inherit from

\end{itemize}

\item[{Returns}] \leavevmode
A meshfill graphics method object

\item[{Return type}] \leavevmode
{\hyperref[vcs/graphics/meshfill:vcs.meshfill.Gfm]{\sphinxcrossref{vcs.meshfill.Gfm}}}

\end{description}\end{quote}

\end{fulllineitems}

\index{createprojection() (vcs.Canvas.Canvas method)}

\begin{fulllineitems}
\phantomsection\label{vcs/Canvas:vcs.Canvas.Canvas.createprojection}\pysiglinewithargsret{\sphinxbfcode{createprojection}}{\emph{name=None}, \emph{source='default'}}{}
Create a new projection graphics method given the the name and the existing
projection graphics method to copy the attributes from. If no existing
projection graphics method is given, then the default projection graphics method will be used as the graphics method
to which the attributes will be copied from.

\begin{notice}{note}{Note:}
If the name provided already exists, then an error will be returned. graphics method
names must be unique.
\end{notice}
\begin{quote}\begin{description}
\item[{Example}] \leavevmode
\begin{Verbatim}[commandchars=\\\{\}]
\PYG{g+gp}{\PYGZgt{}\PYGZgt{}\PYGZgt{} }\PYG{n}{vcs}\PYG{o}{.}\PYG{n}{show}\PYG{p}{(}\PYG{l+s+s1}{\PYGZsq{}}\PYG{l+s+s1}{projection}\PYG{l+s+s1}{\PYGZsq{}}\PYG{p}{)} \PYG{c+c1}{\PYGZsh{} show all available projection}
\PYG{g+go}{*******************Projection Names List**********************}
\PYG{g+gp}{...}
\PYG{g+go}{*******************End Projection Names List**********************}
\PYG{g+gp}{\PYGZgt{}\PYGZgt{}\PYGZgt{} }\PYG{n}{ex}\PYG{o}{=}\PYG{n}{vcs}\PYG{o}{.}\PYG{n}{createprojection}\PYG{p}{(}\PYG{l+s+s1}{\PYGZsq{}}\PYG{l+s+s1}{projection\PYGZus{}ex1}\PYG{l+s+s1}{\PYGZsq{}}\PYG{p}{)} \PYG{c+c1}{\PYGZsh{} Create projection \PYGZsq{}projection\PYGZus{}ex1\PYGZsq{} that inherits from \PYGZsq{}default\PYGZsq{}}
\PYG{g+gp}{\PYGZgt{}\PYGZgt{}\PYGZgt{} }\PYG{n}{vcs}\PYG{o}{.}\PYG{n}{listelements}\PYG{p}{(}\PYG{l+s+s1}{\PYGZsq{}}\PYG{l+s+s1}{projection}\PYG{l+s+s1}{\PYGZsq{}}\PYG{p}{)} \PYG{c+c1}{\PYGZsh{} should now contain the \PYGZsq{}projection\PYGZus{}ex1\PYGZsq{} projection}
\PYG{g+go}{[...\PYGZsq{}projection\PYGZus{}ex1\PYGZsq{}...]}
\PYG{g+gp}{\PYGZgt{}\PYGZgt{}\PYGZgt{} }\PYG{n}{ex2}\PYG{o}{=}\PYG{n}{vcs}\PYG{o}{.}\PYG{n}{createprojection}\PYG{p}{(}\PYG{l+s+s1}{\PYGZsq{}}\PYG{l+s+s1}{projection\PYGZus{}ex2}\PYG{l+s+s1}{\PYGZsq{}}\PYG{p}{,}\PYG{l+s+s1}{\PYGZsq{}}\PYG{l+s+s1}{polar}\PYG{l+s+s1}{\PYGZsq{}}\PYG{p}{)} \PYG{c+c1}{\PYGZsh{} create \PYGZsq{}projection\PYGZus{}ex2\PYGZsq{} from \PYGZsq{}polar\PYGZsq{} template}
\PYG{g+gp}{\PYGZgt{}\PYGZgt{}\PYGZgt{} }\PYG{n}{vcs}\PYG{o}{.}\PYG{n}{listelements}\PYG{p}{(}\PYG{l+s+s1}{\PYGZsq{}}\PYG{l+s+s1}{projection}\PYG{l+s+s1}{\PYGZsq{}}\PYG{p}{)} \PYG{c+c1}{\PYGZsh{} should now contain the \PYGZsq{}projection\PYGZus{}ex2\PYGZsq{} projection}
\PYG{g+go}{[...\PYGZsq{}projection\PYGZus{}ex2\PYGZsq{}...]}
\end{Verbatim}

\item[{Parameters}] \leavevmode\begin{itemize}
\item {} 
\textbf{\texttt{name}} (\href{https://docs.python.org/2/library/functions.html\#str}{\emph{\texttt{str}}}) -- The name of the created object

\item {} 
\textbf{\texttt{source}} (\emph{\texttt{a projection or a string name of a projection}}) -- The object to inherit from

\end{itemize}

\item[{Returns}] \leavevmode
A projection graphics method object

\item[{Return type}] \leavevmode
{\hyperref[vcs/misc/projection:vcs.projection.Proj]{\sphinxcrossref{vcs.projection.Proj}}}

\end{description}\end{quote}

\end{fulllineitems}

\index{createscatter() (vcs.Canvas.Canvas method)}

\begin{fulllineitems}
\phantomsection\label{vcs/Canvas:vcs.Canvas.Canvas.createscatter}\pysiglinewithargsret{\sphinxbfcode{createscatter}}{\emph{name=None}, \emph{source='default'}}{}
Create a new scatter graphics method given the the name and the existing
scatter graphics method to copy the attributes from. If no existing
scatter graphics method is given, then the default scatter graphics method will be used as the graphics method
to which the attributes will be copied from.

\begin{notice}{note}{Note:}
If the name provided already exists, then an error will be returned. graphics method
names must be unique.
\end{notice}
\begin{quote}\begin{description}
\item[{Example}] \leavevmode
\begin{Verbatim}[commandchars=\\\{\}]
\PYG{g+gp}{\PYGZgt{}\PYGZgt{}\PYGZgt{} }\PYG{n}{vcs}\PYG{o}{.}\PYG{n}{show}\PYG{p}{(}\PYG{l+s+s1}{\PYGZsq{}}\PYG{l+s+s1}{scatter}\PYG{l+s+s1}{\PYGZsq{}}\PYG{p}{)} \PYG{c+c1}{\PYGZsh{} show all available scatter}
\PYG{g+go}{*******************Scatter Names List**********************}
\PYG{g+gp}{...}
\PYG{g+go}{*******************End Scatter Names List**********************}
\PYG{g+gp}{\PYGZgt{}\PYGZgt{}\PYGZgt{} }\PYG{n}{ex}\PYG{o}{=}\PYG{n}{vcs}\PYG{o}{.}\PYG{n}{createscatter}\PYG{p}{(}\PYG{l+s+s1}{\PYGZsq{}}\PYG{l+s+s1}{scatter\PYGZus{}ex1}\PYG{l+s+s1}{\PYGZsq{}}\PYG{p}{)} \PYG{c+c1}{\PYGZsh{} Create scatter \PYGZsq{}scatter\PYGZus{}ex1\PYGZsq{} that inherits from \PYGZsq{}default\PYGZsq{}}
\PYG{g+gp}{\PYGZgt{}\PYGZgt{}\PYGZgt{} }\PYG{n}{vcs}\PYG{o}{.}\PYG{n}{listelements}\PYG{p}{(}\PYG{l+s+s1}{\PYGZsq{}}\PYG{l+s+s1}{scatter}\PYG{l+s+s1}{\PYGZsq{}}\PYG{p}{)} \PYG{c+c1}{\PYGZsh{} should now contain the \PYGZsq{}scatter\PYGZus{}ex1\PYGZsq{} scatter}
\PYG{g+go}{[...\PYGZsq{}scatter\PYGZus{}ex1\PYGZsq{}...]}
\end{Verbatim}

\item[{Parameters}] \leavevmode\begin{itemize}
\item {} 
\textbf{\texttt{name}} (\href{https://docs.python.org/2/library/functions.html\#str}{\emph{\texttt{str}}}) -- The name of the created object

\item {} 
\textbf{\texttt{source}} (\emph{\texttt{a scatter or a string name of a scatter}}) -- The object to inherit from

\item {} 
\textbf{\texttt{xaxis}} (\emph{\texttt{cdms2.axis.TransientAxis}}) -- Axis object to replace the slab -1 dim axis

\item {} 
\textbf{\texttt{yaxis}} (\emph{\texttt{cdms2.axis.TransientAxis}}) -- Axis object to replace the slab -2 dim axis, only if slab has more than 1D

\item {} 
\textbf{\texttt{zaxis}} (\emph{\texttt{cdms2.axis.TransientAxis}}) -- Axis object to replace the slab -3 dim axis, only if slab has more than 2D

\item {} 
\textbf{\texttt{taxis}} (\emph{\texttt{cdms2.axis.TransientAxis}}) -- Axis object to replace the slab -4 dim axis, only if slab has more than 3D

\item {} 
\textbf{\texttt{waxis}} (\emph{\texttt{cdms2.axis.TransientAxis}}) -- Axis object to replace the slab -5 dim axis, only if slab has more than 4D

\item {} 
\textbf{\texttt{xrev}} (\href{https://docs.python.org/2/library/functions.html\#bool}{\emph{\texttt{bool}}}) -- reverse x axis

\item {} 
\textbf{\texttt{yrev}} (\href{https://docs.python.org/2/library/functions.html\#bool}{\emph{\texttt{bool}}}) -- reverse y axis, only if slab has more than 1D

\item {} 
\textbf{\texttt{xarray}} (\href{https://docs.python.org/2/library/array.html\#module-array}{\emph{\texttt{array}}}) -- Values to use instead of x axis

\item {} 
\textbf{\texttt{yarray}} (\href{https://docs.python.org/2/library/array.html\#module-array}{\emph{\texttt{array}}}) -- Values to use instead of y axis, only if var has more than 1D

\item {} 
\textbf{\texttt{zarray}} (\href{https://docs.python.org/2/library/array.html\#module-array}{\emph{\texttt{array}}}) -- Values to use instead of z axis, only if var has more than 2D

\item {} 
\textbf{\texttt{tarray}} (\href{https://docs.python.org/2/library/array.html\#module-array}{\emph{\texttt{array}}}) -- Values to use instead of t axis, only if var has more than 3D

\item {} 
\textbf{\texttt{warray}} (\href{https://docs.python.org/2/library/array.html\#module-array}{\emph{\texttt{array}}}) -- Values to use instead of w axis, only if var has more than 4D

\item {} 
\textbf{\texttt{continents}} (\href{https://docs.python.org/2/library/functions.html\#int}{\emph{\texttt{int}}}) -- continents type number

\item {} 
\textbf{\texttt{name}} -- replaces variable name on plot

\item {} 
\textbf{\texttt{time}} (\emph{\texttt{A cdtime object}}) -- replaces time name on plot

\item {} 
\textbf{\texttt{units}} (\href{https://docs.python.org/2/library/functions.html\#str}{\emph{\texttt{str}}}) -- replaces units value on plot

\item {} 
\textbf{\texttt{ymd}} (\href{https://docs.python.org/2/library/functions.html\#str}{\emph{\texttt{str}}}) -- replaces year/month/day on plot

\item {} 
\textbf{\texttt{hms}} (\href{https://docs.python.org/2/library/functions.html\#str}{\emph{\texttt{str}}}) -- replaces hh/mm/ss on plot

\item {} 
\textbf{\texttt{file\_comment}} (\href{https://docs.python.org/2/library/functions.html\#str}{\emph{\texttt{str}}}) -- replaces file\_comment on plot

\item {} 
\textbf{\texttt{xbounds}} (\href{https://docs.python.org/2/library/array.html\#module-array}{\emph{\texttt{array}}}) -- Values to use instead of x axis bounds values

\item {} 
\textbf{\texttt{ybounds}} (\href{https://docs.python.org/2/library/array.html\#module-array}{\emph{\texttt{array}}}) -- Values to use instead of y axis bounds values (if exist)

\item {} 
\textbf{\texttt{xname}} (\href{https://docs.python.org/2/library/functions.html\#str}{\emph{\texttt{str}}}) -- replace xaxis name on plot

\item {} 
\textbf{\texttt{yname}} (\href{https://docs.python.org/2/library/functions.html\#str}{\emph{\texttt{str}}}) -- replace yaxis name on plot (if exists)

\item {} 
\textbf{\texttt{zname}} (\href{https://docs.python.org/2/library/functions.html\#str}{\emph{\texttt{str}}}) -- replace zaxis name on plot (if exists)

\item {} 
\textbf{\texttt{tname}} (\href{https://docs.python.org/2/library/functions.html\#str}{\emph{\texttt{str}}}) -- replace taxis name on plot (if exists)

\item {} 
\textbf{\texttt{wname}} (\href{https://docs.python.org/2/library/functions.html\#str}{\emph{\texttt{str}}}) -- replace waxis name on plot (if exists)

\item {} 
\textbf{\texttt{xunits}} (\href{https://docs.python.org/2/library/functions.html\#str}{\emph{\texttt{str}}}) -- replace xaxis units on plot

\item {} 
\textbf{\texttt{yunits}} (\href{https://docs.python.org/2/library/functions.html\#str}{\emph{\texttt{str}}}) -- replace yaxis units on plot (if exists)

\item {} 
\textbf{\texttt{zunits}} (\href{https://docs.python.org/2/library/functions.html\#str}{\emph{\texttt{str}}}) -- replace zaxis units on plot (if exists)

\item {} 
\textbf{\texttt{tunits}} (\href{https://docs.python.org/2/library/functions.html\#str}{\emph{\texttt{str}}}) -- replace taxis units on plot (if exists)

\item {} 
\textbf{\texttt{wunits}} (\href{https://docs.python.org/2/library/functions.html\#str}{\emph{\texttt{str}}}) -- replace waxis units on plot (if exists)

\item {} 
\textbf{\texttt{xweights}} (\href{https://docs.python.org/2/library/array.html\#module-array}{\emph{\texttt{array}}}) -- replace xaxis weights used for computing mean

\item {} 
\textbf{\texttt{yweights}} (\href{https://docs.python.org/2/library/array.html\#module-array}{\emph{\texttt{array}}}) -- replace xaxis weights used for computing mean

\item {} 
\textbf{\texttt{comment1}} (\href{https://docs.python.org/2/library/functions.html\#str}{\emph{\texttt{str}}}) -- replaces comment1 on plot

\item {} 
\textbf{\texttt{comment2}} (\href{https://docs.python.org/2/library/functions.html\#str}{\emph{\texttt{str}}}) -- replaces comment2 on plot

\item {} 
\textbf{\texttt{comment3}} (\href{https://docs.python.org/2/library/functions.html\#str}{\emph{\texttt{str}}}) -- replaces comment3 on plot

\item {} 
\textbf{\texttt{comment4}} (\href{https://docs.python.org/2/library/functions.html\#str}{\emph{\texttt{str}}}) -- replaces comment4 on plot

\item {} 
\textbf{\texttt{long\_name}} (\href{https://docs.python.org/2/library/functions.html\#str}{\emph{\texttt{str}}}) -- replaces long\_name on plot

\item {} 
\textbf{\texttt{grid}} (\emph{\texttt{cdms2.grid.TransientRectGrid}}) -- replaces array grid (if exists)

\item {} 
\textbf{\texttt{bg}} (\emph{\texttt{bool/int}}) -- plots in background mode

\item {} 
\textbf{\texttt{ratio}} (\index{xmtics1 (vcs.Canvas.Canvas attribute)}\index{xmtics2 (vcs.Canvas.Canvas attribute)}\index{ymtics1 (vcs.Canvas.Canvas attribute)}\index{ymtics2 (vcs.Canvas.Canvas attribute)}\index{xticlabels1 (vcs.Canvas.Canvas attribute)}\index{xticlabels2 (vcs.Canvas.Canvas attribute)}\index{yticlabels1 (vcs.Canvas.Canvas attribute)}\index{yticlabels2 (vcs.Canvas.Canvas attribute)}\index{projection (vcs.Canvas.Canvas attribute)}\index{datawc\_x1 (vcs.Canvas.Canvas attribute)}\index{datawc\_x2 (vcs.Canvas.Canvas attribute)}\index{datawc\_y1 (vcs.Canvas.Canvas attribute)}\index{datawc\_y2 (vcs.Canvas.Canvas attribute)}\index{datawc\_timeunits (vcs.Canvas.Canvas attribute)}\index{datawc\_calendar (vcs.Canvas.Canvas attribute)}) -- sets the y/x ratio ,if passed as a string with `t' at the end, will aslo moves the ticks

\item {} 
\textbf{\texttt{xaxisconvert}} (\href{https://docs.python.org/2/library/functions.html\#str}{\emph{\texttt{str}}}) -- (Ex: `linear') converting xaxis linear/log/log10/ln/exp/area\_wt

\item {} 
\textbf{\texttt{yaxisconvert}} (\href{https://docs.python.org/2/library/functions.html\#str}{\emph{\texttt{str}}}) -- (Ex: `linear') converting yaxis linear/log/log10/ln/exp/area\_wt

\item {} 
\textbf{\texttt{new\_GM\_name}} (\href{https://docs.python.org/2/library/functions.html\#str}{\emph{\texttt{str}}}) -- (Ex: `my\_awesome\_gm') name of the new graphics method object. If no name is given, then one will be created for use.

\item {} 
\textbf{\texttt{source\_GM\_name}} -- (Ex: `default') copy the contents of the source object to the newly created one. If no name is given, then the `default' graphics methond contents is copied over to the new object.

\end{itemize}

\item[{Returns}] \leavevmode
A scatter graphics method

\item[{Return type}] \leavevmode
{\hyperref[vcs/graphics/unified1D:vcs.unified1D.G1d]{\sphinxcrossref{vcs.unified1D.G1d}}}

\end{description}\end{quote}

\end{fulllineitems}

\index{createtaylordiagram() (vcs.Canvas.Canvas method)}

\begin{fulllineitems}
\phantomsection\label{vcs/Canvas:vcs.Canvas.Canvas.createtaylordiagram}\pysiglinewithargsret{\sphinxbfcode{createtaylordiagram}}{\emph{name=None}, \emph{source='default'}}{}
Create a new taylordiagram graphics method given the the name and the existing
taylordiagram graphics method to copy the attributes from. If no existing
taylordiagram graphics method is given, then the default taylordiagram graphics method will be used as the graphics method
to which the attributes will be copied from.

\begin{notice}{note}{Note:}
If the name provided already exists, then an error will be returned. graphics method
names must be unique.
\end{notice}
\begin{quote}\begin{description}
\item[{Example}] \leavevmode
\begin{Verbatim}[commandchars=\\\{\}]
\PYG{g+gp}{\PYGZgt{}\PYGZgt{}\PYGZgt{} }\PYG{n}{vcs}\PYG{o}{.}\PYG{n}{show}\PYG{p}{(}\PYG{l+s+s1}{\PYGZsq{}}\PYG{l+s+s1}{taylordiagram}\PYG{l+s+s1}{\PYGZsq{}}\PYG{p}{)} \PYG{c+c1}{\PYGZsh{} show all available taylordiagram}
\PYG{g+go}{*******************Taylordiagram Names List**********************}
\PYG{g+gp}{...}
\PYG{g+go}{*******************End Taylordiagram Names List**********************}
\PYG{g+gp}{\PYGZgt{}\PYGZgt{}\PYGZgt{} }\PYG{n}{ex}\PYG{o}{=}\PYG{n}{vcs}\PYG{o}{.}\PYG{n}{createtaylordiagram}\PYG{p}{(}\PYG{l+s+s1}{\PYGZsq{}}\PYG{l+s+s1}{taylordiagram\PYGZus{}ex1}\PYG{l+s+s1}{\PYGZsq{}}\PYG{p}{)} \PYG{c+c1}{\PYGZsh{} Create taylordiagram \PYGZsq{}taylordiagram\PYGZus{}ex1\PYGZsq{} that inherits from \PYGZsq{}default\PYGZsq{}}
\PYG{g+gp}{\PYGZgt{}\PYGZgt{}\PYGZgt{} }\PYG{n}{vcs}\PYG{o}{.}\PYG{n}{listelements}\PYG{p}{(}\PYG{l+s+s1}{\PYGZsq{}}\PYG{l+s+s1}{taylordiagram}\PYG{l+s+s1}{\PYGZsq{}}\PYG{p}{)} \PYG{c+c1}{\PYGZsh{} should now contain the \PYGZsq{}taylordiagram\PYGZus{}ex1\PYGZsq{} taylordiagram}
\PYG{g+go}{[...\PYGZsq{}taylordiagram\PYGZus{}ex1\PYGZsq{}...]}
\end{Verbatim}

\item[{Parameters}] \leavevmode\begin{itemize}
\item {} 
\textbf{\texttt{name}} (\href{https://docs.python.org/2/library/functions.html\#str}{\emph{\texttt{str}}}) -- The name of the created object

\item {} 
\textbf{\texttt{source}} (\emph{\texttt{a taylordiagram or a string name of a}}) -- The object to inherit from

\end{itemize}

\item[{Returns}] \leavevmode
A taylordiagram graphics method object

\item[{Return type}] \leavevmode
{\hyperref[vcs/graphics/taylor:vcs.taylor.Gtd]{\sphinxcrossref{vcs.taylor.Gtd}}}

\end{description}\end{quote}

\end{fulllineitems}

\index{createtemplate() (vcs.Canvas.Canvas method)}

\begin{fulllineitems}
\phantomsection\label{vcs/Canvas:vcs.Canvas.Canvas.createtemplate}\pysiglinewithargsret{\sphinxbfcode{createtemplate}}{\emph{name=None}, \emph{source='default'}}{}
Create a new template graphics method given the the name and the existing
template graphics method to copy the attributes from. If no existing
template graphics method is given, then the default template graphics method will be used as the graphics method
to which the attributes will be copied from.

\begin{notice}{note}{Note:}
If the name provided already exists, then an error will be returned. graphics method
names must be unique.
\end{notice}
\begin{quote}\begin{description}
\item[{Example}] \leavevmode
\begin{Verbatim}[commandchars=\\\{\}]
\PYG{g+gp}{\PYGZgt{}\PYGZgt{}\PYGZgt{} }\PYG{n}{vcs}\PYG{o}{.}\PYG{n}{show}\PYG{p}{(}\PYG{l+s+s1}{\PYGZsq{}}\PYG{l+s+s1}{template}\PYG{l+s+s1}{\PYGZsq{}}\PYG{p}{)} \PYG{c+c1}{\PYGZsh{} show all available template}
\PYG{g+go}{*******************Template Names List**********************}
\PYG{g+gp}{...}
\PYG{g+go}{*******************End Template Names List**********************}
\PYG{g+gp}{\PYGZgt{}\PYGZgt{}\PYGZgt{} }\PYG{n}{ex}\PYG{o}{=}\PYG{n}{vcs}\PYG{o}{.}\PYG{n}{createtemplate}\PYG{p}{(}\PYG{l+s+s1}{\PYGZsq{}}\PYG{l+s+s1}{template\PYGZus{}ex1}\PYG{l+s+s1}{\PYGZsq{}}\PYG{p}{)} \PYG{c+c1}{\PYGZsh{} Create template \PYGZsq{}template\PYGZus{}ex1\PYGZsq{} that inherits from \PYGZsq{}default\PYGZsq{}}
\PYG{g+gp}{\PYGZgt{}\PYGZgt{}\PYGZgt{} }\PYG{n}{vcs}\PYG{o}{.}\PYG{n}{listelements}\PYG{p}{(}\PYG{l+s+s1}{\PYGZsq{}}\PYG{l+s+s1}{template}\PYG{l+s+s1}{\PYGZsq{}}\PYG{p}{)} \PYG{c+c1}{\PYGZsh{} should now contain the \PYGZsq{}template\PYGZus{}ex1\PYGZsq{} template}
\PYG{g+go}{[...\PYGZsq{}template\PYGZus{}ex1\PYGZsq{}...]}
\PYG{g+gp}{\PYGZgt{}\PYGZgt{}\PYGZgt{} }\PYG{n}{ex2}\PYG{o}{=}\PYG{n}{vcs}\PYG{o}{.}\PYG{n}{createtemplate}\PYG{p}{(}\PYG{l+s+s1}{\PYGZsq{}}\PYG{l+s+s1}{template\PYGZus{}ex2}\PYG{l+s+s1}{\PYGZsq{}}\PYG{p}{,}\PYG{l+s+s1}{\PYGZsq{}}\PYG{l+s+s1}{polar}\PYG{l+s+s1}{\PYGZsq{}}\PYG{p}{)} \PYG{c+c1}{\PYGZsh{} create \PYGZsq{}template\PYGZus{}ex2\PYGZsq{} from \PYGZsq{}polar\PYGZsq{} template}
\PYG{g+gp}{\PYGZgt{}\PYGZgt{}\PYGZgt{} }\PYG{n}{vcs}\PYG{o}{.}\PYG{n}{listelements}\PYG{p}{(}\PYG{l+s+s1}{\PYGZsq{}}\PYG{l+s+s1}{template}\PYG{l+s+s1}{\PYGZsq{}}\PYG{p}{)} \PYG{c+c1}{\PYGZsh{} should now contain the \PYGZsq{}template\PYGZus{}ex2\PYGZsq{} template}
\PYG{g+go}{[...\PYGZsq{}template\PYGZus{}ex2\PYGZsq{}...]}
\end{Verbatim}

\item[{Parameters}] \leavevmode\begin{itemize}
\item {} 
\textbf{\texttt{name}} (\href{https://docs.python.org/2/library/functions.html\#str}{\emph{\texttt{str}}}) -- The name of the created object

\item {} 
\textbf{\texttt{source}} (\emph{\texttt{a template or a string name of a template}}) -- The object to inherit from

\end{itemize}

\item[{Returns}] \leavevmode
A template

\item[{Return type}] \leavevmode
{\hyperref[vcs/template/template:vcs.template.P]{\sphinxcrossref{vcs.template.P}}}

\end{description}\end{quote}

\end{fulllineitems}

\index{createtext() (vcs.Canvas.Canvas method)}

\begin{fulllineitems}
\phantomsection\label{vcs/Canvas:vcs.Canvas.Canvas.createtext}\pysiglinewithargsret{\sphinxbfcode{createtext}}{\emph{Tt\_name=None}, \emph{Tt\_source='default'}, \emph{To\_name=None}, \emph{To\_source='default'}, \emph{font=None}, \emph{spacing=None}, \emph{expansion=None}, \emph{color=None}, \emph{priority=None}, \emph{viewport=None}, \emph{worldcoordinate=None}, \emph{x=None}, \emph{y=None}, \emph{height=None}, \emph{angle=None}, \emph{path=None}, \emph{halign=None}, \emph{valign=None}, \emph{projection=None}}{}
Create a new textcombined secondary method given the the name and the existing
textcombined secondary method to copy the attributes from. If no existing
textcombined secondary method is given, then the default textcombined secondary method will be used as the graphics method
to which the attributes will be copied from.

\begin{notice}{note}{Note:}
If the name provided already exists, then an error will be returned. secondary method
names must be unique.
\end{notice}
\begin{quote}\begin{description}
\item[{Example}] \leavevmode
\begin{Verbatim}[commandchars=\\\{\}]
\PYG{g+gp}{\PYGZgt{}\PYGZgt{}\PYGZgt{} }\PYG{n}{vcs}\PYG{o}{.}\PYG{n}{show}\PYG{p}{(}\PYG{l+s+s1}{\PYGZsq{}}\PYG{l+s+s1}{textcombined}\PYG{l+s+s1}{\PYGZsq{}}\PYG{p}{)} \PYG{c+c1}{\PYGZsh{} show all available textcombined}
\PYG{g+go}{*******************Textcombined Names List**********************}
\PYG{g+gp}{...}
\PYG{g+go}{*******************End Textcombined Names List**********************}
\PYG{g+gp}{\PYGZgt{}\PYGZgt{}\PYGZgt{} }\PYG{n}{a}\PYG{o}{.}\PYG{n}{createtextcombined}\PYG{p}{(}\PYG{l+s+s1}{\PYGZsq{}}\PYG{l+s+s1}{EXAMPLE\PYGZus{}tt}\PYG{l+s+s1}{\PYGZsq{}}\PYG{p}{,} \PYG{l+s+s1}{\PYGZsq{}}\PYG{l+s+s1}{qa}\PYG{l+s+s1}{\PYGZsq{}}\PYG{p}{,} \PYG{l+s+s1}{\PYGZsq{}}\PYG{l+s+s1}{EXAMPLE\PYGZus{}tto}\PYG{l+s+s1}{\PYGZsq{}}\PYG{p}{,} \PYG{l+s+s1}{\PYGZsq{}}\PYG{l+s+s1}{7left}\PYG{l+s+s1}{\PYGZsq{}}\PYG{p}{)} \PYG{c+c1}{\PYGZsh{} Create \PYGZsq{}EXAMPLE\PYGZus{}tt\PYGZsq{} and \PYGZsq{}EXAMPLE\PYGZus{}tto\PYGZsq{}}
\PYG{g+go}{\PYGZlt{}vcs.textcombined.Tc ...\PYGZgt{}}
\PYG{g+gp}{\PYGZgt{}\PYGZgt{}\PYGZgt{} }\PYG{n}{vcs}\PYG{o}{.}\PYG{n}{listelements}\PYG{p}{(}\PYG{l+s+s1}{\PYGZsq{}}\PYG{l+s+s1}{textcombined}\PYG{l+s+s1}{\PYGZsq{}}\PYG{p}{)} \PYG{c+c1}{\PYGZsh{} should now contain the \PYGZsq{}qa\PYGZus{}tt:::left\PYGZus{}tto\PYGZsq{} textcombined}
\PYG{g+go}{[...\PYGZsq{}qa\PYGZus{}tt:::left\PYGZus{}tto\PYGZsq{}...]}
\end{Verbatim}

\item[{Parameters}] \leavevmode\begin{itemize}
\item {} 
\textbf{\texttt{Tt\_name}} (\href{https://docs.python.org/2/library/functions.html\#str}{\emph{\texttt{str}}}) -- Name of created object

\item {} 
\textbf{\texttt{Tt\_source}} (\emph{\texttt{str or vcs.texttable.Tt}}) -- Texttable object to inherit from. Can be a texttable, or a string name of a texttable.

\item {} 
\textbf{\texttt{To\_name}} (\href{https://docs.python.org/2/library/functions.html\#str}{\emph{\texttt{str}}}) -- Name of the textcombined's text orientation  (to be created)

\item {} 
\textbf{\texttt{To\_source}} (\emph{\texttt{str or vcs.textorientation.To}}) -- Name of the textorientation to inherit. Can be a textorientation, or a string name of a textorientation.

\item {} 
\textbf{\texttt{font}} (\emph{\texttt{int or str}}) -- Which font to use (index or name).

\item {} 
\textbf{\texttt{spacing}} (\emph{\texttt{DEPRECATED}}) -- DEPRECATED

\item {} 
\textbf{\texttt{expansion}} (\emph{\texttt{DEPRECATED}}) -- DEPRECATED

\item {} 
\textbf{\texttt{color}} (\emph{\texttt{str or int}}) -- 
A color name from the \href{https://en.wikipedia.org/wiki/X11\_color\_names}{X11 Color Names list},
or an integer value from 0-255, or an RGB/RGBA tuple/list (e.g. (0,100,0), (100,100,0,50))


\item {} 
\textbf{\texttt{priority}} (\href{https://docs.python.org/2/library/functions.html\#int}{\emph{\texttt{int}}}) -- The layer on which the object will be drawn.

\item {} 
\textbf{\texttt{viewport}} (\emph{\texttt{list of floats}}) -- 4 floats between 0 and 1. These specify the area that the X/Y values are mapped to inside of the canvas

\item {} 
\textbf{\texttt{worldcoordinate}} (\emph{\texttt{list of floats}}) -- List of 4 floats (xmin, xmax, ymin, ymax)

\item {} 
\textbf{\texttt{x}} (\emph{\texttt{list of floats}}) -- List of lists of x coordinates. Values must be between worldcoordinate{[}0{]} and worldcoordinate{[}1{]}.

\item {} 
\textbf{\texttt{y}} (\emph{\texttt{list of floats}}) -- List of lists of y coordinates. Values must be between worldcoordinate{[}2{]} and worldcoordinate{[}3{]}.

\item {} 
\textbf{\texttt{height}} (\href{https://docs.python.org/2/library/functions.html\#int}{\emph{\texttt{int}}}) -- Size of the font

\item {} 
\textbf{\texttt{angle}} (\href{https://docs.python.org/2/library/functions.html\#int}{\emph{\texttt{int}}}) -- Angle of the text, in degrees

\item {} 
\textbf{\texttt{path}} (\emph{\texttt{DEPRECATED}}) -- DEPRECATED

\item {} 
\textbf{\texttt{halign}} (\href{https://docs.python.org/2/library/functions.html\#str}{\emph{\texttt{str}}}) -- Horizontal alignment of the text. One of {[}''left'', ``center'', ``right''{]}.

\item {} 
\textbf{\texttt{valign}} (\href{https://docs.python.org/2/library/functions.html\#str}{\emph{\texttt{str}}}) -- Vertical alignment of the text. One of {[}''top'', ``center'', ``botom''{]}.

\item {} 
\textbf{\texttt{projection}} (\emph{\texttt{str or projection object}}) -- Specify a geographic projection used to convert x/y from spherical coordinates into 2D coordinates.

\end{itemize}

\item[{Returns}] \leavevmode
A VCS text object

\item[{Return type}] \leavevmode
{\hyperref[vcs/secondary/textcombined:vcs.textcombined.Tc]{\sphinxcrossref{vcs.textcombined.Tc}}}

\end{description}\end{quote}

\end{fulllineitems}

\index{createtextcombined() (vcs.Canvas.Canvas method)}

\begin{fulllineitems}
\phantomsection\label{vcs/Canvas:vcs.Canvas.Canvas.createtextcombined}\pysiglinewithargsret{\sphinxbfcode{createtextcombined}}{\emph{Tt\_name=None}, \emph{Tt\_source='default'}, \emph{To\_name=None}, \emph{To\_source='default'}, \emph{font=None}, \emph{spacing=None}, \emph{expansion=None}, \emph{color=None}, \emph{priority=None}, \emph{viewport=None}, \emph{worldcoordinate=None}, \emph{x=None}, \emph{y=None}, \emph{height=None}, \emph{angle=None}, \emph{path=None}, \emph{halign=None}, \emph{valign=None}, \emph{projection=None}}{}
Create a new textcombined secondary method given the the name and the existing
textcombined secondary method to copy the attributes from. If no existing
textcombined secondary method is given, then the default textcombined secondary method will be used as the graphics method
to which the attributes will be copied from.

\begin{notice}{note}{Note:}
If the name provided already exists, then an error will be returned. secondary method
names must be unique.
\end{notice}
\begin{quote}\begin{description}
\item[{Example}] \leavevmode
\begin{Verbatim}[commandchars=\\\{\}]
\PYG{g+gp}{\PYGZgt{}\PYGZgt{}\PYGZgt{} }\PYG{n}{vcs}\PYG{o}{.}\PYG{n}{show}\PYG{p}{(}\PYG{l+s+s1}{\PYGZsq{}}\PYG{l+s+s1}{textcombined}\PYG{l+s+s1}{\PYGZsq{}}\PYG{p}{)} \PYG{c+c1}{\PYGZsh{} show all available textcombined}
\PYG{g+go}{*******************Textcombined Names List**********************}
\PYG{g+gp}{...}
\PYG{g+go}{*******************End Textcombined Names List**********************}
\PYG{g+gp}{\PYGZgt{}\PYGZgt{}\PYGZgt{} }\PYG{n}{a}\PYG{o}{.}\PYG{n}{createtextcombined}\PYG{p}{(}\PYG{l+s+s1}{\PYGZsq{}}\PYG{l+s+s1}{EXAMPLE\PYGZus{}tt}\PYG{l+s+s1}{\PYGZsq{}}\PYG{p}{,} \PYG{l+s+s1}{\PYGZsq{}}\PYG{l+s+s1}{qa}\PYG{l+s+s1}{\PYGZsq{}}\PYG{p}{,} \PYG{l+s+s1}{\PYGZsq{}}\PYG{l+s+s1}{EXAMPLE\PYGZus{}tto}\PYG{l+s+s1}{\PYGZsq{}}\PYG{p}{,} \PYG{l+s+s1}{\PYGZsq{}}\PYG{l+s+s1}{7left}\PYG{l+s+s1}{\PYGZsq{}}\PYG{p}{)} \PYG{c+c1}{\PYGZsh{} Create \PYGZsq{}EXAMPLE\PYGZus{}tt\PYGZsq{} and \PYGZsq{}EXAMPLE\PYGZus{}tto\PYGZsq{}}
\PYG{g+go}{\PYGZlt{}vcs.textcombined.Tc ...\PYGZgt{}}
\PYG{g+gp}{\PYGZgt{}\PYGZgt{}\PYGZgt{} }\PYG{n}{vcs}\PYG{o}{.}\PYG{n}{listelements}\PYG{p}{(}\PYG{l+s+s1}{\PYGZsq{}}\PYG{l+s+s1}{textcombined}\PYG{l+s+s1}{\PYGZsq{}}\PYG{p}{)} \PYG{c+c1}{\PYGZsh{} should now contain the \PYGZsq{}qa\PYGZus{}tt:::left\PYGZus{}tto\PYGZsq{} textcombined}
\PYG{g+go}{[...\PYGZsq{}qa\PYGZus{}tt:::left\PYGZus{}tto\PYGZsq{}...]}
\end{Verbatim}

\item[{Parameters}] \leavevmode\begin{itemize}
\item {} 
\textbf{\texttt{Tt\_name}} (\href{https://docs.python.org/2/library/functions.html\#str}{\emph{\texttt{str}}}) -- Name of created object

\item {} 
\textbf{\texttt{Tt\_source}} (\emph{\texttt{str or vcs.texttable.Tt}}) -- Texttable object to inherit from. Can be a texttable, or a string name of a texttable.

\item {} 
\textbf{\texttt{To\_name}} (\href{https://docs.python.org/2/library/functions.html\#str}{\emph{\texttt{str}}}) -- Name of the textcombined's text orientation  (to be created)

\item {} 
\textbf{\texttt{To\_source}} (\emph{\texttt{str or vcs.textorientation.To}}) -- Name of the textorientation to inherit. Can be a textorientation, or a string name of a textorientation.

\item {} 
\textbf{\texttt{font}} (\emph{\texttt{int or str}}) -- Which font to use (index or name).

\item {} 
\textbf{\texttt{spacing}} (\emph{\texttt{DEPRECATED}}) -- DEPRECATED

\item {} 
\textbf{\texttt{expansion}} (\emph{\texttt{DEPRECATED}}) -- DEPRECATED

\item {} 
\textbf{\texttt{color}} (\emph{\texttt{str or int}}) -- 
A color name from the \href{https://en.wikipedia.org/wiki/X11\_color\_names}{X11 Color Names list},
or an integer value from 0-255, or an RGB/RGBA tuple/list (e.g. (0,100,0), (100,100,0,50))


\item {} 
\textbf{\texttt{priority}} (\href{https://docs.python.org/2/library/functions.html\#int}{\emph{\texttt{int}}}) -- The layer on which the object will be drawn.

\item {} 
\textbf{\texttt{viewport}} (\emph{\texttt{list of floats}}) -- 4 floats between 0 and 1. These specify the area that the X/Y values are mapped to inside of the canvas

\item {} 
\textbf{\texttt{worldcoordinate}} (\emph{\texttt{list of floats}}) -- List of 4 floats (xmin, xmax, ymin, ymax)

\item {} 
\textbf{\texttt{x}} (\emph{\texttt{list of floats}}) -- List of lists of x coordinates. Values must be between worldcoordinate{[}0{]} and worldcoordinate{[}1{]}.

\item {} 
\textbf{\texttt{y}} (\emph{\texttt{list of floats}}) -- List of lists of y coordinates. Values must be between worldcoordinate{[}2{]} and worldcoordinate{[}3{]}.

\item {} 
\textbf{\texttt{height}} (\href{https://docs.python.org/2/library/functions.html\#int}{\emph{\texttt{int}}}) -- Size of the font

\item {} 
\textbf{\texttt{angle}} (\href{https://docs.python.org/2/library/functions.html\#int}{\emph{\texttt{int}}}) -- Angle of the text, in degrees

\item {} 
\textbf{\texttt{path}} (\emph{\texttt{DEPRECATED}}) -- DEPRECATED

\item {} 
\textbf{\texttt{halign}} (\href{https://docs.python.org/2/library/functions.html\#str}{\emph{\texttt{str}}}) -- Horizontal alignment of the text. One of {[}''left'', ``center'', ``right''{]}.

\item {} 
\textbf{\texttt{valign}} (\href{https://docs.python.org/2/library/functions.html\#str}{\emph{\texttt{str}}}) -- Vertical alignment of the text. One of {[}''top'', ``center'', ``botom''{]}.

\item {} 
\textbf{\texttt{projection}} (\emph{\texttt{str or projection object}}) -- Specify a geographic projection used to convert x/y from spherical coordinates into 2D coordinates.

\end{itemize}

\item[{Returns}] \leavevmode
A VCS text object

\item[{Return type}] \leavevmode
{\hyperref[vcs/secondary/textcombined:vcs.textcombined.Tc]{\sphinxcrossref{vcs.textcombined.Tc}}}

\end{description}\end{quote}

\end{fulllineitems}

\index{createtextorientation() (vcs.Canvas.Canvas method)}

\begin{fulllineitems}
\phantomsection\label{vcs/Canvas:vcs.Canvas.Canvas.createtextorientation}\pysiglinewithargsret{\sphinxbfcode{createtextorientation}}{\emph{name=None}, \emph{source='default'}}{}
Create a new textorientation secondary method given the the name and the existing
textorientation secondary method to copy the attributes from. If no existing
textorientation secondary method is given, then the default textorientation secondary method will be used as the graphics method
to which the attributes will be copied from.

\begin{notice}{note}{Note:}
If the name provided already exists, then an error will be returned. secondary method
names must be unique.
\end{notice}
\begin{quote}\begin{description}
\item[{Example}] \leavevmode
\begin{Verbatim}[commandchars=\\\{\}]
\PYG{g+gp}{\PYGZgt{}\PYGZgt{}\PYGZgt{} }\PYG{n}{vcs}\PYG{o}{.}\PYG{n}{show}\PYG{p}{(}\PYG{l+s+s1}{\PYGZsq{}}\PYG{l+s+s1}{textorientation}\PYG{l+s+s1}{\PYGZsq{}}\PYG{p}{)} \PYG{c+c1}{\PYGZsh{} show all available textorientation}
\PYG{g+go}{*******************Textorientation Names List**********************}
\PYG{g+gp}{...}
\PYG{g+go}{*******************End Textorientation Names List**********************}
\PYG{g+gp}{\PYGZgt{}\PYGZgt{}\PYGZgt{} }\PYG{n}{ex}\PYG{o}{=}\PYG{n}{vcs}\PYG{o}{.}\PYG{n}{createtextorientation}\PYG{p}{(}\PYG{l+s+s1}{\PYGZsq{}}\PYG{l+s+s1}{textorientation\PYGZus{}ex1}\PYG{l+s+s1}{\PYGZsq{}}\PYG{p}{)} \PYG{c+c1}{\PYGZsh{} Create textorientation \PYGZsq{}textorientation\PYGZus{}ex1\PYGZsq{} that inherits from \PYGZsq{}default\PYGZsq{}}
\PYG{g+gp}{\PYGZgt{}\PYGZgt{}\PYGZgt{} }\PYG{n}{vcs}\PYG{o}{.}\PYG{n}{listelements}\PYG{p}{(}\PYG{l+s+s1}{\PYGZsq{}}\PYG{l+s+s1}{textorientation}\PYG{l+s+s1}{\PYGZsq{}}\PYG{p}{)} \PYG{c+c1}{\PYGZsh{} should now contain the \PYGZsq{}textorientation\PYGZus{}ex1\PYGZsq{} textorientation}
\PYG{g+go}{[...\PYGZsq{}textorientation\PYGZus{}ex1\PYGZsq{}...]}
\PYG{g+gp}{\PYGZgt{}\PYGZgt{}\PYGZgt{} }\PYG{n}{ex2}\PYG{o}{=}\PYG{n}{vcs}\PYG{o}{.}\PYG{n}{createtextorientation}\PYG{p}{(}\PYG{l+s+s1}{\PYGZsq{}}\PYG{l+s+s1}{textorientation\PYGZus{}ex2}\PYG{l+s+s1}{\PYGZsq{}}\PYG{p}{,}\PYG{l+s+s1}{\PYGZsq{}}\PYG{l+s+s1}{bigger}\PYG{l+s+s1}{\PYGZsq{}}\PYG{p}{)} \PYG{c+c1}{\PYGZsh{} create \PYGZsq{}textorientation\PYGZus{}ex2\PYGZsq{} from \PYGZsq{}bigger\PYGZsq{} template}
\PYG{g+gp}{\PYGZgt{}\PYGZgt{}\PYGZgt{} }\PYG{n}{vcs}\PYG{o}{.}\PYG{n}{listelements}\PYG{p}{(}\PYG{l+s+s1}{\PYGZsq{}}\PYG{l+s+s1}{textorientation}\PYG{l+s+s1}{\PYGZsq{}}\PYG{p}{)} \PYG{c+c1}{\PYGZsh{} should now contain the \PYGZsq{}textorientation\PYGZus{}ex2\PYGZsq{} textorientation}
\PYG{g+go}{[...\PYGZsq{}textorientation\PYGZus{}ex2\PYGZsq{}...]}
\end{Verbatim}

\item[{Parameters}] \leavevmode\begin{itemize}
\item {} 
\textbf{\texttt{name}} (\href{https://docs.python.org/2/library/functions.html\#str}{\emph{\texttt{str}}}) -- The name of the created object

\item {} 
\textbf{\texttt{source}} (\emph{\texttt{a textorientation or a string name of a textorientation}}) -- The object to inherit from

\end{itemize}

\item[{Returns}] \leavevmode
A textorientation secondary method

\item[{Return type}] \leavevmode
{\hyperref[vcs/secondary/textorientation:vcs.textorientation.To]{\sphinxcrossref{vcs.textorientation.To}}}

\end{description}\end{quote}

\end{fulllineitems}

\index{createtexttable() (vcs.Canvas.Canvas method)}

\begin{fulllineitems}
\phantomsection\label{vcs/Canvas:vcs.Canvas.Canvas.createtexttable}\pysiglinewithargsret{\sphinxbfcode{createtexttable}}{\emph{name=None}, \emph{source='default'}, \emph{font=None}, \emph{spacing=None}, \emph{expansion=None}, \emph{color=None}, \emph{priority=None}, \emph{viewport=None}, \emph{worldcoordinate=None}, \emph{x=None}, \emph{y=None}}{}
Create a new texttable secondary method given the the name and the existing
texttable secondary method to copy the attributes from. If no existing
texttable secondary method is given, then the default texttable secondary method will be used as the graphics method
to which the attributes will be copied from.

\begin{notice}{note}{Note:}
If the name provided already exists, then an error will be returned. secondary method
names must be unique.
\end{notice}
\begin{quote}\begin{description}
\item[{Example}] \leavevmode
\begin{Verbatim}[commandchars=\\\{\}]
\PYG{g+gp}{\PYGZgt{}\PYGZgt{}\PYGZgt{} }\PYG{n}{vcs}\PYG{o}{.}\PYG{n}{show}\PYG{p}{(}\PYG{l+s+s1}{\PYGZsq{}}\PYG{l+s+s1}{texttable}\PYG{l+s+s1}{\PYGZsq{}}\PYG{p}{)} \PYG{c+c1}{\PYGZsh{} show all available texttable}
\PYG{g+go}{*******************Texttable Names List**********************}
\PYG{g+gp}{...}
\PYG{g+go}{*******************End Texttable Names List**********************}
\PYG{g+gp}{\PYGZgt{}\PYGZgt{}\PYGZgt{} }\PYG{n}{ex}\PYG{o}{=}\PYG{n}{vcs}\PYG{o}{.}\PYG{n}{createtexttable}\PYG{p}{(}\PYG{l+s+s1}{\PYGZsq{}}\PYG{l+s+s1}{texttable\PYGZus{}ex1}\PYG{l+s+s1}{\PYGZsq{}}\PYG{p}{)} \PYG{c+c1}{\PYGZsh{} Create texttable \PYGZsq{}texttable\PYGZus{}ex1\PYGZsq{} that inherits from \PYGZsq{}default\PYGZsq{}}
\PYG{g+gp}{\PYGZgt{}\PYGZgt{}\PYGZgt{} }\PYG{n}{vcs}\PYG{o}{.}\PYG{n}{listelements}\PYG{p}{(}\PYG{l+s+s1}{\PYGZsq{}}\PYG{l+s+s1}{texttable}\PYG{l+s+s1}{\PYGZsq{}}\PYG{p}{)} \PYG{c+c1}{\PYGZsh{} should now contain the \PYGZsq{}texttable\PYGZus{}ex1\PYGZsq{} texttable}
\PYG{g+go}{[...\PYGZsq{}texttable\PYGZus{}ex1\PYGZsq{}...]}
\PYG{g+gp}{\PYGZgt{}\PYGZgt{}\PYGZgt{} }\PYG{n}{ex2}\PYG{o}{=}\PYG{n}{vcs}\PYG{o}{.}\PYG{n}{createtexttable}\PYG{p}{(}\PYG{l+s+s1}{\PYGZsq{}}\PYG{l+s+s1}{texttable\PYGZus{}ex2}\PYG{l+s+s1}{\PYGZsq{}}\PYG{p}{,}\PYG{l+s+s1}{\PYGZsq{}}\PYG{l+s+s1}{bigger}\PYG{l+s+s1}{\PYGZsq{}}\PYG{p}{)} \PYG{c+c1}{\PYGZsh{} create \PYGZsq{}texttable\PYGZus{}ex2\PYGZsq{} from \PYGZsq{}bigger\PYGZsq{} template}
\PYG{g+gp}{\PYGZgt{}\PYGZgt{}\PYGZgt{} }\PYG{n}{vcs}\PYG{o}{.}\PYG{n}{listelements}\PYG{p}{(}\PYG{l+s+s1}{\PYGZsq{}}\PYG{l+s+s1}{texttable}\PYG{l+s+s1}{\PYGZsq{}}\PYG{p}{)} \PYG{c+c1}{\PYGZsh{} should now contain the \PYGZsq{}texttable\PYGZus{}ex2\PYGZsq{} texttable}
\PYG{g+go}{[...\PYGZsq{}texttable\PYGZus{}ex2\PYGZsq{}...]}
\end{Verbatim}

\item[{Parameters}] \leavevmode\begin{itemize}
\item {} 
\textbf{\texttt{name}} (\href{https://docs.python.org/2/library/functions.html\#str}{\emph{\texttt{str}}}) -- Name of created object

\item {} 
\textbf{\texttt{source}} (\href{https://docs.python.org/2/library/functions.html\#str}{\emph{\texttt{str}}}) -- a texttable, or string name of a texttable

\item {} 
\textbf{\texttt{font}} (\emph{\texttt{int or string}}) -- Which font to use (index or name).

\item {} 
\textbf{\texttt{expansion}} (\emph{\texttt{DEPRECATED}}) -- DEPRECATED

\item {} 
\textbf{\texttt{color}} (\emph{\texttt{str or int}}) -- 
A color name from the \href{https://en.wikipedia.org/wiki/X11\_color\_names}{X11 Color Names list},
or an integer value from 0-255, or an RGB/RGBA tuple/list (e.g. (0,100,0), (100,100,0,50))


\item {} 
\textbf{\texttt{priority}} (\href{https://docs.python.org/2/library/functions.html\#int}{\emph{\texttt{int}}}) -- The layer on which the texttable will be drawn.

\item {} 
\textbf{\texttt{viewport}} (\emph{\texttt{list of floats}}) -- 4 floats between 0 and 1. These specify the area that the X/Y values are mapped to inside of the canvas

\item {} 
\textbf{\texttt{worldcoordinate}} (\emph{\texttt{list of floats}}) -- List of 4 floats (xmin, xmax, ymin, ymax)

\item {} 
\textbf{\texttt{x}} (\emph{\texttt{list of floats}}) -- List of lists of x coordinates. Values must be between worldcoordinate{[}0{]} and worldcoordinate{[}1{]}.

\item {} 
\textbf{\texttt{y}} (\emph{\texttt{list of floats}}) -- List of lists of y coordinates. Values must be between worldcoordinate{[}2{]} and worldcoordinate{[}3{]}.

\end{itemize}

\item[{Returns}] \leavevmode
A texttable graphics method object

\item[{Return type}] \leavevmode
{\hyperref[vcs/secondary/texttable:vcs.texttable.Tt]{\sphinxcrossref{vcs.texttable.Tt}}}

\end{description}\end{quote}

\end{fulllineitems}

\index{createvector() (vcs.Canvas.Canvas method)}

\begin{fulllineitems}
\phantomsection\label{vcs/Canvas:vcs.Canvas.Canvas.createvector}\pysiglinewithargsret{\sphinxbfcode{createvector}}{\emph{name=None}, \emph{source='default'}}{}
Create a new vector graphics method given the the name and the existing
vector graphics method to copy the attributes from. If no existing
vector graphics method is given, then the default vector graphics method will be used as the graphics method
to which the attributes will be copied from.

\begin{notice}{note}{Note:}
If the name provided already exists, then an error will be returned. graphics method
names must be unique.
\end{notice}
\begin{quote}\begin{description}
\item[{Example}] \leavevmode
\begin{Verbatim}[commandchars=\\\{\}]
\PYG{g+gp}{\PYGZgt{}\PYGZgt{}\PYGZgt{} }\PYG{n}{vcs}\PYG{o}{.}\PYG{n}{show}\PYG{p}{(}\PYG{l+s+s1}{\PYGZsq{}}\PYG{l+s+s1}{vector}\PYG{l+s+s1}{\PYGZsq{}}\PYG{p}{)} \PYG{c+c1}{\PYGZsh{} show all available vector}
\PYG{g+go}{*******************Vector Names List**********************}
\PYG{g+gp}{...}
\PYG{g+go}{*******************End Vector Names List**********************}
\PYG{g+gp}{\PYGZgt{}\PYGZgt{}\PYGZgt{} }\PYG{n}{ex}\PYG{o}{=}\PYG{n}{vcs}\PYG{o}{.}\PYG{n}{createvector}\PYG{p}{(}\PYG{l+s+s1}{\PYGZsq{}}\PYG{l+s+s1}{vector\PYGZus{}ex1}\PYG{l+s+s1}{\PYGZsq{}}\PYG{p}{)} \PYG{c+c1}{\PYGZsh{} Create vector \PYGZsq{}vector\PYGZus{}ex1\PYGZsq{} that inherits from \PYGZsq{}default\PYGZsq{}}
\PYG{g+gp}{\PYGZgt{}\PYGZgt{}\PYGZgt{} }\PYG{n}{vcs}\PYG{o}{.}\PYG{n}{listelements}\PYG{p}{(}\PYG{l+s+s1}{\PYGZsq{}}\PYG{l+s+s1}{vector}\PYG{l+s+s1}{\PYGZsq{}}\PYG{p}{)} \PYG{c+c1}{\PYGZsh{} should now contain the \PYGZsq{}vector\PYGZus{}ex1\PYGZsq{} vector}
\PYG{g+go}{[...\PYGZsq{}vector\PYGZus{}ex1\PYGZsq{}...]}
\end{Verbatim}

\item[{Parameters}] \leavevmode\begin{itemize}
\item {} 
\textbf{\texttt{name}} (\href{https://docs.python.org/2/library/functions.html\#str}{\emph{\texttt{str}}}) -- The name of the created object

\item {} 
\textbf{\texttt{source}} (\emph{\texttt{a vector or a string name of a vector}}) -- The object to inherit from

\end{itemize}

\item[{Returns}] \leavevmode
A vector graphics method object

\item[{Return type}] \leavevmode
{\hyperref[vcs/graphics/vector:vcs.vector.Gv]{\sphinxcrossref{vcs.vector.Gv}}}

\end{description}\end{quote}

\end{fulllineitems}

\index{createxvsy() (vcs.Canvas.Canvas method)}

\begin{fulllineitems}
\phantomsection\label{vcs/Canvas:vcs.Canvas.Canvas.createxvsy}\pysiglinewithargsret{\sphinxbfcode{createxvsy}}{\emph{name=None}, \emph{source='default'}}{}
Create a new xvsy graphics method given the the name and the existing
xvsy graphics method to copy the attributes from. If no existing
xvsy graphics method is given, then the default xvsy graphics method will be used as the graphics method
to which the attributes will be copied from.

\begin{notice}{note}{Note:}
If the name provided already exists, then an error will be returned. graphics method
names must be unique.
\end{notice}
\begin{quote}\begin{description}
\item[{Example}] \leavevmode
\begin{Verbatim}[commandchars=\\\{\}]
\PYG{g+gp}{\PYGZgt{}\PYGZgt{}\PYGZgt{} }\PYG{n}{vcs}\PYG{o}{.}\PYG{n}{show}\PYG{p}{(}\PYG{l+s+s1}{\PYGZsq{}}\PYG{l+s+s1}{xvsy}\PYG{l+s+s1}{\PYGZsq{}}\PYG{p}{)} \PYG{c+c1}{\PYGZsh{} show all available xvsy}
\PYG{g+go}{*******************Xvsy Names List**********************}
\PYG{g+gp}{...}
\PYG{g+go}{*******************End Xvsy Names List**********************}
\PYG{g+gp}{\PYGZgt{}\PYGZgt{}\PYGZgt{} }\PYG{n}{ex}\PYG{o}{=}\PYG{n}{vcs}\PYG{o}{.}\PYG{n}{createxvsy}\PYG{p}{(}\PYG{l+s+s1}{\PYGZsq{}}\PYG{l+s+s1}{xvsy\PYGZus{}ex1}\PYG{l+s+s1}{\PYGZsq{}}\PYG{p}{)} \PYG{c+c1}{\PYGZsh{} Create xvsy \PYGZsq{}xvsy\PYGZus{}ex1\PYGZsq{} that inherits from \PYGZsq{}default\PYGZsq{}}
\PYG{g+gp}{\PYGZgt{}\PYGZgt{}\PYGZgt{} }\PYG{n}{vcs}\PYG{o}{.}\PYG{n}{listelements}\PYG{p}{(}\PYG{l+s+s1}{\PYGZsq{}}\PYG{l+s+s1}{xvsy}\PYG{l+s+s1}{\PYGZsq{}}\PYG{p}{)} \PYG{c+c1}{\PYGZsh{} should now contain the \PYGZsq{}xvsy\PYGZus{}ex1\PYGZsq{} xvsy}
\PYG{g+go}{[...\PYGZsq{}xvsy\PYGZus{}ex1\PYGZsq{}...]}
\end{Verbatim}

\item[{Parameters}] \leavevmode\begin{itemize}
\item {} 
\textbf{\texttt{name}} (\href{https://docs.python.org/2/library/functions.html\#str}{\emph{\texttt{str}}}) -- The name of the created object

\item {} 
\textbf{\texttt{source}} (\emph{\texttt{a xvsy or a string name of a xvsy}}) -- The object to inherit from

\item {} 
\textbf{\texttt{xaxis}} (\emph{\texttt{cdms2.axis.TransientAxis}}) -- Axis object to replace the slab -1 dim axis

\item {} 
\textbf{\texttt{yaxis}} (\emph{\texttt{cdms2.axis.TransientAxis}}) -- Axis object to replace the slab -2 dim axis, only if slab has more than 1D

\item {} 
\textbf{\texttt{zaxis}} (\emph{\texttt{cdms2.axis.TransientAxis}}) -- Axis object to replace the slab -3 dim axis, only if slab has more than 2D

\item {} 
\textbf{\texttt{taxis}} (\emph{\texttt{cdms2.axis.TransientAxis}}) -- Axis object to replace the slab -4 dim axis, only if slab has more than 3D

\item {} 
\textbf{\texttt{waxis}} (\emph{\texttt{cdms2.axis.TransientAxis}}) -- Axis object to replace the slab -5 dim axis, only if slab has more than 4D

\item {} 
\textbf{\texttt{xrev}} (\href{https://docs.python.org/2/library/functions.html\#bool}{\emph{\texttt{bool}}}) -- reverse x axis

\item {} 
\textbf{\texttt{yrev}} (\href{https://docs.python.org/2/library/functions.html\#bool}{\emph{\texttt{bool}}}) -- reverse y axis, only if slab has more than 1D

\item {} 
\textbf{\texttt{xarray}} (\href{https://docs.python.org/2/library/array.html\#module-array}{\emph{\texttt{array}}}) -- Values to use instead of x axis

\item {} 
\textbf{\texttt{yarray}} (\href{https://docs.python.org/2/library/array.html\#module-array}{\emph{\texttt{array}}}) -- Values to use instead of y axis, only if var has more than 1D

\item {} 
\textbf{\texttt{zarray}} (\href{https://docs.python.org/2/library/array.html\#module-array}{\emph{\texttt{array}}}) -- Values to use instead of z axis, only if var has more than 2D

\item {} 
\textbf{\texttt{tarray}} (\href{https://docs.python.org/2/library/array.html\#module-array}{\emph{\texttt{array}}}) -- Values to use instead of t axis, only if var has more than 3D

\item {} 
\textbf{\texttt{warray}} (\href{https://docs.python.org/2/library/array.html\#module-array}{\emph{\texttt{array}}}) -- Values to use instead of w axis, only if var has more than 4D

\item {} 
\textbf{\texttt{continents}} (\href{https://docs.python.org/2/library/functions.html\#int}{\emph{\texttt{int}}}) -- continents type number

\item {} 
\textbf{\texttt{name}} -- replaces variable name on plot

\item {} 
\textbf{\texttt{time}} (\emph{\texttt{A cdtime object}}) -- replaces time name on plot

\item {} 
\textbf{\texttt{units}} (\href{https://docs.python.org/2/library/functions.html\#str}{\emph{\texttt{str}}}) -- replaces units value on plot

\item {} 
\textbf{\texttt{ymd}} (\href{https://docs.python.org/2/library/functions.html\#str}{\emph{\texttt{str}}}) -- replaces year/month/day on plot

\item {} 
\textbf{\texttt{hms}} (\href{https://docs.python.org/2/library/functions.html\#str}{\emph{\texttt{str}}}) -- replaces hh/mm/ss on plot

\item {} 
\textbf{\texttt{file\_comment}} (\href{https://docs.python.org/2/library/functions.html\#str}{\emph{\texttt{str}}}) -- replaces file\_comment on plot

\item {} 
\textbf{\texttt{xbounds}} (\href{https://docs.python.org/2/library/array.html\#module-array}{\emph{\texttt{array}}}) -- Values to use instead of x axis bounds values

\item {} 
\textbf{\texttt{ybounds}} (\href{https://docs.python.org/2/library/array.html\#module-array}{\emph{\texttt{array}}}) -- Values to use instead of y axis bounds values (if exist)

\item {} 
\textbf{\texttt{xname}} (\href{https://docs.python.org/2/library/functions.html\#str}{\emph{\texttt{str}}}) -- replace xaxis name on plot

\item {} 
\textbf{\texttt{yname}} (\href{https://docs.python.org/2/library/functions.html\#str}{\emph{\texttt{str}}}) -- replace yaxis name on plot (if exists)

\item {} 
\textbf{\texttt{zname}} (\href{https://docs.python.org/2/library/functions.html\#str}{\emph{\texttt{str}}}) -- replace zaxis name on plot (if exists)

\item {} 
\textbf{\texttt{tname}} (\href{https://docs.python.org/2/library/functions.html\#str}{\emph{\texttt{str}}}) -- replace taxis name on plot (if exists)

\item {} 
\textbf{\texttt{wname}} (\href{https://docs.python.org/2/library/functions.html\#str}{\emph{\texttt{str}}}) -- replace waxis name on plot (if exists)

\item {} 
\textbf{\texttt{xunits}} (\href{https://docs.python.org/2/library/functions.html\#str}{\emph{\texttt{str}}}) -- replace xaxis units on plot

\item {} 
\textbf{\texttt{yunits}} (\href{https://docs.python.org/2/library/functions.html\#str}{\emph{\texttt{str}}}) -- replace yaxis units on plot (if exists)

\item {} 
\textbf{\texttt{zunits}} (\href{https://docs.python.org/2/library/functions.html\#str}{\emph{\texttt{str}}}) -- replace zaxis units on plot (if exists)

\item {} 
\textbf{\texttt{tunits}} (\href{https://docs.python.org/2/library/functions.html\#str}{\emph{\texttt{str}}}) -- replace taxis units on plot (if exists)

\item {} 
\textbf{\texttt{wunits}} (\href{https://docs.python.org/2/library/functions.html\#str}{\emph{\texttt{str}}}) -- replace waxis units on plot (if exists)

\item {} 
\textbf{\texttt{xweights}} (\href{https://docs.python.org/2/library/array.html\#module-array}{\emph{\texttt{array}}}) -- replace xaxis weights used for computing mean

\item {} 
\textbf{\texttt{yweights}} (\href{https://docs.python.org/2/library/array.html\#module-array}{\emph{\texttt{array}}}) -- replace xaxis weights used for computing mean

\item {} 
\textbf{\texttt{comment1}} (\href{https://docs.python.org/2/library/functions.html\#str}{\emph{\texttt{str}}}) -- replaces comment1 on plot

\item {} 
\textbf{\texttt{comment2}} (\href{https://docs.python.org/2/library/functions.html\#str}{\emph{\texttt{str}}}) -- replaces comment2 on plot

\item {} 
\textbf{\texttt{comment3}} (\href{https://docs.python.org/2/library/functions.html\#str}{\emph{\texttt{str}}}) -- replaces comment3 on plot

\item {} 
\textbf{\texttt{comment4}} (\href{https://docs.python.org/2/library/functions.html\#str}{\emph{\texttt{str}}}) -- replaces comment4 on plot

\item {} 
\textbf{\texttt{long\_name}} (\href{https://docs.python.org/2/library/functions.html\#str}{\emph{\texttt{str}}}) -- replaces long\_name on plot

\item {} 
\textbf{\texttt{grid}} (\emph{\texttt{cdms2.grid.TransientRectGrid}}) -- replaces array grid (if exists)

\item {} 
\textbf{\texttt{bg}} (\emph{\texttt{bool/int}}) -- plots in background mode

\item {} 
\textbf{\texttt{ratio}} (\index{xmtics1 (vcs.Canvas.Canvas attribute)}\index{xmtics2 (vcs.Canvas.Canvas attribute)}\index{ymtics1 (vcs.Canvas.Canvas attribute)}\index{ymtics2 (vcs.Canvas.Canvas attribute)}\index{xticlabels1 (vcs.Canvas.Canvas attribute)}\index{xticlabels2 (vcs.Canvas.Canvas attribute)}\index{yticlabels1 (vcs.Canvas.Canvas attribute)}\index{yticlabels2 (vcs.Canvas.Canvas attribute)}\index{projection (vcs.Canvas.Canvas attribute)}\index{datawc\_x1 (vcs.Canvas.Canvas attribute)}\index{datawc\_x2 (vcs.Canvas.Canvas attribute)}\index{datawc\_y1 (vcs.Canvas.Canvas attribute)}\index{datawc\_y2 (vcs.Canvas.Canvas attribute)}\index{datawc\_timeunits (vcs.Canvas.Canvas attribute)}\index{datawc\_calendar (vcs.Canvas.Canvas attribute)}) -- sets the y/x ratio ,if passed as a string with `t' at the end, will aslo moves the ticks

\item {} 
\textbf{\texttt{xaxisconvert}} (\href{https://docs.python.org/2/library/functions.html\#str}{\emph{\texttt{str}}}) -- (Ex: `linear') converting xaxis linear/log/log10/ln/exp/area\_wt

\item {} 
\textbf{\texttt{yaxisconvert}} (\href{https://docs.python.org/2/library/functions.html\#str}{\emph{\texttt{str}}}) -- (Ex: `linear') converting yaxis linear/log/log10/ln/exp/area\_wt

\item {} 
\textbf{\texttt{new\_GM\_name}} (\href{https://docs.python.org/2/library/functions.html\#str}{\emph{\texttt{str}}}) -- (Ex: `my\_awesome\_gm') name of the new graphics method object. If no name is given, then one will be created for use.

\item {} 
\textbf{\texttt{source\_GM\_name}} -- (Ex: `default') copy the contents of the source object to the newly created one. If no name is given, then the `default' graphics methond contents is copied over to the new object.

\end{itemize}

\item[{Returns}] \leavevmode
A XvsY graphics method object

\item[{Return type}] \leavevmode
{\hyperref[vcs/graphics/unified1D:vcs.unified1D.G1d]{\sphinxcrossref{vcs.unified1D.G1d}}}

\end{description}\end{quote}

\end{fulllineitems}

\index{createxyvsy() (vcs.Canvas.Canvas method)}

\begin{fulllineitems}
\phantomsection\label{vcs/Canvas:vcs.Canvas.Canvas.createxyvsy}\pysiglinewithargsret{\sphinxbfcode{createxyvsy}}{\emph{name=None}, \emph{source='default'}}{}
Create a new xyvsy graphics method given the the name and the existing
xyvsy graphics method to copy the attributes from. If no existing
xyvsy graphics method is given, then the default xyvsy graphics method will be used as the graphics method
to which the attributes will be copied from.

\begin{notice}{note}{Note:}
If the name provided already exists, then an error will be returned. graphics method
names must be unique.
\end{notice}
\begin{quote}\begin{description}
\item[{Example}] \leavevmode
\begin{Verbatim}[commandchars=\\\{\}]
\PYG{g+gp}{\PYGZgt{}\PYGZgt{}\PYGZgt{} }\PYG{n}{vcs}\PYG{o}{.}\PYG{n}{show}\PYG{p}{(}\PYG{l+s+s1}{\PYGZsq{}}\PYG{l+s+s1}{xyvsy}\PYG{l+s+s1}{\PYGZsq{}}\PYG{p}{)} \PYG{c+c1}{\PYGZsh{} show all available xyvsy}
\PYG{g+go}{*******************Xyvsy Names List**********************}
\PYG{g+gp}{...}
\PYG{g+go}{*******************End Xyvsy Names List**********************}
\PYG{g+gp}{\PYGZgt{}\PYGZgt{}\PYGZgt{} }\PYG{n}{ex}\PYG{o}{=}\PYG{n}{vcs}\PYG{o}{.}\PYG{n}{createxyvsy}\PYG{p}{(}\PYG{l+s+s1}{\PYGZsq{}}\PYG{l+s+s1}{xyvsy\PYGZus{}ex1}\PYG{l+s+s1}{\PYGZsq{}}\PYG{p}{)} \PYG{c+c1}{\PYGZsh{} Create xyvsy \PYGZsq{}xyvsy\PYGZus{}ex1\PYGZsq{} that inherits from \PYGZsq{}default\PYGZsq{}}
\PYG{g+gp}{\PYGZgt{}\PYGZgt{}\PYGZgt{} }\PYG{n}{vcs}\PYG{o}{.}\PYG{n}{listelements}\PYG{p}{(}\PYG{l+s+s1}{\PYGZsq{}}\PYG{l+s+s1}{xyvsy}\PYG{l+s+s1}{\PYGZsq{}}\PYG{p}{)} \PYG{c+c1}{\PYGZsh{} should now contain the \PYGZsq{}xyvsy\PYGZus{}ex1\PYGZsq{} xyvsy}
\PYG{g+go}{[...\PYGZsq{}xyvsy\PYGZus{}ex1\PYGZsq{}...]}
\end{Verbatim}

\item[{Parameters}] \leavevmode\begin{itemize}
\item {} 
\textbf{\texttt{name}} (\href{https://docs.python.org/2/library/functions.html\#str}{\emph{\texttt{str}}}) -- The name of the created object

\item {} 
\textbf{\texttt{source}} (\emph{\texttt{a xyvsy or a string name of a xyvsy}}) -- The object to inherit from

\item {} 
\textbf{\texttt{xaxis}} (\emph{\texttt{cdms2.axis.TransientAxis}}) -- Axis object to replace the slab -1 dim axis

\item {} 
\textbf{\texttt{yaxis}} (\emph{\texttt{cdms2.axis.TransientAxis}}) -- Axis object to replace the slab -2 dim axis, only if slab has more than 1D

\item {} 
\textbf{\texttt{zaxis}} (\emph{\texttt{cdms2.axis.TransientAxis}}) -- Axis object to replace the slab -3 dim axis, only if slab has more than 2D

\item {} 
\textbf{\texttt{taxis}} (\emph{\texttt{cdms2.axis.TransientAxis}}) -- Axis object to replace the slab -4 dim axis, only if slab has more than 3D

\item {} 
\textbf{\texttt{waxis}} (\emph{\texttt{cdms2.axis.TransientAxis}}) -- Axis object to replace the slab -5 dim axis, only if slab has more than 4D

\item {} 
\textbf{\texttt{xrev}} (\href{https://docs.python.org/2/library/functions.html\#bool}{\emph{\texttt{bool}}}) -- reverse x axis

\item {} 
\textbf{\texttt{yrev}} (\href{https://docs.python.org/2/library/functions.html\#bool}{\emph{\texttt{bool}}}) -- reverse y axis, only if slab has more than 1D

\item {} 
\textbf{\texttt{xarray}} (\href{https://docs.python.org/2/library/array.html\#module-array}{\emph{\texttt{array}}}) -- Values to use instead of x axis

\item {} 
\textbf{\texttt{yarray}} (\href{https://docs.python.org/2/library/array.html\#module-array}{\emph{\texttt{array}}}) -- Values to use instead of y axis, only if var has more than 1D

\item {} 
\textbf{\texttt{zarray}} (\href{https://docs.python.org/2/library/array.html\#module-array}{\emph{\texttt{array}}}) -- Values to use instead of z axis, only if var has more than 2D

\item {} 
\textbf{\texttt{tarray}} (\href{https://docs.python.org/2/library/array.html\#module-array}{\emph{\texttt{array}}}) -- Values to use instead of t axis, only if var has more than 3D

\item {} 
\textbf{\texttt{warray}} (\href{https://docs.python.org/2/library/array.html\#module-array}{\emph{\texttt{array}}}) -- Values to use instead of w axis, only if var has more than 4D

\item {} 
\textbf{\texttt{continents}} (\href{https://docs.python.org/2/library/functions.html\#int}{\emph{\texttt{int}}}) -- continents type number

\item {} 
\textbf{\texttt{name}} -- replaces variable name on plot

\item {} 
\textbf{\texttt{time}} (\emph{\texttt{A cdtime object}}) -- replaces time name on plot

\item {} 
\textbf{\texttt{units}} (\href{https://docs.python.org/2/library/functions.html\#str}{\emph{\texttt{str}}}) -- replaces units value on plot

\item {} 
\textbf{\texttt{ymd}} (\href{https://docs.python.org/2/library/functions.html\#str}{\emph{\texttt{str}}}) -- replaces year/month/day on plot

\item {} 
\textbf{\texttt{hms}} (\href{https://docs.python.org/2/library/functions.html\#str}{\emph{\texttt{str}}}) -- replaces hh/mm/ss on plot

\item {} 
\textbf{\texttt{file\_comment}} (\href{https://docs.python.org/2/library/functions.html\#str}{\emph{\texttt{str}}}) -- replaces file\_comment on plot

\item {} 
\textbf{\texttt{xbounds}} (\href{https://docs.python.org/2/library/array.html\#module-array}{\emph{\texttt{array}}}) -- Values to use instead of x axis bounds values

\item {} 
\textbf{\texttt{ybounds}} (\href{https://docs.python.org/2/library/array.html\#module-array}{\emph{\texttt{array}}}) -- Values to use instead of y axis bounds values (if exist)

\item {} 
\textbf{\texttt{xname}} (\href{https://docs.python.org/2/library/functions.html\#str}{\emph{\texttt{str}}}) -- replace xaxis name on plot

\item {} 
\textbf{\texttt{yname}} (\href{https://docs.python.org/2/library/functions.html\#str}{\emph{\texttt{str}}}) -- replace yaxis name on plot (if exists)

\item {} 
\textbf{\texttt{zname}} (\href{https://docs.python.org/2/library/functions.html\#str}{\emph{\texttt{str}}}) -- replace zaxis name on plot (if exists)

\item {} 
\textbf{\texttt{tname}} (\href{https://docs.python.org/2/library/functions.html\#str}{\emph{\texttt{str}}}) -- replace taxis name on plot (if exists)

\item {} 
\textbf{\texttt{wname}} (\href{https://docs.python.org/2/library/functions.html\#str}{\emph{\texttt{str}}}) -- replace waxis name on plot (if exists)

\item {} 
\textbf{\texttt{xunits}} (\href{https://docs.python.org/2/library/functions.html\#str}{\emph{\texttt{str}}}) -- replace xaxis units on plot

\item {} 
\textbf{\texttt{yunits}} (\href{https://docs.python.org/2/library/functions.html\#str}{\emph{\texttt{str}}}) -- replace yaxis units on plot (if exists)

\item {} 
\textbf{\texttt{zunits}} (\href{https://docs.python.org/2/library/functions.html\#str}{\emph{\texttt{str}}}) -- replace zaxis units on plot (if exists)

\item {} 
\textbf{\texttt{tunits}} (\href{https://docs.python.org/2/library/functions.html\#str}{\emph{\texttt{str}}}) -- replace taxis units on plot (if exists)

\item {} 
\textbf{\texttt{wunits}} (\href{https://docs.python.org/2/library/functions.html\#str}{\emph{\texttt{str}}}) -- replace waxis units on plot (if exists)

\item {} 
\textbf{\texttt{xweights}} (\href{https://docs.python.org/2/library/array.html\#module-array}{\emph{\texttt{array}}}) -- replace xaxis weights used for computing mean

\item {} 
\textbf{\texttt{yweights}} (\href{https://docs.python.org/2/library/array.html\#module-array}{\emph{\texttt{array}}}) -- replace xaxis weights used for computing mean

\item {} 
\textbf{\texttt{comment1}} (\href{https://docs.python.org/2/library/functions.html\#str}{\emph{\texttt{str}}}) -- replaces comment1 on plot

\item {} 
\textbf{\texttt{comment2}} (\href{https://docs.python.org/2/library/functions.html\#str}{\emph{\texttt{str}}}) -- replaces comment2 on plot

\item {} 
\textbf{\texttt{comment3}} (\href{https://docs.python.org/2/library/functions.html\#str}{\emph{\texttt{str}}}) -- replaces comment3 on plot

\item {} 
\textbf{\texttt{comment4}} (\href{https://docs.python.org/2/library/functions.html\#str}{\emph{\texttt{str}}}) -- replaces comment4 on plot

\item {} 
\textbf{\texttt{long\_name}} (\href{https://docs.python.org/2/library/functions.html\#str}{\emph{\texttt{str}}}) -- replaces long\_name on plot

\item {} 
\textbf{\texttt{grid}} (\emph{\texttt{cdms2.grid.TransientRectGrid}}) -- replaces array grid (if exists)

\item {} 
\textbf{\texttt{bg}} (\emph{\texttt{bool/int}}) -- plots in background mode

\item {} 
\textbf{\texttt{ratio}} (\index{xmtics1 (vcs.Canvas.Canvas attribute)}\index{xmtics2 (vcs.Canvas.Canvas attribute)}\index{ymtics1 (vcs.Canvas.Canvas attribute)}\index{ymtics2 (vcs.Canvas.Canvas attribute)}\index{xticlabels1 (vcs.Canvas.Canvas attribute)}\index{xticlabels2 (vcs.Canvas.Canvas attribute)}\index{yticlabels1 (vcs.Canvas.Canvas attribute)}\index{yticlabels2 (vcs.Canvas.Canvas attribute)}\index{projection (vcs.Canvas.Canvas attribute)}\index{datawc\_x1 (vcs.Canvas.Canvas attribute)}\index{datawc\_x2 (vcs.Canvas.Canvas attribute)}\index{datawc\_y1 (vcs.Canvas.Canvas attribute)}\index{datawc\_y2 (vcs.Canvas.Canvas attribute)}\index{datawc\_timeunits (vcs.Canvas.Canvas attribute)}\index{datawc\_calendar (vcs.Canvas.Canvas attribute)}) -- sets the y/x ratio ,if passed as a string with `t' at the end, will aslo moves the ticks

\item {} 
\textbf{\texttt{xaxisconvert}} (\href{https://docs.python.org/2/library/functions.html\#str}{\emph{\texttt{str}}}) -- (Ex: `linear') converting xaxis linear/log/log10/ln/exp/area\_wt

\item {} 
\textbf{\texttt{yaxisconvert}} (\href{https://docs.python.org/2/library/functions.html\#str}{\emph{\texttt{str}}}) -- (Ex: `linear') converting yaxis linear/log/log10/ln/exp/area\_wt

\item {} 
\textbf{\texttt{new\_GM\_name}} (\href{https://docs.python.org/2/library/functions.html\#str}{\emph{\texttt{str}}}) -- (Ex: `my\_awesome\_gm') name of the new graphics method object. If no name is given, then one will be created for use.

\item {} 
\textbf{\texttt{source\_GM\_name}} -- (Ex: `default') copy the contents of the source object to the newly created one. If no name is given, then the `default' graphics methond contents is copied over to the new object.

\end{itemize}

\item[{Returns}] \leavevmode
A XYvsY graphics method object

\item[{Return type}] \leavevmode
{\hyperref[vcs/graphics/unified1D:vcs.unified1D.G1d]{\sphinxcrossref{vcs.unified1D.G1d}}}

\end{description}\end{quote}

\end{fulllineitems}

\index{createyxvsx() (vcs.Canvas.Canvas method)}

\begin{fulllineitems}
\phantomsection\label{vcs/Canvas:vcs.Canvas.Canvas.createyxvsx}\pysiglinewithargsret{\sphinxbfcode{createyxvsx}}{\emph{name=None}, \emph{source='default'}}{}
Create a new yxvsx graphics method given the the name and the existing
yxvsx graphics method to copy the attributes from. If no existing
yxvsx graphics method is given, then the default yxvsx graphics method will be used as the graphics method
to which the attributes will be copied from.

\begin{notice}{note}{Note:}
If the name provided already exists, then an error will be returned. graphics method
names must be unique.
\end{notice}
\begin{quote}\begin{description}
\item[{Example}] \leavevmode
\begin{Verbatim}[commandchars=\\\{\}]
\PYG{g+gp}{\PYGZgt{}\PYGZgt{}\PYGZgt{} }\PYG{n}{vcs}\PYG{o}{.}\PYG{n}{show}\PYG{p}{(}\PYG{l+s+s1}{\PYGZsq{}}\PYG{l+s+s1}{yxvsx}\PYG{l+s+s1}{\PYGZsq{}}\PYG{p}{)} \PYG{c+c1}{\PYGZsh{} show all available yxvsx}
\PYG{g+go}{*******************Yxvsx Names List**********************}
\PYG{g+gp}{...}
\PYG{g+go}{*******************End Yxvsx Names List**********************}
\PYG{g+gp}{\PYGZgt{}\PYGZgt{}\PYGZgt{} }\PYG{n}{ex}\PYG{o}{=}\PYG{n}{vcs}\PYG{o}{.}\PYG{n}{createyxvsx}\PYG{p}{(}\PYG{l+s+s1}{\PYGZsq{}}\PYG{l+s+s1}{yxvsx\PYGZus{}ex1}\PYG{l+s+s1}{\PYGZsq{}}\PYG{p}{)} \PYG{c+c1}{\PYGZsh{} Create yxvsx \PYGZsq{}yxvsx\PYGZus{}ex1\PYGZsq{} that inherits from \PYGZsq{}default\PYGZsq{}}
\PYG{g+gp}{\PYGZgt{}\PYGZgt{}\PYGZgt{} }\PYG{n}{vcs}\PYG{o}{.}\PYG{n}{listelements}\PYG{p}{(}\PYG{l+s+s1}{\PYGZsq{}}\PYG{l+s+s1}{yxvsx}\PYG{l+s+s1}{\PYGZsq{}}\PYG{p}{)} \PYG{c+c1}{\PYGZsh{} should now contain the \PYGZsq{}yxvsx\PYGZus{}ex1\PYGZsq{} yxvsx}
\PYG{g+go}{[...\PYGZsq{}yxvsx\PYGZus{}ex1\PYGZsq{}...]}
\end{Verbatim}

\item[{Parameters}] \leavevmode\begin{itemize}
\item {} 
\textbf{\texttt{name}} (\href{https://docs.python.org/2/library/functions.html\#str}{\emph{\texttt{str}}}) -- The name of the created object

\item {} 
\textbf{\texttt{source}} (\emph{\texttt{a yxvsy or a string name of a yxvsy}}) -- The object to inherit from

\item {} 
\textbf{\texttt{xaxis}} (\emph{\texttt{cdms2.axis.TransientAxis}}) -- Axis object to replace the slab -1 dim axis

\item {} 
\textbf{\texttt{yaxis}} (\emph{\texttt{cdms2.axis.TransientAxis}}) -- Axis object to replace the slab -2 dim axis, only if slab has more than 1D

\item {} 
\textbf{\texttt{zaxis}} (\emph{\texttt{cdms2.axis.TransientAxis}}) -- Axis object to replace the slab -3 dim axis, only if slab has more than 2D

\item {} 
\textbf{\texttt{taxis}} (\emph{\texttt{cdms2.axis.TransientAxis}}) -- Axis object to replace the slab -4 dim axis, only if slab has more than 3D

\item {} 
\textbf{\texttt{waxis}} (\emph{\texttt{cdms2.axis.TransientAxis}}) -- Axis object to replace the slab -5 dim axis, only if slab has more than 4D

\item {} 
\textbf{\texttt{xrev}} (\href{https://docs.python.org/2/library/functions.html\#bool}{\emph{\texttt{bool}}}) -- reverse x axis

\item {} 
\textbf{\texttt{yrev}} (\href{https://docs.python.org/2/library/functions.html\#bool}{\emph{\texttt{bool}}}) -- reverse y axis, only if slab has more than 1D

\item {} 
\textbf{\texttt{xarray}} (\href{https://docs.python.org/2/library/array.html\#module-array}{\emph{\texttt{array}}}) -- Values to use instead of x axis

\item {} 
\textbf{\texttt{yarray}} (\href{https://docs.python.org/2/library/array.html\#module-array}{\emph{\texttt{array}}}) -- Values to use instead of y axis, only if var has more than 1D

\item {} 
\textbf{\texttt{zarray}} (\href{https://docs.python.org/2/library/array.html\#module-array}{\emph{\texttt{array}}}) -- Values to use instead of z axis, only if var has more than 2D

\item {} 
\textbf{\texttt{tarray}} (\href{https://docs.python.org/2/library/array.html\#module-array}{\emph{\texttt{array}}}) -- Values to use instead of t axis, only if var has more than 3D

\item {} 
\textbf{\texttt{warray}} (\href{https://docs.python.org/2/library/array.html\#module-array}{\emph{\texttt{array}}}) -- Values to use instead of w axis, only if var has more than 4D

\item {} 
\textbf{\texttt{continents}} (\href{https://docs.python.org/2/library/functions.html\#int}{\emph{\texttt{int}}}) -- continents type number

\item {} 
\textbf{\texttt{name}} -- replaces variable name on plot

\item {} 
\textbf{\texttt{time}} (\emph{\texttt{A cdtime object}}) -- replaces time name on plot

\item {} 
\textbf{\texttt{units}} (\href{https://docs.python.org/2/library/functions.html\#str}{\emph{\texttt{str}}}) -- replaces units value on plot

\item {} 
\textbf{\texttt{ymd}} (\href{https://docs.python.org/2/library/functions.html\#str}{\emph{\texttt{str}}}) -- replaces year/month/day on plot

\item {} 
\textbf{\texttt{hms}} (\href{https://docs.python.org/2/library/functions.html\#str}{\emph{\texttt{str}}}) -- replaces hh/mm/ss on plot

\item {} 
\textbf{\texttt{file\_comment}} (\href{https://docs.python.org/2/library/functions.html\#str}{\emph{\texttt{str}}}) -- replaces file\_comment on plot

\item {} 
\textbf{\texttt{xbounds}} (\href{https://docs.python.org/2/library/array.html\#module-array}{\emph{\texttt{array}}}) -- Values to use instead of x axis bounds values

\item {} 
\textbf{\texttt{ybounds}} (\href{https://docs.python.org/2/library/array.html\#module-array}{\emph{\texttt{array}}}) -- Values to use instead of y axis bounds values (if exist)

\item {} 
\textbf{\texttt{xname}} (\href{https://docs.python.org/2/library/functions.html\#str}{\emph{\texttt{str}}}) -- replace xaxis name on plot

\item {} 
\textbf{\texttt{yname}} (\href{https://docs.python.org/2/library/functions.html\#str}{\emph{\texttt{str}}}) -- replace yaxis name on plot (if exists)

\item {} 
\textbf{\texttt{zname}} (\href{https://docs.python.org/2/library/functions.html\#str}{\emph{\texttt{str}}}) -- replace zaxis name on plot (if exists)

\item {} 
\textbf{\texttt{tname}} (\href{https://docs.python.org/2/library/functions.html\#str}{\emph{\texttt{str}}}) -- replace taxis name on plot (if exists)

\item {} 
\textbf{\texttt{wname}} (\href{https://docs.python.org/2/library/functions.html\#str}{\emph{\texttt{str}}}) -- replace waxis name on plot (if exists)

\item {} 
\textbf{\texttt{xunits}} (\href{https://docs.python.org/2/library/functions.html\#str}{\emph{\texttt{str}}}) -- replace xaxis units on plot

\item {} 
\textbf{\texttt{yunits}} (\href{https://docs.python.org/2/library/functions.html\#str}{\emph{\texttt{str}}}) -- replace yaxis units on plot (if exists)

\item {} 
\textbf{\texttt{zunits}} (\href{https://docs.python.org/2/library/functions.html\#str}{\emph{\texttt{str}}}) -- replace zaxis units on plot (if exists)

\item {} 
\textbf{\texttt{tunits}} (\href{https://docs.python.org/2/library/functions.html\#str}{\emph{\texttt{str}}}) -- replace taxis units on plot (if exists)

\item {} 
\textbf{\texttt{wunits}} (\href{https://docs.python.org/2/library/functions.html\#str}{\emph{\texttt{str}}}) -- replace waxis units on plot (if exists)

\item {} 
\textbf{\texttt{xweights}} (\href{https://docs.python.org/2/library/array.html\#module-array}{\emph{\texttt{array}}}) -- replace xaxis weights used for computing mean

\item {} 
\textbf{\texttt{yweights}} (\href{https://docs.python.org/2/library/array.html\#module-array}{\emph{\texttt{array}}}) -- replace xaxis weights used for computing mean

\item {} 
\textbf{\texttt{comment1}} (\href{https://docs.python.org/2/library/functions.html\#str}{\emph{\texttt{str}}}) -- replaces comment1 on plot

\item {} 
\textbf{\texttt{comment2}} (\href{https://docs.python.org/2/library/functions.html\#str}{\emph{\texttt{str}}}) -- replaces comment2 on plot

\item {} 
\textbf{\texttt{comment3}} (\href{https://docs.python.org/2/library/functions.html\#str}{\emph{\texttt{str}}}) -- replaces comment3 on plot

\item {} 
\textbf{\texttt{comment4}} (\href{https://docs.python.org/2/library/functions.html\#str}{\emph{\texttt{str}}}) -- replaces comment4 on plot

\item {} 
\textbf{\texttt{long\_name}} (\href{https://docs.python.org/2/library/functions.html\#str}{\emph{\texttt{str}}}) -- replaces long\_name on plot

\item {} 
\textbf{\texttt{grid}} (\emph{\texttt{cdms2.grid.TransientRectGrid}}) -- replaces array grid (if exists)

\item {} 
\textbf{\texttt{bg}} (\emph{\texttt{bool/int}}) -- plots in background mode

\item {} 
\textbf{\texttt{ratio}} (\index{xmtics1 (vcs.Canvas.Canvas attribute)}\index{xmtics2 (vcs.Canvas.Canvas attribute)}\index{ymtics1 (vcs.Canvas.Canvas attribute)}\index{ymtics2 (vcs.Canvas.Canvas attribute)}\index{xticlabels1 (vcs.Canvas.Canvas attribute)}\index{xticlabels2 (vcs.Canvas.Canvas attribute)}\index{yticlabels1 (vcs.Canvas.Canvas attribute)}\index{yticlabels2 (vcs.Canvas.Canvas attribute)}\index{projection (vcs.Canvas.Canvas attribute)}\index{datawc\_x1 (vcs.Canvas.Canvas attribute)}\index{datawc\_x2 (vcs.Canvas.Canvas attribute)}\index{datawc\_y1 (vcs.Canvas.Canvas attribute)}\index{datawc\_y2 (vcs.Canvas.Canvas attribute)}\index{datawc\_timeunits (vcs.Canvas.Canvas attribute)}\index{datawc\_calendar (vcs.Canvas.Canvas attribute)}) -- sets the y/x ratio ,if passed as a string with `t' at the end, will aslo moves the ticks

\item {} 
\textbf{\texttt{xaxisconvert}} (\href{https://docs.python.org/2/library/functions.html\#str}{\emph{\texttt{str}}}) -- (Ex: `linear') converting xaxis linear/log/log10/ln/exp/area\_wt

\item {} 
\textbf{\texttt{yaxisconvert}} (\href{https://docs.python.org/2/library/functions.html\#str}{\emph{\texttt{str}}}) -- (Ex: `linear') converting yaxis linear/log/log10/ln/exp/area\_wt

\item {} 
\textbf{\texttt{new\_GM\_name}} (\href{https://docs.python.org/2/library/functions.html\#str}{\emph{\texttt{str}}}) -- (Ex: `my\_awesome\_gm') name of the new graphics method object. If no name is given, then one will be created for use.

\item {} 
\textbf{\texttt{source\_GM\_name}} -- (Ex: `default') copy the contents of the source object to the newly created one. If no name is given, then the `default' graphics methond contents is copied over to the new object.

\end{itemize}

\item[{Returns}] \leavevmode
A YXvsX graphics method object

\item[{Return type}] \leavevmode
{\hyperref[vcs/graphics/unified1D:vcs.unified1D.G1d]{\sphinxcrossref{vcs.unified1D.G1d}}}

\end{description}\end{quote}

\end{fulllineitems}

\index{destroy() (vcs.Canvas.Canvas method)}

\begin{fulllineitems}
\phantomsection\label{vcs/Canvas:vcs.Canvas.Canvas.destroy}\pysiglinewithargsret{\sphinxbfcode{destroy}}{}{}
Destroy the VCS Canvas. It will deallocate the VCS Canvas object.
\begin{quote}\begin{description}
\item[{Example}] \leavevmode
\begin{Verbatim}[commandchars=\\\{\}]
\PYG{g+gp}{\PYGZgt{}\PYGZgt{}\PYGZgt{} }\PYG{n}{a}\PYG{o}{=}\PYG{n}{vcs}\PYG{o}{.}\PYG{n}{init}\PYG{p}{(}\PYG{p}{)}
\PYG{g+gp}{\PYGZgt{}\PYGZgt{}\PYGZgt{} }\PYG{n}{array} \PYG{o}{=} \PYG{p}{[}\PYG{n+nb}{range}\PYG{p}{(}\PYG{l+m+mi}{1}\PYG{p}{,} \PYG{l+m+mi}{11}\PYG{p}{)} \PYG{k}{for} \PYG{n}{\PYGZus{}} \PYG{o+ow}{in} \PYG{n+nb}{range}\PYG{p}{(}\PYG{l+m+mi}{1}\PYG{p}{,} \PYG{l+m+mi}{11}\PYG{p}{)}\PYG{p}{]}
\PYG{g+gp}{\PYGZgt{}\PYGZgt{}\PYGZgt{} }\PYG{n}{a}\PYG{o}{.}\PYG{n}{plot}\PYG{p}{(}\PYG{n}{array}\PYG{p}{,}\PYG{l+s+s1}{\PYGZsq{}}\PYG{l+s+s1}{default}\PYG{l+s+s1}{\PYGZsq{}}\PYG{p}{,}\PYG{l+s+s1}{\PYGZsq{}}\PYG{l+s+s1}{isofill}\PYG{l+s+s1}{\PYGZsq{}}\PYG{p}{,}\PYG{l+s+s1}{\PYGZsq{}}\PYG{l+s+s1}{quick}\PYG{l+s+s1}{\PYGZsq{}}\PYG{p}{)}
\PYG{g+go}{\PYGZlt{}vcs.displayplot.Dp ...\PYGZgt{}}
\PYG{g+gp}{\PYGZgt{}\PYGZgt{}\PYGZgt{} }\PYG{n}{a}\PYG{o}{.}\PYG{n}{destroy}\PYG{p}{(}\PYG{p}{)}
\end{Verbatim}

\end{description}\end{quote}

\end{fulllineitems}

\index{drawfillarea() (vcs.Canvas.Canvas method)}

\begin{fulllineitems}
\phantomsection\label{vcs/Canvas:vcs.Canvas.Canvas.drawfillarea}\pysiglinewithargsret{\sphinxbfcode{drawfillarea}}{\emph{name=None, style=1, index=1, color=241, priority=1, viewport={[}0.0, 1.0, 0.0, 1.0{]}, worldcoordinate={[}0.0, 1.0, 0.0, 1.0{]}, x=None, y=None, bg=0}}{}
Generate and draw a fillarea object on the VCS Canvas.
\begin{quote}\begin{description}
\item[{Example}] \leavevmode
\begin{Verbatim}[commandchars=\\\{\}]
\PYG{g+gp}{\PYGZgt{}\PYGZgt{}\PYGZgt{} }\PYG{n}{a}\PYG{o}{=}\PYG{n}{vcs}\PYG{o}{.}\PYG{n}{init}\PYG{p}{(}\PYG{p}{)}
\PYG{g+gp}{\PYGZgt{}\PYGZgt{}\PYGZgt{} }\PYG{n}{a}\PYG{o}{.}\PYG{n}{show}\PYG{p}{(}\PYG{l+s+s1}{\PYGZsq{}}\PYG{l+s+s1}{fillarea}\PYG{l+s+s1}{\PYGZsq{}}\PYG{p}{)} \PYG{c+c1}{\PYGZsh{} Show all the existing fillarea objects}
\PYG{g+go}{*******************Fillarea Names List**********************}
\PYG{g+gp}{...}
\PYG{g+go}{*******************End Fillarea Names List**********************}
\PYG{g+gp}{\PYGZgt{}\PYGZgt{}\PYGZgt{} }\PYG{n}{fa}\PYG{o}{=}\PYG{n}{a}\PYG{o}{.}\PYG{n}{drawfillarea}\PYG{p}{(}\PYG{n}{name}\PYG{o}{=}\PYG{l+s+s1}{\PYGZsq{}}\PYG{l+s+s1}{red}\PYG{l+s+s1}{\PYGZsq{}}\PYG{p}{,} \PYG{n}{style}\PYG{o}{=}\PYG{l+m+mi}{1}\PYG{p}{,} \PYG{n}{color}\PYG{o}{=}\PYG{l+m+mi}{242}\PYG{p}{,}
\PYG{g+gp}{... }             \PYG{n}{priority}\PYG{o}{=}\PYG{l+m+mi}{1}\PYG{p}{,} \PYG{n}{viewport}\PYG{o}{=}\PYG{p}{[}\PYG{l+m+mi}{0}\PYG{p}{,} \PYG{l+m+mf}{1.0}\PYG{p}{,} \PYG{l+m+mi}{0}\PYG{p}{,} \PYG{l+m+mf}{1.0}\PYG{p}{]}\PYG{p}{,}
\PYG{g+gp}{... }             \PYG{n}{worldcoordinate}\PYG{o}{=}\PYG{p}{[}\PYG{l+m+mi}{0}\PYG{p}{,}\PYG{l+m+mi}{100}\PYG{p}{,} \PYG{l+m+mi}{0}\PYG{p}{,}\PYG{l+m+mi}{50}\PYG{p}{]}\PYG{p}{,}
\PYG{g+gp}{... }             \PYG{n}{x}\PYG{o}{=}\PYG{p}{[}\PYG{l+m+mi}{0}\PYG{p}{,}\PYG{l+m+mi}{20}\PYG{p}{,}\PYG{l+m+mi}{40}\PYG{p}{,}\PYG{l+m+mi}{60}\PYG{p}{,}\PYG{l+m+mi}{80}\PYG{p}{,}\PYG{l+m+mi}{100}\PYG{p}{]}\PYG{p}{,}
\PYG{g+gp}{... }             \PYG{n}{y}\PYG{o}{=}\PYG{p}{[}\PYG{l+m+mi}{0}\PYG{p}{,}\PYG{l+m+mi}{10}\PYG{p}{,}\PYG{l+m+mi}{20}\PYG{p}{,}\PYG{l+m+mi}{30}\PYG{p}{,}\PYG{l+m+mi}{40}\PYG{p}{,}\PYG{l+m+mi}{50}\PYG{p}{]}\PYG{p}{,} \PYG{n}{bg}\PYG{o}{=}\PYG{l+m+mi}{0} \PYG{p}{)} \PYG{c+c1}{\PYGZsh{} Create instance of fillarea object \PYGZsq{}red\PYGZsq{}}
\PYG{g+gp}{\PYGZgt{}\PYGZgt{}\PYGZgt{} }\PYG{n}{a}\PYG{o}{.}\PYG{n}{fillarea}\PYG{p}{(}\PYG{n}{fa}\PYG{p}{)} \PYG{c+c1}{\PYGZsh{} Plot using specified fillarea object}
\PYG{g+go}{\PYGZlt{}vcs.displayplot.Dp ...\PYGZgt{}}
\end{Verbatim}

\item[{Parameters}] \leavevmode\begin{itemize}
\item {} 
\textbf{\texttt{name}} (\href{https://docs.python.org/2/library/functions.html\#str}{\emph{\texttt{str}}}) -- Name of created object

\item {} 
\textbf{\texttt{style}} (\href{https://docs.python.org/2/library/functions.html\#str}{\emph{\texttt{str}}}) -- One of ``hatch'', ``solid'', or ``pattern''.

\item {} 
\textbf{\texttt{index}} -- 
Specifies which \href{http://uvcdat.llnl.gov/gallery/fullsize/pattern\_chart.png}{pattern}


\end{itemize}

\end{description}\end{quote}

to fill the fillarea with. Accepts ints from 1-20.
\begin{quote}\begin{description}
\item[{Parameters}] \leavevmode
\textbf{\texttt{color}} -- 
A color name from the \href{https://en.wikipedia.org/wiki/X11\_color\_names}{X11 Color Names list},


\end{description}\end{quote}

or an integer value from 0-255, or an RGB/RGBA tuple/list (e.g. (0,100,0), (100,100,0,50))
\begin{quote}\begin{description}
\item[{Parameters}] \leavevmode\begin{itemize}
\item {} 
\textbf{\texttt{priority}} (\href{https://docs.python.org/2/library/functions.html\#int}{\emph{\texttt{int}}}) -- The layer on which the fillarea will be drawn.

\item {} 
\textbf{\texttt{viewport}} (\emph{\texttt{list of floats}}) -- 4 floats between 0 and 1.
These specify the area that the X/Y values are mapped to inside of the canvas

\item {} 
\textbf{\texttt{worldcoordinate}} (\emph{\texttt{list of floats}}) -- List of 4 floats (xmin, xmax, ymin, ymax)

\item {} 
\textbf{\texttt{x}} (\emph{\texttt{list of floats}}) -- List of lists of x coordinates. Values must be between worldcoordinate{[}0{]} and worldcoordinate{[}1{]}.

\item {} 
\textbf{\texttt{y}} (\emph{\texttt{list of floats}}) -- List of lists of y coordinates. Values must be between worldcoordinate{[}2{]} and worldcoordinate{[}3{]}.

\item {} 
\textbf{\texttt{bg}} (\href{https://docs.python.org/2/library/functions.html\#bool}{\emph{\texttt{bool}}}) -- Boolean value. True =\textgreater{} object drawn in background (not shown on canvas).
False =\textgreater{} object shown on canvas.

\end{itemize}

\item[{Returns}] \leavevmode
A fillarea object

\item[{Return type}] \leavevmode
{\hyperref[vcs/secondary/fillarea:vcs.fillarea.Tf]{\sphinxcrossref{vcs.fillarea.Tf}}}

\end{description}\end{quote}

\end{fulllineitems}

\index{drawline() (vcs.Canvas.Canvas method)}

\begin{fulllineitems}
\phantomsection\label{vcs/Canvas:vcs.Canvas.Canvas.drawline}\pysiglinewithargsret{\sphinxbfcode{drawline}}{\emph{name=None, ltype='solid', width=1, color=241, priority=1, viewport={[}0.0, 1.0, 0.0, 1.0{]}, worldcoordinate={[}0.0, 1.0, 0.0, 1.0{]}, x=None, y=None, projection='default', bg=0}}{}
Generate and draw a line object on the VCS Canvas.
\begin{quote}\begin{description}
\item[{Example}] \leavevmode
\begin{Verbatim}[commandchars=\\\{\}]
\PYG{g+gp}{\PYGZgt{}\PYGZgt{}\PYGZgt{} }\PYG{n}{a}\PYG{o}{=}\PYG{n}{vcs}\PYG{o}{.}\PYG{n}{init}\PYG{p}{(}\PYG{p}{)}
\PYG{g+gp}{\PYGZgt{}\PYGZgt{}\PYGZgt{} }\PYG{n}{a}\PYG{o}{.}\PYG{n}{show}\PYG{p}{(}\PYG{l+s+s1}{\PYGZsq{}}\PYG{l+s+s1}{line}\PYG{l+s+s1}{\PYGZsq{}}\PYG{p}{)} \PYG{c+c1}{\PYGZsh{} Show all the existing line objects}
\PYG{g+go}{*******************Line Names List**********************}
\PYG{g+gp}{...}
\PYG{g+go}{*******************End Line Names List**********************}
\PYG{g+gp}{\PYGZgt{}\PYGZgt{}\PYGZgt{} }\PYG{n}{ln}\PYG{o}{=}\PYG{n}{a}\PYG{o}{.}\PYG{n}{drawline}\PYG{p}{(}\PYG{n}{name}\PYG{o}{=}\PYG{l+s+s1}{\PYGZsq{}}\PYG{l+s+s1}{red}\PYG{l+s+s1}{\PYGZsq{}}\PYG{p}{,} \PYG{n}{ltype}\PYG{o}{=}\PYG{l+s+s1}{\PYGZsq{}}\PYG{l+s+s1}{dash}\PYG{l+s+s1}{\PYGZsq{}}\PYG{p}{,} \PYG{n}{width}\PYG{o}{=}\PYG{l+m+mi}{2}\PYG{p}{,}
\PYG{g+gp}{... }             \PYG{n}{color}\PYG{o}{=}\PYG{l+m+mi}{242}\PYG{p}{,} \PYG{n}{priority}\PYG{o}{=}\PYG{l+m+mi}{1}\PYG{p}{,} \PYG{n}{viewport}\PYG{o}{=}\PYG{p}{[}\PYG{l+m+mi}{0}\PYG{p}{,} \PYG{l+m+mf}{1.0}\PYG{p}{,} \PYG{l+m+mi}{0}\PYG{p}{,} \PYG{l+m+mf}{1.0}\PYG{p}{]}\PYG{p}{,}
\PYG{g+gp}{... }             \PYG{n}{worldcoordinate}\PYG{o}{=}\PYG{p}{[}\PYG{l+m+mi}{0}\PYG{p}{,}\PYG{l+m+mi}{100}\PYG{p}{,} \PYG{l+m+mi}{0}\PYG{p}{,}\PYG{l+m+mi}{50}\PYG{p}{]}\PYG{p}{,}
\PYG{g+gp}{... }             \PYG{n}{x}\PYG{o}{=}\PYG{p}{[}\PYG{l+m+mi}{0}\PYG{p}{,}\PYG{l+m+mi}{20}\PYG{p}{,}\PYG{l+m+mi}{40}\PYG{p}{,}\PYG{l+m+mi}{60}\PYG{p}{,}\PYG{l+m+mi}{80}\PYG{p}{,}\PYG{l+m+mi}{100}\PYG{p}{]}\PYG{p}{,}
\PYG{g+gp}{... }             \PYG{n}{y}\PYG{o}{=}\PYG{p}{[}\PYG{l+m+mi}{0}\PYG{p}{,}\PYG{l+m+mi}{10}\PYG{p}{,}\PYG{l+m+mi}{20}\PYG{p}{,}\PYG{l+m+mi}{30}\PYG{p}{,}\PYG{l+m+mi}{40}\PYG{p}{,}\PYG{l+m+mi}{50}\PYG{p}{]} \PYG{p}{)}
\PYG{g+gp}{\PYGZgt{}\PYGZgt{}\PYGZgt{} }\PYG{n}{a}\PYG{o}{.}\PYG{n}{line}\PYG{p}{(}\PYG{n}{ln}\PYG{p}{)} \PYG{c+c1}{\PYGZsh{} Plot using specified line object}
\PYG{g+go}{\PYGZlt{}vcs.displayplot.Dp ...\PYGZgt{}}
\end{Verbatim}

\item[{Parameters}] \leavevmode\begin{itemize}
\item {} 
\textbf{\texttt{name}} (\href{https://docs.python.org/2/library/functions.html\#str}{\emph{\texttt{str}}}) -- Name of created object

\item {} 
\textbf{\texttt{ltype}} (\href{https://docs.python.org/2/library/functions.html\#str}{\emph{\texttt{str}}}) -- One of ``dash'', ``dash-dot'', ``solid'', ``dot'', or ``long-dash''.

\item {} 
\textbf{\texttt{width}} (\href{https://docs.python.org/2/library/functions.html\#int}{\emph{\texttt{int}}}) -- Thickness of the line to be drawn

\item {} 
\textbf{\texttt{color}} (\emph{\texttt{str or int}}) -- 
A color name from the \href{https://en.wikipedia.org/wiki/X11\_color\_names}{X11 Color Names list},
or an integer value from 0-255, or an RGB/RGBA tuple/list (e.g. (0,100,0), (100,100,0,50))


\item {} 
\textbf{\texttt{priority}} (\href{https://docs.python.org/2/library/functions.html\#int}{\emph{\texttt{int}}}) -- The layer on which the line will be drawn.

\item {} 
\textbf{\texttt{viewport}} (\emph{\texttt{list of floats}}) -- 4 floats between 0 and 1.
These specify the area that the X/Y values are mapped to inside of the canvas

\item {} 
\textbf{\texttt{worldcoordinate}} (\emph{\texttt{list of floats}}) -- List of 4 floats (xmin, xmax, ymin, ymax)

\item {} 
\textbf{\texttt{x}} (\emph{\texttt{list of floats}}) -- List of lists of x coordinates. Values must be between worldcoordinate{[}0{]} and worldcoordinate{[}1{]}.

\item {} 
\textbf{\texttt{y}} (\emph{\texttt{list of floats}}) -- List of lists of y coordinates. Values must be between worldcoordinate{[}2{]} and worldcoordinate{[}3{]}.

\item {} 
\textbf{\texttt{projection}} (\emph{\texttt{str or projection object}}) -- Specify a geographic projection used to convert x/y
from spherical coordinates into 2D coordinates.

\end{itemize}

\item[{Returns}] \leavevmode
A VCS line object

\item[{Return type}] \leavevmode
{\hyperref[vcs/secondary/line:vcs.line.Tl]{\sphinxcrossref{vcs.line.Tl}}}

\end{description}\end{quote}

\end{fulllineitems}

\index{drawlogooff() (vcs.Canvas.Canvas method)}

\begin{fulllineitems}
\phantomsection\label{vcs/Canvas:vcs.Canvas.Canvas.drawlogooff}\pysiglinewithargsret{\sphinxbfcode{drawlogooff}}{}{}
Hide UV-CDAT logo on the canvas
\begin{quote}\begin{description}
\item[{Example}] \leavevmode
\begin{Verbatim}[commandchars=\\\{\}]
\PYG{g+gp}{\PYGZgt{}\PYGZgt{}\PYGZgt{} }\PYG{n}{a}\PYG{o}{=}\PYG{n}{vcs}\PYG{o}{.}\PYG{n}{init}\PYG{p}{(}\PYG{p}{)}
\PYG{g+gp}{\PYGZgt{}\PYGZgt{}\PYGZgt{} }\PYG{n}{a}\PYG{o}{.}\PYG{n}{drawlogooff}\PYG{p}{(}\PYG{p}{)}
\PYG{g+gp}{\PYGZgt{}\PYGZgt{}\PYGZgt{} }\PYG{n}{a}\PYG{o}{.}\PYG{n}{getdrawlogo}\PYG{p}{(}\PYG{p}{)}
\PYG{g+go}{False}
\end{Verbatim}

\end{description}\end{quote}

\end{fulllineitems}

\index{drawlogoon() (vcs.Canvas.Canvas method)}

\begin{fulllineitems}
\phantomsection\label{vcs/Canvas:vcs.Canvas.Canvas.drawlogoon}\pysiglinewithargsret{\sphinxbfcode{drawlogoon}}{}{}
Show UV-CDAT logo on the canvas
\begin{quote}\begin{description}
\item[{Example}] \leavevmode
\begin{Verbatim}[commandchars=\\\{\}]
\PYG{g+gp}{\PYGZgt{}\PYGZgt{}\PYGZgt{} }\PYG{n}{a}\PYG{o}{=}\PYG{n}{vcs}\PYG{o}{.}\PYG{n}{init}\PYG{p}{(}\PYG{p}{)}
\PYG{g+gp}{\PYGZgt{}\PYGZgt{}\PYGZgt{} }\PYG{n}{a}\PYG{o}{.}\PYG{n}{drawlogoon}\PYG{p}{(}\PYG{p}{)}
\PYG{g+gp}{\PYGZgt{}\PYGZgt{}\PYGZgt{} }\PYG{n}{a}\PYG{o}{.}\PYG{n}{getdrawlogo}\PYG{p}{(}\PYG{p}{)}
\PYG{g+go}{True}
\end{Verbatim}

\end{description}\end{quote}

\end{fulllineitems}

\index{drawmarker() (vcs.Canvas.Canvas method)}

\begin{fulllineitems}
\phantomsection\label{vcs/Canvas:vcs.Canvas.Canvas.drawmarker}\pysiglinewithargsret{\sphinxbfcode{drawmarker}}{\emph{name=None, mtype='solid', size=1, color=241, priority=1, viewport={[}0.0, 1.0, 0.0, 1.0{]}, worldcoordinate={[}0.0, 1.0, 0.0, 1.0{]}, x=None, y=None, bg=0}}{}
Generate and draw a marker object on the VCS Canvas.
\begin{quote}\begin{description}
\item[{Example}] \leavevmode
\begin{Verbatim}[commandchars=\\\{\}]
\PYG{g+gp}{\PYGZgt{}\PYGZgt{}\PYGZgt{} }\PYG{n}{a}\PYG{o}{=}\PYG{n}{vcs}\PYG{o}{.}\PYG{n}{init}\PYG{p}{(}\PYG{p}{)}
\PYG{g+gp}{\PYGZgt{}\PYGZgt{}\PYGZgt{} }\PYG{n}{a}\PYG{o}{.}\PYG{n}{show}\PYG{p}{(}\PYG{l+s+s1}{\PYGZsq{}}\PYG{l+s+s1}{marker}\PYG{l+s+s1}{\PYGZsq{}}\PYG{p}{)}  \PYG{c+c1}{\PYGZsh{} Show all the existing marker objects}
\PYG{g+go}{*******************Marker Names List**********************}
\PYG{g+gp}{...}
\PYG{g+go}{*******************End Marker Names List**********************}
\PYG{g+gp}{\PYGZgt{}\PYGZgt{}\PYGZgt{} }\PYG{n}{mrk}\PYG{o}{=}\PYG{n}{a}\PYG{o}{.}\PYG{n}{drawmarker}\PYG{p}{(}\PYG{n}{name}\PYG{o}{=}\PYG{l+s+s1}{\PYGZsq{}}\PYG{l+s+s1}{red}\PYG{l+s+s1}{\PYGZsq{}}\PYG{p}{,} \PYG{n}{mtype}\PYG{o}{=}\PYG{l+s+s1}{\PYGZsq{}}\PYG{l+s+s1}{dot}\PYG{l+s+s1}{\PYGZsq{}}\PYG{p}{,} \PYG{n}{size}\PYG{o}{=}\PYG{l+m+mi}{2}\PYG{p}{,}
\PYG{g+gp}{... }             \PYG{n}{color}\PYG{o}{=}\PYG{l+m+mi}{242}\PYG{p}{,} \PYG{n}{priority}\PYG{o}{=}\PYG{l+m+mi}{1}\PYG{p}{,} \PYG{n}{viewport}\PYG{o}{=}\PYG{p}{[}\PYG{l+m+mi}{0}\PYG{p}{,} \PYG{l+m+mf}{1.0}\PYG{p}{,} \PYG{l+m+mi}{0}\PYG{p}{,} \PYG{l+m+mf}{1.0}\PYG{p}{]}\PYG{p}{,}
\PYG{g+gp}{... }             \PYG{n}{worldcoordinate}\PYG{o}{=}\PYG{p}{[}\PYG{l+m+mi}{0}\PYG{p}{,}\PYG{l+m+mi}{100}\PYG{p}{,} \PYG{l+m+mi}{0}\PYG{p}{,}\PYG{l+m+mi}{50}\PYG{p}{]}\PYG{p}{,}
\PYG{g+gp}{... }             \PYG{n}{x}\PYG{o}{=}\PYG{p}{[}\PYG{l+m+mi}{0}\PYG{p}{,}\PYG{l+m+mi}{20}\PYG{p}{,}\PYG{l+m+mi}{40}\PYG{p}{,}\PYG{l+m+mi}{60}\PYG{p}{,}\PYG{l+m+mi}{80}\PYG{p}{,}\PYG{l+m+mi}{100}\PYG{p}{]}\PYG{p}{,}
\PYG{g+gp}{... }             \PYG{n}{y}\PYG{o}{=}\PYG{p}{[}\PYG{l+m+mi}{0}\PYG{p}{,}\PYG{l+m+mi}{10}\PYG{p}{,}\PYG{l+m+mi}{20}\PYG{p}{,}\PYG{l+m+mi}{30}\PYG{p}{,}\PYG{l+m+mi}{40}\PYG{p}{,}\PYG{l+m+mi}{50}\PYG{p}{]} \PYG{p}{)} \PYG{c+c1}{\PYGZsh{} Create instance of marker object \PYGZsq{}red\PYGZsq{}}
\PYG{g+gp}{\PYGZgt{}\PYGZgt{}\PYGZgt{} }\PYG{n}{a}\PYG{o}{.}\PYG{n}{marker}\PYG{p}{(}\PYG{n}{mrk}\PYG{p}{)} \PYG{c+c1}{\PYGZsh{} Plot using specified marker object}
\PYG{g+go}{\PYGZlt{}vcs.displayplot.Dp ...\PYGZgt{}}
\end{Verbatim}

\item[{Parameters}] \leavevmode\begin{itemize}
\item {} 
\textbf{\texttt{name}} (\href{https://docs.python.org/2/library/functions.html\#str}{\emph{\texttt{str}}}) -- Name of created object

\item {} 
\textbf{\texttt{mtype}} (\href{https://docs.python.org/2/library/functions.html\#str}{\emph{\texttt{str}}}) -- Marker type, i.e. `dot', `plus', `star, etc.

\item {} 
\textbf{\texttt{size}} (\href{https://docs.python.org/2/library/functions.html\#int}{\emph{\texttt{int}}}) -- Size of the marker to draw

\item {} 
\textbf{\texttt{color}} (\emph{\texttt{str or int}}) -- 
A color name from the \href{https://en.wikipedia.org/wiki/X11\_color\_names}{X11 Color Names list},
or an integer value from 0-255, or an RGB/RGBA tuple/list (e.g. (0,100,0), (100,100,0,50))


\item {} 
\textbf{\texttt{priority}} (\href{https://docs.python.org/2/library/functions.html\#int}{\emph{\texttt{int}}}) -- The layer on which the marker will be drawn.

\item {} 
\textbf{\texttt{viewport}} (\emph{\texttt{list of floats}}) -- 4 floats between 0 and 1.
These specify the area that the X/Y values are mapped to inside of the canvas

\item {} 
\textbf{\texttt{worldcoordinate}} (\emph{\texttt{list of floats}}) -- List of 4 floats (xmin, xmax, ymin, ymax)

\item {} 
\textbf{\texttt{x}} (\emph{\texttt{list of floats}}) -- List of lists of x coordinates. Values must be between worldcoordinate{[}0{]} and worldcoordinate{[}1{]}.

\item {} 
\textbf{\texttt{y}} (\emph{\texttt{list of floats}}) -- List of lists of y coordinates. Values must be between worldcoordinate{[}2{]} and worldcoordinate{[}3{]}.

\end{itemize}

\item[{Returns}] \leavevmode
A drawmarker object

\item[{Return type}] \leavevmode
{\hyperref[vcs/secondary/marker:vcs.marker.Tm]{\sphinxcrossref{vcs.marker.Tm}}}

\end{description}\end{quote}

\end{fulllineitems}

\index{drawtext() (vcs.Canvas.Canvas method)}

\begin{fulllineitems}
\phantomsection\label{vcs/Canvas:vcs.Canvas.Canvas.drawtext}\pysiglinewithargsret{\sphinxbfcode{drawtext}}{\emph{Tt\_name=None, To\_name=None, string=None, font=1, spacing=2, expansion=100, color=241, height=14, angle=0, path='right', halign='left', valign='half', priority=1, viewport={[}0.0, 1.0, 0.0, 1.0{]}, worldcoordinate={[}0.0, 1.0, 0.0, 1.0{]}, x=None, y=None, bg=0}}{}
Generate and draw a textcombined object on the VCS Canvas.
\begin{quote}\begin{description}
\item[{Example}] \leavevmode
\begin{Verbatim}[commandchars=\\\{\}]
\PYG{g+gp}{\PYGZgt{}\PYGZgt{}\PYGZgt{} }\PYG{n}{a}\PYG{o}{=}\PYG{n}{vcs}\PYG{o}{.}\PYG{n}{init}\PYG{p}{(}\PYG{p}{)}
\PYG{g+gp}{\PYGZgt{}\PYGZgt{}\PYGZgt{} }\PYG{n}{a}\PYG{o}{.}\PYG{n}{show}\PYG{p}{(}\PYG{l+s+s1}{\PYGZsq{}}\PYG{l+s+s1}{texttable}\PYG{l+s+s1}{\PYGZsq{}}\PYG{p}{)}  \PYG{c+c1}{\PYGZsh{} Show all the existing texttable objects}
\PYG{g+go}{*******************Texttable Names List**********************}
\PYG{g+gp}{...}
\PYG{g+go}{*******************End Texttable Names List**********************}
\PYG{g+gp}{\PYGZgt{}\PYGZgt{}\PYGZgt{} }\PYG{n}{tc}\PYG{o}{=}\PYG{n}{a}\PYG{o}{.}\PYG{n}{drawtextcombined}\PYG{p}{(}\PYG{n}{Tt\PYGZus{}name} \PYG{o}{=} \PYG{l+s+s1}{\PYGZsq{}}\PYG{l+s+s1}{std\PYGZus{}example}\PYG{l+s+s1}{\PYGZsq{}}\PYG{p}{,} \PYG{n}{To\PYGZus{}name}\PYG{o}{=}\PYG{l+s+s1}{\PYGZsq{}}\PYG{l+s+s1}{7left\PYGZus{}example}\PYG{l+s+s1}{\PYGZsq{}}\PYG{p}{,} \PYG{n}{string}\PYG{o}{=}\PYG{l+s+s1}{\PYGZsq{}}\PYG{l+s+s1}{Hello example!}\PYG{l+s+s1}{\PYGZsq{}}\PYG{p}{,} \PYG{n}{spacing}\PYG{o}{=}\PYG{l+m+mi}{5}\PYG{p}{,}
\PYG{g+gp}{... }                  \PYG{n}{color}\PYG{o}{=}\PYG{l+m+mi}{242}\PYG{p}{,} \PYG{n}{priority}\PYG{o}{=}\PYG{l+m+mi}{1}\PYG{p}{,} \PYG{n}{viewport}\PYG{o}{=}\PYG{p}{[}\PYG{l+m+mi}{0}\PYG{p}{,} \PYG{l+m+mf}{1.0}\PYG{p}{,} \PYG{l+m+mi}{0}\PYG{p}{,} \PYG{l+m+mf}{1.0}\PYG{p}{]}\PYG{p}{,}
\PYG{g+gp}{... }                  \PYG{n}{worldcoordinate}\PYG{o}{=}\PYG{p}{[}\PYG{l+m+mi}{0}\PYG{p}{,}\PYG{l+m+mi}{100}\PYG{p}{,} \PYG{l+m+mi}{0}\PYG{p}{,}\PYG{l+m+mi}{50}\PYG{p}{]}\PYG{p}{,}
\PYG{g+gp}{... }                  \PYG{n}{x}\PYG{o}{=}\PYG{p}{[}\PYG{l+m+mi}{0}\PYG{p}{,}\PYG{l+m+mi}{20}\PYG{p}{,}\PYG{l+m+mi}{40}\PYG{p}{,}\PYG{l+m+mi}{60}\PYG{p}{,}\PYG{l+m+mi}{80}\PYG{p}{,}\PYG{l+m+mi}{100}\PYG{p}{]}\PYG{p}{,}
\PYG{g+gp}{... }                  \PYG{n}{y}\PYG{o}{=}\PYG{p}{[}\PYG{l+m+mi}{0}\PYG{p}{,}\PYG{l+m+mi}{10}\PYG{p}{,}\PYG{l+m+mi}{20}\PYG{p}{,}\PYG{l+m+mi}{30}\PYG{p}{,}\PYG{l+m+mi}{40}\PYG{p}{,}\PYG{l+m+mi}{50}\PYG{p}{]}\PYG{p}{)} \PYG{c+c1}{\PYGZsh{} Create instance of texttable object \PYGZsq{}red\PYGZsq{}}
\PYG{g+gp}{\PYGZgt{}\PYGZgt{}\PYGZgt{} }\PYG{n}{a}\PYG{o}{.}\PYG{n}{textcombined}\PYG{p}{(}\PYG{n}{tc}\PYG{p}{)} \PYG{c+c1}{\PYGZsh{} Plot using specified texttable object}
\end{Verbatim}

\item[{Parameters}] \leavevmode\begin{itemize}
\item {} 
\textbf{\texttt{name}} (\href{https://docs.python.org/2/library/functions.html\#str}{\emph{\texttt{str}}}) -- Name of created object

\item {} 
\textbf{\texttt{style}} (\href{https://docs.python.org/2/library/functions.html\#str}{\emph{\texttt{str}}}) -- One of ``hatch'', ``solid'', or ``pattern''.

\item {} 
\textbf{\texttt{index}} (\href{https://docs.python.org/2/library/functions.html\#int}{\emph{\texttt{int}}}) -- 
Specifies which \href{http://uvcdat.llnl.gov/gallery/fullsize/pattern\_chart.png}{pattern}
to fill the fillarea with. Accepts ints from 1-20.


\item {} 
\textbf{\texttt{color}} (\emph{\texttt{str or int}}) -- 
A color name from the \href{https://en.wikipedia.org/wiki/X11\_color\_names}{X11 Color Names list},
or an integer value from 0-255, or an RGB/RGBA tuple/list (e.g. (0,100,0), (100,100,0,50))


\item {} 
\textbf{\texttt{priority}} (\href{https://docs.python.org/2/library/functions.html\#int}{\emph{\texttt{int}}}) -- The layer on which the fillarea will be drawn.

\item {} 
\textbf{\texttt{viewport}} (\emph{\texttt{list of floats}}) -- 4 floats between 0 and 1.
These specify the area that the X/Y values are mapped to inside of the canvas

\item {} 
\textbf{\texttt{worldcoordinate}} (\emph{\texttt{list of floats}}) -- List of 4 floats (xmin, xmax, ymin, ymax)

\item {} 
\textbf{\texttt{x}} (\emph{\texttt{list of floats}}) -- List of lists of x coordinates. Values must be between worldcoordinate{[}0{]} and worldcoordinate{[}1{]}.

\item {} 
\textbf{\texttt{y}} (\emph{\texttt{list of floats}}) -- List of lists of y coordinates. Values must be between worldcoordinate{[}2{]} and worldcoordinate{[}3{]}.

\item {} 
\textbf{\texttt{bg}} (\href{https://docs.python.org/2/library/functions.html\#bool}{\emph{\texttt{bool}}}) -- Boolean value. True =\textgreater{} object drawn in background (not shown on canvas).
False =\textgreater{} object shown on canvas.

\end{itemize}

\item[{Returns}] \leavevmode
A texttable object

\item[{Return type}] \leavevmode
{\hyperref[vcs/secondary/texttable:vcs.texttable.Tt]{\sphinxcrossref{vcs.texttable.Tt}}}

\end{description}\end{quote}

\end{fulllineitems}

\index{drawtextcombined() (vcs.Canvas.Canvas method)}

\begin{fulllineitems}
\phantomsection\label{vcs/Canvas:vcs.Canvas.Canvas.drawtextcombined}\pysiglinewithargsret{\sphinxbfcode{drawtextcombined}}{\emph{Tt\_name=None, To\_name=None, string=None, font=1, spacing=2, expansion=100, color=241, height=14, angle=0, path='right', halign='left', valign='half', priority=1, viewport={[}0.0, 1.0, 0.0, 1.0{]}, worldcoordinate={[}0.0, 1.0, 0.0, 1.0{]}, x=None, y=None, bg=0}}{}
Generate and draw a textcombined object on the VCS Canvas.
\begin{quote}\begin{description}
\item[{Example}] \leavevmode
\begin{Verbatim}[commandchars=\\\{\}]
\PYG{g+gp}{\PYGZgt{}\PYGZgt{}\PYGZgt{} }\PYG{n}{a}\PYG{o}{=}\PYG{n}{vcs}\PYG{o}{.}\PYG{n}{init}\PYG{p}{(}\PYG{p}{)}
\PYG{g+gp}{\PYGZgt{}\PYGZgt{}\PYGZgt{} }\PYG{n}{a}\PYG{o}{.}\PYG{n}{show}\PYG{p}{(}\PYG{l+s+s1}{\PYGZsq{}}\PYG{l+s+s1}{texttable}\PYG{l+s+s1}{\PYGZsq{}}\PYG{p}{)}  \PYG{c+c1}{\PYGZsh{} Show all the existing texttable objects}
\PYG{g+go}{*******************Texttable Names List**********************}
\PYG{g+gp}{...}
\PYG{g+go}{*******************End Texttable Names List**********************}
\PYG{g+gp}{\PYGZgt{}\PYGZgt{}\PYGZgt{} }\PYG{n}{tc}\PYG{o}{=}\PYG{n}{a}\PYG{o}{.}\PYG{n}{drawtextcombined}\PYG{p}{(}\PYG{n}{Tt\PYGZus{}name} \PYG{o}{=} \PYG{l+s+s1}{\PYGZsq{}}\PYG{l+s+s1}{std\PYGZus{}example}\PYG{l+s+s1}{\PYGZsq{}}\PYG{p}{,} \PYG{n}{To\PYGZus{}name}\PYG{o}{=}\PYG{l+s+s1}{\PYGZsq{}}\PYG{l+s+s1}{7left\PYGZus{}example}\PYG{l+s+s1}{\PYGZsq{}}\PYG{p}{,} \PYG{n}{string}\PYG{o}{=}\PYG{l+s+s1}{\PYGZsq{}}\PYG{l+s+s1}{Hello example!}\PYG{l+s+s1}{\PYGZsq{}}\PYG{p}{,} \PYG{n}{spacing}\PYG{o}{=}\PYG{l+m+mi}{5}\PYG{p}{,}
\PYG{g+gp}{... }                  \PYG{n}{color}\PYG{o}{=}\PYG{l+m+mi}{242}\PYG{p}{,} \PYG{n}{priority}\PYG{o}{=}\PYG{l+m+mi}{1}\PYG{p}{,} \PYG{n}{viewport}\PYG{o}{=}\PYG{p}{[}\PYG{l+m+mi}{0}\PYG{p}{,} \PYG{l+m+mf}{1.0}\PYG{p}{,} \PYG{l+m+mi}{0}\PYG{p}{,} \PYG{l+m+mf}{1.0}\PYG{p}{]}\PYG{p}{,}
\PYG{g+gp}{... }                  \PYG{n}{worldcoordinate}\PYG{o}{=}\PYG{p}{[}\PYG{l+m+mi}{0}\PYG{p}{,}\PYG{l+m+mi}{100}\PYG{p}{,} \PYG{l+m+mi}{0}\PYG{p}{,}\PYG{l+m+mi}{50}\PYG{p}{]}\PYG{p}{,}
\PYG{g+gp}{... }                  \PYG{n}{x}\PYG{o}{=}\PYG{p}{[}\PYG{l+m+mi}{0}\PYG{p}{,}\PYG{l+m+mi}{20}\PYG{p}{,}\PYG{l+m+mi}{40}\PYG{p}{,}\PYG{l+m+mi}{60}\PYG{p}{,}\PYG{l+m+mi}{80}\PYG{p}{,}\PYG{l+m+mi}{100}\PYG{p}{]}\PYG{p}{,}
\PYG{g+gp}{... }                  \PYG{n}{y}\PYG{o}{=}\PYG{p}{[}\PYG{l+m+mi}{0}\PYG{p}{,}\PYG{l+m+mi}{10}\PYG{p}{,}\PYG{l+m+mi}{20}\PYG{p}{,}\PYG{l+m+mi}{30}\PYG{p}{,}\PYG{l+m+mi}{40}\PYG{p}{,}\PYG{l+m+mi}{50}\PYG{p}{]}\PYG{p}{)} \PYG{c+c1}{\PYGZsh{} Create instance of texttable object \PYGZsq{}red\PYGZsq{}}
\PYG{g+gp}{\PYGZgt{}\PYGZgt{}\PYGZgt{} }\PYG{n}{a}\PYG{o}{.}\PYG{n}{textcombined}\PYG{p}{(}\PYG{n}{tc}\PYG{p}{)} \PYG{c+c1}{\PYGZsh{} Plot using specified texttable object}
\end{Verbatim}

\item[{Parameters}] \leavevmode\begin{itemize}
\item {} 
\textbf{\texttt{name}} (\href{https://docs.python.org/2/library/functions.html\#str}{\emph{\texttt{str}}}) -- Name of created object

\item {} 
\textbf{\texttt{style}} (\href{https://docs.python.org/2/library/functions.html\#str}{\emph{\texttt{str}}}) -- One of ``hatch'', ``solid'', or ``pattern''.

\item {} 
\textbf{\texttt{index}} (\href{https://docs.python.org/2/library/functions.html\#int}{\emph{\texttt{int}}}) -- 
Specifies which \href{http://uvcdat.llnl.gov/gallery/fullsize/pattern\_chart.png}{pattern}
to fill the fillarea with. Accepts ints from 1-20.


\item {} 
\textbf{\texttt{color}} (\emph{\texttt{str or int}}) -- 
A color name from the \href{https://en.wikipedia.org/wiki/X11\_color\_names}{X11 Color Names list},
or an integer value from 0-255, or an RGB/RGBA tuple/list (e.g. (0,100,0), (100,100,0,50))


\item {} 
\textbf{\texttt{priority}} (\href{https://docs.python.org/2/library/functions.html\#int}{\emph{\texttt{int}}}) -- The layer on which the fillarea will be drawn.

\item {} 
\textbf{\texttt{viewport}} (\emph{\texttt{list of floats}}) -- 4 floats between 0 and 1.
These specify the area that the X/Y values are mapped to inside of the canvas

\item {} 
\textbf{\texttt{worldcoordinate}} (\emph{\texttt{list of floats}}) -- List of 4 floats (xmin, xmax, ymin, ymax)

\item {} 
\textbf{\texttt{x}} (\emph{\texttt{list of floats}}) -- List of lists of x coordinates. Values must be between worldcoordinate{[}0{]} and worldcoordinate{[}1{]}.

\item {} 
\textbf{\texttt{y}} (\emph{\texttt{list of floats}}) -- List of lists of y coordinates. Values must be between worldcoordinate{[}2{]} and worldcoordinate{[}3{]}.

\item {} 
\textbf{\texttt{bg}} (\href{https://docs.python.org/2/library/functions.html\#bool}{\emph{\texttt{bool}}}) -- Boolean value. True =\textgreater{} object drawn in background (not shown on canvas).
False =\textgreater{} object shown on canvas.

\end{itemize}

\item[{Returns}] \leavevmode
A texttable object

\item[{Return type}] \leavevmode
{\hyperref[vcs/secondary/texttable:vcs.texttable.Tt]{\sphinxcrossref{vcs.texttable.Tt}}}

\end{description}\end{quote}

\end{fulllineitems}

\index{eps() (vcs.Canvas.Canvas method)}

\begin{fulllineitems}
\phantomsection\label{vcs/Canvas:vcs.Canvas.Canvas.eps}\pysiglinewithargsret{\sphinxbfcode{eps}}{\emph{file}, \emph{mode='r'}, \emph{orientation=None}, \emph{width=None}, \emph{height=None}, \emph{units='inches'}, \emph{textAsPaths=True}}{}
In some cases, the user may want to save the plot out as an Encapsulated
PostScript image. This routine allows the user to save the VCS canvas output
as an Encapsulated PostScript file.
This file can be converted to other image formats with the aid of xv and other
such imaging tools found freely on the web.
\begin{quote}\begin{description}
\item[{Example}] \leavevmode
\begin{Verbatim}[commandchars=\\\{\}]
\PYG{g+gp}{\PYGZgt{}\PYGZgt{}\PYGZgt{} }\PYG{n}{a}\PYG{o}{=}\PYG{n}{vcs}\PYG{o}{.}\PYG{n}{init}\PYG{p}{(}\PYG{p}{)}
\PYG{g+gp}{\PYGZgt{}\PYGZgt{}\PYGZgt{} }\PYG{n}{array} \PYG{o}{=} \PYG{p}{[}\PYG{n+nb}{range}\PYG{p}{(}\PYG{l+m+mi}{10}\PYG{p}{)} \PYG{k}{for} \PYG{n}{\PYGZus{}} \PYG{o+ow}{in} \PYG{n+nb}{range}\PYG{p}{(}\PYG{l+m+mi}{10}\PYG{p}{)}\PYG{p}{]}
\PYG{g+gp}{\PYGZgt{}\PYGZgt{}\PYGZgt{} }\PYG{n}{a}\PYG{o}{.}\PYG{n}{plot}\PYG{p}{(}\PYG{n}{array}\PYG{p}{)}
\PYG{g+go}{\PYGZlt{}vcs.displayplot.Dp ...\PYGZgt{}}
\PYG{g+gp}{\PYGZgt{}\PYGZgt{}\PYGZgt{} }\PYG{n}{a}\PYG{o}{.}\PYG{n}{postscript}\PYG{p}{(}\PYG{l+s+s1}{\PYGZsq{}}\PYG{l+s+s1}{example}\PYG{l+s+s1}{\PYGZsq{}}\PYG{p}{)} \PYG{c+c1}{\PYGZsh{} Overwrite a postscript file}
\PYG{g+gp}{\PYGZgt{}\PYGZgt{}\PYGZgt{} }\PYG{n}{a}\PYG{o}{.}\PYG{n}{postscript}\PYG{p}{(}\PYG{l+s+s1}{\PYGZsq{}}\PYG{l+s+s1}{example}\PYG{l+s+s1}{\PYGZsq{}}\PYG{p}{,} \PYG{l+s+s1}{\PYGZsq{}}\PYG{l+s+s1}{a}\PYG{l+s+s1}{\PYGZsq{}}\PYG{p}{)} \PYG{c+c1}{\PYGZsh{} Append postscript to an existing file}
\PYG{g+gp}{\PYGZgt{}\PYGZgt{}\PYGZgt{} }\PYG{n}{a}\PYG{o}{.}\PYG{n}{postscript}\PYG{p}{(}\PYG{l+s+s1}{\PYGZsq{}}\PYG{l+s+s1}{example}\PYG{l+s+s1}{\PYGZsq{}}\PYG{p}{,} \PYG{l+s+s1}{\PYGZsq{}}\PYG{l+s+s1}{r}\PYG{l+s+s1}{\PYGZsq{}}\PYG{p}{)} \PYG{c+c1}{\PYGZsh{} Overwrite an existing file}
\PYG{g+gp}{\PYGZgt{}\PYGZgt{}\PYGZgt{} }\PYG{n}{a}\PYG{o}{.}\PYG{n}{postscript}\PYG{p}{(}\PYG{l+s+s1}{\PYGZsq{}}\PYG{l+s+s1}{example}\PYG{l+s+s1}{\PYGZsq{}}\PYG{p}{,} \PYG{n}{mode}\PYG{o}{=}\PYG{l+s+s1}{\PYGZsq{}}\PYG{l+s+s1}{a}\PYG{l+s+s1}{\PYGZsq{}}\PYG{p}{)} \PYG{c+c1}{\PYGZsh{} Append postscript to an existing file}
\PYG{g+gp}{\PYGZgt{}\PYGZgt{}\PYGZgt{} }\PYG{n}{a}\PYG{o}{.}\PYG{n}{postscript}\PYG{p}{(}\PYG{l+s+s1}{\PYGZsq{}}\PYG{l+s+s1}{example}\PYG{l+s+s1}{\PYGZsq{}}\PYG{p}{,} \PYG{n}{width}\PYG{o}{=}\PYG{l+m+mf}{11.5}\PYG{p}{,} \PYG{n}{height}\PYG{o}{=} \PYG{l+m+mf}{8.5}\PYG{p}{)} \PYG{c+c1}{\PYGZsh{} US Legal (default)}
\PYG{g+gp}{\PYGZgt{}\PYGZgt{}\PYGZgt{} }\PYG{n}{a}\PYG{o}{.}\PYG{n}{postscript}\PYG{p}{(}\PYG{l+s+s1}{\PYGZsq{}}\PYG{l+s+s1}{example}\PYG{l+s+s1}{\PYGZsq{}}\PYG{p}{,} \PYG{n}{width}\PYG{o}{=}\PYG{l+m+mi}{21}\PYG{p}{,} \PYG{n}{height}\PYG{o}{=}\PYG{l+m+mf}{29.7}\PYG{p}{,} \PYG{n}{units}\PYG{o}{=}\PYG{l+s+s1}{\PYGZsq{}}\PYG{l+s+s1}{cm}\PYG{l+s+s1}{\PYGZsq{}}\PYG{p}{)} \PYG{c+c1}{\PYGZsh{} A4}
\end{Verbatim}

\item[{Parameters}] \leavevmode\begin{itemize}
\item {} 
\textbf{\texttt{file}} (\href{https://docs.python.org/2/library/functions.html\#str}{\emph{\texttt{str}}}) -- String name of the desired output file

\item {} 
\textbf{\texttt{mode}} (\href{https://docs.python.org/2/library/functions.html\#str}{\emph{\texttt{str}}}) -- The mode in which to open the file. One of `r' or `a'.

\item {} 
\textbf{\texttt{orientation}} (\href{https://docs.python.org/2/library/constants.html\#None}{\emph{\texttt{None}}}) -- Deprecated.

\item {} 
\textbf{\texttt{width}} (\href{https://docs.python.org/2/library/functions.html\#float}{\emph{\texttt{float}}}) -- Width of the output image, in the unit of measurement specified

\item {} 
\textbf{\texttt{height}} (\href{https://docs.python.org/2/library/functions.html\#float}{\emph{\texttt{float}}}) -- Height of the output image, in the unit of measurement specified

\item {} 
\textbf{\texttt{units}} (\href{https://docs.python.org/2/library/functions.html\#str}{\emph{\texttt{str}}}) -- One of {[}'inches', `in', `cm', `mm', `pixel', `pixels', `dot', `dots'{]}. Defaults to `inches'.

\end{itemize}

\end{description}\end{quote}

\end{fulllineitems}

\index{ffmpeg() (vcs.Canvas.Canvas method)}

\begin{fulllineitems}
\phantomsection\label{vcs/Canvas:vcs.Canvas.Canvas.ffmpeg}\pysiglinewithargsret{\sphinxbfcode{ffmpeg}}{\emph{movie}, \emph{files}, \emph{bitrate=1024}, \emph{rate=None}, \emph{options=None}}{}
MPEG output from a list of valid files.
Can output to more than just mpeg format.

\begin{notice}{note}{Note:}
ffmpeg ALWAYS overwrites the output file
\end{notice}

\begin{notice}{note}{Audio configuration}

via the options arg you can add audio file to your movie (see ffmpeg help)
\end{notice}
\begin{quote}\begin{description}
\item[{Example}] \leavevmode
\begin{Verbatim}[commandchars=\\\{\}]
\PYG{g+gp}{\PYGZgt{}\PYGZgt{}\PYGZgt{} }\PYG{n}{a}\PYG{o}{=}\PYG{n}{vcs}\PYG{o}{.}\PYG{n}{init}\PYG{p}{(}\PYG{p}{)}
\PYG{g+gp}{\PYGZgt{}\PYGZgt{}\PYGZgt{} }\PYG{k+kn}{import} \PYG{n+nn}{cdms2}
\PYG{g+gp}{\PYGZgt{}\PYGZgt{}\PYGZgt{} }\PYG{n}{f} \PYG{o}{=} \PYG{n}{cdms2}\PYG{o}{.}\PYG{n}{open}\PYG{p}{(}\PYG{n}{vcs}\PYG{o}{.}\PYG{n}{sample\PYGZus{}data}\PYG{o}{+}\PYG{l+s+s1}{\PYGZsq{}}\PYG{l+s+s1}{/clt.nc}\PYG{l+s+s1}{\PYGZsq{}}\PYG{p}{)}
\PYG{g+gp}{\PYGZgt{}\PYGZgt{}\PYGZgt{} }\PYG{n}{v} \PYG{o}{=} \PYG{n}{f}\PYG{p}{(}\PYG{l+s+s1}{\PYGZsq{}}\PYG{l+s+s1}{v}\PYG{l+s+s1}{\PYGZsq{}}\PYG{p}{)} \PYG{c+c1}{\PYGZsh{} use the data file to create a cdms2 slab}
\PYG{g+gp}{\PYGZgt{}\PYGZgt{}\PYGZgt{} }\PYG{n}{u} \PYG{o}{=} \PYG{n}{f}\PYG{p}{(}\PYG{l+s+s1}{\PYGZsq{}}\PYG{l+s+s1}{u}\PYG{l+s+s1}{\PYGZsq{}}\PYG{p}{)} \PYG{c+c1}{\PYGZsh{} use the data file to create a cdms2 slab}
\PYG{g+gp}{\PYGZgt{}\PYGZgt{}\PYGZgt{} }\PYG{n}{png\PYGZus{}files} \PYG{o}{=} \PYG{p}{[}\PYG{p}{]} \PYG{c+c1}{\PYGZsh{} for saving file names to make the mpeg}
\PYG{g+gp}{\PYGZgt{}\PYGZgt{}\PYGZgt{} }\PYG{n}{plots} \PYG{o}{=} \PYG{p}{[}\PYG{p}{]} \PYG{c+c1}{\PYGZsh{} for saving plots for later reference}
\PYG{g+gp}{\PYGZgt{}\PYGZgt{}\PYGZgt{} }\PYG{k}{for} \PYG{n}{i} \PYG{o+ow}{in} \PYG{n+nb}{range}\PYG{p}{(}\PYG{l+m+mi}{10}\PYG{p}{)}\PYG{p}{:} \PYG{c+c1}{\PYGZsh{} create a number of pngs to use for an mpeg}
\PYG{g+gp}{... }    \PYG{n}{a}\PYG{o}{.}\PYG{n}{clear}\PYG{p}{(}\PYG{p}{)}
\PYG{g+gp}{... }    \PYG{k}{if} \PYG{p}{(}\PYG{n}{i}\PYG{o}{\PYGZpc{}}\PYG{l+m+mi}{2}\PYG{p}{)}\PYG{p}{:}
\PYG{g+gp}{... }        \PYG{n}{plots}\PYG{o}{.}\PYG{n}{append}\PYG{p}{(}\PYG{n}{a}\PYG{o}{.}\PYG{n}{plot}\PYG{p}{(}\PYG{n}{u}\PYG{p}{,}\PYG{n}{v}\PYG{p}{)}\PYG{p}{)}
\PYG{g+gp}{... }    \PYG{k}{else}\PYG{p}{:}
\PYG{g+gp}{... }        \PYG{n}{plots}\PYG{o}{.}\PYG{n}{append}\PYG{p}{(}\PYG{n}{a}\PYG{o}{.}\PYG{n}{plot}\PYG{p}{(}\PYG{n}{v}\PYG{p}{,}\PYG{n}{u}\PYG{p}{)}\PYG{p}{)}
\PYG{g+gp}{... }    \PYG{n}{a}\PYG{o}{.}\PYG{n}{png}\PYG{p}{(}\PYG{l+s+s1}{\PYGZsq{}}\PYG{l+s+s1}{my\PYGZus{}png\PYGZus{}\PYGZus{}}\PYG{l+s+si}{\PYGZpc{}i}\PYG{l+s+s1}{\PYGZsq{}} \PYG{o}{\PYGZpc{}} \PYG{n}{i}\PYG{p}{)}
\PYG{g+gp}{... }    \PYG{n}{png\PYGZus{}files}\PYG{o}{.}\PYG{n}{append}\PYG{p}{(}\PYG{l+s+s1}{\PYGZsq{}}\PYG{l+s+s1}{my\PYGZus{}png\PYGZus{}\PYGZus{}}\PYG{l+s+si}{\PYGZpc{}i}\PYG{l+s+s1}{.png}\PYG{l+s+s1}{\PYGZsq{}} \PYG{o}{\PYGZpc{}} \PYG{n}{i}\PYG{p}{)}
\PYG{g+gp}{\PYGZgt{}\PYGZgt{}\PYGZgt{} }\PYG{n}{a}\PYG{o}{.}\PYG{n}{ffmpeg}\PYG{p}{(}\PYG{l+s+s1}{\PYGZsq{}}\PYG{l+s+s1}{mymovie.mpeg}\PYG{l+s+s1}{\PYGZsq{}}\PYG{p}{,}\PYG{n}{png\PYGZus{}files}\PYG{p}{)} \PYG{c+c1}{\PYGZsh{} generates from list of files}
\PYG{g+go}{True}
\PYG{g+gp}{\PYGZgt{}\PYGZgt{}\PYGZgt{} }\PYG{n}{a}\PYG{o}{.}\PYG{n}{ffmpeg}\PYG{p}{(}\PYG{l+s+s1}{\PYGZsq{}}\PYG{l+s+s1}{mymovie.mpeg}\PYG{l+s+s1}{\PYGZsq{}}\PYG{p}{,}\PYG{n}{files}\PYG{o}{=}\PYG{l+s+s2}{\PYGZdq{}}\PYG{l+s+s2}{my\PYGZus{}png\PYGZus{}\PYGZus{}[0\PYGZhy{}9]*}\PYG{l+s+s2}{\PYGZbs{}}\PYG{l+s+s2}{.png}\PYG{l+s+s2}{\PYGZdq{}}\PYG{p}{)} \PYG{c+c1}{\PYGZsh{} generate from files with name matching regex}
\PYG{g+go}{True}
\PYG{g+gp}{\PYGZgt{}\PYGZgt{}\PYGZgt{} }\PYG{n}{a}\PYG{o}{.}\PYG{n}{ffmpeg}\PYG{p}{(}\PYG{l+s+s1}{\PYGZsq{}}\PYG{l+s+s1}{mymovie.mpeg}\PYG{l+s+s1}{\PYGZsq{}}\PYG{p}{,}\PYG{n}{png\PYGZus{}files}\PYG{p}{,}\PYG{n}{bitrate}\PYG{o}{=}\PYG{l+m+mi}{512}\PYG{p}{)} \PYG{c+c1}{\PYGZsh{} generates mpeg at 512kbit}
\PYG{g+go}{True}
\PYG{g+gp}{\PYGZgt{}\PYGZgt{}\PYGZgt{} }\PYG{n}{a}\PYG{o}{.}\PYG{n}{ffmpeg}\PYG{p}{(}\PYG{l+s+s1}{\PYGZsq{}}\PYG{l+s+s1}{mymovie.mpeg}\PYG{l+s+s1}{\PYGZsq{}}\PYG{p}{,}\PYG{n}{png\PYGZus{}files}\PYG{p}{,}\PYG{n}{rate}\PYG{o}{=}\PYG{l+m+mi}{50}\PYG{p}{)} \PYG{c+c1}{\PYGZsh{} generates movie with 50 frame per second}
\PYG{g+go}{True}
\end{Verbatim}

\item[{Parameters}] \leavevmode\begin{itemize}
\item {} 
\textbf{\texttt{movie}} (\href{https://docs.python.org/2/library/functions.html\#str}{\emph{\texttt{str}}}) -- Output video file name

\item {} 
\textbf{\texttt{files}} (\emph{\texttt{str, list, or tuple}}) -- String file name

\item {} 
\textbf{\texttt{rate}} (\href{https://docs.python.org/2/library/functions.html\#str}{\emph{\texttt{str}}}) -- Desired output framerate

\item {} 
\textbf{\texttt{options}} (\href{https://docs.python.org/2/library/functions.html\#str}{\emph{\texttt{str}}}) -- Additional FFMPEG arguments

\end{itemize}

\item[{Returns}] \leavevmode
The output string generated by ffmpeg program

\item[{Return type}] \leavevmode
\href{https://docs.python.org/2/library/functions.html\#str}{str}

\end{description}\end{quote}

\end{fulllineitems}

\index{fillarea() (vcs.Canvas.Canvas method)}

\begin{fulllineitems}
\phantomsection\label{vcs/Canvas:vcs.Canvas.Canvas.fillarea}\pysiglinewithargsret{\sphinxbfcode{fillarea}}{\emph{*args}, \emph{**parms}}{}
Generate a fillarea plot

Plot a fillarea segment on the Vcs Canvas. If no fillarea class
object is given, then an error will be returned.
\begin{quote}\begin{description}
\item[{Example}] \leavevmode
\begin{Verbatim}[commandchars=\\\{\}]
\PYG{g+gp}{\PYGZgt{}\PYGZgt{}\PYGZgt{} }\PYG{n}{a}\PYG{o}{=}\PYG{n}{vcs}\PYG{o}{.}\PYG{n}{init}\PYG{p}{(}\PYG{p}{)}
\PYG{g+gp}{\PYGZgt{}\PYGZgt{}\PYGZgt{} }\PYG{n}{a}\PYG{o}{.}\PYG{n}{show}\PYG{p}{(}\PYG{l+s+s1}{\PYGZsq{}}\PYG{l+s+s1}{fillarea}\PYG{l+s+s1}{\PYGZsq{}}\PYG{p}{)} \PYG{c+c1}{\PYGZsh{} Show all the existing fillarea objects}
\PYG{g+go}{*******************Fillarea Names List**********************}
\PYG{g+gp}{...}
\PYG{g+go}{*******************End Fillarea Names List**********************}
\PYG{g+gp}{\PYGZgt{}\PYGZgt{}\PYGZgt{} }\PYG{n}{fa}\PYG{o}{=}\PYG{n}{a}\PYG{o}{.}\PYG{n}{createfillarea}\PYG{p}{(}\PYG{p}{)} \PYG{c+c1}{\PYGZsh{} Create instance of default fillarea}
\PYG{g+gp}{\PYGZgt{}\PYGZgt{}\PYGZgt{} }\PYG{n}{fa}\PYG{o}{.}\PYG{n}{style}\PYG{o}{=}\PYG{l+m+mi}{1} \PYG{c+c1}{\PYGZsh{} Set the fillarea style}
\PYG{g+gp}{\PYGZgt{}\PYGZgt{}\PYGZgt{} }\PYG{n}{fa}\PYG{o}{.}\PYG{n}{index}\PYG{o}{=}\PYG{l+m+mi}{4} \PYG{c+c1}{\PYGZsh{} Set the fillarea index}
\PYG{g+gp}{\PYGZgt{}\PYGZgt{}\PYGZgt{} }\PYG{n}{fa}\PYG{o}{.}\PYG{n}{color} \PYG{o}{=} \PYG{l+m+mi}{242} \PYG{c+c1}{\PYGZsh{} Set the fillarea color}
\PYG{g+gp}{\PYGZgt{}\PYGZgt{}\PYGZgt{} }\PYG{n}{fa}\PYG{o}{.}\PYG{n}{x}\PYG{o}{=}\PYG{p}{[}\PYG{l+m+mf}{0.25}\PYG{p}{,}\PYG{l+m+mf}{0.75}\PYG{p}{]} \PYG{c+c1}{\PYGZsh{} Set the x value points}
\PYG{g+gp}{\PYGZgt{}\PYGZgt{}\PYGZgt{} }\PYG{n}{fa}\PYG{o}{.}\PYG{n}{y}\PYG{o}{=}\PYG{p}{[}\PYG{l+m+mf}{0.2}\PYG{p}{,}\PYG{l+m+mf}{0.5}\PYG{p}{]} \PYG{c+c1}{\PYGZsh{} Set the y value points}
\PYG{g+gp}{\PYGZgt{}\PYGZgt{}\PYGZgt{} }\PYG{n}{a}\PYG{o}{.}\PYG{n}{fillarea}\PYG{p}{(}\PYG{n}{fa}\PYG{p}{)} \PYG{c+c1}{\PYGZsh{} Plot using specified fillarea object}
\PYG{g+go}{\PYGZlt{}vcs.displayplot.Dp ...\PYGZgt{}}
\end{Verbatim}

\item[{Returns}] \leavevmode
A fillarea object

\item[{Return type}] \leavevmode
{\hyperref[vcs/misc/displayplot:vcs.displayplot.Dp]{\sphinxcrossref{vcs.displayplot.Dp}}}

\end{description}\end{quote}

\end{fulllineitems}

\index{flush() (vcs.Canvas.Canvas method)}

\begin{fulllineitems}
\phantomsection\label{vcs/Canvas:vcs.Canvas.Canvas.flush}\pysiglinewithargsret{\sphinxbfcode{flush}}{\emph{*args}}{}
The flush command executes all buffered X events in the queue.
\begin{quote}\begin{description}
\item[{Example}] \leavevmode
\begin{Verbatim}[commandchars=\\\{\}]
\PYG{g+gp}{\PYGZgt{}\PYGZgt{}\PYGZgt{} }\PYG{n}{a}\PYG{o}{=}\PYG{n}{vcs}\PYG{o}{.}\PYG{n}{init}\PYG{p}{(}\PYG{p}{)}
\PYG{g+gp}{\PYGZgt{}\PYGZgt{}\PYGZgt{} }\PYG{n}{array} \PYG{o}{=} \PYG{p}{[}\PYG{n+nb}{range}\PYG{p}{(}\PYG{l+m+mi}{1}\PYG{p}{,} \PYG{l+m+mi}{11}\PYG{p}{)} \PYG{k}{for} \PYG{n}{\PYGZus{}} \PYG{o+ow}{in} \PYG{n+nb}{range}\PYG{p}{(}\PYG{l+m+mi}{1}\PYG{p}{,} \PYG{l+m+mi}{11}\PYG{p}{)}\PYG{p}{]}
\PYG{g+gp}{\PYGZgt{}\PYGZgt{}\PYGZgt{} }\PYG{n}{a}\PYG{o}{.}\PYG{n}{plot}\PYG{p}{(}\PYG{n}{array}\PYG{p}{,}\PYG{l+s+s1}{\PYGZsq{}}\PYG{l+s+s1}{default}\PYG{l+s+s1}{\PYGZsq{}}\PYG{p}{,}\PYG{l+s+s1}{\PYGZsq{}}\PYG{l+s+s1}{isofill}\PYG{l+s+s1}{\PYGZsq{}}\PYG{p}{,}\PYG{l+s+s1}{\PYGZsq{}}\PYG{l+s+s1}{quick}\PYG{l+s+s1}{\PYGZsq{}}\PYG{p}{)}
\PYG{g+go}{\PYGZlt{}vcs.displayplot.Dp ...\PYGZgt{}}
\PYG{g+gp}{\PYGZgt{}\PYGZgt{}\PYGZgt{} }\PYG{n}{a}\PYG{o}{.}\PYG{n}{flush}\PYG{p}{(}\PYG{p}{)}
\end{Verbatim}

\end{description}\end{quote}

\end{fulllineitems}

\index{geometry() (vcs.Canvas.Canvas method)}

\begin{fulllineitems}
\phantomsection\label{vcs/Canvas:vcs.Canvas.Canvas.geometry}\pysiglinewithargsret{\sphinxbfcode{geometry}}{\emph{*args}}{}
The geometry command is used to set the size and position of the VCS canvas.
\begin{quote}\begin{description}
\item[{Example}] \leavevmode
\begin{Verbatim}[commandchars=\\\{\}]
\PYG{g+gp}{\PYGZgt{}\PYGZgt{}\PYGZgt{} }\PYG{n}{a}\PYG{o}{=}\PYG{n}{vcs}\PYG{o}{.}\PYG{n}{init}\PYG{p}{(}\PYG{p}{)}
\PYG{g+gp}{\PYGZgt{}\PYGZgt{}\PYGZgt{} }\PYG{n}{array} \PYG{o}{=} \PYG{p}{[}\PYG{n+nb}{range}\PYG{p}{(}\PYG{l+m+mi}{1}\PYG{p}{,} \PYG{l+m+mi}{11}\PYG{p}{)} \PYG{k}{for} \PYG{n}{\PYGZus{}} \PYG{o+ow}{in} \PYG{n+nb}{range}\PYG{p}{(}\PYG{l+m+mi}{1}\PYG{p}{,} \PYG{l+m+mi}{11}\PYG{p}{)}\PYG{p}{]}
\PYG{g+gp}{\PYGZgt{}\PYGZgt{}\PYGZgt{} }\PYG{n}{a}\PYG{o}{.}\PYG{n}{plot}\PYG{p}{(}\PYG{n}{array}\PYG{p}{,}\PYG{l+s+s1}{\PYGZsq{}}\PYG{l+s+s1}{default}\PYG{l+s+s1}{\PYGZsq{}}\PYG{p}{,}\PYG{l+s+s1}{\PYGZsq{}}\PYG{l+s+s1}{isofill}\PYG{l+s+s1}{\PYGZsq{}}\PYG{p}{,}\PYG{l+s+s1}{\PYGZsq{}}\PYG{l+s+s1}{quick}\PYG{l+s+s1}{\PYGZsq{}}\PYG{p}{)}
\PYG{g+go}{\PYGZlt{}vcs.displayplot.Dp ...\PYGZgt{}}
\PYG{g+gp}{\PYGZgt{}\PYGZgt{}\PYGZgt{} }\PYG{n}{a}\PYG{o}{.}\PYG{n}{geometry}\PYG{p}{(}\PYG{l+m+mi}{450}\PYG{p}{,}\PYG{l+m+mi}{337}\PYG{p}{)}
\end{Verbatim}

\end{description}\end{quote}

\end{fulllineitems}

\index{get3d\_dual\_scalar() (vcs.Canvas.Canvas method)}

\begin{fulllineitems}
\phantomsection\label{vcs/Canvas:vcs.Canvas.Canvas.get3d_dual_scalar}\pysiglinewithargsret{\sphinxbfcode{get3d\_dual\_scalar}}{\emph{Gfdv3d\_name\_src='default'}}{}
VCS contains a list of graphics methods. This function will create a
dv3d class object from an existing VCS dv3d graphics method. If
no dv3d name is given, then dv3d `default' will be used.

\begin{notice}{note}{Note:}
VCS does not allow the modification of `default' attribute sets.
However, a `default' attribute set that has been copied under a
different name can be modified. (See the {\hyperref[vcs/misc/manageElements:vcs.manageElements.create3d_dual_scalar]{\sphinxcrossref{\sphinxcode{vcs.manageElements.create3d\_dual\_scalar()}}}} function.)
\end{notice}
\begin{quote}\begin{description}
\item[{Example}] \leavevmode
\begin{Verbatim}[commandchars=\\\{\}]
\PYG{g+gp}{\PYGZgt{}\PYGZgt{}\PYGZgt{} }\PYG{n}{a}\PYG{o}{=}\PYG{n}{vcs}\PYG{o}{.}\PYG{n}{init}\PYG{p}{(}\PYG{p}{)}
\PYG{g+gp}{\PYGZgt{}\PYGZgt{}\PYGZgt{} }\PYG{n}{vcs}\PYG{o}{.}\PYG{n}{listelements}\PYG{p}{(}\PYG{l+s+s1}{\PYGZsq{}}\PYG{l+s+s1}{3d\PYGZus{}dual\PYGZus{}scalar}\PYG{l+s+s1}{\PYGZsq{}}\PYG{p}{)} \PYG{c+c1}{\PYGZsh{} Show all the existing 3d\PYGZus{}dual\PYGZus{}scalar graphics methods}
\PYG{g+go}{[...]}
\PYG{g+gp}{\PYGZgt{}\PYGZgt{}\PYGZgt{} }\PYG{n}{ex}\PYG{o}{=}\PYG{n}{vcs}\PYG{o}{.}\PYG{n}{get3d\PYGZus{}dual\PYGZus{}scalar}\PYG{p}{(}\PYG{p}{)}  \PYG{c+c1}{\PYGZsh{} instance of \PYGZsq{}default\PYGZsq{} 3d\PYGZus{}dual\PYGZus{}scalar graphics method}
\PYG{g+gp}{\PYGZgt{}\PYGZgt{}\PYGZgt{} }\PYG{k+kn}{import} \PYG{n+nn}{cdms2} \PYG{c+c1}{\PYGZsh{} Need cdms2 to create a slab}
\PYG{g+gp}{\PYGZgt{}\PYGZgt{}\PYGZgt{} }\PYG{n}{f} \PYG{o}{=} \PYG{n}{cdms2}\PYG{o}{.}\PYG{n}{open}\PYG{p}{(}\PYG{n}{vcs}\PYG{o}{.}\PYG{n}{sample\PYGZus{}data}\PYG{o}{+}\PYG{l+s+s1}{\PYGZsq{}}\PYG{l+s+s1}{/clt.nc}\PYG{l+s+s1}{\PYGZsq{}}\PYG{p}{)} \PYG{c+c1}{\PYGZsh{} use cdms2 to open a data file}
\PYG{g+gp}{\PYGZgt{}\PYGZgt{}\PYGZgt{} }\PYG{n}{slab1} \PYG{o}{=} \PYG{n}{f}\PYG{p}{(}\PYG{l+s+s1}{\PYGZsq{}}\PYG{l+s+s1}{u}\PYG{l+s+s1}{\PYGZsq{}}\PYG{p}{)} \PYG{c+c1}{\PYGZsh{} use the data file to create a cdms2 slab}
\PYG{g+gp}{\PYGZgt{}\PYGZgt{}\PYGZgt{} }\PYG{n}{slab2} \PYG{o}{=} \PYG{n}{f}\PYG{p}{(}\PYG{l+s+s1}{\PYGZsq{}}\PYG{l+s+s1}{v}\PYG{l+s+s1}{\PYGZsq{}}\PYG{p}{)} \PYG{c+c1}{\PYGZsh{} need 2 slabs, so get another}
\PYG{g+gp}{\PYGZgt{}\PYGZgt{}\PYGZgt{} }\PYG{n}{a}\PYG{o}{.}\PYG{n}{plot}\PYG{p}{(}\PYG{n}{ex}\PYG{p}{,} \PYG{n}{slab1}\PYG{p}{,} \PYG{n}{slab2}\PYG{p}{)} \PYG{c+c1}{\PYGZsh{} plot using specified 3d\PYGZus{}dual\PYGZus{}scalar object}
\PYG{g+go}{\PYGZlt{}vcs.displayplot.Dp ...\PYGZgt{}}
\end{Verbatim}

\item[{Parameters}] \leavevmode
\textbf{\texttt{Gfdv3d\_name\_src}} (\href{https://docs.python.org/2/library/functions.html\#str}{\emph{\texttt{str}}}) -- String name of an existing 3d\_dual\_scalar VCS object

\item[{Returns}] \leavevmode
A pre-existing 3d\_dual\_scalar VCS object

\item[{Return type}] \leavevmode
vcs.dv3d.Gf3DDualScalar

\end{description}\end{quote}

\end{fulllineitems}

\index{get3d\_scalar() (vcs.Canvas.Canvas method)}

\begin{fulllineitems}
\phantomsection\label{vcs/Canvas:vcs.Canvas.Canvas.get3d_scalar}\pysiglinewithargsret{\sphinxbfcode{get3d\_scalar}}{\emph{Gfdv3d\_name\_src='default'}}{}
VCS contains a list of graphics methods. This function will create a
dv3d class object from an existing VCS dv3d graphics method. If
no dv3d name is given, then dv3d `default' will be used.

\begin{notice}{note}{Note:}
VCS does not allow the modification of `default' attribute sets.
However, a `default' attribute set that has been copied under a
different name can be modified. (See the {\hyperref[vcs/misc/manageElements:vcs.manageElements.create3d_scalar]{\sphinxcrossref{\sphinxcode{vcs.manageElements.create3d\_scalar()}}}} function.)
\end{notice}
\begin{quote}\begin{description}
\item[{Example}] \leavevmode
\begin{Verbatim}[commandchars=\\\{\}]
\PYG{g+gp}{\PYGZgt{}\PYGZgt{}\PYGZgt{} }\PYG{n}{a}\PYG{o}{=}\PYG{n}{vcs}\PYG{o}{.}\PYG{n}{init}\PYG{p}{(}\PYG{p}{)}
\PYG{g+gp}{\PYGZgt{}\PYGZgt{}\PYGZgt{} }\PYG{n}{vcs}\PYG{o}{.}\PYG{n}{listelements}\PYG{p}{(}\PYG{l+s+s1}{\PYGZsq{}}\PYG{l+s+s1}{3d\PYGZus{}scalar}\PYG{l+s+s1}{\PYGZsq{}}\PYG{p}{)} \PYG{c+c1}{\PYGZsh{} Show all the existing 3d\PYGZus{}scalar graphics methods}
\PYG{g+go}{[...]}
\PYG{g+gp}{\PYGZgt{}\PYGZgt{}\PYGZgt{} }\PYG{n}{ex}\PYG{o}{=}\PYG{n}{vcs}\PYG{o}{.}\PYG{n}{get3d\PYGZus{}scalar}\PYG{p}{(}\PYG{p}{)}  \PYG{c+c1}{\PYGZsh{} instance of \PYGZsq{}default\PYGZsq{} 3d\PYGZus{}scalar graphics method}
\PYG{g+gp}{\PYGZgt{}\PYGZgt{}\PYGZgt{} }\PYG{k+kn}{import} \PYG{n+nn}{cdms2} \PYG{c+c1}{\PYGZsh{} Need cdms2 to create a slab}
\PYG{g+gp}{\PYGZgt{}\PYGZgt{}\PYGZgt{} }\PYG{n}{f} \PYG{o}{=} \PYG{n}{cdms2}\PYG{o}{.}\PYG{n}{open}\PYG{p}{(}\PYG{n}{vcs}\PYG{o}{.}\PYG{n}{sample\PYGZus{}data}\PYG{o}{+}\PYG{l+s+s1}{\PYGZsq{}}\PYG{l+s+s1}{/clt.nc}\PYG{l+s+s1}{\PYGZsq{}}\PYG{p}{)} \PYG{c+c1}{\PYGZsh{} use cdms2 to open a data file}
\PYG{g+gp}{\PYGZgt{}\PYGZgt{}\PYGZgt{} }\PYG{n}{slab1} \PYG{o}{=} \PYG{n}{f}\PYG{p}{(}\PYG{l+s+s1}{\PYGZsq{}}\PYG{l+s+s1}{u}\PYG{l+s+s1}{\PYGZsq{}}\PYG{p}{)} \PYG{c+c1}{\PYGZsh{} use the data file to create a cdms2 slab}
\PYG{g+gp}{\PYGZgt{}\PYGZgt{}\PYGZgt{} }\PYG{n}{a}\PYG{o}{.}\PYG{n}{plot}\PYG{p}{(}\PYG{n}{ex}\PYG{p}{,} \PYG{n}{slab1}\PYG{p}{)} \PYG{c+c1}{\PYGZsh{} plot using specified 3d\PYGZus{}scalar object}
\PYG{g+go}{\PYGZlt{}vcs.displayplot.Dp ...\PYGZgt{}}
\end{Verbatim}

\item[{Parameters}] \leavevmode
\textbf{\texttt{Gfdv3d\_name\_src}} (\href{https://docs.python.org/2/library/functions.html\#str}{\emph{\texttt{str}}}) -- String name of an existing 3d\_scalar VCS object.

\item[{Returns}] \leavevmode
A pre-existing 3d\_scalar VCS object

\item[{Return type}] \leavevmode
vcs.dv3d.Gf3Dscalar

\end{description}\end{quote}

\end{fulllineitems}

\index{get3d\_vector() (vcs.Canvas.Canvas method)}

\begin{fulllineitems}
\phantomsection\label{vcs/Canvas:vcs.Canvas.Canvas.get3d_vector}\pysiglinewithargsret{\sphinxbfcode{get3d\_vector}}{\emph{Gfdv3d\_name\_src='default'}}{}
VCS contains a list of graphics methods. This function will create a
dv3d class object from an existing VCS dv3d graphics method. If
no dv3d name is given, then dv3d `default' will be used.

\begin{notice}{note}{Note:}
VCS does not allow the modification of `default' attribute sets.
However, a `default' attribute set that has been copied under a
different name can be modified. (See the {\hyperref[vcs/misc/manageElements:vcs.manageElements.create3d_vector]{\sphinxcrossref{\sphinxcode{vcs.manageElements.create3d\_vector()}}}} function.)
\end{notice}
\begin{quote}\begin{description}
\item[{Example}] \leavevmode
\begin{Verbatim}[commandchars=\\\{\}]
\PYG{g+gp}{\PYGZgt{}\PYGZgt{}\PYGZgt{} }\PYG{n}{a}\PYG{o}{=}\PYG{n}{vcs}\PYG{o}{.}\PYG{n}{init}\PYG{p}{(}\PYG{p}{)}
\PYG{g+gp}{\PYGZgt{}\PYGZgt{}\PYGZgt{} }\PYG{n}{vcs}\PYG{o}{.}\PYG{n}{listelements}\PYG{p}{(}\PYG{l+s+s1}{\PYGZsq{}}\PYG{l+s+s1}{3d\PYGZus{}vector}\PYG{l+s+s1}{\PYGZsq{}}\PYG{p}{)} \PYG{c+c1}{\PYGZsh{} Show all the existing 3d\PYGZus{}vector graphics methods}
\PYG{g+go}{[...]}
\PYG{g+gp}{\PYGZgt{}\PYGZgt{}\PYGZgt{} }\PYG{n}{ex}\PYG{o}{=}\PYG{n}{vcs}\PYG{o}{.}\PYG{n}{get3d\PYGZus{}vector}\PYG{p}{(}\PYG{p}{)}  \PYG{c+c1}{\PYGZsh{} instance of \PYGZsq{}default\PYGZsq{} 3d\PYGZus{}vector graphics method}
\PYG{g+gp}{\PYGZgt{}\PYGZgt{}\PYGZgt{} }\PYG{k+kn}{import} \PYG{n+nn}{cdms2} \PYG{c+c1}{\PYGZsh{} Need cdms2 to create a slab}
\PYG{g+gp}{\PYGZgt{}\PYGZgt{}\PYGZgt{} }\PYG{n}{f} \PYG{o}{=} \PYG{n}{cdms2}\PYG{o}{.}\PYG{n}{open}\PYG{p}{(}\PYG{n}{vcs}\PYG{o}{.}\PYG{n}{sample\PYGZus{}data}\PYG{o}{+}\PYG{l+s+s1}{\PYGZsq{}}\PYG{l+s+s1}{/clt.nc}\PYG{l+s+s1}{\PYGZsq{}}\PYG{p}{)} \PYG{c+c1}{\PYGZsh{} use cdms2 to open a data file}
\PYG{g+gp}{\PYGZgt{}\PYGZgt{}\PYGZgt{} }\PYG{n}{slab1} \PYG{o}{=} \PYG{n}{f}\PYG{p}{(}\PYG{l+s+s1}{\PYGZsq{}}\PYG{l+s+s1}{u}\PYG{l+s+s1}{\PYGZsq{}}\PYG{p}{)} \PYG{c+c1}{\PYGZsh{} use the data file to create a cdms2 slab}
\PYG{g+gp}{\PYGZgt{}\PYGZgt{}\PYGZgt{} }\PYG{n}{slab2} \PYG{o}{=} \PYG{n}{f}\PYG{p}{(}\PYG{l+s+s1}{\PYGZsq{}}\PYG{l+s+s1}{v}\PYG{l+s+s1}{\PYGZsq{}}\PYG{p}{)} \PYG{c+c1}{\PYGZsh{} need 2 slabs, so get another}
\PYG{g+gp}{\PYGZgt{}\PYGZgt{}\PYGZgt{} }\PYG{n}{a}\PYG{o}{.}\PYG{n}{plot}\PYG{p}{(}\PYG{n}{ex}\PYG{p}{,} \PYG{n}{slab1}\PYG{p}{,} \PYG{n}{slab2}\PYG{p}{)} \PYG{c+c1}{\PYGZsh{} plot using specified 3d\PYGZus{}vector object}
\PYG{g+go}{\PYGZlt{}vcs.displayplot.Dp ...\PYGZgt{}}
\end{Verbatim}

\item[{Parameters}] \leavevmode
\textbf{\texttt{Gfdv3d\_name\_src}} (\href{https://docs.python.org/2/library/functions.html\#str}{\emph{\texttt{str}}}) -- String name of an existing 3d\_vector VCS object

\item[{Returns}] \leavevmode
A pre-existing 3d\_vector VCS object

\item[{Return type}] \leavevmode
vcs.dv3d.Gf3Dvector

\end{description}\end{quote}

\end{fulllineitems}

\index{get\_selected\_display() (vcs.Canvas.Canvas method)}

\begin{fulllineitems}
\phantomsection\label{vcs/Canvas:vcs.Canvas.Canvas.get_selected_display}\pysiglinewithargsret{\sphinxbfcode{get\_selected\_display}}{}{}~
\DUrole{versionmodified}{Deprecated since version 2.0: }This function is no longer supported.

\end{fulllineitems}

\index{getboxfill() (vcs.Canvas.Canvas method)}

\begin{fulllineitems}
\phantomsection\label{vcs/Canvas:vcs.Canvas.Canvas.getboxfill}\pysiglinewithargsret{\sphinxbfcode{getboxfill}}{\emph{Gfb\_name\_src='default'}}{}
VCS contains a list of graphics methods. This function will create a
boxfill class object from an existing VCS boxfill graphics method. If
no boxfill name is given, then boxfill `default' will be used.

\begin{notice}{note}{Note:}
VCS does not allow the modification of `default' attribute sets.
However, a `default' attribute set that has been copied under a
different name can be modified. (See the {\hyperref[vcs/misc/manageElements:vcs.manageElements.createboxfill]{\sphinxcrossref{\sphinxcode{vcs.manageElements.createboxfill()}}}} function.)
\end{notice}
\begin{quote}\begin{description}
\item[{Example}] \leavevmode
\begin{Verbatim}[commandchars=\\\{\}]
\PYG{g+gp}{\PYGZgt{}\PYGZgt{}\PYGZgt{} }\PYG{n}{a}\PYG{o}{=}\PYG{n}{vcs}\PYG{o}{.}\PYG{n}{init}\PYG{p}{(}\PYG{p}{)}
\PYG{g+gp}{\PYGZgt{}\PYGZgt{}\PYGZgt{} }\PYG{n}{vcs}\PYG{o}{.}\PYG{n}{listelements}\PYG{p}{(}\PYG{l+s+s1}{\PYGZsq{}}\PYG{l+s+s1}{boxfill}\PYG{l+s+s1}{\PYGZsq{}}\PYG{p}{)} \PYG{c+c1}{\PYGZsh{} Show all the existing boxfill graphics methods}
\PYG{g+go}{[...]}
\PYG{g+gp}{\PYGZgt{}\PYGZgt{}\PYGZgt{} }\PYG{n}{ex}\PYG{o}{=}\PYG{n}{vcs}\PYG{o}{.}\PYG{n}{getboxfill}\PYG{p}{(}\PYG{p}{)}  \PYG{c+c1}{\PYGZsh{} instance of \PYGZsq{}default\PYGZsq{} boxfill graphics method}
\PYG{g+gp}{\PYGZgt{}\PYGZgt{}\PYGZgt{} }\PYG{k+kn}{import} \PYG{n+nn}{cdms2} \PYG{c+c1}{\PYGZsh{} Need cdms2 to create a slab}
\PYG{g+gp}{\PYGZgt{}\PYGZgt{}\PYGZgt{} }\PYG{n}{f} \PYG{o}{=} \PYG{n}{cdms2}\PYG{o}{.}\PYG{n}{open}\PYG{p}{(}\PYG{n}{vcs}\PYG{o}{.}\PYG{n}{sample\PYGZus{}data}\PYG{o}{+}\PYG{l+s+s1}{\PYGZsq{}}\PYG{l+s+s1}{/clt.nc}\PYG{l+s+s1}{\PYGZsq{}}\PYG{p}{)} \PYG{c+c1}{\PYGZsh{} use cdms2 to open a data file}
\PYG{g+gp}{\PYGZgt{}\PYGZgt{}\PYGZgt{} }\PYG{n}{slab1} \PYG{o}{=} \PYG{n}{f}\PYG{p}{(}\PYG{l+s+s1}{\PYGZsq{}}\PYG{l+s+s1}{u}\PYG{l+s+s1}{\PYGZsq{}}\PYG{p}{)} \PYG{c+c1}{\PYGZsh{} use the data file to create a cdms2 slab}
\PYG{g+gp}{\PYGZgt{}\PYGZgt{}\PYGZgt{} }\PYG{n}{a}\PYG{o}{.}\PYG{n}{boxfill}\PYG{p}{(}\PYG{n}{ex}\PYG{p}{,} \PYG{n}{slab1}\PYG{p}{)} \PYG{c+c1}{\PYGZsh{} plot using specified boxfill object}
\PYG{g+go}{\PYGZlt{}vcs.displayplot.Dp ...\PYGZgt{}}
\PYG{g+gp}{\PYGZgt{}\PYGZgt{}\PYGZgt{} }\PYG{n}{ex2}\PYG{o}{=}\PYG{n}{vcs}\PYG{o}{.}\PYG{n}{getboxfill}\PYG{p}{(}\PYG{l+s+s1}{\PYGZsq{}}\PYG{l+s+s1}{polar}\PYG{l+s+s1}{\PYGZsq{}}\PYG{p}{)}  \PYG{c+c1}{\PYGZsh{} instance of \PYGZsq{}polar\PYGZsq{} boxfill graphics method}
\PYG{g+gp}{\PYGZgt{}\PYGZgt{}\PYGZgt{} }\PYG{n}{a}\PYG{o}{.}\PYG{n}{boxfill}\PYG{p}{(}\PYG{n}{ex2}\PYG{p}{,} \PYG{n}{slab1}\PYG{p}{)} \PYG{c+c1}{\PYGZsh{} plot using specified boxfill object}
\PYG{g+go}{\PYGZlt{}vcs.displayplot.Dp ...\PYGZgt{}}
\end{Verbatim}

\item[{Parameters}] \leavevmode\begin{itemize}
\item {} 
\textbf{\texttt{Gfb\_name\_src}} (\href{https://docs.python.org/2/library/functions.html\#str}{\emph{\texttt{str}}}) -- String name of an existing boxfill VCS object

\item {} 
\textbf{\texttt{xaxis}} (\emph{\texttt{cdms2.axis.TransientAxis}}) -- Axis object to replace the slab -1 dim axis

\item {} 
\textbf{\texttt{yaxis}} (\emph{\texttt{cdms2.axis.TransientAxis}}) -- Axis object to replace the slab -2 dim axis, only if slab has more than 1D

\item {} 
\textbf{\texttt{zaxis}} (\emph{\texttt{cdms2.axis.TransientAxis}}) -- Axis object to replace the slab -3 dim axis, only if slab has more than 2D

\item {} 
\textbf{\texttt{taxis}} (\emph{\texttt{cdms2.axis.TransientAxis}}) -- Axis object to replace the slab -4 dim axis, only if slab has more than 3D

\item {} 
\textbf{\texttt{waxis}} (\emph{\texttt{cdms2.axis.TransientAxis}}) -- Axis object to replace the slab -5 dim axis, only if slab has more than 4D

\item {} 
\textbf{\texttt{xrev}} (\href{https://docs.python.org/2/library/functions.html\#bool}{\emph{\texttt{bool}}}) -- reverse x axis

\item {} 
\textbf{\texttt{yrev}} (\href{https://docs.python.org/2/library/functions.html\#bool}{\emph{\texttt{bool}}}) -- reverse y axis, only if slab has more than 1D

\item {} 
\textbf{\texttt{xarray}} (\href{https://docs.python.org/2/library/array.html\#module-array}{\emph{\texttt{array}}}) -- Values to use instead of x axis

\item {} 
\textbf{\texttt{yarray}} (\href{https://docs.python.org/2/library/array.html\#module-array}{\emph{\texttt{array}}}) -- Values to use instead of y axis, only if var has more than 1D

\item {} 
\textbf{\texttt{zarray}} (\href{https://docs.python.org/2/library/array.html\#module-array}{\emph{\texttt{array}}}) -- Values to use instead of z axis, only if var has more than 2D

\item {} 
\textbf{\texttt{tarray}} (\href{https://docs.python.org/2/library/array.html\#module-array}{\emph{\texttt{array}}}) -- Values to use instead of t axis, only if var has more than 3D

\item {} 
\textbf{\texttt{warray}} (\href{https://docs.python.org/2/library/array.html\#module-array}{\emph{\texttt{array}}}) -- Values to use instead of w axis, only if var has more than 4D

\item {} 
\textbf{\texttt{continents}} (\href{https://docs.python.org/2/library/functions.html\#int}{\emph{\texttt{int}}}) -- continents type number

\item {} 
\textbf{\texttt{name}} (\href{https://docs.python.org/2/library/functions.html\#str}{\emph{\texttt{str}}}) -- replaces variable name on plot

\item {} 
\textbf{\texttt{time}} (\emph{\texttt{A cdtime object}}) -- replaces time name on plot

\item {} 
\textbf{\texttt{units}} (\href{https://docs.python.org/2/library/functions.html\#str}{\emph{\texttt{str}}}) -- replaces units value on plot

\item {} 
\textbf{\texttt{ymd}} (\href{https://docs.python.org/2/library/functions.html\#str}{\emph{\texttt{str}}}) -- replaces year/month/day on plot

\item {} 
\textbf{\texttt{hms}} (\href{https://docs.python.org/2/library/functions.html\#str}{\emph{\texttt{str}}}) -- replaces hh/mm/ss on plot

\item {} 
\textbf{\texttt{file\_comment}} (\href{https://docs.python.org/2/library/functions.html\#str}{\emph{\texttt{str}}}) -- replaces file\_comment on plot

\item {} 
\textbf{\texttt{xbounds}} (\href{https://docs.python.org/2/library/array.html\#module-array}{\emph{\texttt{array}}}) -- Values to use instead of x axis bounds values

\item {} 
\textbf{\texttt{ybounds}} (\href{https://docs.python.org/2/library/array.html\#module-array}{\emph{\texttt{array}}}) -- Values to use instead of y axis bounds values (if exist)

\item {} 
\textbf{\texttt{xname}} (\href{https://docs.python.org/2/library/functions.html\#str}{\emph{\texttt{str}}}) -- replace xaxis name on plot

\item {} 
\textbf{\texttt{yname}} (\href{https://docs.python.org/2/library/functions.html\#str}{\emph{\texttt{str}}}) -- replace yaxis name on plot (if exists)

\item {} 
\textbf{\texttt{zname}} (\href{https://docs.python.org/2/library/functions.html\#str}{\emph{\texttt{str}}}) -- replace zaxis name on plot (if exists)

\item {} 
\textbf{\texttt{tname}} (\href{https://docs.python.org/2/library/functions.html\#str}{\emph{\texttt{str}}}) -- replace taxis name on plot (if exists)

\item {} 
\textbf{\texttt{wname}} (\href{https://docs.python.org/2/library/functions.html\#str}{\emph{\texttt{str}}}) -- replace waxis name on plot (if exists)

\item {} 
\textbf{\texttt{xunits}} (\href{https://docs.python.org/2/library/functions.html\#str}{\emph{\texttt{str}}}) -- replace xaxis units on plot

\item {} 
\textbf{\texttt{yunits}} (\href{https://docs.python.org/2/library/functions.html\#str}{\emph{\texttt{str}}}) -- replace yaxis units on plot (if exists)

\item {} 
\textbf{\texttt{zunits}} (\href{https://docs.python.org/2/library/functions.html\#str}{\emph{\texttt{str}}}) -- replace zaxis units on plot (if exists)

\item {} 
\textbf{\texttt{tunits}} (\href{https://docs.python.org/2/library/functions.html\#str}{\emph{\texttt{str}}}) -- replace taxis units on plot (if exists)

\item {} 
\textbf{\texttt{wunits}} (\href{https://docs.python.org/2/library/functions.html\#str}{\emph{\texttt{str}}}) -- replace waxis units on plot (if exists)

\item {} 
\textbf{\texttt{xweights}} (\href{https://docs.python.org/2/library/array.html\#module-array}{\emph{\texttt{array}}}) -- replace xaxis weights used for computing mean

\item {} 
\textbf{\texttt{yweights}} (\href{https://docs.python.org/2/library/array.html\#module-array}{\emph{\texttt{array}}}) -- replace xaxis weights used for computing mean

\item {} 
\textbf{\texttt{comment1}} (\href{https://docs.python.org/2/library/functions.html\#str}{\emph{\texttt{str}}}) -- replaces comment1 on plot

\item {} 
\textbf{\texttt{comment2}} (\href{https://docs.python.org/2/library/functions.html\#str}{\emph{\texttt{str}}}) -- replaces comment2 on plot

\item {} 
\textbf{\texttt{comment3}} (\href{https://docs.python.org/2/library/functions.html\#str}{\emph{\texttt{str}}}) -- replaces comment3 on plot

\item {} 
\textbf{\texttt{comment4}} (\href{https://docs.python.org/2/library/functions.html\#str}{\emph{\texttt{str}}}) -- replaces comment4 on plot

\item {} 
\textbf{\texttt{long\_name}} (\href{https://docs.python.org/2/library/functions.html\#str}{\emph{\texttt{str}}}) -- replaces long\_name on plot

\item {} 
\textbf{\texttt{grid}} (\emph{\texttt{cdms2.grid.TransientRectGrid}}) -- replaces array grid (if exists)

\item {} 
\textbf{\texttt{bg}} (\emph{\texttt{bool/int}}) -- plots in background mode

\item {} 
\textbf{\texttt{ratio}} (\index{xmtics1 (vcs.Canvas.Canvas attribute)}\index{xmtics2 (vcs.Canvas.Canvas attribute)}\index{ymtics1 (vcs.Canvas.Canvas attribute)}\index{ymtics2 (vcs.Canvas.Canvas attribute)}\index{xticlabels1 (vcs.Canvas.Canvas attribute)}\index{xticlabels2 (vcs.Canvas.Canvas attribute)}\index{yticlabels1 (vcs.Canvas.Canvas attribute)}\index{yticlabels2 (vcs.Canvas.Canvas attribute)}\index{projection (vcs.Canvas.Canvas attribute)}\index{datawc\_x1 (vcs.Canvas.Canvas attribute)}\index{datawc\_x2 (vcs.Canvas.Canvas attribute)}\index{datawc\_y1 (vcs.Canvas.Canvas attribute)}\index{datawc\_y2 (vcs.Canvas.Canvas attribute)}\index{datawc\_timeunits (vcs.Canvas.Canvas attribute)}\index{datawc\_calendar (vcs.Canvas.Canvas attribute)}) -- sets the y/x ratio ,if passed as a string with `t' at the end, will aslo moves the ticks

\item {} 
\textbf{\texttt{xaxisconvert}} (\href{https://docs.python.org/2/library/functions.html\#str}{\emph{\texttt{str}}}) -- (Ex: `linear') converting xaxis linear/log/log10/ln/exp/area\_wt

\item {} 
\textbf{\texttt{yaxisconvert}} (\href{https://docs.python.org/2/library/functions.html\#str}{\emph{\texttt{str}}}) -- (Ex: `linear') converting yaxis linear/log/log10/ln/exp/area\_wt

\item {} 
\textbf{\texttt{GM\_name}} -- (Ex: `default') retrieve the graphics method object of the given name. If no name is given, then retrieve the `default' graphics method.

\end{itemize}

\item[{Returns}] \leavevmode
A pre-existing boxfill graphics method

\item[{Return type}] \leavevmode
{\hyperref[vcs/graphics/boxfill:vcs.boxfill.Gfb]{\sphinxcrossref{vcs.boxfill.Gfb}}}

\end{description}\end{quote}

\end{fulllineitems}

\index{getcolorcell() (vcs.Canvas.Canvas method)}

\begin{fulllineitems}
\phantomsection\label{vcs/Canvas:vcs.Canvas.Canvas.getcolorcell}\pysiglinewithargsret{\sphinxbfcode{getcolorcell}}{\emph{*args}}{}
Gets the colorcell of the provided object's colormap at the specified cell index.
If no object is provided, or if the provided object has no colormap, the default colormap is used.
\begin{quote}\begin{description}
\item[{Example}] \leavevmode
\begin{Verbatim}[commandchars=\\\{\}]
\PYG{g+gp}{\PYGZgt{}\PYGZgt{}\PYGZgt{} }\PYG{n}{a}\PYG{o}{=}\PYG{n}{vcs}\PYG{o}{.}\PYG{n}{init}\PYG{p}{(}\PYG{p}{)}
\PYG{g+gp}{\PYGZgt{}\PYGZgt{}\PYGZgt{} }\PYG{n}{b}\PYG{o}{=}\PYG{n}{vcs}\PYG{o}{.}\PYG{n}{createboxfill}\PYG{p}{(}\PYG{p}{)}
\PYG{g+gp}{\PYGZgt{}\PYGZgt{}\PYGZgt{} }\PYG{n}{b}\PYG{o}{.}\PYG{n}{colormap}\PYG{o}{=}\PYG{l+s+s1}{\PYGZsq{}}\PYG{l+s+s1}{rainbow}\PYG{l+s+s1}{\PYGZsq{}}
\PYG{g+gp}{\PYGZgt{}\PYGZgt{}\PYGZgt{} }\PYG{n}{a}\PYG{o}{.}\PYG{n}{getcolorcell}\PYG{p}{(}\PYG{l+m+mi}{2}\PYG{p}{,}\PYG{n}{b}\PYG{p}{)}
\PYG{g+go}{[85, 85, 85, 100.0]}
\end{Verbatim}

\item[{Parameters}] \leavevmode\begin{itemize}
\item {} 
\textbf{\texttt{cell}} (\href{https://docs.python.org/2/library/functions.html\#int}{\emph{\texttt{int}}}) -- An integer value indicating the index of the desired colorcell.

\item {} 
\textbf{\texttt{obj}} (\emph{\texttt{Any VCS object capable of containing a colormap}}) -- Optional parameter containing the object to extract a colormap from.

\end{itemize}

\item[{Returns}] \leavevmode
The RGBA values of the colormap at the specified cell index.

\item[{Return type}] \leavevmode
{\hyperref[vcs/graphics/boxfill:vcs.boxfill.Gfb.list]{\sphinxcrossref{list}}}

\end{description}\end{quote}

\end{fulllineitems}

\index{getcolormap() (vcs.Canvas.Canvas method)}

\begin{fulllineitems}
\phantomsection\label{vcs/Canvas:vcs.Canvas.Canvas.getcolormap}\pysiglinewithargsret{\sphinxbfcode{getcolormap}}{\emph{Cp\_name\_src='default'}}{}
VCS contains a list of secondary methods. This function will create a
colormap class object from an existing VCS colormap secondary method. If
no colormap name is given, then colormap `default' will be used.

\begin{notice}{note}{Note:}
VCS does not allow the modification of `default' attribute sets.
However, a `default' attribute set that has been copied under a
different name can be modified. (See the {\hyperref[vcs/misc/manageElements:vcs.manageElements.createcolormap]{\sphinxcrossref{\sphinxcode{vcs.manageElements.createcolormap()}}}} function.)
\end{notice}
\begin{quote}\begin{description}
\item[{Example}] \leavevmode
\begin{Verbatim}[commandchars=\\\{\}]
\PYG{g+gp}{\PYGZgt{}\PYGZgt{}\PYGZgt{} }\PYG{n}{a}\PYG{o}{=}\PYG{n}{vcs}\PYG{o}{.}\PYG{n}{init}\PYG{p}{(}\PYG{p}{)}
\PYG{g+gp}{\PYGZgt{}\PYGZgt{}\PYGZgt{} }\PYG{n}{vcs}\PYG{o}{.}\PYG{n}{listelements}\PYG{p}{(}\PYG{l+s+s1}{\PYGZsq{}}\PYG{l+s+s1}{colormap}\PYG{l+s+s1}{\PYGZsq{}}\PYG{p}{)} \PYG{c+c1}{\PYGZsh{} Show all the existing colormap secondary methods}
\PYG{g+go}{[...]}
\PYG{g+gp}{\PYGZgt{}\PYGZgt{}\PYGZgt{} }\PYG{n}{ex}\PYG{o}{=}\PYG{n}{vcs}\PYG{o}{.}\PYG{n}{getcolormap}\PYG{p}{(}\PYG{p}{)}  \PYG{c+c1}{\PYGZsh{} instance of \PYGZsq{}default\PYGZsq{} colormap secondary method}
\PYG{g+gp}{\PYGZgt{}\PYGZgt{}\PYGZgt{} }\PYG{n}{ex2}\PYG{o}{=}\PYG{n}{vcs}\PYG{o}{.}\PYG{n}{getcolormap}\PYG{p}{(}\PYG{l+s+s1}{\PYGZsq{}}\PYG{l+s+s1}{rainbow}\PYG{l+s+s1}{\PYGZsq{}}\PYG{p}{)}  \PYG{c+c1}{\PYGZsh{} instance of \PYGZsq{}rainbow\PYGZsq{} colormap secondary method}
\end{Verbatim}

\item[{Parameters}] \leavevmode
\textbf{\texttt{Cp\_name\_src}} (\href{https://docs.python.org/2/library/functions.html\#str}{\emph{\texttt{str}}}) -- String name of an existing colormap VCS object

\item[{Returns}] \leavevmode
A pre-existing VCS colormap object

\item[{Return type}] \leavevmode
{\hyperref[vcs/misc/colormap:vcs.colormap.Cp]{\sphinxcrossref{vcs.colormap.Cp}}}

\end{description}\end{quote}

\end{fulllineitems}

\index{getcolormapname() (vcs.Canvas.Canvas method)}

\begin{fulllineitems}
\phantomsection\label{vcs/Canvas:vcs.Canvas.Canvas.getcolormapname}\pysiglinewithargsret{\sphinxbfcode{getcolormapname}}{}{}
Returns the name of the colormap this canvas is set to use by default.

To set that colormap, use {\color{red}\bfseries{}:ref:{}`setcolormap{}`\_}.

\end{fulllineitems}

\index{getcontinentstype() (vcs.Canvas.Canvas method)}

\begin{fulllineitems}
\phantomsection\label{vcs/Canvas:vcs.Canvas.Canvas.getcontinentstype}\pysiglinewithargsret{\sphinxbfcode{getcontinentstype}}{\emph{*args}}{}
Retrieve continents type from VCS; either an integer between 0 and 11 or the
path to a custom continentstype.
\begin{quote}\begin{description}
\item[{Example}] \leavevmode
\begin{Verbatim}[commandchars=\\\{\}]
\PYG{g+gp}{\PYGZgt{}\PYGZgt{}\PYGZgt{} }\PYG{n}{a}\PYG{o}{=}\PYG{n}{vcs}\PYG{o}{.}\PYG{n}{init}\PYG{p}{(}\PYG{p}{)}
\PYG{g+gp}{\PYGZgt{}\PYGZgt{}\PYGZgt{} }\PYG{n}{cont\PYGZus{}type} \PYG{o}{=} \PYG{n}{a}\PYG{o}{.}\PYG{n}{getcontinentstype}\PYG{p}{(}\PYG{p}{)} \PYG{c+c1}{\PYGZsh{} Get the continents type}
\end{Verbatim}

\item[{Returns}] \leavevmode
An int between 1 and 0, or the path to a custom continentstype

\item[{Return type}] \leavevmode
int or system filepath

\end{description}\end{quote}

\end{fulllineitems}

\index{getdrawlogo() (vcs.Canvas.Canvas method)}

\begin{fulllineitems}
\phantomsection\label{vcs/Canvas:vcs.Canvas.Canvas.getdrawlogo}\pysiglinewithargsret{\sphinxbfcode{getdrawlogo}}{}{}
Returns value of draw logo. By default, draw logo is set to True.
\begin{quote}\begin{description}
\item[{Example}] \leavevmode
\begin{Verbatim}[commandchars=\\\{\}]
\PYG{g+gp}{\PYGZgt{}\PYGZgt{}\PYGZgt{} }\PYG{n}{a}\PYG{o}{=}\PYG{n}{vcs}\PYG{o}{.}\PYG{n}{init}\PYG{p}{(}\PYG{p}{)}
\PYG{g+gp}{\PYGZgt{}\PYGZgt{}\PYGZgt{} }\PYG{n}{a}\PYG{o}{.}\PYG{n}{getdrawlogo}\PYG{p}{(}\PYG{p}{)}
\PYG{g+go}{True}
\PYG{g+gp}{\PYGZgt{}\PYGZgt{}\PYGZgt{} }\PYG{n}{a}\PYG{o}{.}\PYG{n}{drawlogooff}\PYG{p}{(}\PYG{p}{)}
\PYG{g+gp}{\PYGZgt{}\PYGZgt{}\PYGZgt{} }\PYG{n}{a}\PYG{o}{.}\PYG{n}{getdrawlogo}\PYG{p}{(}\PYG{p}{)}
\PYG{g+go}{False}
\end{Verbatim}

\item[{Returns}] \leavevmode
Boolean value of system variable which indicates whether logo will be drawn

\item[{Return type}] \leavevmode
\href{https://docs.python.org/2/library/functions.html\#bool}{bool}

\end{description}\end{quote}

\end{fulllineitems}

\index{getfillarea() (vcs.Canvas.Canvas method)}

\begin{fulllineitems}
\phantomsection\label{vcs/Canvas:vcs.Canvas.Canvas.getfillarea}\pysiglinewithargsret{\sphinxbfcode{getfillarea}}{\emph{name='default'}, \emph{style=None}, \emph{index=None}, \emph{color=None}, \emph{priority=None}, \emph{viewport=None}, \emph{worldcoordinate=None}, \emph{x=None}, \emph{y=None}}{}
VCS contains a list of secondary methods. This function will create a
fillarea class object from an existing VCS fillarea secondary method. If
no fillarea name is given, then fillarea `default' will be used.

\begin{notice}{note}{Note:}
VCS does not allow the modification of `default' attribute sets.
However, a `default' attribute set that has been copied under a
different name can be modified. (See the {\hyperref[vcs/misc/manageElements:vcs.manageElements.createfillarea]{\sphinxcrossref{\sphinxcode{vcs.manageElements.createfillarea()}}}} function.)
\end{notice}
\begin{quote}\begin{description}
\item[{Example}] \leavevmode
\begin{Verbatim}[commandchars=\\\{\}]
\PYG{g+gp}{\PYGZgt{}\PYGZgt{}\PYGZgt{} }\PYG{n}{a}\PYG{o}{=}\PYG{n}{vcs}\PYG{o}{.}\PYG{n}{init}\PYG{p}{(}\PYG{p}{)}
\PYG{g+gp}{\PYGZgt{}\PYGZgt{}\PYGZgt{} }\PYG{n}{vcs}\PYG{o}{.}\PYG{n}{listelements}\PYG{p}{(}\PYG{l+s+s1}{\PYGZsq{}}\PYG{l+s+s1}{fillarea}\PYG{l+s+s1}{\PYGZsq{}}\PYG{p}{)} \PYG{c+c1}{\PYGZsh{} Show all the existing fillarea secondary methods}
\PYG{g+go}{[...]}
\PYG{g+gp}{\PYGZgt{}\PYGZgt{}\PYGZgt{} }\PYG{n}{ex}\PYG{o}{=}\PYG{n}{vcs}\PYG{o}{.}\PYG{n}{getfillarea}\PYG{p}{(}\PYG{p}{)}  \PYG{c+c1}{\PYGZsh{} instance of \PYGZsq{}default\PYGZsq{} fillarea secondary method}
\PYG{g+gp}{\PYGZgt{}\PYGZgt{}\PYGZgt{} }\PYG{n}{a}\PYG{o}{.}\PYG{n}{fillarea}\PYG{p}{(}\PYG{n}{ex}\PYG{p}{)} \PYG{c+c1}{\PYGZsh{} plot using specified fillarea object}
\PYG{g+go}{\PYGZlt{}vcs.displayplot.Dp ...\PYGZgt{}}
\end{Verbatim}

\item[{Parameters}] \leavevmode\begin{itemize}
\item {} 
\textbf{\texttt{name}} (\href{https://docs.python.org/2/library/functions.html\#str}{\emph{\texttt{str}}}) -- String name of an existing fillarea VCS object

\item {} 
\textbf{\texttt{style}} (\href{https://docs.python.org/2/library/functions.html\#str}{\emph{\texttt{str}}}) -- One of ``hatch'', ``solid'', or ``pattern''.

\item {} 
\textbf{\texttt{index}} (\href{https://docs.python.org/2/library/functions.html\#int}{\emph{\texttt{int}}}) -- 
Specifies which \href{http://uvcdat.llnl.gov/gallery/fullsize/pattern\_chart.png}{pattern} to fill with.
Accepts ints from 1-20.


\item {} 
\textbf{\texttt{color}} (\emph{\texttt{str or int}}) -- 
A color name from the \href{https://en.wikipedia.org/wiki/X11\_color\_names}{X11 Color Names list},
or an integer value from 0-255, or an RGB/RGBA tuple/list (e.g. (0,100,0), (100,100,0,50))


\item {} 
\textbf{\texttt{priority}} (\href{https://docs.python.org/2/library/functions.html\#int}{\emph{\texttt{int}}}) -- The layer on which the texttable will be drawn.

\item {} 
\textbf{\texttt{viewport}} (\emph{\texttt{list of floats}}) -- 4 floats between 0 and 1. These specify the area that the X/Y values are mapped to inside of the canvas

\item {} 
\textbf{\texttt{worldcoordinate}} (\emph{\texttt{list of floats}}) -- List of 4 floats (xmin, xmax, ymin, ymax)

\item {} 
\textbf{\texttt{x}} (\emph{\texttt{list of floats}}) -- List of lists of x coordinates. Values must be between worldcoordinate{[}0{]} and worldcoordinate{[}1{]}.

\item {} 
\textbf{\texttt{y}} (\emph{\texttt{list of floats}}) -- List of lists of y coordinates. Values must be between worldcoordinate{[}2{]} and worldcoordinate{[}3{]}.

\end{itemize}

\item[{Returns}] \leavevmode
A fillarea secondary object

\item[{Return type}] \leavevmode
{\hyperref[vcs/secondary/fillarea:vcs.fillarea.Tf]{\sphinxcrossref{vcs.fillarea.Tf}}}

\end{description}\end{quote}

\end{fulllineitems}

\index{getfont() (vcs.Canvas.Canvas method)}

\begin{fulllineitems}
\phantomsection\label{vcs/Canvas:vcs.Canvas.Canvas.getfont}\pysiglinewithargsret{\sphinxbfcode{getfont}}{\emph{font}}{}
Get the font name/number associated with a font number/name
\begin{quote}\begin{description}
\item[{Example}] \leavevmode
\begin{Verbatim}[commandchars=\\\{\}]
\PYG{g+gp}{\PYGZgt{}\PYGZgt{}\PYGZgt{} }\PYG{n}{a}\PYG{o}{=}\PYG{n}{vcs}\PYG{o}{.}\PYG{n}{init}\PYG{p}{(}\PYG{p}{)}
\PYG{g+gp}{\PYGZgt{}\PYGZgt{}\PYGZgt{} }\PYG{n}{font\PYGZus{}names}\PYG{o}{=}\PYG{p}{[}\PYG{p}{]}
\PYG{g+gp}{\PYGZgt{}\PYGZgt{}\PYGZgt{} }\PYG{k}{for} \PYG{n}{i} \PYG{o+ow}{in} \PYG{n+nb}{range}\PYG{p}{(}\PYG{l+m+mi}{1}\PYG{p}{,}\PYG{l+m+mi}{17}\PYG{p}{)}\PYG{p}{:}
\PYG{g+gp}{... }    \PYG{n}{font\PYGZus{}names}\PYG{o}{.}\PYG{n}{append}\PYG{p}{(}\PYG{n+nb}{str}\PYG{p}{(}\PYG{n}{a}\PYG{o}{.}\PYG{n}{getfont}\PYG{p}{(}\PYG{n}{i}\PYG{p}{)}\PYG{p}{)}\PYG{p}{)} \PYG{c+c1}{\PYGZsh{} font\PYGZus{}names is now filled with all font names}
\PYG{g+gp}{\PYGZgt{}\PYGZgt{}\PYGZgt{} }\PYG{n}{font\PYGZus{}names}
\PYG{g+go}{[\PYGZsq{}default\PYGZsq{}, ...]}
\PYG{g+gp}{\PYGZgt{}\PYGZgt{}\PYGZgt{} }\PYG{n}{font\PYGZus{}numbers} \PYG{o}{=} \PYG{p}{[}\PYG{p}{]}
\PYG{g+gp}{\PYGZgt{}\PYGZgt{}\PYGZgt{} }\PYG{k}{for} \PYG{n}{name} \PYG{o+ow}{in} \PYG{n}{font\PYGZus{}names}\PYG{p}{:}
\PYG{g+gp}{... }    \PYG{n}{font\PYGZus{}numbers}\PYG{o}{.}\PYG{n}{append}\PYG{p}{(}\PYG{n}{a}\PYG{o}{.}\PYG{n}{getfont}\PYG{p}{(}\PYG{n}{name}\PYG{p}{)}\PYG{p}{)} \PYG{c+c1}{\PYGZsh{} font\PYGZus{}numbers is now filled with all font numbers}
\PYG{g+gp}{\PYGZgt{}\PYGZgt{}\PYGZgt{} }\PYG{n}{font\PYGZus{}numbers}
\PYG{g+go}{[1, 2, 3, 4, 5, 6, 7, 8, 9, 10, 11, 12, 13, 14, 15, 16]}
\end{Verbatim}

\item[{Parameters}] \leavevmode
\textbf{\texttt{font}} (\emph{\texttt{int or str}}) -- The font name/number

\item[{Returns}] \leavevmode
If font parameter was a string, will return the integer associated with that string.
If font parameter was an integer, will return the string associated with that integer.

\item[{Return type}] \leavevmode
int or str

\end{description}\end{quote}

\end{fulllineitems}

\index{getfontname() (vcs.Canvas.Canvas method)}

\begin{fulllineitems}
\phantomsection\label{vcs/Canvas:vcs.Canvas.Canvas.getfontname}\pysiglinewithargsret{\sphinxbfcode{getfontname}}{\emph{number}}{}
Retrieve a font name for a given font index.
\begin{quote}\begin{description}
\item[{Parameters}] \leavevmode
\textbf{\texttt{number}} (\href{https://docs.python.org/2/library/functions.html\#int}{\emph{\texttt{int}}}) -- Index of the font to get the name of.

\end{description}\end{quote}

\end{fulllineitems}

\index{getfontnumber() (vcs.Canvas.Canvas method)}

\begin{fulllineitems}
\phantomsection\label{vcs/Canvas:vcs.Canvas.Canvas.getfontnumber}\pysiglinewithargsret{\sphinxbfcode{getfontnumber}}{\emph{name}}{}
Retrieve a font index for a given font name.
\begin{quote}\begin{description}
\item[{Parameters}] \leavevmode
\textbf{\texttt{name}} (\href{https://docs.python.org/2/library/functions.html\#str}{\emph{\texttt{str}}}) -- Name of the font to get the index of.

\end{description}\end{quote}

\end{fulllineitems}

\index{getisofill() (vcs.Canvas.Canvas method)}

\begin{fulllineitems}
\phantomsection\label{vcs/Canvas:vcs.Canvas.Canvas.getisofill}\pysiglinewithargsret{\sphinxbfcode{getisofill}}{\emph{Gfi\_name\_src='default'}}{}
VCS contains a list of graphics methods. This function will create a
isofill class object from an existing VCS isofill graphics method. If
no isofill name is given, then isofill `default' will be used.

\begin{notice}{note}{Note:}
VCS does not allow the modification of `default' attribute sets.
However, a `default' attribute set that has been copied under a
different name can be modified. (See the {\hyperref[vcs/misc/manageElements:vcs.manageElements.createisofill]{\sphinxcrossref{\sphinxcode{vcs.manageElements.createisofill()}}}} function.)
\end{notice}
\begin{quote}\begin{description}
\item[{Example}] \leavevmode
\begin{Verbatim}[commandchars=\\\{\}]
\PYG{g+gp}{\PYGZgt{}\PYGZgt{}\PYGZgt{} }\PYG{n}{a}\PYG{o}{=}\PYG{n}{vcs}\PYG{o}{.}\PYG{n}{init}\PYG{p}{(}\PYG{p}{)}
\PYG{g+gp}{\PYGZgt{}\PYGZgt{}\PYGZgt{} }\PYG{n}{vcs}\PYG{o}{.}\PYG{n}{listelements}\PYG{p}{(}\PYG{l+s+s1}{\PYGZsq{}}\PYG{l+s+s1}{isofill}\PYG{l+s+s1}{\PYGZsq{}}\PYG{p}{)} \PYG{c+c1}{\PYGZsh{} Show all the existing isofill graphics methods}
\PYG{g+go}{[...]}
\PYG{g+gp}{\PYGZgt{}\PYGZgt{}\PYGZgt{} }\PYG{n}{ex}\PYG{o}{=}\PYG{n}{vcs}\PYG{o}{.}\PYG{n}{getisofill}\PYG{p}{(}\PYG{p}{)}  \PYG{c+c1}{\PYGZsh{} instance of \PYGZsq{}default\PYGZsq{} isofill graphics method}
\PYG{g+gp}{\PYGZgt{}\PYGZgt{}\PYGZgt{} }\PYG{k+kn}{import} \PYG{n+nn}{cdms2} \PYG{c+c1}{\PYGZsh{} Need cdms2 to create a slab}
\PYG{g+gp}{\PYGZgt{}\PYGZgt{}\PYGZgt{} }\PYG{n}{f} \PYG{o}{=} \PYG{n}{cdms2}\PYG{o}{.}\PYG{n}{open}\PYG{p}{(}\PYG{n}{vcs}\PYG{o}{.}\PYG{n}{sample\PYGZus{}data}\PYG{o}{+}\PYG{l+s+s1}{\PYGZsq{}}\PYG{l+s+s1}{/clt.nc}\PYG{l+s+s1}{\PYGZsq{}}\PYG{p}{)} \PYG{c+c1}{\PYGZsh{} use cdms2 to open a data file}
\PYG{g+gp}{\PYGZgt{}\PYGZgt{}\PYGZgt{} }\PYG{n}{slab1} \PYG{o}{=} \PYG{n}{f}\PYG{p}{(}\PYG{l+s+s1}{\PYGZsq{}}\PYG{l+s+s1}{u}\PYG{l+s+s1}{\PYGZsq{}}\PYG{p}{)} \PYG{c+c1}{\PYGZsh{} use the data file to create a cdms2 slab}
\PYG{g+gp}{\PYGZgt{}\PYGZgt{}\PYGZgt{} }\PYG{n}{a}\PYG{o}{.}\PYG{n}{isofill}\PYG{p}{(}\PYG{n}{ex}\PYG{p}{,} \PYG{n}{slab1}\PYG{p}{)} \PYG{c+c1}{\PYGZsh{} plot using specified isofill object}
\PYG{g+go}{\PYGZlt{}vcs.displayplot.Dp ...\PYGZgt{}}
\PYG{g+gp}{\PYGZgt{}\PYGZgt{}\PYGZgt{} }\PYG{n}{ex2}\PYG{o}{=}\PYG{n}{vcs}\PYG{o}{.}\PYG{n}{getisofill}\PYG{p}{(}\PYG{l+s+s1}{\PYGZsq{}}\PYG{l+s+s1}{polar}\PYG{l+s+s1}{\PYGZsq{}}\PYG{p}{)}  \PYG{c+c1}{\PYGZsh{} instance of \PYGZsq{}polar\PYGZsq{} isofill graphics method}
\PYG{g+gp}{\PYGZgt{}\PYGZgt{}\PYGZgt{} }\PYG{n}{a}\PYG{o}{.}\PYG{n}{isofill}\PYG{p}{(}\PYG{n}{ex2}\PYG{p}{,} \PYG{n}{slab1}\PYG{p}{)} \PYG{c+c1}{\PYGZsh{} plot using specified isofill object}
\PYG{g+go}{\PYGZlt{}vcs.displayplot.Dp ...\PYGZgt{}}
\end{Verbatim}

\item[{Parameters}] \leavevmode\begin{itemize}
\item {} 
\textbf{\texttt{Gfi\_name\_src}} (\href{https://docs.python.org/2/library/functions.html\#str}{\emph{\texttt{str}}}) -- String name of an existing isofill VCS object

\item {} 
\textbf{\texttt{xaxis}} (\emph{\texttt{cdms2.axis.TransientAxis}}) -- Axis object to replace the slab -1 dim axis

\item {} 
\textbf{\texttt{yaxis}} (\emph{\texttt{cdms2.axis.TransientAxis}}) -- Axis object to replace the slab -2 dim axis, only if slab has more than 1D

\item {} 
\textbf{\texttt{zaxis}} (\emph{\texttt{cdms2.axis.TransientAxis}}) -- Axis object to replace the slab -3 dim axis, only if slab has more than 2D

\item {} 
\textbf{\texttt{taxis}} (\emph{\texttt{cdms2.axis.TransientAxis}}) -- Axis object to replace the slab -4 dim axis, only if slab has more than 3D

\item {} 
\textbf{\texttt{waxis}} (\emph{\texttt{cdms2.axis.TransientAxis}}) -- Axis object to replace the slab -5 dim axis, only if slab has more than 4D

\item {} 
\textbf{\texttt{xrev}} (\href{https://docs.python.org/2/library/functions.html\#bool}{\emph{\texttt{bool}}}) -- reverse x axis

\item {} 
\textbf{\texttt{yrev}} (\href{https://docs.python.org/2/library/functions.html\#bool}{\emph{\texttt{bool}}}) -- reverse y axis, only if slab has more than 1D

\item {} 
\textbf{\texttt{xarray}} (\href{https://docs.python.org/2/library/array.html\#module-array}{\emph{\texttt{array}}}) -- Values to use instead of x axis

\item {} 
\textbf{\texttt{yarray}} (\href{https://docs.python.org/2/library/array.html\#module-array}{\emph{\texttt{array}}}) -- Values to use instead of y axis, only if var has more than 1D

\item {} 
\textbf{\texttt{zarray}} (\href{https://docs.python.org/2/library/array.html\#module-array}{\emph{\texttt{array}}}) -- Values to use instead of z axis, only if var has more than 2D

\item {} 
\textbf{\texttt{tarray}} (\href{https://docs.python.org/2/library/array.html\#module-array}{\emph{\texttt{array}}}) -- Values to use instead of t axis, only if var has more than 3D

\item {} 
\textbf{\texttt{warray}} (\href{https://docs.python.org/2/library/array.html\#module-array}{\emph{\texttt{array}}}) -- Values to use instead of w axis, only if var has more than 4D

\item {} 
\textbf{\texttt{continents}} (\href{https://docs.python.org/2/library/functions.html\#int}{\emph{\texttt{int}}}) -- continents type number

\item {} 
\textbf{\texttt{name}} (\href{https://docs.python.org/2/library/functions.html\#str}{\emph{\texttt{str}}}) -- replaces variable name on plot

\item {} 
\textbf{\texttt{time}} (\emph{\texttt{A cdtime object}}) -- replaces time name on plot

\item {} 
\textbf{\texttt{units}} (\href{https://docs.python.org/2/library/functions.html\#str}{\emph{\texttt{str}}}) -- replaces units value on plot

\item {} 
\textbf{\texttt{ymd}} (\href{https://docs.python.org/2/library/functions.html\#str}{\emph{\texttt{str}}}) -- replaces year/month/day on plot

\item {} 
\textbf{\texttt{hms}} (\href{https://docs.python.org/2/library/functions.html\#str}{\emph{\texttt{str}}}) -- replaces hh/mm/ss on plot

\item {} 
\textbf{\texttt{file\_comment}} (\href{https://docs.python.org/2/library/functions.html\#str}{\emph{\texttt{str}}}) -- replaces file\_comment on plot

\item {} 
\textbf{\texttt{xbounds}} (\href{https://docs.python.org/2/library/array.html\#module-array}{\emph{\texttt{array}}}) -- Values to use instead of x axis bounds values

\item {} 
\textbf{\texttt{ybounds}} (\href{https://docs.python.org/2/library/array.html\#module-array}{\emph{\texttt{array}}}) -- Values to use instead of y axis bounds values (if exist)

\item {} 
\textbf{\texttt{xname}} (\href{https://docs.python.org/2/library/functions.html\#str}{\emph{\texttt{str}}}) -- replace xaxis name on plot

\item {} 
\textbf{\texttt{yname}} (\href{https://docs.python.org/2/library/functions.html\#str}{\emph{\texttt{str}}}) -- replace yaxis name on plot (if exists)

\item {} 
\textbf{\texttt{zname}} (\href{https://docs.python.org/2/library/functions.html\#str}{\emph{\texttt{str}}}) -- replace zaxis name on plot (if exists)

\item {} 
\textbf{\texttt{tname}} (\href{https://docs.python.org/2/library/functions.html\#str}{\emph{\texttt{str}}}) -- replace taxis name on plot (if exists)

\item {} 
\textbf{\texttt{wname}} (\href{https://docs.python.org/2/library/functions.html\#str}{\emph{\texttt{str}}}) -- replace waxis name on plot (if exists)

\item {} 
\textbf{\texttt{xunits}} (\href{https://docs.python.org/2/library/functions.html\#str}{\emph{\texttt{str}}}) -- replace xaxis units on plot

\item {} 
\textbf{\texttt{yunits}} (\href{https://docs.python.org/2/library/functions.html\#str}{\emph{\texttt{str}}}) -- replace yaxis units on plot (if exists)

\item {} 
\textbf{\texttt{zunits}} (\href{https://docs.python.org/2/library/functions.html\#str}{\emph{\texttt{str}}}) -- replace zaxis units on plot (if exists)

\item {} 
\textbf{\texttt{tunits}} (\href{https://docs.python.org/2/library/functions.html\#str}{\emph{\texttt{str}}}) -- replace taxis units on plot (if exists)

\item {} 
\textbf{\texttt{wunits}} (\href{https://docs.python.org/2/library/functions.html\#str}{\emph{\texttt{str}}}) -- replace waxis units on plot (if exists)

\item {} 
\textbf{\texttt{xweights}} (\href{https://docs.python.org/2/library/array.html\#module-array}{\emph{\texttt{array}}}) -- replace xaxis weights used for computing mean

\item {} 
\textbf{\texttt{yweights}} (\href{https://docs.python.org/2/library/array.html\#module-array}{\emph{\texttt{array}}}) -- replace xaxis weights used for computing mean

\item {} 
\textbf{\texttt{comment1}} (\href{https://docs.python.org/2/library/functions.html\#str}{\emph{\texttt{str}}}) -- replaces comment1 on plot

\item {} 
\textbf{\texttt{comment2}} (\href{https://docs.python.org/2/library/functions.html\#str}{\emph{\texttt{str}}}) -- replaces comment2 on plot

\item {} 
\textbf{\texttt{comment3}} (\href{https://docs.python.org/2/library/functions.html\#str}{\emph{\texttt{str}}}) -- replaces comment3 on plot

\item {} 
\textbf{\texttt{comment4}} (\href{https://docs.python.org/2/library/functions.html\#str}{\emph{\texttt{str}}}) -- replaces comment4 on plot

\item {} 
\textbf{\texttt{long\_name}} (\href{https://docs.python.org/2/library/functions.html\#str}{\emph{\texttt{str}}}) -- replaces long\_name on plot

\item {} 
\textbf{\texttt{grid}} (\emph{\texttt{cdms2.grid.TransientRectGrid}}) -- replaces array grid (if exists)

\item {} 
\textbf{\texttt{bg}} (\emph{\texttt{bool/int}}) -- plots in background mode

\item {} 
\textbf{\texttt{ratio}} (\index{xmtics1 (vcs.Canvas.Canvas attribute)}\index{xmtics2 (vcs.Canvas.Canvas attribute)}\index{ymtics1 (vcs.Canvas.Canvas attribute)}\index{ymtics2 (vcs.Canvas.Canvas attribute)}\index{xticlabels1 (vcs.Canvas.Canvas attribute)}\index{xticlabels2 (vcs.Canvas.Canvas attribute)}\index{yticlabels1 (vcs.Canvas.Canvas attribute)}\index{yticlabels2 (vcs.Canvas.Canvas attribute)}\index{projection (vcs.Canvas.Canvas attribute)}\index{datawc\_x1 (vcs.Canvas.Canvas attribute)}\index{datawc\_x2 (vcs.Canvas.Canvas attribute)}\index{datawc\_y1 (vcs.Canvas.Canvas attribute)}\index{datawc\_y2 (vcs.Canvas.Canvas attribute)}\index{datawc\_timeunits (vcs.Canvas.Canvas attribute)}\index{datawc\_calendar (vcs.Canvas.Canvas attribute)}) -- sets the y/x ratio ,if passed as a string with `t' at the end, will aslo moves the ticks

\item {} 
\textbf{\texttt{xaxisconvert}} (\href{https://docs.python.org/2/library/functions.html\#str}{\emph{\texttt{str}}}) -- (Ex: `linear') converting xaxis linear/log/log10/ln/exp/area\_wt

\item {} 
\textbf{\texttt{yaxisconvert}} (\href{https://docs.python.org/2/library/functions.html\#str}{\emph{\texttt{str}}}) -- (Ex: `linear') converting yaxis linear/log/log10/ln/exp/area\_wt

\item {} 
\textbf{\texttt{GM\_name}} -- (Ex: `default') retrieve the graphics method object of the given name. If no name is given, then retrieve the `default' graphics method.

\end{itemize}

\item[{Returns}] \leavevmode
The specified isofill VCS object

\item[{Return type}] \leavevmode
{\hyperref[vcs/graphics/isofill:vcs.isofill.Gfi]{\sphinxcrossref{vcs.isofill.Gfi}}}

\end{description}\end{quote}

\end{fulllineitems}

\index{getisoline() (vcs.Canvas.Canvas method)}

\begin{fulllineitems}
\phantomsection\label{vcs/Canvas:vcs.Canvas.Canvas.getisoline}\pysiglinewithargsret{\sphinxbfcode{getisoline}}{\emph{Gi\_name\_src='default'}}{}
VCS contains a list of graphics methods. This function will create a
isoline class object from an existing VCS isoline graphics method. If
no isoline name is given, then isoline `default' will be used.

\begin{notice}{note}{Note:}
VCS does not allow the modification of `default' attribute sets.
However, a `default' attribute set that has been copied under a
different name can be modified. (See the {\hyperref[vcs/misc/manageElements:vcs.manageElements.createisoline]{\sphinxcrossref{\sphinxcode{vcs.manageElements.createisoline()}}}} function.)
\end{notice}
\begin{quote}\begin{description}
\item[{Example}] \leavevmode
\begin{Verbatim}[commandchars=\\\{\}]
\PYG{g+gp}{\PYGZgt{}\PYGZgt{}\PYGZgt{} }\PYG{n}{a}\PYG{o}{=}\PYG{n}{vcs}\PYG{o}{.}\PYG{n}{init}\PYG{p}{(}\PYG{p}{)}
\PYG{g+gp}{\PYGZgt{}\PYGZgt{}\PYGZgt{} }\PYG{n}{vcs}\PYG{o}{.}\PYG{n}{listelements}\PYG{p}{(}\PYG{l+s+s1}{\PYGZsq{}}\PYG{l+s+s1}{isoline}\PYG{l+s+s1}{\PYGZsq{}}\PYG{p}{)} \PYG{c+c1}{\PYGZsh{} Show all the existing isoline graphics methods}
\PYG{g+go}{[...]}
\PYG{g+gp}{\PYGZgt{}\PYGZgt{}\PYGZgt{} }\PYG{n}{ex}\PYG{o}{=}\PYG{n}{vcs}\PYG{o}{.}\PYG{n}{getisoline}\PYG{p}{(}\PYG{p}{)}  \PYG{c+c1}{\PYGZsh{} instance of \PYGZsq{}default\PYGZsq{} isoline graphics method}
\PYG{g+gp}{\PYGZgt{}\PYGZgt{}\PYGZgt{} }\PYG{k+kn}{import} \PYG{n+nn}{cdms2} \PYG{c+c1}{\PYGZsh{} Need cdms2 to create a slab}
\PYG{g+gp}{\PYGZgt{}\PYGZgt{}\PYGZgt{} }\PYG{n}{f} \PYG{o}{=} \PYG{n}{cdms2}\PYG{o}{.}\PYG{n}{open}\PYG{p}{(}\PYG{n}{vcs}\PYG{o}{.}\PYG{n}{sample\PYGZus{}data}\PYG{o}{+}\PYG{l+s+s1}{\PYGZsq{}}\PYG{l+s+s1}{/clt.nc}\PYG{l+s+s1}{\PYGZsq{}}\PYG{p}{)} \PYG{c+c1}{\PYGZsh{} use cdms2 to open a data file}
\PYG{g+gp}{\PYGZgt{}\PYGZgt{}\PYGZgt{} }\PYG{n}{slab1} \PYG{o}{=} \PYG{n}{f}\PYG{p}{(}\PYG{l+s+s1}{\PYGZsq{}}\PYG{l+s+s1}{u}\PYG{l+s+s1}{\PYGZsq{}}\PYG{p}{)} \PYG{c+c1}{\PYGZsh{} use the data file to create a cdms2 slab}
\PYG{g+gp}{\PYGZgt{}\PYGZgt{}\PYGZgt{} }\PYG{n}{a}\PYG{o}{.}\PYG{n}{isoline}\PYG{p}{(}\PYG{n}{ex}\PYG{p}{,} \PYG{n}{slab1}\PYG{p}{)} \PYG{c+c1}{\PYGZsh{} plot using specified isoline object}
\PYG{g+go}{\PYGZlt{}vcs.displayplot.Dp ...\PYGZgt{}}
\PYG{g+gp}{\PYGZgt{}\PYGZgt{}\PYGZgt{} }\PYG{n}{ex2}\PYG{o}{=}\PYG{n}{vcs}\PYG{o}{.}\PYG{n}{getisoline}\PYG{p}{(}\PYG{l+s+s1}{\PYGZsq{}}\PYG{l+s+s1}{polar}\PYG{l+s+s1}{\PYGZsq{}}\PYG{p}{)}  \PYG{c+c1}{\PYGZsh{} instance of \PYGZsq{}polar\PYGZsq{} isoline graphics method}
\PYG{g+gp}{\PYGZgt{}\PYGZgt{}\PYGZgt{} }\PYG{n}{a}\PYG{o}{.}\PYG{n}{isoline}\PYG{p}{(}\PYG{n}{ex2}\PYG{p}{,} \PYG{n}{slab1}\PYG{p}{)} \PYG{c+c1}{\PYGZsh{} plot using specified isoline object}
\PYG{g+go}{\PYGZlt{}vcs.displayplot.Dp ...\PYGZgt{}}
\end{Verbatim}

\item[{Parameters}] \leavevmode\begin{itemize}
\item {} 
\textbf{\texttt{Gi\_name\_src}} (\href{https://docs.python.org/2/library/functions.html\#str}{\emph{\texttt{str}}}) -- String name of an existing isoline VCS object

\item {} 
\textbf{\texttt{xaxis}} (\emph{\texttt{cdms2.axis.TransientAxis}}) -- Axis object to replace the slab -1 dim axis

\item {} 
\textbf{\texttt{yaxis}} (\emph{\texttt{cdms2.axis.TransientAxis}}) -- Axis object to replace the slab -2 dim axis, only if slab has more than 1D

\item {} 
\textbf{\texttt{zaxis}} (\emph{\texttt{cdms2.axis.TransientAxis}}) -- Axis object to replace the slab -3 dim axis, only if slab has more than 2D

\item {} 
\textbf{\texttt{taxis}} (\emph{\texttt{cdms2.axis.TransientAxis}}) -- Axis object to replace the slab -4 dim axis, only if slab has more than 3D

\item {} 
\textbf{\texttt{waxis}} (\emph{\texttt{cdms2.axis.TransientAxis}}) -- Axis object to replace the slab -5 dim axis, only if slab has more than 4D

\item {} 
\textbf{\texttt{xrev}} (\href{https://docs.python.org/2/library/functions.html\#bool}{\emph{\texttt{bool}}}) -- reverse x axis

\item {} 
\textbf{\texttt{yrev}} (\href{https://docs.python.org/2/library/functions.html\#bool}{\emph{\texttt{bool}}}) -- reverse y axis, only if slab has more than 1D

\item {} 
\textbf{\texttt{xarray}} (\href{https://docs.python.org/2/library/array.html\#module-array}{\emph{\texttt{array}}}) -- Values to use instead of x axis

\item {} 
\textbf{\texttt{yarray}} (\href{https://docs.python.org/2/library/array.html\#module-array}{\emph{\texttt{array}}}) -- Values to use instead of y axis, only if var has more than 1D

\item {} 
\textbf{\texttt{zarray}} (\href{https://docs.python.org/2/library/array.html\#module-array}{\emph{\texttt{array}}}) -- Values to use instead of z axis, only if var has more than 2D

\item {} 
\textbf{\texttt{tarray}} (\href{https://docs.python.org/2/library/array.html\#module-array}{\emph{\texttt{array}}}) -- Values to use instead of t axis, only if var has more than 3D

\item {} 
\textbf{\texttt{warray}} (\href{https://docs.python.org/2/library/array.html\#module-array}{\emph{\texttt{array}}}) -- Values to use instead of w axis, only if var has more than 4D

\item {} 
\textbf{\texttt{continents}} (\href{https://docs.python.org/2/library/functions.html\#int}{\emph{\texttt{int}}}) -- continents type number

\item {} 
\textbf{\texttt{name}} (\href{https://docs.python.org/2/library/functions.html\#str}{\emph{\texttt{str}}}) -- replaces variable name on plot

\item {} 
\textbf{\texttt{time}} (\emph{\texttt{A cdtime object}}) -- replaces time name on plot

\item {} 
\textbf{\texttt{units}} (\href{https://docs.python.org/2/library/functions.html\#str}{\emph{\texttt{str}}}) -- replaces units value on plot

\item {} 
\textbf{\texttt{ymd}} (\href{https://docs.python.org/2/library/functions.html\#str}{\emph{\texttt{str}}}) -- replaces year/month/day on plot

\item {} 
\textbf{\texttt{hms}} (\href{https://docs.python.org/2/library/functions.html\#str}{\emph{\texttt{str}}}) -- replaces hh/mm/ss on plot

\item {} 
\textbf{\texttt{file\_comment}} (\href{https://docs.python.org/2/library/functions.html\#str}{\emph{\texttt{str}}}) -- replaces file\_comment on plot

\item {} 
\textbf{\texttt{xbounds}} (\href{https://docs.python.org/2/library/array.html\#module-array}{\emph{\texttt{array}}}) -- Values to use instead of x axis bounds values

\item {} 
\textbf{\texttt{ybounds}} (\href{https://docs.python.org/2/library/array.html\#module-array}{\emph{\texttt{array}}}) -- Values to use instead of y axis bounds values (if exist)

\item {} 
\textbf{\texttt{xname}} (\href{https://docs.python.org/2/library/functions.html\#str}{\emph{\texttt{str}}}) -- replace xaxis name on plot

\item {} 
\textbf{\texttt{yname}} (\href{https://docs.python.org/2/library/functions.html\#str}{\emph{\texttt{str}}}) -- replace yaxis name on plot (if exists)

\item {} 
\textbf{\texttt{zname}} (\href{https://docs.python.org/2/library/functions.html\#str}{\emph{\texttt{str}}}) -- replace zaxis name on plot (if exists)

\item {} 
\textbf{\texttt{tname}} (\href{https://docs.python.org/2/library/functions.html\#str}{\emph{\texttt{str}}}) -- replace taxis name on plot (if exists)

\item {} 
\textbf{\texttt{wname}} (\href{https://docs.python.org/2/library/functions.html\#str}{\emph{\texttt{str}}}) -- replace waxis name on plot (if exists)

\item {} 
\textbf{\texttt{xunits}} (\href{https://docs.python.org/2/library/functions.html\#str}{\emph{\texttt{str}}}) -- replace xaxis units on plot

\item {} 
\textbf{\texttt{yunits}} (\href{https://docs.python.org/2/library/functions.html\#str}{\emph{\texttt{str}}}) -- replace yaxis units on plot (if exists)

\item {} 
\textbf{\texttt{zunits}} (\href{https://docs.python.org/2/library/functions.html\#str}{\emph{\texttt{str}}}) -- replace zaxis units on plot (if exists)

\item {} 
\textbf{\texttt{tunits}} (\href{https://docs.python.org/2/library/functions.html\#str}{\emph{\texttt{str}}}) -- replace taxis units on plot (if exists)

\item {} 
\textbf{\texttt{wunits}} (\href{https://docs.python.org/2/library/functions.html\#str}{\emph{\texttt{str}}}) -- replace waxis units on plot (if exists)

\item {} 
\textbf{\texttt{xweights}} (\href{https://docs.python.org/2/library/array.html\#module-array}{\emph{\texttt{array}}}) -- replace xaxis weights used for computing mean

\item {} 
\textbf{\texttt{yweights}} (\href{https://docs.python.org/2/library/array.html\#module-array}{\emph{\texttt{array}}}) -- replace xaxis weights used for computing mean

\item {} 
\textbf{\texttt{comment1}} (\href{https://docs.python.org/2/library/functions.html\#str}{\emph{\texttt{str}}}) -- replaces comment1 on plot

\item {} 
\textbf{\texttt{comment2}} (\href{https://docs.python.org/2/library/functions.html\#str}{\emph{\texttt{str}}}) -- replaces comment2 on plot

\item {} 
\textbf{\texttt{comment3}} (\href{https://docs.python.org/2/library/functions.html\#str}{\emph{\texttt{str}}}) -- replaces comment3 on plot

\item {} 
\textbf{\texttt{comment4}} (\href{https://docs.python.org/2/library/functions.html\#str}{\emph{\texttt{str}}}) -- replaces comment4 on plot

\item {} 
\textbf{\texttt{long\_name}} (\href{https://docs.python.org/2/library/functions.html\#str}{\emph{\texttt{str}}}) -- replaces long\_name on plot

\item {} 
\textbf{\texttt{grid}} (\emph{\texttt{cdms2.grid.TransientRectGrid}}) -- replaces array grid (if exists)

\item {} 
\textbf{\texttt{bg}} (\emph{\texttt{bool/int}}) -- plots in background mode

\item {} 
\textbf{\texttt{ratio}} (\index{xmtics1 (vcs.Canvas.Canvas attribute)}\index{xmtics2 (vcs.Canvas.Canvas attribute)}\index{ymtics1 (vcs.Canvas.Canvas attribute)}\index{ymtics2 (vcs.Canvas.Canvas attribute)}\index{xticlabels1 (vcs.Canvas.Canvas attribute)}\index{xticlabels2 (vcs.Canvas.Canvas attribute)}\index{yticlabels1 (vcs.Canvas.Canvas attribute)}\index{yticlabels2 (vcs.Canvas.Canvas attribute)}\index{projection (vcs.Canvas.Canvas attribute)}\index{datawc\_x1 (vcs.Canvas.Canvas attribute)}\index{datawc\_x2 (vcs.Canvas.Canvas attribute)}\index{datawc\_y1 (vcs.Canvas.Canvas attribute)}\index{datawc\_y2 (vcs.Canvas.Canvas attribute)}\index{datawc\_timeunits (vcs.Canvas.Canvas attribute)}\index{datawc\_calendar (vcs.Canvas.Canvas attribute)}) -- sets the y/x ratio ,if passed as a string with `t' at the end, will aslo moves the ticks

\item {} 
\textbf{\texttt{xaxisconvert}} (\href{https://docs.python.org/2/library/functions.html\#str}{\emph{\texttt{str}}}) -- (Ex: `linear') converting xaxis linear/log/log10/ln/exp/area\_wt

\item {} 
\textbf{\texttt{yaxisconvert}} (\href{https://docs.python.org/2/library/functions.html\#str}{\emph{\texttt{str}}}) -- (Ex: `linear') converting yaxis linear/log/log10/ln/exp/area\_wt

\item {} 
\textbf{\texttt{GM\_name}} -- (Ex: `default') retrieve the graphics method object of the given name. If no name is given, then retrieve the `default' graphics method.

\end{itemize}

\item[{Returns}] \leavevmode
The requested isoline VCS object

\item[{Return type}] \leavevmode
{\hyperref[vcs/graphics/isoline:vcs.isoline.Gi]{\sphinxcrossref{vcs.isoline.Gi}}}

\end{description}\end{quote}

\end{fulllineitems}

\index{getline() (vcs.Canvas.Canvas method)}

\begin{fulllineitems}
\phantomsection\label{vcs/Canvas:vcs.Canvas.Canvas.getline}\pysiglinewithargsret{\sphinxbfcode{getline}}{\emph{name='default'}, \emph{ltype=None}, \emph{width=None}, \emph{color=None}, \emph{priority=None}, \emph{viewport=None}, \emph{worldcoordinate=None}, \emph{x=None}, \emph{y=None}}{}
VCS contains a list of secondary methods. This function will create a
line class object from an existing VCS line secondary method. If
no line name is given, then line `default' will be used.

\begin{notice}{note}{Note:}
VCS does not allow the modification of `default' attribute sets.
However, a `default' attribute set that has been copied under a
different name can be modified. (See the {\hyperref[vcs/misc/manageElements:vcs.manageElements.createline]{\sphinxcrossref{\sphinxcode{vcs.manageElements.createline()}}}} function.)
\end{notice}
\begin{quote}\begin{description}
\item[{Example}] \leavevmode
\begin{Verbatim}[commandchars=\\\{\}]
\PYG{g+gp}{\PYGZgt{}\PYGZgt{}\PYGZgt{} }\PYG{n}{a}\PYG{o}{=}\PYG{n}{vcs}\PYG{o}{.}\PYG{n}{init}\PYG{p}{(}\PYG{p}{)}
\PYG{g+gp}{\PYGZgt{}\PYGZgt{}\PYGZgt{} }\PYG{n}{vcs}\PYG{o}{.}\PYG{n}{listelements}\PYG{p}{(}\PYG{l+s+s1}{\PYGZsq{}}\PYG{l+s+s1}{line}\PYG{l+s+s1}{\PYGZsq{}}\PYG{p}{)} \PYG{c+c1}{\PYGZsh{} Show all the existing line secondary methods}
\PYG{g+go}{[...]}
\PYG{g+gp}{\PYGZgt{}\PYGZgt{}\PYGZgt{} }\PYG{n}{ex}\PYG{o}{=}\PYG{n}{vcs}\PYG{o}{.}\PYG{n}{getline}\PYG{p}{(}\PYG{p}{)}  \PYG{c+c1}{\PYGZsh{} instance of \PYGZsq{}default\PYGZsq{} line secondary method}
\PYG{g+gp}{\PYGZgt{}\PYGZgt{}\PYGZgt{} }\PYG{n}{a}\PYG{o}{.}\PYG{n}{line}\PYG{p}{(}\PYG{n}{ex}\PYG{p}{)} \PYG{c+c1}{\PYGZsh{} plot using specified line object}
\PYG{g+go}{\PYGZlt{}vcs.displayplot.Dp ...\PYGZgt{}}
\PYG{g+gp}{\PYGZgt{}\PYGZgt{}\PYGZgt{} }\PYG{n}{ex2}\PYG{o}{=}\PYG{n}{vcs}\PYG{o}{.}\PYG{n}{getline}\PYG{p}{(}\PYG{l+s+s1}{\PYGZsq{}}\PYG{l+s+s1}{red}\PYG{l+s+s1}{\PYGZsq{}}\PYG{p}{)}  \PYG{c+c1}{\PYGZsh{} instance of \PYGZsq{}red\PYGZsq{} line secondary method}
\PYG{g+gp}{\PYGZgt{}\PYGZgt{}\PYGZgt{} }\PYG{n}{a}\PYG{o}{.}\PYG{n}{line}\PYG{p}{(}\PYG{n}{ex2}\PYG{p}{)} \PYG{c+c1}{\PYGZsh{} plot using specified line object}
\PYG{g+go}{\PYGZlt{}vcs.displayplot.Dp ...\PYGZgt{}}
\end{Verbatim}

\item[{Parameters}] \leavevmode\begin{itemize}
\item {} 
\textbf{\texttt{name}} (\href{https://docs.python.org/2/library/functions.html\#str}{\emph{\texttt{str}}}) -- Name of created object

\item {} 
\textbf{\texttt{ltype}} (\href{https://docs.python.org/2/library/functions.html\#str}{\emph{\texttt{str}}}) -- One of ``dash'', ``dash-dot'', ``solid'', ``dot'', or ``long-dash''.

\item {} 
\textbf{\texttt{width}} (\href{https://docs.python.org/2/library/functions.html\#int}{\emph{\texttt{int}}}) -- Thickness of the line to be created

\item {} 
\textbf{\texttt{color}} (\emph{\texttt{str or int}}) -- 
A color name from the \href{https://en.wikipedia.org/wiki/X11\_color\_names}{X11 Color Names list},
or an integer value from 0-255, or an RGB/RGBA tuple/list (e.g. (0,100,0), (100,100,0,50))


\item {} 
\textbf{\texttt{priority}} (\href{https://docs.python.org/2/library/functions.html\#int}{\emph{\texttt{int}}}) -- The layer on which the marker will be drawn.

\item {} 
\textbf{\texttt{viewport}} (\emph{\texttt{list of floats}}) -- 4 floats between 0 and 1. These specify the area that the X/Y values are mapped to inside of the canvas

\item {} 
\textbf{\texttt{worldcoordinate}} (\emph{\texttt{list of floats}}) -- List of 4 floats (xmin, xmax, ymin, ymax)

\item {} 
\textbf{\texttt{x}} (\emph{\texttt{list of floats}}) -- List of lists of x coordinates. Values must be between worldcoordinate{[}0{]} and worldcoordinate{[}1{]}.

\item {} 
\textbf{\texttt{y}} (\emph{\texttt{list of floats}}) -- List of lists of y coordinates. Values must be between worldcoordinate{[}2{]} and worldcoordinate{[}3{]}.

\end{itemize}

\item[{Returns}] \leavevmode
A VCS line object

\item[{Return type}] \leavevmode
{\hyperref[vcs/secondary/line:vcs.line.Tl]{\sphinxcrossref{vcs.line.Tl}}}

\end{description}\end{quote}

\end{fulllineitems}

\index{getmarker() (vcs.Canvas.Canvas method)}

\begin{fulllineitems}
\phantomsection\label{vcs/Canvas:vcs.Canvas.Canvas.getmarker}\pysiglinewithargsret{\sphinxbfcode{getmarker}}{\emph{name='default'}, \emph{mtype=None}, \emph{size=None}, \emph{color=None}, \emph{priority=None}, \emph{viewport=None}, \emph{worldcoordinate=None}, \emph{x=None}, \emph{y=None}}{}
VCS contains a list of secondary methods. This function will create a
marker class object from an existing VCS marker secondary method. If
no marker name is given, then marker `default' will be used.

\begin{notice}{note}{Note:}
VCS does not allow the modification of `default' attribute sets.
However, a `default' attribute set that has been copied under a
different name can be modified. (See the {\hyperref[vcs/misc/manageElements:vcs.manageElements.createmarker]{\sphinxcrossref{\sphinxcode{vcs.manageElements.createmarker()}}}} function.)
\end{notice}
\begin{quote}\begin{description}
\item[{Example}] \leavevmode
\begin{Verbatim}[commandchars=\\\{\}]
\PYG{g+gp}{\PYGZgt{}\PYGZgt{}\PYGZgt{} }\PYG{n}{a}\PYG{o}{=}\PYG{n}{vcs}\PYG{o}{.}\PYG{n}{init}\PYG{p}{(}\PYG{p}{)}
\PYG{g+gp}{\PYGZgt{}\PYGZgt{}\PYGZgt{} }\PYG{n}{vcs}\PYG{o}{.}\PYG{n}{listelements}\PYG{p}{(}\PYG{l+s+s1}{\PYGZsq{}}\PYG{l+s+s1}{marker}\PYG{l+s+s1}{\PYGZsq{}}\PYG{p}{)} \PYG{c+c1}{\PYGZsh{} Show all the existing marker secondary methods}
\PYG{g+go}{[...]}
\PYG{g+gp}{\PYGZgt{}\PYGZgt{}\PYGZgt{} }\PYG{n}{ex}\PYG{o}{=}\PYG{n}{vcs}\PYG{o}{.}\PYG{n}{getmarker}\PYG{p}{(}\PYG{p}{)}  \PYG{c+c1}{\PYGZsh{} instance of \PYGZsq{}default\PYGZsq{} marker secondary method}
\PYG{g+gp}{\PYGZgt{}\PYGZgt{}\PYGZgt{} }\PYG{n}{a}\PYG{o}{.}\PYG{n}{marker}\PYG{p}{(}\PYG{n}{ex}\PYG{p}{)} \PYG{c+c1}{\PYGZsh{} plot using specified marker object}
\PYG{g+go}{\PYGZlt{}vcs.displayplot.Dp ...\PYGZgt{}}
\PYG{g+gp}{\PYGZgt{}\PYGZgt{}\PYGZgt{} }\PYG{n}{ex2}\PYG{o}{=}\PYG{n}{vcs}\PYG{o}{.}\PYG{n}{getmarker}\PYG{p}{(}\PYG{l+s+s1}{\PYGZsq{}}\PYG{l+s+s1}{red}\PYG{l+s+s1}{\PYGZsq{}}\PYG{p}{)}  \PYG{c+c1}{\PYGZsh{} instance of \PYGZsq{}red\PYGZsq{} marker secondary method}
\PYG{g+gp}{\PYGZgt{}\PYGZgt{}\PYGZgt{} }\PYG{n}{a}\PYG{o}{.}\PYG{n}{marker}\PYG{p}{(}\PYG{n}{ex2}\PYG{p}{)} \PYG{c+c1}{\PYGZsh{} plot using specified marker object}
\PYG{g+go}{\PYGZlt{}vcs.displayplot.Dp ...\PYGZgt{}}
\end{Verbatim}

\item[{Parameters}] \leavevmode\begin{itemize}
\item {} 
\textbf{\texttt{name}} (\href{https://docs.python.org/2/library/functions.html\#str}{\emph{\texttt{str}}}) -- Name of created object

\item {} 
\textbf{\texttt{source}} (\href{https://docs.python.org/2/library/functions.html\#str}{\emph{\texttt{str}}}) -- A marker, or string name of a marker

\item {} 
\textbf{\texttt{mtype}} (\href{https://docs.python.org/2/library/functions.html\#str}{\emph{\texttt{str}}}) -- Specifies the type of marker, i.e. ``dot'', ``circle''

\item {} 
\textbf{\texttt{size}} (\href{https://docs.python.org/2/library/functions.html\#int}{\emph{\texttt{int}}}) -- Size of the marker

\item {} 
\textbf{\texttt{color}} (\emph{\texttt{str or int}}) -- 
A color name from the \href{https://en.wikipedia.org/wiki/X11\_color\_names}{X11 Color Names list},
or an integer value from 0-255, or an RGB/RGBA tuple/list (e.g. (0,100,0), (100,100,0,50))


\item {} 
\textbf{\texttt{priority}} (\href{https://docs.python.org/2/library/functions.html\#int}{\emph{\texttt{int}}}) -- The layer on which the marker will be drawn.

\item {} 
\textbf{\texttt{viewport}} (\emph{\texttt{list of floats}}) -- 4 floats between 0 and 1. These specify the area that the X/Y values are mapped to inside of the canvas

\item {} 
\textbf{\texttt{worldcoordinate}} (\emph{\texttt{list of floats}}) -- List of 4 floats (xmin, xmax, ymin, ymax)

\item {} 
\textbf{\texttt{x}} (\emph{\texttt{list of floats}}) -- List of lists of x coordinates. Values must be between worldcoordinate{[}0{]} and worldcoordinate{[}1{]}.

\item {} 
\textbf{\texttt{y}} (\emph{\texttt{list of floats}}) -- List of lists of y coordinates. Values must be between worldcoordinate{[}2{]} and worldcoordinate{[}3{]}.

\end{itemize}

\item[{Returns}] \leavevmode
A marker graphics method object

\item[{Return type}] \leavevmode
{\hyperref[vcs/secondary/marker:vcs.marker.Tm]{\sphinxcrossref{vcs.marker.Tm}}}

\end{description}\end{quote}

\end{fulllineitems}

\index{getmeshfill() (vcs.Canvas.Canvas method)}

\begin{fulllineitems}
\phantomsection\label{vcs/Canvas:vcs.Canvas.Canvas.getmeshfill}\pysiglinewithargsret{\sphinxbfcode{getmeshfill}}{\emph{Gfm\_name\_src='default'}}{}
VCS contains a list of graphics methods. This function will create a
meshfill class object from an existing VCS meshfill graphics method. If
no meshfill name is given, then meshfill `default' will be used.

\begin{notice}{note}{Note:}
VCS does not allow the modification of `default' attribute sets.
However, a `default' attribute set that has been copied under a
different name can be modified. (See the {\hyperref[vcs/misc/manageElements:vcs.manageElements.createmeshfill]{\sphinxcrossref{\sphinxcode{vcs.manageElements.createmeshfill()}}}} function.)
\end{notice}
\begin{quote}\begin{description}
\item[{Example}] \leavevmode
\begin{Verbatim}[commandchars=\\\{\}]
\PYG{g+gp}{\PYGZgt{}\PYGZgt{}\PYGZgt{} }\PYG{n}{a}\PYG{o}{=}\PYG{n}{vcs}\PYG{o}{.}\PYG{n}{init}\PYG{p}{(}\PYG{p}{)}
\PYG{g+gp}{\PYGZgt{}\PYGZgt{}\PYGZgt{} }\PYG{n}{vcs}\PYG{o}{.}\PYG{n}{listelements}\PYG{p}{(}\PYG{l+s+s1}{\PYGZsq{}}\PYG{l+s+s1}{meshfill}\PYG{l+s+s1}{\PYGZsq{}}\PYG{p}{)} \PYG{c+c1}{\PYGZsh{} Show all the existing meshfill graphics methods}
\PYG{g+go}{[...]}
\PYG{g+gp}{\PYGZgt{}\PYGZgt{}\PYGZgt{} }\PYG{n}{ex}\PYG{o}{=}\PYG{n}{vcs}\PYG{o}{.}\PYG{n}{getmeshfill}\PYG{p}{(}\PYG{p}{)}  \PYG{c+c1}{\PYGZsh{} instance of \PYGZsq{}default\PYGZsq{} meshfill graphics method}
\PYG{g+gp}{\PYGZgt{}\PYGZgt{}\PYGZgt{} }\PYG{k+kn}{import} \PYG{n+nn}{cdms2} \PYG{c+c1}{\PYGZsh{} Need cdms2 to create a slab}
\PYG{g+gp}{\PYGZgt{}\PYGZgt{}\PYGZgt{} }\PYG{n}{f} \PYG{o}{=} \PYG{n}{cdms2}\PYG{o}{.}\PYG{n}{open}\PYG{p}{(}\PYG{n}{vcs}\PYG{o}{.}\PYG{n}{sample\PYGZus{}data}\PYG{o}{+}\PYG{l+s+s1}{\PYGZsq{}}\PYG{l+s+s1}{/clt.nc}\PYG{l+s+s1}{\PYGZsq{}}\PYG{p}{)} \PYG{c+c1}{\PYGZsh{} use cdms2 to open a data file}
\PYG{g+gp}{\PYGZgt{}\PYGZgt{}\PYGZgt{} }\PYG{n}{slab1} \PYG{o}{=} \PYG{n}{f}\PYG{p}{(}\PYG{l+s+s1}{\PYGZsq{}}\PYG{l+s+s1}{u}\PYG{l+s+s1}{\PYGZsq{}}\PYG{p}{)} \PYG{c+c1}{\PYGZsh{} use the data file to create a cdms2 slab}
\PYG{g+gp}{\PYGZgt{}\PYGZgt{}\PYGZgt{} }\PYG{n}{a}\PYG{o}{.}\PYG{n}{meshfill}\PYG{p}{(}\PYG{n}{ex}\PYG{p}{,} \PYG{n}{slab1}\PYG{p}{)} \PYG{c+c1}{\PYGZsh{} plot using specified meshfill object}
\PYG{g+go}{\PYGZlt{}vcs.displayplot.Dp ...\PYGZgt{}}
\PYG{g+gp}{\PYGZgt{}\PYGZgt{}\PYGZgt{} }\PYG{n}{ex2}\PYG{o}{=}\PYG{n}{vcs}\PYG{o}{.}\PYG{n}{getmeshfill}\PYG{p}{(}\PYG{l+s+s1}{\PYGZsq{}}\PYG{l+s+s1}{a\PYGZus{}polar\PYGZus{}meshfill}\PYG{l+s+s1}{\PYGZsq{}}\PYG{p}{)}  \PYG{c+c1}{\PYGZsh{} instance of \PYGZsq{}a\PYGZus{}polar\PYGZus{}meshfill\PYGZsq{} meshfill graphics method}
\PYG{g+gp}{\PYGZgt{}\PYGZgt{}\PYGZgt{} }\PYG{n}{a}\PYG{o}{.}\PYG{n}{meshfill}\PYG{p}{(}\PYG{n}{ex2}\PYG{p}{,} \PYG{n}{slab1}\PYG{p}{)} \PYG{c+c1}{\PYGZsh{} plot using specified meshfill object}
\PYG{g+go}{\PYGZlt{}vcs.displayplot.Dp ...\PYGZgt{}}
\end{Verbatim}

\item[{Parameters}] \leavevmode
\textbf{\texttt{Gfm\_name\_src}} (\href{https://docs.python.org/2/library/functions.html\#str}{\emph{\texttt{str}}}) -- String name of an existing meshfill VCS object

\item[{Returns}] \leavevmode
A meshfill VCS object

\item[{Return type}] \leavevmode
{\hyperref[vcs/graphics/meshfill:vcs.meshfill.Gfm]{\sphinxcrossref{vcs.meshfill.Gfm}}}

\end{description}\end{quote}

\end{fulllineitems}

\index{getplot() (vcs.Canvas.Canvas method)}

\begin{fulllineitems}
\phantomsection\label{vcs/Canvas:vcs.Canvas.Canvas.getplot}\pysiglinewithargsret{\sphinxbfcode{getplot}}{\emph{Dp\_name\_src='default'}, \emph{template=None}}{}~
\DUrole{versionmodified}{Deprecated since version 2.0: }The getplot function is deprecated. Do not use it.

This function will create a display plot object from an existing display
plot object from an existing VCS plot. If no display plot name
is given, then None is returned.
\begin{quote}\begin{description}
\item[{Parameters}] \leavevmode\begin{itemize}
\item {} 
\textbf{\texttt{Dp\_name\_src}} (\href{https://docs.python.org/2/library/functions.html\#str}{\emph{\texttt{str}}}) -- String name of an existing display plot object

\item {} 
\textbf{\texttt{template}} -- The displayplot template to inherit from

\end{itemize}

\item[{Returns}] \leavevmode
A VCS displayplot object

\item[{Return type}] \leavevmode
{\hyperref[vcs/misc/displayplot:vcs.displayplot.Dp]{\sphinxcrossref{vcs.displayplot.Dp}}}

\end{description}\end{quote}

\end{fulllineitems}

\index{getprojection() (vcs.Canvas.Canvas method)}

\begin{fulllineitems}
\phantomsection\label{vcs/Canvas:vcs.Canvas.Canvas.getprojection}\pysiglinewithargsret{\sphinxbfcode{getprojection}}{\emph{Proj\_name\_src='default'}}{}
VCS contains a list of graphics methods. This function will create a
projection class object from an existing VCS projection graphics method. If
no projection name is given, then projection `default' will be used.

\begin{notice}{note}{Note:}
VCS does not allow the modification of `default' attribute sets.
However, a `default' attribute set that has been copied under a
different name can be modified. (See the {\hyperref[vcs/misc/manageElements:vcs.manageElements.createprojection]{\sphinxcrossref{\sphinxcode{vcs.manageElements.createprojection()}}}} function.)
\end{notice}
\begin{quote}\begin{description}
\item[{Example}] \leavevmode
\begin{Verbatim}[commandchars=\\\{\}]
\PYG{g+gp}{\PYGZgt{}\PYGZgt{}\PYGZgt{} }\PYG{n}{a}\PYG{o}{=}\PYG{n}{vcs}\PYG{o}{.}\PYG{n}{init}\PYG{p}{(}\PYG{p}{)}
\PYG{g+gp}{\PYGZgt{}\PYGZgt{}\PYGZgt{} }\PYG{n}{vcs}\PYG{o}{.}\PYG{n}{listelements}\PYG{p}{(}\PYG{l+s+s1}{\PYGZsq{}}\PYG{l+s+s1}{projection}\PYG{l+s+s1}{\PYGZsq{}}\PYG{p}{)} \PYG{c+c1}{\PYGZsh{} Show all the existing projection graphics methods}
\PYG{g+go}{[...]}
\PYG{g+gp}{\PYGZgt{}\PYGZgt{}\PYGZgt{} }\PYG{n}{ex}\PYG{o}{=}\PYG{n}{vcs}\PYG{o}{.}\PYG{n}{getprojection}\PYG{p}{(}\PYG{p}{)}  \PYG{c+c1}{\PYGZsh{} instance of \PYGZsq{}default\PYGZsq{} projection graphics method}
\PYG{g+gp}{\PYGZgt{}\PYGZgt{}\PYGZgt{} }\PYG{k+kn}{import} \PYG{n+nn}{cdms2} \PYG{c+c1}{\PYGZsh{} Need cdms2 to create a slab}
\PYG{g+gp}{\PYGZgt{}\PYGZgt{}\PYGZgt{} }\PYG{n}{f} \PYG{o}{=} \PYG{n}{cdms2}\PYG{o}{.}\PYG{n}{open}\PYG{p}{(}\PYG{n}{vcs}\PYG{o}{.}\PYG{n}{sample\PYGZus{}data}\PYG{o}{+}\PYG{l+s+s1}{\PYGZsq{}}\PYG{l+s+s1}{/clt.nc}\PYG{l+s+s1}{\PYGZsq{}}\PYG{p}{)} \PYG{c+c1}{\PYGZsh{} use cdms2 to open a data file}
\PYG{g+gp}{\PYGZgt{}\PYGZgt{}\PYGZgt{} }\PYG{n}{slab1} \PYG{o}{=} \PYG{n}{f}\PYG{p}{(}\PYG{l+s+s1}{\PYGZsq{}}\PYG{l+s+s1}{u}\PYG{l+s+s1}{\PYGZsq{}}\PYG{p}{)} \PYG{c+c1}{\PYGZsh{} use the data file to create a cdms2 slab}
\PYG{g+gp}{\PYGZgt{}\PYGZgt{}\PYGZgt{} }\PYG{n}{a}\PYG{o}{.}\PYG{n}{plot}\PYG{p}{(}\PYG{n}{ex}\PYG{p}{,} \PYG{n}{slab1}\PYG{p}{)} \PYG{c+c1}{\PYGZsh{} plot using specified projection object}
\PYG{g+go}{\PYGZlt{}vcs.displayplot.Dp ...\PYGZgt{}}
\PYG{g+gp}{\PYGZgt{}\PYGZgt{}\PYGZgt{} }\PYG{n}{ex2}\PYG{o}{=}\PYG{n}{vcs}\PYG{o}{.}\PYG{n}{getprojection}\PYG{p}{(}\PYG{l+s+s1}{\PYGZsq{}}\PYG{l+s+s1}{polar}\PYG{l+s+s1}{\PYGZsq{}}\PYG{p}{)}  \PYG{c+c1}{\PYGZsh{} instance of \PYGZsq{}polar\PYGZsq{} projection graphics method}
\PYG{g+gp}{\PYGZgt{}\PYGZgt{}\PYGZgt{} }\PYG{n}{a}\PYG{o}{.}\PYG{n}{plot}\PYG{p}{(}\PYG{n}{ex2}\PYG{p}{,} \PYG{n}{slab1}\PYG{p}{)} \PYG{c+c1}{\PYGZsh{} plot using specified projection object}
\PYG{g+go}{\PYGZlt{}vcs.displayplot.Dp ...\PYGZgt{}}
\end{Verbatim}

\item[{Parameters}] \leavevmode
\textbf{\texttt{Proj\_name\_src}} (\href{https://docs.python.org/2/library/functions.html\#str}{\emph{\texttt{str}}}) -- String name of an existing VCS projection object

\item[{Returns}] \leavevmode
A VCS projection object

\item[{Return type}] \leavevmode
{\hyperref[vcs/misc/projection:vcs.projection.Proj]{\sphinxcrossref{vcs.projection.Proj}}}

\end{description}\end{quote}

\end{fulllineitems}

\index{getscatter() (vcs.Canvas.Canvas method)}

\begin{fulllineitems}
\phantomsection\label{vcs/Canvas:vcs.Canvas.Canvas.getscatter}\pysiglinewithargsret{\sphinxbfcode{getscatter}}{\emph{GSp\_name\_src='default'}}{}
VCS contains a list of graphics methods. This function will create a
scatter class object from an existing VCS scatter graphics method. If
no scatter name is given, then scatter `'{\color{red}\bfseries{}default\_scatter\_}`' will be used.

\begin{notice}{note}{Note:}
VCS does not allow the modification of `default' attribute sets.
However, a `default' attribute set that has been copied under a
different name can be modified. (See the {\hyperref[vcs/misc/manageElements:vcs.manageElements.createscatter]{\sphinxcrossref{\sphinxcode{vcs.manageElements.createscatter()}}}} function.)
\end{notice}
\begin{quote}\begin{description}
\item[{Example}] \leavevmode
\begin{Verbatim}[commandchars=\\\{\}]
\PYG{g+gp}{\PYGZgt{}\PYGZgt{}\PYGZgt{} }\PYG{n}{a}\PYG{o}{=}\PYG{n}{vcs}\PYG{o}{.}\PYG{n}{init}\PYG{p}{(}\PYG{p}{)}
\PYG{g+gp}{\PYGZgt{}\PYGZgt{}\PYGZgt{} }\PYG{n}{vcs}\PYG{o}{.}\PYG{n}{listelements}\PYG{p}{(}\PYG{l+s+s1}{\PYGZsq{}}\PYG{l+s+s1}{scatter}\PYG{l+s+s1}{\PYGZsq{}}\PYG{p}{)} \PYG{c+c1}{\PYGZsh{} Show all the existing scatter graphics methods}
\PYG{g+go}{[...]}
\PYG{g+gp}{\PYGZgt{}\PYGZgt{}\PYGZgt{} }\PYG{n}{ex}\PYG{o}{=}\PYG{n}{vcs}\PYG{o}{.}\PYG{n}{getscatter}\PYG{p}{(}\PYG{l+s+s1}{\PYGZsq{}}\PYG{l+s+s1}{default\PYGZus{}scatter\PYGZus{}}\PYG{l+s+s1}{\PYGZsq{}}\PYG{p}{)}  \PYG{c+c1}{\PYGZsh{} instance of \PYGZsq{}\PYGZsq{}default\PYGZus{}scatter\PYGZus{}\PYGZsq{}\PYGZsq{} scatter graphics method}
\PYG{g+gp}{\PYGZgt{}\PYGZgt{}\PYGZgt{} }\PYG{k+kn}{import} \PYG{n+nn}{cdms2} \PYG{c+c1}{\PYGZsh{} Need cdms2 to create a slab}
\PYG{g+gp}{\PYGZgt{}\PYGZgt{}\PYGZgt{} }\PYG{n}{f} \PYG{o}{=} \PYG{n}{cdms2}\PYG{o}{.}\PYG{n}{open}\PYG{p}{(}\PYG{n}{vcs}\PYG{o}{.}\PYG{n}{sample\PYGZus{}data}\PYG{o}{+}\PYG{l+s+s1}{\PYGZsq{}}\PYG{l+s+s1}{/clt.nc}\PYG{l+s+s1}{\PYGZsq{}}\PYG{p}{)} \PYG{c+c1}{\PYGZsh{} use cdms2 to open a data file}
\PYG{g+gp}{\PYGZgt{}\PYGZgt{}\PYGZgt{} }\PYG{n}{slab1} \PYG{o}{=} \PYG{n}{f}\PYG{p}{(}\PYG{l+s+s1}{\PYGZsq{}}\PYG{l+s+s1}{u}\PYG{l+s+s1}{\PYGZsq{}}\PYG{p}{)} \PYG{c+c1}{\PYGZsh{} use the data file to create a cdms2 slab}
\PYG{g+gp}{\PYGZgt{}\PYGZgt{}\PYGZgt{} }\PYG{n}{slab2} \PYG{o}{=} \PYG{n}{f}\PYG{p}{(}\PYG{l+s+s1}{\PYGZsq{}}\PYG{l+s+s1}{v}\PYG{l+s+s1}{\PYGZsq{}}\PYG{p}{)} \PYG{c+c1}{\PYGZsh{} need 2 slabs, so get another}
\PYG{g+gp}{\PYGZgt{}\PYGZgt{}\PYGZgt{} }\PYG{n}{a}\PYG{o}{.}\PYG{n}{scatter}\PYG{p}{(}\PYG{n}{ex}\PYG{p}{,} \PYG{n}{slab1}\PYG{p}{,} \PYG{n}{slab2}\PYG{p}{)} \PYG{c+c1}{\PYGZsh{} plot using specified scatter object}
\PYG{g+go}{\PYGZlt{}vcs.displayplot.Dp ...\PYGZgt{}}
\end{Verbatim}

\item[{Parameters}] \leavevmode\begin{itemize}
\item {} 
\textbf{\texttt{GSp\_name\_src}} (\href{https://docs.python.org/2/library/functions.html\#str}{\emph{\texttt{str}}}) -- String name of an existing scatter VCS object.

\item {} 
\textbf{\texttt{xaxis}} (\emph{\texttt{cdms2.axis.TransientAxis}}) -- Axis object to replace the slab -1 dim axis

\item {} 
\textbf{\texttt{yaxis}} (\emph{\texttt{cdms2.axis.TransientAxis}}) -- Axis object to replace the slab -2 dim axis, only if slab has more than 1D

\item {} 
\textbf{\texttt{zaxis}} (\emph{\texttt{cdms2.axis.TransientAxis}}) -- Axis object to replace the slab -3 dim axis, only if slab has more than 2D

\item {} 
\textbf{\texttt{taxis}} (\emph{\texttt{cdms2.axis.TransientAxis}}) -- Axis object to replace the slab -4 dim axis, only if slab has more than 3D

\item {} 
\textbf{\texttt{waxis}} (\emph{\texttt{cdms2.axis.TransientAxis}}) -- Axis object to replace the slab -5 dim axis, only if slab has more than 4D

\item {} 
\textbf{\texttt{xrev}} (\href{https://docs.python.org/2/library/functions.html\#bool}{\emph{\texttt{bool}}}) -- reverse x axis

\item {} 
\textbf{\texttt{yrev}} (\href{https://docs.python.org/2/library/functions.html\#bool}{\emph{\texttt{bool}}}) -- reverse y axis, only if slab has more than 1D

\item {} 
\textbf{\texttt{xarray}} (\href{https://docs.python.org/2/library/array.html\#module-array}{\emph{\texttt{array}}}) -- Values to use instead of x axis

\item {} 
\textbf{\texttt{yarray}} (\href{https://docs.python.org/2/library/array.html\#module-array}{\emph{\texttt{array}}}) -- Values to use instead of y axis, only if var has more than 1D

\item {} 
\textbf{\texttt{zarray}} (\href{https://docs.python.org/2/library/array.html\#module-array}{\emph{\texttt{array}}}) -- Values to use instead of z axis, only if var has more than 2D

\item {} 
\textbf{\texttt{tarray}} (\href{https://docs.python.org/2/library/array.html\#module-array}{\emph{\texttt{array}}}) -- Values to use instead of t axis, only if var has more than 3D

\item {} 
\textbf{\texttt{warray}} (\href{https://docs.python.org/2/library/array.html\#module-array}{\emph{\texttt{array}}}) -- Values to use instead of w axis, only if var has more than 4D

\item {} 
\textbf{\texttt{continents}} (\href{https://docs.python.org/2/library/functions.html\#int}{\emph{\texttt{int}}}) -- continents type number

\item {} 
\textbf{\texttt{name}} (\href{https://docs.python.org/2/library/functions.html\#str}{\emph{\texttt{str}}}) -- replaces variable name on plot

\item {} 
\textbf{\texttt{time}} (\emph{\texttt{A cdtime object}}) -- replaces time name on plot

\item {} 
\textbf{\texttt{units}} (\href{https://docs.python.org/2/library/functions.html\#str}{\emph{\texttt{str}}}) -- replaces units value on plot

\item {} 
\textbf{\texttt{ymd}} (\href{https://docs.python.org/2/library/functions.html\#str}{\emph{\texttt{str}}}) -- replaces year/month/day on plot

\item {} 
\textbf{\texttt{hms}} (\href{https://docs.python.org/2/library/functions.html\#str}{\emph{\texttt{str}}}) -- replaces hh/mm/ss on plot

\item {} 
\textbf{\texttt{file\_comment}} (\href{https://docs.python.org/2/library/functions.html\#str}{\emph{\texttt{str}}}) -- replaces file\_comment on plot

\item {} 
\textbf{\texttt{xbounds}} (\href{https://docs.python.org/2/library/array.html\#module-array}{\emph{\texttt{array}}}) -- Values to use instead of x axis bounds values

\item {} 
\textbf{\texttt{ybounds}} (\href{https://docs.python.org/2/library/array.html\#module-array}{\emph{\texttt{array}}}) -- Values to use instead of y axis bounds values (if exist)

\item {} 
\textbf{\texttt{xname}} (\href{https://docs.python.org/2/library/functions.html\#str}{\emph{\texttt{str}}}) -- replace xaxis name on plot

\item {} 
\textbf{\texttt{yname}} (\href{https://docs.python.org/2/library/functions.html\#str}{\emph{\texttt{str}}}) -- replace yaxis name on plot (if exists)

\item {} 
\textbf{\texttt{zname}} (\href{https://docs.python.org/2/library/functions.html\#str}{\emph{\texttt{str}}}) -- replace zaxis name on plot (if exists)

\item {} 
\textbf{\texttt{tname}} (\href{https://docs.python.org/2/library/functions.html\#str}{\emph{\texttt{str}}}) -- replace taxis name on plot (if exists)

\item {} 
\textbf{\texttt{wname}} (\href{https://docs.python.org/2/library/functions.html\#str}{\emph{\texttt{str}}}) -- replace waxis name on plot (if exists)

\item {} 
\textbf{\texttt{xunits}} (\href{https://docs.python.org/2/library/functions.html\#str}{\emph{\texttt{str}}}) -- replace xaxis units on plot

\item {} 
\textbf{\texttt{yunits}} (\href{https://docs.python.org/2/library/functions.html\#str}{\emph{\texttt{str}}}) -- replace yaxis units on plot (if exists)

\item {} 
\textbf{\texttt{zunits}} (\href{https://docs.python.org/2/library/functions.html\#str}{\emph{\texttt{str}}}) -- replace zaxis units on plot (if exists)

\item {} 
\textbf{\texttt{tunits}} (\href{https://docs.python.org/2/library/functions.html\#str}{\emph{\texttt{str}}}) -- replace taxis units on plot (if exists)

\item {} 
\textbf{\texttt{wunits}} (\href{https://docs.python.org/2/library/functions.html\#str}{\emph{\texttt{str}}}) -- replace waxis units on plot (if exists)

\item {} 
\textbf{\texttt{xweights}} (\href{https://docs.python.org/2/library/array.html\#module-array}{\emph{\texttt{array}}}) -- replace xaxis weights used for computing mean

\item {} 
\textbf{\texttt{yweights}} (\href{https://docs.python.org/2/library/array.html\#module-array}{\emph{\texttt{array}}}) -- replace xaxis weights used for computing mean

\item {} 
\textbf{\texttt{comment1}} (\href{https://docs.python.org/2/library/functions.html\#str}{\emph{\texttt{str}}}) -- replaces comment1 on plot

\item {} 
\textbf{\texttt{comment2}} (\href{https://docs.python.org/2/library/functions.html\#str}{\emph{\texttt{str}}}) -- replaces comment2 on plot

\item {} 
\textbf{\texttt{comment3}} (\href{https://docs.python.org/2/library/functions.html\#str}{\emph{\texttt{str}}}) -- replaces comment3 on plot

\item {} 
\textbf{\texttt{comment4}} (\href{https://docs.python.org/2/library/functions.html\#str}{\emph{\texttt{str}}}) -- replaces comment4 on plot

\item {} 
\textbf{\texttt{long\_name}} (\href{https://docs.python.org/2/library/functions.html\#str}{\emph{\texttt{str}}}) -- replaces long\_name on plot

\item {} 
\textbf{\texttt{grid}} (\emph{\texttt{cdms2.grid.TransientRectGrid}}) -- replaces array grid (if exists)

\item {} 
\textbf{\texttt{bg}} (\emph{\texttt{bool/int}}) -- plots in background mode

\item {} 
\textbf{\texttt{ratio}} (\index{xmtics1 (vcs.Canvas.Canvas attribute)}\index{xmtics2 (vcs.Canvas.Canvas attribute)}\index{ymtics1 (vcs.Canvas.Canvas attribute)}\index{ymtics2 (vcs.Canvas.Canvas attribute)}\index{xticlabels1 (vcs.Canvas.Canvas attribute)}\index{xticlabels2 (vcs.Canvas.Canvas attribute)}\index{yticlabels1 (vcs.Canvas.Canvas attribute)}\index{yticlabels2 (vcs.Canvas.Canvas attribute)}\index{projection (vcs.Canvas.Canvas attribute)}\index{datawc\_x1 (vcs.Canvas.Canvas attribute)}\index{datawc\_x2 (vcs.Canvas.Canvas attribute)}\index{datawc\_y1 (vcs.Canvas.Canvas attribute)}\index{datawc\_y2 (vcs.Canvas.Canvas attribute)}\index{datawc\_timeunits (vcs.Canvas.Canvas attribute)}\index{datawc\_calendar (vcs.Canvas.Canvas attribute)}) -- sets the y/x ratio ,if passed as a string with `t' at the end, will aslo moves the ticks

\item {} 
\textbf{\texttt{xaxisconvert}} (\href{https://docs.python.org/2/library/functions.html\#str}{\emph{\texttt{str}}}) -- (Ex: `linear') converting xaxis linear/log/log10/ln/exp/area\_wt

\item {} 
\textbf{\texttt{yaxisconvert}} (\href{https://docs.python.org/2/library/functions.html\#str}{\emph{\texttt{str}}}) -- (Ex: `linear') converting yaxis linear/log/log10/ln/exp/area\_wt

\item {} 
\textbf{\texttt{GM\_name}} -- (Ex: `default') retrieve the graphics method object of the given name. If no name is given, then retrieve the `default' graphics method.

\end{itemize}

\item[{Returns}] \leavevmode
A scatter graphics method object

\item[{Return type}] \leavevmode
{\hyperref[vcs/graphics/unified1D:vcs.unified1D.G1d]{\sphinxcrossref{vcs.unified1D.G1d}}}

\end{description}\end{quote}

\end{fulllineitems}

\index{gettaylordiagram() (vcs.Canvas.Canvas method)}

\begin{fulllineitems}
\phantomsection\label{vcs/Canvas:vcs.Canvas.Canvas.gettaylordiagram}\pysiglinewithargsret{\sphinxbfcode{gettaylordiagram}}{\emph{Gtd\_name\_src='default'}}{}
VCS contains a list of graphics methods. This function will create a
taylordiagram class object from an existing VCS taylordiagram graphics method. If
no taylordiagram name is given, then taylordiagram `default' will be used.

\begin{notice}{note}{Note:}
VCS does not allow the modification of `default' attribute sets.
However, a `default' attribute set that has been copied under a
different name can be modified. (See the {\hyperref[vcs/misc/manageElements:vcs.manageElements.createtaylordiagram]{\sphinxcrossref{\sphinxcode{vcs.manageElements.createtaylordiagram()}}}} function.)
\end{notice}
\begin{quote}\begin{description}
\item[{Example}] \leavevmode
\begin{Verbatim}[commandchars=\\\{\}]
\PYG{g+gp}{\PYGZgt{}\PYGZgt{}\PYGZgt{} }\PYG{n}{a}\PYG{o}{=}\PYG{n}{vcs}\PYG{o}{.}\PYG{n}{init}\PYG{p}{(}\PYG{p}{)}
\PYG{g+gp}{\PYGZgt{}\PYGZgt{}\PYGZgt{} }\PYG{n}{vcs}\PYG{o}{.}\PYG{n}{listelements}\PYG{p}{(}\PYG{l+s+s1}{\PYGZsq{}}\PYG{l+s+s1}{taylordiagram}\PYG{l+s+s1}{\PYGZsq{}}\PYG{p}{)} \PYG{c+c1}{\PYGZsh{} Show all the existing taylordiagram graphics methods}
\PYG{g+go}{[...]}
\PYG{g+gp}{\PYGZgt{}\PYGZgt{}\PYGZgt{} }\PYG{n}{ex}\PYG{o}{=}\PYG{n}{vcs}\PYG{o}{.}\PYG{n}{gettaylordiagram}\PYG{p}{(}\PYG{p}{)}  \PYG{c+c1}{\PYGZsh{} instance of \PYGZsq{}default\PYGZsq{} taylordiagram graphics method}
\PYG{g+gp}{\PYGZgt{}\PYGZgt{}\PYGZgt{} }\PYG{k+kn}{import} \PYG{n+nn}{cdms2} \PYG{c+c1}{\PYGZsh{} Need cdms2 to create a slab}
\PYG{g+gp}{\PYGZgt{}\PYGZgt{}\PYGZgt{} }\PYG{n}{f} \PYG{o}{=} \PYG{n}{cdms2}\PYG{o}{.}\PYG{n}{open}\PYG{p}{(}\PYG{n}{vcs}\PYG{o}{.}\PYG{n}{sample\PYGZus{}data}\PYG{o}{+}\PYG{l+s+s1}{\PYGZsq{}}\PYG{l+s+s1}{/clt.nc}\PYG{l+s+s1}{\PYGZsq{}}\PYG{p}{)} \PYG{c+c1}{\PYGZsh{} use cdms2 to open a data file}
\PYG{g+gp}{\PYGZgt{}\PYGZgt{}\PYGZgt{} }\PYG{n}{slab1} \PYG{o}{=} \PYG{n}{f}\PYG{p}{(}\PYG{l+s+s1}{\PYGZsq{}}\PYG{l+s+s1}{u}\PYG{l+s+s1}{\PYGZsq{}}\PYG{p}{)} \PYG{c+c1}{\PYGZsh{} use the data file to create a cdms2 slab}
\PYG{g+gp}{\PYGZgt{}\PYGZgt{}\PYGZgt{} }\PYG{n}{a}\PYG{o}{.}\PYG{n}{taylordiagram}\PYG{p}{(}\PYG{n}{ex}\PYG{p}{,} \PYG{n}{slab1}\PYG{p}{)} \PYG{c+c1}{\PYGZsh{} plot using specified taylordiagram object}
\PYG{g+go}{\PYGZlt{}vcs.displayplot.Dp ...\PYGZgt{}}
\end{Verbatim}

\item[{Parameters}] \leavevmode
\textbf{\texttt{Gtd\_name\_src}} (\href{https://docs.python.org/2/library/functions.html\#str}{\emph{\texttt{str}}}) -- String name of an existing taylordiagram VCS object

\item[{Returns}] \leavevmode
A taylordiagram VCS object

\item[{Return type}] \leavevmode
{\hyperref[vcs/graphics/taylor:vcs.taylor.Gtd]{\sphinxcrossref{vcs.taylor.Gtd}}}

\end{description}\end{quote}

\end{fulllineitems}

\index{gettemplate() (vcs.Canvas.Canvas method)}

\begin{fulllineitems}
\phantomsection\label{vcs/Canvas:vcs.Canvas.Canvas.gettemplate}\pysiglinewithargsret{\sphinxbfcode{gettemplate}}{\emph{Pt\_name\_src='default'}}{}
VCS contains a list of graphics methods. This function will create a
template class object from an existing VCS template graphics method. If
no template name is given, then template `default' will be used.

\begin{notice}{note}{Note:}
VCS does not allow the modification of `default' attribute sets.
However, a `default' attribute set that has been copied under a
different name can be modified. (See the {\hyperref[vcs/misc/manageElements:vcs.manageElements.createtemplate]{\sphinxcrossref{\sphinxcode{vcs.manageElements.createtemplate()}}}} function.)
\end{notice}
\begin{quote}\begin{description}
\item[{Example}] \leavevmode
\begin{Verbatim}[commandchars=\\\{\}]
\PYG{g+gp}{\PYGZgt{}\PYGZgt{}\PYGZgt{} }\PYG{n}{a}\PYG{o}{=}\PYG{n}{vcs}\PYG{o}{.}\PYG{n}{init}\PYG{p}{(}\PYG{p}{)}
\PYG{g+gp}{\PYGZgt{}\PYGZgt{}\PYGZgt{} }\PYG{n}{vcs}\PYG{o}{.}\PYG{n}{listelements}\PYG{p}{(}\PYG{l+s+s1}{\PYGZsq{}}\PYG{l+s+s1}{template}\PYG{l+s+s1}{\PYGZsq{}}\PYG{p}{)} \PYG{c+c1}{\PYGZsh{} Show all the existing template graphics methods}
\PYG{g+go}{[...]}
\PYG{g+gp}{\PYGZgt{}\PYGZgt{}\PYGZgt{} }\PYG{n}{ex}\PYG{o}{=}\PYG{n}{vcs}\PYG{o}{.}\PYG{n}{gettemplate}\PYG{p}{(}\PYG{p}{)}  \PYG{c+c1}{\PYGZsh{} instance of \PYGZsq{}default\PYGZsq{} template graphics method}
\PYG{g+gp}{\PYGZgt{}\PYGZgt{}\PYGZgt{} }\PYG{k+kn}{import} \PYG{n+nn}{cdms2} \PYG{c+c1}{\PYGZsh{} Need cdms2 to create a slab}
\PYG{g+gp}{\PYGZgt{}\PYGZgt{}\PYGZgt{} }\PYG{n}{f} \PYG{o}{=} \PYG{n}{cdms2}\PYG{o}{.}\PYG{n}{open}\PYG{p}{(}\PYG{n}{vcs}\PYG{o}{.}\PYG{n}{sample\PYGZus{}data}\PYG{o}{+}\PYG{l+s+s1}{\PYGZsq{}}\PYG{l+s+s1}{/clt.nc}\PYG{l+s+s1}{\PYGZsq{}}\PYG{p}{)} \PYG{c+c1}{\PYGZsh{} use cdms2 to open a data file}
\PYG{g+gp}{\PYGZgt{}\PYGZgt{}\PYGZgt{} }\PYG{n}{slab1} \PYG{o}{=} \PYG{n}{f}\PYG{p}{(}\PYG{l+s+s1}{\PYGZsq{}}\PYG{l+s+s1}{u}\PYG{l+s+s1}{\PYGZsq{}}\PYG{p}{)} \PYG{c+c1}{\PYGZsh{} use the data file to create a cdms2 slab}
\PYG{g+gp}{\PYGZgt{}\PYGZgt{}\PYGZgt{} }\PYG{n}{a}\PYG{o}{.}\PYG{n}{plot}\PYG{p}{(}\PYG{n}{ex}\PYG{p}{,} \PYG{n}{slab1}\PYG{p}{)} \PYG{c+c1}{\PYGZsh{} plot using specified template object}
\PYG{g+go}{\PYGZlt{}vcs.displayplot.Dp ...\PYGZgt{}}
\PYG{g+gp}{\PYGZgt{}\PYGZgt{}\PYGZgt{} }\PYG{n}{ex2}\PYG{o}{=}\PYG{n}{vcs}\PYG{o}{.}\PYG{n}{gettemplate}\PYG{p}{(}\PYG{l+s+s1}{\PYGZsq{}}\PYG{l+s+s1}{polar}\PYG{l+s+s1}{\PYGZsq{}}\PYG{p}{)}  \PYG{c+c1}{\PYGZsh{} instance of \PYGZsq{}polar\PYGZsq{} template graphics method}
\PYG{g+gp}{\PYGZgt{}\PYGZgt{}\PYGZgt{} }\PYG{n}{a}\PYG{o}{.}\PYG{n}{plot}\PYG{p}{(}\PYG{n}{ex2}\PYG{p}{,} \PYG{n}{slab1}\PYG{p}{)} \PYG{c+c1}{\PYGZsh{} plot using specified template object}
\PYG{g+go}{\PYGZlt{}vcs.displayplot.Dp ...\PYGZgt{}}
\end{Verbatim}

\item[{Parameters}] \leavevmode
\textbf{\texttt{Pt\_name\_src}} -- String name of an existing template VCS object

\item[{Returns}] \leavevmode
A VCS template object

\item[{Return type}] \leavevmode
{\hyperref[vcs/template/template:vcs.template.P]{\sphinxcrossref{vcs.template.P}}}

\end{description}\end{quote}

\end{fulllineitems}

\index{gettext() (vcs.Canvas.Canvas method)}

\begin{fulllineitems}
\phantomsection\label{vcs/Canvas:vcs.Canvas.Canvas.gettext}\pysiglinewithargsret{\sphinxbfcode{gettext}}{\emph{Tt\_name\_src='default'}, \emph{To\_name\_src=None}, \emph{string=None}, \emph{font=None}, \emph{spacing=None}, \emph{expansion=None}, \emph{color=None}, \emph{priority=None}, \emph{viewport=None}, \emph{worldcoordinate=None}, \emph{x=None}, \emph{y=None}, \emph{height=None}, \emph{angle=None}, \emph{path=None}, \emph{halign=None}, \emph{valign=None}}{}
VCS contains a list of secondary methods. This function will create a
textcombined class object from an existing VCS textcombined secondary method. If
no textcombined name is given, then textcombined `EXAMPLE\_tt:::EXAMPLE\_tto' will be used.

\begin{notice}{note}{Note:}
VCS does not allow the modification of `default' attribute sets.
However, a `default' attribute set that has been copied under a
different name can be modified. (See the {\hyperref[vcs/misc/manageElements:vcs.manageElements.createtextcombined]{\sphinxcrossref{\sphinxcode{vcs.manageElements.createtextcombined()}}}} function.)
\end{notice}
\begin{quote}\begin{description}
\item[{Example}] \leavevmode
\begin{Verbatim}[commandchars=\\\{\}]
\PYG{g+gp}{\PYGZgt{}\PYGZgt{}\PYGZgt{} }\PYG{n}{a}\PYG{o}{=}\PYG{n}{vcs}\PYG{o}{.}\PYG{n}{init}\PYG{p}{(}\PYG{p}{)}
\PYG{g+gp}{\PYGZgt{}\PYGZgt{}\PYGZgt{} }\PYG{n}{vcs}\PYG{o}{.}\PYG{n}{listelements}\PYG{p}{(}\PYG{l+s+s1}{\PYGZsq{}}\PYG{l+s+s1}{textcombined}\PYG{l+s+s1}{\PYGZsq{}}\PYG{p}{)} \PYG{c+c1}{\PYGZsh{} Show all the existing textcombined secondary methods}
\PYG{g+go}{[...]}
\PYG{g+gp}{\PYGZgt{}\PYGZgt{}\PYGZgt{} }\PYG{n}{a}\PYG{o}{.}\PYG{n}{createtextcombined}\PYG{p}{(}\PYG{l+s+s1}{\PYGZsq{}}\PYG{l+s+s1}{EXAMPLE\PYGZus{}tt}\PYG{l+s+s1}{\PYGZsq{}}\PYG{p}{,} \PYG{l+s+s1}{\PYGZsq{}}\PYG{l+s+s1}{qa}\PYG{l+s+s1}{\PYGZsq{}}\PYG{p}{,} \PYG{l+s+s1}{\PYGZsq{}}\PYG{l+s+s1}{EXAMPLE\PYGZus{}tto}\PYG{l+s+s1}{\PYGZsq{}}\PYG{p}{,} \PYG{l+s+s1}{\PYGZsq{}}\PYG{l+s+s1}{7left}\PYG{l+s+s1}{\PYGZsq{}}\PYG{p}{)} \PYG{c+c1}{\PYGZsh{} Create \PYGZsq{}EXAMPLE\PYGZus{}tt\PYGZsq{} and \PYGZsq{}EXAMPLE\PYGZus{}tto\PYGZsq{}}
\PYG{g+go}{\PYGZlt{}vcs.textcombined.Tc ...\PYGZgt{}}
\PYG{g+gp}{\PYGZgt{}\PYGZgt{}\PYGZgt{} }\PYG{n}{ex}\PYG{o}{=}\PYG{n}{vcs}\PYG{o}{.}\PYG{n}{gettextcombined}\PYG{p}{(}\PYG{l+s+s1}{\PYGZsq{}}\PYG{l+s+s1}{EXAMPLE\PYGZus{}tt}\PYG{l+s+s1}{\PYGZsq{}}\PYG{p}{,} \PYG{l+s+s1}{\PYGZsq{}}\PYG{l+s+s1}{EXAMPLE\PYGZus{}tto}\PYG{l+s+s1}{\PYGZsq{}}\PYG{p}{)}  \PYG{c+c1}{\PYGZsh{} instance of \PYGZsq{}EXAMPLE\PYGZus{}tt:::EXAMPLE\PYGZus{}tto\PYGZsq{} textcombined secondary method}
\PYG{g+gp}{\PYGZgt{}\PYGZgt{}\PYGZgt{} }\PYG{n}{a}\PYG{o}{.}\PYG{n}{textcombined}\PYG{p}{(}\PYG{n}{ex}\PYG{p}{)} \PYG{c+c1}{\PYGZsh{} plot using specified textcombined object}
\PYG{g+go}{\PYGZlt{}vcs.displayplot.Dp ...\PYGZgt{}}
\end{Verbatim}

\item[{Parameters}] \leavevmode\begin{itemize}
\item {} 
\textbf{\texttt{Tt\_name\_src}} (\href{https://docs.python.org/2/library/functions.html\#str}{\emph{\texttt{str}}}) -- Name of created object

\item {} 
\textbf{\texttt{To\_name\_src}} (\href{https://docs.python.org/2/library/functions.html\#str}{\emph{\texttt{str}}}) -- Name of parent textorientation object

\item {} 
\textbf{\texttt{string}} -- Text to render

\item {} 
\textbf{\texttt{string}} -- list of str

\item {} 
\textbf{\texttt{font}} (\emph{\texttt{int or str}}) -- Which font to use (index or name)

\item {} 
\textbf{\texttt{spacing}} (\emph{\texttt{DEPRECATED}}) -- DEPRECATED

\item {} 
\textbf{\texttt{expansion}} (\emph{\texttt{DEPRECATED}}) -- DEPRECATED

\item {} 
\textbf{\texttt{color}} (\emph{\texttt{str or int}}) -- 
A color name from the \href{https://en.wikipedia.org/wiki/X11\_color\_names}{X11 Color Names list},
or an integer value from 0-255, or an RGB/RGBA tuple/list (e.g. (0,100,0), (100,100,0,50))


\item {} 
\textbf{\texttt{priority}} (\href{https://docs.python.org/2/library/functions.html\#int}{\emph{\texttt{int}}}) -- The layer on which the object will be drawn.

\item {} 
\textbf{\texttt{viewport}} (\emph{\texttt{list of floats}}) -- 4 floats between 0 and 1. These specify the area that the X/Y values are mapped to inside of the canvas

\item {} 
\textbf{\texttt{worldcoordinate}} (\emph{\texttt{list of floats}}) -- List of 4 floats (xmin, xmax, ymin, ymax)

\item {} 
\textbf{\texttt{x}} (\emph{\texttt{list of floats}}) -- List of lists of x coordinates. Values must be between worldcoordinate{[}0{]} and worldcoordinate{[}1{]}.

\item {} 
\textbf{\texttt{y}} (\emph{\texttt{list of floats}}) -- List of lists of y coordinates. Values must be between worldcoordinate{[}2{]} and worldcoordinate{[}3{]}.

\item {} 
\textbf{\texttt{height}} (\href{https://docs.python.org/2/library/functions.html\#int}{\emph{\texttt{int}}}) -- Size of the font

\item {} 
\textbf{\texttt{angle}} (\emph{\texttt{list of int}}) -- Angle of the rendered text, in degrees

\item {} 
\textbf{\texttt{path}} (\emph{\texttt{DEPRECATED}}) -- DEPRECATED

\item {} 
\textbf{\texttt{halign}} (\href{https://docs.python.org/2/library/functions.html\#str}{\emph{\texttt{str}}}) -- Horizontal alignment of the text. One of {[}''left'', ``center'', ``right''{]}

\item {} 
\textbf{\texttt{valign}} (\href{https://docs.python.org/2/library/functions.html\#str}{\emph{\texttt{str}}}) -- Vertical alignment of the text. One of {[}''top'', ``center'', ``bottom''{]}

\end{itemize}

\item[{Returns}] \leavevmode
A textcombined object

\item[{Return type}] \leavevmode
{\hyperref[vcs/secondary/textcombined:vcs.textcombined.Tc]{\sphinxcrossref{vcs.textcombined.Tc}}}

\end{description}\end{quote}

\end{fulllineitems}

\index{gettextcombined() (vcs.Canvas.Canvas method)}

\begin{fulllineitems}
\phantomsection\label{vcs/Canvas:vcs.Canvas.Canvas.gettextcombined}\pysiglinewithargsret{\sphinxbfcode{gettextcombined}}{\emph{Tt\_name\_src='default'}, \emph{To\_name\_src=None}, \emph{string=None}, \emph{font=None}, \emph{spacing=None}, \emph{expansion=None}, \emph{color=None}, \emph{priority=None}, \emph{viewport=None}, \emph{worldcoordinate=None}, \emph{x=None}, \emph{y=None}, \emph{height=None}, \emph{angle=None}, \emph{path=None}, \emph{halign=None}, \emph{valign=None}}{}
VCS contains a list of secondary methods. This function will create a
textcombined class object from an existing VCS textcombined secondary method. If
no textcombined name is given, then textcombined `EXAMPLE\_tt:::EXAMPLE\_tto' will be used.

\begin{notice}{note}{Note:}
VCS does not allow the modification of `default' attribute sets.
However, a `default' attribute set that has been copied under a
different name can be modified. (See the {\hyperref[vcs/misc/manageElements:vcs.manageElements.createtextcombined]{\sphinxcrossref{\sphinxcode{vcs.manageElements.createtextcombined()}}}} function.)
\end{notice}
\begin{quote}\begin{description}
\item[{Example}] \leavevmode
\begin{Verbatim}[commandchars=\\\{\}]
\PYG{g+gp}{\PYGZgt{}\PYGZgt{}\PYGZgt{} }\PYG{n}{a}\PYG{o}{=}\PYG{n}{vcs}\PYG{o}{.}\PYG{n}{init}\PYG{p}{(}\PYG{p}{)}
\PYG{g+gp}{\PYGZgt{}\PYGZgt{}\PYGZgt{} }\PYG{n}{vcs}\PYG{o}{.}\PYG{n}{listelements}\PYG{p}{(}\PYG{l+s+s1}{\PYGZsq{}}\PYG{l+s+s1}{textcombined}\PYG{l+s+s1}{\PYGZsq{}}\PYG{p}{)} \PYG{c+c1}{\PYGZsh{} Show all the existing textcombined secondary methods}
\PYG{g+go}{[...]}
\PYG{g+gp}{\PYGZgt{}\PYGZgt{}\PYGZgt{} }\PYG{n}{a}\PYG{o}{.}\PYG{n}{createtextcombined}\PYG{p}{(}\PYG{l+s+s1}{\PYGZsq{}}\PYG{l+s+s1}{EXAMPLE\PYGZus{}tt}\PYG{l+s+s1}{\PYGZsq{}}\PYG{p}{,} \PYG{l+s+s1}{\PYGZsq{}}\PYG{l+s+s1}{qa}\PYG{l+s+s1}{\PYGZsq{}}\PYG{p}{,} \PYG{l+s+s1}{\PYGZsq{}}\PYG{l+s+s1}{EXAMPLE\PYGZus{}tto}\PYG{l+s+s1}{\PYGZsq{}}\PYG{p}{,} \PYG{l+s+s1}{\PYGZsq{}}\PYG{l+s+s1}{7left}\PYG{l+s+s1}{\PYGZsq{}}\PYG{p}{)} \PYG{c+c1}{\PYGZsh{} Create \PYGZsq{}EXAMPLE\PYGZus{}tt\PYGZsq{} and \PYGZsq{}EXAMPLE\PYGZus{}tto\PYGZsq{}}
\PYG{g+go}{\PYGZlt{}vcs.textcombined.Tc ...\PYGZgt{}}
\PYG{g+gp}{\PYGZgt{}\PYGZgt{}\PYGZgt{} }\PYG{n}{ex}\PYG{o}{=}\PYG{n}{vcs}\PYG{o}{.}\PYG{n}{gettextcombined}\PYG{p}{(}\PYG{l+s+s1}{\PYGZsq{}}\PYG{l+s+s1}{EXAMPLE\PYGZus{}tt}\PYG{l+s+s1}{\PYGZsq{}}\PYG{p}{,} \PYG{l+s+s1}{\PYGZsq{}}\PYG{l+s+s1}{EXAMPLE\PYGZus{}tto}\PYG{l+s+s1}{\PYGZsq{}}\PYG{p}{)}  \PYG{c+c1}{\PYGZsh{} instance of \PYGZsq{}EXAMPLE\PYGZus{}tt:::EXAMPLE\PYGZus{}tto\PYGZsq{} textcombined secondary method}
\PYG{g+gp}{\PYGZgt{}\PYGZgt{}\PYGZgt{} }\PYG{n}{a}\PYG{o}{.}\PYG{n}{textcombined}\PYG{p}{(}\PYG{n}{ex}\PYG{p}{)} \PYG{c+c1}{\PYGZsh{} plot using specified textcombined object}
\PYG{g+go}{\PYGZlt{}vcs.displayplot.Dp ...\PYGZgt{}}
\end{Verbatim}

\item[{Parameters}] \leavevmode\begin{itemize}
\item {} 
\textbf{\texttt{Tt\_name\_src}} (\href{https://docs.python.org/2/library/functions.html\#str}{\emph{\texttt{str}}}) -- Name of created object

\item {} 
\textbf{\texttt{To\_name\_src}} (\href{https://docs.python.org/2/library/functions.html\#str}{\emph{\texttt{str}}}) -- Name of parent textorientation object

\item {} 
\textbf{\texttt{string}} -- Text to render

\item {} 
\textbf{\texttt{string}} -- list of str

\item {} 
\textbf{\texttt{font}} (\emph{\texttt{int or str}}) -- Which font to use (index or name)

\item {} 
\textbf{\texttt{spacing}} (\emph{\texttt{DEPRECATED}}) -- DEPRECATED

\item {} 
\textbf{\texttt{expansion}} (\emph{\texttt{DEPRECATED}}) -- DEPRECATED

\item {} 
\textbf{\texttt{color}} (\emph{\texttt{str or int}}) -- 
A color name from the \href{https://en.wikipedia.org/wiki/X11\_color\_names}{X11 Color Names list},
or an integer value from 0-255, or an RGB/RGBA tuple/list (e.g. (0,100,0), (100,100,0,50))


\item {} 
\textbf{\texttt{priority}} (\href{https://docs.python.org/2/library/functions.html\#int}{\emph{\texttt{int}}}) -- The layer on which the object will be drawn.

\item {} 
\textbf{\texttt{viewport}} (\emph{\texttt{list of floats}}) -- 4 floats between 0 and 1. These specify the area that the X/Y values are mapped to inside of the canvas

\item {} 
\textbf{\texttt{worldcoordinate}} (\emph{\texttt{list of floats}}) -- List of 4 floats (xmin, xmax, ymin, ymax)

\item {} 
\textbf{\texttt{x}} (\emph{\texttt{list of floats}}) -- List of lists of x coordinates. Values must be between worldcoordinate{[}0{]} and worldcoordinate{[}1{]}.

\item {} 
\textbf{\texttt{y}} (\emph{\texttt{list of floats}}) -- List of lists of y coordinates. Values must be between worldcoordinate{[}2{]} and worldcoordinate{[}3{]}.

\item {} 
\textbf{\texttt{height}} (\href{https://docs.python.org/2/library/functions.html\#int}{\emph{\texttt{int}}}) -- Size of the font

\item {} 
\textbf{\texttt{angle}} (\emph{\texttt{list of int}}) -- Angle of the rendered text, in degrees

\item {} 
\textbf{\texttt{path}} (\emph{\texttt{DEPRECATED}}) -- DEPRECATED

\item {} 
\textbf{\texttt{halign}} (\href{https://docs.python.org/2/library/functions.html\#str}{\emph{\texttt{str}}}) -- Horizontal alignment of the text. One of {[}''left'', ``center'', ``right''{]}

\item {} 
\textbf{\texttt{valign}} (\href{https://docs.python.org/2/library/functions.html\#str}{\emph{\texttt{str}}}) -- Vertical alignment of the text. One of {[}''top'', ``center'', ``bottom''{]}

\end{itemize}

\item[{Returns}] \leavevmode
A textcombined object

\item[{Return type}] \leavevmode
{\hyperref[vcs/secondary/textcombined:vcs.textcombined.Tc]{\sphinxcrossref{vcs.textcombined.Tc}}}

\end{description}\end{quote}

\end{fulllineitems}

\index{gettextextent() (vcs.Canvas.Canvas method)}

\begin{fulllineitems}
\phantomsection\label{vcs/Canvas:vcs.Canvas.Canvas.gettextextent}\pysiglinewithargsret{\sphinxbfcode{gettextextent}}{\emph{textobject}}{}
Returns the coordinate of the box surrounding a text object once printed
\begin{quote}\begin{description}
\item[{Example}] \leavevmode
\begin{Verbatim}[commandchars=\\\{\}]
\PYG{g+gp}{\PYGZgt{}\PYGZgt{}\PYGZgt{} }\PYG{n}{a}\PYG{o}{=}\PYG{n}{vcs}\PYG{o}{.}\PYG{n}{init}\PYG{p}{(}\PYG{p}{)}
\PYG{g+gp}{\PYGZgt{}\PYGZgt{}\PYGZgt{} }\PYG{n}{t}\PYG{o}{=}\PYG{n}{a}\PYG{o}{.}\PYG{n}{createtext}\PYG{p}{(}\PYG{p}{)}
\PYG{g+gp}{\PYGZgt{}\PYGZgt{}\PYGZgt{} }\PYG{n}{t}\PYG{o}{.}\PYG{n}{x}\PYG{o}{=}\PYG{p}{[}\PYG{o}{.}\PYG{l+m+mi}{5}\PYG{p}{]}
\PYG{g+gp}{\PYGZgt{}\PYGZgt{}\PYGZgt{} }\PYG{n}{t}\PYG{o}{.}\PYG{n}{y}\PYG{o}{=}\PYG{p}{[}\PYG{o}{.}\PYG{l+m+mi}{5}\PYG{p}{]}
\PYG{g+gp}{\PYGZgt{}\PYGZgt{}\PYGZgt{} }\PYG{n}{t}\PYG{o}{.}\PYG{n}{string}\PYG{o}{=}\PYG{p}{[}\PYG{l+s+s1}{\PYGZsq{}}\PYG{l+s+s1}{Hello World}\PYG{l+s+s1}{\PYGZsq{}}\PYG{p}{]}
\PYG{g+gp}{\PYGZgt{}\PYGZgt{}\PYGZgt{} }\PYG{n}{a}\PYG{o}{.}\PYG{n}{gettextextent}\PYG{p}{(}\PYG{n}{t}\PYG{p}{)}
\PYG{g+go}{[[...]]}
\end{Verbatim}

\item[{Parameters}] \leavevmode
\textbf{\texttt{textobject}} ({\hyperref[vcs/Canvas:vcs.Canvas.Canvas.textcombined]{\sphinxcrossref{\emph{\texttt{textcombined}}}}}) -- A VCS text object

\item[{Returns}] \leavevmode
list of floats containing the coordinates of the text object's bounding box.

\item[{Return type}] \leavevmode
{\hyperref[vcs/graphics/boxfill:vcs.boxfill.Gfb.list]{\sphinxcrossref{list}}}

\end{description}\end{quote}

\end{fulllineitems}

\index{gettextorientation() (vcs.Canvas.Canvas method)}

\begin{fulllineitems}
\phantomsection\label{vcs/Canvas:vcs.Canvas.Canvas.gettextorientation}\pysiglinewithargsret{\sphinxbfcode{gettextorientation}}{\emph{To\_name\_src='default'}}{}
VCS contains a list of secondary methods. This function will create a
textorientation class object from an existing VCS textorientation secondary method. If
no textorientation name is given, then textorientation `default' will be used.

\begin{notice}{note}{Note:}
VCS does not allow the modification of `default' attribute sets.
However, a `default' attribute set that has been copied under a
different name can be modified. (See the {\hyperref[vcs/misc/manageElements:vcs.manageElements.createtextorientation]{\sphinxcrossref{\sphinxcode{vcs.manageElements.createtextorientation()}}}} function.)
\end{notice}
\begin{quote}\begin{description}
\item[{Example}] \leavevmode
\begin{Verbatim}[commandchars=\\\{\}]
\PYG{g+gp}{\PYGZgt{}\PYGZgt{}\PYGZgt{} }\PYG{n}{a}\PYG{o}{=}\PYG{n}{vcs}\PYG{o}{.}\PYG{n}{init}\PYG{p}{(}\PYG{p}{)}
\PYG{g+gp}{\PYGZgt{}\PYGZgt{}\PYGZgt{} }\PYG{n}{vcs}\PYG{o}{.}\PYG{n}{listelements}\PYG{p}{(}\PYG{l+s+s1}{\PYGZsq{}}\PYG{l+s+s1}{textorientation}\PYG{l+s+s1}{\PYGZsq{}}\PYG{p}{)} \PYG{c+c1}{\PYGZsh{} Show all the existing textorientation secondary methods}
\PYG{g+go}{[...]}
\PYG{g+gp}{\PYGZgt{}\PYGZgt{}\PYGZgt{} }\PYG{n}{ex}\PYG{o}{=}\PYG{n}{vcs}\PYG{o}{.}\PYG{n}{gettextorientation}\PYG{p}{(}\PYG{p}{)}  \PYG{c+c1}{\PYGZsh{} instance of \PYGZsq{}default\PYGZsq{} textorientation secondary method}
\PYG{g+gp}{\PYGZgt{}\PYGZgt{}\PYGZgt{} }\PYG{n}{ex2}\PYG{o}{=}\PYG{n}{vcs}\PYG{o}{.}\PYG{n}{gettextorientation}\PYG{p}{(}\PYG{l+s+s1}{\PYGZsq{}}\PYG{l+s+s1}{bigger}\PYG{l+s+s1}{\PYGZsq{}}\PYG{p}{)}  \PYG{c+c1}{\PYGZsh{} instance of \PYGZsq{}bigger\PYGZsq{} textorientation secondary method}
\end{Verbatim}

\item[{Parameters}] \leavevmode
\textbf{\texttt{To\_name\_src}} (\href{https://docs.python.org/2/library/functions.html\#str}{\emph{\texttt{str}}}) -- String name of an existing textorientation VCS object

\item[{Returns}] \leavevmode
A textorientation VCS object

\item[{Return type}] \leavevmode
{\hyperref[vcs/secondary/textorientation:vcs.textorientation.To]{\sphinxcrossref{vcs.textorientation.To}}}

\end{description}\end{quote}

\end{fulllineitems}

\index{gettexttable() (vcs.Canvas.Canvas method)}

\begin{fulllineitems}
\phantomsection\label{vcs/Canvas:vcs.Canvas.Canvas.gettexttable}\pysiglinewithargsret{\sphinxbfcode{gettexttable}}{\emph{name='default'}, \emph{font=None}, \emph{spacing=None}, \emph{expansion=None}, \emph{color=None}, \emph{priority=None}, \emph{viewport=None}, \emph{worldcoordinate=None}, \emph{x=None}, \emph{y=None}}{}
VCS contains a list of secondary methods. This function will create a
texttable class object from an existing VCS texttable secondary method. If
no texttable name is given, then texttable `default' will be used.

\begin{notice}{note}{Note:}
VCS does not allow the modification of `default' attribute sets.
However, a `default' attribute set that has been copied under a
different name can be modified. (See the {\hyperref[vcs/misc/manageElements:vcs.manageElements.createtexttable]{\sphinxcrossref{\sphinxcode{vcs.manageElements.createtexttable()}}}} function.)
\end{notice}
\begin{quote}\begin{description}
\item[{Example}] \leavevmode
\begin{Verbatim}[commandchars=\\\{\}]
\PYG{g+gp}{\PYGZgt{}\PYGZgt{}\PYGZgt{} }\PYG{n}{a}\PYG{o}{=}\PYG{n}{vcs}\PYG{o}{.}\PYG{n}{init}\PYG{p}{(}\PYG{p}{)}
\PYG{g+gp}{\PYGZgt{}\PYGZgt{}\PYGZgt{} }\PYG{n}{vcs}\PYG{o}{.}\PYG{n}{listelements}\PYG{p}{(}\PYG{l+s+s1}{\PYGZsq{}}\PYG{l+s+s1}{texttable}\PYG{l+s+s1}{\PYGZsq{}}\PYG{p}{)} \PYG{c+c1}{\PYGZsh{} Show all the existing texttable secondary methods}
\PYG{g+go}{[...]}
\PYG{g+gp}{\PYGZgt{}\PYGZgt{}\PYGZgt{} }\PYG{n}{ex}\PYG{o}{=}\PYG{n}{vcs}\PYG{o}{.}\PYG{n}{gettexttable}\PYG{p}{(}\PYG{p}{)}  \PYG{c+c1}{\PYGZsh{} instance of \PYGZsq{}default\PYGZsq{} texttable secondary method}
\PYG{g+gp}{\PYGZgt{}\PYGZgt{}\PYGZgt{} }\PYG{n}{ex2}\PYG{o}{=}\PYG{n}{vcs}\PYG{o}{.}\PYG{n}{gettexttable}\PYG{p}{(}\PYG{l+s+s1}{\PYGZsq{}}\PYG{l+s+s1}{bigger}\PYG{l+s+s1}{\PYGZsq{}}\PYG{p}{)}  \PYG{c+c1}{\PYGZsh{} instance of \PYGZsq{}bigger\PYGZsq{} texttable secondary method}
\end{Verbatim}

\item[{Parameters}] \leavevmode\begin{itemize}
\item {} 
\textbf{\texttt{name}} (\href{https://docs.python.org/2/library/functions.html\#str}{\emph{\texttt{str}}}) -- String name of an existing VCS texttable object

\item {} 
\textbf{\texttt{font}} -- 
???


\item {} 
\textbf{\texttt{expansion}} -- 
???


\item {} 
\textbf{\texttt{color}} (\emph{\texttt{str or int}}) -- 
A color name from the \href{https://en.wikipedia.org/wiki/X11\_color\_names}{X11 Color Names list},
or an integer value from 0-255, or an RGB/RGBA tuple/list (e.g. (0,100,0), (100,100,0,50))


\item {} 
\textbf{\texttt{priority}} (\href{https://docs.python.org/2/library/functions.html\#int}{\emph{\texttt{int}}}) -- The layer on which the texttable will be drawn.

\item {} 
\textbf{\texttt{viewport}} (\emph{\texttt{list of floats}}) -- 4 floats between 0 and 1. These specify the area that the X/Y values are mapped to inside of the canvas

\item {} 
\textbf{\texttt{worldcoordinate}} (\emph{\texttt{list of floats}}) -- List of 4 floats (xmin, xmax, ymin, ymax)

\item {} 
\textbf{\texttt{x}} (\emph{\texttt{list of floats}}) -- List of lists of x coordinates. Values must be between worldcoordinate{[}0{]} and worldcoordinate{[}1{]}.

\item {} 
\textbf{\texttt{y}} (\emph{\texttt{list of floats}}) -- List of lists of y coordinates. Values must be between worldcoordinate{[}2{]} and worldcoordinate{[}3{]}.

\end{itemize}

\item[{Returns}] \leavevmode
A texttable graphics method object

\item[{Return type}] \leavevmode
{\hyperref[vcs/secondary/texttable:vcs.texttable.Tt]{\sphinxcrossref{vcs.texttable.Tt}}}

\end{description}\end{quote}

\end{fulllineitems}

\index{getvector() (vcs.Canvas.Canvas method)}

\begin{fulllineitems}
\phantomsection\label{vcs/Canvas:vcs.Canvas.Canvas.getvector}\pysiglinewithargsret{\sphinxbfcode{getvector}}{\emph{Gv\_name\_src='default'}}{}
VCS contains a list of graphics methods. This function will create a
vector class object from an existing VCS vector graphics method. If
no vector name is given, then vector `default' will be used.

\begin{notice}{note}{Note:}
VCS does not allow the modification of `default' attribute sets.
However, a `default' attribute set that has been copied under a
different name can be modified. (See the {\hyperref[vcs/misc/manageElements:vcs.manageElements.createvector]{\sphinxcrossref{\sphinxcode{vcs.manageElements.createvector()}}}} function.)
\end{notice}
\begin{quote}\begin{description}
\item[{Example}] \leavevmode
\begin{Verbatim}[commandchars=\\\{\}]
\PYG{g+gp}{\PYGZgt{}\PYGZgt{}\PYGZgt{} }\PYG{n}{a}\PYG{o}{=}\PYG{n}{vcs}\PYG{o}{.}\PYG{n}{init}\PYG{p}{(}\PYG{p}{)}
\PYG{g+gp}{\PYGZgt{}\PYGZgt{}\PYGZgt{} }\PYG{n}{vcs}\PYG{o}{.}\PYG{n}{listelements}\PYG{p}{(}\PYG{l+s+s1}{\PYGZsq{}}\PYG{l+s+s1}{vector}\PYG{l+s+s1}{\PYGZsq{}}\PYG{p}{)} \PYG{c+c1}{\PYGZsh{} Show all the existing vector graphics methods}
\PYG{g+go}{[...]}
\PYG{g+gp}{\PYGZgt{}\PYGZgt{}\PYGZgt{} }\PYG{n}{ex}\PYG{o}{=}\PYG{n}{vcs}\PYG{o}{.}\PYG{n}{getvector}\PYG{p}{(}\PYG{p}{)}  \PYG{c+c1}{\PYGZsh{} instance of \PYGZsq{}default\PYGZsq{} vector graphics method}
\PYG{g+gp}{\PYGZgt{}\PYGZgt{}\PYGZgt{} }\PYG{k+kn}{import} \PYG{n+nn}{cdms2} \PYG{c+c1}{\PYGZsh{} Need cdms2 to create a slab}
\PYG{g+gp}{\PYGZgt{}\PYGZgt{}\PYGZgt{} }\PYG{n}{f} \PYG{o}{=} \PYG{n}{cdms2}\PYG{o}{.}\PYG{n}{open}\PYG{p}{(}\PYG{n}{vcs}\PYG{o}{.}\PYG{n}{sample\PYGZus{}data}\PYG{o}{+}\PYG{l+s+s1}{\PYGZsq{}}\PYG{l+s+s1}{/clt.nc}\PYG{l+s+s1}{\PYGZsq{}}\PYG{p}{)} \PYG{c+c1}{\PYGZsh{} use cdms2 to open a data file}
\PYG{g+gp}{\PYGZgt{}\PYGZgt{}\PYGZgt{} }\PYG{n}{slab1} \PYG{o}{=} \PYG{n}{f}\PYG{p}{(}\PYG{l+s+s1}{\PYGZsq{}}\PYG{l+s+s1}{u}\PYG{l+s+s1}{\PYGZsq{}}\PYG{p}{)} \PYG{c+c1}{\PYGZsh{} use the data file to create a cdms2 slab}
\PYG{g+gp}{\PYGZgt{}\PYGZgt{}\PYGZgt{} }\PYG{n}{slab2} \PYG{o}{=} \PYG{n}{f}\PYG{p}{(}\PYG{l+s+s1}{\PYGZsq{}}\PYG{l+s+s1}{v}\PYG{l+s+s1}{\PYGZsq{}}\PYG{p}{)} \PYG{c+c1}{\PYGZsh{} need 2 slabs, so get another}
\PYG{g+gp}{\PYGZgt{}\PYGZgt{}\PYGZgt{} }\PYG{n}{a}\PYG{o}{.}\PYG{n}{vector}\PYG{p}{(}\PYG{n}{ex}\PYG{p}{,} \PYG{n}{slab1}\PYG{p}{,} \PYG{n}{slab2}\PYG{p}{)} \PYG{c+c1}{\PYGZsh{} plot using specified vector object}
\PYG{g+go}{\PYGZlt{}vcs.displayplot.Dp ...\PYGZgt{}}
\end{Verbatim}

\item[{Parameters}] \leavevmode
\textbf{\texttt{Gv\_name\_src}} (\href{https://docs.python.org/2/library/functions.html\#str}{\emph{\texttt{str}}}) -- String name of an existing vector VCS object

\item[{Returns}] \leavevmode
A vector graphics method object

\item[{Return type}] \leavevmode
{\hyperref[vcs/graphics/vector:vcs.vector.Gv]{\sphinxcrossref{vcs.vector.Gv}}}

\end{description}\end{quote}

\end{fulllineitems}

\index{getxvsy() (vcs.Canvas.Canvas method)}

\begin{fulllineitems}
\phantomsection\label{vcs/Canvas:vcs.Canvas.Canvas.getxvsy}\pysiglinewithargsret{\sphinxbfcode{getxvsy}}{\emph{GXY\_name\_src='default'}}{}
VCS contains a list of graphics methods. This function will create a
xvsy class object from an existing VCS xvsy graphics method. If
no xvsy name is given, then xvsy `{\color{red}\bfseries{}default\_xvsy\_}` will be used.

\begin{notice}{note}{Note:}
VCS does not allow the modification of `default' attribute sets.
However, a `default' attribute set that has been copied under a
different name can be modified. (See the {\hyperref[vcs/misc/manageElements:vcs.manageElements.createxvsy]{\sphinxcrossref{\sphinxcode{vcs.manageElements.createxvsy()}}}} function.)
\end{notice}
\begin{quote}\begin{description}
\item[{Example}] \leavevmode
\begin{Verbatim}[commandchars=\\\{\}]
\PYG{g+gp}{\PYGZgt{}\PYGZgt{}\PYGZgt{} }\PYG{n}{a}\PYG{o}{=}\PYG{n}{vcs}\PYG{o}{.}\PYG{n}{init}\PYG{p}{(}\PYG{p}{)}
\PYG{g+gp}{\PYGZgt{}\PYGZgt{}\PYGZgt{} }\PYG{n}{vcs}\PYG{o}{.}\PYG{n}{listelements}\PYG{p}{(}\PYG{l+s+s1}{\PYGZsq{}}\PYG{l+s+s1}{xvsy}\PYG{l+s+s1}{\PYGZsq{}}\PYG{p}{)} \PYG{c+c1}{\PYGZsh{} Show all the existing xvsy graphics methods}
\PYG{g+go}{[...]}
\PYG{g+gp}{\PYGZgt{}\PYGZgt{}\PYGZgt{} }\PYG{n}{ex}\PYG{o}{=}\PYG{n}{vcs}\PYG{o}{.}\PYG{n}{getxvsy}\PYG{p}{(}\PYG{p}{)}  \PYG{c+c1}{\PYGZsh{} instance of \PYGZsq{}default\PYGZus{}xvsy\PYGZus{}\PYGZsq{} xvsy graphics method}
\PYG{g+gp}{\PYGZgt{}\PYGZgt{}\PYGZgt{} }\PYG{k+kn}{import} \PYG{n+nn}{cdms2} \PYG{c+c1}{\PYGZsh{} Need cdms2 to create a slab}
\PYG{g+gp}{\PYGZgt{}\PYGZgt{}\PYGZgt{} }\PYG{n}{f} \PYG{o}{=} \PYG{n}{cdms2}\PYG{o}{.}\PYG{n}{open}\PYG{p}{(}\PYG{n}{vcs}\PYG{o}{.}\PYG{n}{sample\PYGZus{}data}\PYG{o}{+}\PYG{l+s+s1}{\PYGZsq{}}\PYG{l+s+s1}{/clt.nc}\PYG{l+s+s1}{\PYGZsq{}}\PYG{p}{)} \PYG{c+c1}{\PYGZsh{} use cdms2 to open a data file}
\PYG{g+gp}{\PYGZgt{}\PYGZgt{}\PYGZgt{} }\PYG{n}{slab1} \PYG{o}{=} \PYG{n}{f}\PYG{p}{(}\PYG{l+s+s1}{\PYGZsq{}}\PYG{l+s+s1}{u}\PYG{l+s+s1}{\PYGZsq{}}\PYG{p}{)} \PYG{c+c1}{\PYGZsh{} use the data file to create a cdms2 slab}
\PYG{g+gp}{\PYGZgt{}\PYGZgt{}\PYGZgt{} }\PYG{n}{slab2} \PYG{o}{=} \PYG{n}{f}\PYG{p}{(}\PYG{l+s+s1}{\PYGZsq{}}\PYG{l+s+s1}{v}\PYG{l+s+s1}{\PYGZsq{}}\PYG{p}{)} \PYG{c+c1}{\PYGZsh{} need 2 slabs, so get another}
\PYG{g+gp}{\PYGZgt{}\PYGZgt{}\PYGZgt{} }\PYG{n}{a}\PYG{o}{.}\PYG{n}{xvsy}\PYG{p}{(}\PYG{n}{ex}\PYG{p}{,} \PYG{n}{slab1}\PYG{p}{,} \PYG{n}{slab2}\PYG{p}{)} \PYG{c+c1}{\PYGZsh{} plot using specified xvsy object}
\PYG{g+go}{\PYGZlt{}vcs.displayplot.Dp ...\PYGZgt{}}
\end{Verbatim}

\item[{Parameters}] \leavevmode\begin{itemize}
\item {} 
\textbf{\texttt{GXY\_name\_src}} (\href{https://docs.python.org/2/library/functions.html\#str}{\emph{\texttt{str}}}) -- String name of a 1d graphics method

\item {} 
\textbf{\texttt{xaxis}} (\emph{\texttt{cdms2.axis.TransientAxis}}) -- Axis object to replace the slab -1 dim axis

\item {} 
\textbf{\texttt{yaxis}} (\emph{\texttt{cdms2.axis.TransientAxis}}) -- Axis object to replace the slab -2 dim axis, only if slab has more than 1D

\item {} 
\textbf{\texttt{zaxis}} (\emph{\texttt{cdms2.axis.TransientAxis}}) -- Axis object to replace the slab -3 dim axis, only if slab has more than 2D

\item {} 
\textbf{\texttt{taxis}} (\emph{\texttt{cdms2.axis.TransientAxis}}) -- Axis object to replace the slab -4 dim axis, only if slab has more than 3D

\item {} 
\textbf{\texttt{waxis}} (\emph{\texttt{cdms2.axis.TransientAxis}}) -- Axis object to replace the slab -5 dim axis, only if slab has more than 4D

\item {} 
\textbf{\texttt{xrev}} (\href{https://docs.python.org/2/library/functions.html\#bool}{\emph{\texttt{bool}}}) -- reverse x axis

\item {} 
\textbf{\texttt{yrev}} (\href{https://docs.python.org/2/library/functions.html\#bool}{\emph{\texttt{bool}}}) -- reverse y axis, only if slab has more than 1D

\item {} 
\textbf{\texttt{xarray}} (\href{https://docs.python.org/2/library/array.html\#module-array}{\emph{\texttt{array}}}) -- Values to use instead of x axis

\item {} 
\textbf{\texttt{yarray}} (\href{https://docs.python.org/2/library/array.html\#module-array}{\emph{\texttt{array}}}) -- Values to use instead of y axis, only if var has more than 1D

\item {} 
\textbf{\texttt{zarray}} (\href{https://docs.python.org/2/library/array.html\#module-array}{\emph{\texttt{array}}}) -- Values to use instead of z axis, only if var has more than 2D

\item {} 
\textbf{\texttt{tarray}} (\href{https://docs.python.org/2/library/array.html\#module-array}{\emph{\texttt{array}}}) -- Values to use instead of t axis, only if var has more than 3D

\item {} 
\textbf{\texttt{warray}} (\href{https://docs.python.org/2/library/array.html\#module-array}{\emph{\texttt{array}}}) -- Values to use instead of w axis, only if var has more than 4D

\item {} 
\textbf{\texttt{continents}} (\href{https://docs.python.org/2/library/functions.html\#int}{\emph{\texttt{int}}}) -- continents type number

\item {} 
\textbf{\texttt{name}} (\href{https://docs.python.org/2/library/functions.html\#str}{\emph{\texttt{str}}}) -- replaces variable name on plot

\item {} 
\textbf{\texttt{time}} (\emph{\texttt{A cdtime object}}) -- replaces time name on plot

\item {} 
\textbf{\texttt{units}} (\href{https://docs.python.org/2/library/functions.html\#str}{\emph{\texttt{str}}}) -- replaces units value on plot

\item {} 
\textbf{\texttt{ymd}} (\href{https://docs.python.org/2/library/functions.html\#str}{\emph{\texttt{str}}}) -- replaces year/month/day on plot

\item {} 
\textbf{\texttt{hms}} (\href{https://docs.python.org/2/library/functions.html\#str}{\emph{\texttt{str}}}) -- replaces hh/mm/ss on plot

\item {} 
\textbf{\texttt{file\_comment}} (\href{https://docs.python.org/2/library/functions.html\#str}{\emph{\texttt{str}}}) -- replaces file\_comment on plot

\item {} 
\textbf{\texttt{xbounds}} (\href{https://docs.python.org/2/library/array.html\#module-array}{\emph{\texttt{array}}}) -- Values to use instead of x axis bounds values

\item {} 
\textbf{\texttt{ybounds}} (\href{https://docs.python.org/2/library/array.html\#module-array}{\emph{\texttt{array}}}) -- Values to use instead of y axis bounds values (if exist)

\item {} 
\textbf{\texttt{xname}} (\href{https://docs.python.org/2/library/functions.html\#str}{\emph{\texttt{str}}}) -- replace xaxis name on plot

\item {} 
\textbf{\texttt{yname}} (\href{https://docs.python.org/2/library/functions.html\#str}{\emph{\texttt{str}}}) -- replace yaxis name on plot (if exists)

\item {} 
\textbf{\texttt{zname}} (\href{https://docs.python.org/2/library/functions.html\#str}{\emph{\texttt{str}}}) -- replace zaxis name on plot (if exists)

\item {} 
\textbf{\texttt{tname}} (\href{https://docs.python.org/2/library/functions.html\#str}{\emph{\texttt{str}}}) -- replace taxis name on plot (if exists)

\item {} 
\textbf{\texttt{wname}} (\href{https://docs.python.org/2/library/functions.html\#str}{\emph{\texttt{str}}}) -- replace waxis name on plot (if exists)

\item {} 
\textbf{\texttt{xunits}} (\href{https://docs.python.org/2/library/functions.html\#str}{\emph{\texttt{str}}}) -- replace xaxis units on plot

\item {} 
\textbf{\texttt{yunits}} (\href{https://docs.python.org/2/library/functions.html\#str}{\emph{\texttt{str}}}) -- replace yaxis units on plot (if exists)

\item {} 
\textbf{\texttt{zunits}} (\href{https://docs.python.org/2/library/functions.html\#str}{\emph{\texttt{str}}}) -- replace zaxis units on plot (if exists)

\item {} 
\textbf{\texttt{tunits}} (\href{https://docs.python.org/2/library/functions.html\#str}{\emph{\texttt{str}}}) -- replace taxis units on plot (if exists)

\item {} 
\textbf{\texttt{wunits}} (\href{https://docs.python.org/2/library/functions.html\#str}{\emph{\texttt{str}}}) -- replace waxis units on plot (if exists)

\item {} 
\textbf{\texttt{xweights}} (\href{https://docs.python.org/2/library/array.html\#module-array}{\emph{\texttt{array}}}) -- replace xaxis weights used for computing mean

\item {} 
\textbf{\texttt{yweights}} (\href{https://docs.python.org/2/library/array.html\#module-array}{\emph{\texttt{array}}}) -- replace xaxis weights used for computing mean

\item {} 
\textbf{\texttt{comment1}} (\href{https://docs.python.org/2/library/functions.html\#str}{\emph{\texttt{str}}}) -- replaces comment1 on plot

\item {} 
\textbf{\texttt{comment2}} (\href{https://docs.python.org/2/library/functions.html\#str}{\emph{\texttt{str}}}) -- replaces comment2 on plot

\item {} 
\textbf{\texttt{comment3}} (\href{https://docs.python.org/2/library/functions.html\#str}{\emph{\texttt{str}}}) -- replaces comment3 on plot

\item {} 
\textbf{\texttt{comment4}} (\href{https://docs.python.org/2/library/functions.html\#str}{\emph{\texttt{str}}}) -- replaces comment4 on plot

\item {} 
\textbf{\texttt{long\_name}} (\href{https://docs.python.org/2/library/functions.html\#str}{\emph{\texttt{str}}}) -- replaces long\_name on plot

\item {} 
\textbf{\texttt{grid}} (\emph{\texttt{cdms2.grid.TransientRectGrid}}) -- replaces array grid (if exists)

\item {} 
\textbf{\texttt{bg}} (\emph{\texttt{bool/int}}) -- plots in background mode

\item {} 
\textbf{\texttt{ratio}} (\index{xmtics1 (vcs.Canvas.Canvas attribute)}\index{xmtics2 (vcs.Canvas.Canvas attribute)}\index{ymtics1 (vcs.Canvas.Canvas attribute)}\index{ymtics2 (vcs.Canvas.Canvas attribute)}\index{xticlabels1 (vcs.Canvas.Canvas attribute)}\index{xticlabels2 (vcs.Canvas.Canvas attribute)}\index{yticlabels1 (vcs.Canvas.Canvas attribute)}\index{yticlabels2 (vcs.Canvas.Canvas attribute)}\index{projection (vcs.Canvas.Canvas attribute)}\index{datawc\_x1 (vcs.Canvas.Canvas attribute)}\index{datawc\_x2 (vcs.Canvas.Canvas attribute)}\index{datawc\_y1 (vcs.Canvas.Canvas attribute)}\index{datawc\_y2 (vcs.Canvas.Canvas attribute)}\index{datawc\_timeunits (vcs.Canvas.Canvas attribute)}\index{datawc\_calendar (vcs.Canvas.Canvas attribute)}) -- sets the y/x ratio ,if passed as a string with `t' at the end, will aslo moves the ticks

\item {} 
\textbf{\texttt{xaxisconvert}} (\href{https://docs.python.org/2/library/functions.html\#str}{\emph{\texttt{str}}}) -- (Ex: `linear') converting xaxis linear/log/log10/ln/exp/area\_wt

\item {} 
\textbf{\texttt{yaxisconvert}} (\href{https://docs.python.org/2/library/functions.html\#str}{\emph{\texttt{str}}}) -- (Ex: `linear') converting yaxis linear/log/log10/ln/exp/area\_wt

\item {} 
\textbf{\texttt{GM\_name}} -- (Ex: `default') retrieve the graphics method object of the given name. If no name is given, then retrieve the `default' graphics method.

\end{itemize}

\item[{Returns}] \leavevmode
A XvsY graphics method object

\item[{Return type}] \leavevmode
{\hyperref[vcs/graphics/unified1D:vcs.unified1D.G1d]{\sphinxcrossref{vcs.unified1D.G1d}}}

\end{description}\end{quote}

\end{fulllineitems}

\index{getxyvsy() (vcs.Canvas.Canvas method)}

\begin{fulllineitems}
\phantomsection\label{vcs/Canvas:vcs.Canvas.Canvas.getxyvsy}\pysiglinewithargsret{\sphinxbfcode{getxyvsy}}{\emph{GXy\_name\_src='default'}}{}
VCS contains a list of graphics methods. This function will create a
xyvsy class object from an existing VCS xyvsy graphics method. If
no xyvsy name is given, then xyvsy `'{\color{red}\bfseries{}default\_xyvsy\_}`' will be used.

\begin{notice}{note}{Note:}
VCS does not allow the modification of `default' attribute sets.
However, a `default' attribute set that has been copied under a
different name can be modified. (See the {\hyperref[vcs/misc/manageElements:vcs.manageElements.createxyvsy]{\sphinxcrossref{\sphinxcode{vcs.manageElements.createxyvsy()}}}} function.)
\end{notice}
\begin{quote}\begin{description}
\item[{Example}] \leavevmode
\begin{Verbatim}[commandchars=\\\{\}]
\PYG{g+gp}{\PYGZgt{}\PYGZgt{}\PYGZgt{} }\PYG{n}{a}\PYG{o}{=}\PYG{n}{vcs}\PYG{o}{.}\PYG{n}{init}\PYG{p}{(}\PYG{p}{)}
\PYG{g+gp}{\PYGZgt{}\PYGZgt{}\PYGZgt{} }\PYG{n}{vcs}\PYG{o}{.}\PYG{n}{listelements}\PYG{p}{(}\PYG{l+s+s1}{\PYGZsq{}}\PYG{l+s+s1}{xyvsy}\PYG{l+s+s1}{\PYGZsq{}}\PYG{p}{)} \PYG{c+c1}{\PYGZsh{} Show all the existing xyvsy graphics methods}
\PYG{g+go}{[...]}
\PYG{g+gp}{\PYGZgt{}\PYGZgt{}\PYGZgt{} }\PYG{n}{ex}\PYG{o}{=}\PYG{n}{vcs}\PYG{o}{.}\PYG{n}{getxyvsy}\PYG{p}{(}\PYG{l+s+s1}{\PYGZsq{}}\PYG{l+s+s1}{default\PYGZus{}xyvsy\PYGZus{}}\PYG{l+s+s1}{\PYGZsq{}}\PYG{p}{)}  \PYG{c+c1}{\PYGZsh{} instance of \PYGZsq{}\PYGZsq{}default\PYGZus{}xyvsy\PYGZus{}\PYGZsq{}\PYGZsq{} xyvsy graphics method}
\PYG{g+gp}{\PYGZgt{}\PYGZgt{}\PYGZgt{} }\PYG{k+kn}{import} \PYG{n+nn}{cdms2} \PYG{c+c1}{\PYGZsh{} Need cdms2 to create a slab}
\PYG{g+gp}{\PYGZgt{}\PYGZgt{}\PYGZgt{} }\PYG{n}{f} \PYG{o}{=} \PYG{n}{cdms2}\PYG{o}{.}\PYG{n}{open}\PYG{p}{(}\PYG{n}{vcs}\PYG{o}{.}\PYG{n}{sample\PYGZus{}data}\PYG{o}{+}\PYG{l+s+s1}{\PYGZsq{}}\PYG{l+s+s1}{/clt.nc}\PYG{l+s+s1}{\PYGZsq{}}\PYG{p}{)} \PYG{c+c1}{\PYGZsh{} use cdms2 to open a data file}
\PYG{g+gp}{\PYGZgt{}\PYGZgt{}\PYGZgt{} }\PYG{n}{slab1} \PYG{o}{=} \PYG{n}{f}\PYG{p}{(}\PYG{l+s+s1}{\PYGZsq{}}\PYG{l+s+s1}{u}\PYG{l+s+s1}{\PYGZsq{}}\PYG{p}{)} \PYG{c+c1}{\PYGZsh{} use the data file to create a cdms2 slab}
\PYG{g+gp}{\PYGZgt{}\PYGZgt{}\PYGZgt{} }\PYG{n}{a}\PYG{o}{.}\PYG{n}{xyvsy}\PYG{p}{(}\PYG{n}{ex}\PYG{p}{,} \PYG{n}{slab1}\PYG{p}{)} \PYG{c+c1}{\PYGZsh{} plot using specified xyvsy object}
\PYG{g+go}{\PYGZlt{}vcs.displayplot.Dp ...\PYGZgt{}}
\end{Verbatim}

\item[{Parameters}] \leavevmode\begin{itemize}
\item {} 
\textbf{\texttt{GXy\_name\_src}} (\href{https://docs.python.org/2/library/functions.html\#str}{\emph{\texttt{str}}}) -- String name of an existing Xyvsy graphics method

\item {} 
\textbf{\texttt{xaxis}} (\emph{\texttt{cdms2.axis.TransientAxis}}) -- Axis object to replace the slab -1 dim axis

\item {} 
\textbf{\texttt{yaxis}} (\emph{\texttt{cdms2.axis.TransientAxis}}) -- Axis object to replace the slab -2 dim axis, only if slab has more than 1D

\item {} 
\textbf{\texttt{zaxis}} (\emph{\texttt{cdms2.axis.TransientAxis}}) -- Axis object to replace the slab -3 dim axis, only if slab has more than 2D

\item {} 
\textbf{\texttt{taxis}} (\emph{\texttt{cdms2.axis.TransientAxis}}) -- Axis object to replace the slab -4 dim axis, only if slab has more than 3D

\item {} 
\textbf{\texttt{waxis}} (\emph{\texttt{cdms2.axis.TransientAxis}}) -- Axis object to replace the slab -5 dim axis, only if slab has more than 4D

\item {} 
\textbf{\texttt{xrev}} (\href{https://docs.python.org/2/library/functions.html\#bool}{\emph{\texttt{bool}}}) -- reverse x axis

\item {} 
\textbf{\texttt{yrev}} (\href{https://docs.python.org/2/library/functions.html\#bool}{\emph{\texttt{bool}}}) -- reverse y axis, only if slab has more than 1D

\item {} 
\textbf{\texttt{xarray}} (\href{https://docs.python.org/2/library/array.html\#module-array}{\emph{\texttt{array}}}) -- Values to use instead of x axis

\item {} 
\textbf{\texttt{yarray}} (\href{https://docs.python.org/2/library/array.html\#module-array}{\emph{\texttt{array}}}) -- Values to use instead of y axis, only if var has more than 1D

\item {} 
\textbf{\texttt{zarray}} (\href{https://docs.python.org/2/library/array.html\#module-array}{\emph{\texttt{array}}}) -- Values to use instead of z axis, only if var has more than 2D

\item {} 
\textbf{\texttt{tarray}} (\href{https://docs.python.org/2/library/array.html\#module-array}{\emph{\texttt{array}}}) -- Values to use instead of t axis, only if var has more than 3D

\item {} 
\textbf{\texttt{warray}} (\href{https://docs.python.org/2/library/array.html\#module-array}{\emph{\texttt{array}}}) -- Values to use instead of w axis, only if var has more than 4D

\item {} 
\textbf{\texttt{continents}} (\href{https://docs.python.org/2/library/functions.html\#int}{\emph{\texttt{int}}}) -- continents type number

\item {} 
\textbf{\texttt{name}} (\href{https://docs.python.org/2/library/functions.html\#str}{\emph{\texttt{str}}}) -- replaces variable name on plot

\item {} 
\textbf{\texttt{time}} (\emph{\texttt{A cdtime object}}) -- replaces time name on plot

\item {} 
\textbf{\texttt{units}} (\href{https://docs.python.org/2/library/functions.html\#str}{\emph{\texttt{str}}}) -- replaces units value on plot

\item {} 
\textbf{\texttt{ymd}} (\href{https://docs.python.org/2/library/functions.html\#str}{\emph{\texttt{str}}}) -- replaces year/month/day on plot

\item {} 
\textbf{\texttt{hms}} (\href{https://docs.python.org/2/library/functions.html\#str}{\emph{\texttt{str}}}) -- replaces hh/mm/ss on plot

\item {} 
\textbf{\texttt{file\_comment}} (\href{https://docs.python.org/2/library/functions.html\#str}{\emph{\texttt{str}}}) -- replaces file\_comment on plot

\item {} 
\textbf{\texttt{xbounds}} (\href{https://docs.python.org/2/library/array.html\#module-array}{\emph{\texttt{array}}}) -- Values to use instead of x axis bounds values

\item {} 
\textbf{\texttt{ybounds}} (\href{https://docs.python.org/2/library/array.html\#module-array}{\emph{\texttt{array}}}) -- Values to use instead of y axis bounds values (if exist)

\item {} 
\textbf{\texttt{xname}} (\href{https://docs.python.org/2/library/functions.html\#str}{\emph{\texttt{str}}}) -- replace xaxis name on plot

\item {} 
\textbf{\texttt{yname}} (\href{https://docs.python.org/2/library/functions.html\#str}{\emph{\texttt{str}}}) -- replace yaxis name on plot (if exists)

\item {} 
\textbf{\texttt{zname}} (\href{https://docs.python.org/2/library/functions.html\#str}{\emph{\texttt{str}}}) -- replace zaxis name on plot (if exists)

\item {} 
\textbf{\texttt{tname}} (\href{https://docs.python.org/2/library/functions.html\#str}{\emph{\texttt{str}}}) -- replace taxis name on plot (if exists)

\item {} 
\textbf{\texttt{wname}} (\href{https://docs.python.org/2/library/functions.html\#str}{\emph{\texttt{str}}}) -- replace waxis name on plot (if exists)

\item {} 
\textbf{\texttt{xunits}} (\href{https://docs.python.org/2/library/functions.html\#str}{\emph{\texttt{str}}}) -- replace xaxis units on plot

\item {} 
\textbf{\texttt{yunits}} (\href{https://docs.python.org/2/library/functions.html\#str}{\emph{\texttt{str}}}) -- replace yaxis units on plot (if exists)

\item {} 
\textbf{\texttt{zunits}} (\href{https://docs.python.org/2/library/functions.html\#str}{\emph{\texttt{str}}}) -- replace zaxis units on plot (if exists)

\item {} 
\textbf{\texttt{tunits}} (\href{https://docs.python.org/2/library/functions.html\#str}{\emph{\texttt{str}}}) -- replace taxis units on plot (if exists)

\item {} 
\textbf{\texttt{wunits}} (\href{https://docs.python.org/2/library/functions.html\#str}{\emph{\texttt{str}}}) -- replace waxis units on plot (if exists)

\item {} 
\textbf{\texttt{xweights}} (\href{https://docs.python.org/2/library/array.html\#module-array}{\emph{\texttt{array}}}) -- replace xaxis weights used for computing mean

\item {} 
\textbf{\texttt{yweights}} (\href{https://docs.python.org/2/library/array.html\#module-array}{\emph{\texttt{array}}}) -- replace xaxis weights used for computing mean

\item {} 
\textbf{\texttt{comment1}} (\href{https://docs.python.org/2/library/functions.html\#str}{\emph{\texttt{str}}}) -- replaces comment1 on plot

\item {} 
\textbf{\texttt{comment2}} (\href{https://docs.python.org/2/library/functions.html\#str}{\emph{\texttt{str}}}) -- replaces comment2 on plot

\item {} 
\textbf{\texttt{comment3}} (\href{https://docs.python.org/2/library/functions.html\#str}{\emph{\texttt{str}}}) -- replaces comment3 on plot

\item {} 
\textbf{\texttt{comment4}} (\href{https://docs.python.org/2/library/functions.html\#str}{\emph{\texttt{str}}}) -- replaces comment4 on plot

\item {} 
\textbf{\texttt{long\_name}} (\href{https://docs.python.org/2/library/functions.html\#str}{\emph{\texttt{str}}}) -- replaces long\_name on plot

\item {} 
\textbf{\texttt{grid}} (\emph{\texttt{cdms2.grid.TransientRectGrid}}) -- replaces array grid (if exists)

\item {} 
\textbf{\texttt{bg}} (\emph{\texttt{bool/int}}) -- plots in background mode

\item {} 
\textbf{\texttt{ratio}} (\index{xmtics1 (vcs.Canvas.Canvas attribute)}\index{xmtics2 (vcs.Canvas.Canvas attribute)}\index{ymtics1 (vcs.Canvas.Canvas attribute)}\index{ymtics2 (vcs.Canvas.Canvas attribute)}\index{xticlabels1 (vcs.Canvas.Canvas attribute)}\index{xticlabels2 (vcs.Canvas.Canvas attribute)}\index{yticlabels1 (vcs.Canvas.Canvas attribute)}\index{yticlabels2 (vcs.Canvas.Canvas attribute)}\index{projection (vcs.Canvas.Canvas attribute)}\index{datawc\_x1 (vcs.Canvas.Canvas attribute)}\index{datawc\_x2 (vcs.Canvas.Canvas attribute)}\index{datawc\_y1 (vcs.Canvas.Canvas attribute)}\index{datawc\_y2 (vcs.Canvas.Canvas attribute)}\index{datawc\_timeunits (vcs.Canvas.Canvas attribute)}\index{datawc\_calendar (vcs.Canvas.Canvas attribute)}) -- sets the y/x ratio ,if passed as a string with `t' at the end, will aslo moves the ticks

\item {} 
\textbf{\texttt{xaxisconvert}} (\href{https://docs.python.org/2/library/functions.html\#str}{\emph{\texttt{str}}}) -- (Ex: `linear') converting xaxis linear/log/log10/ln/exp/area\_wt

\item {} 
\textbf{\texttt{yaxisconvert}} (\href{https://docs.python.org/2/library/functions.html\#str}{\emph{\texttt{str}}}) -- (Ex: `linear') converting yaxis linear/log/log10/ln/exp/area\_wt

\item {} 
\textbf{\texttt{GM\_name}} -- (Ex: `default') retrieve the graphics method object of the given name. If no name is given, then retrieve the `default' graphics method.

\end{itemize}

\item[{Returns}] \leavevmode
An XYvsY graphics method object

\item[{Return type}] \leavevmode
{\hyperref[vcs/graphics/unified1D:vcs.unified1D.G1d]{\sphinxcrossref{vcs.unified1D.G1d}}}

\end{description}\end{quote}

\end{fulllineitems}

\index{getyxvsx() (vcs.Canvas.Canvas method)}

\begin{fulllineitems}
\phantomsection\label{vcs/Canvas:vcs.Canvas.Canvas.getyxvsx}\pysiglinewithargsret{\sphinxbfcode{getyxvsx}}{\emph{GYx\_name\_src='default'}}{}
VCS contains a list of graphics methods. This function will create a
yxvsx class object from an existing VCS yxvsx graphics method. If
no yxvsx name is given, then yxvsx `{\color{red}\bfseries{}default\_yxvsx\_}` will be used.

\begin{notice}{note}{Note:}
VCS does not allow the modification of `default' attribute sets.
However, a `default' attribute set that has been copied under a
different name can be modified. (See the {\hyperref[vcs/misc/manageElements:vcs.manageElements.createyxvsx]{\sphinxcrossref{\sphinxcode{vcs.manageElements.createyxvsx()}}}} function.)
\end{notice}
\begin{quote}\begin{description}
\item[{Example}] \leavevmode
\begin{Verbatim}[commandchars=\\\{\}]
\PYG{g+gp}{\PYGZgt{}\PYGZgt{}\PYGZgt{} }\PYG{n}{a}\PYG{o}{=}\PYG{n}{vcs}\PYG{o}{.}\PYG{n}{init}\PYG{p}{(}\PYG{p}{)}
\PYG{g+gp}{\PYGZgt{}\PYGZgt{}\PYGZgt{} }\PYG{n}{vcs}\PYG{o}{.}\PYG{n}{listelements}\PYG{p}{(}\PYG{l+s+s1}{\PYGZsq{}}\PYG{l+s+s1}{yxvsx}\PYG{l+s+s1}{\PYGZsq{}}\PYG{p}{)} \PYG{c+c1}{\PYGZsh{} Show all the existing yxvsx graphics methods}
\PYG{g+go}{[...]}
\PYG{g+gp}{\PYGZgt{}\PYGZgt{}\PYGZgt{} }\PYG{n}{ex}\PYG{o}{=}\PYG{n}{vcs}\PYG{o}{.}\PYG{n}{getyxvsx}\PYG{p}{(}\PYG{p}{)}  \PYG{c+c1}{\PYGZsh{} instance of \PYGZsq{}default\PYGZus{}yxvsx\PYGZus{}\PYGZsq{} yxvsx graphics method}
\PYG{g+gp}{\PYGZgt{}\PYGZgt{}\PYGZgt{} }\PYG{k+kn}{import} \PYG{n+nn}{cdms2} \PYG{c+c1}{\PYGZsh{} Need cdms2 to create a slab}
\PYG{g+gp}{\PYGZgt{}\PYGZgt{}\PYGZgt{} }\PYG{n}{f} \PYG{o}{=} \PYG{n}{cdms2}\PYG{o}{.}\PYG{n}{open}\PYG{p}{(}\PYG{n}{vcs}\PYG{o}{.}\PYG{n}{sample\PYGZus{}data}\PYG{o}{+}\PYG{l+s+s1}{\PYGZsq{}}\PYG{l+s+s1}{/clt.nc}\PYG{l+s+s1}{\PYGZsq{}}\PYG{p}{)} \PYG{c+c1}{\PYGZsh{} use cdms2 to open a data file}
\PYG{g+gp}{\PYGZgt{}\PYGZgt{}\PYGZgt{} }\PYG{n}{slab1} \PYG{o}{=} \PYG{n}{f}\PYG{p}{(}\PYG{l+s+s1}{\PYGZsq{}}\PYG{l+s+s1}{u}\PYG{l+s+s1}{\PYGZsq{}}\PYG{p}{)} \PYG{c+c1}{\PYGZsh{} use the data file to create a cdms2 slab}
\PYG{g+gp}{\PYGZgt{}\PYGZgt{}\PYGZgt{} }\PYG{n}{a}\PYG{o}{.}\PYG{n}{yxvsx}\PYG{p}{(}\PYG{n}{ex}\PYG{p}{,} \PYG{n}{slab1}\PYG{p}{)} \PYG{c+c1}{\PYGZsh{} plot using specified yxvsx object}
\PYG{g+go}{\PYGZlt{}vcs.displayplot.Dp ...\PYGZgt{}}
\end{Verbatim}

\item[{Parameters}] \leavevmode\begin{itemize}
\item {} 
\textbf{\texttt{GYx\_name\_src}} (\href{https://docs.python.org/2/library/functions.html\#str}{\emph{\texttt{str}}}) -- String name of an existing Yxvsx graphics method

\item {} 
\textbf{\texttt{xaxis}} (\emph{\texttt{cdms2.axis.TransientAxis}}) -- Axis object to replace the slab -1 dim axis

\item {} 
\textbf{\texttt{yaxis}} (\emph{\texttt{cdms2.axis.TransientAxis}}) -- Axis object to replace the slab -2 dim axis, only if slab has more than 1D

\item {} 
\textbf{\texttt{zaxis}} (\emph{\texttt{cdms2.axis.TransientAxis}}) -- Axis object to replace the slab -3 dim axis, only if slab has more than 2D

\item {} 
\textbf{\texttt{taxis}} (\emph{\texttt{cdms2.axis.TransientAxis}}) -- Axis object to replace the slab -4 dim axis, only if slab has more than 3D

\item {} 
\textbf{\texttt{waxis}} (\emph{\texttt{cdms2.axis.TransientAxis}}) -- Axis object to replace the slab -5 dim axis, only if slab has more than 4D

\item {} 
\textbf{\texttt{xrev}} (\href{https://docs.python.org/2/library/functions.html\#bool}{\emph{\texttt{bool}}}) -- reverse x axis

\item {} 
\textbf{\texttt{yrev}} (\href{https://docs.python.org/2/library/functions.html\#bool}{\emph{\texttt{bool}}}) -- reverse y axis, only if slab has more than 1D

\item {} 
\textbf{\texttt{xarray}} (\href{https://docs.python.org/2/library/array.html\#module-array}{\emph{\texttt{array}}}) -- Values to use instead of x axis

\item {} 
\textbf{\texttt{yarray}} (\href{https://docs.python.org/2/library/array.html\#module-array}{\emph{\texttt{array}}}) -- Values to use instead of y axis, only if var has more than 1D

\item {} 
\textbf{\texttt{zarray}} (\href{https://docs.python.org/2/library/array.html\#module-array}{\emph{\texttt{array}}}) -- Values to use instead of z axis, only if var has more than 2D

\item {} 
\textbf{\texttt{tarray}} (\href{https://docs.python.org/2/library/array.html\#module-array}{\emph{\texttt{array}}}) -- Values to use instead of t axis, only if var has more than 3D

\item {} 
\textbf{\texttt{warray}} (\href{https://docs.python.org/2/library/array.html\#module-array}{\emph{\texttt{array}}}) -- Values to use instead of w axis, only if var has more than 4D

\item {} 
\textbf{\texttt{continents}} (\href{https://docs.python.org/2/library/functions.html\#int}{\emph{\texttt{int}}}) -- continents type number

\item {} 
\textbf{\texttt{name}} (\href{https://docs.python.org/2/library/functions.html\#str}{\emph{\texttt{str}}}) -- replaces variable name on plot

\item {} 
\textbf{\texttt{time}} (\emph{\texttt{A cdtime object}}) -- replaces time name on plot

\item {} 
\textbf{\texttt{units}} (\href{https://docs.python.org/2/library/functions.html\#str}{\emph{\texttt{str}}}) -- replaces units value on plot

\item {} 
\textbf{\texttt{ymd}} (\href{https://docs.python.org/2/library/functions.html\#str}{\emph{\texttt{str}}}) -- replaces year/month/day on plot

\item {} 
\textbf{\texttt{hms}} (\href{https://docs.python.org/2/library/functions.html\#str}{\emph{\texttt{str}}}) -- replaces hh/mm/ss on plot

\item {} 
\textbf{\texttt{file\_comment}} (\href{https://docs.python.org/2/library/functions.html\#str}{\emph{\texttt{str}}}) -- replaces file\_comment on plot

\item {} 
\textbf{\texttt{xbounds}} (\href{https://docs.python.org/2/library/array.html\#module-array}{\emph{\texttt{array}}}) -- Values to use instead of x axis bounds values

\item {} 
\textbf{\texttt{ybounds}} (\href{https://docs.python.org/2/library/array.html\#module-array}{\emph{\texttt{array}}}) -- Values to use instead of y axis bounds values (if exist)

\item {} 
\textbf{\texttt{xname}} (\href{https://docs.python.org/2/library/functions.html\#str}{\emph{\texttt{str}}}) -- replace xaxis name on plot

\item {} 
\textbf{\texttt{yname}} (\href{https://docs.python.org/2/library/functions.html\#str}{\emph{\texttt{str}}}) -- replace yaxis name on plot (if exists)

\item {} 
\textbf{\texttt{zname}} (\href{https://docs.python.org/2/library/functions.html\#str}{\emph{\texttt{str}}}) -- replace zaxis name on plot (if exists)

\item {} 
\textbf{\texttt{tname}} (\href{https://docs.python.org/2/library/functions.html\#str}{\emph{\texttt{str}}}) -- replace taxis name on plot (if exists)

\item {} 
\textbf{\texttt{wname}} (\href{https://docs.python.org/2/library/functions.html\#str}{\emph{\texttt{str}}}) -- replace waxis name on plot (if exists)

\item {} 
\textbf{\texttt{xunits}} (\href{https://docs.python.org/2/library/functions.html\#str}{\emph{\texttt{str}}}) -- replace xaxis units on plot

\item {} 
\textbf{\texttt{yunits}} (\href{https://docs.python.org/2/library/functions.html\#str}{\emph{\texttt{str}}}) -- replace yaxis units on plot (if exists)

\item {} 
\textbf{\texttt{zunits}} (\href{https://docs.python.org/2/library/functions.html\#str}{\emph{\texttt{str}}}) -- replace zaxis units on plot (if exists)

\item {} 
\textbf{\texttt{tunits}} (\href{https://docs.python.org/2/library/functions.html\#str}{\emph{\texttt{str}}}) -- replace taxis units on plot (if exists)

\item {} 
\textbf{\texttt{wunits}} (\href{https://docs.python.org/2/library/functions.html\#str}{\emph{\texttt{str}}}) -- replace waxis units on plot (if exists)

\item {} 
\textbf{\texttt{xweights}} (\href{https://docs.python.org/2/library/array.html\#module-array}{\emph{\texttt{array}}}) -- replace xaxis weights used for computing mean

\item {} 
\textbf{\texttt{yweights}} (\href{https://docs.python.org/2/library/array.html\#module-array}{\emph{\texttt{array}}}) -- replace xaxis weights used for computing mean

\item {} 
\textbf{\texttt{comment1}} (\href{https://docs.python.org/2/library/functions.html\#str}{\emph{\texttt{str}}}) -- replaces comment1 on plot

\item {} 
\textbf{\texttt{comment2}} (\href{https://docs.python.org/2/library/functions.html\#str}{\emph{\texttt{str}}}) -- replaces comment2 on plot

\item {} 
\textbf{\texttt{comment3}} (\href{https://docs.python.org/2/library/functions.html\#str}{\emph{\texttt{str}}}) -- replaces comment3 on plot

\item {} 
\textbf{\texttt{comment4}} (\href{https://docs.python.org/2/library/functions.html\#str}{\emph{\texttt{str}}}) -- replaces comment4 on plot

\item {} 
\textbf{\texttt{long\_name}} (\href{https://docs.python.org/2/library/functions.html\#str}{\emph{\texttt{str}}}) -- replaces long\_name on plot

\item {} 
\textbf{\texttt{grid}} (\emph{\texttt{cdms2.grid.TransientRectGrid}}) -- replaces array grid (if exists)

\item {} 
\textbf{\texttt{bg}} (\emph{\texttt{bool/int}}) -- plots in background mode

\item {} 
\textbf{\texttt{ratio}} (\index{xmtics1 (vcs.Canvas.Canvas attribute)}\index{xmtics2 (vcs.Canvas.Canvas attribute)}\index{ymtics1 (vcs.Canvas.Canvas attribute)}\index{ymtics2 (vcs.Canvas.Canvas attribute)}\index{xticlabels1 (vcs.Canvas.Canvas attribute)}\index{xticlabels2 (vcs.Canvas.Canvas attribute)}\index{yticlabels1 (vcs.Canvas.Canvas attribute)}\index{yticlabels2 (vcs.Canvas.Canvas attribute)}\index{projection (vcs.Canvas.Canvas attribute)}\index{datawc\_x1 (vcs.Canvas.Canvas attribute)}\index{datawc\_x2 (vcs.Canvas.Canvas attribute)}\index{datawc\_y1 (vcs.Canvas.Canvas attribute)}\index{datawc\_y2 (vcs.Canvas.Canvas attribute)}\index{datawc\_timeunits (vcs.Canvas.Canvas attribute)}\index{datawc\_calendar (vcs.Canvas.Canvas attribute)}) -- sets the y/x ratio ,if passed as a string with `t' at the end, will aslo moves the ticks

\item {} 
\textbf{\texttt{xaxisconvert}} (\href{https://docs.python.org/2/library/functions.html\#str}{\emph{\texttt{str}}}) -- (Ex: `linear') converting xaxis linear/log/log10/ln/exp/area\_wt

\item {} 
\textbf{\texttt{yaxisconvert}} (\href{https://docs.python.org/2/library/functions.html\#str}{\emph{\texttt{str}}}) -- (Ex: `linear') converting yaxis linear/log/log10/ln/exp/area\_wt

\item {} 
\textbf{\texttt{GM\_name}} -- (Ex: `default') retrieve the graphics method object of the given name. If no name is given, then retrieve the `default' graphics method.

\end{itemize}

\item[{Returns}] \leavevmode
A Yxvsx graphics method object

\item[{Return type}] \leavevmode
{\hyperref[vcs/graphics/unified1D:vcs.unified1D.G1d]{\sphinxcrossref{vcs.unified1D.G1d}}}

\end{description}\end{quote}

\end{fulllineitems}

\index{gif() (vcs.Canvas.Canvas method)}

\begin{fulllineitems}
\phantomsection\label{vcs/Canvas:vcs.Canvas.Canvas.gif}\pysiglinewithargsret{\sphinxbfcode{gif}}{\emph{filename='noname.gif'}, \emph{merge='r'}, \emph{orientation=None}, \emph{geometry=`1600x1200'}}{}
In some cases, the user may want to save the plot out as a gif image. This
routine allows the user to save the VCS canvas output as a SUN gif file.
This file can be converted to other gif formats with the aid of xv and other
such imaging tools found freely on the web.

By default, the page orientation is in Landscape mode (l). To translate the page
orientation to portrait mode (p), set the orientation = `p'.

The GIF command is used to create or append to a gif file. There are two modes
for saving a gif file: `Append' mode (a) appends gif output to an existing gif
file; `Replace' (r) mode overwrites an existing gif file with new gif output.
The default mode is to overwrite an existing gif file (i.e. mode (r)).
\begin{quote}\begin{description}
\item[{Example}] \leavevmode
\begin{Verbatim}[commandchars=\\\{\}]
\PYG{g+gp}{\PYGZgt{}\PYGZgt{}\PYGZgt{} }\PYG{n}{a}\PYG{o}{=}\PYG{n}{vcs}\PYG{o}{.}\PYG{n}{init}\PYG{p}{(}\PYG{p}{)}
\PYG{g+gp}{\PYGZgt{}\PYGZgt{}\PYGZgt{} }\PYG{n}{array} \PYG{o}{=} \PYG{p}{[}\PYG{n+nb}{range}\PYG{p}{(}\PYG{l+m+mi}{1}\PYG{p}{,} \PYG{l+m+mi}{11}\PYG{p}{)} \PYG{k}{for} \PYG{n}{\PYGZus{}} \PYG{o+ow}{in} \PYG{n+nb}{range}\PYG{p}{(}\PYG{l+m+mi}{1}\PYG{p}{,} \PYG{l+m+mi}{11}\PYG{p}{)}\PYG{p}{]}
\PYG{g+gp}{\PYGZgt{}\PYGZgt{}\PYGZgt{} }\PYG{n}{a}\PYG{o}{.}\PYG{n}{plot}\PYG{p}{(}\PYG{n}{array}\PYG{p}{)}
\PYG{g+go}{\PYGZlt{}vcs.displayplot.Dp ...\PYGZgt{}}
\PYG{g+gp}{\PYGZgt{}\PYGZgt{}\PYGZgt{} }\PYG{n}{a}\PYG{o}{.}\PYG{n}{gif}\PYG{p}{(}\PYG{n}{filename}\PYG{o}{=}\PYG{l+s+s1}{\PYGZsq{}}\PYG{l+s+s1}{example.gif}\PYG{l+s+s1}{\PYGZsq{}}\PYG{p}{,} \PYG{n}{merge}\PYG{o}{=}\PYG{l+s+s1}{\PYGZsq{}}\PYG{l+s+s1}{a}\PYG{l+s+s1}{\PYGZsq{}}\PYG{p}{,} \PYG{n}{orientation}\PYG{o}{=}\PYG{l+s+s1}{\PYGZsq{}}\PYG{l+s+s1}{l}\PYG{l+s+s1}{\PYGZsq{}}\PYG{p}{,} \PYG{n}{geometry}\PYG{o}{=}\PYG{l+s+s1}{\PYGZsq{}}\PYG{l+s+s1}{800x600}\PYG{l+s+s1}{\PYGZsq{}}\PYG{p}{)}
\PYG{g+gp}{\PYGZgt{}\PYGZgt{}\PYGZgt{} }\PYG{n}{a}\PYG{o}{.}\PYG{n}{gif}\PYG{p}{(}\PYG{l+s+s1}{\PYGZsq{}}\PYG{l+s+s1}{example}\PYG{l+s+s1}{\PYGZsq{}}\PYG{p}{)} \PYG{c+c1}{\PYGZsh{} overwrite existing gif file (default is merge=\PYGZsq{}r\PYGZsq{})}
\PYG{g+gp}{\PYGZgt{}\PYGZgt{}\PYGZgt{} }\PYG{n}{a}\PYG{o}{.}\PYG{n}{gif}\PYG{p}{(}\PYG{l+s+s1}{\PYGZsq{}}\PYG{l+s+s1}{example}\PYG{l+s+s1}{\PYGZsq{}}\PYG{p}{,}\PYG{n}{merge}\PYG{o}{=}\PYG{l+s+s1}{\PYGZsq{}}\PYG{l+s+s1}{r}\PYG{l+s+s1}{\PYGZsq{}}\PYG{p}{)} \PYG{c+c1}{\PYGZsh{} overwrite existing gif file}
\PYG{g+gp}{\PYGZgt{}\PYGZgt{}\PYGZgt{} }\PYG{n}{a}\PYG{o}{.}\PYG{n}{gif}\PYG{p}{(}\PYG{l+s+s1}{\PYGZsq{}}\PYG{l+s+s1}{example}\PYG{l+s+s1}{\PYGZsq{}}\PYG{p}{,}\PYG{n}{merge}\PYG{o}{=}\PYG{l+s+s1}{\PYGZsq{}}\PYG{l+s+s1}{a}\PYG{l+s+s1}{\PYGZsq{}}\PYG{p}{)} \PYG{c+c1}{\PYGZsh{} merge gif image into existing gif file}
\PYG{g+gp}{\PYGZgt{}\PYGZgt{}\PYGZgt{} }\PYG{n}{a}\PYG{o}{.}\PYG{n}{gif}\PYG{p}{(}\PYG{l+s+s1}{\PYGZsq{}}\PYG{l+s+s1}{example}\PYG{l+s+s1}{\PYGZsq{}}\PYG{p}{,}\PYG{n}{orientation}\PYG{o}{=}\PYG{l+s+s1}{\PYGZsq{}}\PYG{l+s+s1}{l}\PYG{l+s+s1}{\PYGZsq{}}\PYG{p}{)} \PYG{c+c1}{\PYGZsh{} merge gif image into existing gif file with landscape orientation}
\PYG{g+gp}{\PYGZgt{}\PYGZgt{}\PYGZgt{} }\PYG{n}{a}\PYG{o}{.}\PYG{n}{gif}\PYG{p}{(}\PYG{l+s+s1}{\PYGZsq{}}\PYG{l+s+s1}{example}\PYG{l+s+s1}{\PYGZsq{}}\PYG{p}{,}\PYG{n}{orientation}\PYG{o}{=}\PYG{l+s+s1}{\PYGZsq{}}\PYG{l+s+s1}{p}\PYG{l+s+s1}{\PYGZsq{}}\PYG{p}{)} \PYG{c+c1}{\PYGZsh{} merge gif image into existing gif file with portrait orientation}
\PYG{g+gp}{\PYGZgt{}\PYGZgt{}\PYGZgt{} }\PYG{n}{a}\PYG{o}{.}\PYG{n}{gif}\PYG{p}{(}\PYG{l+s+s1}{\PYGZsq{}}\PYG{l+s+s1}{example}\PYG{l+s+s1}{\PYGZsq{}}\PYG{p}{,}\PYG{n}{geometry}\PYG{o}{=}\PYG{l+s+s1}{\PYGZsq{}}\PYG{l+s+s1}{600x500}\PYG{l+s+s1}{\PYGZsq{}}\PYG{p}{)} \PYG{c+c1}{\PYGZsh{} merge gif image into existing gif file and set gif geometry}
\end{Verbatim}

\end{description}\end{quote}

\end{fulllineitems}

\index{grid() (vcs.Canvas.Canvas method)}

\begin{fulllineitems}
\phantomsection\label{vcs/Canvas:vcs.Canvas.Canvas.grid}\pysiglinewithargsret{\sphinxbfcode{grid}}{\emph{*args}}{}
Set the default plotting region for variables that have more dimension values
than the graphics method. This will also be used for animating plots over the
third and fourth dimensions.
\begin{quote}\begin{description}
\item[{Example}] \leavevmode
\begin{Verbatim}[commandchars=\\\{\}]
\PYG{g+gp}{\PYGZgt{}\PYGZgt{}\PYGZgt{} }\PYG{n}{a}\PYG{o}{=}\PYG{n}{vcs}\PYG{o}{.}\PYG{n}{init}\PYG{p}{(}\PYG{p}{)}
\PYG{g+gp}{\PYGZgt{}\PYGZgt{}\PYGZgt{} }\PYG{n}{a}\PYG{o}{.}\PYG{n}{grid}\PYG{p}{(}\PYG{l+m+mi}{12}\PYG{p}{,}\PYG{l+m+mi}{12}\PYG{p}{,}\PYG{l+m+mi}{0}\PYG{p}{,}\PYG{l+m+mi}{71}\PYG{p}{,}\PYG{l+m+mi}{0}\PYG{p}{,}\PYG{l+m+mi}{45}\PYG{p}{)}
\end{Verbatim}

\end{description}\end{quote}

\begin{notice}{note}{Not Yet Implemented}

:py:func{}`vcs.Canvas.grid{}`\_ does not work.
\end{notice}

\end{fulllineitems}

\index{isinfile() (vcs.Canvas.Canvas method)}

\begin{fulllineitems}
\phantomsection\label{vcs/Canvas:vcs.Canvas.Canvas.isinfile}\pysiglinewithargsret{\sphinxbfcode{isinfile}}{\emph{GM}, \emph{file=None}}{}
Checks if a graphic method is stored in a file
if no file name is passed then looks into the initial.attributes file
\begin{quote}\begin{description}
\item[{Parameters}] \leavevmode\begin{itemize}
\item {} 
\textbf{\texttt{GM}} (\href{https://docs.python.org/2/library/functions.html\#str}{\emph{\texttt{str}}}) -- The graphics method to search for

\item {} 
\textbf{\texttt{file}} (\href{https://docs.python.org/2/library/functions.html\#str}{\emph{\texttt{str}}}) -- String name of the file to search

\end{itemize}

\item[{Returns}] \leavevmode

???


\item[{Return type}] \leavevmode

???


\end{description}\end{quote}

\end{fulllineitems}

\index{islandscape() (vcs.Canvas.Canvas method)}

\begin{fulllineitems}
\phantomsection\label{vcs/Canvas:vcs.Canvas.Canvas.islandscape}\pysiglinewithargsret{\sphinxbfcode{islandscape}}{}{}
Indicates if VCS's orientation is landscape.

Returns a 1 if orientation is landscape.
Otherwise, it will return a 0, indicating false (not in landscape mode).
\begin{quote}\begin{description}
\item[{Example}] \leavevmode
\begin{Verbatim}[commandchars=\\\{\}]
\PYG{g+gp}{\PYGZgt{}\PYGZgt{}\PYGZgt{} }\PYG{n}{a}\PYG{o}{=}\PYG{n}{vcs}\PYG{o}{.}\PYG{n}{init}\PYG{p}{(}\PYG{p}{)}
\PYG{g+gp}{\PYGZgt{}\PYGZgt{}\PYGZgt{} }\PYG{n}{array} \PYG{o}{=} \PYG{p}{[}\PYG{n+nb}{range}\PYG{p}{(}\PYG{l+m+mi}{10}\PYG{p}{)} \PYG{k}{for} \PYG{n}{\PYGZus{}} \PYG{o+ow}{in} \PYG{n+nb}{range}\PYG{p}{(}\PYG{l+m+mi}{10}\PYG{p}{)}\PYG{p}{]}
\PYG{g+gp}{\PYGZgt{}\PYGZgt{}\PYGZgt{} }\PYG{n}{a}\PYG{o}{.}\PYG{n}{plot}\PYG{p}{(}\PYG{n}{array}\PYG{p}{)}
\PYG{g+go}{\PYGZlt{}vcs.displayplot.Dp ...\PYGZgt{}}
\PYG{g+gp}{\PYGZgt{}\PYGZgt{}\PYGZgt{} }\PYG{k}{if} \PYG{n}{a}\PYG{o}{.}\PYG{n}{islandscape}\PYG{p}{(}\PYG{p}{)}\PYG{p}{:}
\PYG{g+gp}{... }    \PYG{n}{a}\PYG{o}{.}\PYG{n}{portrait}\PYG{p}{(}\PYG{p}{)} \PYG{c+c1}{\PYGZsh{} Set VCS\PYGZsq{}s orientation to portrait mode}
\end{Verbatim}

\item[{Returns}] \leavevmode
Integer indicating VCS is in landscape mode (1), or not (0)

\item[{Return type}] \leavevmode
\href{https://docs.python.org/2/library/functions.html\#int}{int}

\end{description}\end{quote}

\end{fulllineitems}

\index{isofill() (vcs.Canvas.Canvas method)}

\begin{fulllineitems}
\phantomsection\label{vcs/Canvas:vcs.Canvas.Canvas.isofill}\pysiglinewithargsret{\sphinxbfcode{isofill}}{\emph{*args}, \emph{**parms}}{}~\begin{quote}

Generate a isofill plot given the data, isofill graphics method, and
template. If no isofill class object is given, then the `default' isofill
graphics method is used. Similarly, if no template class object is given,
then the `default' template is used.
\begin{quote}\begin{description}
\item[{Example}] \leavevmode
\begin{Verbatim}[commandchars=\\\{\}]
\PYG{g+gp}{\PYGZgt{}\PYGZgt{}\PYGZgt{} }\PYG{n}{a}\PYG{o}{=}\PYG{n}{vcs}\PYG{o}{.}\PYG{n}{init}\PYG{p}{(}\PYG{p}{)}
\PYG{g+gp}{\PYGZgt{}\PYGZgt{}\PYGZgt{} }\PYG{n}{a}\PYG{o}{.}\PYG{n}{show}\PYG{p}{(}\PYG{l+s+s1}{\PYGZsq{}}\PYG{l+s+s1}{isofill}\PYG{l+s+s1}{\PYGZsq{}}\PYG{p}{)} \PYG{c+c1}{\PYGZsh{} Show all the existing isofill graphics methods}
\PYG{g+go}{*******************Isofill Names List**********************}
\PYG{g+gp}{...}
\PYG{g+go}{*******************End Isofill Names List**********************}
\PYG{g+gp}{\PYGZgt{}\PYGZgt{}\PYGZgt{} }\PYG{n}{iso}\PYG{o}{=}\PYG{n}{a}\PYG{o}{.}\PYG{n}{getisofill}\PYG{p}{(}\PYG{l+s+s1}{\PYGZsq{}}\PYG{l+s+s1}{quick}\PYG{l+s+s1}{\PYGZsq{}}\PYG{p}{)} \PYG{c+c1}{\PYGZsh{} Create instance of \PYGZsq{}quick\PYGZsq{}}
\PYG{g+gp}{\PYGZgt{}\PYGZgt{}\PYGZgt{} }\PYG{k+kn}{import} \PYG{n+nn}{cdms2} \PYG{c+c1}{\PYGZsh{} Need cdms2 to create a slab}
\PYG{g+gp}{\PYGZgt{}\PYGZgt{}\PYGZgt{} }\PYG{n}{f} \PYG{o}{=} \PYG{n}{cdms2}\PYG{o}{.}\PYG{n}{open}\PYG{p}{(}\PYG{n}{vcs}\PYG{o}{.}\PYG{n}{sample\PYGZus{}data}\PYG{o}{+}\PYG{l+s+s1}{\PYGZsq{}}\PYG{l+s+s1}{/clt.nc}\PYG{l+s+s1}{\PYGZsq{}}\PYG{p}{)} \PYG{c+c1}{\PYGZsh{} use cdms2 to open a data file}
\PYG{g+gp}{\PYGZgt{}\PYGZgt{}\PYGZgt{} }\PYG{n}{slab} \PYG{o}{=} \PYG{n}{f}\PYG{p}{(}\PYG{l+s+s1}{\PYGZsq{}}\PYG{l+s+s1}{clt}\PYG{l+s+s1}{\PYGZsq{}}\PYG{p}{)} \PYG{c+c1}{\PYGZsh{} use the data file to create a cdms2 slab}
\PYG{g+gp}{\PYGZgt{}\PYGZgt{}\PYGZgt{} }\PYG{n}{a}\PYG{o}{.}\PYG{n}{isofill}\PYG{p}{(}\PYG{n}{slab}\PYG{p}{,}\PYG{n}{iso}\PYG{p}{)} \PYG{c+c1}{\PYGZsh{} Plot array using specified iso and default template}
\PYG{g+go}{\PYGZlt{}vcs.displayplot.Dp ...\PYGZgt{}}
\PYG{g+gp}{\PYGZgt{}\PYGZgt{}\PYGZgt{} }\PYG{n}{a}\PYG{o}{.}\PYG{n}{clear}\PYG{p}{(}\PYG{p}{)} \PYG{c+c1}{\PYGZsh{} Clear VCS canvas}
\PYG{g+gp}{\PYGZgt{}\PYGZgt{}\PYGZgt{} }\PYG{n}{template} \PYG{o}{=} \PYG{n}{a}\PYG{o}{.}\PYG{n}{gettemplate}\PYG{p}{(}\PYG{l+s+s1}{\PYGZsq{}}\PYG{l+s+s1}{hovmuller}\PYG{l+s+s1}{\PYGZsq{}}\PYG{p}{)}
\PYG{g+gp}{\PYGZgt{}\PYGZgt{}\PYGZgt{} }\PYG{n}{a}\PYG{o}{.}\PYG{n}{isofill}\PYG{p}{(}\PYG{n}{slab}\PYG{p}{,}\PYG{n}{iso}\PYG{p}{,}\PYG{n}{template}\PYG{p}{)} \PYG{c+c1}{\PYGZsh{} Plot array using specified iso and template}
\PYG{g+go}{\PYGZlt{}vcs.displayplot.Dp ...\PYGZgt{}}
\end{Verbatim}

\end{description}\end{quote}
\end{quote}
\begin{quote}\begin{description}
\item[{Parameters}] \leavevmode\begin{itemize}
\item {} 
\textbf{\texttt{xaxis}} (\emph{\texttt{cdms2.axis.TransientAxis}}) -- Axis object to replace the slab -1 dim axis

\item {} 
\textbf{\texttt{yaxis}} (\emph{\texttt{cdms2.axis.TransientAxis}}) -- Axis object to replace the slab -2 dim axis, only if slab has more than 1D

\item {} 
\textbf{\texttt{zaxis}} (\emph{\texttt{cdms2.axis.TransientAxis}}) -- Axis object to replace the slab -3 dim axis, only if slab has more than 2D

\item {} 
\textbf{\texttt{taxis}} (\emph{\texttt{cdms2.axis.TransientAxis}}) -- Axis object to replace the slab -4 dim axis, only if slab has more than 3D

\item {} 
\textbf{\texttt{waxis}} (\emph{\texttt{cdms2.axis.TransientAxis}}) -- Axis object to replace the slab -5 dim axis, only if slab has more than 4D

\item {} 
\textbf{\texttt{xrev}} (\href{https://docs.python.org/2/library/functions.html\#bool}{\emph{\texttt{bool}}}) -- reverse x axis

\item {} 
\textbf{\texttt{yrev}} (\href{https://docs.python.org/2/library/functions.html\#bool}{\emph{\texttt{bool}}}) -- reverse y axis, only if slab has more than 1D

\item {} 
\textbf{\texttt{xarray}} (\href{https://docs.python.org/2/library/array.html\#module-array}{\emph{\texttt{array}}}) -- Values to use instead of x axis

\item {} 
\textbf{\texttt{yarray}} (\href{https://docs.python.org/2/library/array.html\#module-array}{\emph{\texttt{array}}}) -- Values to use instead of y axis, only if var has more than 1D

\item {} 
\textbf{\texttt{zarray}} (\href{https://docs.python.org/2/library/array.html\#module-array}{\emph{\texttt{array}}}) -- Values to use instead of z axis, only if var has more than 2D

\item {} 
\textbf{\texttt{tarray}} (\href{https://docs.python.org/2/library/array.html\#module-array}{\emph{\texttt{array}}}) -- Values to use instead of t axis, only if var has more than 3D

\item {} 
\textbf{\texttt{warray}} (\href{https://docs.python.org/2/library/array.html\#module-array}{\emph{\texttt{array}}}) -- Values to use instead of w axis, only if var has more than 4D

\item {} 
\textbf{\texttt{continents}} (\href{https://docs.python.org/2/library/functions.html\#int}{\emph{\texttt{int}}}) -- continents type number

\item {} 
\textbf{\texttt{name}} (\href{https://docs.python.org/2/library/functions.html\#str}{\emph{\texttt{str}}}) -- replaces variable name on plot

\item {} 
\textbf{\texttt{time}} (\emph{\texttt{A cdtime object}}) -- replaces time name on plot

\item {} 
\textbf{\texttt{units}} (\href{https://docs.python.org/2/library/functions.html\#str}{\emph{\texttt{str}}}) -- replaces units value on plot

\item {} 
\textbf{\texttt{ymd}} (\href{https://docs.python.org/2/library/functions.html\#str}{\emph{\texttt{str}}}) -- replaces year/month/day on plot

\item {} 
\textbf{\texttt{hms}} (\href{https://docs.python.org/2/library/functions.html\#str}{\emph{\texttt{str}}}) -- replaces hh/mm/ss on plot

\item {} 
\textbf{\texttt{file\_comment}} (\href{https://docs.python.org/2/library/functions.html\#str}{\emph{\texttt{str}}}) -- replaces file\_comment on plot

\item {} 
\textbf{\texttt{xbounds}} (\href{https://docs.python.org/2/library/array.html\#module-array}{\emph{\texttt{array}}}) -- Values to use instead of x axis bounds values

\item {} 
\textbf{\texttt{ybounds}} (\href{https://docs.python.org/2/library/array.html\#module-array}{\emph{\texttt{array}}}) -- Values to use instead of y axis bounds values (if exist)

\item {} 
\textbf{\texttt{xname}} (\href{https://docs.python.org/2/library/functions.html\#str}{\emph{\texttt{str}}}) -- replace xaxis name on plot

\item {} 
\textbf{\texttt{yname}} (\href{https://docs.python.org/2/library/functions.html\#str}{\emph{\texttt{str}}}) -- replace yaxis name on plot (if exists)

\item {} 
\textbf{\texttt{zname}} (\href{https://docs.python.org/2/library/functions.html\#str}{\emph{\texttt{str}}}) -- replace zaxis name on plot (if exists)

\item {} 
\textbf{\texttt{tname}} (\href{https://docs.python.org/2/library/functions.html\#str}{\emph{\texttt{str}}}) -- replace taxis name on plot (if exists)

\item {} 
\textbf{\texttt{wname}} (\href{https://docs.python.org/2/library/functions.html\#str}{\emph{\texttt{str}}}) -- replace waxis name on plot (if exists)

\item {} 
\textbf{\texttt{xunits}} (\href{https://docs.python.org/2/library/functions.html\#str}{\emph{\texttt{str}}}) -- replace xaxis units on plot

\item {} 
\textbf{\texttt{yunits}} (\href{https://docs.python.org/2/library/functions.html\#str}{\emph{\texttt{str}}}) -- replace yaxis units on plot (if exists)

\item {} 
\textbf{\texttt{zunits}} (\href{https://docs.python.org/2/library/functions.html\#str}{\emph{\texttt{str}}}) -- replace zaxis units on plot (if exists)

\item {} 
\textbf{\texttt{tunits}} (\href{https://docs.python.org/2/library/functions.html\#str}{\emph{\texttt{str}}}) -- replace taxis units on plot (if exists)

\item {} 
\textbf{\texttt{wunits}} (\href{https://docs.python.org/2/library/functions.html\#str}{\emph{\texttt{str}}}) -- replace waxis units on plot (if exists)

\item {} 
\textbf{\texttt{xweights}} (\href{https://docs.python.org/2/library/array.html\#module-array}{\emph{\texttt{array}}}) -- replace xaxis weights used for computing mean

\item {} 
\textbf{\texttt{yweights}} (\href{https://docs.python.org/2/library/array.html\#module-array}{\emph{\texttt{array}}}) -- replace xaxis weights used for computing mean

\item {} 
\textbf{\texttt{comment1}} (\href{https://docs.python.org/2/library/functions.html\#str}{\emph{\texttt{str}}}) -- replaces comment1 on plot

\item {} 
\textbf{\texttt{comment2}} (\href{https://docs.python.org/2/library/functions.html\#str}{\emph{\texttt{str}}}) -- replaces comment2 on plot

\item {} 
\textbf{\texttt{comment3}} (\href{https://docs.python.org/2/library/functions.html\#str}{\emph{\texttt{str}}}) -- replaces comment3 on plot

\item {} 
\textbf{\texttt{comment4}} (\href{https://docs.python.org/2/library/functions.html\#str}{\emph{\texttt{str}}}) -- replaces comment4 on plot

\item {} 
\textbf{\texttt{long\_name}} (\href{https://docs.python.org/2/library/functions.html\#str}{\emph{\texttt{str}}}) -- replaces long\_name on plot

\item {} 
\textbf{\texttt{grid}} (\emph{\texttt{cdms2.grid.TransientRectGrid}}) -- replaces array grid (if exists)

\item {} 
\textbf{\texttt{bg}} (\emph{\texttt{bool/int}}) -- plots in background mode

\item {} 
\textbf{\texttt{ratio}} (\index{xmtics1 (vcs.Canvas.Canvas attribute)}\index{xmtics2 (vcs.Canvas.Canvas attribute)}\index{ymtics1 (vcs.Canvas.Canvas attribute)}\index{ymtics2 (vcs.Canvas.Canvas attribute)}\index{xticlabels1 (vcs.Canvas.Canvas attribute)}\index{xticlabels2 (vcs.Canvas.Canvas attribute)}\index{yticlabels1 (vcs.Canvas.Canvas attribute)}\index{yticlabels2 (vcs.Canvas.Canvas attribute)}\index{projection (vcs.Canvas.Canvas attribute)}\index{datawc\_x1 (vcs.Canvas.Canvas attribute)}\index{datawc\_x2 (vcs.Canvas.Canvas attribute)}\index{datawc\_y1 (vcs.Canvas.Canvas attribute)}\index{datawc\_y2 (vcs.Canvas.Canvas attribute)}\index{datawc\_timeunits (vcs.Canvas.Canvas attribute)}\index{datawc\_calendar (vcs.Canvas.Canvas attribute)}) -- sets the y/x ratio ,if passed as a string with `t' at the end, will aslo moves the ticks

\item {} 
\textbf{\texttt{xaxisconvert}} (\href{https://docs.python.org/2/library/functions.html\#str}{\emph{\texttt{str}}}) -- (Ex: `linear') converting xaxis linear/log/log10/ln/exp/area\_wt

\item {} 
\textbf{\texttt{yaxisconvert}} (\href{https://docs.python.org/2/library/functions.html\#str}{\emph{\texttt{str}}}) -- (Ex: `linear') converting yaxis linear/log/log10/ln/exp/area\_wt

\item {} 
\textbf{\texttt{slab}} (\href{https://docs.python.org/2/library/array.html\#module-array}{\emph{\texttt{array}}}) -- (Ex: {[}{[}0, 1{]}{]}) Data at least 2D, last 2 dimensions will be plotted

\end{itemize}

\item[{Returns}] \leavevmode
Display Plot object representing the plot.

\item[{Return type}] \leavevmode

vcs.displayplot.Dp
\begin{quote}\begin{description}
\item[{returns}] \leavevmode
A VCS displayplot object.

\item[{rtype}] \leavevmode
vcs.displayplot.Dp

\end{description}\end{quote}


\end{description}\end{quote}

\end{fulllineitems}

\index{isoline() (vcs.Canvas.Canvas method)}

\begin{fulllineitems}
\phantomsection\label{vcs/Canvas:vcs.Canvas.Canvas.isoline}\pysiglinewithargsret{\sphinxbfcode{isoline}}{\emph{*args}, \emph{**parms}}{}~\begin{quote}

Generate a isoline plot given the data, isoline graphics method, and
template. If no isoline class object is given, then the `default' isoline
graphics method is used. Similarly, if no template class object is given,
then the `default' template is used.
\begin{quote}\begin{description}
\item[{Example}] \leavevmode
\begin{Verbatim}[commandchars=\\\{\}]
\PYG{g+gp}{\PYGZgt{}\PYGZgt{}\PYGZgt{} }\PYG{n}{a}\PYG{o}{=}\PYG{n}{vcs}\PYG{o}{.}\PYG{n}{init}\PYG{p}{(}\PYG{p}{)}
\PYG{g+gp}{\PYGZgt{}\PYGZgt{}\PYGZgt{} }\PYG{n}{a}\PYG{o}{.}\PYG{n}{show}\PYG{p}{(}\PYG{l+s+s1}{\PYGZsq{}}\PYG{l+s+s1}{isoline}\PYG{l+s+s1}{\PYGZsq{}}\PYG{p}{)} \PYG{c+c1}{\PYGZsh{} Show all the existing isoline graphics methods}
\PYG{g+go}{*******************Isoline Names List**********************}
\PYG{g+gp}{...}
\PYG{g+go}{*******************End Isoline Names List**********************}
\PYG{g+gp}{\PYGZgt{}\PYGZgt{}\PYGZgt{} }\PYG{n}{iso}\PYG{o}{=}\PYG{n}{a}\PYG{o}{.}\PYG{n}{getisoline}\PYG{p}{(}\PYG{l+s+s1}{\PYGZsq{}}\PYG{l+s+s1}{quick}\PYG{l+s+s1}{\PYGZsq{}}\PYG{p}{)} \PYG{c+c1}{\PYGZsh{} Create instance of \PYGZsq{}quick\PYGZsq{}}
\PYG{g+gp}{\PYGZgt{}\PYGZgt{}\PYGZgt{} }\PYG{n}{array} \PYG{o}{=} \PYG{p}{[}\PYG{n+nb}{range}\PYG{p}{(}\PYG{l+m+mi}{1}\PYG{p}{,} \PYG{l+m+mi}{11}\PYG{p}{)} \PYG{k}{for} \PYG{n}{\PYGZus{}} \PYG{o+ow}{in} \PYG{n+nb}{range}\PYG{p}{(}\PYG{l+m+mi}{1}\PYG{p}{,} \PYG{l+m+mi}{11}\PYG{p}{)}\PYG{p}{]}
\PYG{g+gp}{\PYGZgt{}\PYGZgt{}\PYGZgt{} }\PYG{n}{a}\PYG{o}{.}\PYG{n}{isoline}\PYG{p}{(}\PYG{n}{array}\PYG{p}{,}\PYG{n}{iso}\PYG{p}{)} \PYG{c+c1}{\PYGZsh{} Plot array using specified iso and default template}
\PYG{g+go}{\PYGZlt{}vcs.displayplot.Dp ...\PYGZgt{}}
\PYG{g+gp}{\PYGZgt{}\PYGZgt{}\PYGZgt{} }\PYG{n}{a}\PYG{o}{.}\PYG{n}{clear}\PYG{p}{(}\PYG{p}{)} \PYG{c+c1}{\PYGZsh{} Clear VCS canvas}
\PYG{g+gp}{\PYGZgt{}\PYGZgt{}\PYGZgt{} }\PYG{n}{template} \PYG{o}{=} \PYG{n}{a}\PYG{o}{.}\PYG{n}{gettemplate}\PYG{p}{(}\PYG{l+s+s1}{\PYGZsq{}}\PYG{l+s+s1}{hovmuller}\PYG{l+s+s1}{\PYGZsq{}}\PYG{p}{)}
\PYG{g+gp}{\PYGZgt{}\PYGZgt{}\PYGZgt{} }\PYG{n}{a}\PYG{o}{.}\PYG{n}{isoline}\PYG{p}{(}\PYG{n}{array}\PYG{p}{,}\PYG{n}{iso}\PYG{p}{,}\PYG{n}{template}\PYG{p}{)}  \PYG{c+c1}{\PYGZsh{} Plot array using specified iso and template}
\PYG{g+go}{\PYGZlt{}vcs.displayplot.Dp ...\PYGZgt{}}
\end{Verbatim}

\end{description}\end{quote}
\end{quote}
\begin{quote}\begin{description}
\item[{Parameters}] \leavevmode\begin{itemize}
\item {} 
\textbf{\texttt{xaxis}} (\emph{\texttt{cdms2.axis.TransientAxis}}) -- Axis object to replace the slab -1 dim axis

\item {} 
\textbf{\texttt{yaxis}} (\emph{\texttt{cdms2.axis.TransientAxis}}) -- Axis object to replace the slab -2 dim axis, only if slab has more than 1D

\item {} 
\textbf{\texttt{zaxis}} (\emph{\texttt{cdms2.axis.TransientAxis}}) -- Axis object to replace the slab -3 dim axis, only if slab has more than 2D

\item {} 
\textbf{\texttt{taxis}} (\emph{\texttt{cdms2.axis.TransientAxis}}) -- Axis object to replace the slab -4 dim axis, only if slab has more than 3D

\item {} 
\textbf{\texttt{waxis}} (\emph{\texttt{cdms2.axis.TransientAxis}}) -- Axis object to replace the slab -5 dim axis, only if slab has more than 4D

\item {} 
\textbf{\texttt{xrev}} (\href{https://docs.python.org/2/library/functions.html\#bool}{\emph{\texttt{bool}}}) -- reverse x axis

\item {} 
\textbf{\texttt{yrev}} (\href{https://docs.python.org/2/library/functions.html\#bool}{\emph{\texttt{bool}}}) -- reverse y axis, only if slab has more than 1D

\item {} 
\textbf{\texttt{xarray}} (\href{https://docs.python.org/2/library/array.html\#module-array}{\emph{\texttt{array}}}) -- Values to use instead of x axis

\item {} 
\textbf{\texttt{yarray}} (\href{https://docs.python.org/2/library/array.html\#module-array}{\emph{\texttt{array}}}) -- Values to use instead of y axis, only if var has more than 1D

\item {} 
\textbf{\texttt{zarray}} (\href{https://docs.python.org/2/library/array.html\#module-array}{\emph{\texttt{array}}}) -- Values to use instead of z axis, only if var has more than 2D

\item {} 
\textbf{\texttt{tarray}} (\href{https://docs.python.org/2/library/array.html\#module-array}{\emph{\texttt{array}}}) -- Values to use instead of t axis, only if var has more than 3D

\item {} 
\textbf{\texttt{warray}} (\href{https://docs.python.org/2/library/array.html\#module-array}{\emph{\texttt{array}}}) -- Values to use instead of w axis, only if var has more than 4D

\item {} 
\textbf{\texttt{continents}} (\href{https://docs.python.org/2/library/functions.html\#int}{\emph{\texttt{int}}}) -- continents type number

\item {} 
\textbf{\texttt{name}} (\href{https://docs.python.org/2/library/functions.html\#str}{\emph{\texttt{str}}}) -- replaces variable name on plot

\item {} 
\textbf{\texttt{time}} (\emph{\texttt{A cdtime object}}) -- replaces time name on plot

\item {} 
\textbf{\texttt{units}} (\href{https://docs.python.org/2/library/functions.html\#str}{\emph{\texttt{str}}}) -- replaces units value on plot

\item {} 
\textbf{\texttt{ymd}} (\href{https://docs.python.org/2/library/functions.html\#str}{\emph{\texttt{str}}}) -- replaces year/month/day on plot

\item {} 
\textbf{\texttt{hms}} (\href{https://docs.python.org/2/library/functions.html\#str}{\emph{\texttt{str}}}) -- replaces hh/mm/ss on plot

\item {} 
\textbf{\texttt{file\_comment}} (\href{https://docs.python.org/2/library/functions.html\#str}{\emph{\texttt{str}}}) -- replaces file\_comment on plot

\item {} 
\textbf{\texttt{xbounds}} (\href{https://docs.python.org/2/library/array.html\#module-array}{\emph{\texttt{array}}}) -- Values to use instead of x axis bounds values

\item {} 
\textbf{\texttt{ybounds}} (\href{https://docs.python.org/2/library/array.html\#module-array}{\emph{\texttt{array}}}) -- Values to use instead of y axis bounds values (if exist)

\item {} 
\textbf{\texttt{xname}} (\href{https://docs.python.org/2/library/functions.html\#str}{\emph{\texttt{str}}}) -- replace xaxis name on plot

\item {} 
\textbf{\texttt{yname}} (\href{https://docs.python.org/2/library/functions.html\#str}{\emph{\texttt{str}}}) -- replace yaxis name on plot (if exists)

\item {} 
\textbf{\texttt{zname}} (\href{https://docs.python.org/2/library/functions.html\#str}{\emph{\texttt{str}}}) -- replace zaxis name on plot (if exists)

\item {} 
\textbf{\texttt{tname}} (\href{https://docs.python.org/2/library/functions.html\#str}{\emph{\texttt{str}}}) -- replace taxis name on plot (if exists)

\item {} 
\textbf{\texttt{wname}} (\href{https://docs.python.org/2/library/functions.html\#str}{\emph{\texttt{str}}}) -- replace waxis name on plot (if exists)

\item {} 
\textbf{\texttt{xunits}} (\href{https://docs.python.org/2/library/functions.html\#str}{\emph{\texttt{str}}}) -- replace xaxis units on plot

\item {} 
\textbf{\texttt{yunits}} (\href{https://docs.python.org/2/library/functions.html\#str}{\emph{\texttt{str}}}) -- replace yaxis units on plot (if exists)

\item {} 
\textbf{\texttt{zunits}} (\href{https://docs.python.org/2/library/functions.html\#str}{\emph{\texttt{str}}}) -- replace zaxis units on plot (if exists)

\item {} 
\textbf{\texttt{tunits}} (\href{https://docs.python.org/2/library/functions.html\#str}{\emph{\texttt{str}}}) -- replace taxis units on plot (if exists)

\item {} 
\textbf{\texttt{wunits}} (\href{https://docs.python.org/2/library/functions.html\#str}{\emph{\texttt{str}}}) -- replace waxis units on plot (if exists)

\item {} 
\textbf{\texttt{xweights}} (\href{https://docs.python.org/2/library/array.html\#module-array}{\emph{\texttt{array}}}) -- replace xaxis weights used for computing mean

\item {} 
\textbf{\texttt{yweights}} (\href{https://docs.python.org/2/library/array.html\#module-array}{\emph{\texttt{array}}}) -- replace xaxis weights used for computing mean

\item {} 
\textbf{\texttt{comment1}} (\href{https://docs.python.org/2/library/functions.html\#str}{\emph{\texttt{str}}}) -- replaces comment1 on plot

\item {} 
\textbf{\texttt{comment2}} (\href{https://docs.python.org/2/library/functions.html\#str}{\emph{\texttt{str}}}) -- replaces comment2 on plot

\item {} 
\textbf{\texttt{comment3}} (\href{https://docs.python.org/2/library/functions.html\#str}{\emph{\texttt{str}}}) -- replaces comment3 on plot

\item {} 
\textbf{\texttt{comment4}} (\href{https://docs.python.org/2/library/functions.html\#str}{\emph{\texttt{str}}}) -- replaces comment4 on plot

\item {} 
\textbf{\texttt{long\_name}} (\href{https://docs.python.org/2/library/functions.html\#str}{\emph{\texttt{str}}}) -- replaces long\_name on plot

\item {} 
\textbf{\texttt{grid}} (\emph{\texttt{cdms2.grid.TransientRectGrid}}) -- replaces array grid (if exists)

\item {} 
\textbf{\texttt{bg}} (\emph{\texttt{bool/int}}) -- plots in background mode

\item {} 
\textbf{\texttt{ratio}} (\index{xmtics1 (vcs.Canvas.Canvas attribute)}\index{xmtics2 (vcs.Canvas.Canvas attribute)}\index{ymtics1 (vcs.Canvas.Canvas attribute)}\index{ymtics2 (vcs.Canvas.Canvas attribute)}\index{xticlabels1 (vcs.Canvas.Canvas attribute)}\index{xticlabels2 (vcs.Canvas.Canvas attribute)}\index{yticlabels1 (vcs.Canvas.Canvas attribute)}\index{yticlabels2 (vcs.Canvas.Canvas attribute)}\index{projection (vcs.Canvas.Canvas attribute)}\index{datawc\_x1 (vcs.Canvas.Canvas attribute)}\index{datawc\_x2 (vcs.Canvas.Canvas attribute)}\index{datawc\_y1 (vcs.Canvas.Canvas attribute)}\index{datawc\_y2 (vcs.Canvas.Canvas attribute)}\index{datawc\_timeunits (vcs.Canvas.Canvas attribute)}\index{datawc\_calendar (vcs.Canvas.Canvas attribute)}) -- sets the y/x ratio ,if passed as a string with `t' at the end, will aslo moves the ticks

\item {} 
\textbf{\texttt{xaxisconvert}} (\href{https://docs.python.org/2/library/functions.html\#str}{\emph{\texttt{str}}}) -- (Ex: `linear') converting xaxis linear/log/log10/ln/exp/area\_wt

\item {} 
\textbf{\texttt{yaxisconvert}} (\href{https://docs.python.org/2/library/functions.html\#str}{\emph{\texttt{str}}}) -- (Ex: `linear') converting yaxis linear/log/log10/ln/exp/area\_wt

\item {} 
\textbf{\texttt{slab}} (\href{https://docs.python.org/2/library/array.html\#module-array}{\emph{\texttt{array}}}) -- (Ex: {[}{[}0, 1{]}{]}) Data at least 2D, last 2 dimensions will be plotted

\end{itemize}

\item[{Returns}] \leavevmode
Display Plot object representing the plot.

\item[{Return type}] \leavevmode

vcs.displayplot.Dp
\begin{quote}\begin{description}
\item[{returns}] \leavevmode
A VCS displayplot object.

\item[{rtype}] \leavevmode
vcs.displayplot.Dp

\end{description}\end{quote}


\end{description}\end{quote}

\end{fulllineitems}

\index{isopened() (vcs.Canvas.Canvas method)}

\begin{fulllineitems}
\phantomsection\label{vcs/Canvas:vcs.Canvas.Canvas.isopened}\pysiglinewithargsret{\sphinxbfcode{isopened}}{}{}
Returns a boolean value indicating whether the canvas is opened or not.
\begin{quote}\begin{description}
\item[{Returns}] \leavevmode
A boolean value indicating whether the Canvas is opened (1), or closed (0)

\item[{Return type}] \leavevmode
\href{https://docs.python.org/2/library/functions.html\#bool}{bool}

\end{description}\end{quote}

\end{fulllineitems}

\index{isportrait() (vcs.Canvas.Canvas method)}

\begin{fulllineitems}
\phantomsection\label{vcs/Canvas:vcs.Canvas.Canvas.isportrait}\pysiglinewithargsret{\sphinxbfcode{isportrait}}{}{}
Indicates if VCS's orientation is portrait.
\begin{quote}\begin{description}
\item[{Example}] \leavevmode
\begin{Verbatim}[commandchars=\\\{\}]
\PYG{g+gp}{\PYGZgt{}\PYGZgt{}\PYGZgt{} }\PYG{n}{a}\PYG{o}{=}\PYG{n}{vcs}\PYG{o}{.}\PYG{n}{init}\PYG{p}{(}\PYG{p}{)}
\PYG{g+gp}{\PYGZgt{}\PYGZgt{}\PYGZgt{} }\PYG{n}{array} \PYG{o}{=} \PYG{p}{[}\PYG{n+nb}{range}\PYG{p}{(}\PYG{l+m+mi}{10}\PYG{p}{)} \PYG{k}{for} \PYG{n}{\PYGZus{}} \PYG{o+ow}{in} \PYG{n+nb}{range}\PYG{p}{(}\PYG{l+m+mi}{10}\PYG{p}{)}\PYG{p}{]}
\PYG{g+gp}{\PYGZgt{}\PYGZgt{}\PYGZgt{} }\PYG{n}{a}\PYG{o}{.}\PYG{n}{plot}\PYG{p}{(}\PYG{n}{array}\PYG{p}{)}
\PYG{g+go}{\PYGZlt{}vcs.displayplot.Dp ...\PYGZgt{}}
\PYG{g+gp}{\PYGZgt{}\PYGZgt{}\PYGZgt{} }\PYG{k}{if} \PYG{n}{a}\PYG{o}{.}\PYG{n}{isportrait}\PYG{p}{(}\PYG{p}{)}\PYG{p}{:}
\PYG{g+gp}{... }    \PYG{n}{a}\PYG{o}{.}\PYG{n}{landscape}\PYG{p}{(}\PYG{p}{)} \PYG{c+c1}{\PYGZsh{} Set VCS\PYGZsq{}s orientation to landscape mode}
\end{Verbatim}

\item[{Returns}] \leavevmode
Returns a 1 if orientation is portrait, or 0 if not in portrait mode

\item[{Return type}] \leavevmode
\href{https://docs.python.org/2/library/functions.html\#bool}{bool}

\end{description}\end{quote}

\end{fulllineitems}

\index{landscape() (vcs.Canvas.Canvas method)}

\begin{fulllineitems}
\phantomsection\label{vcs/Canvas:vcs.Canvas.Canvas.landscape}\pysiglinewithargsret{\sphinxbfcode{landscape}}{\emph{width=-99}, \emph{height=-99}, \emph{x=-99}, \emph{y=-99}, \emph{clear=0}}{}
Change the VCS Canvas orientation to Landscape.

\begin{notice}{note}{Note:}
The (width, height) and (x, y) arguments work in pairs. That is, you must
set (width, height) or (x, y) together to see any change in the VCS Canvas.

If the portrait method is called  with arguments before displaying a VCS Canvas,
then the arguments (width, height, x, y, and clear) will have no effect on the
canvas.
\end{notice}

\begin{notice}{warning}{Warning:}
If the visible plot on the VCS Canvas is not adjusted properly, then resize
the screen with the point. Some X servers are not handling the threads properly
to keep up with the demands of the X client.
\end{notice}
\begin{quote}\begin{description}
\item[{Example}] \leavevmode
\begin{Verbatim}[commandchars=\\\{\}]
\PYG{g+gp}{\PYGZgt{}\PYGZgt{}\PYGZgt{} }\PYG{n}{a}\PYG{o}{=}\PYG{n}{vcs}\PYG{o}{.}\PYG{n}{init}\PYG{p}{(}\PYG{p}{)}
\PYG{g+gp}{\PYGZgt{}\PYGZgt{}\PYGZgt{} }\PYG{n}{array} \PYG{o}{=} \PYG{p}{[}\PYG{n+nb}{range}\PYG{p}{(}\PYG{l+m+mi}{1}\PYG{p}{,} \PYG{l+m+mi}{11}\PYG{p}{)} \PYG{k}{for} \PYG{n}{\PYGZus{}} \PYG{o+ow}{in} \PYG{n+nb}{range}\PYG{p}{(}\PYG{l+m+mi}{1}\PYG{p}{,} \PYG{l+m+mi}{11}\PYG{p}{)}\PYG{p}{]}
\PYG{g+gp}{\PYGZgt{}\PYGZgt{}\PYGZgt{} }\PYG{n}{a}\PYG{o}{.}\PYG{n}{plot}\PYG{p}{(}\PYG{n}{array}\PYG{p}{)}
\PYG{g+go}{\PYGZlt{}vcs.displayplot.Dp ...\PYGZgt{}}
\PYG{g+gp}{\PYGZgt{}\PYGZgt{}\PYGZgt{} }\PYG{n}{a}\PYG{o}{.}\PYG{n}{landscape}\PYG{p}{(}\PYG{p}{)} \PYG{c+c1}{\PYGZsh{} Change the VCS Canvas orientation and set object flag to landscape}
\PYG{g+gp}{\PYGZgt{}\PYGZgt{}\PYGZgt{} }\PYG{n}{a}\PYG{o}{.}\PYG{n}{landscape}\PYG{p}{(}\PYG{n}{clear}\PYG{o}{=}\PYG{l+m+mi}{1}\PYG{p}{)} \PYG{c+c1}{\PYGZsh{} Change the VCS Canvas to landscape and clear the page}
\PYG{g+gp}{\PYGZgt{}\PYGZgt{}\PYGZgt{} }\PYG{n}{a}\PYG{o}{.}\PYG{n}{landscape}\PYG{p}{(}\PYG{n}{width} \PYG{o}{=} \PYG{l+m+mi}{400}\PYG{p}{,} \PYG{n}{height} \PYG{o}{=} \PYG{l+m+mi}{337}\PYG{p}{)} \PYG{c+c1}{\PYGZsh{} Change to landscape and set the window size}
\PYG{g+gp}{\PYGZgt{}\PYGZgt{}\PYGZgt{} }\PYG{n}{a}\PYG{o}{.}\PYG{n}{landscape}\PYG{p}{(}\PYG{n}{x}\PYG{o}{=}\PYG{l+m+mi}{100}\PYG{p}{,} \PYG{n}{y} \PYG{o}{=} \PYG{l+m+mi}{200}\PYG{p}{)} \PYG{c+c1}{\PYGZsh{} Change to landscape and set the x and y screen position}
\PYG{g+gp}{\PYGZgt{}\PYGZgt{}\PYGZgt{} }\PYG{n}{a}\PYG{o}{.}\PYG{n}{landscape}\PYG{p}{(}\PYG{n}{width} \PYG{o}{=} \PYG{l+m+mi}{400}\PYG{p}{,} \PYG{n}{height} \PYG{o}{=} \PYG{l+m+mi}{337}\PYG{p}{,} \PYG{n}{x}\PYG{o}{=}\PYG{l+m+mi}{100}\PYG{p}{,} \PYG{n}{y} \PYG{o}{=} \PYG{l+m+mi}{200}\PYG{p}{,} \PYG{n}{clear}\PYG{o}{=}\PYG{l+m+mi}{1}\PYG{p}{)} \PYG{c+c1}{\PYGZsh{} landscape with more settings}
\end{Verbatim}

\item[{Parameters}] \leavevmode\begin{itemize}
\item {} 
\textbf{\texttt{width}} (\href{https://docs.python.org/2/library/functions.html\#int}{\emph{\texttt{int}}}) -- Width of the canvas, in pixels

\item {} 
\textbf{\texttt{height}} (\href{https://docs.python.org/2/library/functions.html\#int}{\emph{\texttt{int}}}) -- Height of the canvas, in pixels

\item {} 
\textbf{\texttt{x}} (\href{https://docs.python.org/2/library/functions.html\#int}{\emph{\texttt{int}}}) -- Unused

\item {} 
\textbf{\texttt{y}} (\href{https://docs.python.org/2/library/functions.html\#int}{\emph{\texttt{int}}}) -- Unused

\item {} 
\textbf{\texttt{clear}} (\href{https://docs.python.org/2/library/functions.html\#int}{\emph{\texttt{int}}}) -- Indicates the canvas should be cleared (1),
or should not be cleared (0), when orientation is changed.

\end{itemize}

\end{description}\end{quote}

\end{fulllineitems}

\index{line() (vcs.Canvas.Canvas method)}

\begin{fulllineitems}
\phantomsection\label{vcs/Canvas:vcs.Canvas.Canvas.line}\pysiglinewithargsret{\sphinxbfcode{line}}{\emph{*args}, \emph{**parms}}{}
Plot a line segment on the Vcs Canvas. If no line class
object is given, then an error will be returned.
\begin{quote}\begin{description}
\item[{Example}] \leavevmode
\begin{Verbatim}[commandchars=\\\{\}]
\PYG{g+gp}{\PYGZgt{}\PYGZgt{}\PYGZgt{} }\PYG{n}{a}\PYG{o}{=}\PYG{n}{vcs}\PYG{o}{.}\PYG{n}{init}\PYG{p}{(}\PYG{p}{)}
\PYG{g+gp}{\PYGZgt{}\PYGZgt{}\PYGZgt{} }\PYG{n}{a}\PYG{o}{.}\PYG{n}{show}\PYG{p}{(}\PYG{l+s+s1}{\PYGZsq{}}\PYG{l+s+s1}{line}\PYG{l+s+s1}{\PYGZsq{}}\PYG{p}{)}  \PYG{c+c1}{\PYGZsh{} Show all the existing line objects}
\PYG{g+go}{*******************Line Names List**********************}
\PYG{g+gp}{...}
\PYG{g+go}{*******************End Line Names List**********************}
\PYG{g+gp}{\PYGZgt{}\PYGZgt{}\PYGZgt{} }\PYG{n}{ln}\PYG{o}{=}\PYG{n}{a}\PYG{o}{.}\PYG{n}{getline}\PYG{p}{(}\PYG{l+s+s1}{\PYGZsq{}}\PYG{l+s+s1}{red}\PYG{l+s+s1}{\PYGZsq{}}\PYG{p}{)} \PYG{c+c1}{\PYGZsh{} Create instance of \PYGZsq{}red\PYGZsq{}}
\PYG{g+gp}{\PYGZgt{}\PYGZgt{}\PYGZgt{} }\PYG{n}{ln}\PYG{o}{.}\PYG{n}{width}\PYG{o}{=}\PYG{l+m+mi}{4} \PYG{c+c1}{\PYGZsh{} Set the line width}
\PYG{g+gp}{\PYGZgt{}\PYGZgt{}\PYGZgt{} }\PYG{n}{ln}\PYG{o}{.}\PYG{n}{color} \PYG{o}{=} \PYG{l+m+mi}{242} \PYG{c+c1}{\PYGZsh{} Set the line color}
\PYG{g+gp}{\PYGZgt{}\PYGZgt{}\PYGZgt{} }\PYG{n}{ln}\PYG{o}{.}\PYG{n}{type} \PYG{o}{=} \PYG{l+m+mi}{4} \PYG{c+c1}{\PYGZsh{} Set the line type}
\PYG{g+gp}{\PYGZgt{}\PYGZgt{}\PYGZgt{} }\PYG{n}{ln}\PYG{o}{.}\PYG{n}{x}\PYG{o}{=}\PYG{p}{[}\PYG{p}{[}\PYG{l+m+mf}{0.0}\PYG{p}{,}\PYG{l+m+mf}{2.0}\PYG{p}{,}\PYG{l+m+mf}{2.0}\PYG{p}{,}\PYG{l+m+mf}{0.0}\PYG{p}{,}\PYG{l+m+mf}{0.0}\PYG{p}{]}\PYG{p}{,} \PYG{p}{[}\PYG{l+m+mf}{0.5}\PYG{p}{,}\PYG{l+m+mf}{1.5}\PYG{p}{]}\PYG{p}{]} \PYG{c+c1}{\PYGZsh{} Set the x value points}
\PYG{g+gp}{\PYGZgt{}\PYGZgt{}\PYGZgt{} }\PYG{n}{ln}\PYG{o}{.}\PYG{n}{y}\PYG{o}{=}\PYG{p}{[}\PYG{p}{[}\PYG{l+m+mf}{0.0}\PYG{p}{,}\PYG{l+m+mf}{0.0}\PYG{p}{,}\PYG{l+m+mf}{2.0}\PYG{p}{,}\PYG{l+m+mf}{2.0}\PYG{p}{,}\PYG{l+m+mf}{0.0}\PYG{p}{]}\PYG{p}{,} \PYG{p}{[}\PYG{l+m+mf}{1.0}\PYG{p}{,}\PYG{l+m+mf}{1.0}\PYG{p}{]}\PYG{p}{]} \PYG{c+c1}{\PYGZsh{} Set the y value points}
\PYG{g+gp}{\PYGZgt{}\PYGZgt{}\PYGZgt{} }\PYG{n}{a}\PYG{o}{.}\PYG{n}{line}\PYG{p}{(}\PYG{n}{ln}\PYG{p}{)} \PYG{c+c1}{\PYGZsh{} Plot using specified line object}
\PYG{g+go}{\PYGZlt{}vcs.displayplot.Dp ...\PYGZgt{}}
\end{Verbatim}

\item[{Returns}] \leavevmode
A VCS displayplot object.

\item[{Return type}] \leavevmode
{\hyperref[vcs/misc/displayplot:vcs.displayplot.Dp]{\sphinxcrossref{vcs.displayplot.Dp}}}

\end{description}\end{quote}

\end{fulllineitems}

\index{listelements() (vcs.Canvas.Canvas method)}

\begin{fulllineitems}
\phantomsection\label{vcs/Canvas:vcs.Canvas.Canvas.listelements}\pysiglinewithargsret{\sphinxbfcode{listelements}}{\emph{*args}}{}
Returns a Python list of all the VCS class objects.

The list that will be returned:
{[}`1d', `3d\_dual\_scalar', `3d\_scalar', `3d\_vector', `boxfill', `colormap', `display', `fillarea',
\begin{quote}

`font', `fontNumber', `isofill', `isoline', `line', `list', `marker', `meshfill', `projection',
`scatter', `taylordiagram', `template', `textcombined', `textorientation', `texttable',
`vector', `xvsy', `xyvsy', `yxvsx'{]}
\end{quote}
\begin{quote}\begin{description}
\item[{Example}] \leavevmode
\begin{Verbatim}[commandchars=\\\{\}]
\PYG{g+gp}{\PYGZgt{}\PYGZgt{}\PYGZgt{} }\PYG{n}{a}\PYG{o}{=}\PYG{n}{vcs}\PYG{o}{.}\PYG{n}{init}\PYG{p}{(}\PYG{p}{)}
\PYG{g+gp}{\PYGZgt{}\PYGZgt{}\PYGZgt{} }\PYG{n}{a}\PYG{o}{.}\PYG{n}{listelements}\PYG{p}{(}\PYG{p}{)}
\PYG{g+go}{[\PYGZsq{}1d\PYGZsq{}, \PYGZsq{}3d\PYGZus{}dual\PYGZus{}scalar\PYGZsq{}, \PYGZsq{}3d\PYGZus{}scalar\PYGZsq{}, ...]}
\end{Verbatim}

\item[{Returns}] \leavevmode
A list of string names of all VCS class objects

\item[{Return type}] \leavevmode
{\hyperref[vcs/graphics/boxfill:vcs.boxfill.Gfb.list]{\sphinxcrossref{list}}}

\end{description}\end{quote}

\end{fulllineitems}

\index{marker() (vcs.Canvas.Canvas method)}

\begin{fulllineitems}
\phantomsection\label{vcs/Canvas:vcs.Canvas.Canvas.marker}\pysiglinewithargsret{\sphinxbfcode{marker}}{\emph{*args}, \emph{**parms}}{}
Plot a marker segment on the Vcs Canvas. If no marker class
object is given, then an error will be returned.
\begin{quote}\begin{description}
\item[{Example}] \leavevmode
\begin{Verbatim}[commandchars=\\\{\}]
\PYG{g+gp}{\PYGZgt{}\PYGZgt{}\PYGZgt{} }\PYG{n}{a}\PYG{o}{=}\PYG{n}{vcs}\PYG{o}{.}\PYG{n}{init}\PYG{p}{(}\PYG{p}{)}
\PYG{g+gp}{\PYGZgt{}\PYGZgt{}\PYGZgt{} }\PYG{n}{a}\PYG{o}{.}\PYG{n}{show}\PYG{p}{(}\PYG{l+s+s1}{\PYGZsq{}}\PYG{l+s+s1}{marker}\PYG{l+s+s1}{\PYGZsq{}}\PYG{p}{)} \PYG{c+c1}{\PYGZsh{} Show all the existing marker objects}
\PYG{g+go}{*******************Marker Names List**********************}
\PYG{g+gp}{...}
\PYG{g+go}{*******************End Marker Names List**********************}
\PYG{g+gp}{\PYGZgt{}\PYGZgt{}\PYGZgt{} }\PYG{n}{mrk}\PYG{o}{=}\PYG{n}{a}\PYG{o}{.}\PYG{n}{getmarker}\PYG{p}{(}\PYG{l+s+s1}{\PYGZsq{}}\PYG{l+s+s1}{red}\PYG{l+s+s1}{\PYGZsq{}}\PYG{p}{)} \PYG{c+c1}{\PYGZsh{} Create instance of \PYGZsq{}red\PYGZsq{}}
\PYG{g+gp}{\PYGZgt{}\PYGZgt{}\PYGZgt{} }\PYG{n}{mrk}\PYG{o}{.}\PYG{n}{size}\PYG{o}{=}\PYG{l+m+mi}{4} \PYG{c+c1}{\PYGZsh{} Set the marker size}
\PYG{g+gp}{\PYGZgt{}\PYGZgt{}\PYGZgt{} }\PYG{n}{mrk}\PYG{o}{.}\PYG{n}{color} \PYG{o}{=} \PYG{l+m+mi}{242} \PYG{c+c1}{\PYGZsh{} Set the marker color}
\PYG{g+gp}{\PYGZgt{}\PYGZgt{}\PYGZgt{} }\PYG{n}{mrk}\PYG{o}{.}\PYG{n}{type} \PYG{o}{=} \PYG{l+m+mi}{4} \PYG{c+c1}{\PYGZsh{} Set the marker type}
\PYG{g+gp}{\PYGZgt{}\PYGZgt{}\PYGZgt{} }\PYG{n}{mrk}\PYG{o}{.}\PYG{n}{x}\PYG{o}{=}\PYG{p}{[}\PYG{p}{[}\PYG{l+m+mf}{0.0}\PYG{p}{,}\PYG{l+m+mf}{2.0}\PYG{p}{,}\PYG{l+m+mf}{2.0}\PYG{p}{,}\PYG{l+m+mf}{0.0}\PYG{p}{,}\PYG{l+m+mf}{0.0}\PYG{p}{]}\PYG{p}{,} \PYG{p}{[}\PYG{l+m+mf}{0.5}\PYG{p}{,}\PYG{l+m+mf}{1.5}\PYG{p}{]}\PYG{p}{]} \PYG{c+c1}{\PYGZsh{} Set the x value points}
\PYG{g+gp}{\PYGZgt{}\PYGZgt{}\PYGZgt{} }\PYG{n}{mrk}\PYG{o}{.}\PYG{n}{y}\PYG{o}{=}\PYG{p}{[}\PYG{p}{[}\PYG{l+m+mf}{0.0}\PYG{p}{,}\PYG{l+m+mf}{0.0}\PYG{p}{,}\PYG{l+m+mf}{2.0}\PYG{p}{,}\PYG{l+m+mf}{2.0}\PYG{p}{,}\PYG{l+m+mf}{0.0}\PYG{p}{]}\PYG{p}{,} \PYG{p}{[}\PYG{l+m+mf}{1.0}\PYG{p}{,}\PYG{l+m+mf}{1.0}\PYG{p}{]}\PYG{p}{]} \PYG{c+c1}{\PYGZsh{} Set the y value points}
\PYG{g+gp}{\PYGZgt{}\PYGZgt{}\PYGZgt{} }\PYG{n}{a}\PYG{o}{.}\PYG{n}{marker}\PYG{p}{(}\PYG{n}{mrk}\PYG{p}{)} \PYG{c+c1}{\PYGZsh{} Plot using specified marker object}
\PYG{g+go}{\PYGZlt{}vcs.displayplot.Dp ...\PYGZgt{}}
\end{Verbatim}

\item[{Returns}] \leavevmode
a VCS displayplot object

\item[{Return type}] \leavevmode
{\hyperref[vcs/misc/displayplot:vcs.displayplot.Dp]{\sphinxcrossref{vcs.displayplot.Dp}}}

\end{description}\end{quote}

\end{fulllineitems}

\index{meshfill() (vcs.Canvas.Canvas method)}

\begin{fulllineitems}
\phantomsection\label{vcs/Canvas:vcs.Canvas.Canvas.meshfill}\pysiglinewithargsret{\sphinxbfcode{meshfill}}{\emph{*args}, \emph{**parms}}{}
Generate a meshfill plot given the data, the mesh, a meshfill graphics method, and
a template. If no meshfill class object is given, then the `default' meshfill
graphics method is used. Similarly, if no template class object is given,
then the `default' template is used.

Format:
This function expects 1D data (any extra dimension will be used for animation)
In addition the mesh array must be of the same shape than data with 2 additional dimension
representing the vertices coordinates for the Y (0) and X (1) dimension
Let's say you want to plot a spatial assuming mesh containing 10,000 grid cell, then data must be shape (10000,)
or (n1,n2,n3,...,10000) if additional dimensions exist (ex time,level), these dimension would be used only
for animation and will be ignored in the rest of this example.
The shape of the mesh, assuming 4 vertices per grid cell, must be (1000,2,4), where the array {[}:,0,:{]}
represent the Y coordinates of the vertices (clockwise or counterclockwise) and the array {[}:,1:{]}
represents the X coordinates of the vertices (the same clockwise/counterclockwise than the Y coordinates)
In brief you'd have:
data.shape=(10000,)
mesh.shape=(10000,2,4)
\begin{quote}\begin{description}
\item[{Example}] \leavevmode
\begin{Verbatim}[commandchars=\\\{\}]
\PYG{g+gp}{\PYGZgt{}\PYGZgt{}\PYGZgt{} }\PYG{n}{a}\PYG{o}{=}\PYG{n}{vcs}\PYG{o}{.}\PYG{n}{init}\PYG{p}{(}\PYG{p}{)}
\PYG{g+gp}{\PYGZgt{}\PYGZgt{}\PYGZgt{} }\PYG{n}{a}\PYG{o}{.}\PYG{n}{show}\PYG{p}{(}\PYG{l+s+s1}{\PYGZsq{}}\PYG{l+s+s1}{meshfill}\PYG{l+s+s1}{\PYGZsq{}}\PYG{p}{)} \PYG{c+c1}{\PYGZsh{} Show all the existing meshfill graphics methods}
\PYG{g+go}{*******************Meshfill Names List**********************}
\PYG{g+gp}{...}
\PYG{g+go}{*******************End Meshfill Names List**********************}
\PYG{g+gp}{\PYGZgt{}\PYGZgt{}\PYGZgt{} }\PYG{k+kn}{import} \PYG{n+nn}{cdms2} \PYG{c+c1}{\PYGZsh{} Need cdms2 to create a slab}
\PYG{g+gp}{\PYGZgt{}\PYGZgt{}\PYGZgt{} }\PYG{n}{f} \PYG{o}{=} \PYG{n}{cdms2}\PYG{o}{.}\PYG{n}{open}\PYG{p}{(}\PYG{n}{vcs}\PYG{o}{.}\PYG{n}{sample\PYGZus{}data}\PYG{o}{+}\PYG{l+s+s1}{\PYGZsq{}}\PYG{l+s+s1}{/clt.nc}\PYG{l+s+s1}{\PYGZsq{}}\PYG{p}{)} \PYG{c+c1}{\PYGZsh{} use cdms2 to open a data file}
\PYG{g+gp}{\PYGZgt{}\PYGZgt{}\PYGZgt{} }\PYG{n}{slab} \PYG{o}{=} \PYG{n}{f}\PYG{p}{(}\PYG{l+s+s1}{\PYGZsq{}}\PYG{l+s+s1}{clt}\PYG{l+s+s1}{\PYGZsq{}}\PYG{p}{)} \PYG{c+c1}{\PYGZsh{} use the data file to create a cdms2 slab}
\PYG{g+gp}{\PYGZgt{}\PYGZgt{}\PYGZgt{} }\PYG{n}{mesh}\PYG{o}{=}\PYG{n}{a}\PYG{o}{.}\PYG{n}{getmeshfill}\PYG{p}{(}\PYG{p}{)} \PYG{c+c1}{\PYGZsh{} Create instance of \PYGZsq{}default\PYGZsq{}}
\PYG{g+gp}{\PYGZgt{}\PYGZgt{}\PYGZgt{} }\PYG{n}{a}\PYG{o}{.}\PYG{n}{meshfill}\PYG{p}{(}\PYG{n}{slab}\PYG{p}{,}\PYG{n}{mesh}\PYG{p}{)} \PYG{c+c1}{\PYGZsh{} Plot array using specified mesh and default template}
\PYG{g+go}{\PYGZlt{}vcs.displayplot.Dp ...\PYGZgt{}}
\PYG{g+gp}{\PYGZgt{}\PYGZgt{}\PYGZgt{} }\PYG{n}{a}\PYG{o}{.}\PYG{n}{clear}\PYG{p}{(}\PYG{p}{)} \PYG{c+c1}{\PYGZsh{} Clear VCS canvas}
\PYG{g+gp}{\PYGZgt{}\PYGZgt{}\PYGZgt{} }\PYG{n}{a}\PYG{o}{.}\PYG{n}{meshfill}\PYG{p}{(}\PYG{n}{slab}\PYG{p}{,}\PYG{n}{mesh}\PYG{p}{,}\PYG{l+s+s1}{\PYGZsq{}}\PYG{l+s+s1}{quick}\PYG{l+s+s1}{\PYGZsq{}}\PYG{p}{,}\PYG{l+s+s1}{\PYGZsq{}}\PYG{l+s+s1}{a\PYGZus{}polar\PYGZus{}meshfill}\PYG{l+s+s1}{\PYGZsq{}}\PYG{p}{)}  \PYG{c+c1}{\PYGZsh{} Plot slab with polar mesh, quick template}
\PYG{g+go}{\PYGZlt{}vcs.displayplot.Dp ...\PYGZgt{}}
\end{Verbatim}

\item[{Returns}] \leavevmode
A VCS displayplot object.

\item[{Return type}] \leavevmode
{\hyperref[vcs/misc/displayplot:vcs.displayplot.Dp]{\sphinxcrossref{vcs.displayplot.Dp}}}

\end{description}\end{quote}

\end{fulllineitems}

\index{objecthelp() (vcs.Canvas.Canvas method)}

\begin{fulllineitems}
\phantomsection\label{vcs/Canvas:vcs.Canvas.Canvas.objecthelp}\pysiglinewithargsret{\sphinxbfcode{objecthelp}}{\emph{*arg}}{}
Print out information on the VCS object. See example below on its use.
\begin{quote}\begin{description}
\item[{Example}] \leavevmode
\begin{Verbatim}[commandchars=\\\{\}]
\PYG{g+gp}{\PYGZgt{}\PYGZgt{}\PYGZgt{} }\PYG{n}{a}\PYG{o}{=}\PYG{n}{vcs}\PYG{o}{.}\PYG{n}{init}\PYG{p}{(}\PYG{p}{)}
\PYG{g+gp}{\PYGZgt{}\PYGZgt{}\PYGZgt{} }\PYG{n}{ln}\PYG{o}{=}\PYG{n}{a}\PYG{o}{.}\PYG{n}{getline}\PYG{p}{(}\PYG{l+s+s1}{\PYGZsq{}}\PYG{l+s+s1}{red}\PYG{l+s+s1}{\PYGZsq{}}\PYG{p}{)} \PYG{c+c1}{\PYGZsh{} Get a VCS line object}
\PYG{g+gp}{\PYGZgt{}\PYGZgt{}\PYGZgt{} }\PYG{n}{a}\PYG{o}{.}\PYG{n}{objecthelp}\PYG{p}{(}\PYG{n}{ln}\PYG{p}{)} \PYG{c+c1}{\PYGZsh{} This will print out information on how to use ln}
\end{Verbatim}

\end{description}\end{quote}

\end{fulllineitems}

\index{open() (vcs.Canvas.Canvas method)}

\begin{fulllineitems}
\phantomsection\label{vcs/Canvas:vcs.Canvas.Canvas.open}\pysiglinewithargsret{\sphinxbfcode{open}}{\emph{width=None}, \emph{height=None}, \emph{**kargs}}{}
Open VCS Canvas object. This routine really just manages the VCS canvas. It will
popup the VCS Canvas for viewing. It can be used to display the VCS Canvas.
\begin{quote}\begin{description}
\item[{Example}] \leavevmode
\begin{Verbatim}[commandchars=\\\{\}]
\PYG{g+gp}{\PYGZgt{}\PYGZgt{}\PYGZgt{} }\PYG{n}{a}\PYG{o}{=}\PYG{n}{vcs}\PYG{o}{.}\PYG{n}{init}\PYG{p}{(}\PYG{p}{)}
\PYG{g+gp}{\PYGZgt{}\PYGZgt{}\PYGZgt{} }\PYG{n}{a}\PYG{o}{.}\PYG{n}{open}\PYG{p}{(}\PYG{p}{)}
\PYG{g+gp}{\PYGZgt{}\PYGZgt{}\PYGZgt{} }\PYG{n}{a}\PYG{o}{.}\PYG{n}{open}\PYG{p}{(}\PYG{l+m+mi}{800}\PYG{p}{,}\PYG{l+m+mi}{600}\PYG{p}{)}
\end{Verbatim}

\item[{Parameters}] \leavevmode\begin{itemize}
\item {} 
\textbf{\texttt{width}} (\href{https://docs.python.org/2/library/functions.html\#int}{\emph{\texttt{int}}}) -- Integer representing the desire width of the opened window in pixels

\item {} 
\textbf{\texttt{height}} (\href{https://docs.python.org/2/library/functions.html\#int}{\emph{\texttt{int}}}) -- Integer representing the desire height of the opened window in pixels

\end{itemize}

\end{description}\end{quote}

\end{fulllineitems}

\index{orientation() (vcs.Canvas.Canvas method)}

\begin{fulllineitems}
\phantomsection\label{vcs/Canvas:vcs.Canvas.Canvas.orientation}\pysiglinewithargsret{\sphinxbfcode{orientation}}{\emph{*args}, \emph{**kargs}}{}
Return canvas orientation.

The current implementation does not use any args or kargs.
\begin{quote}\begin{description}
\item[{Example}] \leavevmode
\begin{Verbatim}[commandchars=\\\{\}]
\PYG{g+gp}{\PYGZgt{}\PYGZgt{}\PYGZgt{} }\PYG{n}{a}\PYG{o}{=}\PYG{n}{vcs}\PYG{o}{.}\PYG{n}{init}\PYG{p}{(}\PYG{p}{)}
\PYG{g+gp}{\PYGZgt{}\PYGZgt{}\PYGZgt{} }\PYG{n}{a}\PYG{o}{.}\PYG{n}{orientation}\PYG{p}{(}\PYG{p}{)} \PYG{c+c1}{\PYGZsh{} Show current orientation of the canvas}
\PYG{g+go}{\PYGZsq{}landscape\PYGZsq{}}
\end{Verbatim}

\item[{Returns}] \leavevmode
A string indicating the orientation of the canvas, i.e. `landscape' or `portrait'

\item[{Return type}] \leavevmode
\href{https://docs.python.org/2/library/functions.html\#str}{str}

\end{description}\end{quote}

\end{fulllineitems}

\index{pdf() (vcs.Canvas.Canvas method)}

\begin{fulllineitems}
\phantomsection\label{vcs/Canvas:vcs.Canvas.Canvas.pdf}\pysiglinewithargsret{\sphinxbfcode{pdf}}{\emph{file}, \emph{width=None}, \emph{height=None}, \emph{units='inches'}, \emph{textAsPaths=True}}{}
PDF output is another form of vector graphics.

\begin{notice}{note}{Note:}
The textAsPaths parameter preserves custom fonts, but text can no longer be edited in the file
\end{notice}
\begin{quote}\begin{description}
\item[{Example}] \leavevmode
\begin{Verbatim}[commandchars=\\\{\}]
\PYG{g+gp}{\PYGZgt{}\PYGZgt{}\PYGZgt{} }\PYG{n}{a}\PYG{o}{=}\PYG{n}{vcs}\PYG{o}{.}\PYG{n}{init}\PYG{p}{(}\PYG{p}{)}
\PYG{g+gp}{\PYGZgt{}\PYGZgt{}\PYGZgt{} }\PYG{n}{array} \PYG{o}{=} \PYG{p}{[}\PYG{n+nb}{range}\PYG{p}{(}\PYG{l+m+mi}{1}\PYG{p}{,} \PYG{l+m+mi}{11}\PYG{p}{)} \PYG{k}{for} \PYG{n}{\PYGZus{}} \PYG{o+ow}{in} \PYG{n+nb}{range}\PYG{p}{(}\PYG{l+m+mi}{1}\PYG{p}{,} \PYG{l+m+mi}{11}\PYG{p}{)}\PYG{p}{]}
\PYG{g+gp}{\PYGZgt{}\PYGZgt{}\PYGZgt{} }\PYG{n}{a}\PYG{o}{.}\PYG{n}{plot}\PYG{p}{(}\PYG{n}{array}\PYG{p}{)}
\PYG{g+go}{\PYGZlt{}vcs.displayplot.Dp ...\PYGZgt{}}
\PYG{g+gp}{\PYGZgt{}\PYGZgt{}\PYGZgt{} }\PYG{n}{a}\PYG{o}{.}\PYG{n}{pdf}\PYG{p}{(}\PYG{l+s+s1}{\PYGZsq{}}\PYG{l+s+s1}{example}\PYG{l+s+s1}{\PYGZsq{}}\PYG{p}{)} \PYG{c+c1}{\PYGZsh{} Overwrite a postscript file}
\PYG{g+gp}{\PYGZgt{}\PYGZgt{}\PYGZgt{} }\PYG{n}{a}\PYG{o}{.}\PYG{n}{pdf}\PYG{p}{(}\PYG{l+s+s1}{\PYGZsq{}}\PYG{l+s+s1}{example}\PYG{l+s+s1}{\PYGZsq{}}\PYG{p}{,} \PYG{n}{width}\PYG{o}{=}\PYG{l+m+mf}{11.5}\PYG{p}{,} \PYG{n}{height}\PYG{o}{=} \PYG{l+m+mf}{8.5}\PYG{p}{)} \PYG{c+c1}{\PYGZsh{} US Legal}
\PYG{g+gp}{\PYGZgt{}\PYGZgt{}\PYGZgt{} }\PYG{n}{a}\PYG{o}{.}\PYG{n}{pdf}\PYG{p}{(}\PYG{l+s+s1}{\PYGZsq{}}\PYG{l+s+s1}{example}\PYG{l+s+s1}{\PYGZsq{}}\PYG{p}{,} \PYG{n}{width}\PYG{o}{=}\PYG{l+m+mi}{21}\PYG{p}{,} \PYG{n}{height}\PYG{o}{=}\PYG{l+m+mf}{29.7}\PYG{p}{,} \PYG{n}{units}\PYG{o}{=}\PYG{l+s+s1}{\PYGZsq{}}\PYG{l+s+s1}{cm}\PYG{l+s+s1}{\PYGZsq{}}\PYG{p}{)} \PYG{c+c1}{\PYGZsh{} A4}
\end{Verbatim}

\item[{Parameters}] \leavevmode\begin{itemize}
\item {} 
\textbf{\texttt{file}} (\href{https://docs.python.org/2/library/functions.html\#str}{\emph{\texttt{str}}}) -- Desired string name of the output file

\item {} 
\textbf{\texttt{width}} (\href{https://docs.python.org/2/library/functions.html\#int}{\emph{\texttt{int}}}) -- Integer specifying the desired width of the output, measured in the chosen units

\item {} 
\textbf{\texttt{height}} (\href{https://docs.python.org/2/library/functions.html\#int}{\emph{\texttt{int}}}) -- Integer specifying the desired height of the output, measured in the chosen units

\item {} 
\textbf{\texttt{units}} (\href{https://docs.python.org/2/library/functions.html\#str}{\emph{\texttt{str}}}) -- Must be one of {[}'inches', `in', `cm', `mm', `pixel', `pixels', `dot', `dots'{]}. Default is `inches'.

\item {} 
\textbf{\texttt{textAsPaths}} (\href{https://docs.python.org/2/library/functions.html\#bool}{\emph{\texttt{bool}}}) -- Specifies whether to render text objects as paths.

\end{itemize}

\end{description}\end{quote}

\end{fulllineitems}

\index{plot() (vcs.Canvas.Canvas method)}

\begin{fulllineitems}
\phantomsection\label{vcs/Canvas:vcs.Canvas.Canvas.plot}\pysiglinewithargsret{\sphinxbfcode{plot}}{\emph{*actual\_args}, \emph{**keyargs}}{}~\begin{quote}

Plot an array(s) of data given a template and graphics method. The VCS template is
used to define where the data and variable attributes will be displayed on the VCS
Canvas. The VCS graphics method is used to define how the array(s) will be shown
on the VCS Canvas.


\begin{fulllineitems}
\pysigline{\sphinxbfcode{Plot~Usage:}}~
\begin{Verbatim}[commandchars=\\\{\}]
\PYG{n}{plot}\PYG{p}{(}\PYG{n}{array1}\PYG{o}{=}\PYG{n+nb+bp}{None}\PYG{p}{,} \PYG{n}{array2}\PYG{o}{=}\PYG{n+nb+bp}{None}\PYG{p}{,} \PYG{n}{template\PYGZus{}name}\PYG{o}{=}\PYG{n+nb+bp}{None}\PYG{p}{,}
    \PYG{n}{graphics\PYGZus{}method}\PYG{o}{=}\PYG{n+nb+bp}{None}\PYG{p}{,} \PYG{n}{graphics\PYGZus{}name}\PYG{o}{=}\PYG{n+nb+bp}{None}\PYG{p}{,}
    \PYG{p}{[}\PYG{n}{key}\PYG{o}{=}\PYG{n}{value} \PYG{p}{[}\PYG{p}{,} \PYG{n}{key}\PYG{o}{=}\PYG{n}{value} \PYG{p}{[}\PYG{p}{,} \PYG{o}{.}\PYG{o}{.}\PYG{o}{.}\PYG{p}{]}\PYG{p}{]}\PYG{p}{]}\PYG{p}{)}
\end{Verbatim}

\begin{notice}{note}{Note:}
array1 and array2 are NumPy arrays.
\end{notice}

\end{fulllineitems}



\begin{fulllineitems}
\pysigline{\sphinxbfcode{Plot~attribute~keywords:}}~
\begin{notice}{note}{Note:}
More specific attributes take precedence over general attributes. In particular,
specific attributes override variable object attributes, dimension attributes and
arrays override axis objects, which override grid objects, which override variable
objects.

For example, if both `file\_comment' and `variable' keywords are specified, the value of
`file\_comment' is used instead of the file comment in the parent of variable. Similarly,
if both `xaxis' and `grid' keywords are specified, the value of `xaxis' takes precedence
over the x-axis of grid.
\end{notice}
\begin{itemize}
\item {} 
ratio {[}default is none{]}
\begin{itemize}
\item {} 
None: let the self.ratio attribute decide

\item {} 
0,'off': overwrite self.ratio and do nothing about the ratio

\item {} 
`auto': computes an automatic ratio

\item {} 
`3',3: y dim will be 3 times bigger than x dim (restricted to original tempalte.data area

\item {} 
Adding a `t' at the end of the ratio, makes the tickmarks and boxes move along.

\end{itemize}

\item {} 
Dimension attribute keys (dimension length=n):
\begin{itemize}
\item {} 
x or y Dimension values
\begin{quote}

\begin{Verbatim}[commandchars=\\\{\}]
\PYG{p}{[}\PYG{n}{x}\PYG{o}{\textbar{}}\PYG{n}{y}\PYG{o}{\textbar{}}\PYG{n}{z}\PYG{o}{\textbar{}}\PYG{n}{t}\PYG{o}{\textbar{}}\PYG{n}{w}\PYG{p}{]}\PYG{n}{array} \PYG{o}{=} \PYG{n}{NumPy} \PYG{n}{array} \PYG{n}{of} \PYG{n}{length} \PYG{n}{n}
\PYG{p}{[}\PYG{n}{x}\PYG{o}{\textbar{}}\PYG{n}{y}\PYG{o}{\textbar{}}\PYG{n}{z}\PYG{o}{\textbar{}}\PYG{n}{t}\PYG{o}{\textbar{}}\PYG{n}{w}\PYG{p}{]}\PYG{n}{array} \PYG{o}{=} \PYG{n}{NumPy} \PYG{n}{array} \PYG{n}{of} \PYG{n}{length} \PYG{n}{n}
\end{Verbatim}
\end{quote}

\item {} 
x or y Dimension boundaries
\begin{quote}

\begin{Verbatim}[commandchars=\\\{\}]
\PYG{p}{[}\PYG{n}{x}\PYG{o}{\textbar{}}\PYG{n}{y}\PYG{p}{]}\PYG{n}{bounds} \PYG{o}{=} \PYG{n}{NumPy} \PYG{n}{array} \PYG{n}{of} \PYG{n}{shape} \PYG{p}{(}\PYG{n}{n}\PYG{p}{,}\PYG{l+m+mi}{2}\PYG{p}{)}
\end{Verbatim}
\end{quote}

\end{itemize}

\item {} 
CDMS object:
\begin{itemize}
\item {} 
x or y Axis
\begin{quote}

\begin{Verbatim}[commandchars=\\\{\}]
\PYG{p}{[}\PYG{n}{x}\PYG{o}{\textbar{}}\PYG{n}{y}\PYG{o}{\textbar{}}\PYG{n}{z}\PYG{o}{\textbar{}}\PYG{n}{t}\PYG{o}{\textbar{}}\PYG{n}{w}\PYG{p}{]}\PYG{n}{axis} \PYG{o}{=} \PYG{n}{CDMS} \PYG{n}{axis} \PYG{n+nb}{object}
\end{Verbatim}
\end{quote}

\item {} 
Grid object (e.g. grid=var.getGrid())
\begin{quote}

\begin{Verbatim}[commandchars=\\\{\}]
\PYG{n}{grid} \PYG{o}{=} \PYG{n}{CDMS} \PYG{n}{grid} \PYG{n+nb}{object}
\end{Verbatim}
\end{quote}

\item {} 
Variable object
\begin{quote}

\begin{Verbatim}[commandchars=\\\{\}]
\PYG{n}{variable} \PYG{o}{=} \PYG{n}{CDMS} \PYG{n}{variable} \PYG{n+nb}{object}
\end{Verbatim}
\end{quote}

\end{itemize}

\item {} 
Other:
\begin{itemize}
\item {} 
Reverse the direction of the x or y axis:
\begin{quote}

\begin{Verbatim}[commandchars=\\\{\}]
\PYG{p}{[}\PYG{n}{x}\PYG{o}{\textbar{}}\PYG{n}{y}\PYG{p}{]}\PYG{n}{rev} \PYG{o}{=} \PYG{l+m+mi}{0}\PYG{o}{\textbar{}}\PYG{l+m+mi}{1}
\end{Verbatim}

\begin{notice}{note}{Note:}
For example, xrev = 1 would reverse the
direction of the x-axis
\end{notice}
\end{quote}

\item {} 
Continental outlines:
\begin{quote}

\begin{Verbatim}[commandchars=\\\{\}]
\PYG{n}{continents} \PYG{o}{=} \PYG{l+m+mi}{0}\PYG{p}{,}\PYG{l+m+mi}{1}\PYG{p}{,}\PYG{l+m+mi}{2}\PYG{p}{,}\PYG{l+m+mi}{3}\PYG{p}{,}\PYG{l+m+mi}{4}\PYG{p}{,}\PYG{l+m+mi}{5}\PYG{p}{,}\PYG{l+m+mi}{6}\PYG{p}{,}\PYG{l+m+mi}{7}\PYG{p}{,}\PYG{l+m+mi}{8}\PYG{p}{,}\PYG{l+m+mi}{9}\PYG{p}{,}\PYG{l+m+mi}{10}\PYG{p}{,}\PYG{l+m+mi}{11}
\PYG{c+c1}{\PYGZsh{} VCS line object to define continent appearance}
\PYG{n}{continents\PYGZus{}line} \PYG{o}{=} \PYG{n}{vcs}\PYG{o}{.}\PYG{n}{getline}\PYG{p}{(}\PYG{l+s+s2}{\PYGZdq{}}\PYG{l+s+s2}{default}\PYG{l+s+s2}{\PYGZdq{}}\PYG{p}{)}
\end{Verbatim}

\begin{notice}{note}{Note:}
If continents \textgreater{}=1, plot continental outlines.
By default: plot of xaxis is longitude, yaxis is latitude
-OR- xname is `longitude' and yname is `latitude'
\end{notice}
\begin{itemize}
\item {} 
List of continents-type values (integers from 0-11)
\begin{itemize}
\item {} 
0 signifies ``No Continents''

\item {} 
1 signifies ``Fine Continents''

\item {} 
2 signifies ``Coarse Continents''

\item {} 
3 signifies ``United States''

\item {} 
4 signifies ``Political Borders''

\item {} 
5 signifies ``Rivers''

\end{itemize}

\end{itemize}

\begin{notice}{note}{Note:}
Values 6 through 11 signify the line type defined by
the files data\_continent\_other7 through data\_continent\_other12.
\end{notice}
\end{quote}

\item {} 
To set whether the displayplot object generated by this plot is stored
\begin{quote}

\begin{Verbatim}[commandchars=\\\{\}]
\PYG{n}{donotstoredisplay} \PYG{o}{=} \PYG{n+nb+bp}{True}\PYG{o}{\textbar{}}\PYG{n+nb+bp}{False}
\end{Verbatim}
\end{quote}

\item {} 
Whether to actually render the plot or not (useful for doing a bunch of plots in a row)
\begin{quote}

\begin{Verbatim}[commandchars=\\\{\}]
\PYG{n}{render} \PYG{o}{=} \PYG{n+nb+bp}{True}\PYG{o}{\textbar{}}\PYG{n+nb+bp}{False}
\end{Verbatim}
\end{quote}

\item {} 
VCS Display plot name (used to prevent duplicate display plots)
\begin{quote}

\begin{Verbatim}[commandchars=\\\{\}]
\PYG{n}{display\PYGZus{}name} \PYG{o}{=} \PYG{l+s+s2}{\PYGZdq{}}\PYG{l+s+s2}{\PYGZus{}\PYGZus{}display\PYGZus{}123}\PYG{l+s+s2}{\PYGZdq{}}
\end{Verbatim}
\end{quote}

\item {} 
Ratio of height/width for the plot; autot and auto will choose a ``good'' ratio for you.
\begin{quote}

\begin{Verbatim}[commandchars=\\\{\}]
\PYG{n}{ratio} \PYG{o}{=} \PYG{l+m+mf}{1.5}\PYG{o}{\textbar{}}\PYG{l+s+s2}{\PYGZdq{}}\PYG{l+s+s2}{autot}\PYG{l+s+s2}{\PYGZdq{}}\PYG{o}{\textbar{}}\PYG{l+s+s2}{\PYGZdq{}}\PYG{l+s+s2}{auto}\PYG{l+s+s2}{\PYGZdq{}}
\end{Verbatim}
\end{quote}

\item {} 
Plot the actual grid or the dual grid
\begin{quote}

\begin{Verbatim}[commandchars=\\\{\}]
\PYG{n}{plot\PYGZus{}based\PYGZus{}dual\PYGZus{}grid} \PYG{o}{=} \PYG{n+nb+bp}{True} \PYG{o}{\textbar{}} \PYG{n+nb+bp}{False}
\end{Verbatim}

\begin{notice}{note}{Note:}
This is based on what is needed by the plot: isofill, isoline, vector need
point attributes, boxfill and meshfill need cell attributes
the default is True (if the parameter is not specified).
\end{notice}
\end{quote}

\item {} 
Graphics Output in Background Mode:
\begin{quote}

\begin{Verbatim}[commandchars=\\\{\}]
\PYG{c+c1}{\PYGZsh{} if ==1, create images in the background}
\PYG{n}{bg} \PYG{o}{=} \PYG{l+m+mi}{0}\PYG{o}{\textbar{}}\PYG{l+m+mi}{1}
\end{Verbatim}
\end{quote}

\end{itemize}

\end{itemize}

\end{fulllineitems}

\begin{quote}\begin{description}
\item[{Example}] \leavevmode
\begin{Verbatim}[commandchars=\\\{\}]
\PYG{g+gp}{\PYGZgt{}\PYGZgt{}\PYGZgt{} }\PYG{n}{a}\PYG{o}{=}\PYG{n}{vcs}\PYG{o}{.}\PYG{n}{init}\PYG{p}{(}\PYG{p}{)}
\PYG{g+gp}{\PYGZgt{}\PYGZgt{}\PYGZgt{} }\PYG{k+kn}{import} \PYG{n+nn}{cdms2} \PYG{c+c1}{\PYGZsh{} Need cdms2 to create a slab}
\PYG{g+gp}{\PYGZgt{}\PYGZgt{}\PYGZgt{} }\PYG{n}{f} \PYG{o}{=} \PYG{n}{cdms2}\PYG{o}{.}\PYG{n}{open}\PYG{p}{(}\PYG{n}{vcs}\PYG{o}{.}\PYG{n}{sample\PYGZus{}data}\PYG{o}{+}\PYG{l+s+s1}{\PYGZsq{}}\PYG{l+s+s1}{/clt.nc}\PYG{l+s+s1}{\PYGZsq{}}\PYG{p}{)} \PYG{c+c1}{\PYGZsh{} use cdms2 to open a data file}
\PYG{g+gp}{\PYGZgt{}\PYGZgt{}\PYGZgt{} }\PYG{n}{slab1} \PYG{o}{=} \PYG{n}{f}\PYG{p}{(}\PYG{l+s+s1}{\PYGZsq{}}\PYG{l+s+s1}{u}\PYG{l+s+s1}{\PYGZsq{}}\PYG{p}{)} \PYG{c+c1}{\PYGZsh{} use the data file to create a cdms2 slab}
\PYG{g+gp}{\PYGZgt{}\PYGZgt{}\PYGZgt{} }\PYG{n}{slab2} \PYG{o}{=} \PYG{n}{f}\PYG{p}{(}\PYG{l+s+s1}{\PYGZsq{}}\PYG{l+s+s1}{v}\PYG{l+s+s1}{\PYGZsq{}}\PYG{p}{)} \PYG{c+c1}{\PYGZsh{} need 2 slabs, so get another}
\PYG{g+gp}{\PYGZgt{}\PYGZgt{}\PYGZgt{} }\PYG{n}{a}\PYG{o}{.}\PYG{n}{plot}\PYG{p}{(}\PYG{n}{slab1}\PYG{p}{)} \PYG{c+c1}{\PYGZsh{} this call will use default settings for template and boxfill}
\PYG{g+go}{\PYGZlt{}vcs.displayplot.Dp ...\PYGZgt{}}
\PYG{g+gp}{\PYGZgt{}\PYGZgt{}\PYGZgt{} }\PYG{n}{a}\PYG{o}{.}\PYG{n}{plot}\PYG{p}{(}\PYG{n}{slab1}\PYG{p}{,} \PYG{l+s+s1}{\PYGZsq{}}\PYG{l+s+s1}{polar}\PYG{l+s+s1}{\PYGZsq{}}\PYG{p}{,} \PYG{l+s+s1}{\PYGZsq{}}\PYG{l+s+s1}{isofill}\PYG{l+s+s1}{\PYGZsq{}}\PYG{p}{,} \PYG{l+s+s1}{\PYGZsq{}}\PYG{l+s+s1}{polar}\PYG{l+s+s1}{\PYGZsq{}}\PYG{p}{)} \PYG{c+c1}{\PYGZsh{} this is specifying the template and graphics method}
\PYG{g+go}{\PYGZlt{}vcs.displayplot.Dp ...\PYGZgt{}}
\PYG{g+gp}{\PYGZgt{}\PYGZgt{}\PYGZgt{} }\PYG{n}{t}\PYG{o}{=}\PYG{n}{a}\PYG{o}{.}\PYG{n}{gettemplate}\PYG{p}{(}\PYG{l+s+s1}{\PYGZsq{}}\PYG{l+s+s1}{polar}\PYG{l+s+s1}{\PYGZsq{}}\PYG{p}{)} \PYG{c+c1}{\PYGZsh{} get the polar template}
\PYG{g+gp}{\PYGZgt{}\PYGZgt{}\PYGZgt{} }\PYG{n}{vec}\PYG{o}{=}\PYG{n}{a}\PYG{o}{.}\PYG{n}{getvector}\PYG{p}{(}\PYG{p}{)} \PYG{c+c1}{\PYGZsh{} get default vector}
\PYG{g+gp}{\PYGZgt{}\PYGZgt{}\PYGZgt{} }\PYG{n}{a}\PYG{o}{.}\PYG{n}{plot}\PYG{p}{(}\PYG{n}{slab1}\PYG{p}{,} \PYG{n}{slab2}\PYG{p}{,} \PYG{n}{t}\PYG{p}{,} \PYG{n}{vec}\PYG{p}{)} \PYG{c+c1}{\PYGZsh{} plot the data as a vector using the \PYGZsq{}AMIP\PYGZsq{} template}
\PYG{g+go}{\PYGZlt{}vcs.displayplot.Dp ...\PYGZgt{}}
\PYG{g+gp}{\PYGZgt{}\PYGZgt{}\PYGZgt{} }\PYG{n}{a}\PYG{o}{.}\PYG{n}{clear}\PYG{p}{(}\PYG{p}{)} \PYG{c+c1}{\PYGZsh{} clear the VCS Canvas of all plots}
\PYG{g+gp}{\PYGZgt{}\PYGZgt{}\PYGZgt{} }\PYG{n}{box}\PYG{o}{=}\PYG{n}{a}\PYG{o}{.}\PYG{n}{getboxfill}\PYG{p}{(}\PYG{p}{)} \PYG{c+c1}{\PYGZsh{} get default boxfill graphics method}
\PYG{g+gp}{\PYGZgt{}\PYGZgt{}\PYGZgt{} }\PYG{n}{a}\PYG{o}{.}\PYG{n}{plot}\PYG{p}{(}\PYG{n}{box}\PYG{p}{,}\PYG{n}{t}\PYG{p}{,}\PYG{n}{slab2}\PYG{p}{)} \PYG{c+c1}{\PYGZsh{} plot array data using box \PYGZsq{}new\PYGZsq{} and template \PYGZsq{}t\PYGZsq{}}
\PYG{g+go}{\PYGZlt{}vcs.displayplot.Dp ...\PYGZgt{}}
\end{Verbatim}

\end{description}\end{quote}
\end{quote}
\begin{quote}\begin{description}
\item[{Parameters}] \leavevmode\begin{itemize}
\item {} 
\textbf{\texttt{slab2}} (\href{https://docs.python.org/2/library/array.html\#module-array}{\emph{\texttt{array}}}) -- Data at least 1D, last dimension(s) will be plotted

\item {} 
\textbf{\texttt{template}} (\emph{\texttt{str/vcs.template.P}}) -- (`default') vcs template to use

\item {} 
\textbf{\texttt{gm}} (\emph{\texttt{VCS graphics method object}}) -- (Ex: `default') graphic method to use

\item {} 
\textbf{\texttt{xaxis}} (\emph{\texttt{cdms2.axis.TransientAxis}}) -- Axis object to replace the slab -1 dim axis

\item {} 
\textbf{\texttt{yaxis}} (\emph{\texttt{cdms2.axis.TransientAxis}}) -- Axis object to replace the slab -2 dim axis, only if slab has more than 1D

\item {} 
\textbf{\texttt{zaxis}} (\emph{\texttt{cdms2.axis.TransientAxis}}) -- Axis object to replace the slab -3 dim axis, only if slab has more than 2D

\item {} 
\textbf{\texttt{taxis}} (\emph{\texttt{cdms2.axis.TransientAxis}}) -- Axis object to replace the slab -4 dim axis, only if slab has more than 3D

\item {} 
\textbf{\texttt{waxis}} (\emph{\texttt{cdms2.axis.TransientAxis}}) -- Axis object to replace the slab -5 dim axis, only if slab has more than 4D

\item {} 
\textbf{\texttt{xrev}} (\href{https://docs.python.org/2/library/functions.html\#bool}{\emph{\texttt{bool}}}) -- reverse x axis

\item {} 
\textbf{\texttt{yrev}} (\href{https://docs.python.org/2/library/functions.html\#bool}{\emph{\texttt{bool}}}) -- reverse y axis, only if slab has more than 1D

\item {} 
\textbf{\texttt{xarray}} (\href{https://docs.python.org/2/library/array.html\#module-array}{\emph{\texttt{array}}}) -- Values to use instead of x axis

\item {} 
\textbf{\texttt{yarray}} (\href{https://docs.python.org/2/library/array.html\#module-array}{\emph{\texttt{array}}}) -- Values to use instead of y axis, only if var has more than 1D

\item {} 
\textbf{\texttt{zarray}} (\href{https://docs.python.org/2/library/array.html\#module-array}{\emph{\texttt{array}}}) -- Values to use instead of z axis, only if var has more than 2D

\item {} 
\textbf{\texttt{tarray}} (\href{https://docs.python.org/2/library/array.html\#module-array}{\emph{\texttt{array}}}) -- Values to use instead of t axis, only if var has more than 3D

\item {} 
\textbf{\texttt{warray}} (\href{https://docs.python.org/2/library/array.html\#module-array}{\emph{\texttt{array}}}) -- Values to use instead of w axis, only if var has more than 4D

\item {} 
\textbf{\texttt{continents}} (\href{https://docs.python.org/2/library/functions.html\#int}{\emph{\texttt{int}}}) -- continents type number

\item {} 
\textbf{\texttt{name}} (\href{https://docs.python.org/2/library/functions.html\#str}{\emph{\texttt{str}}}) -- replaces variable name on plot

\item {} 
\textbf{\texttt{time}} (\emph{\texttt{A cdtime object}}) -- replaces time name on plot

\item {} 
\textbf{\texttt{units}} (\href{https://docs.python.org/2/library/functions.html\#str}{\emph{\texttt{str}}}) -- replaces units value on plot

\item {} 
\textbf{\texttt{ymd}} (\href{https://docs.python.org/2/library/functions.html\#str}{\emph{\texttt{str}}}) -- replaces year/month/day on plot

\item {} 
\textbf{\texttt{hms}} (\href{https://docs.python.org/2/library/functions.html\#str}{\emph{\texttt{str}}}) -- replaces hh/mm/ss on plot

\item {} 
\textbf{\texttt{file\_comment}} (\href{https://docs.python.org/2/library/functions.html\#str}{\emph{\texttt{str}}}) -- replaces file\_comment on plot

\item {} 
\textbf{\texttt{xbounds}} (\href{https://docs.python.org/2/library/array.html\#module-array}{\emph{\texttt{array}}}) -- Values to use instead of x axis bounds values

\item {} 
\textbf{\texttt{ybounds}} (\href{https://docs.python.org/2/library/array.html\#module-array}{\emph{\texttt{array}}}) -- Values to use instead of y axis bounds values (if exist)

\item {} 
\textbf{\texttt{xname}} (\href{https://docs.python.org/2/library/functions.html\#str}{\emph{\texttt{str}}}) -- replace xaxis name on plot

\item {} 
\textbf{\texttt{yname}} (\href{https://docs.python.org/2/library/functions.html\#str}{\emph{\texttt{str}}}) -- replace yaxis name on plot (if exists)

\item {} 
\textbf{\texttt{zname}} (\href{https://docs.python.org/2/library/functions.html\#str}{\emph{\texttt{str}}}) -- replace zaxis name on plot (if exists)

\item {} 
\textbf{\texttt{tname}} (\href{https://docs.python.org/2/library/functions.html\#str}{\emph{\texttt{str}}}) -- replace taxis name on plot (if exists)

\item {} 
\textbf{\texttt{wname}} (\href{https://docs.python.org/2/library/functions.html\#str}{\emph{\texttt{str}}}) -- replace waxis name on plot (if exists)

\item {} 
\textbf{\texttt{xunits}} (\href{https://docs.python.org/2/library/functions.html\#str}{\emph{\texttt{str}}}) -- replace xaxis units on plot

\item {} 
\textbf{\texttt{yunits}} (\href{https://docs.python.org/2/library/functions.html\#str}{\emph{\texttt{str}}}) -- replace yaxis units on plot (if exists)

\item {} 
\textbf{\texttt{zunits}} (\href{https://docs.python.org/2/library/functions.html\#str}{\emph{\texttt{str}}}) -- replace zaxis units on plot (if exists)

\item {} 
\textbf{\texttt{tunits}} (\href{https://docs.python.org/2/library/functions.html\#str}{\emph{\texttt{str}}}) -- replace taxis units on plot (if exists)

\item {} 
\textbf{\texttt{wunits}} (\href{https://docs.python.org/2/library/functions.html\#str}{\emph{\texttt{str}}}) -- replace waxis units on plot (if exists)

\item {} 
\textbf{\texttt{xweights}} (\href{https://docs.python.org/2/library/array.html\#module-array}{\emph{\texttt{array}}}) -- replace xaxis weights used for computing mean

\item {} 
\textbf{\texttt{yweights}} (\href{https://docs.python.org/2/library/array.html\#module-array}{\emph{\texttt{array}}}) -- replace xaxis weights used for computing mean

\item {} 
\textbf{\texttt{comment1}} (\href{https://docs.python.org/2/library/functions.html\#str}{\emph{\texttt{str}}}) -- replaces comment1 on plot

\item {} 
\textbf{\texttt{comment2}} (\href{https://docs.python.org/2/library/functions.html\#str}{\emph{\texttt{str}}}) -- replaces comment2 on plot

\item {} 
\textbf{\texttt{comment3}} (\href{https://docs.python.org/2/library/functions.html\#str}{\emph{\texttt{str}}}) -- replaces comment3 on plot

\item {} 
\textbf{\texttt{comment4}} (\href{https://docs.python.org/2/library/functions.html\#str}{\emph{\texttt{str}}}) -- replaces comment4 on plot

\item {} 
\textbf{\texttt{long\_name}} (\href{https://docs.python.org/2/library/functions.html\#str}{\emph{\texttt{str}}}) -- replaces long\_name on plot

\item {} 
\textbf{\texttt{grid}} (\emph{\texttt{cdms2.grid.TransientRectGrid}}) -- replaces array grid (if exists)

\item {} 
\textbf{\texttt{bg}} (\emph{\texttt{bool/int}}) -- plots in background mode

\item {} 
\textbf{\texttt{ratio}} (\index{xmtics1 (vcs.Canvas.Canvas attribute)}\index{xmtics2 (vcs.Canvas.Canvas attribute)}\index{ymtics1 (vcs.Canvas.Canvas attribute)}\index{ymtics2 (vcs.Canvas.Canvas attribute)}\index{xticlabels1 (vcs.Canvas.Canvas attribute)}\index{xticlabels2 (vcs.Canvas.Canvas attribute)}\index{yticlabels1 (vcs.Canvas.Canvas attribute)}\index{yticlabels2 (vcs.Canvas.Canvas attribute)}\index{projection (vcs.Canvas.Canvas attribute)}\index{datawc\_x1 (vcs.Canvas.Canvas attribute)}\index{datawc\_x2 (vcs.Canvas.Canvas attribute)}\index{datawc\_y1 (vcs.Canvas.Canvas attribute)}\index{datawc\_y2 (vcs.Canvas.Canvas attribute)}\index{datawc\_timeunits (vcs.Canvas.Canvas attribute)}\index{datawc\_calendar (vcs.Canvas.Canvas attribute)}) -- sets the y/x ratio ,if passed as a string with `t' at the end, will aslo moves the ticks

\item {} 
\textbf{\texttt{xaxisconvert}} (\href{https://docs.python.org/2/library/functions.html\#str}{\emph{\texttt{str}}}) -- (Ex: `linear') converting xaxis linear/log/log10/ln/exp/area\_wt

\item {} 
\textbf{\texttt{yaxisconvert}} (\href{https://docs.python.org/2/library/functions.html\#str}{\emph{\texttt{str}}}) -- (Ex: `linear') converting yaxis linear/log/log10/ln/exp/area\_wt

\item {} 
\textbf{\texttt{slab\_or\_primary\_object}} (\href{https://docs.python.org/2/library/array.html\#module-array}{\emph{\texttt{array}}}) -- Data at least 1D, last dimension(s) will be plotted, or secondary vcs object

\end{itemize}

\item[{Returns}] \leavevmode
Display Plot object representing the plot.

\item[{Return type}] \leavevmode

vcs.displayplot.Dp
\begin{quote}\begin{description}
\item[{returns}] \leavevmode
A VCS display plot object

\item[{rtype}] \leavevmode
vcs.displayplot.Dp

\end{description}\end{quote}


\end{description}\end{quote}

\end{fulllineitems}

\index{png() (vcs.Canvas.Canvas method)}

\begin{fulllineitems}
\phantomsection\label{vcs/Canvas:vcs.Canvas.Canvas.png}\pysiglinewithargsret{\sphinxbfcode{png}}{\emph{file}, \emph{width=None}, \emph{height=None}, \emph{units=None}, \emph{draw\_white\_background=True}, \emph{**args}}{}
PNG output, dimensions set via setbgoutputdimensions
\begin{quote}\begin{description}
\item[{Example}] \leavevmode
\begin{Verbatim}[commandchars=\\\{\}]
\PYG{g+gp}{\PYGZgt{}\PYGZgt{}\PYGZgt{} }\PYG{n}{a}\PYG{o}{=}\PYG{n}{vcs}\PYG{o}{.}\PYG{n}{init}\PYG{p}{(}\PYG{p}{)}
\PYG{g+gp}{\PYGZgt{}\PYGZgt{}\PYGZgt{} }\PYG{n}{array} \PYG{o}{=} \PYG{p}{[}\PYG{n+nb}{range}\PYG{p}{(}\PYG{l+m+mi}{1}\PYG{p}{,} \PYG{l+m+mi}{11}\PYG{p}{)} \PYG{k}{for} \PYG{n}{\PYGZus{}} \PYG{o+ow}{in} \PYG{n+nb}{range}\PYG{p}{(}\PYG{l+m+mi}{1}\PYG{p}{,} \PYG{l+m+mi}{11}\PYG{p}{)}\PYG{p}{]}
\PYG{g+gp}{\PYGZgt{}\PYGZgt{}\PYGZgt{} }\PYG{n}{a}\PYG{o}{.}\PYG{n}{plot}\PYG{p}{(}\PYG{n}{array}\PYG{p}{)}
\PYG{g+go}{\PYGZlt{}vcs.displayplot.Dp ...\PYGZgt{}}
\PYG{g+gp}{\PYGZgt{}\PYGZgt{}\PYGZgt{} }\PYG{n}{a}\PYG{o}{.}\PYG{n}{png}\PYG{p}{(}\PYG{l+s+s1}{\PYGZsq{}}\PYG{l+s+s1}{example}\PYG{l+s+s1}{\PYGZsq{}}\PYG{p}{)} \PYG{c+c1}{\PYGZsh{} Overwrite a png file}
\end{Verbatim}

\item[{Parameters}] \leavevmode\begin{itemize}
\item {} 
\textbf{\texttt{file}} (\href{https://docs.python.org/2/library/functions.html\#str}{\emph{\texttt{str}}}) -- Output image filename

\item {} 
\textbf{\texttt{width}} (\href{https://docs.python.org/2/library/functions.html\#float}{\emph{\texttt{float}}}) -- Float representing the desired width of the output png, using the specified unit of measurement

\item {} 
\textbf{\texttt{height}} (\href{https://docs.python.org/2/library/functions.html\#float}{\emph{\texttt{float}}}) -- Float representing the desired height of the output png, using the specified unit of measurement

\item {} 
\textbf{\texttt{units}} (\href{https://docs.python.org/2/library/functions.html\#str}{\emph{\texttt{str}}}) -- One of {[}'inches', `in', `cm', `mm', `pixel', `pixels', `dot', `dots'{]}. Defaults to `inches'.

\item {} 
\textbf{\texttt{draw\_white\_background}} (\href{https://docs.python.org/2/library/functions.html\#bool}{\emph{\texttt{bool}}}) -- Boolean value indicating if the background should be white. Defaults to True.

\end{itemize}

\end{description}\end{quote}

\end{fulllineitems}

\index{portrait() (vcs.Canvas.Canvas method)}

\begin{fulllineitems}
\phantomsection\label{vcs/Canvas:vcs.Canvas.Canvas.portrait}\pysiglinewithargsret{\sphinxbfcode{portrait}}{\emph{width=-99}, \emph{height=-99}, \emph{x=-99}, \emph{y=-99}, \emph{clear=0}}{}
Change the VCS Canvas orientation to Portrait.

\begin{notice}{note}{Note:}
If the current orientation of the canvas is already portrait, nothing happens.
\end{notice}
\begin{quote}\begin{description}
\item[{Example}] \leavevmode
\begin{Verbatim}[commandchars=\\\{\}]
\PYG{g+gp}{\PYGZgt{}\PYGZgt{}\PYGZgt{} }\PYG{n}{a}\PYG{o}{=}\PYG{n}{vcs}\PYG{o}{.}\PYG{n}{init}\PYG{p}{(}\PYG{p}{)}
\PYG{g+gp}{\PYGZgt{}\PYGZgt{}\PYGZgt{} }\PYG{n}{array} \PYG{o}{=} \PYG{p}{[}\PYG{n+nb}{range}\PYG{p}{(}\PYG{l+m+mi}{1}\PYG{p}{,} \PYG{l+m+mi}{11}\PYG{p}{)} \PYG{k}{for} \PYG{n}{\PYGZus{}} \PYG{o+ow}{in} \PYG{n+nb}{range}\PYG{p}{(}\PYG{l+m+mi}{1}\PYG{p}{,} \PYG{l+m+mi}{11}\PYG{p}{)}\PYG{p}{]}
\PYG{g+gp}{\PYGZgt{}\PYGZgt{}\PYGZgt{} }\PYG{n}{a}\PYG{o}{.}\PYG{n}{plot}\PYG{p}{(}\PYG{n}{array}\PYG{p}{)}
\PYG{g+go}{\PYGZlt{}vcs.displayplot.Dp ...\PYGZgt{}}
\PYG{g+gp}{\PYGZgt{}\PYGZgt{}\PYGZgt{} }\PYG{n}{a}\PYG{o}{.}\PYG{n}{portrait}\PYG{p}{(}\PYG{p}{)}  \PYG{c+c1}{\PYGZsh{} Change the VCS Canvas orientation and set object flag to portrait}
\PYG{g+gp}{\PYGZgt{}\PYGZgt{}\PYGZgt{} }\PYG{n}{a}\PYG{o}{.}\PYG{n}{portrait}\PYG{p}{(}\PYG{n}{clear}\PYG{o}{=}\PYG{l+m+mi}{1}\PYG{p}{)} \PYG{c+c1}{\PYGZsh{} Change the VCS Canvas to portrait and clear the page}
\PYG{g+gp}{\PYGZgt{}\PYGZgt{}\PYGZgt{} }\PYG{n}{a}\PYG{o}{.}\PYG{n}{portrait}\PYG{p}{(}\PYG{n}{width} \PYG{o}{=} \PYG{l+m+mi}{337}\PYG{p}{,} \PYG{n}{height} \PYG{o}{=} \PYG{l+m+mi}{400}\PYG{p}{)} \PYG{c+c1}{\PYGZsh{} Change to portrait and set the window size}
\PYG{g+gp}{\PYGZgt{}\PYGZgt{}\PYGZgt{} }\PYG{n}{a}\PYG{o}{.}\PYG{n}{portrait}\PYG{p}{(}\PYG{n}{x}\PYG{o}{=}\PYG{l+m+mi}{100}\PYG{p}{,} \PYG{n}{y} \PYG{o}{=} \PYG{l+m+mi}{200}\PYG{p}{)} \PYG{c+c1}{\PYGZsh{} Change to portrait and set the x and y screen position}
\PYG{g+gp}{\PYGZgt{}\PYGZgt{}\PYGZgt{} }\PYG{n}{a}\PYG{o}{.}\PYG{n}{portrait}\PYG{p}{(}\PYG{n}{width} \PYG{o}{=} \PYG{l+m+mi}{337}\PYG{p}{,} \PYG{n}{height} \PYG{o}{=} \PYG{l+m+mi}{400}\PYG{p}{,} \PYG{n}{x}\PYG{o}{=}\PYG{l+m+mi}{100}\PYG{p}{,} \PYG{n}{y} \PYG{o}{=} \PYG{l+m+mi}{200}\PYG{p}{,} \PYG{n}{clear}\PYG{o}{=}\PYG{l+m+mi}{1}\PYG{p}{)} \PYG{c+c1}{\PYGZsh{} portrait, with more specifications}
\end{Verbatim}

\item[{Parameters}] \leavevmode\begin{itemize}
\item {} 
\textbf{\texttt{width}} (\href{https://docs.python.org/2/library/functions.html\#int}{\emph{\texttt{int}}}) -- Width to set the canvas to (in pixels)

\item {} 
\textbf{\texttt{height}} (\href{https://docs.python.org/2/library/functions.html\#int}{\emph{\texttt{int}}}) -- Height to set the canvas to (in pixels)

\item {} 
\textbf{\texttt{x}} (\href{https://docs.python.org/2/library/constants.html\#None}{\emph{\texttt{None}}}) -- Unused.

\item {} 
\textbf{\texttt{y}} (\href{https://docs.python.org/2/library/constants.html\#None}{\emph{\texttt{None}}}) -- Unused.

\item {} 
\textbf{\texttt{clear}} (\href{https://docs.python.org/2/library/functions.html\#int}{\emph{\texttt{int}}}) -- 0: Do not clear the canvas when orientation is changed.
1: clear the canvas when orientation is changed.

\end{itemize}

\end{description}\end{quote}

\end{fulllineitems}

\index{postscript() (vcs.Canvas.Canvas method)}

\begin{fulllineitems}
\phantomsection\label{vcs/Canvas:vcs.Canvas.Canvas.postscript}\pysiglinewithargsret{\sphinxbfcode{postscript}}{\emph{file}, \emph{mode='r'}, \emph{orientation=None}, \emph{width=None}, \emph{height=None}, \emph{units='inches'}, \emph{textAsPaths=True}}{}
Postscript output is another form of vector graphics. It is larger than its CGM output
counter part, because it is stored out in ASCII format.

There are two modes for saving a postscript file: {\color{red}\bfseries{}{}`}Append' (a) mode appends postscript
output to an existing postscript file; and {\color{red}\bfseries{}{}`}Replace' (r) mode overwrites an existing
postscript file with new postscript output. The default mode is to overwrite an existing
postscript file (i.e. mode (r)).

\begin{notice}{note}{Note:}
The textAsPaths parameter preserves custom fonts, but text can no longer be edited in the file
\end{notice}
\begin{quote}\begin{description}
\item[{Example}] \leavevmode
\begin{Verbatim}[commandchars=\\\{\}]
\PYG{g+gp}{\PYGZgt{}\PYGZgt{}\PYGZgt{} }\PYG{n}{a}\PYG{o}{=}\PYG{n}{vcs}\PYG{o}{.}\PYG{n}{init}\PYG{p}{(}\PYG{p}{)}
\PYG{g+gp}{\PYGZgt{}\PYGZgt{}\PYGZgt{} }\PYG{n}{array} \PYG{o}{=} \PYG{p}{[}\PYG{n+nb}{range}\PYG{p}{(}\PYG{l+m+mi}{1}\PYG{p}{,} \PYG{l+m+mi}{11}\PYG{p}{)} \PYG{k}{for} \PYG{n}{\PYGZus{}} \PYG{o+ow}{in} \PYG{n+nb}{range}\PYG{p}{(}\PYG{l+m+mi}{1}\PYG{p}{,} \PYG{l+m+mi}{11}\PYG{p}{)}\PYG{p}{]}
\PYG{g+gp}{\PYGZgt{}\PYGZgt{}\PYGZgt{} }\PYG{n}{a}\PYG{o}{.}\PYG{n}{plot}\PYG{p}{(}\PYG{n}{array}\PYG{p}{)}
\PYG{g+go}{\PYGZlt{}vcs.displayplot.Dp ...\PYGZgt{}}
\PYG{g+gp}{\PYGZgt{}\PYGZgt{}\PYGZgt{} }\PYG{n}{a}\PYG{o}{.}\PYG{n}{postscript}\PYG{p}{(}\PYG{l+s+s1}{\PYGZsq{}}\PYG{l+s+s1}{example}\PYG{l+s+s1}{\PYGZsq{}}\PYG{p}{)} \PYG{c+c1}{\PYGZsh{} Overwrite a postscript file}
\PYG{g+gp}{\PYGZgt{}\PYGZgt{}\PYGZgt{} }\PYG{n}{a}\PYG{o}{.}\PYG{n}{postscript}\PYG{p}{(}\PYG{l+s+s1}{\PYGZsq{}}\PYG{l+s+s1}{example}\PYG{l+s+s1}{\PYGZsq{}}\PYG{p}{,} \PYG{l+s+s1}{\PYGZsq{}}\PYG{l+s+s1}{a}\PYG{l+s+s1}{\PYGZsq{}}\PYG{p}{)} \PYG{c+c1}{\PYGZsh{} Append postscript to an existing file}
\PYG{g+gp}{\PYGZgt{}\PYGZgt{}\PYGZgt{} }\PYG{n}{a}\PYG{o}{.}\PYG{n}{postscript}\PYG{p}{(}\PYG{l+s+s1}{\PYGZsq{}}\PYG{l+s+s1}{example}\PYG{l+s+s1}{\PYGZsq{}}\PYG{p}{,} \PYG{l+s+s1}{\PYGZsq{}}\PYG{l+s+s1}{r}\PYG{l+s+s1}{\PYGZsq{}}\PYG{p}{)} \PYG{c+c1}{\PYGZsh{} Overwrite an existing file}
\PYG{g+gp}{\PYGZgt{}\PYGZgt{}\PYGZgt{} }\PYG{n}{a}\PYG{o}{.}\PYG{n}{postscript}\PYG{p}{(}\PYG{l+s+s1}{\PYGZsq{}}\PYG{l+s+s1}{example}\PYG{l+s+s1}{\PYGZsq{}}\PYG{p}{,} \PYG{n}{mode}\PYG{o}{=}\PYG{l+s+s1}{\PYGZsq{}}\PYG{l+s+s1}{a}\PYG{l+s+s1}{\PYGZsq{}}\PYG{p}{)} \PYG{c+c1}{\PYGZsh{} Append postscript to an existing file}
\PYG{g+gp}{\PYGZgt{}\PYGZgt{}\PYGZgt{} }\PYG{n}{a}\PYG{o}{.}\PYG{n}{postscript}\PYG{p}{(}\PYG{l+s+s1}{\PYGZsq{}}\PYG{l+s+s1}{example}\PYG{l+s+s1}{\PYGZsq{}}\PYG{p}{,} \PYG{n}{width}\PYG{o}{=}\PYG{l+m+mf}{11.5}\PYG{p}{,} \PYG{n}{height}\PYG{o}{=} \PYG{l+m+mf}{8.5}\PYG{p}{)} \PYG{c+c1}{\PYGZsh{} US Legal (default)}
\PYG{g+gp}{\PYGZgt{}\PYGZgt{}\PYGZgt{} }\PYG{n}{a}\PYG{o}{.}\PYG{n}{postscript}\PYG{p}{(}\PYG{l+s+s1}{\PYGZsq{}}\PYG{l+s+s1}{example}\PYG{l+s+s1}{\PYGZsq{}}\PYG{p}{,} \PYG{n}{width}\PYG{o}{=}\PYG{l+m+mi}{21}\PYG{p}{,} \PYG{n}{height}\PYG{o}{=}\PYG{l+m+mf}{29.7}\PYG{p}{,} \PYG{n}{units}\PYG{o}{=}\PYG{l+s+s1}{\PYGZsq{}}\PYG{l+s+s1}{cm}\PYG{l+s+s1}{\PYGZsq{}}\PYG{p}{)} \PYG{c+c1}{\PYGZsh{} A4}
\end{Verbatim}

\item[{Parameters}] \leavevmode\begin{itemize}
\item {} 
\textbf{\texttt{file}} (\href{https://docs.python.org/2/library/functions.html\#str}{\emph{\texttt{str}}}) -- String name of the desired output file

\item {} 
\textbf{\texttt{mode}} (\href{https://docs.python.org/2/library/functions.html\#str}{\emph{\texttt{str}}}) -- The mode in which to open the file. One of `r' or `a'.

\item {} 
\textbf{\texttt{orientation}} (\href{https://docs.python.org/2/library/constants.html\#None}{\emph{\texttt{None}}}) -- Deprecated.

\item {} 
\textbf{\texttt{width}} (\href{https://docs.python.org/2/library/functions.html\#int}{\emph{\texttt{int}}}) -- Desired width of the postscript output, in the specified unit of measurement

\item {} 
\textbf{\texttt{height}} (\href{https://docs.python.org/2/library/functions.html\#int}{\emph{\texttt{int}}}) -- Desired height of the postscript output, in the specified unit of measurement

\item {} 
\textbf{\texttt{textAsPaths}} (\href{https://docs.python.org/2/library/functions.html\#bool}{\emph{\texttt{bool}}}) -- Specifies whether to render text objects as paths.

\end{itemize}

\end{description}\end{quote}

\end{fulllineitems}

\index{pstogif() (vcs.Canvas.Canvas method)}

\begin{fulllineitems}
\phantomsection\label{vcs/Canvas:vcs.Canvas.Canvas.pstogif}\pysiglinewithargsret{\sphinxbfcode{pstogif}}{\emph{filename}, \emph{*opt}}{}
In some cases, the user may want to save the plot out as a gif image. This
routine allows the user to convert a postscript file to a gif file.
\begin{quote}\begin{description}
\item[{Example}] \leavevmode
\begin{Verbatim}[commandchars=\\\{\}]
\PYG{g+gp}{\PYGZgt{}\PYGZgt{}\PYGZgt{} }\PYG{n}{a}\PYG{o}{=}\PYG{n}{vcs}\PYG{o}{.}\PYG{n}{init}\PYG{p}{(}\PYG{p}{)}
\PYG{g+gp}{\PYGZgt{}\PYGZgt{}\PYGZgt{} }\PYG{n}{array} \PYG{o}{=} \PYG{p}{[}\PYG{n+nb}{range}\PYG{p}{(}\PYG{l+m+mi}{1}\PYG{p}{,} \PYG{l+m+mi}{11}\PYG{p}{)} \PYG{k}{for} \PYG{n}{\PYGZus{}} \PYG{o+ow}{in} \PYG{n+nb}{range}\PYG{p}{(}\PYG{l+m+mi}{1}\PYG{p}{,} \PYG{l+m+mi}{11}\PYG{p}{)}\PYG{p}{]}
\PYG{g+gp}{\PYGZgt{}\PYGZgt{}\PYGZgt{} }\PYG{n}{a}\PYG{o}{.}\PYG{n}{plot}\PYG{p}{(}\PYG{n}{array}\PYG{p}{)}
\PYG{g+go}{\PYGZlt{}vcs.displayplot.Dp ...\PYGZgt{}}
\PYG{g+gp}{\PYGZgt{}\PYGZgt{}\PYGZgt{} }\PYG{n}{a}\PYG{o}{.}\PYG{n}{pstogif}\PYG{p}{(}\PYG{l+s+s1}{\PYGZsq{}}\PYG{l+s+s1}{filename.ps}\PYG{l+s+s1}{\PYGZsq{}}\PYG{p}{)} \PYG{c+c1}{\PYGZsh{} convert the postscript file to a gif file (l=landscape)}
\PYG{g+gp}{\PYGZgt{}\PYGZgt{}\PYGZgt{} }\PYG{n}{a}\PYG{o}{.}\PYG{n}{pstogif}\PYG{p}{(}\PYG{l+s+s1}{\PYGZsq{}}\PYG{l+s+s1}{filename.ps}\PYG{l+s+s1}{\PYGZsq{}}\PYG{p}{,}\PYG{l+s+s1}{\PYGZsq{}}\PYG{l+s+s1}{l}\PYG{l+s+s1}{\PYGZsq{}}\PYG{p}{)} \PYG{c+c1}{\PYGZsh{} convert the postscript file to a gif file (l=landscape)}
\PYG{g+gp}{\PYGZgt{}\PYGZgt{}\PYGZgt{} }\PYG{n}{a}\PYG{o}{.}\PYG{n}{pstogif}\PYG{p}{(}\PYG{l+s+s1}{\PYGZsq{}}\PYG{l+s+s1}{filename.ps}\PYG{l+s+s1}{\PYGZsq{}}\PYG{p}{,}\PYG{l+s+s1}{\PYGZsq{}}\PYG{l+s+s1}{p}\PYG{l+s+s1}{\PYGZsq{}}\PYG{p}{)} \PYG{c+c1}{\PYGZsh{} convert the postscript file to a gif file (p=portrait)}
\end{Verbatim}

\item[{Parameters}] \leavevmode\begin{itemize}
\item {} 
\textbf{\texttt{filename}} (\href{https://docs.python.org/2/library/functions.html\#str}{\emph{\texttt{str}}}) -- String name of the desired output file

\item {} 
\textbf{\texttt{opt}} (\href{https://docs.python.org/2/library/functions.html\#str}{\emph{\texttt{str}}}) -- One of `l' or `p', indicating landscape or portrait mode, respectively.

\end{itemize}

\item[{Returns}] \leavevmode

???


\item[{Return type}] \leavevmode

???


\end{description}\end{quote}

\end{fulllineitems}

\index{raisecanvas() (vcs.Canvas.Canvas method)}

\begin{fulllineitems}
\phantomsection\label{vcs/Canvas:vcs.Canvas.Canvas.raisecanvas}\pysiglinewithargsret{\sphinxbfcode{raisecanvas}}{\emph{*args}}{}
Raise the VCS Canvas to the top of all open windows.

\end{fulllineitems}

\index{remove\_display\_name() (vcs.Canvas.Canvas method)}

\begin{fulllineitems}
\phantomsection\label{vcs/Canvas:vcs.Canvas.Canvas.remove_display_name}\pysiglinewithargsret{\sphinxbfcode{remove\_display\_name}}{\emph{*args}}{}
Removes a plotted item from the canvas.
\begin{quote}\begin{description}
\item[{Parameters}] \leavevmode
\textbf{\texttt{args}} (\emph{\texttt{str list}}) -- Any number of display names to remove.

\end{description}\end{quote}

\end{fulllineitems}

\index{removeobject() (vcs.Canvas.Canvas method)}

\begin{fulllineitems}
\phantomsection\label{vcs/Canvas:vcs.Canvas.Canvas.removeobject}\pysiglinewithargsret{\sphinxbfcode{removeobject}}{\emph{obj}}{}
The user has the ability to create primary and secondary class
objects. The function allows the user to remove these objects
from the appropriate class list.

Note, To remove the object completely from Python, remember to
use the ``del'' function.

Also note, The user is not allowed to remove a ``default'' class
object.
\begin{quote}\begin{description}
\item[{Example}] \leavevmode
\begin{Verbatim}[commandchars=\\\{\}]
\PYG{g+gp}{\PYGZgt{}\PYGZgt{}\PYGZgt{} }\PYG{n}{a}\PYG{o}{=}\PYG{n}{vcs}\PYG{o}{.}\PYG{n}{init}\PYG{p}{(}\PYG{p}{)}
\PYG{g+gp}{\PYGZgt{}\PYGZgt{}\PYGZgt{} }\PYG{n}{line}\PYG{o}{=}\PYG{n}{a}\PYG{o}{.}\PYG{n}{getline}\PYG{p}{(}\PYG{l+s+s1}{\PYGZsq{}}\PYG{l+s+s1}{red}\PYG{l+s+s1}{\PYGZsq{}}\PYG{p}{)} \PYG{c+c1}{\PYGZsh{} To Modify an existing line object}
\PYG{g+gp}{\PYGZgt{}\PYGZgt{}\PYGZgt{} }\PYG{n}{iso}\PYG{o}{=}\PYG{n}{a}\PYG{o}{.}\PYG{n}{createisoline}\PYG{p}{(}\PYG{l+s+s1}{\PYGZsq{}}\PYG{l+s+s1}{dean}\PYG{l+s+s1}{\PYGZsq{}}\PYG{p}{)} \PYG{c+c1}{\PYGZsh{} Create an instance of an isoline object}
\PYG{g+gp}{\PYGZgt{}\PYGZgt{}\PYGZgt{} }\PYG{n}{a}\PYG{o}{.}\PYG{n}{removeobject}\PYG{p}{(}\PYG{n}{line}\PYG{p}{)} \PYG{c+c1}{\PYGZsh{} Removes line object from VCS list}
\PYG{g+go}{\PYGZsq{}Removed line object red\PYGZsq{}}
\PYG{g+gp}{\PYGZgt{}\PYGZgt{}\PYGZgt{} }\PYG{n}{a}\PYG{o}{.}\PYG{n}{removeobject}\PYG{p}{(}\PYG{n}{iso}\PYG{p}{)} \PYG{c+c1}{\PYGZsh{} Remove isoline object from VCS list}
\PYG{g+go}{\PYGZsq{}Removed isoline object dean\PYGZsq{}}
\end{Verbatim}

\item[{Parameters}] \leavevmode
\textbf{\texttt{obj}} (\emph{\texttt{VCS object}}) -- Any VCS primary or secondary object

\item[{Returns}] \leavevmode
String indicating the specified object was removed

\item[{Return type}] \leavevmode
\href{https://docs.python.org/2/library/functions.html\#str}{str}

\end{description}\end{quote}

\end{fulllineitems}

\index{saveinitialfile() (vcs.Canvas.Canvas method)}

\begin{fulllineitems}
\phantomsection\label{vcs/Canvas:vcs.Canvas.Canvas.saveinitialfile}\pysiglinewithargsret{\sphinxbfcode{saveinitialfile}}{}{}
At start-up, VCS reads a script file named initial.attributes that
defines the initial appearance of the VCS Interface. Although not
required to run VCS, this initial.attributes file contains many
predefined settings to aid the beginning user of VCS.
\begin{quote}\begin{description}
\item[{Example}] \leavevmode
\begin{Verbatim}[commandchars=\\\{\}]
\PYG{g+gp}{\PYGZgt{}\PYGZgt{}\PYGZgt{} }\PYG{n}{a}\PYG{o}{=}\PYG{n}{vcs}\PYG{o}{.}\PYG{n}{init}\PYG{p}{(}\PYG{p}{)}
\PYG{g+gp}{\PYGZgt{}\PYGZgt{}\PYGZgt{} }\PYG{n}{a}\PYG{o}{.}\PYG{n}{saveinitialfile}\PYG{p}{(}\PYG{p}{)}
\end{Verbatim}

\end{description}\end{quote}

\begin{notice}{warning}{Warning:}
This removes first ALL objects generated automatically
(i.e. whose name starts with `\_\_') in order to preserve this, rename objects first
e.g:

\begin{Verbatim}[commandchars=\\\{\}]
\PYG{g+gp}{\PYGZgt{}\PYGZgt{}\PYGZgt{} }\PYG{n}{b}\PYG{o}{=}\PYG{n}{vcs}\PYG{o}{.}\PYG{n}{createboxfill}\PYG{p}{(}\PYG{p}{)}
\PYG{g+gp}{\PYGZgt{}\PYGZgt{}\PYGZgt{} }\PYG{n}{b}\PYG{o}{.}\PYG{n}{rename}\PYG{p}{(}\PYG{l+s+s1}{\PYGZsq{}}\PYG{l+s+s1}{MyBoxfill}\PYG{l+s+s1}{\PYGZsq{}}\PYG{p}{)} \PYG{c+c1}{\PYGZsh{} graphic method is now preserved}
\end{Verbatim}
\end{notice}

\end{fulllineitems}

\index{scatter() (vcs.Canvas.Canvas method)}

\begin{fulllineitems}
\phantomsection\label{vcs/Canvas:vcs.Canvas.Canvas.scatter}\pysiglinewithargsret{\sphinxbfcode{scatter}}{\emph{*args}, \emph{**parms}}{}~\begin{quote}

Generate a scatter plot given the data, scatter graphics method, and
template. If no scatter class object is given, then the `default' scatter
graphics method is used. Similarly, if no template class object is given,
then the `default' template is used.
\begin{quote}\begin{description}
\item[{Example}] \leavevmode
\begin{Verbatim}[commandchars=\\\{\}]
\PYG{g+gp}{\PYGZgt{}\PYGZgt{}\PYGZgt{} }\PYG{n}{a}\PYG{o}{=}\PYG{n}{vcs}\PYG{o}{.}\PYG{n}{init}\PYG{p}{(}\PYG{p}{)}
\PYG{g+gp}{\PYGZgt{}\PYGZgt{}\PYGZgt{} }\PYG{n}{a}\PYG{o}{.}\PYG{n}{show}\PYG{p}{(}\PYG{l+s+s1}{\PYGZsq{}}\PYG{l+s+s1}{scatter}\PYG{l+s+s1}{\PYGZsq{}}\PYG{p}{)} \PYG{c+c1}{\PYGZsh{} Show all the existing scatter graphics methods}
\PYG{g+go}{*******************Scatter Names List**********************}
\PYG{g+gp}{...}
\PYG{g+go}{*******************End Scatter Names List**********************}
\PYG{g+gp}{\PYGZgt{}\PYGZgt{}\PYGZgt{} }\PYG{k+kn}{import} \PYG{n+nn}{cdms2} \PYG{c+c1}{\PYGZsh{} Need cdms2 to create a slab}
\PYG{g+gp}{\PYGZgt{}\PYGZgt{}\PYGZgt{} }\PYG{n}{f} \PYG{o}{=} \PYG{n}{cdms2}\PYG{o}{.}\PYG{n}{open}\PYG{p}{(}\PYG{n}{vcs}\PYG{o}{.}\PYG{n}{sample\PYGZus{}data}\PYG{o}{+}\PYG{l+s+s1}{\PYGZsq{}}\PYG{l+s+s1}{/clt.nc}\PYG{l+s+s1}{\PYGZsq{}}\PYG{p}{)} \PYG{c+c1}{\PYGZsh{} use cdms2 to open a data file}
\PYG{g+gp}{\PYGZgt{}\PYGZgt{}\PYGZgt{} }\PYG{n}{slab1} \PYG{o}{=} \PYG{n}{f}\PYG{p}{(}\PYG{l+s+s1}{\PYGZsq{}}\PYG{l+s+s1}{u}\PYG{l+s+s1}{\PYGZsq{}}\PYG{p}{)} \PYG{c+c1}{\PYGZsh{} use the data file to create a cdms2 slab}
\PYG{g+gp}{\PYGZgt{}\PYGZgt{}\PYGZgt{} }\PYG{n}{slab2} \PYG{o}{=} \PYG{n}{f}\PYG{p}{(}\PYG{l+s+s1}{\PYGZsq{}}\PYG{l+s+s1}{v}\PYG{l+s+s1}{\PYGZsq{}}\PYG{p}{)} \PYG{c+c1}{\PYGZsh{} need 2 slabs, so get another}
\PYG{g+gp}{\PYGZgt{}\PYGZgt{}\PYGZgt{} }\PYG{n}{a}\PYG{o}{.}\PYG{n}{scatter}\PYG{p}{(}\PYG{n}{slab1}\PYG{p}{,} \PYG{n}{slab2}\PYG{p}{)} \PYG{c+c1}{\PYGZsh{} Plot array using specified sct and default template}
\PYG{g+go}{\PYGZlt{}vcs.displayplot.Dp ...\PYGZgt{}}
\PYG{g+gp}{\PYGZgt{}\PYGZgt{}\PYGZgt{} }\PYG{n}{a}\PYG{o}{.}\PYG{n}{clear}\PYG{p}{(}\PYG{p}{)} \PYG{c+c1}{\PYGZsh{} Clear VCS canvas}
\PYG{g+gp}{\PYGZgt{}\PYGZgt{}\PYGZgt{} }\PYG{n}{template}\PYG{o}{=}\PYG{n}{a}\PYG{o}{.}\PYG{n}{gettemplate}\PYG{p}{(}\PYG{l+s+s1}{\PYGZsq{}}\PYG{l+s+s1}{hovmuller}\PYG{l+s+s1}{\PYGZsq{}}\PYG{p}{)}
\PYG{g+gp}{\PYGZgt{}\PYGZgt{}\PYGZgt{} }\PYG{n}{a}\PYG{o}{.}\PYG{n}{scatter}\PYG{p}{(}\PYG{n}{slab1}\PYG{p}{,} \PYG{n}{slab2}\PYG{p}{,} \PYG{n}{template}\PYG{p}{)} \PYG{c+c1}{\PYGZsh{} Plot array using specified sct and template}
\PYG{g+go}{\PYGZlt{}vcs.displayplot.Dp ...\PYGZgt{}}
\end{Verbatim}

\end{description}\end{quote}
\end{quote}
\begin{quote}\begin{description}
\item[{Parameters}] \leavevmode\begin{itemize}
\item {} 
\textbf{\texttt{xaxis}} (\emph{\texttt{cdms2.axis.TransientAxis}}) -- Axis object to replace the slab -1 dim axis

\item {} 
\textbf{\texttt{yaxis}} (\emph{\texttt{cdms2.axis.TransientAxis}}) -- Axis object to replace the slab -2 dim axis, only if slab has more than 1D

\item {} 
\textbf{\texttt{zaxis}} (\emph{\texttt{cdms2.axis.TransientAxis}}) -- Axis object to replace the slab -3 dim axis, only if slab has more than 2D

\item {} 
\textbf{\texttt{taxis}} (\emph{\texttt{cdms2.axis.TransientAxis}}) -- Axis object to replace the slab -4 dim axis, only if slab has more than 3D

\item {} 
\textbf{\texttt{waxis}} (\emph{\texttt{cdms2.axis.TransientAxis}}) -- Axis object to replace the slab -5 dim axis, only if slab has more than 4D

\item {} 
\textbf{\texttt{xrev}} (\href{https://docs.python.org/2/library/functions.html\#bool}{\emph{\texttt{bool}}}) -- reverse x axis

\item {} 
\textbf{\texttt{yrev}} (\href{https://docs.python.org/2/library/functions.html\#bool}{\emph{\texttt{bool}}}) -- reverse y axis, only if slab has more than 1D

\item {} 
\textbf{\texttt{xarray}} (\href{https://docs.python.org/2/library/array.html\#module-array}{\emph{\texttt{array}}}) -- Values to use instead of x axis

\item {} 
\textbf{\texttt{yarray}} (\href{https://docs.python.org/2/library/array.html\#module-array}{\emph{\texttt{array}}}) -- Values to use instead of y axis, only if var has more than 1D

\item {} 
\textbf{\texttt{zarray}} (\href{https://docs.python.org/2/library/array.html\#module-array}{\emph{\texttt{array}}}) -- Values to use instead of z axis, only if var has more than 2D

\item {} 
\textbf{\texttt{tarray}} (\href{https://docs.python.org/2/library/array.html\#module-array}{\emph{\texttt{array}}}) -- Values to use instead of t axis, only if var has more than 3D

\item {} 
\textbf{\texttt{warray}} (\href{https://docs.python.org/2/library/array.html\#module-array}{\emph{\texttt{array}}}) -- Values to use instead of w axis, only if var has more than 4D

\item {} 
\textbf{\texttt{continents}} (\href{https://docs.python.org/2/library/functions.html\#int}{\emph{\texttt{int}}}) -- continents type number

\item {} 
\textbf{\texttt{name}} (\href{https://docs.python.org/2/library/functions.html\#str}{\emph{\texttt{str}}}) -- replaces variable name on plot

\item {} 
\textbf{\texttt{time}} (\emph{\texttt{A cdtime object}}) -- replaces time name on plot

\item {} 
\textbf{\texttt{units}} (\href{https://docs.python.org/2/library/functions.html\#str}{\emph{\texttt{str}}}) -- replaces units value on plot

\item {} 
\textbf{\texttt{ymd}} (\href{https://docs.python.org/2/library/functions.html\#str}{\emph{\texttt{str}}}) -- replaces year/month/day on plot

\item {} 
\textbf{\texttt{hms}} (\href{https://docs.python.org/2/library/functions.html\#str}{\emph{\texttt{str}}}) -- replaces hh/mm/ss on plot

\item {} 
\textbf{\texttt{file\_comment}} (\href{https://docs.python.org/2/library/functions.html\#str}{\emph{\texttt{str}}}) -- replaces file\_comment on plot

\item {} 
\textbf{\texttt{xbounds}} (\href{https://docs.python.org/2/library/array.html\#module-array}{\emph{\texttt{array}}}) -- Values to use instead of x axis bounds values

\item {} 
\textbf{\texttt{ybounds}} (\href{https://docs.python.org/2/library/array.html\#module-array}{\emph{\texttt{array}}}) -- Values to use instead of y axis bounds values (if exist)

\item {} 
\textbf{\texttt{xname}} (\href{https://docs.python.org/2/library/functions.html\#str}{\emph{\texttt{str}}}) -- replace xaxis name on plot

\item {} 
\textbf{\texttt{yname}} (\href{https://docs.python.org/2/library/functions.html\#str}{\emph{\texttt{str}}}) -- replace yaxis name on plot (if exists)

\item {} 
\textbf{\texttt{zname}} (\href{https://docs.python.org/2/library/functions.html\#str}{\emph{\texttt{str}}}) -- replace zaxis name on plot (if exists)

\item {} 
\textbf{\texttt{tname}} (\href{https://docs.python.org/2/library/functions.html\#str}{\emph{\texttt{str}}}) -- replace taxis name on plot (if exists)

\item {} 
\textbf{\texttt{wname}} (\href{https://docs.python.org/2/library/functions.html\#str}{\emph{\texttt{str}}}) -- replace waxis name on plot (if exists)

\item {} 
\textbf{\texttt{xunits}} (\href{https://docs.python.org/2/library/functions.html\#str}{\emph{\texttt{str}}}) -- replace xaxis units on plot

\item {} 
\textbf{\texttt{yunits}} (\href{https://docs.python.org/2/library/functions.html\#str}{\emph{\texttt{str}}}) -- replace yaxis units on plot (if exists)

\item {} 
\textbf{\texttt{zunits}} (\href{https://docs.python.org/2/library/functions.html\#str}{\emph{\texttt{str}}}) -- replace zaxis units on plot (if exists)

\item {} 
\textbf{\texttt{tunits}} (\href{https://docs.python.org/2/library/functions.html\#str}{\emph{\texttt{str}}}) -- replace taxis units on plot (if exists)

\item {} 
\textbf{\texttt{wunits}} (\href{https://docs.python.org/2/library/functions.html\#str}{\emph{\texttt{str}}}) -- replace waxis units on plot (if exists)

\item {} 
\textbf{\texttt{xweights}} (\href{https://docs.python.org/2/library/array.html\#module-array}{\emph{\texttt{array}}}) -- replace xaxis weights used for computing mean

\item {} 
\textbf{\texttt{yweights}} (\href{https://docs.python.org/2/library/array.html\#module-array}{\emph{\texttt{array}}}) -- replace xaxis weights used for computing mean

\item {} 
\textbf{\texttt{comment1}} (\href{https://docs.python.org/2/library/functions.html\#str}{\emph{\texttt{str}}}) -- replaces comment1 on plot

\item {} 
\textbf{\texttt{comment2}} (\href{https://docs.python.org/2/library/functions.html\#str}{\emph{\texttt{str}}}) -- replaces comment2 on plot

\item {} 
\textbf{\texttt{comment3}} (\href{https://docs.python.org/2/library/functions.html\#str}{\emph{\texttt{str}}}) -- replaces comment3 on plot

\item {} 
\textbf{\texttt{comment4}} (\href{https://docs.python.org/2/library/functions.html\#str}{\emph{\texttt{str}}}) -- replaces comment4 on plot

\item {} 
\textbf{\texttt{long\_name}} (\href{https://docs.python.org/2/library/functions.html\#str}{\emph{\texttt{str}}}) -- replaces long\_name on plot

\item {} 
\textbf{\texttt{grid}} (\emph{\texttt{cdms2.grid.TransientRectGrid}}) -- replaces array grid (if exists)

\item {} 
\textbf{\texttt{bg}} (\emph{\texttt{bool/int}}) -- plots in background mode

\item {} 
\textbf{\texttt{ratio}} (\index{xmtics1 (vcs.Canvas.Canvas attribute)}\index{xmtics2 (vcs.Canvas.Canvas attribute)}\index{ymtics1 (vcs.Canvas.Canvas attribute)}\index{ymtics2 (vcs.Canvas.Canvas attribute)}\index{xticlabels1 (vcs.Canvas.Canvas attribute)}\index{xticlabels2 (vcs.Canvas.Canvas attribute)}\index{yticlabels1 (vcs.Canvas.Canvas attribute)}\index{yticlabels2 (vcs.Canvas.Canvas attribute)}\index{projection (vcs.Canvas.Canvas attribute)}\index{datawc\_x1 (vcs.Canvas.Canvas attribute)}\index{datawc\_x2 (vcs.Canvas.Canvas attribute)}\index{datawc\_y1 (vcs.Canvas.Canvas attribute)}\index{datawc\_y2 (vcs.Canvas.Canvas attribute)}\index{datawc\_timeunits (vcs.Canvas.Canvas attribute)}\index{datawc\_calendar (vcs.Canvas.Canvas attribute)}) -- sets the y/x ratio ,if passed as a string with `t' at the end, will aslo moves the ticks

\item {} 
\textbf{\texttt{xaxisconvert}} (\href{https://docs.python.org/2/library/functions.html\#str}{\emph{\texttt{str}}}) -- (Ex: `linear') converting xaxis linear/log/log10/ln/exp/area\_wt

\item {} 
\textbf{\texttt{yaxisconvert}} (\href{https://docs.python.org/2/library/functions.html\#str}{\emph{\texttt{str}}}) -- (Ex: `linear') converting yaxis linear/log/log10/ln/exp/area\_wt

\item {} 
\textbf{\texttt{slab\_or\_primary\_object}} (\href{https://docs.python.org/2/library/array.html\#module-array}{\emph{\texttt{array}}}) -- Data at least 1D, last dimension(s) will be plotted, or secondary vcs object

\end{itemize}

\item[{Returns}] \leavevmode
Display Plot object representing the plot.

\item[{Return type}] \leavevmode

vcs.displayplot.Dp
\begin{quote}\begin{description}
\item[{returns}] \leavevmode
A VCS displayplot object.

\item[{rtype}] \leavevmode
vcs.displayplot.Dp

\end{description}\end{quote}


\end{description}\end{quote}

\end{fulllineitems}

\index{scriptobject() (vcs.Canvas.Canvas method)}

\begin{fulllineitems}
\phantomsection\label{vcs/Canvas:vcs.Canvas.Canvas.scriptobject}\pysiglinewithargsret{\sphinxbfcode{scriptobject}}{\emph{obj}, \emph{script\_filename=None}, \emph{mode=None}}{}
Save individual attributes sets (i.e., individual primary class
objects and/or secondary class objects). These attribute sets
are saved in the user's current directory in one of two formats:
Python script, or a Javascript Object.

\begin{notice}{note}{Note:}
If the the filename has a ''.py'' at the end, it will produce a
Python script. If the filename has a ''.scr'' at the end, it will
produce a VCS script. If neither extensions are given, then by
default a Javascript Object will be produced.
\end{notice}

\begin{notice}{attention}{Attention:}
VCS does not allow the modification of `default' attribute sets,
it will not allow them to be saved as individual script files.
However, a `default' attribute set that has been copied under a
different name can be saved as a script file.
\end{notice}

\begin{notice}{note}{VCS Scripts Deprecated}

SCR scripts are no longer generated by this function
\end{notice}
\begin{quote}\begin{description}
\item[{Example}] \leavevmode
\begin{Verbatim}[commandchars=\\\{\}]
\PYG{g+gp}{\PYGZgt{}\PYGZgt{}\PYGZgt{} }\PYG{n}{a}\PYG{o}{=}\PYG{n}{vcs}\PYG{o}{.}\PYG{n}{init}\PYG{p}{(}\PYG{p}{)}
\PYG{g+gp}{\PYGZgt{}\PYGZgt{}\PYGZgt{} }\PYG{n}{i}\PYG{o}{=}\PYG{n}{a}\PYG{o}{.}\PYG{n}{createisoline}\PYG{p}{(}\PYG{l+s+s1}{\PYGZsq{}}\PYG{l+s+s1}{dean}\PYG{l+s+s1}{\PYGZsq{}}\PYG{p}{)} \PYG{c+c1}{\PYGZsh{} Create an instance of default isoline object}
\PYG{g+gp}{\PYGZgt{}\PYGZgt{}\PYGZgt{} }\PYG{n}{a}\PYG{o}{.}\PYG{n}{scriptobject}\PYG{p}{(}\PYG{n}{i}\PYG{p}{,}\PYG{l+s+s1}{\PYGZsq{}}\PYG{l+s+s1}{ex\PYGZus{}isoline.py}\PYG{l+s+s1}{\PYGZsq{}}\PYG{p}{)} \PYG{c+c1}{\PYGZsh{} Save isoline object as a Python file \PYGZsq{}isoline.py\PYGZsq{}}
\PYG{g+gp}{\PYGZgt{}\PYGZgt{}\PYGZgt{} }\PYG{n}{a}\PYG{o}{.}\PYG{n}{scriptobject}\PYG{p}{(}\PYG{n}{i}\PYG{p}{,}\PYG{l+s+s1}{\PYGZsq{}}\PYG{l+s+s1}{ex\PYGZus{}isoline2}\PYG{l+s+s1}{\PYGZsq{}}\PYG{p}{)} \PYG{c+c1}{\PYGZsh{} Save isoline object as a JSON object \PYGZsq{}isoline2.json\PYGZsq{}}
\end{Verbatim}

\item[{Parameters}] \leavevmode\begin{itemize}
\item {} 
\textbf{\texttt{script\_filename}} (\href{https://docs.python.org/2/library/functions.html\#str}{\emph{\texttt{str}}}) -- Name of the output script file.

\item {} 
\textbf{\texttt{mode}} (\href{https://docs.python.org/2/library/functions.html\#str}{\emph{\texttt{str}}}) -- Mode is either ``w'' for replace or ``a'' for append.

\item {} 
\textbf{\texttt{obj}} (\emph{\texttt{VCS object}}) -- Any VCS primary class or secondary class object.

\end{itemize}

\end{description}\end{quote}

\end{fulllineitems}

\index{setantialiasing() (vcs.Canvas.Canvas method)}

\begin{fulllineitems}
\phantomsection\label{vcs/Canvas:vcs.Canvas.Canvas.setantialiasing}\pysiglinewithargsret{\sphinxbfcode{setantialiasing}}{\emph{antialiasing}}{}
Set antialiasing rate.
\begin{quote}\begin{description}
\item[{Parameters}] \leavevmode
\textbf{\texttt{antialiasing}} (\href{https://docs.python.org/2/library/functions.html\#int}{\emph{\texttt{int}}}) -- Integer from 0-64, representing the antialising rate (0 means no antialiasing).

\end{description}\end{quote}

\end{fulllineitems}

\index{setbgoutputdimensions() (vcs.Canvas.Canvas method)}

\begin{fulllineitems}
\phantomsection\label{vcs/Canvas:vcs.Canvas.Canvas.setbgoutputdimensions}\pysiglinewithargsret{\sphinxbfcode{setbgoutputdimensions}}{\emph{width=None}, \emph{height=None}, \emph{units='inches'}}{}
Sets dimensions for output in bg mode.
\begin{quote}\begin{description}
\item[{Example}] \leavevmode
\begin{Verbatim}[commandchars=\\\{\}]
\PYG{g+gp}{\PYGZgt{}\PYGZgt{}\PYGZgt{} }\PYG{n}{a}\PYG{o}{=}\PYG{n}{vcs}\PYG{o}{.}\PYG{n}{init}\PYG{p}{(}\PYG{p}{)}
\PYG{g+gp}{\PYGZgt{}\PYGZgt{}\PYGZgt{} }\PYG{n}{a}\PYG{o}{.}\PYG{n}{setbgoutputdimensions}\PYG{p}{(}\PYG{n}{width}\PYG{o}{=}\PYG{l+m+mf}{11.5}\PYG{p}{,} \PYG{n}{height}\PYG{o}{=} \PYG{l+m+mf}{8.5}\PYG{p}{)} \PYG{c+c1}{\PYGZsh{} US Legal}
\PYG{g+gp}{\PYGZgt{}\PYGZgt{}\PYGZgt{} }\PYG{n}{a}\PYG{o}{.}\PYG{n}{setbgoutputdimensions}\PYG{p}{(}\PYG{n}{width}\PYG{o}{=}\PYG{l+m+mi}{21}\PYG{p}{,} \PYG{n}{height}\PYG{o}{=}\PYG{l+m+mf}{29.7}\PYG{p}{,} \PYG{n}{units}\PYG{o}{=}\PYG{l+s+s1}{\PYGZsq{}}\PYG{l+s+s1}{cm}\PYG{l+s+s1}{\PYGZsq{}}\PYG{p}{)} \PYG{c+c1}{\PYGZsh{} A4}
\end{Verbatim}

\item[{Parameters}] \leavevmode\begin{itemize}
\item {} 
\textbf{\texttt{width}} (\href{https://docs.python.org/2/library/functions.html\#float}{\emph{\texttt{float}}}) -- Float representing the desired width of the output, using the specified unit of measurement

\item {} 
\textbf{\texttt{height}} (\href{https://docs.python.org/2/library/functions.html\#float}{\emph{\texttt{float}}}) -- Float representing the desired height of the output, using the specified unit of measurement.

\item {} 
\textbf{\texttt{units}} (\href{https://docs.python.org/2/library/functions.html\#str}{\emph{\texttt{str}}}) -- One of {[}'inches', `in', `cm', `mm', `pixel', `pixels', `dot', `dots'{]}. Defaults to `inches'.

\end{itemize}

\end{description}\end{quote}

\end{fulllineitems}

\index{setcolorcell() (vcs.Canvas.Canvas method)}

\begin{fulllineitems}
\phantomsection\label{vcs/Canvas:vcs.Canvas.Canvas.setcolorcell}\pysiglinewithargsret{\sphinxbfcode{setcolorcell}}{\emph{*args}}{}
Set a individual color cell in the active colormap. If default is
the active colormap, then return an error string.

If the the visul display is 16-bit, 24-bit, or 32-bit TrueColor, then a redrawing
of the VCS Canvas is made evertime the color cell is changed.

Note, the user can only change color cells 0 through 239 and R,G,B
value must range from 0 to 100. Where 0 represents no color intensity
and 100 is the greatest color intensity.
\begin{quote}\begin{description}
\item[{Example}] \leavevmode
\begin{Verbatim}[commandchars=\\\{\}]
\PYG{g+gp}{\PYGZgt{}\PYGZgt{}\PYGZgt{} }\PYG{n}{a}\PYG{o}{=}\PYG{n}{vcs}\PYG{o}{.}\PYG{n}{init}\PYG{p}{(}\PYG{p}{)}
\PYG{g+gp}{\PYGZgt{}\PYGZgt{}\PYGZgt{} }\PYG{n}{array} \PYG{o}{=} \PYG{p}{[}\PYG{n+nb}{range}\PYG{p}{(}\PYG{l+m+mi}{1}\PYG{p}{,} \PYG{l+m+mi}{11}\PYG{p}{)} \PYG{k}{for} \PYG{n}{\PYGZus{}} \PYG{o+ow}{in} \PYG{n+nb}{range}\PYG{p}{(}\PYG{l+m+mi}{1}\PYG{p}{,} \PYG{l+m+mi}{11}\PYG{p}{)}\PYG{p}{]}
\PYG{g+gp}{\PYGZgt{}\PYGZgt{}\PYGZgt{} }\PYG{n}{a}\PYG{o}{.}\PYG{n}{plot}\PYG{p}{(}\PYG{n}{array}\PYG{p}{,}\PYG{l+s+s1}{\PYGZsq{}}\PYG{l+s+s1}{default}\PYG{l+s+s1}{\PYGZsq{}}\PYG{p}{,}\PYG{l+s+s1}{\PYGZsq{}}\PYG{l+s+s1}{isofill}\PYG{l+s+s1}{\PYGZsq{}}\PYG{p}{,}\PYG{l+s+s1}{\PYGZsq{}}\PYG{l+s+s1}{quick}\PYG{l+s+s1}{\PYGZsq{}}\PYG{p}{)}
\PYG{g+go}{\PYGZlt{}vcs.displayplot.Dp ...\PYGZgt{}}
\PYG{g+gp}{\PYGZgt{}\PYGZgt{}\PYGZgt{} }\PYG{n}{a}\PYG{o}{.}\PYG{n}{setcolormap}\PYG{p}{(}\PYG{l+s+s2}{\PYGZdq{}}\PYG{l+s+s2}{AMIP}\PYG{l+s+s2}{\PYGZdq{}}\PYG{p}{)}
\PYG{g+gp}{\PYGZgt{}\PYGZgt{}\PYGZgt{} }\PYG{n}{a}\PYG{o}{.}\PYG{n}{setcolorcell}\PYG{p}{(}\PYG{l+m+mi}{11}\PYG{p}{,}\PYG{l+m+mi}{0}\PYG{p}{,}\PYG{l+m+mi}{0}\PYG{p}{,}\PYG{l+m+mi}{0}\PYG{p}{)}
\PYG{g+gp}{\PYGZgt{}\PYGZgt{}\PYGZgt{} }\PYG{n}{a}\PYG{o}{.}\PYG{n}{setcolorcell}\PYG{p}{(}\PYG{l+m+mi}{21}\PYG{p}{,}\PYG{l+m+mi}{100}\PYG{p}{,}\PYG{l+m+mi}{0}\PYG{p}{,}\PYG{l+m+mi}{0}\PYG{p}{)}
\PYG{g+gp}{\PYGZgt{}\PYGZgt{}\PYGZgt{} }\PYG{n}{a}\PYG{o}{.}\PYG{n}{setcolorcell}\PYG{p}{(}\PYG{l+m+mi}{31}\PYG{p}{,}\PYG{l+m+mi}{0}\PYG{p}{,}\PYG{l+m+mi}{100}\PYG{p}{,}\PYG{l+m+mi}{0}\PYG{p}{)}
\PYG{g+gp}{\PYGZgt{}\PYGZgt{}\PYGZgt{} }\PYG{n}{a}\PYG{o}{.}\PYG{n}{setcolorcell}\PYG{p}{(}\PYG{l+m+mi}{41}\PYG{p}{,}\PYG{l+m+mi}{0}\PYG{p}{,}\PYG{l+m+mi}{0}\PYG{p}{,}\PYG{l+m+mi}{100}\PYG{p}{)}
\PYG{g+gp}{\PYGZgt{}\PYGZgt{}\PYGZgt{} }\PYG{n}{a}\PYG{o}{.}\PYG{n}{setcolorcell}\PYG{p}{(}\PYG{l+m+mi}{51}\PYG{p}{,}\PYG{l+m+mi}{100}\PYG{p}{,}\PYG{l+m+mi}{100}\PYG{p}{,}\PYG{l+m+mi}{100}\PYG{p}{)}
\PYG{g+gp}{\PYGZgt{}\PYGZgt{}\PYGZgt{} }\PYG{n}{a}\PYG{o}{.}\PYG{n}{setcolorcell}\PYG{p}{(}\PYG{l+m+mi}{61}\PYG{p}{,}\PYG{l+m+mi}{70}\PYG{p}{,}\PYG{l+m+mi}{70}\PYG{p}{,}\PYG{l+m+mi}{70}\PYG{p}{)}
\PYG{g+gp}{\PYGZgt{}\PYGZgt{}\PYGZgt{} }\PYG{n}{a}\PYG{o}{.}\PYG{n}{plot}\PYG{p}{(}\PYG{n}{array}\PYG{p}{,}\PYG{l+s+s1}{\PYGZsq{}}\PYG{l+s+s1}{default}\PYG{l+s+s1}{\PYGZsq{}}\PYG{p}{,}\PYG{l+s+s1}{\PYGZsq{}}\PYG{l+s+s1}{isofill}\PYG{l+s+s1}{\PYGZsq{}}\PYG{p}{,}\PYG{l+s+s1}{\PYGZsq{}}\PYG{l+s+s1}{quick}\PYG{l+s+s1}{\PYGZsq{}}\PYG{p}{)}
\PYG{g+go}{\PYGZlt{}vcs.displayplot.Dp ...\PYGZgt{}}
\end{Verbatim}

\end{description}\end{quote}

\end{fulllineitems}

\index{setcolormap() (vcs.Canvas.Canvas method)}

\begin{fulllineitems}
\phantomsection\label{vcs/Canvas:vcs.Canvas.Canvas.setcolormap}\pysiglinewithargsret{\sphinxbfcode{setcolormap}}{\emph{name}}{}
It is necessary to change the colormap. This routine will change the VCS
color map.

If the the visual display is 16-bit, 24-bit, or 32-bit TrueColor, then a redrawing
of the VCS Canvas is made every time the colormap is changed.
\begin{quote}\begin{description}
\item[{Example}] \leavevmode
\begin{Verbatim}[commandchars=\\\{\}]
\PYG{g+gp}{\PYGZgt{}\PYGZgt{}\PYGZgt{} }\PYG{n}{a}\PYG{o}{=}\PYG{n}{vcs}\PYG{o}{.}\PYG{n}{init}\PYG{p}{(}\PYG{p}{)}
\PYG{g+gp}{\PYGZgt{}\PYGZgt{}\PYGZgt{} }\PYG{n}{array} \PYG{o}{=} \PYG{p}{[}\PYG{n+nb}{range}\PYG{p}{(}\PYG{l+m+mi}{1}\PYG{p}{,} \PYG{l+m+mi}{11}\PYG{p}{)} \PYG{k}{for} \PYG{n}{\PYGZus{}} \PYG{o+ow}{in} \PYG{n+nb}{range}\PYG{p}{(}\PYG{l+m+mi}{1}\PYG{p}{,} \PYG{l+m+mi}{11}\PYG{p}{)}\PYG{p}{]}
\PYG{g+gp}{\PYGZgt{}\PYGZgt{}\PYGZgt{} }\PYG{n}{a}\PYG{o}{.}\PYG{n}{plot}\PYG{p}{(}\PYG{n}{array}\PYG{p}{,}\PYG{l+s+s1}{\PYGZsq{}}\PYG{l+s+s1}{default}\PYG{l+s+s1}{\PYGZsq{}}\PYG{p}{,}\PYG{l+s+s1}{\PYGZsq{}}\PYG{l+s+s1}{isofill}\PYG{l+s+s1}{\PYGZsq{}}\PYG{p}{,}\PYG{l+s+s1}{\PYGZsq{}}\PYG{l+s+s1}{quick}\PYG{l+s+s1}{\PYGZsq{}}\PYG{p}{)}
\PYG{g+go}{\PYGZlt{}vcs.displayplot.Dp ...\PYGZgt{}}
\PYG{g+gp}{\PYGZgt{}\PYGZgt{}\PYGZgt{} }\PYG{n}{a}\PYG{o}{.}\PYG{n}{setcolormap}\PYG{p}{(}\PYG{l+s+s2}{\PYGZdq{}}\PYG{l+s+s2}{AMIP}\PYG{l+s+s2}{\PYGZdq{}}\PYG{p}{)}
\PYG{g+gp}{\PYGZgt{}\PYGZgt{}\PYGZgt{} }\PYG{n}{a}\PYG{o}{.}\PYG{n}{plot}\PYG{p}{(}\PYG{n}{array}\PYG{p}{,}\PYG{l+s+s1}{\PYGZsq{}}\PYG{l+s+s1}{default}\PYG{l+s+s1}{\PYGZsq{}}\PYG{p}{,}\PYG{l+s+s1}{\PYGZsq{}}\PYG{l+s+s1}{isofill}\PYG{l+s+s1}{\PYGZsq{}}\PYG{p}{,}\PYG{l+s+s1}{\PYGZsq{}}\PYG{l+s+s1}{quick}\PYG{l+s+s1}{\PYGZsq{}}\PYG{p}{)}
\PYG{g+go}{\PYGZlt{}vcs.displayplot.Dp ...\PYGZgt{}}
\end{Verbatim}

\item[{Parameters}] \leavevmode
\textbf{\texttt{name}} (\href{https://docs.python.org/2/library/functions.html\#str}{\emph{\texttt{str}}}) -- Name of the colormap to use

\end{description}\end{quote}

\end{fulllineitems}

\index{setcontinentsline() (vcs.Canvas.Canvas method)}

\begin{fulllineitems}
\phantomsection\label{vcs/Canvas:vcs.Canvas.Canvas.setcontinentsline}\pysiglinewithargsret{\sphinxbfcode{setcontinentsline}}{\emph{line='default'}}{}
One has the option of configuring the appearance of the lines used to
draw continents by providing a VCS Line object.
\begin{quote}\begin{description}
\item[{Example}] \leavevmode
\begin{Verbatim}[commandchars=\\\{\}]
\PYG{g+gp}{\PYGZgt{}\PYGZgt{}\PYGZgt{} }\PYG{n}{a} \PYG{o}{=} \PYG{n}{vcs}\PYG{o}{.}\PYG{n}{init}\PYG{p}{(}\PYG{p}{)}
\PYG{g+gp}{\PYGZgt{}\PYGZgt{}\PYGZgt{} }\PYG{n}{line} \PYG{o}{=} \PYG{n}{vcs}\PYG{o}{.}\PYG{n}{createline}\PYG{p}{(}\PYG{p}{)}
\PYG{g+gp}{\PYGZgt{}\PYGZgt{}\PYGZgt{} }\PYG{n}{line}\PYG{o}{.}\PYG{n}{width} \PYG{o}{=} \PYG{l+m+mi}{5}
\PYG{g+gp}{\PYGZgt{}\PYGZgt{}\PYGZgt{} }\PYG{n}{a}\PYG{o}{.}\PYG{n}{setcontinentsline}\PYG{p}{(}\PYG{n}{line}\PYG{p}{)} \PYG{c+c1}{\PYGZsh{} Use custom continents line}
\PYG{g+gp}{\PYGZgt{}\PYGZgt{}\PYGZgt{} }\PYG{n}{a}\PYG{o}{.}\PYG{n}{setcontinentsline}\PYG{p}{(}\PYG{l+s+s2}{\PYGZdq{}}\PYG{l+s+s2}{default}\PYG{l+s+s2}{\PYGZdq{}}\PYG{p}{)} \PYG{c+c1}{\PYGZsh{} Use default line}
\end{Verbatim}

\item[{Parameters}] \leavevmode
\textbf{\texttt{line}} (str or {\hyperref[vcs/secondary/line:vcs.line.Tl]{\sphinxcrossref{\sphinxcode{vcs.line.Tl}}}}) -- Line to use for drawing continents. Can be a string name of a line, or a VCS line object

\end{description}\end{quote}

\end{fulllineitems}

\index{setcontinentstype() (vcs.Canvas.Canvas method)}

\begin{fulllineitems}
\phantomsection\label{vcs/Canvas:vcs.Canvas.Canvas.setcontinentstype}\pysiglinewithargsret{\sphinxbfcode{setcontinentstype}}{\emph{value}}{}
One has the option of using continental maps that are predefined or that
are user-defined. Predefined continental maps are either internal to VCS
or are specified by external files. User-defined continental maps are
specified by additional external files that must be read as input.
\begin{description}
\item[{The continents-type values are integers ranging from 0 to 11, where:}] \leavevmode
0 signifies ``No Continents''
1 signifies ``Fine Continents''
2 signifies ``Coarse Continents''
3 signifies ``United States'' (with ``Fine Continents'')
4 signifies ``Political Borders'' (with ``Fine Continents'')
5 signifies ``Rivers'' (with ``Fine Continents'')

6 uses a custom continent set

\end{description}

You can also pass a file by path.
\begin{quote}\begin{description}
\item[{Example}] \leavevmode
\begin{Verbatim}[commandchars=\\\{\}]
\PYG{g+gp}{\PYGZgt{}\PYGZgt{}\PYGZgt{} }\PYG{n}{a}\PYG{o}{=}\PYG{n}{vcs}\PYG{o}{.}\PYG{n}{init}\PYG{p}{(}\PYG{p}{)}
\PYG{g+gp}{\PYGZgt{}\PYGZgt{}\PYGZgt{} }\PYG{n}{a}\PYG{o}{.}\PYG{n}{setcontinentstype}\PYG{p}{(}\PYG{l+m+mi}{3}\PYG{p}{)}
\PYG{g+gp}{\PYGZgt{}\PYGZgt{}\PYGZgt{} }\PYG{n}{array} \PYG{o}{=} \PYG{p}{[}\PYG{n+nb}{range}\PYG{p}{(}\PYG{l+m+mi}{1}\PYG{p}{,} \PYG{l+m+mi}{11}\PYG{p}{)} \PYG{k}{for} \PYG{n}{\PYGZus{}} \PYG{o+ow}{in} \PYG{n+nb}{range}\PYG{p}{(}\PYG{l+m+mi}{1}\PYG{p}{,} \PYG{l+m+mi}{11}\PYG{p}{)}\PYG{p}{]}
\PYG{g+gp}{\PYGZgt{}\PYGZgt{}\PYGZgt{} }\PYG{n}{a}\PYG{o}{.}\PYG{n}{plot}\PYG{p}{(}\PYG{n}{array}\PYG{p}{,}\PYG{l+s+s1}{\PYGZsq{}}\PYG{l+s+s1}{default}\PYG{l+s+s1}{\PYGZsq{}}\PYG{p}{,}\PYG{l+s+s1}{\PYGZsq{}}\PYG{l+s+s1}{isofill}\PYG{l+s+s1}{\PYGZsq{}}\PYG{p}{,}\PYG{l+s+s1}{\PYGZsq{}}\PYG{l+s+s1}{quick}\PYG{l+s+s1}{\PYGZsq{}}\PYG{p}{)}
\PYG{g+go}{\PYGZlt{}vcs.displayplot.Dp ...\PYGZgt{}}
\end{Verbatim}

\item[{Parameters}] \leavevmode
\textbf{\texttt{value}} (\href{https://docs.python.org/2/library/functions.html\#int}{\emph{\texttt{int}}}) -- Integer representing continent type, as specified in function description

\end{description}\end{quote}

\end{fulllineitems}

\index{setdefaultfont() (vcs.Canvas.Canvas method)}

\begin{fulllineitems}
\phantomsection\label{vcs/Canvas:vcs.Canvas.Canvas.setdefaultfont}\pysiglinewithargsret{\sphinxbfcode{setdefaultfont}}{\emph{font}}{}
Sets the passed/def show font as the default font for vcs
\begin{quote}\begin{description}
\item[{Parameters}] \leavevmode
\textbf{\texttt{font}} (\emph{\texttt{str or int}}) -- Font name or index to use as default

\end{description}\end{quote}

\end{fulllineitems}

\index{show() (vcs.Canvas.Canvas method)}

\begin{fulllineitems}
\phantomsection\label{vcs/Canvas:vcs.Canvas.Canvas.show}\pysiglinewithargsret{\sphinxbfcode{show}}{\emph{*args}}{}
Creator: \href{http://computation.llnl.gov/about/our-people/highlights/dean-williams}{Dean Williams} (LLNL, AIMS Team)

Lead Developer: \href{https://github.com/doutriaux1}{Charles Doutriaux} (LLNL, AIMS Team)

Contributors: \url{https://github.com/UV-CDAT/uvcdat/graphs/contributors}

Support Email: \href{mailto:uvcdat-support@llnl.gov}{uvcdat-support@llnl.gov}

Project Site: \url{http://uvcdat.llnl.gov/}

Project Repo: \url{https://github.com/UV-CDAT/uvcdat/graphs/contributors}

VCS is a visualization library for scientific data. It has a simple
model for defining a plot, that is decomposed into three parts:
\begin{enumerate}
\item {} 
\textbf{Data}: If it's iterable, we'll plot it... or at least try!
Currently we support numpy arrays, lists (nested and not),
and CDMS2 variables (there's some special support for metadata
from CDMS2 that gives some niceties in your plot, but it's not
mandatory).

\item {} 
\textbf{Graphics Method}: We have a variety of plot types that we
support out-of-the box; you can easily customize every aspect
of them to create the effect that you're looking for. If you can't,
we also support defining your own graphics methods, which you can
share with other users using standard python infrastructure (conda, pip).

\item {} 
\textbf{Template}: Templates control the appearance of everything that
\emph{isn't} your data. They position labels, control fonts, adjust borders,
place legends, and more. They're very flexible, and give the fine-grained
control of your plot that is needed for the truly perfect plot. Once you've
customized them, you can also save them out for later use, and distribute
them to other users.

\end{enumerate}

\end{fulllineitems}

\index{svg() (vcs.Canvas.Canvas method)}

\begin{fulllineitems}
\phantomsection\label{vcs/Canvas:vcs.Canvas.Canvas.svg}\pysiglinewithargsret{\sphinxbfcode{svg}}{\emph{file}, \emph{width=None}, \emph{height=None}, \emph{units='inches'}, \emph{textAsPaths=True}}{}
SVG output is another form of vector graphics.

\begin{notice}{note}{Note:}
The textAsPaths parameter preserves custom fonts, but text can no longer be edited in the file
\end{notice}
\begin{quote}\begin{description}
\item[{Example}] \leavevmode
\begin{Verbatim}[commandchars=\\\{\}]
\PYG{g+gp}{\PYGZgt{}\PYGZgt{}\PYGZgt{} }\PYG{n}{a}\PYG{o}{=}\PYG{n}{vcs}\PYG{o}{.}\PYG{n}{init}\PYG{p}{(}\PYG{p}{)}
\PYG{g+gp}{\PYGZgt{}\PYGZgt{}\PYGZgt{} }\PYG{n}{array} \PYG{o}{=} \PYG{p}{[}\PYG{n+nb}{range}\PYG{p}{(}\PYG{l+m+mi}{1}\PYG{p}{,} \PYG{l+m+mi}{11}\PYG{p}{)} \PYG{k}{for} \PYG{n}{\PYGZus{}} \PYG{o+ow}{in} \PYG{n+nb}{range}\PYG{p}{(}\PYG{l+m+mi}{1}\PYG{p}{,} \PYG{l+m+mi}{11}\PYG{p}{)}\PYG{p}{]}
\PYG{g+gp}{\PYGZgt{}\PYGZgt{}\PYGZgt{} }\PYG{n}{a}\PYG{o}{.}\PYG{n}{plot}\PYG{p}{(}\PYG{n}{array}\PYG{p}{)}
\PYG{g+go}{\PYGZlt{}vcs.displayplot.Dp ...\PYGZgt{}}
\PYG{g+gp}{\PYGZgt{}\PYGZgt{}\PYGZgt{} }\PYG{n}{a}\PYG{o}{.}\PYG{n}{svg}\PYG{p}{(}\PYG{l+s+s1}{\PYGZsq{}}\PYG{l+s+s1}{example}\PYG{l+s+s1}{\PYGZsq{}}\PYG{p}{)} \PYG{c+c1}{\PYGZsh{} Overwrite a postscript file}
\PYG{g+gp}{\PYGZgt{}\PYGZgt{}\PYGZgt{} }\PYG{n}{a}\PYG{o}{.}\PYG{n}{svg}\PYG{p}{(}\PYG{l+s+s1}{\PYGZsq{}}\PYG{l+s+s1}{example}\PYG{l+s+s1}{\PYGZsq{}}\PYG{p}{,} \PYG{n}{width}\PYG{o}{=}\PYG{l+m+mf}{11.5}\PYG{p}{,} \PYG{n}{height}\PYG{o}{=} \PYG{l+m+mf}{8.5}\PYG{p}{)} \PYG{c+c1}{\PYGZsh{} US Legal}
\PYG{g+gp}{\PYGZgt{}\PYGZgt{}\PYGZgt{} }\PYG{n}{a}\PYG{o}{.}\PYG{n}{svg}\PYG{p}{(}\PYG{l+s+s1}{\PYGZsq{}}\PYG{l+s+s1}{example}\PYG{l+s+s1}{\PYGZsq{}}\PYG{p}{,} \PYG{n}{width}\PYG{o}{=}\PYG{l+m+mi}{21}\PYG{p}{,} \PYG{n}{height}\PYG{o}{=}\PYG{l+m+mf}{29.7}\PYG{p}{,} \PYG{n}{units}\PYG{o}{=}\PYG{l+s+s1}{\PYGZsq{}}\PYG{l+s+s1}{cm}\PYG{l+s+s1}{\PYGZsq{}}\PYG{p}{)} \PYG{c+c1}{\PYGZsh{} A4}
\end{Verbatim}

\item[{Parameters}] \leavevmode\begin{itemize}
\item {} 
\textbf{\texttt{file}} -- 

\item {} 
\textbf{\texttt{width}} (\href{https://docs.python.org/2/library/functions.html\#float}{\emph{\texttt{float}}}) -- Float to set width of output SVG, in specified unit of measurement

\item {} 
\textbf{\texttt{height}} (\href{https://docs.python.org/2/library/functions.html\#float}{\emph{\texttt{float}}}) -- Float to set height of output SVG, in specified unit of measurement

\item {} 
\textbf{\texttt{units}} (\href{https://docs.python.org/2/library/functions.html\#str}{\emph{\texttt{str}}}) -- One of {[}'inches', `in', `cm', `mm', `pixel', `pixels', `dot', `dots'{]}. Deafults to `inches'.

\item {} 
\textbf{\texttt{textAsPaths}} (\href{https://docs.python.org/2/library/functions.html\#bool}{\emph{\texttt{bool}}}) -- Specifies whether to render text objects as paths.

\end{itemize}

\end{description}\end{quote}

\end{fulllineitems}

\index{switchfonts() (vcs.Canvas.Canvas method)}

\begin{fulllineitems}
\phantomsection\label{vcs/Canvas:vcs.Canvas.Canvas.switchfonts}\pysiglinewithargsret{\sphinxbfcode{switchfonts}}{\emph{font1}, \emph{font2}}{}
Switch the font numbers of two fonts.
\begin{quote}\begin{description}
\item[{Parameters}] \leavevmode\begin{itemize}
\item {} 
\textbf{\texttt{font1}} (\emph{\texttt{int or str}}) -- The first font

\item {} 
\textbf{\texttt{font2}} (\emph{\texttt{int or str}}) -- The second font

\end{itemize}

\end{description}\end{quote}

\end{fulllineitems}

\index{taylordiagram() (vcs.Canvas.Canvas method)}

\begin{fulllineitems}
\phantomsection\label{vcs/Canvas:vcs.Canvas.Canvas.taylordiagram}\pysiglinewithargsret{\sphinxbfcode{taylordiagram}}{\emph{*args}, \emph{**parms}}{}
Generate a taylordiagram plot given the data, taylordiagram graphics method, and
template. If no taylordiagram class object is given, then the `default' taylordiagram
graphics method is used. Similarly, if no template class object is given,
then the `default' template is used.
\begin{quote}\begin{description}
\item[{Example}] \leavevmode
\begin{Verbatim}[commandchars=\\\{\}]
\PYG{g+gp}{\PYGZgt{}\PYGZgt{}\PYGZgt{} }\PYG{n}{a}\PYG{o}{=}\PYG{n}{vcs}\PYG{o}{.}\PYG{n}{init}\PYG{p}{(}\PYG{p}{)}
\PYG{g+gp}{\PYGZgt{}\PYGZgt{}\PYGZgt{} }\PYG{n}{a}\PYG{o}{.}\PYG{n}{show}\PYG{p}{(}\PYG{l+s+s1}{\PYGZsq{}}\PYG{l+s+s1}{taylordiagram}\PYG{l+s+s1}{\PYGZsq{}}\PYG{p}{)} \PYG{c+c1}{\PYGZsh{} Show all the existing taylordiagram graphics methods}
\PYG{g+go}{*******************Taylordiagram Names List**********************}
\PYG{g+gp}{...}
\PYG{g+go}{*******************End Taylordiagram Names List**********************}
\PYG{g+gp}{\PYGZgt{}\PYGZgt{}\PYGZgt{} }\PYG{n}{td}\PYG{o}{=} \PYG{n}{a}\PYG{o}{.}\PYG{n}{gettaylordiagram}\PYG{p}{(}\PYG{p}{)} \PYG{c+c1}{\PYGZsh{} Create instance of \PYGZsq{}default\PYGZsq{}}
\PYG{g+gp}{\PYGZgt{}\PYGZgt{}\PYGZgt{} }\PYG{n}{array}\PYG{o}{=}\PYG{p}{[}\PYG{n+nb}{range}\PYG{p}{(}\PYG{l+m+mi}{1}\PYG{p}{,} \PYG{l+m+mi}{11}\PYG{p}{)} \PYG{k}{for} \PYG{n}{\PYGZus{}} \PYG{o+ow}{in} \PYG{n+nb}{range}\PYG{p}{(}\PYG{l+m+mi}{1}\PYG{p}{,} \PYG{l+m+mi}{11}\PYG{p}{)}\PYG{p}{]}
\PYG{g+gp}{\PYGZgt{}\PYGZgt{}\PYGZgt{} }\PYG{n}{a}\PYG{o}{.}\PYG{n}{taylordiagram}\PYG{p}{(}\PYG{n}{array}\PYG{p}{,}\PYG{n}{td}\PYG{p}{)} \PYG{c+c1}{\PYGZsh{} Plot array using specified iso and default template}
\PYG{g+go}{\PYGZlt{}vcs.displayplot.Dp ...\PYGZgt{}}
\PYG{g+gp}{\PYGZgt{}\PYGZgt{}\PYGZgt{} }\PYG{n}{a}\PYG{o}{.}\PYG{n}{clear}\PYG{p}{(}\PYG{p}{)} \PYG{c+c1}{\PYGZsh{} Clear VCS canvas}
\PYG{g+gp}{\PYGZgt{}\PYGZgt{}\PYGZgt{} }\PYG{n}{template}\PYG{o}{=}\PYG{n}{a}\PYG{o}{.}\PYG{n}{gettemplate}\PYG{p}{(}\PYG{l+s+s1}{\PYGZsq{}}\PYG{l+s+s1}{hovmuller}\PYG{l+s+s1}{\PYGZsq{}}\PYG{p}{)}
\PYG{g+gp}{\PYGZgt{}\PYGZgt{}\PYGZgt{} }\PYG{n}{a}\PYG{o}{.}\PYG{n}{taylordiagram}\PYG{p}{(}\PYG{n}{array}\PYG{p}{,}\PYG{n}{td}\PYG{p}{,}\PYG{n}{template}\PYG{p}{)} \PYG{c+c1}{\PYGZsh{} Plot array using specified iso and template}
\PYG{g+go}{\PYGZlt{}vcs.displayplot.Dp ...\PYGZgt{}}
\end{Verbatim}

\item[{Returns}] \leavevmode
A VCS displayplot object.

\item[{Return type}] \leavevmode
{\hyperref[vcs/misc/displayplot:vcs.displayplot.Dp]{\sphinxcrossref{vcs.displayplot.Dp}}}

\end{description}\end{quote}

\end{fulllineitems}

\index{text() (vcs.Canvas.Canvas method)}

\begin{fulllineitems}
\phantomsection\label{vcs/Canvas:vcs.Canvas.Canvas.text}\pysiglinewithargsret{\sphinxbfcode{text}}{\emph{*args}, \emph{**parms}}{}
Plot a textcombined segment on the Vcs Canvas. If no textcombined class
object is given, then an error will be returned.

\begin{notice}{note}{Note:}
The text() function is an alias for textcombined().
See example for usage.
\end{notice}
\begin{quote}\begin{description}
\item[{Example}] \leavevmode
\end{description}\end{quote}

\begin{Verbatim}[commandchars=\\\{\}]
\PYG{g+gp}{\PYGZgt{}\PYGZgt{}\PYGZgt{} }\PYG{n}{a}\PYG{o}{=}\PYG{n}{vcs}\PYG{o}{.}\PYG{n}{init}\PYG{p}{(}\PYG{p}{)}
\PYG{g+gp}{\PYGZgt{}\PYGZgt{}\PYGZgt{} }\PYG{n}{a}\PYG{o}{.}\PYG{n}{clean\PYGZus{}auto\PYGZus{}generated\PYGZus{}objects}\PYG{p}{(}\PYG{p}{)}
\PYG{g+gp}{\PYGZgt{}\PYGZgt{}\PYGZgt{} }\PYG{n}{a}\PYG{o}{.}\PYG{n}{show}\PYG{p}{(}\PYG{l+s+s1}{\PYGZsq{}}\PYG{l+s+s1}{texttable}\PYG{l+s+s1}{\PYGZsq{}}\PYG{p}{)} \PYG{c+c1}{\PYGZsh{} Show all the existing texttable objects}
\PYG{g+go}{*******************Texttable Names List**********************}
\PYG{g+gp}{...}
\PYG{g+go}{*******************End Texttable Names List**********************}
\PYG{g+gp}{\PYGZgt{}\PYGZgt{}\PYGZgt{} }\PYG{n}{a}\PYG{o}{.}\PYG{n}{show}\PYG{p}{(}\PYG{l+s+s1}{\PYGZsq{}}\PYG{l+s+s1}{textorientation}\PYG{l+s+s1}{\PYGZsq{}}\PYG{p}{)} \PYG{c+c1}{\PYGZsh{} Show all the existing textorientation objects}
\PYG{g+go}{*******************Textorientation Names List**********************}
\PYG{g+gp}{...}
\PYG{g+go}{*******************End Textorientation Names List**********************}
\PYG{g+gp}{\PYGZgt{}\PYGZgt{}\PYGZgt{} }\PYG{n}{vcs}\PYG{o}{.}\PYG{n}{createtext}\PYG{p}{(}\PYG{l+s+s1}{\PYGZsq{}}\PYG{l+s+s1}{qa\PYGZus{}tta}\PYG{l+s+s1}{\PYGZsq{}}\PYG{p}{,} \PYG{l+s+s1}{\PYGZsq{}}\PYG{l+s+s1}{qa}\PYG{l+s+s1}{\PYGZsq{}}\PYG{p}{,} \PYG{l+s+s1}{\PYGZsq{}}\PYG{l+s+s1}{7left\PYGZus{}tto}\PYG{l+s+s1}{\PYGZsq{}}\PYG{p}{,} \PYG{l+s+s1}{\PYGZsq{}}\PYG{l+s+s1}{7left}\PYG{l+s+s1}{\PYGZsq{}}\PYG{p}{)} \PYG{c+c1}{\PYGZsh{} Create instance of \PYGZsq{}std\PYGZus{}tt\PYGZsq{} and \PYGZsq{}7left\PYGZus{}to\PYGZsq{}}
\PYG{g+go}{\PYGZlt{}vcs.textcombined.Tc object at ...\PYGZgt{}}
\PYG{g+gp}{\PYGZgt{}\PYGZgt{}\PYGZgt{} }\PYG{n}{tc}\PYG{o}{=}\PYG{n}{a}\PYG{o}{.}\PYG{n}{gettext}\PYG{p}{(}\PYG{l+s+s1}{\PYGZsq{}}\PYG{l+s+s1}{qa\PYGZus{}tta}\PYG{l+s+s1}{\PYGZsq{}}\PYG{p}{,} \PYG{l+s+s1}{\PYGZsq{}}\PYG{l+s+s1}{7left\PYGZus{}tto}\PYG{l+s+s1}{\PYGZsq{}}\PYG{p}{)}
\PYG{g+gp}{\PYGZgt{}\PYGZgt{}\PYGZgt{} }\PYG{n}{tc}\PYG{o}{.}\PYG{n}{string}\PYG{o}{=}\PYG{l+s+s1}{\PYGZsq{}}\PYG{l+s+s1}{Text1}\PYG{l+s+s1}{\PYGZsq{}} \PYG{c+c1}{\PYGZsh{} Show the string \PYGZdq{}Text1\PYGZdq{} on the VCS Canvas}
\PYG{g+gp}{\PYGZgt{}\PYGZgt{}\PYGZgt{} }\PYG{n}{tc}\PYG{o}{.}\PYG{n}{font}\PYG{o}{=}\PYG{l+m+mi}{2} \PYG{c+c1}{\PYGZsh{} Set the text size}
\PYG{g+gp}{\PYGZgt{}\PYGZgt{}\PYGZgt{} }\PYG{n}{tc}\PYG{o}{.}\PYG{n}{color}\PYG{o}{=}\PYG{l+m+mi}{242} \PYG{c+c1}{\PYGZsh{} Set the text color}
\PYG{g+gp}{\PYGZgt{}\PYGZgt{}\PYGZgt{} }\PYG{n}{tc}\PYG{o}{.}\PYG{n}{angle}\PYG{o}{=}\PYG{l+m+mi}{45} \PYG{c+c1}{\PYGZsh{} Set the text angle}
\PYG{g+gp}{\PYGZgt{}\PYGZgt{}\PYGZgt{} }\PYG{n}{tc}\PYG{o}{.}\PYG{n}{x}\PYG{o}{=}\PYG{p}{[}\PYG{l+m+mf}{0.5}\PYG{p}{]}
\PYG{g+gp}{\PYGZgt{}\PYGZgt{}\PYGZgt{} }\PYG{n}{tc}\PYG{o}{.}\PYG{n}{y}\PYG{o}{=}\PYG{p}{[}\PYG{l+m+mf}{0.5}\PYG{p}{]}
\PYG{g+gp}{\PYGZgt{}\PYGZgt{}\PYGZgt{} }\PYG{n}{a}\PYG{o}{.}\PYG{n}{textcombined}\PYG{p}{(}\PYG{n}{tc}\PYG{p}{)} \PYG{c+c1}{\PYGZsh{} Plot using specified text object}
\PYG{g+go}{\PYGZlt{}vcs.displayplot.Dp ...\PYGZgt{}}
\end{Verbatim}
\begin{quote}\begin{description}
\item[{Returns}] \leavevmode
A fillarea object

\item[{Return type}] \leavevmode
{\hyperref[vcs/misc/displayplot:vcs.displayplot.Dp]{\sphinxcrossref{vcs.displayplot.Dp}}}

\end{description}\end{quote}

\end{fulllineitems}

\index{textcombined() (vcs.Canvas.Canvas method)}

\begin{fulllineitems}
\phantomsection\label{vcs/Canvas:vcs.Canvas.Canvas.textcombined}\pysiglinewithargsret{\sphinxbfcode{textcombined}}{\emph{*args}, \emph{**parms}}{}
Plot a textcombined segment on the Vcs Canvas. If no textcombined class
object is given, then an error will be returned.

\begin{notice}{note}{Note:}
The text() function is an alias for textcombined().
See example for usage.
\end{notice}
\begin{quote}\begin{description}
\item[{Example}] \leavevmode
\end{description}\end{quote}

\begin{Verbatim}[commandchars=\\\{\}]
\PYG{g+gp}{\PYGZgt{}\PYGZgt{}\PYGZgt{} }\PYG{n}{a}\PYG{o}{=}\PYG{n}{vcs}\PYG{o}{.}\PYG{n}{init}\PYG{p}{(}\PYG{p}{)}
\PYG{g+gp}{\PYGZgt{}\PYGZgt{}\PYGZgt{} }\PYG{n}{a}\PYG{o}{.}\PYG{n}{clean\PYGZus{}auto\PYGZus{}generated\PYGZus{}objects}\PYG{p}{(}\PYG{p}{)}
\PYG{g+gp}{\PYGZgt{}\PYGZgt{}\PYGZgt{} }\PYG{n}{a}\PYG{o}{.}\PYG{n}{show}\PYG{p}{(}\PYG{l+s+s1}{\PYGZsq{}}\PYG{l+s+s1}{texttable}\PYG{l+s+s1}{\PYGZsq{}}\PYG{p}{)} \PYG{c+c1}{\PYGZsh{} Show all the existing texttable objects}
\PYG{g+go}{*******************Texttable Names List**********************}
\PYG{g+gp}{...}
\PYG{g+go}{*******************End Texttable Names List**********************}
\PYG{g+gp}{\PYGZgt{}\PYGZgt{}\PYGZgt{} }\PYG{n}{a}\PYG{o}{.}\PYG{n}{show}\PYG{p}{(}\PYG{l+s+s1}{\PYGZsq{}}\PYG{l+s+s1}{textorientation}\PYG{l+s+s1}{\PYGZsq{}}\PYG{p}{)} \PYG{c+c1}{\PYGZsh{} Show all the existing textorientation objects}
\PYG{g+go}{*******************Textorientation Names List**********************}
\PYG{g+gp}{...}
\PYG{g+go}{*******************End Textorientation Names List**********************}
\PYG{g+gp}{\PYGZgt{}\PYGZgt{}\PYGZgt{} }\PYG{n}{vcs}\PYG{o}{.}\PYG{n}{createtext}\PYG{p}{(}\PYG{l+s+s1}{\PYGZsq{}}\PYG{l+s+s1}{qa\PYGZus{}tta}\PYG{l+s+s1}{\PYGZsq{}}\PYG{p}{,} \PYG{l+s+s1}{\PYGZsq{}}\PYG{l+s+s1}{qa}\PYG{l+s+s1}{\PYGZsq{}}\PYG{p}{,} \PYG{l+s+s1}{\PYGZsq{}}\PYG{l+s+s1}{7left\PYGZus{}tto}\PYG{l+s+s1}{\PYGZsq{}}\PYG{p}{,} \PYG{l+s+s1}{\PYGZsq{}}\PYG{l+s+s1}{7left}\PYG{l+s+s1}{\PYGZsq{}}\PYG{p}{)} \PYG{c+c1}{\PYGZsh{} Create instance of \PYGZsq{}std\PYGZus{}tt\PYGZsq{} and \PYGZsq{}7left\PYGZus{}to\PYGZsq{}}
\PYG{g+go}{\PYGZlt{}vcs.textcombined.Tc object at ...\PYGZgt{}}
\PYG{g+gp}{\PYGZgt{}\PYGZgt{}\PYGZgt{} }\PYG{n}{tc}\PYG{o}{=}\PYG{n}{a}\PYG{o}{.}\PYG{n}{gettext}\PYG{p}{(}\PYG{l+s+s1}{\PYGZsq{}}\PYG{l+s+s1}{qa\PYGZus{}tta}\PYG{l+s+s1}{\PYGZsq{}}\PYG{p}{,} \PYG{l+s+s1}{\PYGZsq{}}\PYG{l+s+s1}{7left\PYGZus{}tto}\PYG{l+s+s1}{\PYGZsq{}}\PYG{p}{)}
\PYG{g+gp}{\PYGZgt{}\PYGZgt{}\PYGZgt{} }\PYG{n}{tc}\PYG{o}{.}\PYG{n}{string}\PYG{o}{=}\PYG{l+s+s1}{\PYGZsq{}}\PYG{l+s+s1}{Text1}\PYG{l+s+s1}{\PYGZsq{}} \PYG{c+c1}{\PYGZsh{} Show the string \PYGZdq{}Text1\PYGZdq{} on the VCS Canvas}
\PYG{g+gp}{\PYGZgt{}\PYGZgt{}\PYGZgt{} }\PYG{n}{tc}\PYG{o}{.}\PYG{n}{font}\PYG{o}{=}\PYG{l+m+mi}{2} \PYG{c+c1}{\PYGZsh{} Set the text size}
\PYG{g+gp}{\PYGZgt{}\PYGZgt{}\PYGZgt{} }\PYG{n}{tc}\PYG{o}{.}\PYG{n}{color}\PYG{o}{=}\PYG{l+m+mi}{242} \PYG{c+c1}{\PYGZsh{} Set the text color}
\PYG{g+gp}{\PYGZgt{}\PYGZgt{}\PYGZgt{} }\PYG{n}{tc}\PYG{o}{.}\PYG{n}{angle}\PYG{o}{=}\PYG{l+m+mi}{45} \PYG{c+c1}{\PYGZsh{} Set the text angle}
\PYG{g+gp}{\PYGZgt{}\PYGZgt{}\PYGZgt{} }\PYG{n}{tc}\PYG{o}{.}\PYG{n}{x}\PYG{o}{=}\PYG{p}{[}\PYG{l+m+mf}{0.5}\PYG{p}{]}
\PYG{g+gp}{\PYGZgt{}\PYGZgt{}\PYGZgt{} }\PYG{n}{tc}\PYG{o}{.}\PYG{n}{y}\PYG{o}{=}\PYG{p}{[}\PYG{l+m+mf}{0.5}\PYG{p}{]}
\PYG{g+gp}{\PYGZgt{}\PYGZgt{}\PYGZgt{} }\PYG{n}{a}\PYG{o}{.}\PYG{n}{textcombined}\PYG{p}{(}\PYG{n}{tc}\PYG{p}{)} \PYG{c+c1}{\PYGZsh{} Plot using specified text object}
\PYG{g+go}{\PYGZlt{}vcs.displayplot.Dp ...\PYGZgt{}}
\end{Verbatim}
\begin{quote}\begin{description}
\item[{Returns}] \leavevmode
A fillarea object

\item[{Return type}] \leavevmode
{\hyperref[vcs/misc/displayplot:vcs.displayplot.Dp]{\sphinxcrossref{vcs.displayplot.Dp}}}

\end{description}\end{quote}

\end{fulllineitems}

\index{update() (vcs.Canvas.Canvas method)}

\begin{fulllineitems}
\phantomsection\label{vcs/Canvas:vcs.Canvas.Canvas.update}\pysiglinewithargsret{\sphinxbfcode{update}}{\emph{*args}, \emph{**kargs}}{}
If a series of commands are given to VCS and the Canvas Mode is
set to manual, then use this function to update the plot(s)
manually.
\begin{quote}\begin{description}
\item[{Example}] \leavevmode
\begin{Verbatim}[commandchars=\\\{\}]
\PYG{g+gp}{\PYGZgt{}\PYGZgt{}\PYGZgt{} }\PYG{n}{a}\PYG{o}{=}\PYG{n}{vcs}\PYG{o}{.}\PYG{n}{init}\PYG{p}{(}\PYG{p}{)}
\PYG{g+gp}{\PYGZgt{}\PYGZgt{}\PYGZgt{} }\PYG{k+kn}{import} \PYG{n+nn}{cdms2} \PYG{c+c1}{\PYGZsh{} We need cdms2 to create a slab}
\PYG{g+gp}{\PYGZgt{}\PYGZgt{}\PYGZgt{} }\PYG{n}{f} \PYG{o}{=} \PYG{n}{cdms2}\PYG{o}{.}\PYG{n}{open}\PYG{p}{(}\PYG{n}{vcs}\PYG{o}{.}\PYG{n}{sample\PYGZus{}data}\PYG{o}{+}\PYG{l+s+s1}{\PYGZsq{}}\PYG{l+s+s1}{/clt.nc}\PYG{l+s+s1}{\PYGZsq{}}\PYG{p}{)} \PYG{c+c1}{\PYGZsh{} use cdms2 to open a data file}
\PYG{g+gp}{\PYGZgt{}\PYGZgt{}\PYGZgt{} }\PYG{n}{s} \PYG{o}{=} \PYG{n}{f}\PYG{p}{(}\PYG{l+s+s1}{\PYGZsq{}}\PYG{l+s+s1}{clt}\PYG{l+s+s1}{\PYGZsq{}}\PYG{p}{)} \PYG{c+c1}{\PYGZsh{} use the data file to create a slab}
\PYG{g+gp}{\PYGZgt{}\PYGZgt{}\PYGZgt{} }\PYG{n}{a}\PYG{o}{.}\PYG{n}{plot}\PYG{p}{(}\PYG{n}{s}\PYG{p}{,}\PYG{l+s+s1}{\PYGZsq{}}\PYG{l+s+s1}{default}\PYG{l+s+s1}{\PYGZsq{}}\PYG{p}{,}\PYG{l+s+s1}{\PYGZsq{}}\PYG{l+s+s1}{boxfill}\PYG{l+s+s1}{\PYGZsq{}}\PYG{p}{,}\PYG{l+s+s1}{\PYGZsq{}}\PYG{l+s+s1}{quick}\PYG{l+s+s1}{\PYGZsq{}}\PYG{p}{)}
\PYG{g+go}{\PYGZlt{}vcs.displayplot.Dp ...\PYGZgt{}}
\PYG{g+gp}{\PYGZgt{}\PYGZgt{}\PYGZgt{} }\PYG{n}{a}\PYG{o}{.}\PYG{n}{mode} \PYG{o}{=} \PYG{l+m+mi}{0} \PYG{c+c1}{\PYGZsh{} Go to manual mode}
\PYG{g+gp}{\PYGZgt{}\PYGZgt{}\PYGZgt{} }\PYG{n}{box}\PYG{o}{=}\PYG{n}{a}\PYG{o}{.}\PYG{n}{getboxfill}\PYG{p}{(}\PYG{l+s+s1}{\PYGZsq{}}\PYG{l+s+s1}{quick}\PYG{l+s+s1}{\PYGZsq{}}\PYG{p}{)}
\PYG{g+gp}{\PYGZgt{}\PYGZgt{}\PYGZgt{} }\PYG{n}{box}\PYG{o}{.}\PYG{n}{color\PYGZus{}1}\PYG{o}{=}\PYG{l+m+mi}{100}
\PYG{g+gp}{\PYGZgt{}\PYGZgt{}\PYGZgt{} }\PYG{n}{box}\PYG{o}{.}\PYG{n}{xticlabels}\PYG{p}{(}\PYG{l+s+s1}{\PYGZsq{}}\PYG{l+s+s1}{lon30}\PYG{l+s+s1}{\PYGZsq{}}\PYG{p}{,}\PYG{l+s+s1}{\PYGZsq{}}\PYG{l+s+s1}{lon30}\PYG{l+s+s1}{\PYGZsq{}}\PYG{p}{)}
\PYG{g+gp}{\PYGZgt{}\PYGZgt{}\PYGZgt{} }\PYG{n}{box}\PYG{o}{.}\PYG{n}{xticlabels}\PYG{p}{(}\PYG{l+s+s1}{\PYGZsq{}}\PYG{l+s+s1}{\PYGZsq{}}\PYG{p}{,}\PYG{l+s+s1}{\PYGZsq{}}\PYG{l+s+s1}{\PYGZsq{}}\PYG{p}{)}
\PYG{g+gp}{\PYGZgt{}\PYGZgt{}\PYGZgt{} }\PYG{n}{box}\PYG{o}{.}\PYG{n}{datawc}\PYG{p}{(}\PYG{l+m+mi}{1}\PYG{n}{e20}\PYG{p}{,}\PYG{l+m+mi}{1}\PYG{n}{e20}\PYG{p}{,}\PYG{l+m+mi}{1}\PYG{n}{e20}\PYG{p}{,}\PYG{l+m+mi}{1}\PYG{n}{e20}\PYG{p}{)}
\PYG{g+gp}{\PYGZgt{}\PYGZgt{}\PYGZgt{} }\PYG{n}{box}\PYG{o}{.}\PYG{n}{datawc}\PYG{p}{(}\PYG{o}{\PYGZhy{}}\PYG{l+m+mf}{45.0}\PYG{p}{,} \PYG{l+m+mf}{45.0}\PYG{p}{,} \PYG{o}{\PYGZhy{}}\PYG{l+m+mf}{90.0}\PYG{p}{,} \PYG{l+m+mf}{90.0}\PYG{p}{)}
\PYG{g+gp}{\PYGZgt{}\PYGZgt{}\PYGZgt{} }\PYG{n}{a}\PYG{o}{.}\PYG{n}{update}\PYG{p}{(}\PYG{p}{)} \PYG{c+c1}{\PYGZsh{} Update the changes manually}
\end{Verbatim}

\end{description}\end{quote}

\end{fulllineitems}

\index{updateorientation() (vcs.Canvas.Canvas method)}

\begin{fulllineitems}
\phantomsection\label{vcs/Canvas:vcs.Canvas.Canvas.updateorientation}\pysiglinewithargsret{\sphinxbfcode{updateorientation}}{\emph{*args}}{}~
\DUrole{versionmodified}{Deprecated since version 2.0: }Use {\hyperref[vcs/Canvas:vcs.Canvas.Canvas.landscape]{\sphinxcrossref{\sphinxcode{landscape()}}}} or {\hyperref[vcs/Canvas:vcs.Canvas.Canvas.portrait]{\sphinxcrossref{\sphinxcode{portrait()}}}} instead.

\end{fulllineitems}

\index{vector() (vcs.Canvas.Canvas method)}

\begin{fulllineitems}
\phantomsection\label{vcs/Canvas:vcs.Canvas.Canvas.vector}\pysiglinewithargsret{\sphinxbfcode{vector}}{\emph{*args}, \emph{**parms}}{}
Generate a vector plot given the data, vector graphics method, and
template. If no vector class object is given, then the `default' vector
graphics method is used. Similarly, if no template class object is given,
then the `default' template is used.
\begin{quote}\begin{description}
\item[{Example}] \leavevmode
\begin{Verbatim}[commandchars=\\\{\}]
\PYG{g+gp}{\PYGZgt{}\PYGZgt{}\PYGZgt{} }\PYG{n}{a}\PYG{o}{=}\PYG{n}{vcs}\PYG{o}{.}\PYG{n}{init}\PYG{p}{(}\PYG{p}{)}
\PYG{g+gp}{\PYGZgt{}\PYGZgt{}\PYGZgt{} }\PYG{n}{a}\PYG{o}{.}\PYG{n}{show}\PYG{p}{(}\PYG{l+s+s1}{\PYGZsq{}}\PYG{l+s+s1}{vector}\PYG{l+s+s1}{\PYGZsq{}}\PYG{p}{)} \PYG{c+c1}{\PYGZsh{} Show all the existing vector graphics methods}
\PYG{g+go}{*******************Vector Names List**********************}
\PYG{g+gp}{...}
\PYG{g+go}{*******************End Vector Names List**********************}
\PYG{g+gp}{\PYGZgt{}\PYGZgt{}\PYGZgt{} }\PYG{k+kn}{import} \PYG{n+nn}{cdms2} \PYG{c+c1}{\PYGZsh{} Need cdms2 to create a slab}
\PYG{g+gp}{\PYGZgt{}\PYGZgt{}\PYGZgt{} }\PYG{n}{f} \PYG{o}{=} \PYG{n}{cdms2}\PYG{o}{.}\PYG{n}{open}\PYG{p}{(}\PYG{n}{vcs}\PYG{o}{.}\PYG{n}{sample\PYGZus{}data}\PYG{o}{+}\PYG{l+s+s1}{\PYGZsq{}}\PYG{l+s+s1}{/clt.nc}\PYG{l+s+s1}{\PYGZsq{}}\PYG{p}{)} \PYG{c+c1}{\PYGZsh{} use cdms2 to open a data file}
\PYG{g+gp}{\PYGZgt{}\PYGZgt{}\PYGZgt{} }\PYG{n}{slab1} \PYG{o}{=} \PYG{n}{f}\PYG{p}{(}\PYG{l+s+s1}{\PYGZsq{}}\PYG{l+s+s1}{u}\PYG{l+s+s1}{\PYGZsq{}}\PYG{p}{)} \PYG{c+c1}{\PYGZsh{} use the data file to create a cdms2 slab}
\PYG{g+gp}{\PYGZgt{}\PYGZgt{}\PYGZgt{} }\PYG{n}{slab2} \PYG{o}{=} \PYG{n}{f}\PYG{p}{(}\PYG{l+s+s1}{\PYGZsq{}}\PYG{l+s+s1}{v}\PYG{l+s+s1}{\PYGZsq{}}\PYG{p}{)} \PYG{c+c1}{\PYGZsh{} vector needs 2 slabs, so get another}
\PYG{g+gp}{\PYGZgt{}\PYGZgt{}\PYGZgt{} }\PYG{n}{a}\PYG{o}{.}\PYG{n}{vector}\PYG{p}{(}\PYG{n}{slab1}\PYG{p}{,} \PYG{n}{slab2}\PYG{p}{)} \PYG{c+c1}{\PYGZsh{} plot vector using slab and default vector}
\PYG{g+go}{\PYGZlt{}vcs.displayplot.Dp ...\PYGZgt{}}
\PYG{g+gp}{\PYGZgt{}\PYGZgt{}\PYGZgt{} }\PYG{n}{a}\PYG{o}{.}\PYG{n}{clear}\PYG{p}{(}\PYG{p}{)} \PYG{c+c1}{\PYGZsh{} Clear VCS canvas}
\PYG{g+gp}{\PYGZgt{}\PYGZgt{}\PYGZgt{} }\PYG{n}{template}\PYG{o}{=}\PYG{n}{a}\PYG{o}{.}\PYG{n}{gettemplate}\PYG{p}{(}\PYG{l+s+s1}{\PYGZsq{}}\PYG{l+s+s1}{hovmuller}\PYG{l+s+s1}{\PYGZsq{}}\PYG{p}{)}
\PYG{g+gp}{\PYGZgt{}\PYGZgt{}\PYGZgt{} }\PYG{n}{a}\PYG{o}{.}\PYG{n}{vector}\PYG{p}{(}\PYG{n}{slab1}\PYG{p}{,} \PYG{n}{slab2}\PYG{p}{,} \PYG{n}{template}\PYG{p}{)} \PYG{c+c1}{\PYGZsh{} Plot array using default vector and specified template}
\PYG{g+go}{\PYGZlt{}vcs.displayplot.Dp ...\PYGZgt{}}
\end{Verbatim}

\item[{Returns}] \leavevmode
A VCS displayplot object.

\item[{Return type}] \leavevmode
{\hyperref[vcs/misc/displayplot:vcs.displayplot.Dp]{\sphinxcrossref{vcs.displayplot.Dp}}}

\end{description}\end{quote}

\end{fulllineitems}

\index{xvsy() (vcs.Canvas.Canvas method)}

\begin{fulllineitems}
\phantomsection\label{vcs/Canvas:vcs.Canvas.Canvas.xvsy}\pysiglinewithargsret{\sphinxbfcode{xvsy}}{\emph{*args}, \emph{**parms}}{}~\begin{quote}

Generate a XvsY plot given the data, XvsY graphics method, and
template. If no XvsY class object is given, then the `default' XvsY
graphics method is used. Similarly, if no template class object is given,
then the `default' template is used.
\begin{quote}\begin{description}
\item[{Example}] \leavevmode
\begin{Verbatim}[commandchars=\\\{\}]
\PYG{g+gp}{\PYGZgt{}\PYGZgt{}\PYGZgt{} }\PYG{n}{a}\PYG{o}{=}\PYG{n}{vcs}\PYG{o}{.}\PYG{n}{init}\PYG{p}{(}\PYG{p}{)}
\PYG{g+gp}{\PYGZgt{}\PYGZgt{}\PYGZgt{} }\PYG{n}{a}\PYG{o}{.}\PYG{n}{show}\PYG{p}{(}\PYG{l+s+s1}{\PYGZsq{}}\PYG{l+s+s1}{xvsy}\PYG{l+s+s1}{\PYGZsq{}}\PYG{p}{)} \PYG{c+c1}{\PYGZsh{} Show all the existing XvsY graphics methods}
\PYG{g+go}{*******************Xvsy Names List**********************}
\PYG{g+gp}{...}
\PYG{g+go}{*******************End Xvsy Names List**********************}
\PYG{g+gp}{\PYGZgt{}\PYGZgt{}\PYGZgt{} }\PYG{n}{xy}\PYG{o}{=}\PYG{n}{a}\PYG{o}{.}\PYG{n}{getxvsy}\PYG{p}{(}\PYG{l+s+s1}{\PYGZsq{}}\PYG{l+s+s1}{default\PYGZus{}xvsy\PYGZus{}}\PYG{l+s+s1}{\PYGZsq{}}\PYG{p}{)} \PYG{c+c1}{\PYGZsh{} Create instance of default xvsy}
\PYG{g+gp}{\PYGZgt{}\PYGZgt{}\PYGZgt{} }\PYG{k+kn}{import} \PYG{n+nn}{cdms2} \PYG{c+c1}{\PYGZsh{} Need cdms2 to create a slab}
\PYG{g+gp}{\PYGZgt{}\PYGZgt{}\PYGZgt{} }\PYG{n}{f} \PYG{o}{=} \PYG{n}{cdms2}\PYG{o}{.}\PYG{n}{open}\PYG{p}{(}\PYG{n}{vcs}\PYG{o}{.}\PYG{n}{sample\PYGZus{}data}\PYG{o}{+}\PYG{l+s+s1}{\PYGZsq{}}\PYG{l+s+s1}{/clt.nc}\PYG{l+s+s1}{\PYGZsq{}}\PYG{p}{)} \PYG{c+c1}{\PYGZsh{} use cdms2 to open a data file}
\PYG{g+gp}{\PYGZgt{}\PYGZgt{}\PYGZgt{} }\PYG{n}{slab1} \PYG{o}{=} \PYG{n}{f}\PYG{p}{(}\PYG{l+s+s1}{\PYGZsq{}}\PYG{l+s+s1}{u}\PYG{l+s+s1}{\PYGZsq{}}\PYG{p}{)} \PYG{c+c1}{\PYGZsh{} use the data file to create a cdms2 slab}
\PYG{g+gp}{\PYGZgt{}\PYGZgt{}\PYGZgt{} }\PYG{n}{slab2} \PYG{o}{=} \PYG{n}{f}\PYG{p}{(}\PYG{l+s+s1}{\PYGZsq{}}\PYG{l+s+s1}{v}\PYG{l+s+s1}{\PYGZsq{}}\PYG{p}{)} \PYG{c+c1}{\PYGZsh{} use the data file to create a cdms2 slab}
\PYG{g+gp}{\PYGZgt{}\PYGZgt{}\PYGZgt{} }\PYG{n}{a}\PYG{o}{.}\PYG{n}{xvsy}\PYG{p}{(}\PYG{n}{slab1}\PYG{p}{,}\PYG{n}{slab2}\PYG{p}{,}\PYG{n}{xy}\PYG{p}{)} \PYG{c+c1}{\PYGZsh{} Plot array using specified xy and default template}
\PYG{g+go}{\PYGZlt{}vcs.displayplot.Dp ...\PYGZgt{}}
\PYG{g+gp}{\PYGZgt{}\PYGZgt{}\PYGZgt{} }\PYG{n}{a}\PYG{o}{.}\PYG{n}{clear}\PYG{p}{(}\PYG{p}{)} \PYG{c+c1}{\PYGZsh{} Clear VCS canvas}
\PYG{g+gp}{\PYGZgt{}\PYGZgt{}\PYGZgt{} }\PYG{n}{template}\PYG{o}{=}\PYG{n}{a}\PYG{o}{.}\PYG{n}{gettemplate}\PYG{p}{(}\PYG{l+s+s1}{\PYGZsq{}}\PYG{l+s+s1}{hovmuller}\PYG{l+s+s1}{\PYGZsq{}}\PYG{p}{)}
\PYG{g+gp}{\PYGZgt{}\PYGZgt{}\PYGZgt{} }\PYG{n}{a}\PYG{o}{.}\PYG{n}{xvsy}\PYG{p}{(}\PYG{n}{slab1}\PYG{p}{,}\PYG{n}{slab2}\PYG{p}{,}\PYG{n}{xy}\PYG{p}{,}\PYG{n}{template}\PYG{p}{)} \PYG{c+c1}{\PYGZsh{} Plot array using specified xy and template}
\PYG{g+go}{\PYGZlt{}vcs.displayplot.Dp ...\PYGZgt{}}
\end{Verbatim}

\end{description}\end{quote}
\end{quote}
\begin{quote}\begin{description}
\item[{Parameters}] \leavevmode\begin{itemize}
\item {} 
\textbf{\texttt{xaxis}} (\emph{\texttt{cdms2.axis.TransientAxis}}) -- Axis object to replace the slab -1 dim axis

\item {} 
\textbf{\texttt{yaxis}} (\emph{\texttt{cdms2.axis.TransientAxis}}) -- Axis object to replace the slab -2 dim axis, only if slab has more than 1D

\item {} 
\textbf{\texttt{zaxis}} (\emph{\texttt{cdms2.axis.TransientAxis}}) -- Axis object to replace the slab -3 dim axis, only if slab has more than 2D

\item {} 
\textbf{\texttt{taxis}} (\emph{\texttt{cdms2.axis.TransientAxis}}) -- Axis object to replace the slab -4 dim axis, only if slab has more than 3D

\item {} 
\textbf{\texttt{waxis}} (\emph{\texttt{cdms2.axis.TransientAxis}}) -- Axis object to replace the slab -5 dim axis, only if slab has more than 4D

\item {} 
\textbf{\texttt{xrev}} (\href{https://docs.python.org/2/library/functions.html\#bool}{\emph{\texttt{bool}}}) -- reverse x axis

\item {} 
\textbf{\texttt{yrev}} (\href{https://docs.python.org/2/library/functions.html\#bool}{\emph{\texttt{bool}}}) -- reverse y axis, only if slab has more than 1D

\item {} 
\textbf{\texttt{xarray}} (\href{https://docs.python.org/2/library/array.html\#module-array}{\emph{\texttt{array}}}) -- Values to use instead of x axis

\item {} 
\textbf{\texttt{yarray}} (\href{https://docs.python.org/2/library/array.html\#module-array}{\emph{\texttt{array}}}) -- Values to use instead of y axis, only if var has more than 1D

\item {} 
\textbf{\texttt{zarray}} (\href{https://docs.python.org/2/library/array.html\#module-array}{\emph{\texttt{array}}}) -- Values to use instead of z axis, only if var has more than 2D

\item {} 
\textbf{\texttt{tarray}} (\href{https://docs.python.org/2/library/array.html\#module-array}{\emph{\texttt{array}}}) -- Values to use instead of t axis, only if var has more than 3D

\item {} 
\textbf{\texttt{warray}} (\href{https://docs.python.org/2/library/array.html\#module-array}{\emph{\texttt{array}}}) -- Values to use instead of w axis, only if var has more than 4D

\item {} 
\textbf{\texttt{continents}} (\href{https://docs.python.org/2/library/functions.html\#int}{\emph{\texttt{int}}}) -- continents type number

\item {} 
\textbf{\texttt{name}} (\href{https://docs.python.org/2/library/functions.html\#str}{\emph{\texttt{str}}}) -- replaces variable name on plot

\item {} 
\textbf{\texttt{time}} (\emph{\texttt{A cdtime object}}) -- replaces time name on plot

\item {} 
\textbf{\texttt{units}} (\href{https://docs.python.org/2/library/functions.html\#str}{\emph{\texttt{str}}}) -- replaces units value on plot

\item {} 
\textbf{\texttt{ymd}} (\href{https://docs.python.org/2/library/functions.html\#str}{\emph{\texttt{str}}}) -- replaces year/month/day on plot

\item {} 
\textbf{\texttt{hms}} (\href{https://docs.python.org/2/library/functions.html\#str}{\emph{\texttt{str}}}) -- replaces hh/mm/ss on plot

\item {} 
\textbf{\texttt{file\_comment}} (\href{https://docs.python.org/2/library/functions.html\#str}{\emph{\texttt{str}}}) -- replaces file\_comment on plot

\item {} 
\textbf{\texttt{xbounds}} (\href{https://docs.python.org/2/library/array.html\#module-array}{\emph{\texttt{array}}}) -- Values to use instead of x axis bounds values

\item {} 
\textbf{\texttt{ybounds}} (\href{https://docs.python.org/2/library/array.html\#module-array}{\emph{\texttt{array}}}) -- Values to use instead of y axis bounds values (if exist)

\item {} 
\textbf{\texttt{xname}} (\href{https://docs.python.org/2/library/functions.html\#str}{\emph{\texttt{str}}}) -- replace xaxis name on plot

\item {} 
\textbf{\texttt{yname}} (\href{https://docs.python.org/2/library/functions.html\#str}{\emph{\texttt{str}}}) -- replace yaxis name on plot (if exists)

\item {} 
\textbf{\texttt{zname}} (\href{https://docs.python.org/2/library/functions.html\#str}{\emph{\texttt{str}}}) -- replace zaxis name on plot (if exists)

\item {} 
\textbf{\texttt{tname}} (\href{https://docs.python.org/2/library/functions.html\#str}{\emph{\texttt{str}}}) -- replace taxis name on plot (if exists)

\item {} 
\textbf{\texttt{wname}} (\href{https://docs.python.org/2/library/functions.html\#str}{\emph{\texttt{str}}}) -- replace waxis name on plot (if exists)

\item {} 
\textbf{\texttt{xunits}} (\href{https://docs.python.org/2/library/functions.html\#str}{\emph{\texttt{str}}}) -- replace xaxis units on plot

\item {} 
\textbf{\texttt{yunits}} (\href{https://docs.python.org/2/library/functions.html\#str}{\emph{\texttt{str}}}) -- replace yaxis units on plot (if exists)

\item {} 
\textbf{\texttt{zunits}} (\href{https://docs.python.org/2/library/functions.html\#str}{\emph{\texttt{str}}}) -- replace zaxis units on plot (if exists)

\item {} 
\textbf{\texttt{tunits}} (\href{https://docs.python.org/2/library/functions.html\#str}{\emph{\texttt{str}}}) -- replace taxis units on plot (if exists)

\item {} 
\textbf{\texttt{wunits}} (\href{https://docs.python.org/2/library/functions.html\#str}{\emph{\texttt{str}}}) -- replace waxis units on plot (if exists)

\item {} 
\textbf{\texttt{xweights}} (\href{https://docs.python.org/2/library/array.html\#module-array}{\emph{\texttt{array}}}) -- replace xaxis weights used for computing mean

\item {} 
\textbf{\texttt{yweights}} (\href{https://docs.python.org/2/library/array.html\#module-array}{\emph{\texttt{array}}}) -- replace xaxis weights used for computing mean

\item {} 
\textbf{\texttt{comment1}} (\href{https://docs.python.org/2/library/functions.html\#str}{\emph{\texttt{str}}}) -- replaces comment1 on plot

\item {} 
\textbf{\texttt{comment2}} (\href{https://docs.python.org/2/library/functions.html\#str}{\emph{\texttt{str}}}) -- replaces comment2 on plot

\item {} 
\textbf{\texttt{comment3}} (\href{https://docs.python.org/2/library/functions.html\#str}{\emph{\texttt{str}}}) -- replaces comment3 on plot

\item {} 
\textbf{\texttt{comment4}} (\href{https://docs.python.org/2/library/functions.html\#str}{\emph{\texttt{str}}}) -- replaces comment4 on plot

\item {} 
\textbf{\texttt{long\_name}} (\href{https://docs.python.org/2/library/functions.html\#str}{\emph{\texttt{str}}}) -- replaces long\_name on plot

\item {} 
\textbf{\texttt{grid}} (\emph{\texttt{cdms2.grid.TransientRectGrid}}) -- replaces array grid (if exists)

\item {} 
\textbf{\texttt{bg}} (\emph{\texttt{bool/int}}) -- plots in background mode

\item {} 
\textbf{\texttt{ratio}} (\index{xmtics1 (vcs.Canvas.Canvas attribute)}\index{xmtics2 (vcs.Canvas.Canvas attribute)}\index{ymtics1 (vcs.Canvas.Canvas attribute)}\index{ymtics2 (vcs.Canvas.Canvas attribute)}\index{xticlabels1 (vcs.Canvas.Canvas attribute)}\index{xticlabels2 (vcs.Canvas.Canvas attribute)}\index{yticlabels1 (vcs.Canvas.Canvas attribute)}\index{yticlabels2 (vcs.Canvas.Canvas attribute)}\index{projection (vcs.Canvas.Canvas attribute)}\index{datawc\_x1 (vcs.Canvas.Canvas attribute)}\index{datawc\_x2 (vcs.Canvas.Canvas attribute)}\index{datawc\_y1 (vcs.Canvas.Canvas attribute)}\index{datawc\_y2 (vcs.Canvas.Canvas attribute)}\index{datawc\_timeunits (vcs.Canvas.Canvas attribute)}\index{datawc\_calendar (vcs.Canvas.Canvas attribute)}) -- sets the y/x ratio ,if passed as a string with `t' at the end, will aslo moves the ticks

\item {} 
\textbf{\texttt{xaxisconvert}} (\href{https://docs.python.org/2/library/functions.html\#str}{\emph{\texttt{str}}}) -- (Ex: `linear') converting xaxis linear/log/log10/ln/exp/area\_wt

\item {} 
\textbf{\texttt{yaxisconvert}} (\href{https://docs.python.org/2/library/functions.html\#str}{\emph{\texttt{str}}}) -- (Ex: `linear') converting yaxis linear/log/log10/ln/exp/area\_wt

\item {} 
\textbf{\texttt{slab\_or\_primary\_object}} (\href{https://docs.python.org/2/library/array.html\#module-array}{\emph{\texttt{array}}}) -- Data at least 1D, last dimension(s) will be plotted, or secondary vcs object

\end{itemize}

\item[{Returns}] \leavevmode
Display Plot object representing the plot.

\item[{Return type}] \leavevmode

vcs.displayplot.Dp
\begin{quote}\begin{description}
\item[{returns}] \leavevmode
A VCS displayplot object.

\item[{rtype}] \leavevmode
vcs.displayplot.Dp

\end{description}\end{quote}


\end{description}\end{quote}

\end{fulllineitems}

\index{xyvsy() (vcs.Canvas.Canvas method)}

\begin{fulllineitems}
\phantomsection\label{vcs/Canvas:vcs.Canvas.Canvas.xyvsy}\pysiglinewithargsret{\sphinxbfcode{xyvsy}}{\emph{*args}, \emph{**parms}}{}~\begin{quote}

Generate a Xyvsy plot given the data, Xyvsy graphics method, and
template. If no Xyvsy class object is given, then the `default' Xyvsy
graphics method is used. Similarly, if no template class object is given,
then the `default' template is used.
\begin{quote}\begin{description}
\item[{Example}] \leavevmode
\begin{Verbatim}[commandchars=\\\{\}]
\PYG{g+gp}{\PYGZgt{}\PYGZgt{}\PYGZgt{} }\PYG{n}{a}\PYG{o}{=}\PYG{n}{vcs}\PYG{o}{.}\PYG{n}{init}\PYG{p}{(}\PYG{p}{)}
\PYG{g+gp}{\PYGZgt{}\PYGZgt{}\PYGZgt{} }\PYG{n}{a}\PYG{o}{.}\PYG{n}{show}\PYG{p}{(}\PYG{l+s+s1}{\PYGZsq{}}\PYG{l+s+s1}{xyvsy}\PYG{l+s+s1}{\PYGZsq{}}\PYG{p}{)} \PYG{c+c1}{\PYGZsh{} Show all the existing Xyvsy graphics methods}
\PYG{g+go}{*******************Xyvsy Names List**********************}
\PYG{g+gp}{...}
\PYG{g+go}{*******************End Xyvsy Names List**********************}
\PYG{g+gp}{\PYGZgt{}\PYGZgt{}\PYGZgt{} }\PYG{n}{xyy}\PYG{o}{=}\PYG{n}{a}\PYG{o}{.}\PYG{n}{getxyvsy}\PYG{p}{(}\PYG{l+s+s1}{\PYGZsq{}}\PYG{l+s+s1}{default\PYGZus{}xyvsy\PYGZus{}}\PYG{l+s+s1}{\PYGZsq{}}\PYG{p}{)} \PYG{c+c1}{\PYGZsh{} Create instance of default xyvsy}
\PYG{g+gp}{\PYGZgt{}\PYGZgt{}\PYGZgt{} }\PYG{n}{array}\PYG{o}{=}\PYG{p}{[}\PYG{n+nb}{range}\PYG{p}{(}\PYG{l+m+mi}{1}\PYG{p}{,} \PYG{l+m+mi}{11}\PYG{p}{)} \PYG{k}{for} \PYG{n}{\PYGZus{}} \PYG{o+ow}{in} \PYG{n+nb}{range}\PYG{p}{(}\PYG{l+m+mi}{1}\PYG{p}{,} \PYG{l+m+mi}{11}\PYG{p}{)}\PYG{p}{]}
\PYG{g+gp}{\PYGZgt{}\PYGZgt{}\PYGZgt{} }\PYG{n}{a}\PYG{o}{.}\PYG{n}{xyvsy}\PYG{p}{(}\PYG{n}{array}\PYG{p}{,}\PYG{n}{xyy}\PYG{p}{)} \PYG{c+c1}{\PYGZsh{} Plot array using specified xyy and default template}
\PYG{g+go}{\PYGZlt{}vcs.displayplot.Dp ...\PYGZgt{}}
\PYG{g+gp}{\PYGZgt{}\PYGZgt{}\PYGZgt{} }\PYG{n}{a}\PYG{o}{.}\PYG{n}{clear}\PYG{p}{(}\PYG{p}{)} \PYG{c+c1}{\PYGZsh{} Clear VCS canvas}
\PYG{g+gp}{\PYGZgt{}\PYGZgt{}\PYGZgt{} }\PYG{n}{template}\PYG{o}{=}\PYG{n}{a}\PYG{o}{.}\PYG{n}{gettemplate}\PYG{p}{(}\PYG{l+s+s1}{\PYGZsq{}}\PYG{l+s+s1}{hovmuller}\PYG{l+s+s1}{\PYGZsq{}}\PYG{p}{)}
\PYG{g+gp}{\PYGZgt{}\PYGZgt{}\PYGZgt{} }\PYG{n}{a}\PYG{o}{.}\PYG{n}{xyvsy}\PYG{p}{(}\PYG{n}{array}\PYG{p}{,}\PYG{n}{xyy}\PYG{p}{,}\PYG{n}{template}\PYG{p}{)} \PYG{c+c1}{\PYGZsh{} Plot array using specified xyy and template}
\PYG{g+go}{\PYGZlt{}vcs.displayplot.Dp ...\PYGZgt{}}
\end{Verbatim}

\end{description}\end{quote}
\end{quote}
\begin{quote}\begin{description}
\item[{Parameters}] \leavevmode\begin{itemize}
\item {} 
\textbf{\texttt{xaxis}} (\emph{\texttt{cdms2.axis.TransientAxis}}) -- Axis object to replace the slab -1 dim axis

\item {} 
\textbf{\texttt{yaxis}} (\emph{\texttt{cdms2.axis.TransientAxis}}) -- Axis object to replace the slab -2 dim axis, only if slab has more than 1D

\item {} 
\textbf{\texttt{zaxis}} (\emph{\texttt{cdms2.axis.TransientAxis}}) -- Axis object to replace the slab -3 dim axis, only if slab has more than 2D

\item {} 
\textbf{\texttt{taxis}} (\emph{\texttt{cdms2.axis.TransientAxis}}) -- Axis object to replace the slab -4 dim axis, only if slab has more than 3D

\item {} 
\textbf{\texttt{waxis}} (\emph{\texttt{cdms2.axis.TransientAxis}}) -- Axis object to replace the slab -5 dim axis, only if slab has more than 4D

\item {} 
\textbf{\texttt{xrev}} (\href{https://docs.python.org/2/library/functions.html\#bool}{\emph{\texttt{bool}}}) -- reverse x axis

\item {} 
\textbf{\texttt{yrev}} (\href{https://docs.python.org/2/library/functions.html\#bool}{\emph{\texttt{bool}}}) -- reverse y axis, only if slab has more than 1D

\item {} 
\textbf{\texttt{xarray}} (\href{https://docs.python.org/2/library/array.html\#module-array}{\emph{\texttt{array}}}) -- Values to use instead of x axis

\item {} 
\textbf{\texttt{yarray}} (\href{https://docs.python.org/2/library/array.html\#module-array}{\emph{\texttt{array}}}) -- Values to use instead of y axis, only if var has more than 1D

\item {} 
\textbf{\texttt{zarray}} (\href{https://docs.python.org/2/library/array.html\#module-array}{\emph{\texttt{array}}}) -- Values to use instead of z axis, only if var has more than 2D

\item {} 
\textbf{\texttt{tarray}} (\href{https://docs.python.org/2/library/array.html\#module-array}{\emph{\texttt{array}}}) -- Values to use instead of t axis, only if var has more than 3D

\item {} 
\textbf{\texttt{warray}} (\href{https://docs.python.org/2/library/array.html\#module-array}{\emph{\texttt{array}}}) -- Values to use instead of w axis, only if var has more than 4D

\item {} 
\textbf{\texttt{continents}} (\href{https://docs.python.org/2/library/functions.html\#int}{\emph{\texttt{int}}}) -- continents type number

\item {} 
\textbf{\texttt{name}} (\href{https://docs.python.org/2/library/functions.html\#str}{\emph{\texttt{str}}}) -- replaces variable name on plot

\item {} 
\textbf{\texttt{time}} (\emph{\texttt{A cdtime object}}) -- replaces time name on plot

\item {} 
\textbf{\texttt{units}} (\href{https://docs.python.org/2/library/functions.html\#str}{\emph{\texttt{str}}}) -- replaces units value on plot

\item {} 
\textbf{\texttt{ymd}} (\href{https://docs.python.org/2/library/functions.html\#str}{\emph{\texttt{str}}}) -- replaces year/month/day on plot

\item {} 
\textbf{\texttt{hms}} (\href{https://docs.python.org/2/library/functions.html\#str}{\emph{\texttt{str}}}) -- replaces hh/mm/ss on plot

\item {} 
\textbf{\texttt{file\_comment}} (\href{https://docs.python.org/2/library/functions.html\#str}{\emph{\texttt{str}}}) -- replaces file\_comment on plot

\item {} 
\textbf{\texttt{xbounds}} (\href{https://docs.python.org/2/library/array.html\#module-array}{\emph{\texttt{array}}}) -- Values to use instead of x axis bounds values

\item {} 
\textbf{\texttt{ybounds}} (\href{https://docs.python.org/2/library/array.html\#module-array}{\emph{\texttt{array}}}) -- Values to use instead of y axis bounds values (if exist)

\item {} 
\textbf{\texttt{xname}} (\href{https://docs.python.org/2/library/functions.html\#str}{\emph{\texttt{str}}}) -- replace xaxis name on plot

\item {} 
\textbf{\texttt{yname}} (\href{https://docs.python.org/2/library/functions.html\#str}{\emph{\texttt{str}}}) -- replace yaxis name on plot (if exists)

\item {} 
\textbf{\texttt{zname}} (\href{https://docs.python.org/2/library/functions.html\#str}{\emph{\texttt{str}}}) -- replace zaxis name on plot (if exists)

\item {} 
\textbf{\texttt{tname}} (\href{https://docs.python.org/2/library/functions.html\#str}{\emph{\texttt{str}}}) -- replace taxis name on plot (if exists)

\item {} 
\textbf{\texttt{wname}} (\href{https://docs.python.org/2/library/functions.html\#str}{\emph{\texttt{str}}}) -- replace waxis name on plot (if exists)

\item {} 
\textbf{\texttt{xunits}} (\href{https://docs.python.org/2/library/functions.html\#str}{\emph{\texttt{str}}}) -- replace xaxis units on plot

\item {} 
\textbf{\texttt{yunits}} (\href{https://docs.python.org/2/library/functions.html\#str}{\emph{\texttt{str}}}) -- replace yaxis units on plot (if exists)

\item {} 
\textbf{\texttt{zunits}} (\href{https://docs.python.org/2/library/functions.html\#str}{\emph{\texttt{str}}}) -- replace zaxis units on plot (if exists)

\item {} 
\textbf{\texttt{tunits}} (\href{https://docs.python.org/2/library/functions.html\#str}{\emph{\texttt{str}}}) -- replace taxis units on plot (if exists)

\item {} 
\textbf{\texttt{wunits}} (\href{https://docs.python.org/2/library/functions.html\#str}{\emph{\texttt{str}}}) -- replace waxis units on plot (if exists)

\item {} 
\textbf{\texttt{xweights}} (\href{https://docs.python.org/2/library/array.html\#module-array}{\emph{\texttt{array}}}) -- replace xaxis weights used for computing mean

\item {} 
\textbf{\texttt{yweights}} (\href{https://docs.python.org/2/library/array.html\#module-array}{\emph{\texttt{array}}}) -- replace xaxis weights used for computing mean

\item {} 
\textbf{\texttt{comment1}} (\href{https://docs.python.org/2/library/functions.html\#str}{\emph{\texttt{str}}}) -- replaces comment1 on plot

\item {} 
\textbf{\texttt{comment2}} (\href{https://docs.python.org/2/library/functions.html\#str}{\emph{\texttt{str}}}) -- replaces comment2 on plot

\item {} 
\textbf{\texttt{comment3}} (\href{https://docs.python.org/2/library/functions.html\#str}{\emph{\texttt{str}}}) -- replaces comment3 on plot

\item {} 
\textbf{\texttt{comment4}} (\href{https://docs.python.org/2/library/functions.html\#str}{\emph{\texttt{str}}}) -- replaces comment4 on plot

\item {} 
\textbf{\texttt{long\_name}} (\href{https://docs.python.org/2/library/functions.html\#str}{\emph{\texttt{str}}}) -- replaces long\_name on plot

\item {} 
\textbf{\texttt{grid}} (\emph{\texttt{cdms2.grid.TransientRectGrid}}) -- replaces array grid (if exists)

\item {} 
\textbf{\texttt{bg}} (\emph{\texttt{bool/int}}) -- plots in background mode

\item {} 
\textbf{\texttt{ratio}} (\index{xmtics1 (vcs.Canvas.Canvas attribute)}\index{xmtics2 (vcs.Canvas.Canvas attribute)}\index{ymtics1 (vcs.Canvas.Canvas attribute)}\index{ymtics2 (vcs.Canvas.Canvas attribute)}\index{xticlabels1 (vcs.Canvas.Canvas attribute)}\index{xticlabels2 (vcs.Canvas.Canvas attribute)}\index{yticlabels1 (vcs.Canvas.Canvas attribute)}\index{yticlabels2 (vcs.Canvas.Canvas attribute)}\index{projection (vcs.Canvas.Canvas attribute)}\index{datawc\_x1 (vcs.Canvas.Canvas attribute)}\index{datawc\_x2 (vcs.Canvas.Canvas attribute)}\index{datawc\_y1 (vcs.Canvas.Canvas attribute)}\index{datawc\_y2 (vcs.Canvas.Canvas attribute)}\index{datawc\_timeunits (vcs.Canvas.Canvas attribute)}\index{datawc\_calendar (vcs.Canvas.Canvas attribute)}) -- sets the y/x ratio ,if passed as a string with `t' at the end, will aslo moves the ticks

\item {} 
\textbf{\texttt{xaxisconvert}} (\href{https://docs.python.org/2/library/functions.html\#str}{\emph{\texttt{str}}}) -- (Ex: `linear') converting xaxis linear/log/log10/ln/exp/area\_wt

\item {} 
\textbf{\texttt{slab}} (\href{https://docs.python.org/2/library/array.html\#module-array}{\emph{\texttt{array}}}) -- (Ex: {[}1, 2{]}) Data at least 1D, last dimension will be plotted

\end{itemize}

\item[{Returns}] \leavevmode
Display Plot object representing the plot.

\item[{Return type}] \leavevmode

vcs.displayplot.Dp
\begin{quote}\begin{description}
\item[{returns}] \leavevmode
A VCS displayplot object.

\item[{rtype}] \leavevmode
vcs.displayplot.Dp

\end{description}\end{quote}


\end{description}\end{quote}

\end{fulllineitems}

\index{yxvsx() (vcs.Canvas.Canvas method)}

\begin{fulllineitems}
\phantomsection\label{vcs/Canvas:vcs.Canvas.Canvas.yxvsx}\pysiglinewithargsret{\sphinxbfcode{yxvsx}}{\emph{*args}, \emph{**parms}}{}~\begin{quote}

Generate a Yxvsx plot given the data, Yxvsx graphics method, and
template. If no Yxvsx class object is given, then the `default' Yxvsx
graphics method is used. Simerly, if no template class object is given,
then the `default' template is used.
\begin{quote}\begin{description}
\item[{Example}] \leavevmode
\begin{Verbatim}[commandchars=\\\{\}]
\PYG{g+gp}{\PYGZgt{}\PYGZgt{}\PYGZgt{} }\PYG{n}{a}\PYG{o}{=}\PYG{n}{vcs}\PYG{o}{.}\PYG{n}{init}\PYG{p}{(}\PYG{p}{)}
\PYG{g+gp}{\PYGZgt{}\PYGZgt{}\PYGZgt{} }\PYG{n}{a}\PYG{o}{.}\PYG{n}{show}\PYG{p}{(}\PYG{l+s+s1}{\PYGZsq{}}\PYG{l+s+s1}{yxvsx}\PYG{l+s+s1}{\PYGZsq{}}\PYG{p}{)} \PYG{c+c1}{\PYGZsh{} Show all the existing Yxvsx graphics methods}
\PYG{g+go}{*******************Yxvsx Names List**********************}
\PYG{g+gp}{...}
\PYG{g+go}{*******************End Yxvsx Names List**********************}
\PYG{g+gp}{\PYGZgt{}\PYGZgt{}\PYGZgt{} }\PYG{n}{yxx}\PYG{o}{=}\PYG{n}{a}\PYG{o}{.}\PYG{n}{getyxvsx}\PYG{p}{(}\PYG{l+s+s1}{\PYGZsq{}}\PYG{l+s+s1}{default\PYGZus{}yxvsx\PYGZus{}}\PYG{l+s+s1}{\PYGZsq{}}\PYG{p}{)} \PYG{c+c1}{\PYGZsh{} Create instance of default yxvsx}
\PYG{g+gp}{\PYGZgt{}\PYGZgt{}\PYGZgt{} }\PYG{n}{array}\PYG{o}{=}\PYG{p}{[}\PYG{n+nb}{range}\PYG{p}{(}\PYG{l+m+mi}{1}\PYG{p}{,} \PYG{l+m+mi}{11}\PYG{p}{)} \PYG{k}{for} \PYG{n}{\PYGZus{}} \PYG{o+ow}{in} \PYG{n+nb}{range}\PYG{p}{(}\PYG{l+m+mi}{1}\PYG{p}{,} \PYG{l+m+mi}{11}\PYG{p}{)}\PYG{p}{]}
\PYG{g+gp}{\PYGZgt{}\PYGZgt{}\PYGZgt{} }\PYG{n}{a}\PYG{o}{.}\PYG{n}{yxvsx}\PYG{p}{(}\PYG{n}{array}\PYG{p}{,}\PYG{n}{yxx}\PYG{p}{)} \PYG{c+c1}{\PYGZsh{} Plot array using specified yxx and default template}
\PYG{g+go}{\PYGZlt{}vcs.displayplot.Dp ...\PYGZgt{}}
\PYG{g+gp}{\PYGZgt{}\PYGZgt{}\PYGZgt{} }\PYG{n}{a}\PYG{o}{.}\PYG{n}{clear}\PYG{p}{(}\PYG{p}{)} \PYG{c+c1}{\PYGZsh{} Clear VCS canvas}
\PYG{g+gp}{\PYGZgt{}\PYGZgt{}\PYGZgt{} }\PYG{n}{template}\PYG{o}{=}\PYG{n}{a}\PYG{o}{.}\PYG{n}{gettemplate}\PYG{p}{(}\PYG{l+s+s1}{\PYGZsq{}}\PYG{l+s+s1}{hovmuller}\PYG{l+s+s1}{\PYGZsq{}}\PYG{p}{)}
\PYG{g+gp}{\PYGZgt{}\PYGZgt{}\PYGZgt{} }\PYG{n}{a}\PYG{o}{.}\PYG{n}{yxvsx}\PYG{p}{(}\PYG{n}{array}\PYG{p}{,}\PYG{n}{yxx}\PYG{p}{,}\PYG{n}{template}\PYG{p}{)} \PYG{c+c1}{\PYGZsh{} Plot array using specified yxx and template}
\PYG{g+go}{\PYGZlt{}vcs.displayplot.Dp ...\PYGZgt{}}
\end{Verbatim}

\end{description}\end{quote}
\end{quote}
\begin{quote}\begin{description}
\item[{Parameters}] \leavevmode\begin{itemize}
\item {} 
\textbf{\texttt{xaxis}} (\emph{\texttt{cdms2.axis.TransientAxis}}) -- Axis object to replace the slab -1 dim axis

\item {} 
\textbf{\texttt{yaxis}} (\emph{\texttt{cdms2.axis.TransientAxis}}) -- Axis object to replace the slab -2 dim axis, only if slab has more than 1D

\item {} 
\textbf{\texttt{zaxis}} (\emph{\texttt{cdms2.axis.TransientAxis}}) -- Axis object to replace the slab -3 dim axis, only if slab has more than 2D

\item {} 
\textbf{\texttt{taxis}} (\emph{\texttt{cdms2.axis.TransientAxis}}) -- Axis object to replace the slab -4 dim axis, only if slab has more than 3D

\item {} 
\textbf{\texttt{waxis}} (\emph{\texttt{cdms2.axis.TransientAxis}}) -- Axis object to replace the slab -5 dim axis, only if slab has more than 4D

\item {} 
\textbf{\texttt{xrev}} (\href{https://docs.python.org/2/library/functions.html\#bool}{\emph{\texttt{bool}}}) -- reverse x axis

\item {} 
\textbf{\texttt{yrev}} (\href{https://docs.python.org/2/library/functions.html\#bool}{\emph{\texttt{bool}}}) -- reverse y axis, only if slab has more than 1D

\item {} 
\textbf{\texttt{xarray}} (\href{https://docs.python.org/2/library/array.html\#module-array}{\emph{\texttt{array}}}) -- Values to use instead of x axis

\item {} 
\textbf{\texttt{yarray}} (\href{https://docs.python.org/2/library/array.html\#module-array}{\emph{\texttt{array}}}) -- Values to use instead of y axis, only if var has more than 1D

\item {} 
\textbf{\texttt{zarray}} (\href{https://docs.python.org/2/library/array.html\#module-array}{\emph{\texttt{array}}}) -- Values to use instead of z axis, only if var has more than 2D

\item {} 
\textbf{\texttt{tarray}} (\href{https://docs.python.org/2/library/array.html\#module-array}{\emph{\texttt{array}}}) -- Values to use instead of t axis, only if var has more than 3D

\item {} 
\textbf{\texttt{warray}} (\href{https://docs.python.org/2/library/array.html\#module-array}{\emph{\texttt{array}}}) -- Values to use instead of w axis, only if var has more than 4D

\item {} 
\textbf{\texttt{continents}} (\href{https://docs.python.org/2/library/functions.html\#int}{\emph{\texttt{int}}}) -- continents type number

\item {} 
\textbf{\texttt{name}} (\href{https://docs.python.org/2/library/functions.html\#str}{\emph{\texttt{str}}}) -- replaces variable name on plot

\item {} 
\textbf{\texttt{time}} (\emph{\texttt{A cdtime object}}) -- replaces time name on plot

\item {} 
\textbf{\texttt{units}} (\href{https://docs.python.org/2/library/functions.html\#str}{\emph{\texttt{str}}}) -- replaces units value on plot

\item {} 
\textbf{\texttt{ymd}} (\href{https://docs.python.org/2/library/functions.html\#str}{\emph{\texttt{str}}}) -- replaces year/month/day on plot

\item {} 
\textbf{\texttt{hms}} (\href{https://docs.python.org/2/library/functions.html\#str}{\emph{\texttt{str}}}) -- replaces hh/mm/ss on plot

\item {} 
\textbf{\texttt{file\_comment}} (\href{https://docs.python.org/2/library/functions.html\#str}{\emph{\texttt{str}}}) -- replaces file\_comment on plot

\item {} 
\textbf{\texttt{xbounds}} (\href{https://docs.python.org/2/library/array.html\#module-array}{\emph{\texttt{array}}}) -- Values to use instead of x axis bounds values

\item {} 
\textbf{\texttt{ybounds}} (\href{https://docs.python.org/2/library/array.html\#module-array}{\emph{\texttt{array}}}) -- Values to use instead of y axis bounds values (if exist)

\item {} 
\textbf{\texttt{xname}} (\href{https://docs.python.org/2/library/functions.html\#str}{\emph{\texttt{str}}}) -- replace xaxis name on plot

\item {} 
\textbf{\texttt{yname}} (\href{https://docs.python.org/2/library/functions.html\#str}{\emph{\texttt{str}}}) -- replace yaxis name on plot (if exists)

\item {} 
\textbf{\texttt{zname}} (\href{https://docs.python.org/2/library/functions.html\#str}{\emph{\texttt{str}}}) -- replace zaxis name on plot (if exists)

\item {} 
\textbf{\texttt{tname}} (\href{https://docs.python.org/2/library/functions.html\#str}{\emph{\texttt{str}}}) -- replace taxis name on plot (if exists)

\item {} 
\textbf{\texttt{wname}} (\href{https://docs.python.org/2/library/functions.html\#str}{\emph{\texttt{str}}}) -- replace waxis name on plot (if exists)

\item {} 
\textbf{\texttt{xunits}} (\href{https://docs.python.org/2/library/functions.html\#str}{\emph{\texttt{str}}}) -- replace xaxis units on plot

\item {} 
\textbf{\texttt{yunits}} (\href{https://docs.python.org/2/library/functions.html\#str}{\emph{\texttt{str}}}) -- replace yaxis units on plot (if exists)

\item {} 
\textbf{\texttt{zunits}} (\href{https://docs.python.org/2/library/functions.html\#str}{\emph{\texttt{str}}}) -- replace zaxis units on plot (if exists)

\item {} 
\textbf{\texttt{tunits}} (\href{https://docs.python.org/2/library/functions.html\#str}{\emph{\texttt{str}}}) -- replace taxis units on plot (if exists)

\item {} 
\textbf{\texttt{wunits}} (\href{https://docs.python.org/2/library/functions.html\#str}{\emph{\texttt{str}}}) -- replace waxis units on plot (if exists)

\item {} 
\textbf{\texttt{xweights}} (\href{https://docs.python.org/2/library/array.html\#module-array}{\emph{\texttt{array}}}) -- replace xaxis weights used for computing mean

\item {} 
\textbf{\texttt{yweights}} (\href{https://docs.python.org/2/library/array.html\#module-array}{\emph{\texttt{array}}}) -- replace xaxis weights used for computing mean

\item {} 
\textbf{\texttt{comment1}} (\href{https://docs.python.org/2/library/functions.html\#str}{\emph{\texttt{str}}}) -- replaces comment1 on plot

\item {} 
\textbf{\texttt{comment2}} (\href{https://docs.python.org/2/library/functions.html\#str}{\emph{\texttt{str}}}) -- replaces comment2 on plot

\item {} 
\textbf{\texttt{comment3}} (\href{https://docs.python.org/2/library/functions.html\#str}{\emph{\texttt{str}}}) -- replaces comment3 on plot

\item {} 
\textbf{\texttt{comment4}} (\href{https://docs.python.org/2/library/functions.html\#str}{\emph{\texttt{str}}}) -- replaces comment4 on plot

\item {} 
\textbf{\texttt{long\_name}} (\href{https://docs.python.org/2/library/functions.html\#str}{\emph{\texttt{str}}}) -- replaces long\_name on plot

\item {} 
\textbf{\texttt{grid}} (\emph{\texttt{cdms2.grid.TransientRectGrid}}) -- replaces array grid (if exists)

\item {} 
\textbf{\texttt{bg}} (\emph{\texttt{bool/int}}) -- plots in background mode

\item {} 
\textbf{\texttt{ratio}} (\index{xmtics1 (vcs.Canvas.Canvas attribute)}\index{xmtics2 (vcs.Canvas.Canvas attribute)}\index{ymtics1 (vcs.Canvas.Canvas attribute)}\index{ymtics2 (vcs.Canvas.Canvas attribute)}\index{xticlabels1 (vcs.Canvas.Canvas attribute)}\index{xticlabels2 (vcs.Canvas.Canvas attribute)}\index{yticlabels1 (vcs.Canvas.Canvas attribute)}\index{yticlabels2 (vcs.Canvas.Canvas attribute)}\index{projection (vcs.Canvas.Canvas attribute)}\index{datawc\_x1 (vcs.Canvas.Canvas attribute)}\index{datawc\_x2 (vcs.Canvas.Canvas attribute)}\index{datawc\_y1 (vcs.Canvas.Canvas attribute)}\index{datawc\_y2 (vcs.Canvas.Canvas attribute)}\index{datawc\_timeunits (vcs.Canvas.Canvas attribute)}\index{datawc\_calendar (vcs.Canvas.Canvas attribute)}) -- sets the y/x ratio ,if passed as a string with `t' at the end, will aslo moves the ticks

\item {} 
\textbf{\texttt{xaxisconvert}} (\href{https://docs.python.org/2/library/functions.html\#str}{\emph{\texttt{str}}}) -- (Ex: `linear') converting xaxis linear/log/log10/ln/exp/area\_wt

\item {} 
\textbf{\texttt{slab}} (\href{https://docs.python.org/2/library/array.html\#module-array}{\emph{\texttt{array}}}) -- (Ex: {[}1, 2{]}) Data at least 1D, last dimension will be plotted

\end{itemize}

\item[{Returns}] \leavevmode
Display Plot object representing the plot.

\item[{Return type}] \leavevmode

vcs.displayplot.Dp
\begin{quote}\begin{description}
\item[{returns}] \leavevmode
A VCS displayplot object.

\item[{rtype}] \leavevmode
vcs.displayplot.Dp

\end{description}\end{quote}


\end{description}\end{quote}

\end{fulllineitems}


\end{fulllineitems}



\section{Graphics Methods}
\label{vcs/graphics/gm:graphics-methods}\label{vcs/graphics/gm::doc}
Graphics methods are VCS' objects for configuring your visualization's data representation. They allow you to set levels, colors, subset your data, set patterns, and much more.


\subsection{boxfill}
\label{vcs/graphics/boxfill:boxfill}\label{vcs/graphics/boxfill::doc}\label{vcs/graphics/boxfill:module-vcs.boxfill}\index{vcs.boxfill (module)}
\# Boxfill (Gfb) module
\index{Gfb (class in vcs.boxfill)}

\begin{fulllineitems}
\phantomsection\label{vcs/graphics/boxfill:vcs.boxfill.Gfb}\pysiglinewithargsret{\sphinxstrong{class }\sphinxcode{vcs.boxfill.}\sphinxbfcode{Gfb}}{\emph{Gfb\_name=None}, \emph{Gfb\_name\_src='default'}}{}
The boxfill graphics method (Gfb) displays a two-dimensional data array
by surrounding each data value by a colored grid box.

This class is used to define a boxfill table entry used in VCS, or it
can be used to change some or all of the attributes in an existing
boxfill table entry.


\begin{fulllineitems}
\pysigline{\sphinxbfcode{General~use~of~a~boxfill:}}~
\begin{Verbatim}[commandchars=\\\{\}]
\PYG{c+c1}{\PYGZsh{} Constructor}
\PYG{n}{a}\PYG{o}{=}\PYG{n}{vcs}\PYG{o}{.}\PYG{n}{init}\PYG{p}{(}\PYG{p}{)}
\PYG{c+c1}{\PYGZsh{} Show predefined boxfill graphics methods}
\PYG{n}{a}\PYG{o}{.}\PYG{n}{show}\PYG{p}{(}\PYG{l+s+s1}{\PYGZsq{}}\PYG{l+s+s1}{boxfill}\PYG{l+s+s1}{\PYGZsq{}}\PYG{p}{)}
\PYG{c+c1}{\PYGZsh{} Change the VCS color map}
\PYG{n}{a}\PYG{o}{.}\PYG{n}{setcolormap}\PYG{p}{(}\PYG{l+s+s2}{\PYGZdq{}}\PYG{l+s+s2}{AMIP}\PYG{l+s+s2}{\PYGZdq{}}\PYG{p}{)}
\PYG{c+c1}{\PYGZsh{} Plot data \PYGZsq{}s\PYGZsq{} with boxfill \PYGZsq{}b\PYGZsq{} and \PYGZsq{}default\PYGZsq{} template}
\PYG{n}{a}\PYG{o}{.}\PYG{n}{boxfill}\PYG{p}{(}\PYG{n}{s}\PYG{p}{,}\PYG{n}{b}\PYG{p}{,}\PYG{l+s+s1}{\PYGZsq{}}\PYG{l+s+s1}{default}\PYG{l+s+s1}{\PYGZsq{}}\PYG{p}{)}
\end{Verbatim}

\end{fulllineitems}



\begin{fulllineitems}
\pysigline{\sphinxbfcode{Updating~a~boxfill:}}~
\begin{Verbatim}[commandchars=\\\{\}]
\PYG{c+c1}{\PYGZsh{} Updates the VCS Canvas at user\PYGZsq{}s request}
\PYG{n}{a}\PYG{o}{.}\PYG{n}{update}\PYG{p}{(}\PYG{p}{)}
\PYG{c+c1}{\PYGZsh{} Set VCS Canvas to automatic update mode}
\PYG{n}{a}\PYG{o}{.}\PYG{n}{mode}\PYG{o}{=}\PYG{l+m+mi}{1}
\PYG{c+c1}{\PYGZsh{} Use update function to update the VCS Canvas}
\PYG{n}{a}\PYG{o}{.}\PYG{n}{mode}\PYG{o}{=}\PYG{l+m+mi}{0}
\end{Verbatim}

\end{fulllineitems}



\begin{fulllineitems}
\pysigline{\sphinxbfcode{Create~a~new~instance~of~boxfill:}}~
\begin{Verbatim}[commandchars=\\\{\}]
\PYG{c+c1}{\PYGZsh{}  Copies content of \PYGZsq{}quick\PYGZsq{} to \PYGZsq{}new\PYGZsq{}}
\PYG{n}{box}\PYG{o}{=}\PYG{n}{a}\PYG{o}{.}\PYG{n}{createboxfill}\PYG{p}{(}\PYG{l+s+s1}{\PYGZsq{}}\PYG{l+s+s1}{new}\PYG{l+s+s1}{\PYGZsq{}}\PYG{p}{,}\PYG{l+s+s1}{\PYGZsq{}}\PYG{l+s+s1}{quick}\PYG{l+s+s1}{\PYGZsq{}}\PYG{p}{)}
\PYG{c+c1}{\PYGZsh{}  Copies content of \PYGZsq{}default\PYGZsq{} to \PYGZsq{}new\PYGZsq{}}
\PYG{n}{box}\PYG{o}{=}\PYG{n}{a}\PYG{o}{.}\PYG{n}{createboxfill}\PYG{p}{(}\PYG{l+s+s1}{\PYGZsq{}}\PYG{l+s+s1}{new}\PYG{l+s+s1}{\PYGZsq{}}\PYG{p}{)}
\end{Verbatim}

\end{fulllineitems}



\begin{fulllineitems}
\pysigline{\sphinxbfcode{Modifying~an~existing~boxfill:}}~
\begin{Verbatim}[commandchars=\\\{\}]
\PYG{n}{fill}\PYG{o}{=}\PYG{n}{a}\PYG{o}{.}\PYG{n}{getboxfill}\PYG{p}{(}\PYG{l+s+s1}{\PYGZsq{}}\PYG{l+s+s1}{quick}\PYG{l+s+s1}{\PYGZsq{}}\PYG{p}{)}

\PYG{c+c1}{\PYGZsh{} Set index using fillarea}
\PYG{n}{box}\PYG{o}{.}\PYG{n}{fillareaindices}\PYG{o}{=}\PYG{p}{(}\PYG{l+m+mi}{7}\PYG{p}{,}\PYG{n}{fill}\PYG{p}{,}\PYG{l+m+mi}{4}\PYG{p}{,}\PYG{l+m+mi}{9}\PYG{p}{,}\PYG{n}{fill}\PYG{p}{,}\PYG{l+m+mi}{15}\PYG{p}{)}
\PYG{c+c1}{\PYGZsh{} list fillarea attributes}
\PYG{n}{fill}\PYG{o}{.}\PYG{n}{list}\PYG{p}{(}\PYG{p}{)}
\PYG{c+c1}{\PYGZsh{} change style}
\PYG{n}{fill}\PYG{o}{.}\PYG{n}{style}\PYG{o}{=}\PYG{l+s+s1}{\PYGZsq{}}\PYG{l+s+s1}{hatch}\PYG{l+s+s1}{\PYGZsq{}}
\PYG{c+c1}{\PYGZsh{} change color}
\PYG{n}{fill}\PYG{o}{.}\PYG{n}{color}\PYG{o}{=}\PYG{l+m+mi}{241}
\PYG{c+c1}{\PYGZsh{} change style index}
\PYG{n}{fill}\PYG{o}{.}\PYG{n}{index}\PYG{o}{=}\PYG{l+m+mi}{3}
\end{Verbatim}

\end{fulllineitems}



\begin{fulllineitems}
\pysigline{\sphinxbfcode{Overview~of~boxfill~attributes:}}~\begin{itemize}
\item {} 
Listing all the boxfill attribute values:
\begin{quote}

\begin{Verbatim}[commandchars=\\\{\}]
\PYG{n}{box}\PYG{o}{.}\PYG{n}{list}\PYG{p}{(}\PYG{p}{)}
\end{Verbatim}
\end{quote}

\item {} 
Setting boxfill attribute values:
\begin{quote}

\begin{Verbatim}[commandchars=\\\{\}]
\PYG{n}{box}\PYG{o}{.}\PYG{n}{projection}\PYG{o}{=}\PYG{l+s+s1}{\PYGZsq{}}\PYG{l+s+s1}{linear}\PYG{l+s+s1}{\PYGZsq{}}
\PYG{n}{lon30}\PYG{o}{=}\PYG{p}{\PYGZob{}}\PYG{o}{\PYGZhy{}}\PYG{l+m+mi}{180}\PYG{p}{:}\PYG{l+s+s1}{\PYGZsq{}}\PYG{l+s+s1}{180W}\PYG{l+s+s1}{\PYGZsq{}}\PYG{p}{,}\PYG{o}{\PYGZhy{}}\PYG{l+m+mi}{150}\PYG{p}{:}\PYG{l+s+s1}{\PYGZsq{}}\PYG{l+s+s1}{150W}\PYG{l+s+s1}{\PYGZsq{}}\PYG{p}{,}\PYG{l+m+mi}{0}\PYG{p}{:}\PYG{l+s+s1}{\PYGZsq{}}\PYG{l+s+s1}{Eq}\PYG{l+s+s1}{\PYGZsq{}}\PYG{p}{\PYGZcb{}}
\PYG{n}{box}\PYG{o}{.}\PYG{n}{xticlabels1}\PYG{o}{=}\PYG{n}{lon30}
\PYG{n}{box}\PYG{o}{.}\PYG{n}{xticlabels2}\PYG{o}{=}\PYG{n}{lon30}
\PYG{c+c1}{\PYGZsh{} Will set them both}
\PYG{n}{box}\PYG{o}{.}\PYG{n}{xticlabels}\PYG{p}{(}\PYG{n}{lon30}\PYG{p}{,} \PYG{n}{lon30}\PYG{p}{)}
\PYG{n}{box}\PYG{o}{.}\PYG{n}{xmtics1}\PYG{o}{=}\PYG{l+s+s1}{\PYGZsq{}}\PYG{l+s+s1}{\PYGZsq{}}
\PYG{n}{box}\PYG{o}{.}\PYG{n}{xmtics2}\PYG{o}{=}\PYG{l+s+s1}{\PYGZsq{}}\PYG{l+s+s1}{\PYGZsq{}}
\PYG{c+c1}{\PYGZsh{} Will set them both}
\PYG{n}{box}\PYG{o}{.}\PYG{n}{xmtics}\PYG{p}{(}\PYG{n}{lon30}\PYG{p}{,} \PYG{n}{lon30}\PYG{p}{)}
\PYG{n}{box}\PYG{o}{.}\PYG{n}{yticlabels1}\PYG{o}{=}\PYG{n}{lat10}
\PYG{n}{box}\PYG{o}{.}\PYG{n}{yticlabels2}\PYG{o}{=}\PYG{n}{lat10}
\PYG{c+c1}{\PYGZsh{} Will set them both}
\PYG{n}{box}\PYG{o}{.}\PYG{n}{yticlabels}\PYG{p}{(}\PYG{n}{lat10}\PYG{p}{,} \PYG{n}{lat10}\PYG{p}{)}
\PYG{n}{box}\PYG{o}{.}\PYG{n}{ymtics1}\PYG{o}{=}\PYG{l+s+s1}{\PYGZsq{}}\PYG{l+s+s1}{\PYGZsq{}}
\PYG{n}{box}\PYG{o}{.}\PYG{n}{ymtics2}\PYG{o}{=}\PYG{l+s+s1}{\PYGZsq{}}\PYG{l+s+s1}{\PYGZsq{}}
\PYG{c+c1}{\PYGZsh{} Will set them both}
\PYG{n}{box}\PYG{o}{.}\PYG{n}{ymtics}\PYG{p}{(}\PYG{n}{lat10}\PYG{p}{,} \PYG{n}{lat10}\PYG{p}{)}
\PYG{n}{box}\PYG{o}{.}\PYG{n}{datawc\PYGZus{}y1}\PYG{o}{=}\PYG{o}{\PYGZhy{}}\PYG{l+m+mf}{90.0}
\PYG{n}{box}\PYG{o}{.}\PYG{n}{datawc\PYGZus{}y2}\PYG{o}{=}\PYG{l+m+mf}{90.0}
\PYG{n}{box}\PYG{o}{.}\PYG{n}{datawc\PYGZus{}x1}\PYG{o}{=}\PYG{o}{\PYGZhy{}}\PYG{l+m+mf}{180.0}
\PYG{n}{box}\PYG{o}{.}\PYG{n}{datawc\PYGZus{}x2}\PYG{o}{=}\PYG{l+m+mf}{180.0}
\PYG{c+c1}{\PYGZsh{} Will set them all}
\PYG{n}{box}\PYG{o}{.}\PYG{n}{datawc}\PYG{p}{(}\PYG{o}{\PYGZhy{}}\PYG{l+m+mi}{90}\PYG{p}{,} \PYG{l+m+mi}{90}\PYG{p}{,} \PYG{o}{\PYGZhy{}}\PYG{l+m+mi}{180}\PYG{p}{,} \PYG{l+m+mi}{180}\PYG{p}{)}
\PYG{n}{box}\PYG{o}{.}\PYG{n}{xaxisconvert}\PYG{o}{=}\PYG{l+s+s1}{\PYGZsq{}}\PYG{l+s+s1}{linear}\PYG{l+s+s1}{\PYGZsq{}}
\PYG{n}{box}\PYG{o}{.}\PYG{n}{yaxisconvert}\PYG{o}{=}\PYG{l+s+s1}{\PYGZsq{}}\PYG{l+s+s1}{linear}\PYG{l+s+s1}{\PYGZsq{}}
\PYG{c+c1}{\PYGZsh{} Will set them both}
\PYG{n}{box}\PYG{o}{.}\PYG{n}{xyscale}\PYG{p}{(}\PYG{l+s+s1}{\PYGZsq{}}\PYG{l+s+s1}{linear}\PYG{l+s+s1}{\PYGZsq{}}\PYG{p}{,} \PYG{l+s+s1}{\PYGZsq{}}\PYG{l+s+s1}{area\PYGZus{}wt}\PYG{l+s+s1}{\PYGZsq{}}\PYG{p}{)}
\PYG{n}{box}\PYG{o}{.}\PYG{n}{level\PYGZus{}1}\PYG{o}{=}\PYG{l+m+mf}{1e20}
\PYG{n}{box}\PYG{o}{.}\PYG{n}{level\PYGZus{}2}\PYG{o}{=}\PYG{l+m+mf}{1e20}
\PYG{n}{box}\PYG{o}{.}\PYG{n}{color\PYGZus{}1}\PYG{o}{=}\PYG{l+m+mi}{16}
\PYG{n}{box}\PYG{o}{.}\PYG{n}{color\PYGZus{}2}\PYG{o}{=}\PYG{l+m+mi}{239}
\PYG{c+c1}{\PYGZsh{} Will set them both}
\PYG{n}{box}\PYG{o}{.}\PYG{n}{colors}\PYG{p}{(}\PYG{l+m+mi}{16}\PYG{p}{,} \PYG{l+m+mi}{239} \PYG{p}{)}
\PYG{c+c1}{\PYGZsh{} \PYGZsq{}linear\PYGZsq{} \PYGZhy{} compute or specify legend}
\PYG{n}{box}\PYG{o}{.}\PYG{n}{boxfill\PYGZus{}type}\PYG{o}{=}\PYG{l+s+s1}{\PYGZsq{}}\PYG{l+s+s1}{linear}\PYG{l+s+s1}{\PYGZsq{}}
\PYG{c+c1}{\PYGZsh{} \PYGZsq{}log10\PYGZsq{} \PYGZhy{} plot using log10}
\PYG{n}{box}\PYG{o}{.}\PYG{n}{boxfill\PYGZus{}type}\PYG{o}{=}\PYG{l+s+s1}{\PYGZsq{}}\PYG{l+s+s1}{log10}\PYG{l+s+s1}{\PYGZsq{}}
\PYG{c+c1}{\PYGZsh{} \PYGZsq{}custom\PYGZsq{} \PYGZhy{} use custom values to display legend evenly}
\PYG{n}{box}\PYG{o}{.}\PYG{n}{boxfill\PYGZus{}type}\PYG{o}{=}\PYG{l+s+s1}{\PYGZsq{}}\PYG{l+s+s1}{custom}\PYG{l+s+s1}{\PYGZsq{}}
\PYG{c+c1}{\PYGZsh{} Hold the legend values}
\PYG{n}{box}\PYG{o}{.}\PYG{n}{legend}\PYG{o}{=}\PYG{n+nb+bp}{None}
\PYG{c+c1}{\PYGZsh{} Show left overflow arrow}
\PYG{n}{box}\PYG{o}{.}\PYG{n}{ext\PYGZus{}1}\PYG{o}{=}\PYG{l+s+s1}{\PYGZsq{}}\PYG{l+s+s1}{n}\PYG{l+s+s1}{\PYGZsq{}}
\PYG{c+c1}{\PYGZsh{} Show right overflow arrow}
\PYG{n}{box}\PYG{o}{.}\PYG{n}{ext\PYGZus{}2}\PYG{o}{=}\PYG{l+s+s1}{\PYGZsq{}}\PYG{l+s+s1}{y}\PYG{l+s+s1}{\PYGZsq{}}
\PYG{c+c1}{\PYGZsh{} Will set them both}
\PYG{n}{box}\PYG{o}{.}\PYG{n}{exts}\PYG{p}{(}\PYG{l+s+s1}{\PYGZsq{}}\PYG{l+s+s1}{n}\PYG{l+s+s1}{\PYGZsq{}}\PYG{p}{,} \PYG{l+s+s1}{\PYGZsq{}}\PYG{l+s+s1}{y}\PYG{l+s+s1}{\PYGZsq{}} \PYG{p}{)}
\PYG{c+c1}{\PYGZsh{} Color index value range 0 to 255}
\PYG{n}{box}\PYG{o}{.}\PYG{n}{missing}\PYG{o}{=}\PYG{l+m+mi}{241}
\end{Verbatim}
\end{quote}

\item {} 
Setting the boxfill levels:
\begin{quote}

\begin{Verbatim}[commandchars=\\\{\}]
\PYG{c+c1}{\PYGZsh{} Case 1: Levels are all contiguous:}
\PYG{n}{box}\PYG{o}{.}\PYG{n}{levels}\PYG{o}{=}\PYG{p}{(}\PYG{p}{[}\PYG{l+m+mi}{0}\PYG{p}{,}\PYG{l+m+mi}{20}\PYG{p}{,}\PYG{l+m+mi}{25}\PYG{p}{,}\PYG{l+m+mi}{30}\PYG{p}{,}\PYG{l+m+mi}{35}\PYG{p}{,}\PYG{l+m+mi}{40}\PYG{p}{]}\PYG{p}{,}\PYG{p}{)}
\PYG{n}{box}\PYG{o}{.}\PYG{n}{levels}\PYG{o}{=}\PYG{p}{(}\PYG{p}{[}\PYG{l+m+mi}{0}\PYG{p}{,}\PYG{l+m+mi}{20}\PYG{p}{,}\PYG{l+m+mi}{25}\PYG{p}{,}\PYG{l+m+mi}{30}\PYG{p}{,}\PYG{l+m+mi}{35}\PYG{p}{,}\PYG{l+m+mi}{40}\PYG{p}{,}\PYG{l+m+mi}{45}\PYG{p}{,}\PYG{l+m+mi}{50}\PYG{p}{]}\PYG{p}{)}
\PYG{n}{box}\PYG{o}{.}\PYG{n}{levels}\PYG{o}{=}\PYG{p}{[}\PYG{l+m+mi}{0}\PYG{p}{,}\PYG{l+m+mi}{20}\PYG{p}{,}\PYG{l+m+mi}{25}\PYG{p}{,}\PYG{l+m+mi}{30}\PYG{p}{,}\PYG{l+m+mi}{35}\PYG{p}{,}\PYG{l+m+mi}{40}\PYG{p}{]}
\PYG{n}{box}\PYG{o}{.}\PYG{n}{levels}\PYG{o}{=}\PYG{p}{(}\PYG{l+m+mf}{0.0}\PYG{p}{,}\PYG{l+m+mf}{20.0}\PYG{p}{,}\PYG{l+m+mf}{25.0}\PYG{p}{,}\PYG{l+m+mf}{30.0}\PYG{p}{,}\PYG{l+m+mf}{35.0}\PYG{p}{,}\PYG{l+m+mf}{40.0}\PYG{p}{,}\PYG{l+m+mf}{50.0}\PYG{p}{)}

\PYG{c+c1}{\PYGZsh{} Case 2: Levels are not contiguous:}
\PYG{n}{box}\PYG{o}{.}\PYG{n}{levels}\PYG{o}{=}\PYG{p}{(}\PYG{p}{[}\PYG{l+m+mi}{0}\PYG{p}{,}\PYG{l+m+mi}{20}\PYG{p}{]}\PYG{p}{,}\PYG{p}{[}\PYG{l+m+mi}{30}\PYG{p}{,}\PYG{l+m+mi}{40}\PYG{p}{]}\PYG{p}{,}\PYG{p}{[}\PYG{l+m+mi}{50}\PYG{p}{,}\PYG{l+m+mi}{60}\PYG{p}{]}\PYG{p}{)}
\PYG{n}{box}\PYG{o}{.}\PYG{n}{levels}\PYG{o}{=}\PYG{p}{(}\PYG{p}{[}\PYG{l+m+mi}{0}\PYG{p}{,}\PYG{l+m+mi}{20}\PYG{p}{,}\PYG{l+m+mi}{25}\PYG{p}{,}\PYG{l+m+mi}{30}\PYG{p}{,}\PYG{l+m+mi}{35}\PYG{p}{,}\PYG{l+m+mi}{40}\PYG{p}{]}\PYG{p}{,}\PYG{p}{[}\PYG{l+m+mi}{30}\PYG{p}{,}\PYG{l+m+mi}{40}\PYG{p}{]}\PYG{p}{,}\PYG{p}{[}\PYG{l+m+mi}{50}\PYG{p}{,}\PYG{l+m+mi}{60}\PYG{p}{]}\PYG{p}{)}
\end{Verbatim}
\end{quote}

\item {} 
Setting the fillarea color indices:
\begin{quote}

\begin{Verbatim}[commandchars=\\\{\}]
\PYG{c+c1}{\PYGZsh{} Three different methods for setting color indices:}
\PYG{n}{box}\PYG{o}{.}\PYG{n}{fillareacolors}\PYG{o}{=}\PYG{p}{(}\PYG{p}{[}\PYG{l+m+mi}{22}\PYG{p}{,}\PYG{l+m+mi}{33}\PYG{p}{,}\PYG{l+m+mi}{44}\PYG{p}{,}\PYG{l+m+mi}{55}\PYG{p}{,}\PYG{l+m+mi}{66}\PYG{p}{,}\PYG{l+m+mi}{77}\PYG{p}{]}\PYG{p}{)}
\PYG{n}{box}\PYG{o}{.}\PYG{n}{fillareacolors}\PYG{o}{=}\PYG{p}{(}\PYG{l+m+mi}{16}\PYG{p}{,}\PYG{l+m+mi}{19}\PYG{p}{,}\PYG{l+m+mi}{33}\PYG{p}{,}\PYG{l+m+mi}{44}\PYG{p}{)}
\PYG{n}{box}\PYG{o}{.}\PYG{n}{fillareacolors}\PYG{o}{=}\PYG{n+nb+bp}{None}
\end{Verbatim}
\end{quote}

\item {} 
Setting the fillarea style:
\begin{quote}

\begin{Verbatim}[commandchars=\\\{\}]
\PYG{n}{box}\PYG{o}{.}\PYG{n}{fillareastyle} \PYG{o}{=} \PYG{l+s+s1}{\PYGZsq{}}\PYG{l+s+s1}{solid}\PYG{l+s+s1}{\PYGZsq{}}
\PYG{n}{box}\PYG{o}{.}\PYG{n}{fillareastyle} \PYG{o}{=} \PYG{l+s+s1}{\PYGZsq{}}\PYG{l+s+s1}{hatch}\PYG{l+s+s1}{\PYGZsq{}}
\PYG{n}{box}\PYG{o}{.}\PYG{n}{fillareastyle} \PYG{o}{=} \PYG{l+s+s1}{\PYGZsq{}}\PYG{l+s+s1}{pattern}\PYG{l+s+s1}{\PYGZsq{}}
\end{Verbatim}
\end{quote}

\item {} 
Setting the fillarea hatch or pattern indices:
\begin{quote}

\begin{Verbatim}[commandchars=\\\{\}]
\PYG{n}{box}\PYG{o}{.}\PYG{n}{fillareaindices}\PYG{o}{=}\PYG{p}{(}\PYG{p}{[}\PYG{l+m+mi}{1}\PYG{p}{,}\PYG{l+m+mi}{3}\PYG{p}{,}\PYG{l+m+mi}{5}\PYG{p}{,}\PYG{l+m+mi}{6}\PYG{p}{,}\PYG{l+m+mi}{9}\PYG{p}{,}\PYG{l+m+mi}{20}\PYG{p}{]}\PYG{p}{)}
\PYG{n}{box}\PYG{o}{.}\PYG{n}{fillareaindices}\PYG{o}{=}\PYG{p}{(}\PYG{l+m+mi}{7}\PYG{p}{,}\PYG{l+m+mi}{1}\PYG{p}{,}\PYG{l+m+mi}{4}\PYG{p}{,}\PYG{l+m+mi}{9}\PYG{p}{,}\PYG{l+m+mi}{6}\PYG{p}{,}\PYG{l+m+mi}{15}\PYG{p}{)}
\end{Verbatim}
\end{quote}

\item {} 
Using the fillarea secondary object (Ex):
\begin{quote}

\begin{Verbatim}[commandchars=\\\{\}]
\PYG{n}{f}\PYG{o}{=}\PYG{n}{createfillarea}\PYG{p}{(}\PYG{l+s+s1}{\PYGZsq{}}\PYG{l+s+s1}{fill1}\PYG{l+s+s1}{\PYGZsq{}}\PYG{p}{)}
\PYG{c+c1}{\PYGZsh{}To Create a new instance of fillarea use:}
\PYG{c+c1}{\PYGZsh{} Copies \PYGZsq{}quick\PYGZsq{} to \PYGZsq{}new\PYGZsq{}}
\PYG{n}{fill}\PYG{o}{=}\PYG{n}{a}\PYG{o}{.}\PYG{n}{createfillarea}\PYG{p}{(}\PYG{l+s+s1}{\PYGZsq{}}\PYG{l+s+s1}{new}\PYG{l+s+s1}{\PYGZsq{}}\PYG{p}{,}\PYG{l+s+s1}{\PYGZsq{}}\PYG{l+s+s1}{quick}\PYG{l+s+s1}{\PYGZsq{}}\PYG{p}{)}
\PYG{c+c1}{\PYGZsh{} Copies \PYGZsq{}default\PYGZsq{} to \PYGZsq{}new\PYGZsq{}}
\PYG{n}{fill}\PYG{o}{=}\PYG{n}{a}\PYG{o}{.}\PYG{n}{createfillarea}\PYG{p}{(}\PYG{l+s+s1}{\PYGZsq{}}\PYG{l+s+s1}{new}\PYG{l+s+s1}{\PYGZsq{}}\PYG{p}{)}
\end{Verbatim}
\end{quote}

\end{itemize}
\phantomsection\label{vcs/graphics/boxfill:boxfill-attribute-descriptions}\begin{itemize}
\item {} 
Attribute descriptions:
\begin{quote}
\index{boxfill\_type (vcs.boxfill.Gfb attribute)}

\begin{fulllineitems}
\phantomsection\label{vcs/graphics/boxfill:vcs.boxfill.Gfb.boxfill_type}\pysiglinewithargsret{\sphinxbfcode{boxfill\_type}}{\emph{str}}{}
Type of boxfill legend. One of `linear', `log10', or `custom'. See examples above for usage.

\end{fulllineitems}

\index{level\_1 (vcs.boxfill.Gfb attribute)}

\begin{fulllineitems}
\phantomsection\label{vcs/graphics/boxfill:vcs.boxfill.Gfb.level_1}\pysiglinewithargsret{\sphinxbfcode{level\_1}}{\emph{float}}{}
Used in conjunction with boxfill\_type linear/log10. Sets the value of the legend's first level

\end{fulllineitems}

\index{level\_2 (vcs.boxfill.Gfb attribute)}

\begin{fulllineitems}
\phantomsection\label{vcs/graphics/boxfill:vcs.boxfill.Gfb.level_2}\pysiglinewithargsret{\sphinxbfcode{level\_2}}{\emph{float}}{}
Used in conjunction with boxfill\_type linear/log10, sets the value of the legend's end level

\end{fulllineitems}

\index{color\_1 (vcs.boxfill.Gfb attribute)}

\begin{fulllineitems}
\phantomsection\label{vcs/graphics/boxfill:vcs.boxfill.Gfb.color_1}\pysiglinewithargsret{\sphinxbfcode{color\_1}}{\emph{float}}{}
Used in conjunction with boxfill\_type linear/log10, sets the legend's color range first value

\end{fulllineitems}

\index{color\_2 (vcs.boxfill.Gfb attribute)}

\begin{fulllineitems}
\phantomsection\label{vcs/graphics/boxfill:vcs.boxfill.Gfb.color_2}\pysiglinewithargsret{\sphinxbfcode{color\_2}}{\emph{float}}{}
Used in conjunction with boxfill\_type linear/log10, sets the legend's color range lasst value

\end{fulllineitems}

\index{levels (vcs.boxfill.Gfb attribute)}

\begin{fulllineitems}
\phantomsection\label{vcs/graphics/boxfill:vcs.boxfill.Gfb.levels}\pysiglinewithargsret{\sphinxbfcode{levels}}{\emph{list of floats}}{}
Used in conjunction for boxfill\_type custom, sets the levels range to use, can be
either a list of contiguous levels, or list of tuples indicating first
and last value of the range.

\end{fulllineitems}

\index{legend (vcs.boxfill.Gfb attribute)}

\begin{fulllineitems}
\phantomsection\label{vcs/graphics/boxfill:vcs.boxfill.Gfb.legend}\pysiglinewithargsret{\sphinxbfcode{legend}}{\emph{\{float:str\}}}{}
Used in conjunction with boxfill\_type linear/log10, replaces the legend values in the dictionary keys with
their associated string.

\end{fulllineitems}

\index{ext\_1 (vcs.boxfill.Gfb attribute)}

\begin{fulllineitems}
\phantomsection\label{vcs/graphics/boxfill:vcs.boxfill.Gfb.ext_1}\pysiglinewithargsret{\sphinxbfcode{ext\_1}}{\emph{str}}{}
Draws an extension arrow on right side (values less than first range value)

\end{fulllineitems}

\index{ext\_2 (vcs.boxfill.Gfb attribute)}

\begin{fulllineitems}
\phantomsection\label{vcs/graphics/boxfill:vcs.boxfill.Gfb.ext_2}\pysiglinewithargsret{\sphinxbfcode{ext\_2}}{\emph{str}}{}
Draws an extension arrow on left side (values greater than last range value)

\end{fulllineitems}

\index{missing (vcs.boxfill.Gfb attribute)}

\begin{fulllineitems}
\phantomsection\label{vcs/graphics/boxfill:vcs.boxfill.Gfb.missing}\pysiglinewithargsret{\sphinxbfcode{missing}}{\emph{int}}{}
Color to use for missing value or values not in defined ranges.

\end{fulllineitems}

\index{xmtics1 (vcs.boxfill.Gfb attribute)}

\begin{fulllineitems}
\phantomsection\label{vcs/graphics/boxfill:vcs.boxfill.Gfb.xmtics1}\pysiglinewithargsret{\sphinxbfcode{xmtics1}}{\emph{str/\{float:str\}}}{}
(Ex: `') dictionary with location of intermediate tics as keys for 1st side of y axis

\end{fulllineitems}

\index{xmtics2 (vcs.boxfill.Gfb attribute)}

\begin{fulllineitems}
\phantomsection\label{vcs/graphics/boxfill:vcs.boxfill.Gfb.xmtics2}\pysiglinewithargsret{\sphinxbfcode{xmtics2}}{\emph{str/\{float:str\}}}{}
(Ex: `') dictionary with location of intermediate tics as keys for 2nd side of y axis

\end{fulllineitems}

\index{ymtics1 (vcs.boxfill.Gfb attribute)}

\begin{fulllineitems}
\phantomsection\label{vcs/graphics/boxfill:vcs.boxfill.Gfb.ymtics1}\pysiglinewithargsret{\sphinxbfcode{ymtics1}}{\emph{str/\{float:str\}}}{}
(Ex: `') dictionary with location of intermediate tics as keys for 1st side of y axis

\end{fulllineitems}

\index{ymtics2 (vcs.boxfill.Gfb attribute)}

\begin{fulllineitems}
\phantomsection\label{vcs/graphics/boxfill:vcs.boxfill.Gfb.ymtics2}\pysiglinewithargsret{\sphinxbfcode{ymtics2}}{\emph{str/\{float:str\}}}{}
(Ex: `') dictionary with location of intermediate tics as keys for 2nd side of y axis

\end{fulllineitems}

\index{xticlabels1 (vcs.boxfill.Gfb attribute)}

\begin{fulllineitems}
\phantomsection\label{vcs/graphics/boxfill:vcs.boxfill.Gfb.xticlabels1}\pysiglinewithargsret{\sphinxbfcode{xticlabels1}}{\emph{str/\{float:str\}}}{}
(Ex: `*') values for labels on 1st side of x axis

\end{fulllineitems}

\index{xticlabels2 (vcs.boxfill.Gfb attribute)}

\begin{fulllineitems}
\phantomsection\label{vcs/graphics/boxfill:vcs.boxfill.Gfb.xticlabels2}\pysiglinewithargsret{\sphinxbfcode{xticlabels2}}{\emph{str/\{float:str\}}}{}
(Ex: `*') values for labels on 2nd side of x axis

\end{fulllineitems}

\index{yticlabels1 (vcs.boxfill.Gfb attribute)}

\begin{fulllineitems}
\phantomsection\label{vcs/graphics/boxfill:vcs.boxfill.Gfb.yticlabels1}\pysiglinewithargsret{\sphinxbfcode{yticlabels1}}{\emph{str/\{float:str\}}}{}
(Ex: `*') values for labels on 1st side of y axis

\end{fulllineitems}

\index{yticlabels2 (vcs.boxfill.Gfb attribute)}

\begin{fulllineitems}
\phantomsection\label{vcs/graphics/boxfill:vcs.boxfill.Gfb.yticlabels2}\pysiglinewithargsret{\sphinxbfcode{yticlabels2}}{\emph{str/\{float:str\}}}{}
(Ex: `*') values for labels on 2nd side of y axis

\end{fulllineitems}

\index{projection (vcs.boxfill.Gfb attribute)}

\begin{fulllineitems}
\phantomsection\label{vcs/graphics/boxfill:vcs.boxfill.Gfb.projection}\pysiglinewithargsret{\sphinxbfcode{projection}}{\emph{str/vcs.projection.Proj}}{}
(Ex: `default') projection to use, name or object

\end{fulllineitems}

\index{datawc\_x1 (vcs.boxfill.Gfb attribute)}

\begin{fulllineitems}
\phantomsection\label{vcs/graphics/boxfill:vcs.boxfill.Gfb.datawc_x1}\pysiglinewithargsret{\sphinxbfcode{datawc\_x1}}{\emph{float}}{}
(Ex: 1.E20) first value of xaxis on plot

\end{fulllineitems}

\index{datawc\_x2 (vcs.boxfill.Gfb attribute)}

\begin{fulllineitems}
\phantomsection\label{vcs/graphics/boxfill:vcs.boxfill.Gfb.datawc_x2}\pysiglinewithargsret{\sphinxbfcode{datawc\_x2}}{\emph{float}}{}
(Ex: 1.E20) second value of xaxis on plot

\end{fulllineitems}

\index{datawc\_y1 (vcs.boxfill.Gfb attribute)}

\begin{fulllineitems}
\phantomsection\label{vcs/graphics/boxfill:vcs.boxfill.Gfb.datawc_y1}\pysiglinewithargsret{\sphinxbfcode{datawc\_y1}}{\emph{float}}{}
(Ex: 1.E20) first value of yaxis on plot

\end{fulllineitems}

\index{datawc\_y2 (vcs.boxfill.Gfb attribute)}

\begin{fulllineitems}
\phantomsection\label{vcs/graphics/boxfill:vcs.boxfill.Gfb.datawc_y2}\pysiglinewithargsret{\sphinxbfcode{datawc\_y2}}{\emph{float}}{}
(Ex: 1.E20) second value of yaxis on plot

\end{fulllineitems}

\index{datawc\_timeunits (vcs.boxfill.Gfb attribute)}

\begin{fulllineitems}
\phantomsection\label{vcs/graphics/boxfill:vcs.boxfill.Gfb.datawc_timeunits}\pysiglinewithargsret{\sphinxbfcode{datawc\_timeunits}}{\emph{str}}{}
(Ex: `days since 2000') units to use when displaying time dimension auto tick

\end{fulllineitems}

\index{datawc\_calendar (vcs.boxfill.Gfb attribute)}

\begin{fulllineitems}
\phantomsection\label{vcs/graphics/boxfill:vcs.boxfill.Gfb.datawc_calendar}\pysiglinewithargsret{\sphinxbfcode{datawc\_calendar}}{\emph{int}}{}
(Ex: 135441) calendar to use when displaying time dimension auto tick, default is proleptic gregorian calendar

\end{fulllineitems}

\end{quote}

\end{itemize}

\end{fulllineitems}

\index{colors() (vcs.boxfill.Gfb method)}

\begin{fulllineitems}
\phantomsection\label{vcs/graphics/boxfill:vcs.boxfill.Gfb.colors}\pysiglinewithargsret{\sphinxbfcode{colors}}{\emph{color1=16}, \emph{color2=239}}{}
Sets the color\_1 and color\_2 properties of the object.
\begin{quote}\begin{description}
\item[{Parameters}] \leavevmode\begin{itemize}
\item {} 
\textbf{\texttt{color1}} (\href{https://docs.python.org/2/library/functions.html\#int}{\emph{\texttt{int}}}) -- Sets the {\hyperref[vcs/graphics/boxfill:vcs.boxfill.Gfb.color_1]{\sphinxcrossref{\sphinxcode{color\_1}}}} value on the object

\item {} 
\textbf{\texttt{color2}} (\href{https://docs.python.org/2/library/functions.html\#int}{\emph{\texttt{int}}}) -- Sets the {\hyperref[vcs/graphics/boxfill:vcs.boxfill.Gfb.color_2]{\sphinxcrossref{\sphinxcode{color\_2}}}} value on the object

\end{itemize}

\end{description}\end{quote}

\end{fulllineitems}

\index{datawc() (vcs.boxfill.Gfb method)}

\begin{fulllineitems}
\phantomsection\label{vcs/graphics/boxfill:vcs.boxfill.Gfb.datawc}\pysiglinewithargsret{\sphinxbfcode{datawc}}{\emph{dsp1=1e+20}, \emph{dsp2=1e+20}, \emph{dsp3=1e+20}, \emph{dsp4=1e+20}}{}
Sets the data world coordinates for object
\begin{quote}\begin{description}
\item[{Parameters}] \leavevmode\begin{itemize}
\item {} 
\textbf{\texttt{dsp1}} (\href{https://docs.python.org/2/library/functions.html\#float}{\emph{\texttt{float}}}) -- Sets the {\hyperref[vcs/graphics/boxfill:vcs.boxfill.Gfb.datawc_y1]{\sphinxcrossref{\sphinxcode{datawc\_y1}}}} property of the object.

\item {} 
\textbf{\texttt{dsp2}} (\href{https://docs.python.org/2/library/functions.html\#float}{\emph{\texttt{float}}}) -- Sets the {\hyperref[vcs/graphics/boxfill:vcs.boxfill.Gfb.datawc_y2]{\sphinxcrossref{\sphinxcode{datawc\_y2}}}} property of the object.

\item {} 
\textbf{\texttt{dsp3}} (\href{https://docs.python.org/2/library/functions.html\#float}{\emph{\texttt{float}}}) -- Sets the {\hyperref[vcs/graphics/boxfill:vcs.boxfill.Gfb.datawc_x1]{\sphinxcrossref{\sphinxcode{datawc\_x1}}}} property of the object.

\item {} 
\textbf{\texttt{dsp4}} (\href{https://docs.python.org/2/library/functions.html\#float}{\emph{\texttt{float}}}) -- Sets the {\hyperref[vcs/graphics/boxfill:vcs.boxfill.Gfb.datawc_x2]{\sphinxcrossref{\sphinxcode{datawc\_x2}}}} property of the object.

\end{itemize}

\end{description}\end{quote}

\end{fulllineitems}

\index{exts() (vcs.boxfill.Gfb method)}

\begin{fulllineitems}
\phantomsection\label{vcs/graphics/boxfill:vcs.boxfill.Gfb.exts}\pysiglinewithargsret{\sphinxbfcode{exts}}{\emph{ext1='n'}, \emph{ext2='y'}}{}
Sets the ext\_1 and ext\_2 values on the object.
\begin{quote}\begin{description}
\item[{Parameters}] \leavevmode\begin{itemize}
\item {} 
\textbf{\texttt{ext1}} (\href{https://docs.python.org/2/library/functions.html\#str}{\emph{\texttt{str}}}) -- Sets the {\hyperref[vcs/graphics/boxfill:vcs.boxfill.Gfb.ext_1]{\sphinxcrossref{\sphinxcode{ext\_1}}}} value on the object. `y' sets it to True, `n' sets it to False.

\item {} 
\textbf{\texttt{ext2}} (\href{https://docs.python.org/2/library/functions.html\#str}{\emph{\texttt{str}}}) -- Sets the {\hyperref[vcs/graphics/boxfill:vcs.boxfill.Gfb.ext_2]{\sphinxcrossref{\sphinxcode{ext\_2}}}} value on the object. `y' sets it to True, `n' sets it to False.

\end{itemize}

\end{description}\end{quote}

\end{fulllineitems}

\index{list() (vcs.boxfill.Gfb method)}

\begin{fulllineitems}
\phantomsection\label{vcs/graphics/boxfill:vcs.boxfill.Gfb.list}\pysiglinewithargsret{\sphinxbfcode{list}}{}{}
Lists the current values of object attributes

\end{fulllineitems}

\index{rename() (vcs.boxfill.Gfb method)}

\begin{fulllineitems}
\phantomsection\label{vcs/graphics/boxfill:vcs.boxfill.Gfb.rename}\pysiglinewithargsret{\sphinxbfcode{rename}}{\emph{newname}}{}
Renames the boxfill in the VCS name table.

\begin{notice}{note}{Note:}
This function will not rename the `default' boxfill.
If rename is called on the `default' boxfill, newname is associated with default in the VCS name table,
but the boxfill's name will not be changed, and will behave in all ways as a `default' boxfill.
\end{notice}
\begin{quote}\begin{description}
\item[{Example}] \leavevmode
\begin{Verbatim}[commandchars=\\\{\}]
\PYG{g+gp}{\PYGZgt{}\PYGZgt{}\PYGZgt{} }\PYG{n}{b}\PYG{o}{=}\PYG{n}{vcs}\PYG{o}{.}\PYG{n}{createboxfill}\PYG{p}{(}\PYG{p}{)}
\PYG{g+gp}{\PYGZgt{}\PYGZgt{}\PYGZgt{} }\PYG{n}{b}\PYG{o}{.}\PYG{n}{name}
\PYG{g+go}{\PYGZsq{}...\PYGZsq{}}
\PYG{g+gp}{\PYGZgt{}\PYGZgt{}\PYGZgt{} }\PYG{n}{vcs}\PYG{o}{.}\PYG{n}{listelements}\PYG{p}{(}\PYG{l+s+s1}{\PYGZsq{}}\PYG{l+s+s1}{boxfill}\PYG{l+s+s1}{\PYGZsq{}}\PYG{p}{)} \PYG{c+c1}{\PYGZsh{} list will include the name show above}
\PYG{g+go}{[...]}
\PYG{g+gp}{\PYGZgt{}\PYGZgt{}\PYGZgt{} }\PYG{n}{b}\PYG{o}{.}\PYG{n}{rename}\PYG{p}{(}\PYG{l+s+s1}{\PYGZsq{}}\PYG{l+s+s1}{foo}\PYG{l+s+s1}{\PYGZsq{}}\PYG{p}{)}
\PYG{g+gp}{\PYGZgt{}\PYGZgt{}\PYGZgt{} }\PYG{n}{b}\PYG{o}{.}\PYG{n}{name}
\PYG{g+go}{\PYGZsq{}foo\PYGZsq{}}
\PYG{g+gp}{\PYGZgt{}\PYGZgt{}\PYGZgt{} }\PYG{n}{vcs}\PYG{o}{.}\PYG{n}{listelements}\PYG{p}{(}\PYG{l+s+s1}{\PYGZsq{}}\PYG{l+s+s1}{boxfill}\PYG{l+s+s1}{\PYGZsq{}}\PYG{p}{)} \PYG{c+c1}{\PYGZsh{} list will include \PYGZsq{}foo\PYGZsq{}, but not the old name}
\PYG{g+go}{[...\PYGZsq{}foo\PYGZsq{}...]}
\end{Verbatim}

\item[{Parameters}] \leavevmode
\textbf{\texttt{newname}} -- The new name you want given to the boxfill

\end{description}\end{quote}

\end{fulllineitems}

\index{script() (vcs.boxfill.Gfb method)}

\begin{fulllineitems}
\phantomsection\label{vcs/graphics/boxfill:vcs.boxfill.Gfb.script}\pysiglinewithargsret{\sphinxbfcode{script}}{\emph{script\_filename}, \emph{mode='a'}}{}
Saves out a copy of the boxfill graphics method in JSON, or Python format to a designated file.
\begin{quote}

\begin{notice}{note}{Note:}
If the the filename has a `.py' at the end, it will produce a
Python script. If no extension is given, then by default a
.json file containing a JSON serialization of the object's
data will be produced.
\end{notice}

\begin{notice}{warning}{Warning:}
VCS Scripts Deprecated.
SCR script files are no longer generated by this function.
\end{notice}
\end{quote}
\begin{quote}\begin{description}
\item[{Example}] \leavevmode
\begin{Verbatim}[commandchars=\\\{\}]
\PYG{g+gp}{\PYGZgt{}\PYGZgt{}\PYGZgt{} }\PYG{n}{a}\PYG{o}{=}\PYG{n}{vcs}\PYG{o}{.}\PYG{n}{init}\PYG{p}{(}\PYG{p}{)} \PYG{c+c1}{\PYGZsh{} Make a Canvas object to work with}
\PYG{g+gp}{\PYGZgt{}\PYGZgt{}\PYGZgt{} }\PYG{n}{ex}\PYG{o}{=}\PYG{n}{a}\PYG{o}{.}\PYG{n}{getboxfill}\PYG{p}{(}\PYG{p}{)} \PYG{c+c1}{\PYGZsh{} Get default boxfill}
\PYG{g+gp}{\PYGZgt{}\PYGZgt{}\PYGZgt{} }\PYG{n}{ex}\PYG{o}{.}\PYG{n}{script}\PYG{p}{(}\PYG{l+s+s1}{\PYGZsq{}}\PYG{l+s+s1}{filename.py}\PYG{l+s+s1}{\PYGZsq{}}\PYG{p}{)} \PYG{c+c1}{\PYGZsh{} Append to a Python script named \PYGZsq{}filename.py\PYGZsq{}}
\PYG{g+gp}{\PYGZgt{}\PYGZgt{}\PYGZgt{} }\PYG{n}{ex}\PYG{o}{.}\PYG{n}{script}\PYG{p}{(}\PYG{l+s+s1}{\PYGZsq{}}\PYG{l+s+s1}{filename}\PYG{l+s+s1}{\PYGZsq{}}\PYG{p}{,}\PYG{l+s+s1}{\PYGZsq{}}\PYG{l+s+s1}{w}\PYG{l+s+s1}{\PYGZsq{}}\PYG{p}{)} \PYG{c+c1}{\PYGZsh{} Create or overwrite a JSON file \PYGZsq{}filename.json\PYGZsq{}.}
\end{Verbatim}

\item[{Parameters}] \leavevmode\begin{itemize}
\item {} 
\textbf{\texttt{script\_filename}} (\href{https://docs.python.org/2/library/functions.html\#str}{\emph{\texttt{str}}}) -- Output name of the script file. If no extension is specified, a .json object is created.

\item {} 
\textbf{\texttt{mode}} (\href{https://docs.python.org/2/library/functions.html\#str}{\emph{\texttt{str}}}) -- Either `w' for replace, or `a' for append. Defaults to `a', if not specified.

\end{itemize}

\end{description}\end{quote}

\end{fulllineitems}

\index{xmtics() (vcs.boxfill.Gfb method)}

\begin{fulllineitems}
\phantomsection\label{vcs/graphics/boxfill:vcs.boxfill.Gfb.xmtics}\pysiglinewithargsret{\sphinxbfcode{xmtics}}{\emph{xmt1='`}, \emph{xmt2='`}}{}
Sets the xmtics1 and xmtics2 values on the object
\begin{quote}\begin{description}
\item[{Parameters}] \leavevmode\begin{itemize}
\item {} 
\textbf{\texttt{xmt1}} (\emph{\texttt{\{float:str\} or str}}) -- Value for {\hyperref[vcs/graphics/boxfill:vcs.boxfill.Gfb.xmtics1]{\sphinxcrossref{\sphinxcode{xmtics1}}}}. Must be a str, or a dictionary object with float:str mappings.

\item {} 
\textbf{\texttt{xmt2}} (\emph{\texttt{\{float:str\} or str}}) -- Value for {\hyperref[vcs/graphics/boxfill:vcs.boxfill.Gfb.xmtics2]{\sphinxcrossref{\sphinxcode{xmtics2}}}}. Must be a str, or a dictionary object with float:str mappings.

\end{itemize}

\end{description}\end{quote}

\end{fulllineitems}

\index{xticlabels() (vcs.boxfill.Gfb method)}

\begin{fulllineitems}
\phantomsection\label{vcs/graphics/boxfill:vcs.boxfill.Gfb.xticlabels}\pysiglinewithargsret{\sphinxbfcode{xticlabels}}{\emph{xtl1='`}, \emph{xtl2='`}}{}
Sets the xticlabels1 and xticlabels2 values on the object
\begin{quote}\begin{description}
\item[{Parameters}] \leavevmode\begin{itemize}
\item {} 
\textbf{\texttt{xtl1}} (\emph{\texttt{\{float:str\} or str}}) -- Sets the object's value for {\hyperref[vcs/graphics/boxfill:vcs.boxfill.Gfb.xticlabels1]{\sphinxcrossref{\sphinxcode{xticlabels1}}}}. Must be  a str, or a dictionary object with float:str mappings.

\item {} 
\textbf{\texttt{xtl2}} (\emph{\texttt{\{float:str\} or str}}) -- Sets the object's value for {\hyperref[vcs/graphics/boxfill:vcs.boxfill.Gfb.xticlabels2]{\sphinxcrossref{\sphinxcode{xticlabels2}}}}. Must be a str, or a dictionary object with float:str mappings.

\end{itemize}

\end{description}\end{quote}

\end{fulllineitems}

\index{xyscale() (vcs.boxfill.Gfb method)}

\begin{fulllineitems}
\phantomsection\label{vcs/graphics/boxfill:vcs.boxfill.Gfb.xyscale}\pysiglinewithargsret{\sphinxbfcode{xyscale}}{\emph{xat='linear'}, \emph{yat='linear'}}{}
Sets xaxisconvert and yaxisconvert values for the object.
\begin{quote}\begin{description}
\item[{Example}] \leavevmode
\begin{Verbatim}[commandchars=\\\{\}]
\PYG{g+gp}{\PYGZgt{}\PYGZgt{}\PYGZgt{} }\PYG{n}{a}\PYG{o}{=}\PYG{n}{vcs}\PYG{o}{.}\PYG{n}{init}\PYG{p}{(}\PYG{p}{)}
\PYG{g+gp}{\PYGZgt{}\PYGZgt{}\PYGZgt{} }\PYG{n}{ex}\PYG{o}{=}\PYG{n}{a}\PYG{o}{.}\PYG{n}{createboxfill}\PYG{p}{(}\PYG{l+s+s1}{\PYGZsq{}}\PYG{l+s+s1}{xyscale\PYGZus{}ex}\PYG{l+s+s1}{\PYGZsq{}}\PYG{p}{)} \PYG{c+c1}{\PYGZsh{} create a boxfill to work with}
\PYG{g+gp}{\PYGZgt{}\PYGZgt{}\PYGZgt{} }\PYG{n}{ex}\PYG{o}{.}\PYG{n}{xyscale}\PYG{p}{(}\PYG{n}{xat}\PYG{o}{=}\PYG{l+s+s1}{\PYGZsq{}}\PYG{l+s+s1}{linear}\PYG{l+s+s1}{\PYGZsq{}}\PYG{p}{,} \PYG{n}{yat}\PYG{o}{=}\PYG{l+s+s1}{\PYGZsq{}}\PYG{l+s+s1}{linear}\PYG{l+s+s1}{\PYGZsq{}}\PYG{p}{)} \PYG{c+c1}{\PYGZsh{} set xaxisconvert and yaxisconvert to \PYGZsq{}linear\PYGZsq{}}
\end{Verbatim}

\item[{Parameters}] \leavevmode\begin{itemize}
\item {} 
\textbf{\texttt{xat}} (\href{https://docs.python.org/2/library/functions.html\#str}{\emph{\texttt{str}}}) -- Set value for x axis conversion.

\item {} 
\textbf{\texttt{yat}} (\href{https://docs.python.org/2/library/functions.html\#str}{\emph{\texttt{str}}}) -- Set value for y axis conversion.

\end{itemize}

\end{description}\end{quote}

\end{fulllineitems}

\index{ymtics() (vcs.boxfill.Gfb method)}

\begin{fulllineitems}
\phantomsection\label{vcs/graphics/boxfill:vcs.boxfill.Gfb.ymtics}\pysiglinewithargsret{\sphinxbfcode{ymtics}}{\emph{ymt1='`}, \emph{ymt2='`}}{}
Sets the ymtics1 and ymtics2 values on the object
\begin{quote}\begin{description}
\item[{Parameters}] \leavevmode\begin{itemize}
\item {} 
\textbf{\texttt{ymt1}} (\emph{\texttt{\{float:str\} or str}}) -- Value for {\hyperref[vcs/graphics/boxfill:vcs.boxfill.Gfb.ymtics1]{\sphinxcrossref{\sphinxcode{ymtics1}}}}. Must be a str, or a dictionary object with float:str mappings.

\item {} 
\textbf{\texttt{ymt2}} (\emph{\texttt{\{float:str\} or str}}) -- Value for {\hyperref[vcs/graphics/boxfill:vcs.boxfill.Gfb.ymtics2]{\sphinxcrossref{\sphinxcode{ymtics2}}}}. Must be a str, or a dictionary object with float:str mappings.

\end{itemize}

\end{description}\end{quote}

\end{fulllineitems}

\index{yticlabels() (vcs.boxfill.Gfb method)}

\begin{fulllineitems}
\phantomsection\label{vcs/graphics/boxfill:vcs.boxfill.Gfb.yticlabels}\pysiglinewithargsret{\sphinxbfcode{yticlabels}}{\emph{ytl1='`}, \emph{ytl2='`}}{}
Sets the yticlabels1 and yticlabels2 values on the object
\begin{quote}\begin{description}
\item[{Parameters}] \leavevmode\begin{itemize}
\item {} 
\textbf{\texttt{ytl1}} (\emph{\texttt{\{float:str\} or str}}) -- Sets the object's value for {\hyperref[vcs/graphics/boxfill:vcs.boxfill.Gfb.yticlabels1]{\sphinxcrossref{\sphinxcode{yticlabels1}}}}. Must be  a str, or a dictionary object with float:str mappings.

\item {} 
\textbf{\texttt{ytl2}} (\emph{\texttt{\{float:str\} or str}}) -- Sets the object's value for {\hyperref[vcs/graphics/boxfill:vcs.boxfill.Gfb.yticlabels2]{\sphinxcrossref{\sphinxcode{yticlabels2}}}}. Must be a str, or a dictionary object with float:str mappings.

\end{itemize}

\end{description}\end{quote}

\end{fulllineitems}


\end{fulllineitems}



\subsection{dv3d}
\label{vcs/graphics/dv3d:dv3d}\label{vcs/graphics/dv3d::doc}\label{vcs/graphics/dv3d:module-vcs.dv3d}\index{vcs.dv3d (module)}
Created on Jun 18, 2014

@author: tpmaxwel
\index{Gfdv3d (class in vcs.dv3d)}

\begin{fulllineitems}
\phantomsection\label{vcs/graphics/dv3d:vcs.dv3d.Gfdv3d}\pysiglinewithargsret{\sphinxstrong{class }\sphinxcode{vcs.dv3d.}\sphinxbfcode{Gfdv3d}}{\emph{Gfdv3d\_name}, \emph{Gfdv3d\_name\_src='default'}}{}
\end{fulllineitems}



\subsection{isofill}
\label{vcs/graphics/isofill:module-vcs.isofill}\label{vcs/graphics/isofill::doc}\label{vcs/graphics/isofill:isofill}\index{vcs.isofill (module)}
\# Isofill (Gfi) module
\index{Gfi (class in vcs.isofill)}

\begin{fulllineitems}
\phantomsection\label{vcs/graphics/isofill:vcs.isofill.Gfi}\pysiglinewithargsret{\sphinxstrong{class }\sphinxcode{vcs.isofill.}\sphinxbfcode{Gfi}}{\emph{Gfi\_name}, \emph{Gfi\_name\_src='default'}}{}
The Isofill graphics method fills the area between selected isolevels
(levels of constant value) of a two-dimensional array with a
user-specified color. The example below shows how to display an isofill
plot on the VCS Canvas and how to create and remove isofill isolevels.

This class is used to define an isofill table entry used in VCS, or it
can be used to change some or all of the isofill attributes in an
existing isofill table entry.


\begin{fulllineitems}
\pysigline{\sphinxbfcode{Useful~Functions:}}~
\begin{Verbatim}[commandchars=\\\{\}]
\PYG{c+c1}{\PYGZsh{} VCS Canvas Constructor}
\PYG{n}{a}\PYG{o}{=}\PYG{n}{vcs}\PYG{o}{.}\PYG{n}{init}\PYG{p}{(}\PYG{p}{)}
\PYG{c+c1}{\PYGZsh{} Show predefined isofill graphics methods}
\PYG{n}{a}\PYG{o}{.}\PYG{n}{show}\PYG{p}{(}\PYG{l+s+s1}{\PYGZsq{}}\PYG{l+s+s1}{isofill}\PYG{l+s+s1}{\PYGZsq{}}\PYG{p}{)}
\PYG{c+c1}{\PYGZsh{} Show predefined fillarea objects}
\PYG{n}{a}\PYG{o}{.}\PYG{n}{show}\PYG{p}{(}\PYG{l+s+s1}{\PYGZsq{}}\PYG{l+s+s1}{fillarea}\PYG{l+s+s1}{\PYGZsq{}}\PYG{p}{)}
\PYG{c+c1}{\PYGZsh{} Show predefined template objects}
\PYG{n}{a}\PYG{o}{.}\PYG{n}{show}\PYG{p}{(}\PYG{l+s+s1}{\PYGZsq{}}\PYG{l+s+s1}{template}\PYG{l+s+s1}{\PYGZsq{}}\PYG{p}{)}
\PYG{c+c1}{\PYGZsh{} Change the VCS color map}
\PYG{n}{a}\PYG{o}{.}\PYG{n}{setcolormap}\PYG{p}{(}\PYG{l+s+s2}{\PYGZdq{}}\PYG{l+s+s2}{AMIP}\PYG{l+s+s2}{\PYGZdq{}}\PYG{p}{)}
\PYG{c+c1}{\PYGZsh{} Create a template}
\PYG{n}{a}\PYG{o}{.}\PYG{n}{createtemplate}\PYG{p}{(}\PYG{l+s+s1}{\PYGZsq{}}\PYG{l+s+s1}{test}\PYG{l+s+s1}{\PYGZsq{}}\PYG{p}{)}
\PYG{c+c1}{\PYGZsh{} Create a fillarea}
\PYG{n}{a}\PYG{o}{.}\PYG{n}{createfillarea}\PYG{p}{(}\PYG{l+s+s1}{\PYGZsq{}}\PYG{l+s+s1}{fill}\PYG{l+s+s1}{\PYGZsq{}}\PYG{p}{)}
\PYG{c+c1}{\PYGZsh{} Get an existing template}
\PYG{n}{a}\PYG{o}{.}\PYG{n}{gettemplate}\PYG{p}{(}\PYG{l+s+s1}{\PYGZsq{}}\PYG{l+s+s1}{AMIP}\PYG{l+s+s1}{\PYGZsq{}}\PYG{p}{)}
\PYG{c+c1}{\PYGZsh{} Get an existing fillarea}
\PYG{n}{a}\PYG{o}{.}\PYG{n}{getfillarea}\PYG{p}{(}\PYG{l+s+s1}{\PYGZsq{}}\PYG{l+s+s1}{def37}\PYG{l+s+s1}{\PYGZsq{}}\PYG{p}{)}
\PYG{c+c1}{\PYGZsh{} Plot array \PYGZsq{}s\PYGZsq{} with isofill \PYGZsq{}i\PYGZsq{} and template \PYGZsq{}t\PYGZsq{}}
\PYG{n}{a}\PYG{o}{.}\PYG{n}{isofill}\PYG{p}{(}\PYG{n}{s}\PYG{p}{,}\PYG{n}{i}\PYG{p}{,}\PYG{n}{t}\PYG{p}{)}
\PYG{c+c1}{\PYGZsh{} Updates the VCS Canvas at user\PYGZsq{}s request}
\PYG{n}{a}\PYG{o}{.}\PYG{n}{update}\PYG{p}{(}\PYG{p}{)}
\end{Verbatim}

\end{fulllineitems}



\begin{fulllineitems}
\pysigline{\sphinxbfcode{Creating~an~isofill~object:}}~
\begin{Verbatim}[commandchars=\\\{\}]
\PYG{c+c1}{\PYGZsh{}Create a VCS Canvas}
\PYG{n}{a}\PYG{o}{=}\PYG{n}{vcs}\PYG{o}{.}\PYG{n}{init}\PYG{p}{(}\PYG{p}{)}
\PYG{c+c1}{\PYGZsh{}Create a new instance of isofill:}
\PYG{c+c1}{\PYGZsh{} Copies content of \PYGZsq{}quick\PYGZsq{} to \PYGZsq{}new\PYGZsq{}}
\PYG{n}{iso}\PYG{o}{=}\PYG{n}{a}\PYG{o}{.}\PYG{n}{createisofill}\PYG{p}{(}\PYG{l+s+s1}{\PYGZsq{}}\PYG{l+s+s1}{new}\PYG{l+s+s1}{\PYGZsq{}}\PYG{p}{,}\PYG{l+s+s1}{\PYGZsq{}}\PYG{l+s+s1}{quick}\PYG{l+s+s1}{\PYGZsq{}}\PYG{p}{)}
\PYG{c+c1}{\PYGZsh{} Copies content of \PYGZsq{}default\PYGZsq{} to \PYGZsq{}new\PYGZsq{}}
\PYG{n}{iso}\PYG{o}{=}\PYG{n}{a}\PYG{o}{.}\PYG{n}{createisofill}\PYG{p}{(}\PYG{l+s+s1}{\PYGZsq{}}\PYG{l+s+s1}{new}\PYG{l+s+s1}{\PYGZsq{}}\PYG{p}{)}
\end{Verbatim}

\end{fulllineitems}



\begin{fulllineitems}
\pysigline{\sphinxbfcode{Modifying~an~existing~isofill:}}~
\begin{Verbatim}[commandchars=\\\{\}]
\PYG{n}{iso}\PYG{o}{=}\PYG{n}{a}\PYG{o}{.}\PYG{n}{getisofill}\PYG{p}{(}\PYG{l+s+s1}{\PYGZsq{}}\PYG{l+s+s1}{AMIP\PYGZus{}psl}\PYG{l+s+s1}{\PYGZsq{}}\PYG{p}{)}
\end{Verbatim}

\end{fulllineitems}



\begin{fulllineitems}
\pysigline{\sphinxbfcode{Overview~of~isofill~attributes:}}~\begin{itemize}
\item {} 
List all isofill attribute values:
\begin{quote}

\begin{Verbatim}[commandchars=\\\{\}]
\PYG{n}{iso}\PYG{o}{.}\PYG{n}{list}\PYG{p}{(}\PYG{p}{)}
\end{Verbatim}
\end{quote}

\item {} 
Set isofill attributes:
\begin{quote}

\begin{Verbatim}[commandchars=\\\{\}]
\PYG{n}{iso}\PYG{o}{.}\PYG{n}{projection}\PYG{o}{=}\PYG{l+s+s1}{\PYGZsq{}}\PYG{l+s+s1}{linear}\PYG{l+s+s1}{\PYGZsq{}}
\PYG{n}{lon30}\PYG{o}{=}\PYG{p}{\PYGZob{}}\PYG{o}{\PYGZhy{}}\PYG{l+m+mi}{180}\PYG{p}{:}\PYG{l+s+s1}{\PYGZsq{}}\PYG{l+s+s1}{180W}\PYG{l+s+s1}{\PYGZsq{}}\PYG{p}{,}\PYG{o}{\PYGZhy{}}\PYG{l+m+mi}{150}\PYG{p}{:}\PYG{l+s+s1}{\PYGZsq{}}\PYG{l+s+s1}{150W}\PYG{l+s+s1}{\PYGZsq{}}\PYG{p}{,}\PYG{l+m+mi}{0}\PYG{p}{:}\PYG{l+s+s1}{\PYGZsq{}}\PYG{l+s+s1}{Eq}\PYG{l+s+s1}{\PYGZsq{}}\PYG{p}{\PYGZcb{}}
\PYG{n}{iso}\PYG{o}{.}\PYG{n}{xticlabels1}\PYG{o}{=}\PYG{n}{lon30}
\PYG{n}{iso}\PYG{o}{.}\PYG{n}{xticlabels2}\PYG{o}{=}\PYG{n}{lon30}
\PYG{c+c1}{\PYGZsh{} Will set them both}
\PYG{n}{iso}\PYG{o}{.}\PYG{n}{xticlabels}\PYG{p}{(}\PYG{n}{lon30}\PYG{p}{,} \PYG{n}{lon30}\PYG{p}{)}
\PYG{n}{iso}\PYG{o}{.}\PYG{n}{xmtics1}\PYG{o}{=}\PYG{l+s+s1}{\PYGZsq{}}\PYG{l+s+s1}{\PYGZsq{}}
\PYG{n}{iso}\PYG{o}{.}\PYG{n}{xmtics2}\PYG{o}{=}\PYG{l+s+s1}{\PYGZsq{}}\PYG{l+s+s1}{\PYGZsq{}}
\PYG{c+c1}{\PYGZsh{} Will set them both}
\PYG{n}{iso}\PYG{o}{.}\PYG{n}{xmtics}\PYG{p}{(}\PYG{n}{lon30}\PYG{p}{,} \PYG{n}{lon30}\PYG{p}{)}
\PYG{n}{iso}\PYG{o}{.}\PYG{n}{yticlabels1}\PYG{o}{=}\PYG{n}{lat10}
\PYG{n}{iso}\PYG{o}{.}\PYG{n}{yticlabels2}\PYG{o}{=}\PYG{n}{lat10}
\PYG{c+c1}{\PYGZsh{} Will set them both}
\PYG{n}{iso}\PYG{o}{.}\PYG{n}{yticlabels}\PYG{p}{(}\PYG{n}{lat10}\PYG{p}{,} \PYG{n}{lat10}\PYG{p}{)}
\PYG{n}{iso}\PYG{o}{.}\PYG{n}{ymtics1}\PYG{o}{=}\PYG{l+s+s1}{\PYGZsq{}}\PYG{l+s+s1}{\PYGZsq{}}
\PYG{n}{iso}\PYG{o}{.}\PYG{n}{ymtics2}\PYG{o}{=}\PYG{l+s+s1}{\PYGZsq{}}\PYG{l+s+s1}{\PYGZsq{}}
\PYG{c+c1}{\PYGZsh{} Will set them both}
\PYG{n}{iso}\PYG{o}{.}\PYG{n}{ymtics}\PYG{p}{(}\PYG{n}{lat10}\PYG{p}{,} \PYG{n}{lat10}\PYG{p}{)}
\PYG{n}{iso}\PYG{o}{.}\PYG{n}{datawc\PYGZus{}y1}\PYG{o}{=}\PYG{o}{\PYGZhy{}}\PYG{l+m+mf}{90.0}
\PYG{n}{iso}\PYG{o}{.}\PYG{n}{datawc\PYGZus{}y2}\PYG{o}{=}\PYG{l+m+mf}{90.0}
\PYG{n}{iso}\PYG{o}{.}\PYG{n}{datawc\PYGZus{}x1}\PYG{o}{=}\PYG{o}{\PYGZhy{}}\PYG{l+m+mf}{180.0}
\PYG{n}{iso}\PYG{o}{.}\PYG{n}{datawc\PYGZus{}x2}\PYG{o}{=}\PYG{l+m+mf}{180.0}
\PYG{c+c1}{\PYGZsh{} Will set them all}
\PYG{n}{iso}\PYG{o}{.}\PYG{n}{datawc}\PYG{p}{(}\PYG{o}{\PYGZhy{}}\PYG{l+m+mi}{90}\PYG{p}{,} \PYG{l+m+mi}{90}\PYG{p}{,} \PYG{o}{\PYGZhy{}}\PYG{l+m+mi}{180}\PYG{p}{,} \PYG{l+m+mi}{180}\PYG{p}{)}
\PYG{n}{iso}\PYG{o}{.}\PYG{n}{xaxisconvert}\PYG{o}{=}\PYG{l+s+s1}{\PYGZsq{}}\PYG{l+s+s1}{linear}\PYG{l+s+s1}{\PYGZsq{}}
\PYG{n}{iso}\PYG{o}{.}\PYG{n}{yaxisconvert}\PYG{o}{=}\PYG{l+s+s1}{\PYGZsq{}}\PYG{l+s+s1}{linear}\PYG{l+s+s1}{\PYGZsq{}}
\PYG{c+c1}{\PYGZsh{} Will set them both}
\PYG{n}{iso}\PYG{o}{.}\PYG{n}{xyscale}\PYG{p}{(}\PYG{l+s+s1}{\PYGZsq{}}\PYG{l+s+s1}{linear}\PYG{l+s+s1}{\PYGZsq{}}\PYG{p}{,} \PYG{l+s+s1}{\PYGZsq{}}\PYG{l+s+s1}{area\PYGZus{}wt}\PYG{l+s+s1}{\PYGZsq{}}\PYG{p}{)}
\PYG{c+c1}{\PYGZsh{} Color index value range 0 to 255}
\PYG{n}{iso}\PYG{o}{.}\PYG{n}{missing}\PYG{o}{=}\PYG{l+m+mi}{241}
\PYG{n}{iso}\PYG{o}{.}\PYG{n}{legend}\PYG{o}{=}\PYG{n+nb+bp}{None}
\PYG{n}{ext\PYGZus{}1}\PYG{o}{=}\PYG{l+s+s1}{\PYGZsq{}}\PYG{l+s+s1}{n}\PYG{l+s+s1}{\PYGZsq{}}
\PYG{n}{ext\PYGZus{}2}\PYG{o}{=}\PYG{l+s+s1}{\PYGZsq{}}\PYG{l+s+s1}{y}\PYG{l+s+s1}{\PYGZsq{}}
\PYG{c+c1}{\PYGZsh{} Will set them both}
\PYG{n}{iso}\PYG{o}{.}\PYG{n}{exts}\PYG{p}{(}\PYG{l+s+s1}{\PYGZsq{}}\PYG{l+s+s1}{n}\PYG{l+s+s1}{\PYGZsq{}}\PYG{p}{,} \PYG{l+s+s1}{\PYGZsq{}}\PYG{l+s+s1}{y}\PYG{l+s+s1}{\PYGZsq{}} \PYG{p}{)}
\end{Verbatim}
\end{quote}

\item {} 
Setting the isofill levels:
\begin{quote}

\begin{Verbatim}[commandchars=\\\{\}]
\PYG{c+c1}{\PYGZsh{} 1) When levels are all contiguous:}
    \PYG{n}{iso}\PYG{o}{.}\PYG{n}{levels}\PYG{o}{=}\PYG{p}{(}\PYG{p}{[}\PYG{l+m+mi}{0}\PYG{p}{,}\PYG{l+m+mi}{20}\PYG{p}{,}\PYG{l+m+mi}{25}\PYG{p}{,}\PYG{l+m+mi}{30}\PYG{p}{,}\PYG{l+m+mi}{35}\PYG{p}{,}\PYG{l+m+mi}{40}\PYG{p}{]}\PYG{p}{,}\PYG{p}{)}
    \PYG{n}{iso}\PYG{o}{.}\PYG{n}{levels}\PYG{o}{=}\PYG{p}{(}\PYG{p}{[}\PYG{l+m+mi}{0}\PYG{p}{,}\PYG{l+m+mi}{20}\PYG{p}{,}\PYG{l+m+mi}{25}\PYG{p}{,}\PYG{l+m+mi}{30}\PYG{p}{,}\PYG{l+m+mi}{35}\PYG{p}{,}\PYG{l+m+mi}{40}\PYG{p}{,}\PYG{l+m+mi}{45}\PYG{p}{,}\PYG{l+m+mi}{50}\PYG{p}{]}\PYG{p}{)}
    \PYG{n}{iso}\PYG{o}{.}\PYG{n}{levels}\PYG{o}{=}\PYG{p}{[}\PYG{l+m+mi}{0}\PYG{p}{,}\PYG{l+m+mi}{20}\PYG{p}{,}\PYG{l+m+mi}{25}\PYG{p}{,}\PYG{l+m+mi}{30}\PYG{p}{,}\PYG{l+m+mi}{35}\PYG{p}{,}\PYG{l+m+mi}{40}\PYG{p}{]}
    \PYG{n}{iso}\PYG{o}{.}\PYG{n}{levels}\PYG{o}{=}\PYG{p}{(}\PYG{l+m+mf}{0.0}\PYG{p}{,}\PYG{l+m+mf}{20.0}\PYG{p}{,}\PYG{l+m+mf}{25.0}\PYG{p}{,}\PYG{l+m+mf}{30.0}\PYG{p}{,}\PYG{l+m+mf}{35.0}\PYG{p}{,}\PYG{l+m+mf}{40.0}\PYG{p}{,}\PYG{l+m+mf}{50.0}\PYG{p}{)}

\PYG{c+c1}{\PYGZsh{} 2) When levels are not contiguous:}
    \PYG{n}{iso}\PYG{o}{.}\PYG{n}{levels}\PYG{o}{=}\PYG{p}{(}\PYG{p}{[}\PYG{l+m+mi}{0}\PYG{p}{,}\PYG{l+m+mi}{20}\PYG{p}{]}\PYG{p}{,}\PYG{p}{[}\PYG{l+m+mi}{30}\PYG{p}{,}\PYG{l+m+mi}{40}\PYG{p}{]}\PYG{p}{,}\PYG{p}{[}\PYG{l+m+mi}{50}\PYG{p}{,}\PYG{l+m+mi}{60}\PYG{p}{]}\PYG{p}{)}
    \PYG{n}{iso}\PYG{o}{.}\PYG{n}{levels}\PYG{o}{=}\PYG{p}{(}\PYG{p}{[}\PYG{l+m+mi}{0}\PYG{p}{,}\PYG{l+m+mi}{20}\PYG{p}{,}\PYG{l+m+mi}{25}\PYG{p}{,}\PYG{l+m+mi}{30}\PYG{p}{,}\PYG{l+m+mi}{35}\PYG{p}{,}\PYG{l+m+mi}{40}\PYG{p}{]}\PYG{p}{,}\PYG{p}{[}\PYG{l+m+mi}{30}\PYG{p}{,}\PYG{l+m+mi}{40}\PYG{p}{]}\PYG{p}{,}\PYG{p}{[}\PYG{l+m+mi}{50}\PYG{p}{,}\PYG{l+m+mi}{60}\PYG{p}{]}\PYG{p}{)}
\end{Verbatim}
\end{quote}

\item {} 
Setting the fillarea color indices:
\begin{quote}

\begin{Verbatim}[commandchars=\\\{\}]
\PYG{n}{iso}\PYG{o}{.}\PYG{n}{fillareacolors}\PYG{o}{=}\PYG{p}{(}\PYG{p}{[}\PYG{l+m+mi}{22}\PYG{p}{,}\PYG{l+m+mi}{33}\PYG{p}{,}\PYG{l+m+mi}{44}\PYG{p}{,}\PYG{l+m+mi}{55}\PYG{p}{,}\PYG{l+m+mi}{66}\PYG{p}{,}\PYG{l+m+mi}{77}\PYG{p}{]}\PYG{p}{)}
\PYG{n}{iso}\PYG{o}{.}\PYG{n}{fillareacolors}\PYG{o}{=}\PYG{p}{(}\PYG{l+m+mi}{16}\PYG{p}{,}\PYG{l+m+mi}{19}\PYG{p}{,}\PYG{l+m+mi}{33}\PYG{p}{,}\PYG{l+m+mi}{44}\PYG{p}{)}
\PYG{n}{iso}\PYG{o}{.}\PYG{n}{fillareacolors}\PYG{o}{=}\PYG{n+nb+bp}{None}
\end{Verbatim}
\end{quote}

\item {} 
Setting the fillarea style:
\begin{quote}

\begin{Verbatim}[commandchars=\\\{\}]
\PYG{n}{iso}\PYG{o}{.}\PYG{n}{fillareastyle} \PYG{o}{=} \PYG{l+s+s1}{\PYGZsq{}}\PYG{l+s+s1}{solid}\PYG{l+s+s1}{\PYGZsq{}}
\PYG{n}{iso}\PYG{o}{.}\PYG{n}{fillareastyle} \PYG{o}{=} \PYG{l+s+s1}{\PYGZsq{}}\PYG{l+s+s1}{hatch}\PYG{l+s+s1}{\PYGZsq{}}
\PYG{n}{iso}\PYG{o}{.}\PYG{n}{fillareastyle} \PYG{o}{=} \PYG{l+s+s1}{\PYGZsq{}}\PYG{l+s+s1}{pattern}\PYG{l+s+s1}{\PYGZsq{}}
\end{Verbatim}
\end{quote}

\item {} 
Setting the fillarea hatch or pattern indices:
\begin{quote}

\begin{Verbatim}[commandchars=\\\{\}]
\PYG{n}{iso}\PYG{o}{.}\PYG{n}{fillareaindices}\PYG{o}{=}\PYG{p}{(}\PYG{p}{[}\PYG{l+m+mi}{1}\PYG{p}{,}\PYG{l+m+mi}{3}\PYG{p}{,}\PYG{l+m+mi}{5}\PYG{p}{,}\PYG{l+m+mi}{6}\PYG{p}{,}\PYG{l+m+mi}{9}\PYG{p}{,}\PYG{l+m+mi}{20}\PYG{p}{]}\PYG{p}{)}
\PYG{n}{iso}\PYG{o}{.}\PYG{n}{fillareaindices}\PYG{o}{=}\PYG{p}{(}\PYG{l+m+mi}{7}\PYG{p}{,}\PYG{l+m+mi}{1}\PYG{p}{,}\PYG{l+m+mi}{4}\PYG{p}{,}\PYG{l+m+mi}{9}\PYG{p}{,}\PYG{l+m+mi}{6}\PYG{p}{,}\PYG{l+m+mi}{15}\PYG{p}{)}
\end{Verbatim}
\end{quote}

\end{itemize}

\end{fulllineitems}



\begin{fulllineitems}
\pysigline{\sphinxbfcode{Using~the~fillarea~secondary~object~(Ex):}}~\begin{itemize}
\item {} 
Create a new instance of fillarea:
\begin{quote}

\begin{Verbatim}[commandchars=\\\{\}]
\PYG{n}{f}\PYG{o}{=}\PYG{n}{createfillarea}\PYG{p}{(}\PYG{l+s+s1}{\PYGZsq{}}\PYG{l+s+s1}{fill1}\PYG{l+s+s1}{\PYGZsq{}}\PYG{p}{)}
\end{Verbatim}
\end{quote}

\item {} 
Create a new isofill:
\begin{quote}

\begin{Verbatim}[commandchars=\\\{\}]
\PYG{c+c1}{\PYGZsh{} Copies \PYGZsq{}quick\PYGZsq{} to \PYGZsq{}new\PYGZsq{}}
\PYG{n}{fill}\PYG{o}{=}\PYG{n}{a}\PYG{o}{.}\PYG{n}{createisofill}\PYG{p}{(}\PYG{l+s+s1}{\PYGZsq{}}\PYG{l+s+s1}{new}\PYG{l+s+s1}{\PYGZsq{}}\PYG{p}{,}\PYG{l+s+s1}{\PYGZsq{}}\PYG{l+s+s1}{quick}\PYG{l+s+s1}{\PYGZsq{}}\PYG{p}{)}
\PYG{c+c1}{\PYGZsh{} Copies \PYGZsq{}default\PYGZsq{} to \PYGZsq{}new\PYGZsq{}}
\PYG{n}{fill}\PYG{o}{=}\PYG{n}{a}\PYG{o}{.}\PYG{n}{createisofill}\PYG{p}{(}\PYG{l+s+s1}{\PYGZsq{}}\PYG{l+s+s1}{new}\PYG{l+s+s1}{\PYGZsq{}}\PYG{p}{)}
\end{Verbatim}
\end{quote}

\item {} 
Modify an existing isofill:
\begin{quote}

\begin{Verbatim}[commandchars=\\\{\}]
\PYG{n}{fill}\PYG{o}{=}\PYG{n}{a}\PYG{o}{.}\PYG{n}{getisofill}\PYG{p}{(}\PYG{l+s+s1}{\PYGZsq{}}\PYG{l+s+s1}{def37}\PYG{l+s+s1}{\PYGZsq{}}\PYG{p}{)}
\end{Verbatim}
\end{quote}

\item {} 
Set index using fillarea
\begin{quote}

\begin{Verbatim}[commandchars=\\\{\}]
\PYG{n}{iso}\PYG{o}{.}\PYG{n}{fillareaindices}\PYG{o}{=}\PYG{p}{(}\PYG{l+m+mi}{7}\PYG{p}{,}\PYG{n}{fill}\PYG{p}{,}\PYG{l+m+mi}{4}\PYG{p}{,}\PYG{l+m+mi}{9}\PYG{p}{,}\PYG{n}{fill}\PYG{p}{,}\PYG{l+m+mi}{15}\PYG{p}{)}
\PYG{c+c1}{\PYGZsh{} list fillarea attributes}
\PYG{n}{fill}\PYG{o}{.}\PYG{n}{list}\PYG{p}{(}\PYG{p}{)}
\PYG{c+c1}{\PYGZsh{} change style}
\PYG{n}{fill}\PYG{o}{.}\PYG{n}{style}\PYG{o}{=}\PYG{l+s+s1}{\PYGZsq{}}\PYG{l+s+s1}{hatch}\PYG{l+s+s1}{\PYGZsq{}}
\PYG{c+c1}{\PYGZsh{} change color}
\PYG{n}{fill}\PYG{o}{.}\PYG{n}{color}\PYG{o}{=}\PYG{l+m+mi}{241}
\PYG{c+c1}{\PYGZsh{} change style index}
\PYG{n}{fill}\PYG{o}{.}\PYG{n}{index}\PYG{o}{=}\PYG{l+m+mi}{3}
\end{Verbatim}
\end{quote}

\end{itemize}

\end{fulllineitems}



\begin{fulllineitems}
\pysigline{\sphinxbfcode{Attribute~descriptions:}}~\index{xmtics1 (vcs.isofill.Gfi attribute)}

\begin{fulllineitems}
\phantomsection\label{vcs/graphics/isofill:vcs.isofill.Gfi.xmtics1}\pysiglinewithargsret{\sphinxbfcode{xmtics1}}{\emph{str/\{float:str\}}}{}
(Ex: `') dictionary with location of intermediate tics as keys for 1st side of y axis

\end{fulllineitems}

\index{xmtics2 (vcs.isofill.Gfi attribute)}

\begin{fulllineitems}
\phantomsection\label{vcs/graphics/isofill:vcs.isofill.Gfi.xmtics2}\pysiglinewithargsret{\sphinxbfcode{xmtics2}}{\emph{str/\{float:str\}}}{}
(Ex: `') dictionary with location of intermediate tics as keys for 2nd side of y axis

\end{fulllineitems}

\index{ymtics1 (vcs.isofill.Gfi attribute)}

\begin{fulllineitems}
\phantomsection\label{vcs/graphics/isofill:vcs.isofill.Gfi.ymtics1}\pysiglinewithargsret{\sphinxbfcode{ymtics1}}{\emph{str/\{float:str\}}}{}
(Ex: `') dictionary with location of intermediate tics as keys for 1st side of y axis

\end{fulllineitems}

\index{ymtics2 (vcs.isofill.Gfi attribute)}

\begin{fulllineitems}
\phantomsection\label{vcs/graphics/isofill:vcs.isofill.Gfi.ymtics2}\pysiglinewithargsret{\sphinxbfcode{ymtics2}}{\emph{str/\{float:str\}}}{}
(Ex: `') dictionary with location of intermediate tics as keys for 2nd side of y axis

\end{fulllineitems}

\index{xticlabels1 (vcs.isofill.Gfi attribute)}

\begin{fulllineitems}
\phantomsection\label{vcs/graphics/isofill:vcs.isofill.Gfi.xticlabels1}\pysiglinewithargsret{\sphinxbfcode{xticlabels1}}{\emph{str/\{float:str\}}}{}
(Ex: `*') values for labels on 1st side of x axis

\end{fulllineitems}

\index{xticlabels2 (vcs.isofill.Gfi attribute)}

\begin{fulllineitems}
\phantomsection\label{vcs/graphics/isofill:vcs.isofill.Gfi.xticlabels2}\pysiglinewithargsret{\sphinxbfcode{xticlabels2}}{\emph{str/\{float:str\}}}{}
(Ex: `*') values for labels on 2nd side of x axis

\end{fulllineitems}

\index{yticlabels1 (vcs.isofill.Gfi attribute)}

\begin{fulllineitems}
\phantomsection\label{vcs/graphics/isofill:vcs.isofill.Gfi.yticlabels1}\pysiglinewithargsret{\sphinxbfcode{yticlabels1}}{\emph{str/\{float:str\}}}{}
(Ex: `*') values for labels on 1st side of y axis

\end{fulllineitems}

\index{yticlabels2 (vcs.isofill.Gfi attribute)}

\begin{fulllineitems}
\phantomsection\label{vcs/graphics/isofill:vcs.isofill.Gfi.yticlabels2}\pysiglinewithargsret{\sphinxbfcode{yticlabels2}}{\emph{str/\{float:str\}}}{}
(Ex: `*') values for labels on 2nd side of y axis

\end{fulllineitems}

\index{projection (vcs.isofill.Gfi attribute)}

\begin{fulllineitems}
\phantomsection\label{vcs/graphics/isofill:vcs.isofill.Gfi.projection}\pysiglinewithargsret{\sphinxbfcode{projection}}{\emph{str/vcs.projection.Proj}}{}
(Ex: `default') projection to use, name or object

\end{fulllineitems}

\index{datawc\_x1 (vcs.isofill.Gfi attribute)}

\begin{fulllineitems}
\phantomsection\label{vcs/graphics/isofill:vcs.isofill.Gfi.datawc_x1}\pysiglinewithargsret{\sphinxbfcode{datawc\_x1}}{\emph{float}}{}
(Ex: 1.E20) first value of xaxis on plot

\end{fulllineitems}

\index{datawc\_x2 (vcs.isofill.Gfi attribute)}

\begin{fulllineitems}
\phantomsection\label{vcs/graphics/isofill:vcs.isofill.Gfi.datawc_x2}\pysiglinewithargsret{\sphinxbfcode{datawc\_x2}}{\emph{float}}{}
(Ex: 1.E20) second value of xaxis on plot

\end{fulllineitems}

\index{datawc\_y1 (vcs.isofill.Gfi attribute)}

\begin{fulllineitems}
\phantomsection\label{vcs/graphics/isofill:vcs.isofill.Gfi.datawc_y1}\pysiglinewithargsret{\sphinxbfcode{datawc\_y1}}{\emph{float}}{}
(Ex: 1.E20) first value of yaxis on plot

\end{fulllineitems}

\index{datawc\_y2 (vcs.isofill.Gfi attribute)}

\begin{fulllineitems}
\phantomsection\label{vcs/graphics/isofill:vcs.isofill.Gfi.datawc_y2}\pysiglinewithargsret{\sphinxbfcode{datawc\_y2}}{\emph{float}}{}
(Ex: 1.E20) second value of yaxis on plot

\end{fulllineitems}

\index{datawc\_timeunits (vcs.isofill.Gfi attribute)}

\begin{fulllineitems}
\phantomsection\label{vcs/graphics/isofill:vcs.isofill.Gfi.datawc_timeunits}\pysiglinewithargsret{\sphinxbfcode{datawc\_timeunits}}{\emph{str}}{}
(Ex: `days since 2000') units to use when displaying time dimension auto tick

\end{fulllineitems}

\index{datawc\_calendar (vcs.isofill.Gfi attribute)}

\begin{fulllineitems}
\phantomsection\label{vcs/graphics/isofill:vcs.isofill.Gfi.datawc_calendar}\pysiglinewithargsret{\sphinxbfcode{datawc\_calendar}}{\emph{int}}{}
(Ex: 135441) calendar to use when displaying time dimension auto tick, default is proleptic gregorian calendar

\end{fulllineitems}

\index{levels (vcs.isofill.Gfi attribute)}

\begin{fulllineitems}
\phantomsection\label{vcs/graphics/isofill:vcs.isofill.Gfi.levels}\pysiglinewithargsret{\sphinxbfcode{levels}}{\emph{{[}float,...{]}/{[}{[}float,float{]},...{]}}}{}
Sets the levels range to use, can be either a list of contiguous levels, or list of tuples
indicating first and last value of the range.

\end{fulllineitems}

\index{fillareacolors (vcs.isofill.Gfi attribute)}

\begin{fulllineitems}
\phantomsection\label{vcs/graphics/isofill:vcs.isofill.Gfi.fillareacolors}\pysiglinewithargsret{\sphinxbfcode{fillareacolors}}{\sphinxoptional{\emph{int}, \emph{...}}}{}
Colors to use for each level

\end{fulllineitems}

\index{fillareastyle (vcs.isofill.Gfi attribute)}

\begin{fulllineitems}
\phantomsection\label{vcs/graphics/isofill:vcs.isofill.Gfi.fillareastyle}\pysiglinewithargsret{\sphinxbfcode{fillareastyle}}{\emph{str}}{}
Style to use for levels filling: solid/pattern/hatch

\end{fulllineitems}

\index{fillareaindices (vcs.isofill.Gfi attribute)}

\begin{fulllineitems}
\phantomsection\label{vcs/graphics/isofill:vcs.isofill.Gfi.fillareaindices}\pysiglinewithargsret{\sphinxbfcode{fillareaindices}}{\sphinxoptional{\emph{int}, \emph{...}}}{}
List of patterns to use when filling a level and using pattern/hatch

\end{fulllineitems}

\index{legend (vcs.isofill.Gfi attribute)}

\begin{fulllineitems}
\phantomsection\label{vcs/graphics/isofill:vcs.isofill.Gfi.legend}\pysiglinewithargsret{\sphinxbfcode{legend}}{\emph{None/\{float:str\}}}{}
Replaces the legend values in the dictionary keys with their associated string

\end{fulllineitems}

\index{ext\_1 (vcs.isofill.Gfi attribute)}

\begin{fulllineitems}
\phantomsection\label{vcs/graphics/isofill:vcs.isofill.Gfi.ext_1}\pysiglinewithargsret{\sphinxbfcode{ext\_1}}{\emph{str}}{}
Draws an extension arrow on right side (values less than first range value)

\end{fulllineitems}

\index{ext\_2 (vcs.isofill.Gfi attribute)}

\begin{fulllineitems}
\phantomsection\label{vcs/graphics/isofill:vcs.isofill.Gfi.ext_2}\pysiglinewithargsret{\sphinxbfcode{ext\_2}}{\emph{str}}{}
Draws an extension arrow on left side (values greater than last range value)

\end{fulllineitems}

\index{missing (vcs.isofill.Gfi attribute)}

\begin{fulllineitems}
\phantomsection\label{vcs/graphics/isofill:vcs.isofill.Gfi.missing}\pysiglinewithargsret{\sphinxbfcode{missing}}{\emph{int}}{}
Color to use for missing value or values not in defined ranges

\end{fulllineitems}


\end{fulllineitems}

\index{colors() (vcs.isofill.Gfi method)}

\begin{fulllineitems}
\phantomsection\label{vcs/graphics/isofill:vcs.isofill.Gfi.colors}\pysiglinewithargsret{\sphinxbfcode{colors}}{\emph{color1=16}, \emph{color2=239}}{}
Sets the color\_1 and color\_2 properties of the object.
\begin{quote}\begin{description}
\item[{Parameters}] \leavevmode\begin{itemize}
\item {} 
\textbf{\texttt{color1}} (\href{https://docs.python.org/2/library/functions.html\#int}{\emph{\texttt{int}}}) -- Sets the \sphinxcode{color\_1} value on the object

\item {} 
\textbf{\texttt{color2}} (\href{https://docs.python.org/2/library/functions.html\#int}{\emph{\texttt{int}}}) -- Sets the \sphinxcode{color\_2} value on the object

\end{itemize}

\end{description}\end{quote}

\end{fulllineitems}

\index{datawc() (vcs.isofill.Gfi method)}

\begin{fulllineitems}
\phantomsection\label{vcs/graphics/isofill:vcs.isofill.Gfi.datawc}\pysiglinewithargsret{\sphinxbfcode{datawc}}{\emph{dsp1=1e+20}, \emph{dsp2=1e+20}, \emph{dsp3=1e+20}, \emph{dsp4=1e+20}}{}
Sets the data world coordinates for object
\begin{quote}\begin{description}
\item[{Parameters}] \leavevmode\begin{itemize}
\item {} 
\textbf{\texttt{dsp1}} (\href{https://docs.python.org/2/library/functions.html\#float}{\emph{\texttt{float}}}) -- Sets the {\hyperref[vcs/graphics/isofill:vcs.isofill.Gfi.datawc_y1]{\sphinxcrossref{\sphinxcode{datawc\_y1}}}} property of the object.

\item {} 
\textbf{\texttt{dsp2}} (\href{https://docs.python.org/2/library/functions.html\#float}{\emph{\texttt{float}}}) -- Sets the {\hyperref[vcs/graphics/isofill:vcs.isofill.Gfi.datawc_y2]{\sphinxcrossref{\sphinxcode{datawc\_y2}}}} property of the object.

\item {} 
\textbf{\texttt{dsp3}} (\href{https://docs.python.org/2/library/functions.html\#float}{\emph{\texttt{float}}}) -- Sets the {\hyperref[vcs/graphics/isofill:vcs.isofill.Gfi.datawc_x1]{\sphinxcrossref{\sphinxcode{datawc\_x1}}}} property of the object.

\item {} 
\textbf{\texttt{dsp4}} (\href{https://docs.python.org/2/library/functions.html\#float}{\emph{\texttt{float}}}) -- Sets the {\hyperref[vcs/graphics/isofill:vcs.isofill.Gfi.datawc_x2]{\sphinxcrossref{\sphinxcode{datawc\_x2}}}} property of the object.

\end{itemize}

\end{description}\end{quote}

\end{fulllineitems}

\index{exts() (vcs.isofill.Gfi method)}

\begin{fulllineitems}
\phantomsection\label{vcs/graphics/isofill:vcs.isofill.Gfi.exts}\pysiglinewithargsret{\sphinxbfcode{exts}}{\emph{ext1='n'}, \emph{ext2='y'}}{}
Sets the ext\_1 and ext\_2 values on the object.
\begin{quote}\begin{description}
\item[{Parameters}] \leavevmode\begin{itemize}
\item {} 
\textbf{\texttt{ext1}} (\href{https://docs.python.org/2/library/functions.html\#str}{\emph{\texttt{str}}}) -- Sets the {\hyperref[vcs/graphics/isofill:vcs.isofill.Gfi.ext_1]{\sphinxcrossref{\sphinxcode{ext\_1}}}} value on the object. `y' sets it to True, `n' sets it to False.

\item {} 
\textbf{\texttt{ext2}} (\href{https://docs.python.org/2/library/functions.html\#str}{\emph{\texttt{str}}}) -- Sets the {\hyperref[vcs/graphics/isofill:vcs.isofill.Gfi.ext_2]{\sphinxcrossref{\sphinxcode{ext\_2}}}} value on the object. `y' sets it to True, `n' sets it to False.

\end{itemize}

\end{description}\end{quote}

\end{fulllineitems}

\index{list() (vcs.isofill.Gfi method)}

\begin{fulllineitems}
\phantomsection\label{vcs/graphics/isofill:vcs.isofill.Gfi.list}\pysiglinewithargsret{\sphinxbfcode{list}}{}{}
Lists the current values of object attributes

\end{fulllineitems}

\index{script() (vcs.isofill.Gfi method)}

\begin{fulllineitems}
\phantomsection\label{vcs/graphics/isofill:vcs.isofill.Gfi.script}\pysiglinewithargsret{\sphinxbfcode{script}}{\emph{script\_filename}, \emph{mode='a'}}{}
Saves out a copy of the isofill graphics method in JSON, or Python format to a designated file.
\begin{quote}

\begin{notice}{note}{Note:}
If the the filename has a `.py' at the end, it will produce a
Python script. If no extension is given, then by default a
.json file containing a JSON serialization of the object's
data will be produced.
\end{notice}

\begin{notice}{warning}{Warning:}
VCS Scripts Deprecated.
SCR script files are no longer generated by this function.
\end{notice}
\end{quote}
\begin{quote}\begin{description}
\item[{Example}] \leavevmode
\begin{Verbatim}[commandchars=\\\{\}]
\PYG{g+gp}{\PYGZgt{}\PYGZgt{}\PYGZgt{} }\PYG{n}{a}\PYG{o}{=}\PYG{n}{vcs}\PYG{o}{.}\PYG{n}{init}\PYG{p}{(}\PYG{p}{)} \PYG{c+c1}{\PYGZsh{} Make a Canvas object to work with}
\PYG{g+gp}{\PYGZgt{}\PYGZgt{}\PYGZgt{} }\PYG{n}{ex}\PYG{o}{=}\PYG{n}{a}\PYG{o}{.}\PYG{n}{getisofill}\PYG{p}{(}\PYG{p}{)} \PYG{c+c1}{\PYGZsh{} Get default isofill}
\PYG{g+gp}{\PYGZgt{}\PYGZgt{}\PYGZgt{} }\PYG{n}{ex}\PYG{o}{.}\PYG{n}{script}\PYG{p}{(}\PYG{l+s+s1}{\PYGZsq{}}\PYG{l+s+s1}{filename.py}\PYG{l+s+s1}{\PYGZsq{}}\PYG{p}{)} \PYG{c+c1}{\PYGZsh{} Append to a Python script named \PYGZsq{}filename.py\PYGZsq{}}
\PYG{g+gp}{\PYGZgt{}\PYGZgt{}\PYGZgt{} }\PYG{n}{ex}\PYG{o}{.}\PYG{n}{script}\PYG{p}{(}\PYG{l+s+s1}{\PYGZsq{}}\PYG{l+s+s1}{filename}\PYG{l+s+s1}{\PYGZsq{}}\PYG{p}{,}\PYG{l+s+s1}{\PYGZsq{}}\PYG{l+s+s1}{w}\PYG{l+s+s1}{\PYGZsq{}}\PYG{p}{)} \PYG{c+c1}{\PYGZsh{} Create or overwrite a JSON file \PYGZsq{}filename.json\PYGZsq{}.}
\end{Verbatim}

\item[{Parameters}] \leavevmode\begin{itemize}
\item {} 
\textbf{\texttt{script\_filename}} (\href{https://docs.python.org/2/library/functions.html\#str}{\emph{\texttt{str}}}) -- Output name of the script file. If no extension is specified, a .json object is created.

\item {} 
\textbf{\texttt{mode}} (\href{https://docs.python.org/2/library/functions.html\#str}{\emph{\texttt{str}}}) -- Either `w' for replace, or `a' for append. Defaults to `a', if not specified.

\end{itemize}

\end{description}\end{quote}

\end{fulllineitems}

\index{xmtics() (vcs.isofill.Gfi method)}

\begin{fulllineitems}
\phantomsection\label{vcs/graphics/isofill:vcs.isofill.Gfi.xmtics}\pysiglinewithargsret{\sphinxbfcode{xmtics}}{\emph{xmt1='`}, \emph{xmt2='`}}{}
Sets the xmtics1 and xmtics2 values on the object
\begin{quote}\begin{description}
\item[{Parameters}] \leavevmode\begin{itemize}
\item {} 
\textbf{\texttt{xmt1}} (\emph{\texttt{\{float:str\} or str}}) -- Value for {\hyperref[vcs/graphics/isofill:vcs.isofill.Gfi.xmtics1]{\sphinxcrossref{\sphinxcode{xmtics1}}}}. Must be a str, or a dictionary object with float:str mappings.

\item {} 
\textbf{\texttt{xmt2}} (\emph{\texttt{\{float:str\} or str}}) -- Value for {\hyperref[vcs/graphics/isofill:vcs.isofill.Gfi.xmtics2]{\sphinxcrossref{\sphinxcode{xmtics2}}}}. Must be a str, or a dictionary object with float:str mappings.

\end{itemize}

\end{description}\end{quote}

\end{fulllineitems}

\index{xticlabels() (vcs.isofill.Gfi method)}

\begin{fulllineitems}
\phantomsection\label{vcs/graphics/isofill:vcs.isofill.Gfi.xticlabels}\pysiglinewithargsret{\sphinxbfcode{xticlabels}}{\emph{xtl1='`}, \emph{xtl2='`}}{}
Sets the xticlabels1 and xticlabels2 values on the object
\begin{quote}\begin{description}
\item[{Parameters}] \leavevmode\begin{itemize}
\item {} 
\textbf{\texttt{xtl1}} (\emph{\texttt{\{float:str\} or str}}) -- Sets the object's value for {\hyperref[vcs/graphics/isofill:vcs.isofill.Gfi.xticlabels1]{\sphinxcrossref{\sphinxcode{xticlabels1}}}}. Must be  a str, or a dictionary object with float:str mappings.

\item {} 
\textbf{\texttt{xtl2}} (\emph{\texttt{\{float:str\} or str}}) -- Sets the object's value for {\hyperref[vcs/graphics/isofill:vcs.isofill.Gfi.xticlabels2]{\sphinxcrossref{\sphinxcode{xticlabels2}}}}. Must be a str, or a dictionary object with float:str mappings.

\end{itemize}

\end{description}\end{quote}

\end{fulllineitems}

\index{xyscale() (vcs.isofill.Gfi method)}

\begin{fulllineitems}
\phantomsection\label{vcs/graphics/isofill:vcs.isofill.Gfi.xyscale}\pysiglinewithargsret{\sphinxbfcode{xyscale}}{\emph{xat='`}, \emph{yat='`}}{}
Sets xaxisconvert and yaxisconvert values for the object.
\begin{quote}\begin{description}
\item[{Example}] \leavevmode
\begin{Verbatim}[commandchars=\\\{\}]
\PYG{g+gp}{\PYGZgt{}\PYGZgt{}\PYGZgt{} }\PYG{n}{a}\PYG{o}{=}\PYG{n}{vcs}\PYG{o}{.}\PYG{n}{init}\PYG{p}{(}\PYG{p}{)}
\PYG{g+gp}{\PYGZgt{}\PYGZgt{}\PYGZgt{} }\PYG{n}{ex}\PYG{o}{=}\PYG{n}{a}\PYG{o}{.}\PYG{n}{createisofill}\PYG{p}{(}\PYG{l+s+s1}{\PYGZsq{}}\PYG{l+s+s1}{xyscale\PYGZus{}ex}\PYG{l+s+s1}{\PYGZsq{}}\PYG{p}{)} \PYG{c+c1}{\PYGZsh{} create a boxfill to work with}
\PYG{g+gp}{\PYGZgt{}\PYGZgt{}\PYGZgt{} }\PYG{n}{ex}\PYG{o}{.}\PYG{n}{xyscale}\PYG{p}{(}\PYG{n}{xat}\PYG{o}{=}\PYG{l+s+s1}{\PYGZsq{}}\PYG{l+s+s1}{linear}\PYG{l+s+s1}{\PYGZsq{}}\PYG{p}{,} \PYG{n}{yat}\PYG{o}{=}\PYG{l+s+s1}{\PYGZsq{}}\PYG{l+s+s1}{linear}\PYG{l+s+s1}{\PYGZsq{}}\PYG{p}{)} \PYG{c+c1}{\PYGZsh{} set xaxisconvert and yaxisconvert to \PYGZsq{}linear\PYGZsq{}}
\end{Verbatim}

\item[{Parameters}] \leavevmode\begin{itemize}
\item {} 
\textbf{\texttt{xat}} (\href{https://docs.python.org/2/library/functions.html\#str}{\emph{\texttt{str}}}) -- Set value for x axis conversion.

\item {} 
\textbf{\texttt{yat}} (\href{https://docs.python.org/2/library/functions.html\#str}{\emph{\texttt{str}}}) -- Set value for y axis conversion.

\end{itemize}

\end{description}\end{quote}

\end{fulllineitems}

\index{ymtics() (vcs.isofill.Gfi method)}

\begin{fulllineitems}
\phantomsection\label{vcs/graphics/isofill:vcs.isofill.Gfi.ymtics}\pysiglinewithargsret{\sphinxbfcode{ymtics}}{\emph{ymt1='`}, \emph{ymt2='`}}{}
Sets the ymtics1 and ymtics2 values on the object
\begin{quote}\begin{description}
\item[{Parameters}] \leavevmode\begin{itemize}
\item {} 
\textbf{\texttt{ymt1}} (\emph{\texttt{\{float:str\} or str}}) -- Value for {\hyperref[vcs/graphics/isofill:vcs.isofill.Gfi.ymtics1]{\sphinxcrossref{\sphinxcode{ymtics1}}}}. Must be a str, or a dictionary object with float:str mappings.

\item {} 
\textbf{\texttt{ymt2}} (\emph{\texttt{\{float:str\} or str}}) -- Value for {\hyperref[vcs/graphics/isofill:vcs.isofill.Gfi.ymtics2]{\sphinxcrossref{\sphinxcode{ymtics2}}}}. Must be a str, or a dictionary object with float:str mappings.

\end{itemize}

\end{description}\end{quote}

\end{fulllineitems}

\index{yticlabels() (vcs.isofill.Gfi method)}

\begin{fulllineitems}
\phantomsection\label{vcs/graphics/isofill:vcs.isofill.Gfi.yticlabels}\pysiglinewithargsret{\sphinxbfcode{yticlabels}}{\emph{ytl1='`}, \emph{ytl2='`}}{}
Sets the yticlabels1 and yticlabels2 values on the object
\begin{quote}\begin{description}
\item[{Parameters}] \leavevmode\begin{itemize}
\item {} 
\textbf{\texttt{ytl1}} (\emph{\texttt{\{float:str\} or str}}) -- Sets the object's value for {\hyperref[vcs/graphics/isofill:vcs.isofill.Gfi.yticlabels1]{\sphinxcrossref{\sphinxcode{yticlabels1}}}}. Must be  a str, or a dictionary object with float:str mappings.

\item {} 
\textbf{\texttt{ytl2}} (\emph{\texttt{\{float:str\} or str}}) -- Sets the object's value for {\hyperref[vcs/graphics/isofill:vcs.isofill.Gfi.yticlabels2]{\sphinxcrossref{\sphinxcode{yticlabels2}}}}. Must be a str, or a dictionary object with float:str mappings.

\end{itemize}

\end{description}\end{quote}

\end{fulllineitems}


\end{fulllineitems}



\subsection{isoline}
\label{vcs/graphics/isoline:isoline}\label{vcs/graphics/isoline::doc}\label{vcs/graphics/isoline:module-vcs.isoline}\index{vcs.isoline (module)}
\# Isoline (Gi) module
\index{Gi (class in vcs.isoline)}

\begin{fulllineitems}
\phantomsection\label{vcs/graphics/isoline:vcs.isoline.Gi}\pysiglinewithargsret{\sphinxstrong{class }\sphinxcode{vcs.isoline.}\sphinxbfcode{Gi}}{\emph{Gi\_name}, \emph{Gi\_name\_src='default'}}{}
The Isoline graphics method (Gi) draws lines of constant value at specified
levels in order to graphically represent a two-dimensional array. It
also labels the values of these isolines on the VCS Canvas. The example
below shows how to plot isolines of different types at specified levels
and how to create isoline labels having user-specified text and line type
and color.

This class is used to define an isoline table entry used in VCS, or it can
be used to change some or all of the isoline attributes in an existing isoline
table entry.


\begin{fulllineitems}
\pysigline{\sphinxbfcode{Useful~Functions:}}~
\begin{Verbatim}[commandchars=\\\{\}]
\PYG{c+c1}{\PYGZsh{} VCS Canvas Constructor}
\PYG{n}{a}\PYG{o}{=}\PYG{n}{vcs}\PYG{o}{.}\PYG{n}{init}\PYG{p}{(}\PYG{p}{)}
\PYG{c+c1}{\PYGZsh{} Show predefined isoline graphics methods}
\PYG{n}{a}\PYG{o}{.}\PYG{n}{show}\PYG{p}{(}\PYG{l+s+s1}{\PYGZsq{}}\PYG{l+s+s1}{isoline}\PYG{l+s+s1}{\PYGZsq{}}\PYG{p}{)}
\PYG{c+c1}{\PYGZsh{} Show predefined VCS line objects}
\PYG{n}{a}\PYG{o}{.}\PYG{n}{show}\PYG{p}{(}\PYG{l+s+s1}{\PYGZsq{}}\PYG{l+s+s1}{line}\PYG{l+s+s1}{\PYGZsq{}}\PYG{p}{)}
\PYG{c+c1}{\PYGZsh{} Change the VCS color map}
\PYG{n}{a}\PYG{o}{.}\PYG{n}{setcolormap}\PYG{p}{(}\PYG{l+s+s2}{\PYGZdq{}}\PYG{l+s+s2}{AMIP}\PYG{l+s+s2}{\PYGZdq{}}\PYG{p}{)}
\PYG{c+c1}{\PYGZsh{} Plot data \PYGZsq{}s\PYGZsq{} with isoline \PYGZsq{}i\PYGZsq{} and \PYGZsq{}default\PYGZsq{} template}
\PYG{n}{a}\PYG{o}{.}\PYG{n}{isoline}\PYG{p}{(}\PYG{n}{s}\PYG{p}{,}\PYG{n}{a}\PYG{p}{,}\PYG{l+s+s1}{\PYGZsq{}}\PYG{l+s+s1}{default}\PYG{l+s+s1}{\PYGZsq{}}\PYG{p}{)}
\PYG{c+c1}{\PYGZsh{} Updates the VCS Canvas at user\PYGZsq{}s request}
\PYG{n}{a}\PYG{o}{.}\PYG{n}{update}\PYG{p}{(}\PYG{p}{)}
\end{Verbatim}

\end{fulllineitems}



\begin{fulllineitems}
\pysigline{\sphinxbfcode{Create~a~canvas~object:}}~
\begin{Verbatim}[commandchars=\\\{\}]
\PYG{n}{a}\PYG{o}{=}\PYG{n}{vcs}\PYG{o}{.}\PYG{n}{init}\PYG{p}{(}\PYG{p}{)}
\end{Verbatim}

\end{fulllineitems}



\begin{fulllineitems}
\pysigline{\sphinxbfcode{Create~a~new~instance~of~isoline:}}~
\begin{Verbatim}[commandchars=\\\{\}]
\PYG{c+c1}{\PYGZsh{} Copies content of \PYGZsq{}quick\PYGZsq{} to \PYGZsq{}new\PYGZsq{}}
\PYG{n}{iso}\PYG{o}{=}\PYG{n}{a}\PYG{o}{.}\PYG{n}{createisoline}\PYG{p}{(}\PYG{l+s+s1}{\PYGZsq{}}\PYG{l+s+s1}{new}\PYG{l+s+s1}{\PYGZsq{}}\PYG{p}{,}\PYG{l+s+s1}{\PYGZsq{}}\PYG{l+s+s1}{quick}\PYG{l+s+s1}{\PYGZsq{}}\PYG{p}{)}
\PYG{c+c1}{\PYGZsh{} Copies content of \PYGZsq{}default\PYGZsq{} to \PYGZsq{}new\PYGZsq{}}
\PYG{n}{iso}\PYG{o}{=}\PYG{n}{a}\PYG{o}{.}\PYG{n}{createisoline}\PYG{p}{(}\PYG{l+s+s1}{\PYGZsq{}}\PYG{l+s+s1}{new}\PYG{l+s+s1}{\PYGZsq{}}\PYG{p}{)}
\end{Verbatim}

\end{fulllineitems}



\begin{fulllineitems}
\pysigline{\sphinxbfcode{Modify~an~existing~isoline:}}~
\begin{Verbatim}[commandchars=\\\{\}]
\PYG{n}{iso}\PYG{o}{=}\PYG{n}{a}\PYG{o}{.}\PYG{n}{getisoline}\PYG{p}{(}\PYG{l+s+s1}{\PYGZsq{}}\PYG{l+s+s1}{AMIP\PYGZus{}psl}\PYG{l+s+s1}{\PYGZsq{}}\PYG{p}{)}
\end{Verbatim}

\end{fulllineitems}

\phantomsection\label{vcs/graphics/isoline:isoline-attributes}

\begin{fulllineitems}
\pysigline{\sphinxbfcode{Overview~of~isoline~attributes:}}~\begin{itemize}
\item {} 
List all the isoline attribute values
\begin{quote}

\begin{Verbatim}[commandchars=\\\{\}]
\PYG{n}{iso}\PYG{o}{.}\PYG{n}{list}\PYG{p}{(}\PYG{p}{)}
\end{Verbatim}
\end{quote}

\item {} 
Set isoline attribute values:
\begin{quote}

\begin{Verbatim}[commandchars=\\\{\}]
\PYG{n}{iso}\PYG{o}{.}\PYG{n}{projection}\PYG{o}{=}\PYG{l+s+s1}{\PYGZsq{}}\PYG{l+s+s1}{linear}\PYG{l+s+s1}{\PYGZsq{}}
\PYG{n}{lon30}\PYG{o}{=}\PYG{p}{\PYGZob{}}\PYG{o}{\PYGZhy{}}\PYG{l+m+mi}{180}\PYG{p}{:}\PYG{l+s+s1}{\PYGZsq{}}\PYG{l+s+s1}{180W}\PYG{l+s+s1}{\PYGZsq{}}\PYG{p}{,}\PYG{o}{\PYGZhy{}}\PYG{l+m+mi}{150}\PYG{p}{:}\PYG{l+s+s1}{\PYGZsq{}}\PYG{l+s+s1}{150W}\PYG{l+s+s1}{\PYGZsq{}}\PYG{p}{,}\PYG{l+m+mi}{0}\PYG{p}{:}\PYG{l+s+s1}{\PYGZsq{}}\PYG{l+s+s1}{Eq}\PYG{l+s+s1}{\PYGZsq{}}\PYG{p}{\PYGZcb{}}
\PYG{n}{iso}\PYG{o}{.}\PYG{n}{xticlabels1}\PYG{o}{=}\PYG{n}{lon30}
\PYG{n}{iso}\PYG{o}{.}\PYG{n}{xticlabels2}\PYG{o}{=}\PYG{n}{lon30}
\PYG{c+c1}{\PYGZsh{} Will set them both}
\PYG{n}{iso}\PYG{o}{.}\PYG{n}{xticlabels}\PYG{p}{(}\PYG{n}{lon30}\PYG{p}{,} \PYG{n}{lon30}\PYG{p}{)}
\PYG{n}{iso}\PYG{o}{.}\PYG{n}{xmtics1}\PYG{o}{=}\PYG{l+s+s1}{\PYGZsq{}}\PYG{l+s+s1}{\PYGZsq{}}
\PYG{n}{iso}\PYG{o}{.}\PYG{n}{xmtics2}\PYG{o}{=}\PYG{l+s+s1}{\PYGZsq{}}\PYG{l+s+s1}{\PYGZsq{}}
\PYG{c+c1}{\PYGZsh{} Will set them both}
\PYG{n}{iso}\PYG{o}{.}\PYG{n}{xmtics}\PYG{p}{(}\PYG{n}{lon30}\PYG{p}{,} \PYG{n}{lon30}\PYG{p}{)}
\PYG{n}{iso}\PYG{o}{.}\PYG{n}{yticlabels1}\PYG{o}{=}\PYG{n}{lat10}
\PYG{n}{iso}\PYG{o}{.}\PYG{n}{yticlabels2}\PYG{o}{=}\PYG{n}{lat10}
\PYG{c+c1}{\PYGZsh{} Will set them both}
\PYG{n}{iso}\PYG{o}{.}\PYG{n}{yticlabels}\PYG{p}{(}\PYG{n}{lat10}\PYG{p}{,} \PYG{n}{lat10}\PYG{p}{)}
\PYG{n}{iso}\PYG{o}{.}\PYG{n}{ymtics1}\PYG{o}{=}\PYG{l+s+s1}{\PYGZsq{}}\PYG{l+s+s1}{\PYGZsq{}}
\PYG{n}{iso}\PYG{o}{.}\PYG{n}{ymtics2}\PYG{o}{=}\PYG{l+s+s1}{\PYGZsq{}}\PYG{l+s+s1}{\PYGZsq{}}
\PYG{c+c1}{\PYGZsh{} Will set them both}
\PYG{n}{iso}\PYG{o}{.}\PYG{n}{ymtics}\PYG{p}{(}\PYG{n}{lat10}\PYG{p}{,} \PYG{n}{lat10}\PYG{p}{)}
\PYG{n}{iso}\PYG{o}{.}\PYG{n}{datawc\PYGZus{}y1}\PYG{o}{=}\PYG{o}{\PYGZhy{}}\PYG{l+m+mf}{90.0}
\PYG{n}{iso}\PYG{o}{.}\PYG{n}{datawc\PYGZus{}y2}\PYG{o}{=}\PYG{l+m+mf}{90.0}
\PYG{n}{iso}\PYG{o}{.}\PYG{n}{datawc\PYGZus{}x1}\PYG{o}{=}\PYG{o}{\PYGZhy{}}\PYG{l+m+mf}{180.0}
\PYG{n}{iso}\PYG{o}{.}\PYG{n}{datawc\PYGZus{}x2}\PYG{o}{=}\PYG{l+m+mf}{180.0}
\PYG{c+c1}{\PYGZsh{} Will set them all}
\PYG{n}{iso}\PYG{o}{.}\PYG{n}{datawc}\PYG{p}{(}\PYG{o}{\PYGZhy{}}\PYG{l+m+mi}{90}\PYG{p}{,} \PYG{l+m+mi}{90}\PYG{p}{,} \PYG{o}{\PYGZhy{}}\PYG{l+m+mi}{180}\PYG{p}{,} \PYG{l+m+mi}{180}\PYG{p}{)}
\PYG{n}{xaxisconvert}\PYG{o}{=}\PYG{l+s+s1}{\PYGZsq{}}\PYG{l+s+s1}{linear}\PYG{l+s+s1}{\PYGZsq{}}
\PYG{n}{yaxisconvert}\PYG{o}{=}\PYG{l+s+s1}{\PYGZsq{}}\PYG{l+s+s1}{linear}\PYG{l+s+s1}{\PYGZsq{}}
\PYG{c+c1}{\PYGZsh{} Will set them both}
\PYG{n}{iso}\PYG{o}{.}\PYG{n}{xyscale}\PYG{p}{(}\PYG{l+s+s1}{\PYGZsq{}}\PYG{l+s+s1}{linear}\PYG{l+s+s1}{\PYGZsq{}}\PYG{p}{,} \PYG{l+s+s1}{\PYGZsq{}}\PYG{l+s+s1}{area\PYGZus{}wt}\PYG{l+s+s1}{\PYGZsq{}}\PYG{p}{)}
\end{Verbatim}
\end{quote}

\item {} 
Setting isoline {\hyperref[vcs/graphics/isoline:vcs.isoline.Gi.level]{\sphinxcrossref{\sphinxcode{level}}}} values:
\begin{quote}

\begin{Verbatim}[commandchars=\\\{\}]
\PYG{c+c1}{\PYGZsh{}1) As a list of tuples (Examples):}
    \PYG{n}{iso}\PYG{o}{.}\PYG{n}{level}\PYG{o}{=}\PYG{p}{[}\PYG{p}{(}\PYG{l+m+mi}{23}\PYG{p}{,}\PYG{l+m+mi}{32}\PYG{p}{,}\PYG{l+m+mi}{45}\PYG{p}{,}\PYG{l+m+mi}{50}\PYG{p}{,}\PYG{l+m+mi}{76}\PYG{p}{)}\PYG{p}{,}\PYG{p}{]}
    \PYG{n}{iso}\PYG{o}{.}\PYG{n}{level}\PYG{o}{=}\PYG{p}{[}\PYG{p}{(}\PYG{l+m+mi}{22}\PYG{p}{,}\PYG{l+m+mi}{33}\PYG{p}{,}\PYG{l+m+mi}{44}\PYG{p}{,}\PYG{l+m+mi}{55}\PYG{p}{,}\PYG{l+m+mi}{66}\PYG{p}{)}\PYG{p}{]}
    \PYG{n}{iso}\PYG{o}{.}\PYG{n}{level}\PYG{o}{=}\PYG{p}{[}\PYG{p}{(}\PYG{l+m+mi}{20}\PYG{p}{,}\PYG{l+m+mf}{0.0}\PYG{p}{)}\PYG{p}{,}\PYG{p}{(}\PYG{l+m+mi}{30}\PYG{p}{,}\PYG{l+m+mi}{0}\PYG{p}{)}\PYG{p}{,}\PYG{p}{(}\PYG{l+m+mi}{50}\PYG{p}{,}\PYG{l+m+mi}{0}\PYG{p}{)}\PYG{p}{]}
    \PYG{n}{iso}\PYG{o}{.}\PYG{n}{level}\PYG{o}{=}\PYG{p}{[}\PYG{p}{(}\PYG{l+m+mi}{23}\PYG{p}{,}\PYG{l+m+mi}{32}\PYG{p}{,}\PYG{l+m+mi}{45}\PYG{p}{,}\PYG{l+m+mi}{50}\PYG{p}{,}\PYG{l+m+mi}{76}\PYG{p}{)}\PYG{p}{,} \PYG{p}{(}\PYG{l+m+mi}{35}\PYG{p}{,} \PYG{l+m+mi}{45}\PYG{p}{,} \PYG{l+m+mi}{55}\PYG{p}{)}\PYG{p}{]}
\PYG{c+c1}{\PYGZsh{}2) As a tuple of lists (Examples):}
    \PYG{n}{iso}\PYG{o}{.}\PYG{n}{level}\PYG{o}{=}\PYG{p}{(}\PYG{p}{[}\PYG{l+m+mi}{23}\PYG{p}{,}\PYG{l+m+mi}{32}\PYG{p}{,}\PYG{l+m+mi}{45}\PYG{p}{,}\PYG{l+m+mi}{50}\PYG{p}{,}\PYG{l+m+mi}{76}\PYG{p}{]}\PYG{p}{,}\PYG{p}{)}
    \PYG{n}{iso}\PYG{o}{.}\PYG{n}{level}\PYG{o}{=}\PYG{p}{(}\PYG{p}{[}\PYG{l+m+mi}{22}\PYG{p}{,}\PYG{l+m+mi}{33}\PYG{p}{,}\PYG{l+m+mi}{44}\PYG{p}{,}\PYG{l+m+mi}{55}\PYG{p}{,}\PYG{l+m+mi}{66}\PYG{p}{]}\PYG{p}{)}
    \PYG{n}{iso}\PYG{o}{.}\PYG{n}{level}\PYG{o}{=}\PYG{p}{(}\PYG{p}{[}\PYG{l+m+mi}{23}\PYG{p}{,}\PYG{l+m+mi}{32}\PYG{p}{,}\PYG{l+m+mi}{45}\PYG{p}{,}\PYG{l+m+mi}{50}\PYG{p}{,}\PYG{l+m+mi}{76}\PYG{p}{]}\PYG{p}{,}\PYG{p}{)}
    \PYG{n}{iso}\PYG{o}{.}\PYG{n}{level}\PYG{o}{=}\PYG{p}{(}\PYG{p}{[}\PYG{l+m+mi}{0}\PYG{p}{,}\PYG{l+m+mi}{20}\PYG{p}{,}\PYG{l+m+mi}{25}\PYG{p}{,}\PYG{l+m+mi}{30}\PYG{p}{,}\PYG{l+m+mi}{35}\PYG{p}{,}\PYG{l+m+mi}{40}\PYG{p}{]}\PYG{p}{,}\PYG{p}{[}\PYG{l+m+mi}{30}\PYG{p}{,}\PYG{l+m+mi}{40}\PYG{p}{]}\PYG{p}{,}\PYG{p}{[}\PYG{l+m+mi}{50}\PYG{p}{,}\PYG{l+m+mi}{60}\PYG{p}{]}\PYG{p}{)}
\PYG{c+c1}{\PYGZsh{}3) As a list of lists (Examples):}
    \PYG{n}{iso}\PYG{o}{.}\PYG{n}{level}\PYG{o}{=}\PYG{p}{[}\PYG{p}{[}\PYG{l+m+mi}{20}\PYG{p}{,}\PYG{l+m+mf}{0.0}\PYG{p}{]}\PYG{p}{,}\PYG{p}{[}\PYG{l+m+mi}{30}\PYG{p}{,}\PYG{l+m+mi}{0}\PYG{p}{]}\PYG{p}{,}\PYG{p}{[}\PYG{l+m+mi}{50}\PYG{p}{,}\PYG{l+m+mi}{0}\PYG{p}{]}\PYG{p}{]}
\PYG{c+c1}{\PYGZsh{}4) As a tuple of tuples (Examples):}
    \PYG{n}{iso}\PYG{o}{.}\PYG{n}{level}\PYG{o}{=}\PYG{p}{(}\PYG{p}{(}\PYG{l+m+mi}{20}\PYG{p}{,}\PYG{l+m+mf}{0.0}\PYG{p}{)}\PYG{p}{,}\PYG{p}{(}\PYG{l+m+mi}{30}\PYG{p}{,}\PYG{l+m+mi}{0}\PYG{p}{)}\PYG{p}{,}\PYG{p}{(}\PYG{l+m+mi}{50}\PYG{p}{,}\PYG{l+m+mi}{0}\PYG{p}{)}\PYG{p}{,}\PYG{p}{(}\PYG{l+m+mi}{60}\PYG{p}{,}\PYG{l+m+mi}{0}\PYG{p}{)}\PYG{p}{,}\PYG{p}{(}\PYG{l+m+mi}{70}\PYG{p}{,}\PYG{l+m+mi}{0}\PYG{p}{)}\PYG{p}{)}
\end{Verbatim}

\begin{notice}{note}{Note:}
A combination of a pairs (i.e., (30,0) or {[}30,0{]})
represents the isoline value plus its increment value.
Thus, to let VCS generate ``default'' isolines:

\begin{Verbatim}[commandchars=\\\{\}]
\PYG{c+c1}{\PYGZsh{} Same as iso.level=((0,1e20),)}
\PYG{n}{iso}\PYG{o}{.}\PYG{n}{level}\PYG{o}{=}\PYG{p}{[}\PYG{p}{[}\PYG{l+m+mi}{0}\PYG{p}{,}\PYG{l+m+mf}{1e20}\PYG{p}{]}\PYG{p}{]}
\end{Verbatim}
\end{notice}
\end{quote}

\item {} 
Displaying isoline labels:
\begin{quote}

\begin{Verbatim}[commandchars=\\\{\}]
\PYG{c+c1}{\PYGZsh{} Same as iso.label=1, will display isoline labels}
\PYG{n}{iso}\PYG{o}{.}\PYG{n}{label}\PYG{o}{=}\PYG{l+s+s1}{\PYGZsq{}}\PYG{l+s+s1}{y}\PYG{l+s+s1}{\PYGZsq{}}
\PYG{c+c1}{\PYGZsh{} Same as iso.label=0, will turn isoline labels off}
\PYG{n}{iso}\PYG{o}{.}\PYG{n}{label}\PYG{o}{=}\PYG{l+s+s1}{\PYGZsq{}}\PYG{l+s+s1}{n}\PYG{l+s+s1}{\PYGZsq{}}
\end{Verbatim}
\end{quote}

\item {} 
Specify the isoline line style (or type):
\begin{quote}

\begin{Verbatim}[commandchars=\\\{\}]
\PYG{c+c1}{\PYGZsh{} The following two lines of code are equivalent.}
\PYG{n}{iso}\PYG{o}{.}\PYG{n}{line}\PYG{o}{=}\PYG{p}{(}\PYG{p}{[}\PYG{l+m+mi}{0}\PYG{p}{,}\PYG{l+m+mi}{1}\PYG{p}{,}\PYG{l+m+mi}{2}\PYG{p}{,}\PYG{l+m+mi}{3}\PYG{p}{,}\PYG{l+m+mi}{4}\PYG{p}{]}\PYG{p}{)}
\PYG{c+c1}{\PYGZsh{} Both specify the isoline style}
\PYG{n}{iso}\PYG{o}{.}\PYG{n}{line}\PYG{o}{=}\PYG{p}{(}\PYG{p}{[}\PYG{l+s+s1}{\PYGZsq{}}\PYG{l+s+s1}{solid, }\PYG{l+s+s1}{\PYGZsq{}}\PYG{n}{dash}\PYG{l+s+s1}{\PYGZsq{}}\PYG{l+s+s1}{, }\PYG{l+s+s1}{\PYGZsq{}}\PYG{n}{dot}\PYG{l+s+s1}{\PYGZsq{}}\PYG{l+s+s1}{, }\PYG{l+s+s1}{\PYGZsq{}}\PYG{n}{dash}\PYG{o}{\PYGZhy{}}\PYG{n}{dot}\PYG{l+s+s1}{\PYGZsq{}}\PYG{l+s+s1}{, }\PYG{l+s+s1}{\PYGZsq{}}\PYG{n+nb}{long}\PYG{o}{\PYGZhy{}}\PYG{n}{dash}\PYG{l+s+s1}{\PYGZsq{}}\PYG{l+s+s1}{])}
\end{Verbatim}
\end{quote}

\item {} 
There are three possibilities for setting the line color indices:
\begin{quote}

\begin{Verbatim}[commandchars=\\\{\}]
\PYG{c+c1}{\PYGZsh{} The following two lines of code are equivalent}
\PYG{c+c1}{\PYGZsh{} Both will set the isoline to a specific color index}
\PYG{n}{iso}\PYG{o}{.}\PYG{n}{linecolors}\PYG{o}{=}\PYG{p}{(}\PYG{l+m+mi}{22}\PYG{p}{,}\PYG{l+m+mi}{33}\PYG{p}{,}\PYG{l+m+mi}{44}\PYG{p}{,}\PYG{l+m+mi}{55}\PYG{p}{,}\PYG{l+m+mi}{66}\PYG{p}{,}\PYG{l+m+mi}{77}\PYG{p}{)}
\PYG{n}{iso}\PYG{o}{.}\PYG{n}{linecolors}\PYG{o}{=}\PYG{p}{(}\PYG{p}{[}\PYG{l+m+mi}{22}\PYG{p}{,}\PYG{l+m+mi}{33}\PYG{p}{,}\PYG{l+m+mi}{44}\PYG{p}{,}\PYG{l+m+mi}{55}\PYG{p}{,}\PYG{l+m+mi}{66}\PYG{p}{,}\PYG{l+m+mi}{77}\PYG{p}{]}\PYG{p}{)}
\PYG{c+c1}{\PYGZsh{} Turns off the line color index}
\PYG{n}{iso}\PYG{o}{.}\PYG{n}{linecolors}\PYG{o}{=}\PYG{n+nb+bp}{None}
\end{Verbatim}
\end{quote}

\item {} 
There are three possibilities for setting the line widths:
\begin{quote}

\begin{Verbatim}[commandchars=\\\{\}]
\PYG{c+c1}{\PYGZsh{} The following two lines of code are equivalent}
\PYG{n}{iso}\PYG{o}{.}\PYG{n}{linewidths}\PYG{o}{=}\PYG{p}{(}\PYG{l+m+mi}{1}\PYG{p}{,}\PYG{l+m+mi}{10}\PYG{p}{,}\PYG{l+m+mi}{3}\PYG{p}{,}\PYG{l+m+mi}{4}\PYG{p}{,}\PYG{l+m+mi}{5}\PYG{p}{,}\PYG{l+m+mi}{6}\PYG{p}{,}\PYG{l+m+mi}{7}\PYG{p}{,}\PYG{l+m+mi}{8}\PYG{p}{)}
\PYG{c+c1}{\PYGZsh{} Both will set the isoline to a specific width size}
\PYG{n}{iso}\PYG{o}{.}\PYG{n}{linewidths}\PYG{o}{=}\PYG{p}{(}\PYG{p}{[}\PYG{l+m+mi}{1}\PYG{p}{,}\PYG{l+m+mi}{2}\PYG{p}{,}\PYG{l+m+mi}{3}\PYG{p}{,}\PYG{l+m+mi}{4}\PYG{p}{,}\PYG{l+m+mi}{5}\PYG{p}{,}\PYG{l+m+mi}{6}\PYG{p}{,}\PYG{l+m+mi}{7}\PYG{p}{,}\PYG{l+m+mi}{8}\PYG{p}{]}\PYG{p}{)}
\PYG{c+c1}{\PYGZsh{} Turns off the line width size}
\PYG{n}{iso}\PYG{o}{.}\PYG{n}{linewidths}\PYG{o}{=}\PYG{n+nb+bp}{None}
\end{Verbatim}

\begin{notice}{note}{Note:}
If the number of line styles, colors or widths are less
than the number of levels, we extend the attribute list
using the last attribute value in the attribute list.
\end{notice}
\end{quote}

\item {} 
There are three ways to specify the text or font number:
\begin{quote}

\begin{Verbatim}[commandchars=\\\{\}]
\PYG{c+c1}{\PYGZsh{} Font numbers are between 1 and 9}
\PYG{n}{iso}\PYG{o}{.}\PYG{n}{text}\PYG{o}{=}\PYG{p}{(}\PYG{l+m+mi}{1}\PYG{p}{,}\PYG{l+m+mi}{2}\PYG{p}{,}\PYG{l+m+mi}{3}\PYG{p}{,}\PYG{l+m+mi}{4}\PYG{p}{,}\PYG{l+m+mi}{5}\PYG{p}{,}\PYG{l+m+mi}{6}\PYG{p}{,}\PYG{l+m+mi}{7}\PYG{p}{,}\PYG{l+m+mi}{8}\PYG{p}{,}\PYG{l+m+mi}{9}\PYG{p}{)}
\PYG{n}{iso}\PYG{o}{.}\PYG{n}{text}\PYG{o}{=}\PYG{p}{[}\PYG{l+m+mi}{9}\PYG{p}{,}\PYG{l+m+mi}{8}\PYG{p}{,}\PYG{l+m+mi}{7}\PYG{p}{,}\PYG{l+m+mi}{6}\PYG{p}{,}\PYG{l+m+mi}{5}\PYG{p}{,}\PYG{l+m+mi}{4}\PYG{p}{,}\PYG{l+m+mi}{3}\PYG{p}{,}\PYG{l+m+mi}{2}\PYG{p}{,}\PYG{l+m+mi}{1}\PYG{p}{]}
\PYG{n}{iso}\PYG{o}{.}\PYG{n}{text}\PYG{o}{=}\PYG{p}{(}\PYG{p}{[}\PYG{l+m+mi}{1}\PYG{p}{,}\PYG{l+m+mi}{3}\PYG{p}{,}\PYG{l+m+mi}{5}\PYG{p}{,}\PYG{l+m+mi}{6}\PYG{p}{,}\PYG{l+m+mi}{9}\PYG{p}{,}\PYG{l+m+mi}{2}\PYG{p}{]}\PYG{p}{)}
\PYG{c+c1}{\PYGZsh{} Removes the text settings}
\PYG{n}{iso}\PYG{o}{.}\PYG{n}{text}\PYG{o}{=}\PYG{n+nb+bp}{None}
\end{Verbatim}
\end{quote}

\item {} 
There are three possibilities for setting the text color indices:
\begin{quote}

\begin{Verbatim}[commandchars=\\\{\}]
\PYG{n}{iso}\PYG{o}{.}\PYG{n}{textcolors}\PYG{o}{=}\PYG{p}{(}\PYG{p}{[}\PYG{l+m+mi}{22}\PYG{p}{,}\PYG{l+m+mi}{33}\PYG{p}{,}\PYG{l+m+mi}{44}\PYG{p}{,}\PYG{l+m+mi}{55}\PYG{p}{,}\PYG{l+m+mi}{66}\PYG{p}{,}\PYG{l+m+mi}{77}\PYG{p}{]}\PYG{p}{)}
\PYG{n}{iso}\PYG{o}{.}\PYG{n}{textcolors}\PYG{o}{=}\PYG{p}{(}\PYG{l+m+mi}{16}\PYG{p}{,}\PYG{l+m+mi}{19}\PYG{p}{,}\PYG{l+m+mi}{33}\PYG{p}{,}\PYG{l+m+mi}{44}\PYG{p}{)}
\PYG{c+c1}{\PYGZsh{} Turns off the text color index}
\PYG{n}{iso}\PYG{o}{.}\PYG{n}{textcolors}\PYG{o}{=}\PYG{n+nb+bp}{None}
\end{Verbatim}
\end{quote}

\end{itemize}
\phantomsection\label{vcs/graphics/isoline:isoline-attribute-descriptions}\begin{itemize}
\item {} 
Attribute descriptions:
\begin{quote}
\index{label (vcs.isoline.Gi attribute)}

\begin{fulllineitems}
\phantomsection\label{vcs/graphics/isoline:vcs.isoline.Gi.label}\pysiglinewithargsret{\sphinxbfcode{label}}{\emph{str}}{}
\end{fulllineitems}


Turn on/off labels on isolines
\index{labelskipdistance (vcs.isoline.Gi attribute)}

\begin{fulllineitems}
\phantomsection\label{vcs/graphics/isoline:vcs.isoline.Gi.labelskipdistance}\pysiglinewithargsret{\sphinxbfcode{labelskipdistance}}{\emph{float}}{}
\end{fulllineitems}


Minimum distance between isoline labels
\index{labelbackgroundcolors (vcs.isoline.Gi attribute)}

\begin{fulllineitems}
\phantomsection\label{vcs/graphics/isoline:vcs.isoline.Gi.labelbackgroundcolors}\pysiglinewithargsret{\sphinxbfcode{labelbackgroundcolors}}{\sphinxoptional{\emph{float}}}{}
\end{fulllineitems}


Background color for isoline labels
\index{labelbackgroundopacities (vcs.isoline.Gi attribute)}

\begin{fulllineitems}
\phantomsection\label{vcs/graphics/isoline:vcs.isoline.Gi.labelbackgroundopacities}\pysiglinewithargsret{\sphinxbfcode{labelbackgroundopacities}}{\sphinxoptional{\emph{float}}}{}
\end{fulllineitems}


Background opacity for isoline labels
\index{level (vcs.isoline.Gi attribute)}

\begin{fulllineitems}
\phantomsection\label{vcs/graphics/isoline:vcs.isoline.Gi.level}\pysiglinewithargsret{\sphinxbfcode{level}}{\sphinxoptional{\emph{float}, \emph{...}}}{}
\end{fulllineitems}


Isocountours to display
\index{clockwise (vcs.isoline.Gi attribute)}

\begin{fulllineitems}
\phantomsection\label{vcs/graphics/isoline:vcs.isoline.Gi.clockwise}\pysiglinewithargsret{\sphinxbfcode{clockwise}}{\sphinxoptional{\emph{int}, \emph{...}}}{}
\end{fulllineitems}


Draw directional arrows
+-(0,1,2) Indicate none/clockwise/clokwise on y axis \textgreater{}0.
Clockwise on x axis positive negative value invert behaviour
\index{scale (vcs.isoline.Gi attribute)}

\begin{fulllineitems}
\phantomsection\label{vcs/graphics/isoline:vcs.isoline.Gi.scale}\pysiglinewithargsret{\sphinxbfcode{scale}}{\sphinxoptional{\emph{float}, \emph{...}}}{}
\end{fulllineitems}


Scales the directional arrow lengths
\index{angle (vcs.isoline.Gi attribute)}

\begin{fulllineitems}
\phantomsection\label{vcs/graphics/isoline:vcs.isoline.Gi.angle}\pysiglinewithargsret{\sphinxbfcode{angle}}{\sphinxoptional{\emph{float}, \emph{...}}}{}
\end{fulllineitems}


Directional arrows head angle
\index{spacing (vcs.isoline.Gi attribute)}

\begin{fulllineitems}
\phantomsection\label{vcs/graphics/isoline:vcs.isoline.Gi.spacing}\pysiglinewithargsret{\sphinxbfcode{spacing}}{\sphinxoptional{\emph{float}, \emph{...}}}{}
\end{fulllineitems}


Scales spacing between directional arrows
\index{xmtics1 (vcs.isoline.Gi attribute)}

\begin{fulllineitems}
\phantomsection\label{vcs/graphics/isoline:vcs.isoline.Gi.xmtics1}\pysiglinewithargsret{\sphinxbfcode{xmtics1}}{\emph{str/\{float:str\}}}{}
(Ex: `') dictionary with location of intermediate tics as keys for 1st side of y axis

\end{fulllineitems}

\index{xmtics2 (vcs.isoline.Gi attribute)}

\begin{fulllineitems}
\phantomsection\label{vcs/graphics/isoline:vcs.isoline.Gi.xmtics2}\pysiglinewithargsret{\sphinxbfcode{xmtics2}}{\emph{str/\{float:str\}}}{}
(Ex: `') dictionary with location of intermediate tics as keys for 2nd side of y axis

\end{fulllineitems}

\index{ymtics1 (vcs.isoline.Gi attribute)}

\begin{fulllineitems}
\phantomsection\label{vcs/graphics/isoline:vcs.isoline.Gi.ymtics1}\pysiglinewithargsret{\sphinxbfcode{ymtics1}}{\emph{str/\{float:str\}}}{}
(Ex: `') dictionary with location of intermediate tics as keys for 1st side of y axis

\end{fulllineitems}

\index{ymtics2 (vcs.isoline.Gi attribute)}

\begin{fulllineitems}
\phantomsection\label{vcs/graphics/isoline:vcs.isoline.Gi.ymtics2}\pysiglinewithargsret{\sphinxbfcode{ymtics2}}{\emph{str/\{float:str\}}}{}
(Ex: `') dictionary with location of intermediate tics as keys for 2nd side of y axis

\end{fulllineitems}

\index{xticlabels1 (vcs.isoline.Gi attribute)}

\begin{fulllineitems}
\phantomsection\label{vcs/graphics/isoline:vcs.isoline.Gi.xticlabels1}\pysiglinewithargsret{\sphinxbfcode{xticlabels1}}{\emph{str/\{float:str\}}}{}
(Ex: `*') values for labels on 1st side of x axis

\end{fulllineitems}

\index{xticlabels2 (vcs.isoline.Gi attribute)}

\begin{fulllineitems}
\phantomsection\label{vcs/graphics/isoline:vcs.isoline.Gi.xticlabels2}\pysiglinewithargsret{\sphinxbfcode{xticlabels2}}{\emph{str/\{float:str\}}}{}
(Ex: `*') values for labels on 2nd side of x axis

\end{fulllineitems}

\index{yticlabels1 (vcs.isoline.Gi attribute)}

\begin{fulllineitems}
\phantomsection\label{vcs/graphics/isoline:vcs.isoline.Gi.yticlabels1}\pysiglinewithargsret{\sphinxbfcode{yticlabels1}}{\emph{str/\{float:str\}}}{}
(Ex: `*') values for labels on 1st side of y axis

\end{fulllineitems}

\index{yticlabels2 (vcs.isoline.Gi attribute)}

\begin{fulllineitems}
\phantomsection\label{vcs/graphics/isoline:vcs.isoline.Gi.yticlabels2}\pysiglinewithargsret{\sphinxbfcode{yticlabels2}}{\emph{str/\{float:str\}}}{}
(Ex: `*') values for labels on 2nd side of y axis

\end{fulllineitems}

\index{projection (vcs.isoline.Gi attribute)}

\begin{fulllineitems}
\phantomsection\label{vcs/graphics/isoline:vcs.isoline.Gi.projection}\pysiglinewithargsret{\sphinxbfcode{projection}}{\emph{str/vcs.projection.Proj}}{}
(Ex: `default') projection to use, name or object

\end{fulllineitems}

\index{datawc\_x1 (vcs.isoline.Gi attribute)}

\begin{fulllineitems}
\phantomsection\label{vcs/graphics/isoline:vcs.isoline.Gi.datawc_x1}\pysiglinewithargsret{\sphinxbfcode{datawc\_x1}}{\emph{float}}{}
(Ex: 1.E20) first value of xaxis on plot

\end{fulllineitems}

\index{datawc\_x2 (vcs.isoline.Gi attribute)}

\begin{fulllineitems}
\phantomsection\label{vcs/graphics/isoline:vcs.isoline.Gi.datawc_x2}\pysiglinewithargsret{\sphinxbfcode{datawc\_x2}}{\emph{float}}{}
(Ex: 1.E20) second value of xaxis on plot

\end{fulllineitems}

\index{datawc\_y1 (vcs.isoline.Gi attribute)}

\begin{fulllineitems}
\phantomsection\label{vcs/graphics/isoline:vcs.isoline.Gi.datawc_y1}\pysiglinewithargsret{\sphinxbfcode{datawc\_y1}}{\emph{float}}{}
(Ex: 1.E20) first value of yaxis on plot

\end{fulllineitems}

\index{datawc\_y2 (vcs.isoline.Gi attribute)}

\begin{fulllineitems}
\phantomsection\label{vcs/graphics/isoline:vcs.isoline.Gi.datawc_y2}\pysiglinewithargsret{\sphinxbfcode{datawc\_y2}}{\emph{float}}{}
(Ex: 1.E20) second value of yaxis on plot

\end{fulllineitems}

\index{datawc\_timeunits (vcs.isoline.Gi attribute)}

\begin{fulllineitems}
\phantomsection\label{vcs/graphics/isoline:vcs.isoline.Gi.datawc_timeunits}\pysiglinewithargsret{\sphinxbfcode{datawc\_timeunits}}{\emph{str}}{}
(Ex: `days since 2000') units to use when displaying time dimension auto tick

\end{fulllineitems}

\index{datawc\_calendar (vcs.isoline.Gi attribute)}

\begin{fulllineitems}
\phantomsection\label{vcs/graphics/isoline:vcs.isoline.Gi.datawc_calendar}\pysiglinewithargsret{\sphinxbfcode{datawc\_calendar}}{\emph{int}}{}
(Ex: 135441) calendar to use when displaying time dimension auto tick, default is proleptic gregorian calendar

line :: ({[}str,...{]}/{[}vcs.line.Tl,...{]}/{[}int,...{]}) ({[}'solid',{]}) line type to use for each isoline, can also pass a line object or line object name

\end{fulllineitems}

\end{quote}

\end{itemize}

\end{fulllineitems}


linecolors :: ({[}int,...{]}) ({[}241{]}) colors to use for each isoline
linewidths :: ({[}float,...{]}) ({[}1.0{]}) list of width for each isoline

text :: (None/{[}vcs.textcombined.Tc,...{]}) (None) text objects or text objects names to use for each countour labels
textcolors :: (None/{[}int,...{]}) (None) colors to use for each countour labels
\index{datawc() (vcs.isoline.Gi method)}

\begin{fulllineitems}
\phantomsection\label{vcs/graphics/isoline:vcs.isoline.Gi.datawc}\pysiglinewithargsret{\sphinxbfcode{datawc}}{\emph{dsp1=1e+20}, \emph{dsp2=1e+20}, \emph{dsp3=1e+20}, \emph{dsp4=1e+20}}{}
Sets the data world coordinates for object
\begin{quote}\begin{description}
\item[{Parameters}] \leavevmode\begin{itemize}
\item {} 
\textbf{\texttt{dsp1}} (\href{https://docs.python.org/2/library/functions.html\#float}{\emph{\texttt{float}}}) -- Sets the {\hyperref[vcs/graphics/isoline:vcs.isoline.Gi.datawc_y1]{\sphinxcrossref{\sphinxcode{datawc\_y1}}}} property of the object.

\item {} 
\textbf{\texttt{dsp2}} (\href{https://docs.python.org/2/library/functions.html\#float}{\emph{\texttt{float}}}) -- Sets the {\hyperref[vcs/graphics/isoline:vcs.isoline.Gi.datawc_y2]{\sphinxcrossref{\sphinxcode{datawc\_y2}}}} property of the object.

\item {} 
\textbf{\texttt{dsp3}} (\href{https://docs.python.org/2/library/functions.html\#float}{\emph{\texttt{float}}}) -- Sets the {\hyperref[vcs/graphics/isoline:vcs.isoline.Gi.datawc_x1]{\sphinxcrossref{\sphinxcode{datawc\_x1}}}} property of the object.

\item {} 
\textbf{\texttt{dsp4}} (\href{https://docs.python.org/2/library/functions.html\#float}{\emph{\texttt{float}}}) -- Sets the {\hyperref[vcs/graphics/isoline:vcs.isoline.Gi.datawc_x2]{\sphinxcrossref{\sphinxcode{datawc\_x2}}}} property of the object.

\end{itemize}

\end{description}\end{quote}

\end{fulllineitems}

\index{list() (vcs.isoline.Gi method)}

\begin{fulllineitems}
\phantomsection\label{vcs/graphics/isoline:vcs.isoline.Gi.list}\pysiglinewithargsret{\sphinxbfcode{list}}{}{}
Lists the current values of object attributes

\end{fulllineitems}

\index{script() (vcs.isoline.Gi method)}

\begin{fulllineitems}
\phantomsection\label{vcs/graphics/isoline:vcs.isoline.Gi.script}\pysiglinewithargsret{\sphinxbfcode{script}}{\emph{script\_filename}, \emph{mode='a'}}{}
Saves out a copy of the isoline graphics method in JSON, or Python format to a designated file.
\begin{quote}

\begin{notice}{note}{Note:}
If the the filename has a `.py' at the end, it will produce a
Python script. If no extension is given, then by default a
.json file containing a JSON serialization of the object's
data will be produced.
\end{notice}

\begin{notice}{warning}{Warning:}
VCS Scripts Deprecated.
SCR script files are no longer generated by this function.
\end{notice}
\end{quote}
\begin{quote}\begin{description}
\item[{Example}] \leavevmode
\begin{Verbatim}[commandchars=\\\{\}]
\PYG{g+gp}{\PYGZgt{}\PYGZgt{}\PYGZgt{} }\PYG{n}{a}\PYG{o}{=}\PYG{n}{vcs}\PYG{o}{.}\PYG{n}{init}\PYG{p}{(}\PYG{p}{)} \PYG{c+c1}{\PYGZsh{} Make a Canvas object to work with}
\PYG{g+gp}{\PYGZgt{}\PYGZgt{}\PYGZgt{} }\PYG{n}{ex}\PYG{o}{=}\PYG{n}{a}\PYG{o}{.}\PYG{n}{getisoline}\PYG{p}{(}\PYG{p}{)} \PYG{c+c1}{\PYGZsh{} Get default isoline}
\PYG{g+gp}{\PYGZgt{}\PYGZgt{}\PYGZgt{} }\PYG{n}{ex}\PYG{o}{.}\PYG{n}{script}\PYG{p}{(}\PYG{l+s+s1}{\PYGZsq{}}\PYG{l+s+s1}{filename.py}\PYG{l+s+s1}{\PYGZsq{}}\PYG{p}{)} \PYG{c+c1}{\PYGZsh{} Append to a Python script named \PYGZsq{}filename.py\PYGZsq{}}
\PYG{g+gp}{\PYGZgt{}\PYGZgt{}\PYGZgt{} }\PYG{n}{ex}\PYG{o}{.}\PYG{n}{script}\PYG{p}{(}\PYG{l+s+s1}{\PYGZsq{}}\PYG{l+s+s1}{filename}\PYG{l+s+s1}{\PYGZsq{}}\PYG{p}{,}\PYG{l+s+s1}{\PYGZsq{}}\PYG{l+s+s1}{w}\PYG{l+s+s1}{\PYGZsq{}}\PYG{p}{)} \PYG{c+c1}{\PYGZsh{} Create or overwrite a JSON file \PYGZsq{}filename.json\PYGZsq{}.}
\end{Verbatim}

\item[{Parameters}] \leavevmode\begin{itemize}
\item {} 
\textbf{\texttt{script\_filename}} (\href{https://docs.python.org/2/library/functions.html\#str}{\emph{\texttt{str}}}) -- Output name of the script file. If no extension is specified, a .json object is created.

\item {} 
\textbf{\texttt{mode}} (\href{https://docs.python.org/2/library/functions.html\#str}{\emph{\texttt{str}}}) -- Either `w' for replace, or `a' for append. Defaults to `a', if not specified.

\end{itemize}

\end{description}\end{quote}

\end{fulllineitems}

\index{xmtics() (vcs.isoline.Gi method)}

\begin{fulllineitems}
\phantomsection\label{vcs/graphics/isoline:vcs.isoline.Gi.xmtics}\pysiglinewithargsret{\sphinxbfcode{xmtics}}{\emph{xmt1='`}, \emph{xmt2='`}}{}
Sets the xmtics1 and xmtics2 values on the object
\begin{quote}\begin{description}
\item[{Parameters}] \leavevmode\begin{itemize}
\item {} 
\textbf{\texttt{xmt1}} (\emph{\texttt{\{float:str\} or str}}) -- Value for {\hyperref[vcs/graphics/isoline:vcs.isoline.Gi.xmtics1]{\sphinxcrossref{\sphinxcode{xmtics1}}}}. Must be a str, or a dictionary object with float:str mappings.

\item {} 
\textbf{\texttt{xmt2}} (\emph{\texttt{\{float:str\} or str}}) -- Value for {\hyperref[vcs/graphics/isoline:vcs.isoline.Gi.xmtics2]{\sphinxcrossref{\sphinxcode{xmtics2}}}}. Must be a str, or a dictionary object with float:str mappings.

\end{itemize}

\end{description}\end{quote}

\end{fulllineitems}

\index{xticlabels() (vcs.isoline.Gi method)}

\begin{fulllineitems}
\phantomsection\label{vcs/graphics/isoline:vcs.isoline.Gi.xticlabels}\pysiglinewithargsret{\sphinxbfcode{xticlabels}}{\emph{xtl1='`}, \emph{xtl2='`}}{}
Sets the xticlabels1 and xticlabels2 values on the object
\begin{quote}\begin{description}
\item[{Parameters}] \leavevmode\begin{itemize}
\item {} 
\textbf{\texttt{xtl1}} (\emph{\texttt{\{float:str\} or str}}) -- Sets the object's value for {\hyperref[vcs/graphics/isoline:vcs.isoline.Gi.xticlabels1]{\sphinxcrossref{\sphinxcode{xticlabels1}}}}. Must be  a str, or a dictionary object with float:str mappings.

\item {} 
\textbf{\texttt{xtl2}} (\emph{\texttt{\{float:str\} or str}}) -- Sets the object's value for {\hyperref[vcs/graphics/isoline:vcs.isoline.Gi.xticlabels2]{\sphinxcrossref{\sphinxcode{xticlabels2}}}}. Must be a str, or a dictionary object with float:str mappings.

\end{itemize}

\end{description}\end{quote}

\end{fulllineitems}

\index{xyscale() (vcs.isoline.Gi method)}

\begin{fulllineitems}
\phantomsection\label{vcs/graphics/isoline:vcs.isoline.Gi.xyscale}\pysiglinewithargsret{\sphinxbfcode{xyscale}}{\emph{xat='`}, \emph{yat='`}}{}
Sets xaxisconvert and yaxisconvert values for the object.
\begin{quote}\begin{description}
\item[{Example}] \leavevmode
\begin{Verbatim}[commandchars=\\\{\}]
\PYG{g+gp}{\PYGZgt{}\PYGZgt{}\PYGZgt{} }\PYG{n}{a}\PYG{o}{=}\PYG{n}{vcs}\PYG{o}{.}\PYG{n}{init}\PYG{p}{(}\PYG{p}{)}
\PYG{g+gp}{\PYGZgt{}\PYGZgt{}\PYGZgt{} }\PYG{n}{ex}\PYG{o}{=}\PYG{n}{a}\PYG{o}{.}\PYG{n}{createisoline}\PYG{p}{(}\PYG{l+s+s1}{\PYGZsq{}}\PYG{l+s+s1}{xyscale\PYGZus{}ex}\PYG{l+s+s1}{\PYGZsq{}}\PYG{p}{)} \PYG{c+c1}{\PYGZsh{} create a boxfill to work with}
\PYG{g+gp}{\PYGZgt{}\PYGZgt{}\PYGZgt{} }\PYG{n}{ex}\PYG{o}{.}\PYG{n}{xyscale}\PYG{p}{(}\PYG{n}{xat}\PYG{o}{=}\PYG{l+s+s1}{\PYGZsq{}}\PYG{l+s+s1}{linear}\PYG{l+s+s1}{\PYGZsq{}}\PYG{p}{,} \PYG{n}{yat}\PYG{o}{=}\PYG{l+s+s1}{\PYGZsq{}}\PYG{l+s+s1}{linear}\PYG{l+s+s1}{\PYGZsq{}}\PYG{p}{)} \PYG{c+c1}{\PYGZsh{} set xaxisconvert and yaxisconvert to \PYGZsq{}linear\PYGZsq{}}
\end{Verbatim}

\item[{Parameters}] \leavevmode\begin{itemize}
\item {} 
\textbf{\texttt{xat}} (\href{https://docs.python.org/2/library/functions.html\#str}{\emph{\texttt{str}}}) -- Set value for x axis conversion.

\item {} 
\textbf{\texttt{yat}} (\href{https://docs.python.org/2/library/functions.html\#str}{\emph{\texttt{str}}}) -- Set value for y axis conversion.

\end{itemize}

\end{description}\end{quote}

\end{fulllineitems}

\index{ymtics() (vcs.isoline.Gi method)}

\begin{fulllineitems}
\phantomsection\label{vcs/graphics/isoline:vcs.isoline.Gi.ymtics}\pysiglinewithargsret{\sphinxbfcode{ymtics}}{\emph{ymt1='`}, \emph{ymt2='`}}{}
Sets the ymtics1 and ymtics2 values on the object
\begin{quote}\begin{description}
\item[{Parameters}] \leavevmode\begin{itemize}
\item {} 
\textbf{\texttt{ymt1}} (\emph{\texttt{\{float:str\} or str}}) -- Value for {\hyperref[vcs/graphics/isoline:vcs.isoline.Gi.ymtics1]{\sphinxcrossref{\sphinxcode{ymtics1}}}}. Must be a str, or a dictionary object with float:str mappings.

\item {} 
\textbf{\texttt{ymt2}} (\emph{\texttt{\{float:str\} or str}}) -- Value for {\hyperref[vcs/graphics/isoline:vcs.isoline.Gi.ymtics2]{\sphinxcrossref{\sphinxcode{ymtics2}}}}. Must be a str, or a dictionary object with float:str mappings.

\end{itemize}

\end{description}\end{quote}

\end{fulllineitems}

\index{yticlabels() (vcs.isoline.Gi method)}

\begin{fulllineitems}
\phantomsection\label{vcs/graphics/isoline:vcs.isoline.Gi.yticlabels}\pysiglinewithargsret{\sphinxbfcode{yticlabels}}{\emph{ytl1='`}, \emph{ytl2='`}}{}
Sets the yticlabels1 and yticlabels2 values on the object
\begin{quote}\begin{description}
\item[{Parameters}] \leavevmode\begin{itemize}
\item {} 
\textbf{\texttt{ytl1}} (\emph{\texttt{\{float:str\} or str}}) -- Sets the object's value for {\hyperref[vcs/graphics/isoline:vcs.isoline.Gi.yticlabels1]{\sphinxcrossref{\sphinxcode{yticlabels1}}}}. Must be  a str, or a dictionary object with float:str mappings.

\item {} 
\textbf{\texttt{ytl2}} (\emph{\texttt{\{float:str\} or str}}) -- Sets the object's value for {\hyperref[vcs/graphics/isoline:vcs.isoline.Gi.yticlabels2]{\sphinxcrossref{\sphinxcode{yticlabels2}}}}. Must be a str, or a dictionary object with float:str mappings.

\end{itemize}

\end{description}\end{quote}

\end{fulllineitems}


\end{fulllineitems}



\subsection{meshfill}
\label{vcs/graphics/meshfill:meshfill}\label{vcs/graphics/meshfill:module-vcs.meshfill}\label{vcs/graphics/meshfill::doc}\index{vcs.meshfill (module)}
\# Meshfill (Gfm) module
\index{Gfm (class in vcs.meshfill)}

\begin{fulllineitems}
\phantomsection\label{vcs/graphics/meshfill:vcs.meshfill.Gfm}\pysiglinewithargsret{\sphinxstrong{class }\sphinxcode{vcs.meshfill.}\sphinxbfcode{Gfm}}{\emph{Gfm\_name}, \emph{Gfm\_name\_src='default'}}{}
The meshfill graphics method (Gfm) displays a two-dimensional data array
by surrounding each data value by a colored grid mesh.

This class is used to define a meshfill table entry used in VCS, or it
can be used to change some or all of the attributes in an existing
meshfill table entry.


\begin{fulllineitems}
\pysigline{\sphinxbfcode{Useful~Functions:}}~
\begin{Verbatim}[commandchars=\\\{\}]
\PYG{c+c1}{\PYGZsh{} VCS Canvas Constructor}
\PYG{n}{a}\PYG{o}{=}\PYG{n}{vcs}\PYG{o}{.}\PYG{n}{init}\PYG{p}{(}\PYG{p}{)}
\PYG{c+c1}{\PYGZsh{} Show predefined meshfill graphics methods}
\PYG{n}{a}\PYG{o}{.}\PYG{n}{show}\PYG{p}{(}\PYG{l+s+s1}{\PYGZsq{}}\PYG{l+s+s1}{meshfill}\PYG{l+s+s1}{\PYGZsq{}}\PYG{p}{)}
\PYG{c+c1}{\PYGZsh{} Change the VCS color map}
\PYG{n}{a}\PYG{o}{.}\PYG{n}{setcolormap}\PYG{p}{(}\PYG{l+s+s2}{\PYGZdq{}}\PYG{l+s+s2}{AMIP}\PYG{l+s+s2}{\PYGZdq{}}\PYG{p}{)}
\PYG{c+c1}{\PYGZsh{} Plot data \PYGZsq{}s\PYGZsq{} with meshfill \PYGZsq{}b\PYGZsq{} and \PYGZsq{}default\PYGZsq{} template}
\PYG{n}{a}\PYG{o}{.}\PYG{n}{meshfill}\PYG{p}{(}\PYG{n}{s}\PYG{p}{,}\PYG{n}{b}\PYG{p}{,}\PYG{l+s+s1}{\PYGZsq{}}\PYG{l+s+s1}{default}\PYG{l+s+s1}{\PYGZsq{}}\PYG{p}{)}
\PYG{c+c1}{\PYGZsh{} Updates the VCS Canvas at user\PYGZsq{}s request}
\PYG{n}{a}\PYG{o}{.}\PYG{n}{update}\PYG{p}{(}\PYG{p}{)}
\end{Verbatim}

\end{fulllineitems}



\begin{fulllineitems}
\pysigline{\sphinxbfcode{Create~a~new~instance~of~meshfill:}}~
\begin{Verbatim}[commandchars=\\\{\}]
\PYG{c+c1}{\PYGZsh{} Copies content of \PYGZsq{}quick\PYGZsq{} to \PYGZsq{}new\PYGZsq{}}
\PYG{n}{mesh}\PYG{o}{=}\PYG{n}{a}\PYG{o}{.}\PYG{n}{createmeshfill}\PYG{p}{(}\PYG{l+s+s1}{\PYGZsq{}}\PYG{l+s+s1}{new}\PYG{l+s+s1}{\PYGZsq{}}\PYG{p}{,}\PYG{l+s+s1}{\PYGZsq{}}\PYG{l+s+s1}{quick}\PYG{l+s+s1}{\PYGZsq{}}\PYG{p}{)}
\PYG{c+c1}{\PYGZsh{} Copies content of \PYGZsq{}default\PYGZsq{} to \PYGZsq{}new\PYGZsq{}}
\PYG{n}{mesh}\PYG{o}{=}\PYG{n}{a}\PYG{o}{.}\PYG{n}{createmeshfill}\PYG{p}{(}\PYG{l+s+s1}{\PYGZsq{}}\PYG{l+s+s1}{new}\PYG{l+s+s1}{\PYGZsq{}}\PYG{p}{)}
\end{Verbatim}

\end{fulllineitems}



\begin{fulllineitems}
\pysigline{\sphinxbfcode{Modify~an~existing~meshfill:}}~
\begin{Verbatim}[commandchars=\\\{\}]
\PYG{n}{mesh}\PYG{o}{=}\PYG{n}{a}\PYG{o}{.}\PYG{n}{getmeshfill}\PYG{p}{(}\PYG{l+s+s1}{\PYGZsq{}}\PYG{l+s+s1}{AMIP\PYGZus{}psl}\PYG{l+s+s1}{\PYGZsq{}}\PYG{p}{)}
\end{Verbatim}

\end{fulllineitems}



\begin{fulllineitems}
\pysigline{\sphinxbfcode{Overview~of~meshfill~object~attributes:}}~
\begin{Verbatim}[commandchars=\\\{\}]

\end{Verbatim}
\begin{itemize}
\item {} 
List all the meshfill attribute values
\begin{quote}

\begin{Verbatim}[commandchars=\\\{\}]
\PYG{n}{mesh}\PYG{o}{.}\PYG{n}{list}\PYG{p}{(}\PYG{p}{)}
\end{Verbatim}
\end{quote}

\item {} 
Setting attributes:
\begin{itemize}
\item {} 
Setting general attributes:
\begin{quote}

\begin{Verbatim}[commandchars=\\\{\}]
\PYG{n}{mesh}\PYG{o}{.}\PYG{n}{projection}\PYG{o}{=}\PYG{l+s+s1}{\PYGZsq{}}\PYG{l+s+s1}{linear}\PYG{l+s+s1}{\PYGZsq{}}
\PYG{n}{lon30}\PYG{o}{=}\PYG{p}{\PYGZob{}}\PYG{o}{\PYGZhy{}}\PYG{l+m+mi}{180}\PYG{p}{:}\PYG{l+s+s1}{\PYGZsq{}}\PYG{l+s+s1}{180W}\PYG{l+s+s1}{\PYGZsq{}}\PYG{p}{,}\PYG{o}{\PYGZhy{}}\PYG{l+m+mi}{150}\PYG{p}{:}\PYG{l+s+s1}{\PYGZsq{}}\PYG{l+s+s1}{150W}\PYG{l+s+s1}{\PYGZsq{}}\PYG{p}{,}\PYG{l+m+mi}{0}\PYG{p}{:}\PYG{l+s+s1}{\PYGZsq{}}\PYG{l+s+s1}{Eq}\PYG{l+s+s1}{\PYGZsq{}}\PYG{p}{\PYGZcb{}}
\PYG{n}{mesh}\PYG{o}{.}\PYG{n}{xticlabels1}\PYG{o}{=}\PYG{n}{lon30}
\PYG{n}{mesh}\PYG{o}{.}\PYG{n}{xticlabels2}\PYG{o}{=}\PYG{n}{lon30}
\PYG{c+c1}{\PYGZsh{} Will set them both}
\PYG{n}{mesh}\PYG{o}{.}\PYG{n}{xticlabels}\PYG{p}{(}\PYG{n}{lon30}\PYG{p}{,} \PYG{n}{lon30}\PYG{p}{)}
\PYG{n}{mesh}\PYG{o}{.}\PYG{n}{xmtics1}\PYG{o}{=}\PYG{l+s+s1}{\PYGZsq{}}\PYG{l+s+s1}{\PYGZsq{}}
\PYG{n}{mesh}\PYG{o}{.}\PYG{n}{xmtics2}\PYG{o}{=}\PYG{l+s+s1}{\PYGZsq{}}\PYG{l+s+s1}{\PYGZsq{}}
\PYG{c+c1}{\PYGZsh{} Will set them both}
\PYG{n}{mesh}\PYG{o}{.}\PYG{n}{xmtics}\PYG{p}{(}\PYG{n}{lon30}\PYG{p}{,} \PYG{n}{lon30}\PYG{p}{)}
\PYG{n}{mesh}\PYG{o}{.}\PYG{n}{yticlabels1}\PYG{o}{=}\PYG{n}{lat10}
\PYG{n}{mesh}\PYG{o}{.}\PYG{n}{yticlabels2}\PYG{o}{=}\PYG{n}{lat10}
\PYG{c+c1}{\PYGZsh{} Will set them both}
\PYG{n}{mesh}\PYG{o}{.}\PYG{n}{yticlabels}\PYG{p}{(}\PYG{n}{lat10}\PYG{p}{,} \PYG{n}{lat10}\PYG{p}{)}
\PYG{n}{mesh}\PYG{o}{.}\PYG{n}{ymtics1}\PYG{o}{=}\PYG{l+s+s1}{\PYGZsq{}}\PYG{l+s+s1}{\PYGZsq{}}
\PYG{n}{mesh}\PYG{o}{.}\PYG{n}{ymtics2}\PYG{o}{=}\PYG{l+s+s1}{\PYGZsq{}}\PYG{l+s+s1}{\PYGZsq{}}
\PYG{c+c1}{\PYGZsh{} Will set them both}
\PYG{n}{mesh}\PYG{o}{.}\PYG{n}{ymtics}\PYG{p}{(}\PYG{n}{lat10}\PYG{p}{,} \PYG{n}{lat10}\PYG{p}{)}
\PYG{n}{mesh}\PYG{o}{.}\PYG{n}{datawc\PYGZus{}y1}\PYG{o}{=}\PYG{o}{\PYGZhy{}}\PYG{l+m+mf}{90.0}
\PYG{n}{mesh}\PYG{o}{.}\PYG{n}{datawc\PYGZus{}y2}\PYG{o}{=}\PYG{l+m+mf}{90.0}
\PYG{n}{mesh}\PYG{o}{.}\PYG{n}{datawc\PYGZus{}x1}\PYG{o}{=}\PYG{o}{\PYGZhy{}}\PYG{l+m+mf}{180.0}
\PYG{n}{mesh}\PYG{o}{.}\PYG{n}{datawc\PYGZus{}x2}\PYG{o}{=}\PYG{l+m+mf}{180.0}
\PYG{c+c1}{\PYGZsh{} Will set them all}
\PYG{n}{mesh}\PYG{o}{.}\PYG{n}{datawc}\PYG{p}{(}\PYG{o}{\PYGZhy{}}\PYG{l+m+mi}{90}\PYG{p}{,} \PYG{l+m+mi}{90}\PYG{p}{,} \PYG{o}{\PYGZhy{}}\PYG{l+m+mi}{180}\PYG{p}{,} \PYG{l+m+mi}{180}\PYG{p}{)}
\PYG{n}{mesh}\PYG{o}{.}\PYG{n}{ext\PYGZus{}1}\PYG{o}{=}\PYG{l+s+s1}{\PYGZsq{}}\PYG{l+s+s1}{n}\PYG{l+s+s1}{\PYGZsq{}}
\PYG{n}{mesh}\PYG{o}{.}\PYG{n}{ext\PYGZus{}2}\PYG{o}{=}\PYG{l+s+s1}{\PYGZsq{}}\PYG{l+s+s1}{y}\PYG{l+s+s1}{\PYGZsq{}}
\PYG{c+c1}{\PYGZsh{} Will set them both}
\PYG{n}{mesh}\PYG{o}{.}\PYG{n}{exts}\PYG{p}{(}\PYG{l+s+s1}{\PYGZsq{}}\PYG{l+s+s1}{n}\PYG{l+s+s1}{\PYGZsq{}}\PYG{p}{,} \PYG{l+s+s1}{\PYGZsq{}}\PYG{l+s+s1}{y}\PYG{l+s+s1}{\PYGZsq{}} \PYG{p}{)}
\PYG{c+c1}{\PYGZsh{} Color index value range 0 to 255}
\PYG{n}{mesh}\PYG{o}{.}\PYG{n}{missing}\PYG{o}{=}\PYG{l+m+mi}{241}
\end{Verbatim}
\end{quote}

\item {} 
There are two possibilities for setting meshfill levels:
\begin{enumerate}
\item {} 
Levels are all contiguous:
\begin{quote}

\begin{Verbatim}[commandchars=\\\{\}]
\PYG{n}{mesh}\PYG{o}{.}\PYG{n}{levels}\PYG{o}{=}\PYG{p}{(}\PYG{p}{[}\PYG{l+m+mi}{0}\PYG{p}{,}\PYG{l+m+mi}{20}\PYG{p}{,}\PYG{l+m+mi}{25}\PYG{p}{,}\PYG{l+m+mi}{30}\PYG{p}{,}\PYG{l+m+mi}{35}\PYG{p}{,}\PYG{l+m+mi}{40}\PYG{p}{]}\PYG{p}{,}\PYG{p}{)}
\PYG{n}{mesh}\PYG{o}{.}\PYG{n}{levels}\PYG{o}{=}\PYG{p}{(}\PYG{p}{[}\PYG{l+m+mi}{0}\PYG{p}{,}\PYG{l+m+mi}{20}\PYG{p}{,}\PYG{l+m+mi}{25}\PYG{p}{,}\PYG{l+m+mi}{30}\PYG{p}{,}\PYG{l+m+mi}{35}\PYG{p}{,}\PYG{l+m+mi}{40}\PYG{p}{,}\PYG{l+m+mi}{45}\PYG{p}{,}\PYG{l+m+mi}{50}\PYG{p}{]}\PYG{p}{)}
\PYG{n}{mesh}\PYG{o}{.}\PYG{n}{levels}\PYG{o}{=}\PYG{p}{[}\PYG{l+m+mi}{0}\PYG{p}{,}\PYG{l+m+mi}{20}\PYG{p}{,}\PYG{l+m+mi}{25}\PYG{p}{,}\PYG{l+m+mi}{30}\PYG{p}{,}\PYG{l+m+mi}{35}\PYG{p}{,}\PYG{l+m+mi}{40}\PYG{p}{]}
\PYG{n}{mesh}\PYG{o}{.}\PYG{n}{levels}\PYG{o}{=}\PYG{p}{(}\PYG{l+m+mf}{0.0}\PYG{p}{,}\PYG{l+m+mf}{20.0}\PYG{p}{,}\PYG{l+m+mf}{25.0}\PYG{p}{,}\PYG{l+m+mf}{30.0}\PYG{p}{,}\PYG{l+m+mf}{35.0}\PYG{p}{,}\PYG{l+m+mf}{40.0}\PYG{p}{,}\PYG{l+m+mf}{50.0}\PYG{p}{)}
\end{Verbatim}
\end{quote}

\item {} 
Levels are not contiguous (Examples):
\begin{quote}

\begin{Verbatim}[commandchars=\\\{\}]
\PYG{n}{mesh}\PYG{o}{.}\PYG{n}{levels}\PYG{o}{=}\PYG{p}{(}\PYG{p}{[}\PYG{l+m+mi}{0}\PYG{p}{,}\PYG{l+m+mi}{20}\PYG{p}{]}\PYG{p}{,}\PYG{p}{[}\PYG{l+m+mi}{30}\PYG{p}{,}\PYG{l+m+mi}{40}\PYG{p}{]}\PYG{p}{,}\PYG{p}{[}\PYG{l+m+mi}{50}\PYG{p}{,}\PYG{l+m+mi}{60}\PYG{p}{]}\PYG{p}{)}
\PYG{n}{mesh}\PYG{o}{.}\PYG{n}{levels}\PYG{o}{=}\PYG{p}{(}\PYG{p}{[}\PYG{l+m+mi}{0}\PYG{p}{,}\PYG{l+m+mi}{20}\PYG{p}{,}\PYG{l+m+mi}{25}\PYG{p}{,}\PYG{l+m+mi}{30}\PYG{p}{,}\PYG{l+m+mi}{35}\PYG{p}{,}\PYG{l+m+mi}{40}\PYG{p}{]}\PYG{p}{,}\PYG{p}{[}\PYG{l+m+mi}{30}\PYG{p}{,}\PYG{l+m+mi}{40}\PYG{p}{]}\PYG{p}{,}\PYG{p}{[}\PYG{l+m+mi}{50}\PYG{p}{,}\PYG{l+m+mi}{60}\PYG{p}{]}\PYG{p}{)}
\end{Verbatim}
\end{quote}

\end{enumerate}

\item {} 
There are three ways to set fillarea color indices:
\begin{quote}

\begin{Verbatim}[commandchars=\\\{\}]
\PYG{n}{mesh}\PYG{o}{.}\PYG{n}{fillareacolors}\PYG{o}{=}\PYG{p}{(}\PYG{p}{[}\PYG{l+m+mi}{22}\PYG{p}{,}\PYG{l+m+mi}{33}\PYG{p}{,}\PYG{l+m+mi}{44}\PYG{p}{,}\PYG{l+m+mi}{55}\PYG{p}{,}\PYG{l+m+mi}{66}\PYG{p}{,}\PYG{l+m+mi}{77}\PYG{p}{]}\PYG{p}{)}
\PYG{n}{mesh}\PYG{o}{.}\PYG{n}{fillareacolors}\PYG{o}{=}\PYG{p}{(}\PYG{l+m+mi}{16}\PYG{p}{,}\PYG{l+m+mi}{19}\PYG{p}{,}\PYG{l+m+mi}{33}\PYG{p}{,}\PYG{l+m+mi}{44}\PYG{p}{)}
\PYG{n}{mesh}\PYG{o}{.}\PYG{n}{fillareacolors}\PYG{o}{=}\PYG{n+nb+bp}{None}
\end{Verbatim}
\end{quote}

\item {} 
There are three ways to set fillarea style:
\begin{quote}

\begin{Verbatim}[commandchars=\\\{\}]
\PYG{n}{mesh}\PYG{o}{.}\PYG{n}{fillareastyle} \PYG{o}{=} \PYG{l+s+s1}{\PYGZsq{}}\PYG{l+s+s1}{solid}\PYG{l+s+s1}{\PYGZsq{}}
\PYG{n}{mesh}\PYG{o}{.}\PYG{n}{fillareastyle} \PYG{o}{=} \PYG{l+s+s1}{\PYGZsq{}}\PYG{l+s+s1}{hatch}\PYG{l+s+s1}{\PYGZsq{}}
\PYG{n}{mesh}\PYG{o}{.}\PYG{n}{fillareastyle} \PYG{o}{=} \PYG{l+s+s1}{\PYGZsq{}}\PYG{l+s+s1}{pattern}\PYG{l+s+s1}{\PYGZsq{}}
\end{Verbatim}
\end{quote}

\item {} 
There are two ways to set fillarea hatch or pattern indices:
\begin{quote}

\begin{Verbatim}[commandchars=\\\{\}]
\PYG{n}{mesh}\PYG{o}{.}\PYG{n}{fillareaindices}\PYG{o}{=}\PYG{p}{(}\PYG{p}{[}\PYG{l+m+mi}{1}\PYG{p}{,}\PYG{l+m+mi}{3}\PYG{p}{,}\PYG{l+m+mi}{5}\PYG{p}{,}\PYG{l+m+mi}{6}\PYG{p}{,}\PYG{l+m+mi}{9}\PYG{p}{,}\PYG{l+m+mi}{20}\PYG{p}{]}\PYG{p}{)}
\PYG{n}{mesh}\PYG{o}{.}\PYG{n}{fillareaindices}\PYG{o}{=}\PYG{p}{(}\PYG{l+m+mi}{7}\PYG{p}{,}\PYG{l+m+mi}{1}\PYG{p}{,}\PYG{l+m+mi}{4}\PYG{p}{,}\PYG{l+m+mi}{9}\PYG{p}{,}\PYG{l+m+mi}{6}\PYG{p}{,}\PYG{l+m+mi}{15}\PYG{p}{)}
\end{Verbatim}
\end{quote}

\end{itemize}

\end{itemize}

\end{fulllineitems}



\begin{fulllineitems}
\pysigline{\sphinxbfcode{Using~the~fillarea~secondary~object:}}~\begin{itemize}
\item {} 
Create a new instance of fillarea:
\begin{quote}

\begin{Verbatim}[commandchars=\\\{\}]
\PYG{c+c1}{\PYGZsh{} Copies \PYGZsq{}quick\PYGZsq{} to \PYGZsq{}new\PYGZsq{}}
\PYG{n}{fill}\PYG{o}{=}\PYG{n}{a}\PYG{o}{.}\PYG{n}{createfillarea}\PYG{p}{(}\PYG{l+s+s1}{\PYGZsq{}}\PYG{l+s+s1}{new}\PYG{l+s+s1}{\PYGZsq{}}\PYG{p}{,}\PYG{l+s+s1}{\PYGZsq{}}\PYG{l+s+s1}{quick}\PYG{l+s+s1}{\PYGZsq{}}\PYG{p}{)}
\PYG{c+c1}{\PYGZsh{} Copies \PYGZsq{}default\PYGZsq{} to \PYGZsq{}new\PYGZsq{}}
\PYG{n}{fill}\PYG{o}{=}\PYG{n}{a}\PYG{o}{.}\PYG{n}{createfillarea}\PYG{p}{(}\PYG{l+s+s1}{\PYGZsq{}}\PYG{l+s+s1}{new}\PYG{l+s+s1}{\PYGZsq{}}\PYG{p}{)}
\end{Verbatim}
\end{quote}

\item {} 
Modify an existing fillarea:
\begin{quote}

\begin{Verbatim}[commandchars=\\\{\}]
\PYG{n}{fill}\PYG{o}{=}\PYG{n}{a}\PYG{o}{.}\PYG{n}{getmfillarea}\PYG{p}{(}\PYG{l+s+s1}{\PYGZsq{}}\PYG{l+s+s1}{def37}\PYG{l+s+s1}{\PYGZsq{}}\PYG{p}{)}
\PYG{c+c1}{\PYGZsh{} Set index using fillarea}
\PYG{n}{mesh}\PYG{o}{.}\PYG{n}{fillareaindices}\PYG{o}{=}\PYG{p}{(}\PYG{l+m+mi}{7}\PYG{p}{,}\PYG{n}{fill}\PYG{p}{,}\PYG{l+m+mi}{4}\PYG{p}{,}\PYG{l+m+mi}{9}\PYG{p}{,}\PYG{n}{fill}\PYG{p}{,}\PYG{l+m+mi}{15}\PYG{p}{)}
\PYG{c+c1}{\PYGZsh{} list fillarea attributes}
\PYG{n}{fill}\PYG{o}{.}\PYG{n}{list}\PYG{p}{(}\PYG{p}{)}
\PYG{c+c1}{\PYGZsh{} change style}
\PYG{n}{fill}\PYG{o}{.}\PYG{n}{style}\PYG{o}{=}\PYG{l+s+s1}{\PYGZsq{}}\PYG{l+s+s1}{hatch}\PYG{l+s+s1}{\PYGZsq{}}
\PYG{c+c1}{\PYGZsh{} change color}
\PYG{n}{fill}\PYG{o}{.}\PYG{n}{color}\PYG{o}{=}\PYG{l+m+mi}{241}
\PYG{c+c1}{\PYGZsh{} change style index}
\PYG{n}{fill}\PYG{o}{.}\PYG{n}{index}\PYG{o}{=}\PYG{l+m+mi}{3}
\end{Verbatim}
\index{xmtics1 (vcs.meshfill.Gfm attribute)}

\begin{fulllineitems}
\phantomsection\label{vcs/graphics/meshfill:vcs.meshfill.Gfm.xmtics1}\pysiglinewithargsret{\sphinxbfcode{xmtics1}}{\emph{str/\{float:str\}}}{}
(Ex: `') dictionary with location of intermediate tics as keys for 1st side of y axis

\end{fulllineitems}

\index{xmtics2 (vcs.meshfill.Gfm attribute)}

\begin{fulllineitems}
\phantomsection\label{vcs/graphics/meshfill:vcs.meshfill.Gfm.xmtics2}\pysiglinewithargsret{\sphinxbfcode{xmtics2}}{\emph{str/\{float:str\}}}{}
(Ex: `') dictionary with location of intermediate tics as keys for 2nd side of y axis

\end{fulllineitems}

\index{ymtics1 (vcs.meshfill.Gfm attribute)}

\begin{fulllineitems}
\phantomsection\label{vcs/graphics/meshfill:vcs.meshfill.Gfm.ymtics1}\pysiglinewithargsret{\sphinxbfcode{ymtics1}}{\emph{str/\{float:str\}}}{}
(Ex: `') dictionary with location of intermediate tics as keys for 1st side of y axis

\end{fulllineitems}

\index{ymtics2 (vcs.meshfill.Gfm attribute)}

\begin{fulllineitems}
\phantomsection\label{vcs/graphics/meshfill:vcs.meshfill.Gfm.ymtics2}\pysiglinewithargsret{\sphinxbfcode{ymtics2}}{\emph{str/\{float:str\}}}{}
(Ex: `') dictionary with location of intermediate tics as keys for 2nd side of y axis

\end{fulllineitems}

\index{xticlabels1 (vcs.meshfill.Gfm attribute)}

\begin{fulllineitems}
\phantomsection\label{vcs/graphics/meshfill:vcs.meshfill.Gfm.xticlabels1}\pysiglinewithargsret{\sphinxbfcode{xticlabels1}}{\emph{str/\{float:str\}}}{}
(Ex: `*') values for labels on 1st side of x axis

\end{fulllineitems}

\index{xticlabels2 (vcs.meshfill.Gfm attribute)}

\begin{fulllineitems}
\phantomsection\label{vcs/graphics/meshfill:vcs.meshfill.Gfm.xticlabels2}\pysiglinewithargsret{\sphinxbfcode{xticlabels2}}{\emph{str/\{float:str\}}}{}
(Ex: `*') values for labels on 2nd side of x axis

\end{fulllineitems}

\index{yticlabels1 (vcs.meshfill.Gfm attribute)}

\begin{fulllineitems}
\phantomsection\label{vcs/graphics/meshfill:vcs.meshfill.Gfm.yticlabels1}\pysiglinewithargsret{\sphinxbfcode{yticlabels1}}{\emph{str/\{float:str\}}}{}
(Ex: `*') values for labels on 1st side of y axis

\end{fulllineitems}

\index{yticlabels2 (vcs.meshfill.Gfm attribute)}

\begin{fulllineitems}
\phantomsection\label{vcs/graphics/meshfill:vcs.meshfill.Gfm.yticlabels2}\pysiglinewithargsret{\sphinxbfcode{yticlabels2}}{\emph{str/\{float:str\}}}{}
(Ex: `*') values for labels on 2nd side of y axis

\end{fulllineitems}

\index{projection (vcs.meshfill.Gfm attribute)}

\begin{fulllineitems}
\phantomsection\label{vcs/graphics/meshfill:vcs.meshfill.Gfm.projection}\pysiglinewithargsret{\sphinxbfcode{projection}}{\emph{str/vcs.projection.Proj}}{}
(Ex: `default') projection to use, name or object

\end{fulllineitems}

\index{datawc\_x1 (vcs.meshfill.Gfm attribute)}

\begin{fulllineitems}
\phantomsection\label{vcs/graphics/meshfill:vcs.meshfill.Gfm.datawc_x1}\pysiglinewithargsret{\sphinxbfcode{datawc\_x1}}{\emph{float}}{}
(Ex: 1.E20) first value of xaxis on plot

\end{fulllineitems}

\index{datawc\_x2 (vcs.meshfill.Gfm attribute)}

\begin{fulllineitems}
\phantomsection\label{vcs/graphics/meshfill:vcs.meshfill.Gfm.datawc_x2}\pysiglinewithargsret{\sphinxbfcode{datawc\_x2}}{\emph{float}}{}
(Ex: 1.E20) second value of xaxis on plot

\end{fulllineitems}

\index{datawc\_y1 (vcs.meshfill.Gfm attribute)}

\begin{fulllineitems}
\phantomsection\label{vcs/graphics/meshfill:vcs.meshfill.Gfm.datawc_y1}\pysiglinewithargsret{\sphinxbfcode{datawc\_y1}}{\emph{float}}{}
(Ex: 1.E20) first value of yaxis on plot

\end{fulllineitems}

\index{datawc\_y2 (vcs.meshfill.Gfm attribute)}

\begin{fulllineitems}
\phantomsection\label{vcs/graphics/meshfill:vcs.meshfill.Gfm.datawc_y2}\pysiglinewithargsret{\sphinxbfcode{datawc\_y2}}{\emph{float}}{}
(Ex: 1.E20) second value of yaxis on plot

\end{fulllineitems}

\index{datawc\_timeunits (vcs.meshfill.Gfm attribute)}

\begin{fulllineitems}
\phantomsection\label{vcs/graphics/meshfill:vcs.meshfill.Gfm.datawc_timeunits}\pysiglinewithargsret{\sphinxbfcode{datawc\_timeunits}}{\emph{str}}{}
(Ex: `days since 2000') units to use when displaying time dimension auto tick

\end{fulllineitems}

\index{datawc\_calendar (vcs.meshfill.Gfm attribute)}

\begin{fulllineitems}
\phantomsection\label{vcs/graphics/meshfill:vcs.meshfill.Gfm.datawc_calendar}\pysiglinewithargsret{\sphinxbfcode{datawc\_calendar}}{\emph{int}}{}
(Ex: 135441) calendar to use when displaying time dimension auto tick, default is proleptic gregorian calendar

\end{fulllineitems}

\index{levels (vcs.meshfill.Gfm attribute)}

\begin{fulllineitems}
\phantomsection\label{vcs/graphics/meshfill:vcs.meshfill.Gfm.levels}\pysiglinewithargsret{\sphinxbfcode{levels}}{\emph{{[}float,...{]}/{[}{[}float,float{]},...{]}}}{}
Sets the levels range to use, can be either a list of contiguous levels, or list of tuples
indicating first and last value of the range.

\end{fulllineitems}

\index{fillareacolors (vcs.meshfill.Gfm attribute)}

\begin{fulllineitems}
\phantomsection\label{vcs/graphics/meshfill:vcs.meshfill.Gfm.fillareacolors}\pysiglinewithargsret{\sphinxbfcode{fillareacolors}}{\sphinxoptional{\emph{int}, \emph{...}}}{}
Colors to use for each level

\end{fulllineitems}

\index{fillareastyle (vcs.meshfill.Gfm attribute)}

\begin{fulllineitems}
\phantomsection\label{vcs/graphics/meshfill:vcs.meshfill.Gfm.fillareastyle}\pysiglinewithargsret{\sphinxbfcode{fillareastyle}}{\emph{str}}{}
Style to use for levels filling: solid/pattern/hatch

\end{fulllineitems}

\index{fillareaindices (vcs.meshfill.Gfm attribute)}

\begin{fulllineitems}
\phantomsection\label{vcs/graphics/meshfill:vcs.meshfill.Gfm.fillareaindices}\pysiglinewithargsret{\sphinxbfcode{fillareaindices}}{\sphinxoptional{\emph{int}, \emph{...}}}{}
List of patterns to use when filling a level and using pattern/hatch

\end{fulllineitems}

\index{legend (vcs.meshfill.Gfm attribute)}

\begin{fulllineitems}
\phantomsection\label{vcs/graphics/meshfill:vcs.meshfill.Gfm.legend}\pysiglinewithargsret{\sphinxbfcode{legend}}{\emph{None/\{float:str\}}}{}
Replaces the legend values in the dictionary keys with their associated string

\end{fulllineitems}

\index{ext\_1 (vcs.meshfill.Gfm attribute)}

\begin{fulllineitems}
\phantomsection\label{vcs/graphics/meshfill:vcs.meshfill.Gfm.ext_1}\pysiglinewithargsret{\sphinxbfcode{ext\_1}}{\emph{str}}{}
Draws an extension arrow on right side (values less than first range value)

\end{fulllineitems}

\index{ext\_2 (vcs.meshfill.Gfm attribute)}

\begin{fulllineitems}
\phantomsection\label{vcs/graphics/meshfill:vcs.meshfill.Gfm.ext_2}\pysiglinewithargsret{\sphinxbfcode{ext\_2}}{\emph{str}}{}
Draws an extension arrow on left side (values greater than last range value)

\end{fulllineitems}

\index{missing (vcs.meshfill.Gfm attribute)}

\begin{fulllineitems}
\phantomsection\label{vcs/graphics/meshfill:vcs.meshfill.Gfm.missing}\pysiglinewithargsret{\sphinxbfcode{missing}}{\emph{int}}{}
Color to use for missing value or values not in defined ranges

\end{fulllineitems}

\end{quote}

\end{itemize}

\end{fulllineitems}


mesh :: (str/int) (0) Draws the mesh
wrap :: ({[}float,float{]}) ({[}0.,0.{]}) Modulo to wrap around on either axis (automatically sets to 360 for longitude axes)
\index{colors() (vcs.meshfill.Gfm method)}

\begin{fulllineitems}
\phantomsection\label{vcs/graphics/meshfill:vcs.meshfill.Gfm.colors}\pysiglinewithargsret{\sphinxbfcode{colors}}{\emph{color1=16}, \emph{color2=239}}{}
Sets the color\_1 and color\_2 properties of the object.
\begin{quote}\begin{description}
\item[{Parameters}] \leavevmode\begin{itemize}
\item {} 
\textbf{\texttt{color1}} (\href{https://docs.python.org/2/library/functions.html\#int}{\emph{\texttt{int}}}) -- Sets the \sphinxcode{color\_1} value on the object

\item {} 
\textbf{\texttt{color2}} (\href{https://docs.python.org/2/library/functions.html\#int}{\emph{\texttt{int}}}) -- Sets the \sphinxcode{color\_2} value on the object

\end{itemize}

\end{description}\end{quote}

\end{fulllineitems}

\index{datawc() (vcs.meshfill.Gfm method)}

\begin{fulllineitems}
\phantomsection\label{vcs/graphics/meshfill:vcs.meshfill.Gfm.datawc}\pysiglinewithargsret{\sphinxbfcode{datawc}}{\emph{dsp1=1e+20}, \emph{dsp2=1e+20}, \emph{dsp3=1e+20}, \emph{dsp4=1e+20}}{}
Sets the data world coordinates for object
\begin{quote}\begin{description}
\item[{Parameters}] \leavevmode\begin{itemize}
\item {} 
\textbf{\texttt{dsp1}} (\href{https://docs.python.org/2/library/functions.html\#float}{\emph{\texttt{float}}}) -- Sets the {\hyperref[vcs/graphics/meshfill:vcs.meshfill.Gfm.datawc_y1]{\sphinxcrossref{\sphinxcode{datawc\_y1}}}} property of the object.

\item {} 
\textbf{\texttt{dsp2}} (\href{https://docs.python.org/2/library/functions.html\#float}{\emph{\texttt{float}}}) -- Sets the {\hyperref[vcs/graphics/meshfill:vcs.meshfill.Gfm.datawc_y2]{\sphinxcrossref{\sphinxcode{datawc\_y2}}}} property of the object.

\item {} 
\textbf{\texttt{dsp3}} (\href{https://docs.python.org/2/library/functions.html\#float}{\emph{\texttt{float}}}) -- Sets the {\hyperref[vcs/graphics/meshfill:vcs.meshfill.Gfm.datawc_x1]{\sphinxcrossref{\sphinxcode{datawc\_x1}}}} property of the object.

\item {} 
\textbf{\texttt{dsp4}} (\href{https://docs.python.org/2/library/functions.html\#float}{\emph{\texttt{float}}}) -- Sets the {\hyperref[vcs/graphics/meshfill:vcs.meshfill.Gfm.datawc_x2]{\sphinxcrossref{\sphinxcode{datawc\_x2}}}} property of the object.

\end{itemize}

\end{description}\end{quote}

\end{fulllineitems}

\index{exts() (vcs.meshfill.Gfm method)}

\begin{fulllineitems}
\phantomsection\label{vcs/graphics/meshfill:vcs.meshfill.Gfm.exts}\pysiglinewithargsret{\sphinxbfcode{exts}}{\emph{ext1='n'}, \emph{ext2='y'}}{}
Sets the ext\_1 and ext\_2 values on the object.
\begin{quote}\begin{description}
\item[{Parameters}] \leavevmode\begin{itemize}
\item {} 
\textbf{\texttt{ext1}} (\href{https://docs.python.org/2/library/functions.html\#str}{\emph{\texttt{str}}}) -- Sets the {\hyperref[vcs/graphics/meshfill:vcs.meshfill.Gfm.ext_1]{\sphinxcrossref{\sphinxcode{ext\_1}}}} value on the object. `y' sets it to True, `n' sets it to False.

\item {} 
\textbf{\texttt{ext2}} (\href{https://docs.python.org/2/library/functions.html\#str}{\emph{\texttt{str}}}) -- Sets the {\hyperref[vcs/graphics/meshfill:vcs.meshfill.Gfm.ext_2]{\sphinxcrossref{\sphinxcode{ext\_2}}}} value on the object. `y' sets it to True, `n' sets it to False.

\end{itemize}

\end{description}\end{quote}

\end{fulllineitems}

\index{list() (vcs.meshfill.Gfm method)}

\begin{fulllineitems}
\phantomsection\label{vcs/graphics/meshfill:vcs.meshfill.Gfm.list}\pysiglinewithargsret{\sphinxbfcode{list}}{}{}
Lists the current values of object attributes

\end{fulllineitems}

\index{script() (vcs.meshfill.Gfm method)}

\begin{fulllineitems}
\phantomsection\label{vcs/graphics/meshfill:vcs.meshfill.Gfm.script}\pysiglinewithargsret{\sphinxbfcode{script}}{\emph{script\_filename}, \emph{mode='a'}}{}
Saves out a copy of the meshfill graphics method in JSON, or Python format to a designated file.
\begin{quote}

\begin{notice}{note}{Note:}
If the the filename has a `.py' at the end, it will produce a
Python script. If no extension is given, then by default a
.json file containing a JSON serialization of the object's
data will be produced.
\end{notice}

\begin{notice}{warning}{Warning:}
VCS Scripts Deprecated.
SCR script files are no longer generated by this function.
\end{notice}
\end{quote}
\begin{quote}\begin{description}
\item[{Example}] \leavevmode
\begin{Verbatim}[commandchars=\\\{\}]
\PYG{g+gp}{\PYGZgt{}\PYGZgt{}\PYGZgt{} }\PYG{n}{a}\PYG{o}{=}\PYG{n}{vcs}\PYG{o}{.}\PYG{n}{init}\PYG{p}{(}\PYG{p}{)} \PYG{c+c1}{\PYGZsh{} Make a Canvas object to work with}
\PYG{g+gp}{\PYGZgt{}\PYGZgt{}\PYGZgt{} }\PYG{n}{ex}\PYG{o}{=}\PYG{n}{a}\PYG{o}{.}\PYG{n}{getmeshfill}\PYG{p}{(}\PYG{p}{)} \PYG{c+c1}{\PYGZsh{} Get default meshfill}
\PYG{g+gp}{\PYGZgt{}\PYGZgt{}\PYGZgt{} }\PYG{n}{ex}\PYG{o}{.}\PYG{n}{script}\PYG{p}{(}\PYG{l+s+s1}{\PYGZsq{}}\PYG{l+s+s1}{filename.py}\PYG{l+s+s1}{\PYGZsq{}}\PYG{p}{)} \PYG{c+c1}{\PYGZsh{} Append to a Python script named \PYGZsq{}filename.py\PYGZsq{}}
\PYG{g+gp}{\PYGZgt{}\PYGZgt{}\PYGZgt{} }\PYG{n}{ex}\PYG{o}{.}\PYG{n}{script}\PYG{p}{(}\PYG{l+s+s1}{\PYGZsq{}}\PYG{l+s+s1}{filename}\PYG{l+s+s1}{\PYGZsq{}}\PYG{p}{,}\PYG{l+s+s1}{\PYGZsq{}}\PYG{l+s+s1}{w}\PYG{l+s+s1}{\PYGZsq{}}\PYG{p}{)} \PYG{c+c1}{\PYGZsh{} Create or overwrite a JSON file \PYGZsq{}filename.json\PYGZsq{}.}
\end{Verbatim}

\item[{Parameters}] \leavevmode\begin{itemize}
\item {} 
\textbf{\texttt{script\_filename}} (\href{https://docs.python.org/2/library/functions.html\#str}{\emph{\texttt{str}}}) -- Output name of the script file. If no extension is specified, a .json object is created.

\item {} 
\textbf{\texttt{mode}} (\href{https://docs.python.org/2/library/functions.html\#str}{\emph{\texttt{str}}}) -- Either `w' for replace, or `a' for append. Defaults to `a', if not specified.

\end{itemize}

\end{description}\end{quote}

\end{fulllineitems}

\index{xmtics() (vcs.meshfill.Gfm method)}

\begin{fulllineitems}
\phantomsection\label{vcs/graphics/meshfill:vcs.meshfill.Gfm.xmtics}\pysiglinewithargsret{\sphinxbfcode{xmtics}}{\emph{xmt1='`}, \emph{xmt2='`}}{}
Sets the xmtics1 and xmtics2 values on the object
\begin{quote}\begin{description}
\item[{Parameters}] \leavevmode\begin{itemize}
\item {} 
\textbf{\texttt{xmt1}} (\emph{\texttt{\{float:str\} or str}}) -- Value for {\hyperref[vcs/graphics/meshfill:vcs.meshfill.Gfm.xmtics1]{\sphinxcrossref{\sphinxcode{xmtics1}}}}. Must be a str, or a dictionary object with float:str mappings.

\item {} 
\textbf{\texttt{xmt2}} (\emph{\texttt{\{float:str\} or str}}) -- Value for {\hyperref[vcs/graphics/meshfill:vcs.meshfill.Gfm.xmtics2]{\sphinxcrossref{\sphinxcode{xmtics2}}}}. Must be a str, or a dictionary object with float:str mappings.

\end{itemize}

\end{description}\end{quote}

\end{fulllineitems}

\index{xticlabels() (vcs.meshfill.Gfm method)}

\begin{fulllineitems}
\phantomsection\label{vcs/graphics/meshfill:vcs.meshfill.Gfm.xticlabels}\pysiglinewithargsret{\sphinxbfcode{xticlabels}}{\emph{xtl1='`}, \emph{xtl2='`}}{}
Sets the xticlabels1 and xticlabels2 values on the object
\begin{quote}\begin{description}
\item[{Parameters}] \leavevmode\begin{itemize}
\item {} 
\textbf{\texttt{xtl1}} (\emph{\texttt{\{float:str\} or str}}) -- Sets the object's value for {\hyperref[vcs/graphics/meshfill:vcs.meshfill.Gfm.xticlabels1]{\sphinxcrossref{\sphinxcode{xticlabels1}}}}. Must be  a str, or a dictionary object with float:str mappings.

\item {} 
\textbf{\texttt{xtl2}} (\emph{\texttt{\{float:str\} or str}}) -- Sets the object's value for {\hyperref[vcs/graphics/meshfill:vcs.meshfill.Gfm.xticlabels2]{\sphinxcrossref{\sphinxcode{xticlabels2}}}}. Must be a str, or a dictionary object with float:str mappings.

\end{itemize}

\end{description}\end{quote}

\end{fulllineitems}

\index{xyscale() (vcs.meshfill.Gfm method)}

\begin{fulllineitems}
\phantomsection\label{vcs/graphics/meshfill:vcs.meshfill.Gfm.xyscale}\pysiglinewithargsret{\sphinxbfcode{xyscale}}{\emph{xat='`}, \emph{yat='`}}{}
Sets xaxisconvert and yaxisconvert values for the object.
\begin{quote}\begin{description}
\item[{Example}] \leavevmode
\begin{Verbatim}[commandchars=\\\{\}]
\PYG{g+gp}{\PYGZgt{}\PYGZgt{}\PYGZgt{} }\PYG{n}{a}\PYG{o}{=}\PYG{n}{vcs}\PYG{o}{.}\PYG{n}{init}\PYG{p}{(}\PYG{p}{)}
\PYG{g+gp}{\PYGZgt{}\PYGZgt{}\PYGZgt{} }\PYG{n}{ex}\PYG{o}{=}\PYG{n}{a}\PYG{o}{.}\PYG{n}{createmeshfill}\PYG{p}{(}\PYG{l+s+s1}{\PYGZsq{}}\PYG{l+s+s1}{xyscale\PYGZus{}ex}\PYG{l+s+s1}{\PYGZsq{}}\PYG{p}{)} \PYG{c+c1}{\PYGZsh{} create a boxfill to work with}
\PYG{g+gp}{\PYGZgt{}\PYGZgt{}\PYGZgt{} }\PYG{n}{ex}\PYG{o}{.}\PYG{n}{xyscale}\PYG{p}{(}\PYG{n}{xat}\PYG{o}{=}\PYG{l+s+s1}{\PYGZsq{}}\PYG{l+s+s1}{linear}\PYG{l+s+s1}{\PYGZsq{}}\PYG{p}{,} \PYG{n}{yat}\PYG{o}{=}\PYG{l+s+s1}{\PYGZsq{}}\PYG{l+s+s1}{linear}\PYG{l+s+s1}{\PYGZsq{}}\PYG{p}{)} \PYG{c+c1}{\PYGZsh{} set xaxisconvert and yaxisconvert to \PYGZsq{}linear\PYGZsq{}}
\end{Verbatim}

\item[{Parameters}] \leavevmode\begin{itemize}
\item {} 
\textbf{\texttt{xat}} (\href{https://docs.python.org/2/library/functions.html\#str}{\emph{\texttt{str}}}) -- Set value for x axis conversion.

\item {} 
\textbf{\texttt{yat}} (\href{https://docs.python.org/2/library/functions.html\#str}{\emph{\texttt{str}}}) -- Set value for y axis conversion.

\end{itemize}

\end{description}\end{quote}

\end{fulllineitems}

\index{ymtics() (vcs.meshfill.Gfm method)}

\begin{fulllineitems}
\phantomsection\label{vcs/graphics/meshfill:vcs.meshfill.Gfm.ymtics}\pysiglinewithargsret{\sphinxbfcode{ymtics}}{\emph{ymt1='`}, \emph{ymt2='`}}{}
Sets the ymtics1 and ymtics2 values on the object
\begin{quote}\begin{description}
\item[{Parameters}] \leavevmode\begin{itemize}
\item {} 
\textbf{\texttt{ymt1}} (\emph{\texttt{\{float:str\} or str}}) -- Value for {\hyperref[vcs/graphics/meshfill:vcs.meshfill.Gfm.ymtics1]{\sphinxcrossref{\sphinxcode{ymtics1}}}}. Must be a str, or a dictionary object with float:str mappings.

\item {} 
\textbf{\texttt{ymt2}} (\emph{\texttt{\{float:str\} or str}}) -- Value for {\hyperref[vcs/graphics/meshfill:vcs.meshfill.Gfm.ymtics2]{\sphinxcrossref{\sphinxcode{ymtics2}}}}. Must be a str, or a dictionary object with float:str mappings.

\end{itemize}

\end{description}\end{quote}

\end{fulllineitems}

\index{yticlabels() (vcs.meshfill.Gfm method)}

\begin{fulllineitems}
\phantomsection\label{vcs/graphics/meshfill:vcs.meshfill.Gfm.yticlabels}\pysiglinewithargsret{\sphinxbfcode{yticlabels}}{\emph{ytl1='`}, \emph{ytl2='`}}{}
Sets the yticlabels1 and yticlabels2 values on the object
\begin{quote}\begin{description}
\item[{Parameters}] \leavevmode\begin{itemize}
\item {} 
\textbf{\texttt{ytl1}} (\emph{\texttt{\{float:str\} or str}}) -- Sets the object's value for {\hyperref[vcs/graphics/meshfill:vcs.meshfill.Gfm.yticlabels1]{\sphinxcrossref{\sphinxcode{yticlabels1}}}}. Must be  a str, or a dictionary object with float:str mappings.

\item {} 
\textbf{\texttt{ytl2}} (\emph{\texttt{\{float:str\} or str}}) -- Sets the object's value for {\hyperref[vcs/graphics/meshfill:vcs.meshfill.Gfm.yticlabels2]{\sphinxcrossref{\sphinxcode{yticlabels2}}}}. Must be a str, or a dictionary object with float:str mappings.

\end{itemize}

\end{description}\end{quote}

\end{fulllineitems}


\end{fulllineitems}



\subsection{taylor}
\label{vcs/graphics/taylor:taylor}\label{vcs/graphics/taylor::doc}\label{vcs/graphics/taylor:module-vcs.taylor}\index{vcs.taylor (module)}\index{Gtd (class in vcs.taylor)}

\begin{fulllineitems}
\phantomsection\label{vcs/graphics/taylor:vcs.taylor.Gtd}\pysiglinewithargsret{\sphinxstrong{class }\sphinxcode{vcs.taylor.}\sphinxbfcode{Gtd}}{\emph{name}, \emph{source='default'}}{}
The Taylor Diagram graphics method (Gtd) is used to plot \href{http://www-pcmdi.llnl.gov/about/staff/Taylor/CV/Taylor\_diagram\_primer.pdf}{Taylor diagrams} on a VCS Canvas.
\href{http://www-pcmdi.llnl.gov/about/staff/Taylor/CV/Taylor\_diagram\_primer.pdf}{Taylor diagrams} provide a way of graphically summarizing how closely a pattern matches observations.
\index{defaultSkillFunction() (vcs.taylor.Gtd method)}

\begin{fulllineitems}
\phantomsection\label{vcs/graphics/taylor:vcs.taylor.Gtd.defaultSkillFunction}\pysiglinewithargsret{\sphinxbfcode{defaultSkillFunction}}{\emph{s}, \emph{R}}{}
Provides a default function for determining the \href{https://en.wikipedia.org/wiki/Forecast\_skill}{skill} with which a model predicts observations.
This function may be used in the function parameter of {\hyperref[vcs/graphics/taylor:vcs.taylor.Gtd.drawSkill]{\sphinxcrossref{\sphinxcode{drawSkill()}}}}, although it may be preferable
to provide a custom function for determining \href{https://en.wikipedia.org/wiki/Forecast\_skill}{skill}, depending on the application.
\begin{quote}\begin{description}
\item[{Parameters}] \leavevmode\begin{itemize}
\item {} 
\textbf{\texttt{s}} (\href{https://docs.python.org/2/library/functions.html\#float}{\emph{\texttt{float}}}) -- A float representing the standard deviation of a model.

\item {} 
\textbf{\texttt{R}} (\href{https://docs.python.org/2/library/functions.html\#float}{\emph{\texttt{float}}}) -- A float representing the correlation of a model.

\end{itemize}

\item[{Returns}] \leavevmode
The \href{https://en.wikipedia.org/wiki/Forecast\_skill}{skill} of a model, computed using this function.

\item[{Return type}] \leavevmode
\href{https://docs.python.org/2/library/functions.html\#float}{float}

\end{description}\end{quote}

\end{fulllineitems}

\index{drawSkill() (vcs.taylor.Gtd method)}

\begin{fulllineitems}
\phantomsection\label{vcs/graphics/taylor:vcs.taylor.Gtd.drawSkill}\pysiglinewithargsret{\sphinxbfcode{drawSkill}}{\emph{canvas}, \emph{values}, \emph{function=None}}{}
Draw a skill score. Default skill score provided in {\hyperref[vcs/graphics/taylor:vcs.taylor.Gtd.defaultSkillFunction]{\sphinxcrossref{\sphinxcode{defaultSkillFunction()}}}}
from Karl taylor, see \href{http://www-pcmdi.llnl.gov/publications/pdf/55.pdf}{PCMDI report series 55} for more information on \href{http://www-pcmdi.llnl.gov/about/staff/Taylor/CV/Taylor\_diagram\_primer.pdf}{Taylor diagrams} and {\color{red}\bfseries{}{}`}skill{}`\_s.

\begin{notice}{note}{Note:}
The function parameter must be provided for drawSkill to work.
The {\hyperref[vcs/graphics/taylor:vcs.taylor.Gtd.defaultSkillFunction]{\sphinxcrossref{\sphinxcode{defaultSkillFunction()}}}} provided in this module can be used to provide a default skill score.
Be aware that, as stated in \href{http://www-pcmdi.llnl.gov/publications/pdf/55.pdf}{PCMDI report series 55} section 5, it is not possible to define
a single skill score that is appropriate for all models.
It may be more suitable to create a custom function for determining the skill score of your model.
\end{notice}
\begin{quote}\begin{description}
\item[{Parameters}] \leavevmode\begin{itemize}
\item {} 
\textbf{\texttt{canvas}} ({\hyperref[vcs/Canvas:vcs.Canvas.Canvas]{\sphinxcrossref{\emph{\texttt{vcs.Canvas.Canvas}}}}}) -- A VCS Canvas object on which to draw the skill score.

\item {} 
\textbf{\texttt{values}} (\emph{\texttt{list/tuple}}) -- A list/tuple used to specify the {\hyperref[vcs/graphics/isoline:vcs.isoline.Gi.level]{\sphinxcrossref{\sphinxcode{levels}}}}
of an {\hyperref[vcs/graphics/isoline:vcs.isoline.Gi]{\sphinxcrossref{\sphinxcode{isoline}}}} object.

\item {} 
\textbf{\texttt{function}} -- A function for determining the skill score of a model.

\end{itemize}

\end{description}\end{quote}

\end{fulllineitems}

\index{getArc() (vcs.taylor.Gtd method)}

\begin{fulllineitems}
\phantomsection\label{vcs/graphics/taylor:vcs.taylor.Gtd.getArc}\pysiglinewithargsret{\sphinxbfcode{getArc}}{\emph{value}, \emph{val1=0.0}, \emph{val2=90.0}, \emph{convert=True}}{}
Return coordinates to draw an arc from 0 to 90 degrees.

\begin{notice}{note}{Note:}
val1 and val2 can be used to limit the arc (in degrees).
\end{notice}
\begin{quote}\begin{description}
\item[{Parameters}] \leavevmode\begin{itemize}
\item {} 
\textbf{\texttt{value}} (\href{https://docs.python.org/2/library/functions.html\#float}{\emph{\texttt{float}}}) -- The radius of the arc to be calculated.

\item {} 
\textbf{\texttt{val1}} (\href{https://docs.python.org/2/library/functions.html\#float}{\emph{\texttt{float}}}) -- Lower limit of the arc to compute.

\item {} 
\textbf{\texttt{val2}} (\href{https://docs.python.org/2/library/functions.html\#float}{\emph{\texttt{float}}}) -- Upper limit of the arc to compute.

\item {} 
\textbf{\texttt{convert}} (\href{https://docs.python.org/2/library/functions.html\#bool}{\emph{\texttt{bool}}}) -- Boolean flag indicating whether

\end{itemize}

\item[{Returns}] \leavevmode
The coordinates for the calculated arc.

\item[{Return type}] \leavevmode
\href{https://docs.python.org/2/library/functions.html\#tuple}{tuple}

\end{description}\end{quote}

\end{fulllineitems}

\index{plot() (vcs.taylor.Gtd method)}

\begin{fulllineitems}
\phantomsection\label{vcs/graphics/taylor:vcs.taylor.Gtd.plot}\pysiglinewithargsret{\sphinxbfcode{plot}}{\emph{data}, \emph{template='deftaylor'}, \emph{skill=None}, \emph{bg=0}, \emph{canvas=None}}{}
Plots an instance of a {\hyperref[vcs/graphics/taylor:vcs.taylor.Gtd]{\sphinxcrossref{\sphinxcode{Taylor diagram}}}} on the provided VCS Canvas.
\begin{quote}\begin{description}
\item[{Parameters}] \leavevmode\begin{itemize}
\item {} 
\textbf{\texttt{data}} -- 

\item {} 
\textbf{\texttt{template}} (\emph{\texttt{str/vcs.template.P}}) -- A {\hyperref[vcs/template/template:vcs.template.P]{\sphinxcrossref{\sphinxcode{VCS template}}}} or a string name of a VCS template.

\item {} 
\textbf{\texttt{skill}} -- 

\item {} 
\textbf{\texttt{bg}} (\emph{\texttt{bool/int}}) -- A boolean/integer flag indicating whether to plot this object in the background.

\item {} 
\textbf{\texttt{canvas}} ({\hyperref[vcs/Canvas:vcs.Canvas.Canvas]{\sphinxcrossref{\emph{\texttt{vcs.Canvas.Canvas}}}}}) -- A VCS Canvas object on which the diagram will be plotted.

\end{itemize}

\end{description}\end{quote}

\end{fulllineitems}

\index{script() (vcs.taylor.Gtd method)}

\begin{fulllineitems}
\phantomsection\label{vcs/graphics/taylor:vcs.taylor.Gtd.script}\pysiglinewithargsret{\sphinxbfcode{script}}{\emph{script\_filename}, \emph{mode='a'}}{}
Saves out a copy of the taylordiagram graphics method in JSON, or Python format to a designated file.
\begin{quote}

\begin{notice}{note}{Note:}
If the the filename has a `.py' at the end, it will produce a
Python script. If no extension is given, then by default a
.json file containing a JSON serialization of the object's
data will be produced.
\end{notice}

\begin{notice}{warning}{Warning:}
VCS Scripts Deprecated.
SCR script files are no longer generated by this function.
\end{notice}
\end{quote}
\begin{quote}\begin{description}
\item[{Example}] \leavevmode
\begin{Verbatim}[commandchars=\\\{\}]
\PYG{g+gp}{\PYGZgt{}\PYGZgt{}\PYGZgt{} }\PYG{n}{a}\PYG{o}{=}\PYG{n}{vcs}\PYG{o}{.}\PYG{n}{init}\PYG{p}{(}\PYG{p}{)} \PYG{c+c1}{\PYGZsh{} Make a Canvas object to work with}
\PYG{g+gp}{\PYGZgt{}\PYGZgt{}\PYGZgt{} }\PYG{n}{ex}\PYG{o}{=}\PYG{n}{a}\PYG{o}{.}\PYG{n}{gettaylordiagram}\PYG{p}{(}\PYG{p}{)} \PYG{c+c1}{\PYGZsh{} Get default taylordiagram}
\PYG{g+gp}{\PYGZgt{}\PYGZgt{}\PYGZgt{} }\PYG{n}{ex}\PYG{o}{.}\PYG{n}{script}\PYG{p}{(}\PYG{l+s+s1}{\PYGZsq{}}\PYG{l+s+s1}{filename.py}\PYG{l+s+s1}{\PYGZsq{}}\PYG{p}{)} \PYG{c+c1}{\PYGZsh{} Append to a Python script named \PYGZsq{}filename.py\PYGZsq{}}
\PYG{g+gp}{\PYGZgt{}\PYGZgt{}\PYGZgt{} }\PYG{n}{ex}\PYG{o}{.}\PYG{n}{script}\PYG{p}{(}\PYG{l+s+s1}{\PYGZsq{}}\PYG{l+s+s1}{filename}\PYG{l+s+s1}{\PYGZsq{}}\PYG{p}{,}\PYG{l+s+s1}{\PYGZsq{}}\PYG{l+s+s1}{w}\PYG{l+s+s1}{\PYGZsq{}}\PYG{p}{)} \PYG{c+c1}{\PYGZsh{} Create or overwrite a JSON file \PYGZsq{}filename.json\PYGZsq{}.}
\end{Verbatim}

\item[{Parameters}] \leavevmode\begin{itemize}
\item {} 
\textbf{\texttt{script\_filename}} (\href{https://docs.python.org/2/library/functions.html\#str}{\emph{\texttt{str}}}) -- Output name of the script file. If no extension is specified, a .json object is created.

\item {} 
\textbf{\texttt{mode}} (\href{https://docs.python.org/2/library/functions.html\#str}{\emph{\texttt{str}}}) -- Either `w' for replace, or `a' for append. Defaults to `a', if not specified.

\end{itemize}

\end{description}\end{quote}

\end{fulllineitems}


\end{fulllineitems}

\index{TDMarker (class in vcs.taylor)}

\begin{fulllineitems}
\phantomsection\label{vcs/graphics/taylor:vcs.taylor.TDMarker}\pysigline{\sphinxstrong{class }\sphinxcode{vcs.taylor.}\sphinxbfcode{TDMarker}}
class
\index{equalize() (vcs.taylor.TDMarker method)}

\begin{fulllineitems}
\phantomsection\label{vcs/graphics/taylor:vcs.taylor.TDMarker.equalize}\pysiglinewithargsret{\sphinxbfcode{equalize}}{}{}
Make sure that we have the same amount of everything
usage self.equalize()
Also updates self.number

\end{fulllineitems}


\end{fulllineitems}



\subsection{unified1D}
\label{vcs/graphics/unified1D:unified1d}\label{vcs/graphics/unified1D:module-vcs.unified1D}\label{vcs/graphics/unified1D::doc}\index{vcs.unified1D (module)}
\# Unification of all 1D gms
\index{G1d (class in vcs.unified1D)}

\begin{fulllineitems}
\phantomsection\label{vcs/graphics/unified1D:vcs.unified1D.G1d}\pysiglinewithargsret{\sphinxstrong{class }\sphinxcode{vcs.unified1D.}\sphinxbfcode{G1d}}{\emph{name}, \emph{name\_src='default'}}{}
This graphics method displays a line plot from 1D data array (i.e. a
plot of Y(x), where y represents the 1D coordinate values, and x can be either Y's axis or another 1D arrays).
The example below
shows how to change line and marker attributes for the Yxvsx graphics method.

This class is used to define an Yxvsx table entry used in VCS, or it can be
used to change some or all of the Yxvsx attributes in an existing Yxvsx table
entry.


\begin{fulllineitems}
\pysigline{\sphinxbfcode{Make~a~Canvas~object:}}
You'll need a Canvas object to work with.

\begin{Verbatim}[commandchars=\\\{\}]
\PYG{c+c1}{\PYGZsh{} VCS Canvas constructor}
\PYG{n}{a}\PYG{o}{=}\PYG{n}{vcs}\PYG{o}{.}\PYG{n}{init}\PYG{p}{(}\PYG{p}{)}
\end{Verbatim}

\end{fulllineitems}



\begin{fulllineitems}
\pysigline{\sphinxbfcode{Create~a~new~instance~of~Yxvsx:}}~
\begin{Verbatim}[commandchars=\\\{\}]
\PYG{c+c1}{\PYGZsh{} Copies content of \PYGZsq{}quick\PYGZsq{} to \PYGZsq{}new\PYGZsq{}}
\PYG{n}{yxx}\PYG{o}{=}\PYG{n}{a}\PYG{o}{.}\PYG{n}{create1D}\PYG{p}{(}\PYG{l+s+s1}{\PYGZsq{}}\PYG{l+s+s1}{new}\PYG{l+s+s1}{\PYGZsq{}}\PYG{p}{,}\PYG{l+s+s1}{\PYGZsq{}}\PYG{l+s+s1}{quick}\PYG{l+s+s1}{\PYGZsq{}}\PYG{p}{)}
\PYG{c+c1}{\PYGZsh{} Copies content of \PYGZsq{}default\PYGZsq{} to \PYGZsq{}new\PYGZsq{}}
\PYG{n}{yxx}\PYG{o}{=}\PYG{n}{a}\PYG{o}{.}\PYG{n}{create1D}\PYG{p}{(}\PYG{l+s+s1}{\PYGZsq{}}\PYG{l+s+s1}{new}\PYG{l+s+s1}{\PYGZsq{}}\PYG{p}{)}
\end{Verbatim}

\end{fulllineitems}



\begin{fulllineitems}
\pysigline{\sphinxbfcode{Modify~an~existing~Yxvsx:}}~\begin{itemize}
\item {} 
Get a YXvsX object to work with:
\begin{quote}

\begin{Verbatim}[commandchars=\\\{\}]
\PYG{n}{yxx}\PYG{o}{=}\PYG{n}{a}\PYG{o}{.}\PYG{n}{get1D}\PYG{p}{(}\PYG{l+s+s1}{\PYGZsq{}}\PYG{l+s+s1}{AMIP\PYGZus{}psl}\PYG{l+s+s1}{\PYGZsq{}}\PYG{p}{)}
\end{Verbatim}
\end{quote}

\item {} 
Overview of YXvsX attributes:
\begin{quote}
\begin{itemize}
\item {} 
To view YXvsX attributes:
\begin{quote}

\begin{Verbatim}[commandchars=\\\{\}]
\PYG{c+c1}{\PYGZsh{} Will list all the Yxvsx attribute values}
\PYG{n}{yxx}\PYG{o}{.}\PYG{n}{list}\PYG{p}{(}\PYG{p}{)}
\end{Verbatim}
\end{quote}

\item {} 
To set the projection attribute:
\begin{quote}

\begin{Verbatim}[commandchars=\\\{\}]
\PYG{n}{yxx}\PYG{o}{.}\PYG{n}{projection}\PYG{o}{=}\PYG{l+s+s1}{\PYGZsq{}}\PYG{l+s+s1}{linear}\PYG{l+s+s1}{\PYGZsq{}}
\end{Verbatim}

\begin{notice}{note}{Note:}
YXvsX projection attribute can only be `linear'
i.e. lon30=\{-180:`180W',-150:`150W',0:'Eq'\}
\end{notice}
\end{quote}

\item {} 
To set axis attributes:
\begin{quote}

\begin{Verbatim}[commandchars=\\\{\}]
\PYG{n}{yxx}\PYG{o}{.}\PYG{n}{xticlabels1}\PYG{o}{=}\PYG{n}{lon30}
\PYG{n}{yxx}\PYG{o}{.}\PYG{n}{xticlabels2}\PYG{o}{=}\PYG{n}{lon30}
\PYG{c+c1}{\PYGZsh{} Will set them both}
\PYG{n}{yxx}\PYG{o}{.}\PYG{n}{xticlabels}\PYG{p}{(}\PYG{n}{lon30}\PYG{p}{,} \PYG{n}{lon30}\PYG{p}{)}
\PYG{n}{yxx}\PYG{o}{.}\PYG{n}{xmtics1}\PYG{o}{=}\PYG{l+s+s1}{\PYGZsq{}}\PYG{l+s+s1}{\PYGZsq{}}
\PYG{n}{yxx}\PYG{o}{.}\PYG{n}{xmtics2}\PYG{o}{=}\PYG{l+s+s1}{\PYGZsq{}}\PYG{l+s+s1}{\PYGZsq{}}
\PYG{c+c1}{\PYGZsh{} Will set them both}
\PYG{n}{yxx}\PYG{o}{.}\PYG{n}{xmtics}\PYG{p}{(}\PYG{n}{lon30}\PYG{p}{,} \PYG{n}{lon30}\PYG{p}{)}
\PYG{n}{yxx}\PYG{o}{.}\PYG{n}{yticlabels1}\PYG{o}{=}\PYG{n}{lat10}
\PYG{n}{yxx}\PYG{o}{.}\PYG{n}{yticlabels2}\PYG{o}{=}\PYG{n}{lat10}
\PYG{c+c1}{\PYGZsh{} Will set them both}
\PYG{n}{yxx}\PYG{o}{.}\PYG{n}{yticlabels}\PYG{p}{(}\PYG{n}{lat10}\PYG{p}{,} \PYG{n}{lat10}\PYG{p}{)}
\PYG{n}{yxx}\PYG{o}{.}\PYG{n}{ymtics1}\PYG{o}{=}\PYG{l+s+s1}{\PYGZsq{}}\PYG{l+s+s1}{\PYGZsq{}}
\PYG{n}{yxx}\PYG{o}{.}\PYG{n}{ymtics2}\PYG{o}{=}\PYG{l+s+s1}{\PYGZsq{}}\PYG{l+s+s1}{\PYGZsq{}}
\PYG{c+c1}{\PYGZsh{} Will set them both}
\PYG{n}{yxx}\PYG{o}{.}\PYG{n}{ymtics}\PYG{p}{(}\PYG{n}{lat10}\PYG{p}{,} \PYG{n}{lat10}\PYG{p}{)}
\PYG{n}{yxx}\PYG{o}{.}\PYG{n}{datawc\PYGZus{}y1}\PYG{o}{=}\PYG{o}{\PYGZhy{}}\PYG{l+m+mf}{90.0}
\PYG{n}{yxx}\PYG{o}{.}\PYG{n}{datawc\PYGZus{}y2}\PYG{o}{=}\PYG{l+m+mf}{90.0}
\PYG{n}{yxx}\PYG{o}{.}\PYG{n}{datawc\PYGZus{}x1}\PYG{o}{=}\PYG{o}{\PYGZhy{}}\PYG{l+m+mf}{180.0}
\PYG{n}{yxx}\PYG{o}{.}\PYG{n}{datawc\PYGZus{}x2}\PYG{o}{=}\PYG{l+m+mf}{180.0}
\PYG{c+c1}{\PYGZsh{} Will set them all}
\PYG{n}{yxx}\PYG{o}{.}\PYG{n}{datawc}\PYG{p}{(}\PYG{o}{\PYGZhy{}}\PYG{l+m+mi}{90}\PYG{p}{,} \PYG{l+m+mi}{90}\PYG{p}{,} \PYG{o}{\PYGZhy{}}\PYG{l+m+mi}{180}\PYG{p}{,} \PYG{l+m+mi}{180}\PYG{p}{)}
\PYG{n}{yxx}\PYG{o}{.}\PYG{n}{xaxisconvert}\PYG{o}{=}\PYG{l+s+s1}{\PYGZsq{}}\PYG{l+s+s1}{linear}\PYG{l+s+s1}{\PYGZsq{}}
\end{Verbatim}
\end{quote}

\item {} 
To specify the Yxvsx line type:
\begin{quote}

\begin{Verbatim}[commandchars=\\\{\}]
\PYG{c+c1}{\PYGZsh{} same as yxx.line = \PYGZsq{}solid\PYGZsq{}}
\PYG{n}{yxx}\PYG{o}{.}\PYG{n}{line}\PYG{o}{=}\PYG{l+m+mi}{0}
\PYG{c+c1}{\PYGZsh{} same as yxx.line = \PYGZsq{}dash\PYGZsq{}}
\PYG{n}{yxx}\PYG{o}{.}\PYG{n}{line}\PYG{o}{=}\PYG{l+m+mi}{1}
\PYG{c+c1}{\PYGZsh{} same as yxx.line = \PYGZsq{}dot\PYGZsq{}}
\PYG{n}{yxx}\PYG{o}{.}\PYG{n}{line}\PYG{o}{=}\PYG{l+m+mi}{2}
\PYG{c+c1}{\PYGZsh{} same as yxx.line = \PYGZsq{}dash\PYGZhy{}dot\PYGZsq{}}
\PYG{n}{yxx}\PYG{o}{.}\PYG{n}{line}\PYG{o}{=}\PYG{l+m+mi}{3}
\PYG{c+c1}{\PYGZsh{} same as yxx.line = \PYGZsq{}long\PYGZhy{}dash}
\PYG{n}{yxx}\PYG{o}{.}\PYG{n}{line}\PYG{o}{=}\PYG{l+m+mi}{4}
\end{Verbatim}
\end{quote}

\item {} 
To specify the Yxvsx line color:
\begin{quote}

\begin{Verbatim}[commandchars=\\\{\}]
\PYG{c+c1}{\PYGZsh{} color range: 16 to 230, default color is black}
\PYG{n}{yxx}\PYG{o}{.}\PYG{n}{linecolor}\PYG{o}{=}\PYG{l+m+mi}{16}
\PYG{c+c1}{\PYGZsh{} width range: 1 to 100, default color is 1}
\PYG{n}{yxx}\PYG{o}{.}\PYG{n}{linewidth}\PYG{o}{=}\PYG{l+m+mi}{1}
\end{Verbatim}
\end{quote}

\item {} 
To specify the Yxvsx marker type:
\begin{quote}

\begin{Verbatim}[commandchars=\\\{\}]
\PYG{c+c1}{\PYGZsh{} Same as yxx.marker=\PYGZsq{}dot\PYGZsq{}}
\PYG{n}{yxx}\PYG{o}{.}\PYG{n}{marker}\PYG{o}{=}\PYG{l+m+mi}{1}
\PYG{c+c1}{\PYGZsh{} Same as yxx.marker=\PYGZsq{}plus\PYGZsq{}}
\PYG{n}{yxx}\PYG{o}{.}\PYG{n}{marker}\PYG{o}{=}\PYG{l+m+mi}{2}
\PYG{c+c1}{\PYGZsh{} Same as yxx.marker=\PYGZsq{}star\PYGZsq{}}
\PYG{n}{yxx}\PYG{o}{.}\PYG{n}{marker}\PYG{o}{=}\PYG{l+m+mi}{3}
\PYG{c+c1}{\PYGZsh{} Same as yxx.marker=\PYGZsq{}circle\PYGZsq{}}
\PYG{n}{yxx}\PYG{o}{.}\PYG{n}{marker}\PYG{o}{=}\PYG{l+m+mi}{4}
\PYG{c+c1}{\PYGZsh{} Same as yxx.marker=\PYGZsq{}cross\PYGZsq{}}
\PYG{n}{yxx}\PYG{o}{.}\PYG{n}{marker}\PYG{o}{=}\PYG{l+m+mi}{5}
\PYG{c+c1}{\PYGZsh{} Same as yxx.marker=\PYGZsq{}diamond\PYGZsq{}}
\PYG{n}{yxx}\PYG{o}{.}\PYG{n}{marker}\PYG{o}{=}\PYG{l+m+mi}{6}
\PYG{c+c1}{\PYGZsh{} Same as yxx.marker=\PYGZsq{}triangle\PYGZus{}up\PYGZsq{}}
\PYG{n}{yxx}\PYG{o}{.}\PYG{n}{marker}\PYG{o}{=}\PYG{l+m+mi}{7}
\PYG{c+c1}{\PYGZsh{} Same as yxx.marker=\PYGZsq{}triangle\PYGZus{}down\PYGZsq{}}
\PYG{n}{yxx}\PYG{o}{.}\PYG{n}{marker}\PYG{o}{=}\PYG{l+m+mi}{8}
\PYG{c+c1}{\PYGZsh{} Same as yxx.marker=\PYGZsq{}triangle\PYGZus{}left\PYGZsq{}}
\PYG{n}{yxx}\PYG{o}{.}\PYG{n}{marker}\PYG{o}{=}\PYG{l+m+mi}{9}
\PYG{c+c1}{\PYGZsh{} Same as yxx.marker=\PYGZsq{}triangle\PYGZus{}right\PYGZsq{}}
\PYG{n}{yxx}\PYG{o}{.}\PYG{n}{marker}\PYG{o}{=}\PYG{l+m+mi}{10}
\PYG{c+c1}{\PYGZsh{} Same as yxx.marker=\PYGZsq{}square\PYGZsq{}}
\PYG{n}{yxx}\PYG{o}{.}\PYG{n}{marker}\PYG{o}{=}\PYG{l+m+mi}{11}
\PYG{c+c1}{\PYGZsh{} Same as yxx.marker=\PYGZsq{}diamond\PYGZus{}fill\PYGZsq{}}
\PYG{n}{yxx}\PYG{o}{.}\PYG{n}{marker}\PYG{o}{=}\PYG{l+m+mi}{12}
\PYG{c+c1}{\PYGZsh{} Same as yxx.marker=\PYGZsq{}triangle\PYGZus{}up\PYGZus{}fill\PYGZsq{}}
\PYG{n}{yxx}\PYG{o}{.}\PYG{n}{marker}\PYG{o}{=}\PYG{l+m+mi}{13}
\PYG{c+c1}{\PYGZsh{} Same as yxx.marker=\PYGZsq{}triangle\PYGZus{}down\PYGZus{}fill\PYGZsq{}}
\PYG{n}{yxx}\PYG{o}{.}\PYG{n}{marker}\PYG{o}{=}\PYG{l+m+mi}{14}
\PYG{c+c1}{\PYGZsh{} Same as yxx.marker=\PYGZsq{}triangle\PYGZus{}left\PYGZus{}fill\PYGZsq{}}
\PYG{n}{yxx}\PYG{o}{.}\PYG{n}{marker}\PYG{o}{=}\PYG{l+m+mi}{15}
\PYG{c+c1}{\PYGZsh{} Same as yxx.marker=\PYGZsq{}triangle\PYGZus{}right\PYGZus{}fill\PYGZsq{}}
\PYG{n}{yxx}\PYG{o}{.}\PYG{n}{marker}\PYG{o}{=}\PYG{l+m+mi}{16}
\PYG{c+c1}{\PYGZsh{} Same as yxx.marker=\PYGZsq{}square\PYGZus{}fill\PYGZsq{}}
\PYG{n}{yxx}\PYG{o}{.}\PYG{n}{marker}\PYG{o}{=}\PYG{l+m+mi}{17}
\PYG{c+c1}{\PYGZsh{} Draw no markers}
\PYG{n}{yxx}\PYG{o}{.}\PYG{n}{marker}\PYG{o}{=}\PYG{n+nb+bp}{None}
\end{Verbatim}
\end{quote}

\item {} 
There are four possibilities for setting the marker color index:
\begin{quote}

\begin{Verbatim}[commandchars=\\\{\}]
\PYG{c+c1}{\PYGZsh{} Same as below}
\PYG{n}{yxx}\PYG{o}{.}\PYG{n}{markercolors}\PYG{o}{=}\PYG{l+m+mi}{22}
\PYG{c+c1}{\PYGZsh{} Same as below}
\PYG{n}{yxx}\PYG{o}{.}\PYG{n}{markercolors}\PYG{o}{=}\PYG{p}{(}\PYG{l+m+mi}{22}\PYG{p}{)}
\PYG{c+c1}{\PYGZsh{} Will set the markers to a specific color index}
\PYG{n}{yxx}\PYG{o}{.}\PYG{n}{markercolors}\PYG{o}{=}\PYG{p}{(}\PYG{p}{[}\PYG{l+m+mi}{22}\PYG{p}{]}\PYG{p}{)}
\PYG{c+c1}{\PYGZsh{} Color index defaults to Black}
\PYG{n}{yxx}\PYG{o}{.}\PYG{n}{markercolors}\PYG{o}{=}\PYG{n+nb+bp}{None}
\end{Verbatim}
\end{quote}

\item {} 
To set the Yxvsx Marker size:
\begin{quote}

\begin{Verbatim}[commandchars=\\\{\}]
\PYG{n}{yxx}\PYG{o}{.}\PYG{n}{markersize}\PYG{o}{=}\PYG{l+m+mi}{5}
\PYG{n}{yxx}\PYG{o}{.}\PYG{n}{markersize}\PYG{o}{=}\PYG{l+m+mi}{55}
\PYG{n}{yxx}\PYG{o}{.}\PYG{n}{markersize}\PYG{o}{=}\PYG{l+m+mi}{100}
\PYG{n}{yxx}\PYG{o}{.}\PYG{n}{markersize}\PYG{o}{=}\PYG{l+m+mi}{300}
\PYG{n}{yxx}\PYG{o}{.}\PYG{n}{markersize}\PYG{o}{=}\PYG{n+nb+bp}{None}
\end{Verbatim}
\end{quote}

\end{itemize}
\index{xmtics1 (vcs.unified1D.G1d attribute)}

\begin{fulllineitems}
\phantomsection\label{vcs/graphics/unified1D:vcs.unified1D.G1d.xmtics1}\pysiglinewithargsret{\sphinxbfcode{xmtics1}}{\emph{str/\{float:str\}}}{}
(Ex: `') dictionary with location of intermediate tics as keys for 1st side of y axis

\end{fulllineitems}

\index{xmtics2 (vcs.unified1D.G1d attribute)}

\begin{fulllineitems}
\phantomsection\label{vcs/graphics/unified1D:vcs.unified1D.G1d.xmtics2}\pysiglinewithargsret{\sphinxbfcode{xmtics2}}{\emph{str/\{float:str\}}}{}
(Ex: `') dictionary with location of intermediate tics as keys for 2nd side of y axis

\end{fulllineitems}

\index{ymtics1 (vcs.unified1D.G1d attribute)}

\begin{fulllineitems}
\phantomsection\label{vcs/graphics/unified1D:vcs.unified1D.G1d.ymtics1}\pysiglinewithargsret{\sphinxbfcode{ymtics1}}{\emph{str/\{float:str\}}}{}
(Ex: `') dictionary with location of intermediate tics as keys for 1st side of y axis

\end{fulllineitems}

\index{ymtics2 (vcs.unified1D.G1d attribute)}

\begin{fulllineitems}
\phantomsection\label{vcs/graphics/unified1D:vcs.unified1D.G1d.ymtics2}\pysiglinewithargsret{\sphinxbfcode{ymtics2}}{\emph{str/\{float:str\}}}{}
(Ex: `') dictionary with location of intermediate tics as keys for 2nd side of y axis

\end{fulllineitems}

\index{xticlabels1 (vcs.unified1D.G1d attribute)}

\begin{fulllineitems}
\phantomsection\label{vcs/graphics/unified1D:vcs.unified1D.G1d.xticlabels1}\pysiglinewithargsret{\sphinxbfcode{xticlabels1}}{\emph{str/\{float:str\}}}{}
(Ex: `*') values for labels on 1st side of x axis

\end{fulllineitems}

\index{xticlabels2 (vcs.unified1D.G1d attribute)}

\begin{fulllineitems}
\phantomsection\label{vcs/graphics/unified1D:vcs.unified1D.G1d.xticlabels2}\pysiglinewithargsret{\sphinxbfcode{xticlabels2}}{\emph{str/\{float:str\}}}{}
(Ex: `*') values for labels on 2nd side of x axis

\end{fulllineitems}

\index{yticlabels1 (vcs.unified1D.G1d attribute)}

\begin{fulllineitems}
\phantomsection\label{vcs/graphics/unified1D:vcs.unified1D.G1d.yticlabels1}\pysiglinewithargsret{\sphinxbfcode{yticlabels1}}{\emph{str/\{float:str\}}}{}
(Ex: `*') values for labels on 1st side of y axis

\end{fulllineitems}

\index{yticlabels2 (vcs.unified1D.G1d attribute)}

\begin{fulllineitems}
\phantomsection\label{vcs/graphics/unified1D:vcs.unified1D.G1d.yticlabels2}\pysiglinewithargsret{\sphinxbfcode{yticlabels2}}{\emph{str/\{float:str\}}}{}
(Ex: `*') values for labels on 2nd side of y axis

\end{fulllineitems}

\index{projection (vcs.unified1D.G1d attribute)}

\begin{fulllineitems}
\phantomsection\label{vcs/graphics/unified1D:vcs.unified1D.G1d.projection}\pysiglinewithargsret{\sphinxbfcode{projection}}{\emph{str/vcs.projection.Proj}}{}
(Ex: `default') projection to use, name or object

\end{fulllineitems}

\index{datawc\_x1 (vcs.unified1D.G1d attribute)}

\begin{fulllineitems}
\phantomsection\label{vcs/graphics/unified1D:vcs.unified1D.G1d.datawc_x1}\pysiglinewithargsret{\sphinxbfcode{datawc\_x1}}{\emph{float}}{}
(Ex: 1.E20) first value of xaxis on plot

\end{fulllineitems}

\index{datawc\_x2 (vcs.unified1D.G1d attribute)}

\begin{fulllineitems}
\phantomsection\label{vcs/graphics/unified1D:vcs.unified1D.G1d.datawc_x2}\pysiglinewithargsret{\sphinxbfcode{datawc\_x2}}{\emph{float}}{}
(Ex: 1.E20) second value of xaxis on plot

\end{fulllineitems}

\index{datawc\_y1 (vcs.unified1D.G1d attribute)}

\begin{fulllineitems}
\phantomsection\label{vcs/graphics/unified1D:vcs.unified1D.G1d.datawc_y1}\pysiglinewithargsret{\sphinxbfcode{datawc\_y1}}{\emph{float}}{}
(Ex: 1.E20) first value of yaxis on plot

\end{fulllineitems}

\index{datawc\_y2 (vcs.unified1D.G1d attribute)}

\begin{fulllineitems}
\phantomsection\label{vcs/graphics/unified1D:vcs.unified1D.G1d.datawc_y2}\pysiglinewithargsret{\sphinxbfcode{datawc\_y2}}{\emph{float}}{}
(Ex: 1.E20) second value of yaxis on plot

\end{fulllineitems}

\index{datawc\_timeunits (vcs.unified1D.G1d attribute)}

\begin{fulllineitems}
\phantomsection\label{vcs/graphics/unified1D:vcs.unified1D.G1d.datawc_timeunits}\pysiglinewithargsret{\sphinxbfcode{datawc\_timeunits}}{\emph{str}}{}
(Ex: `days since 2000') units to use when displaying time dimension auto tick

\end{fulllineitems}

\index{datawc\_calendar (vcs.unified1D.G1d attribute)}

\begin{fulllineitems}
\phantomsection\label{vcs/graphics/unified1D:vcs.unified1D.G1d.datawc_calendar}\pysiglinewithargsret{\sphinxbfcode{datawc\_calendar}}{\emph{int}}{}
(Ex: 135441) calendar to use when displaying time dimension auto tick, default is proleptic gregorian calendar

\end{fulllineitems}

\end{quote}

\end{itemize}

\end{fulllineitems}

\begin{quote}\begin{description}
\item[{Parameters}] \leavevmode
\textbf{\texttt{xaxisconvert}} -- (Ex: `linear') converting xaxis linear/log/log10/ln/exp/area\_wt

\end{description}\end{quote}

linecolor :: (int) (241) colors to use for each isoline
linewidth :: (float) (1.0) list of width for each isoline

marker :: (None/int/str/vcs.marker.Tm) (None) markers type to use
markercolor :: (None/int) (None) color to use for markers
markersize :: (None/int) (None) size of markers
\index{datawc() (vcs.unified1D.G1d method)}

\begin{fulllineitems}
\phantomsection\label{vcs/graphics/unified1D:vcs.unified1D.G1d.datawc}\pysiglinewithargsret{\sphinxbfcode{datawc}}{\emph{dsp1=1e+20}, \emph{dsp2=1e+20}, \emph{dsp3=1e+20}, \emph{dsp4=1e+20}}{}
Sets the data world coordinates for object
\begin{quote}\begin{description}
\item[{Parameters}] \leavevmode\begin{itemize}
\item {} 
\textbf{\texttt{dsp1}} (\href{https://docs.python.org/2/library/functions.html\#float}{\emph{\texttt{float}}}) -- Sets the {\hyperref[vcs/graphics/unified1D:vcs.unified1D.G1d.datawc_y1]{\sphinxcrossref{\sphinxcode{datawc\_y1}}}} property of the object.

\item {} 
\textbf{\texttt{dsp2}} (\href{https://docs.python.org/2/library/functions.html\#float}{\emph{\texttt{float}}}) -- Sets the {\hyperref[vcs/graphics/unified1D:vcs.unified1D.G1d.datawc_y2]{\sphinxcrossref{\sphinxcode{datawc\_y2}}}} property of the object.

\item {} 
\textbf{\texttt{dsp3}} (\href{https://docs.python.org/2/library/functions.html\#float}{\emph{\texttt{float}}}) -- Sets the {\hyperref[vcs/graphics/unified1D:vcs.unified1D.G1d.datawc_x1]{\sphinxcrossref{\sphinxcode{datawc\_x1}}}} property of the object.

\item {} 
\textbf{\texttt{dsp4}} (\href{https://docs.python.org/2/library/functions.html\#float}{\emph{\texttt{float}}}) -- Sets the {\hyperref[vcs/graphics/unified1D:vcs.unified1D.G1d.datawc_x2]{\sphinxcrossref{\sphinxcode{datawc\_x2}}}} property of the object.

\end{itemize}

\end{description}\end{quote}

\end{fulllineitems}

\index{g\_type (vcs.unified1D.G1d attribute)}

\begin{fulllineitems}
\phantomsection\label{vcs/graphics/unified1D:vcs.unified1D.G1d.g_type}\pysigline{\sphinxbfcode{g\_type}}
the 1d graphics method type

\end{fulllineitems}

\index{list() (vcs.unified1D.G1d method)}

\begin{fulllineitems}
\phantomsection\label{vcs/graphics/unified1D:vcs.unified1D.G1d.list}\pysiglinewithargsret{\sphinxbfcode{list}}{}{}
Lists the current values of object attributes

\end{fulllineitems}

\index{script() (vcs.unified1D.G1d method)}

\begin{fulllineitems}
\phantomsection\label{vcs/graphics/unified1D:vcs.unified1D.G1d.script}\pysiglinewithargsret{\sphinxbfcode{script}}{\emph{script\_filename}, \emph{mode='a'}}{}
Saves out a copy of the yxvsx graphics method in JSON, or Python format to a designated file.
\begin{quote}

\begin{notice}{note}{Note:}
If the the filename has a `.py' at the end, it will produce a
Python script. If no extension is given, then by default a
.json file containing a JSON serialization of the object's
data will be produced.
\end{notice}

\begin{notice}{warning}{Warning:}
VCS Scripts Deprecated.
SCR script files are no longer generated by this function.
\end{notice}
\end{quote}
\begin{quote}\begin{description}
\item[{Example}] \leavevmode
\begin{Verbatim}[commandchars=\\\{\}]
\PYG{g+gp}{\PYGZgt{}\PYGZgt{}\PYGZgt{} }\PYG{n}{a}\PYG{o}{=}\PYG{n}{vcs}\PYG{o}{.}\PYG{n}{init}\PYG{p}{(}\PYG{p}{)} \PYG{c+c1}{\PYGZsh{} Make a Canvas object to work with}
\PYG{g+gp}{\PYGZgt{}\PYGZgt{}\PYGZgt{} }\PYG{n}{ex}\PYG{o}{=}\PYG{n}{a}\PYG{o}{.}\PYG{n}{getyxvsx}\PYG{p}{(}\PYG{p}{)} \PYG{c+c1}{\PYGZsh{} Get default yxvsx}
\PYG{g+gp}{\PYGZgt{}\PYGZgt{}\PYGZgt{} }\PYG{n}{ex}\PYG{o}{.}\PYG{n}{script}\PYG{p}{(}\PYG{l+s+s1}{\PYGZsq{}}\PYG{l+s+s1}{filename.py}\PYG{l+s+s1}{\PYGZsq{}}\PYG{p}{)} \PYG{c+c1}{\PYGZsh{} Append to a Python script named \PYGZsq{}filename.py\PYGZsq{}}
\PYG{g+gp}{\PYGZgt{}\PYGZgt{}\PYGZgt{} }\PYG{n}{ex}\PYG{o}{.}\PYG{n}{script}\PYG{p}{(}\PYG{l+s+s1}{\PYGZsq{}}\PYG{l+s+s1}{filename}\PYG{l+s+s1}{\PYGZsq{}}\PYG{p}{,}\PYG{l+s+s1}{\PYGZsq{}}\PYG{l+s+s1}{w}\PYG{l+s+s1}{\PYGZsq{}}\PYG{p}{)} \PYG{c+c1}{\PYGZsh{} Create or overwrite a JSON file \PYGZsq{}filename.json\PYGZsq{}.}
\end{Verbatim}

\item[{Parameters}] \leavevmode\begin{itemize}
\item {} 
\textbf{\texttt{script\_filename}} (\href{https://docs.python.org/2/library/functions.html\#str}{\emph{\texttt{str}}}) -- Output name of the script file. If no extension is specified, a .json object is created.

\item {} 
\textbf{\texttt{mode}} (\href{https://docs.python.org/2/library/functions.html\#str}{\emph{\texttt{str}}}) -- Either `w' for replace, or `a' for append. Defaults to `a', if not specified.

\end{itemize}

\end{description}\end{quote}

\end{fulllineitems}

\index{smooth (vcs.unified1D.G1d attribute)}

\begin{fulllineitems}
\phantomsection\label{vcs/graphics/unified1D:vcs.unified1D.G1d.smooth}\pysigline{\sphinxbfcode{smooth}}
beta parameter for kaiser smoothing

\end{fulllineitems}

\index{xmtics() (vcs.unified1D.G1d method)}

\begin{fulllineitems}
\phantomsection\label{vcs/graphics/unified1D:vcs.unified1D.G1d.xmtics}\pysiglinewithargsret{\sphinxbfcode{xmtics}}{\emph{xmt1='`}, \emph{xmt2='`}}{}
Sets the xmtics1 and xmtics2 values on the object
\begin{quote}\begin{description}
\item[{Parameters}] \leavevmode\begin{itemize}
\item {} 
\textbf{\texttt{xmt1}} (\emph{\texttt{\{float:str\} or str}}) -- Value for {\hyperref[vcs/graphics/unified1D:vcs.unified1D.G1d.xmtics1]{\sphinxcrossref{\sphinxcode{xmtics1}}}}. Must be a str, or a dictionary object with float:str mappings.

\item {} 
\textbf{\texttt{xmt2}} (\emph{\texttt{\{float:str\} or str}}) -- Value for {\hyperref[vcs/graphics/unified1D:vcs.unified1D.G1d.xmtics2]{\sphinxcrossref{\sphinxcode{xmtics2}}}}. Must be a str, or a dictionary object with float:str mappings.

\end{itemize}

\end{description}\end{quote}

\end{fulllineitems}

\index{xticlabels() (vcs.unified1D.G1d method)}

\begin{fulllineitems}
\phantomsection\label{vcs/graphics/unified1D:vcs.unified1D.G1d.xticlabels}\pysiglinewithargsret{\sphinxbfcode{xticlabels}}{\emph{xtl1='`}, \emph{xtl2='`}}{}
Sets the xticlabels1 and xticlabels2 values on the object
\begin{quote}\begin{description}
\item[{Parameters}] \leavevmode\begin{itemize}
\item {} 
\textbf{\texttt{xtl1}} (\emph{\texttt{\{float:str\} or str}}) -- Sets the object's value for {\hyperref[vcs/graphics/unified1D:vcs.unified1D.G1d.xticlabels1]{\sphinxcrossref{\sphinxcode{xticlabels1}}}}. Must be  a str, or a dictionary object with float:str mappings.

\item {} 
\textbf{\texttt{xtl2}} (\emph{\texttt{\{float:str\} or str}}) -- Sets the object's value for {\hyperref[vcs/graphics/unified1D:vcs.unified1D.G1d.xticlabels2]{\sphinxcrossref{\sphinxcode{xticlabels2}}}}. Must be a str, or a dictionary object with float:str mappings.

\end{itemize}

\end{description}\end{quote}

\end{fulllineitems}

\index{ymtics() (vcs.unified1D.G1d method)}

\begin{fulllineitems}
\phantomsection\label{vcs/graphics/unified1D:vcs.unified1D.G1d.ymtics}\pysiglinewithargsret{\sphinxbfcode{ymtics}}{\emph{ymt1='`}, \emph{ymt2='`}}{}
Sets the ymtics1 and ymtics2 values on the object
\begin{quote}\begin{description}
\item[{Parameters}] \leavevmode\begin{itemize}
\item {} 
\textbf{\texttt{ymt1}} (\emph{\texttt{\{float:str\} or str}}) -- Value for {\hyperref[vcs/graphics/unified1D:vcs.unified1D.G1d.ymtics1]{\sphinxcrossref{\sphinxcode{ymtics1}}}}. Must be a str, or a dictionary object with float:str mappings.

\item {} 
\textbf{\texttt{ymt2}} (\emph{\texttt{\{float:str\} or str}}) -- Value for {\hyperref[vcs/graphics/unified1D:vcs.unified1D.G1d.ymtics2]{\sphinxcrossref{\sphinxcode{ymtics2}}}}. Must be a str, or a dictionary object with float:str mappings.

\end{itemize}

\end{description}\end{quote}

\end{fulllineitems}

\index{yticlabels() (vcs.unified1D.G1d method)}

\begin{fulllineitems}
\phantomsection\label{vcs/graphics/unified1D:vcs.unified1D.G1d.yticlabels}\pysiglinewithargsret{\sphinxbfcode{yticlabels}}{\emph{ytl1='`}, \emph{ytl2='`}}{}
Sets the yticlabels1 and yticlabels2 values on the object
\begin{quote}\begin{description}
\item[{Parameters}] \leavevmode\begin{itemize}
\item {} 
\textbf{\texttt{ytl1}} (\emph{\texttt{\{float:str\} or str}}) -- Sets the object's value for {\hyperref[vcs/graphics/unified1D:vcs.unified1D.G1d.yticlabels1]{\sphinxcrossref{\sphinxcode{yticlabels1}}}}. Must be  a str, or a dictionary object with float:str mappings.

\item {} 
\textbf{\texttt{ytl2}} (\emph{\texttt{\{float:str\} or str}}) -- Sets the object's value for {\hyperref[vcs/graphics/unified1D:vcs.unified1D.G1d.yticlabels2]{\sphinxcrossref{\sphinxcode{yticlabels2}}}}. Must be a str, or a dictionary object with float:str mappings.

\end{itemize}

\end{description}\end{quote}

\end{fulllineitems}


\end{fulllineitems}



\subsection{vector}
\label{vcs/graphics/vector:vector}\label{vcs/graphics/vector::doc}\label{vcs/graphics/vector:module-vcs.vector}\index{vcs.vector (module)}
\# Vector (Gv) module
\index{Gv (class in vcs.vector)}

\begin{fulllineitems}
\phantomsection\label{vcs/graphics/vector:vcs.vector.Gv}\pysiglinewithargsret{\sphinxstrong{class }\sphinxcode{vcs.vector.}\sphinxbfcode{Gv}}{\emph{Gv\_name}, \emph{Gv\_name\_src='default'}}{}
The vector graphics method displays a vector plot of a 2D vector field. Vectors
are located at the coordinate locations and point in the direction of the data
vector field. Vector magnitudes are the product of data vector field lengths and
a scaling factor. The example below shows how to modify the vector's line, scale,
alignment, type, and reference.

This class is used to define an vector table entry used in VCS, or it  can be
used to change some or all of the vector attributes in an existing vector table
entry.


\begin{fulllineitems}
\pysigline{\sphinxbfcode{Useful~Functions:}}~
\begin{Verbatim}[commandchars=\\\{\}]
\PYG{c+c1}{\PYGZsh{} Constructor}
\PYG{n}{a}\PYG{o}{=}\PYG{n}{vcs}\PYG{o}{.}\PYG{n}{init}\PYG{p}{(}\PYG{p}{)}
\PYG{c+c1}{\PYGZsh{} Show predefined vector graphics methods}
\PYG{n}{a}\PYG{o}{.}\PYG{n}{show}\PYG{p}{(}\PYG{l+s+s1}{\PYGZsq{}}\PYG{l+s+s1}{vector}\PYG{l+s+s1}{\PYGZsq{}}\PYG{p}{)}
\PYG{c+c1}{\PYGZsh{} Show predefined VCS line objects}
\PYG{n}{a}\PYG{o}{.}\PYG{n}{show}\PYG{p}{(}\PYG{l+s+s1}{\PYGZsq{}}\PYG{l+s+s1}{line}\PYG{l+s+s1}{\PYGZsq{}}\PYG{p}{)}
\PYG{c+c1}{\PYGZsh{} Change the VCS color Map}
\PYG{n}{a}\PYG{o}{.}\PYG{n}{setcolormap}\PYG{p}{(}\PYG{l+s+s2}{\PYGZdq{}}\PYG{l+s+s2}{AMIP}\PYG{l+s+s2}{\PYGZdq{}}\PYG{p}{)}
\PYG{c+c1}{\PYGZsh{} Plot data \PYGZsq{}s1\PYGZsq{}, and \PYGZsq{}s2\PYGZsq{} with vector \PYGZsq{}v\PYGZsq{} and \PYGZsq{}default\PYGZsq{} template}
\PYG{n}{a}\PYG{o}{.}\PYG{n}{vector}\PYG{p}{(}\PYG{n}{s1}\PYG{p}{,} \PYG{n}{s2}\PYG{p}{,} \PYG{n}{v}\PYG{p}{,}\PYG{l+s+s1}{\PYGZsq{}}\PYG{l+s+s1}{default}\PYG{l+s+s1}{\PYGZsq{}}\PYG{p}{)}
\PYG{c+c1}{\PYGZsh{} Updates the VCS Canvas at user\PYGZsq{}s request}
\PYG{n}{a}\PYG{o}{.}\PYG{n}{update}\PYG{p}{(}\PYG{p}{)}
\end{Verbatim}

\end{fulllineitems}



\begin{fulllineitems}
\pysigline{\sphinxbfcode{Make~a~Canvas~object~to~work~with:}}~
\begin{Verbatim}[commandchars=\\\{\}]
\PYG{n}{a}\PYG{o}{=}\PYG{n}{vcs}\PYG{o}{.}\PYG{n}{init}\PYG{p}{(}\PYG{p}{)}
\end{Verbatim}

\end{fulllineitems}



\begin{fulllineitems}
\pysigline{\sphinxbfcode{Create~a~new~instance~of~vector:}}~
\begin{Verbatim}[commandchars=\\\{\}]
\PYG{c+c1}{\PYGZsh{} Copies content of \PYGZsq{}quick\PYGZsq{} to \PYGZsq{}new\PYGZsq{}}
\PYG{n}{vc}\PYG{o}{=}\PYG{n}{a}\PYG{o}{.}\PYG{n}{createvector}\PYG{p}{(}\PYG{l+s+s1}{\PYGZsq{}}\PYG{l+s+s1}{new}\PYG{l+s+s1}{\PYGZsq{}}\PYG{p}{,}\PYG{l+s+s1}{\PYGZsq{}}\PYG{l+s+s1}{quick}\PYG{l+s+s1}{\PYGZsq{}}\PYG{p}{)}
\PYG{c+c1}{\PYGZsh{} Copies content of \PYGZsq{}default\PYGZsq{} to \PYGZsq{}new\PYGZsq{}}
\PYG{n}{vc}\PYG{o}{=}\PYG{n}{a}\PYG{o}{.}\PYG{n}{createvector}\PYG{p}{(}\PYG{l+s+s1}{\PYGZsq{}}\PYG{l+s+s1}{new}\PYG{l+s+s1}{\PYGZsq{}}\PYG{p}{)}
\end{Verbatim}

\end{fulllineitems}



\begin{fulllineitems}
\pysigline{\sphinxbfcode{Modify~an~existing~vector:}}~
\begin{Verbatim}[commandchars=\\\{\}]
\PYG{n}{vc}\PYG{o}{=}\PYG{n}{a}\PYG{o}{.}\PYG{n}{getvector}\PYG{p}{(}\PYG{l+s+s1}{\PYGZsq{}}\PYG{l+s+s1}{AMIP\PYGZus{}psl}\PYG{l+s+s1}{\PYGZsq{}}\PYG{p}{)}
\end{Verbatim}

\end{fulllineitems}



\begin{fulllineitems}
\pysigline{\sphinxbfcode{Overview~of~vector~attributes:}}~\begin{itemize}
\item {} 
List all attributes:
\begin{quote}

\begin{Verbatim}[commandchars=\\\{\}]
\PYG{c+c1}{\PYGZsh{} Will list all the vector attribute values}
\PYG{n}{vc}\PYG{o}{.}\PYG{n}{list}\PYG{p}{(}\PYG{p}{)}
\end{Verbatim}
\end{quote}

\item {} 
Set axis attributes:
\begin{quote}

\begin{Verbatim}[commandchars=\\\{\}]
\PYG{c+c1}{\PYGZsh{} Can only be \PYGZsq{}linear\PYGZsq{}}
\PYG{n}{vc}\PYG{o}{.}\PYG{n}{projection}\PYG{o}{=}\PYG{l+s+s1}{\PYGZsq{}}\PYG{l+s+s1}{linear}\PYG{l+s+s1}{\PYGZsq{}}
\PYG{n}{lon30}\PYG{o}{=}\PYG{p}{\PYGZob{}}\PYG{o}{\PYGZhy{}}\PYG{l+m+mi}{180}\PYG{p}{:}\PYG{l+s+s1}{\PYGZsq{}}\PYG{l+s+s1}{180W}\PYG{l+s+s1}{\PYGZsq{}}\PYG{p}{,}\PYG{o}{\PYGZhy{}}\PYG{l+m+mi}{150}\PYG{p}{:}\PYG{l+s+s1}{\PYGZsq{}}\PYG{l+s+s1}{150W}\PYG{l+s+s1}{\PYGZsq{}}\PYG{p}{,}\PYG{l+m+mi}{0}\PYG{p}{:}\PYG{l+s+s1}{\PYGZsq{}}\PYG{l+s+s1}{Eq}\PYG{l+s+s1}{\PYGZsq{}}\PYG{p}{\PYGZcb{}}
\PYG{n}{vc}\PYG{o}{.}\PYG{n}{xticlabels1}\PYG{o}{=}\PYG{n}{lon30}
\PYG{n}{vc}\PYG{o}{.}\PYG{n}{xticlabels2}\PYG{o}{=}\PYG{n}{lon30}
\PYG{c+c1}{\PYGZsh{} Will set them both}
\PYG{n}{vc}\PYG{o}{.}\PYG{n}{xticlabels}\PYG{p}{(}\PYG{n}{lon30}\PYG{p}{,} \PYG{n}{lon30}\PYG{p}{)}
\PYG{n}{vc}\PYG{o}{.}\PYG{n}{xmtics1}\PYG{o}{=}\PYG{l+s+s1}{\PYGZsq{}}\PYG{l+s+s1}{\PYGZsq{}}
\PYG{n}{vc}\PYG{o}{.}\PYG{n}{xmtics2}\PYG{o}{=}\PYG{l+s+s1}{\PYGZsq{}}\PYG{l+s+s1}{\PYGZsq{}}
\PYG{c+c1}{\PYGZsh{} Will set them both}
\PYG{n}{vc}\PYG{o}{.}\PYG{n}{xmtics}\PYG{p}{(}\PYG{n}{lon30}\PYG{p}{,} \PYG{n}{lon30}\PYG{p}{)}
\PYG{n}{vc}\PYG{o}{.}\PYG{n}{yticlabels1}\PYG{o}{=}\PYG{n}{lat10}
\PYG{n}{vc}\PYG{o}{.}\PYG{n}{yticlabels2}\PYG{o}{=}\PYG{n}{lat10}
\PYG{c+c1}{\PYGZsh{} Will set them both}
\PYG{n}{vc}\PYG{o}{.}\PYG{n}{yticlabels}\PYG{p}{(}\PYG{n}{lat10}\PYG{p}{,} \PYG{n}{lat10}\PYG{p}{)}
\PYG{n}{vc}\PYG{o}{.}\PYG{n}{ymtics1}\PYG{o}{=}\PYG{l+s+s1}{\PYGZsq{}}\PYG{l+s+s1}{\PYGZsq{}}
\PYG{n}{vc}\PYG{o}{.}\PYG{n}{ymtics2}\PYG{o}{=}\PYG{l+s+s1}{\PYGZsq{}}\PYG{l+s+s1}{\PYGZsq{}}
\PYG{c+c1}{\PYGZsh{} Will set them both}
\PYG{n}{vc}\PYG{o}{.}\PYG{n}{ymtics}\PYG{p}{(}\PYG{n}{lat10}\PYG{p}{,} \PYG{n}{lat10}\PYG{p}{)}
\PYG{n}{vc}\PYG{o}{.}\PYG{n}{datawc\PYGZus{}y1}\PYG{o}{=}\PYG{o}{\PYGZhy{}}\PYG{l+m+mf}{90.0}
\PYG{n}{vc}\PYG{o}{.}\PYG{n}{datawc\PYGZus{}y2}\PYG{o}{=}\PYG{l+m+mf}{90.0}
\PYG{n}{vc}\PYG{o}{.}\PYG{n}{datawc\PYGZus{}x1}\PYG{o}{=}\PYG{o}{\PYGZhy{}}\PYG{l+m+mf}{180.0}
\PYG{n}{vc}\PYG{o}{.}\PYG{n}{datawc\PYGZus{}x2}\PYG{o}{=}\PYG{l+m+mf}{180.0}
\PYG{c+c1}{\PYGZsh{} Will set them all}
\PYG{n}{vc}\PYG{o}{.}\PYG{n}{datawc}\PYG{p}{(}\PYG{o}{\PYGZhy{}}\PYG{l+m+mi}{90}\PYG{p}{,} \PYG{l+m+mi}{90}\PYG{p}{,} \PYG{o}{\PYGZhy{}}\PYG{l+m+mi}{180}\PYG{p}{,} \PYG{l+m+mi}{180}\PYG{p}{)}
\PYG{n}{xaxisconvert}\PYG{o}{=}\PYG{l+s+s1}{\PYGZsq{}}\PYG{l+s+s1}{linear}\PYG{l+s+s1}{\PYGZsq{}}
\PYG{n}{yaxisconvert}\PYG{o}{=}\PYG{l+s+s1}{\PYGZsq{}}\PYG{l+s+s1}{linear}\PYG{l+s+s1}{\PYGZsq{}}
\PYG{c+c1}{\PYGZsh{} Will set them both}
\PYG{n}{vc}\PYG{o}{.}\PYG{n}{xyscale}\PYG{p}{(}\PYG{l+s+s1}{\PYGZsq{}}\PYG{l+s+s1}{linear}\PYG{l+s+s1}{\PYGZsq{}}\PYG{p}{,} \PYG{l+s+s1}{\PYGZsq{}}\PYG{l+s+s1}{area\PYGZus{}wt}\PYG{l+s+s1}{\PYGZsq{}}\PYG{p}{)}
\end{Verbatim}
\end{quote}

\item {} 
Specify the line style:
\begin{quote}

\begin{Verbatim}[commandchars=\\\{\}]
\PYG{c+c1}{\PYGZsh{} Same as vc.line=\PYGZsq{}solid\PYGZsq{}}
\PYG{n}{vc}\PYG{o}{.}\PYG{n}{line}\PYG{o}{=}\PYG{l+m+mi}{0}
\PYG{c+c1}{\PYGZsh{} Same as vc.line=\PYGZsq{}dash\PYGZsq{}}
\PYG{n}{vc}\PYG{o}{.}\PYG{n}{line}\PYG{o}{=}\PYG{l+m+mi}{1}
\PYG{c+c1}{\PYGZsh{} Same as vc.line=\PYGZsq{}dot\PYGZsq{}}
\PYG{n}{vc}\PYG{o}{.}\PYG{n}{line}\PYG{o}{=}\PYG{l+m+mi}{2}
\PYG{c+c1}{\PYGZsh{} Same as vc.line=\PYGZsq{}dash\PYGZhy{}dot\PYGZsq{}}
\PYG{n}{vc}\PYG{o}{.}\PYG{n}{line}\PYG{o}{=}\PYG{l+m+mi}{3}
\PYG{c+c1}{\PYGZsh{} Same as vc.line=\PYGZsq{}long\PYGZhy{}dot\PYGZsq{}}
\PYG{n}{vc}\PYG{o}{.}\PYG{n}{line}\PYG{o}{=}\PYG{l+m+mi}{4}
\end{Verbatim}
\end{quote}

\item {} 
Specify the line color of the vectors:
\begin{quote}

\begin{Verbatim}[commandchars=\\\{\}]
\PYG{c+c1}{\PYGZsh{} Color range: 16 to 230, default line color is black}
\PYG{n}{vc}\PYG{o}{.}\PYG{n}{linecolor}\PYG{o}{=}\PYG{l+m+mi}{16}
\PYG{c+c1}{\PYGZsh{} Width range: 1 to 100, default size is 1}
\PYG{n}{vc}\PYG{o}{.}\PYG{n}{linewidth}\PYG{o}{=}\PYG{l+m+mi}{1}
\end{Verbatim}
\end{quote}

\item {} 
Specify the vector scale factor:
\begin{quote}

\begin{Verbatim}[commandchars=\\\{\}]
\PYG{c+c1}{\PYGZsh{} Can be an integer or float}
\PYG{n}{vc}\PYG{o}{.}\PYG{n}{scale}\PYG{o}{=}\PYG{l+m+mf}{2.0}
\end{Verbatim}
\end{quote}

\item {} 
Specify the vector alignment:
\begin{quote}

\begin{Verbatim}[commandchars=\\\{\}]
\PYG{c+c1}{\PYGZsh{} Same as vc.alignment=\PYGZsq{}head\PYGZsq{}}
\PYG{n}{vc}\PYG{o}{.}\PYG{n}{alignment}\PYG{o}{=}\PYG{l+m+mi}{0}
\PYG{c+c1}{\PYGZsh{} Same as vc.alignment=\PYGZsq{}center\PYGZsq{}}
\PYG{n}{vc}\PYG{o}{.}\PYG{n}{alignment}\PYG{o}{=}\PYG{l+m+mi}{1}
\PYG{c+c1}{\PYGZsh{} Same as vc.alignment=\PYGZsq{}tail\PYGZsq{}}
\PYG{n}{vc}\PYG{o}{.}\PYG{n}{alignment}\PYG{o}{=}\PYG{l+m+mi}{2}
\end{Verbatim}
\end{quote}

\item {} 
Specify the vector type:
\begin{quote}

\begin{Verbatim}[commandchars=\\\{\}]
\PYG{c+c1}{\PYGZsh{} Same as vc.type=\PYGZsq{}arrow head\PYGZsq{}}
\PYG{n}{vc}\PYG{o}{.}\PYG{n}{type}\PYG{o}{=}\PYG{l+m+mi}{0}
\PYG{c+c1}{\PYGZsh{} Same as vc.type=\PYGZsq{}wind barbs\PYGZsq{}}
\PYG{n}{vc}\PYG{o}{.}\PYG{n}{type}\PYG{o}{=}\PYG{l+m+mi}{1}
\PYG{c+c1}{\PYGZsh{} Same as vc.type=\PYGZsq{}solid arrow head\PYGZsq{}}
\PYG{n}{vc}\PYG{o}{.}\PYG{n}{type}\PYG{o}{=}\PYG{l+m+mi}{2}
\end{Verbatim}
\end{quote}

\item {} 
Specify the vector reference:
\begin{quote}

\begin{Verbatim}[commandchars=\\\{\}]
\PYG{c+c1}{\PYGZsh{} Can be an integer or float}
\PYG{n}{vc}\PYG{o}{.}\PYG{n}{reference}\PYG{o}{=}\PYG{l+m+mi}{4}
\end{Verbatim}
\end{quote}

\end{itemize}

\end{fulllineitems}

\index{script() (vcs.vector.Gv method)}

\begin{fulllineitems}
\phantomsection\label{vcs/graphics/vector:vcs.vector.Gv.script}\pysiglinewithargsret{\sphinxbfcode{script}}{\emph{script\_filename=None}, \emph{mode=None}}{}
Saves out a copy of the vector graphics method in JSON, or Python format to a designated file.
\begin{quote}

\begin{notice}{note}{Note:}
If the the filename has a `.py' at the end, it will produce a
Python script. If no extension is given, then by default a
.json file containing a JSON serialization of the object's
data will be produced.
\end{notice}

\begin{notice}{warning}{Warning:}
VCS Scripts Deprecated.
SCR script files are no longer generated by this function.
\end{notice}
\end{quote}
\begin{quote}\begin{description}
\item[{Example}] \leavevmode
\begin{Verbatim}[commandchars=\\\{\}]
\PYG{g+gp}{\PYGZgt{}\PYGZgt{}\PYGZgt{} }\PYG{n}{a}\PYG{o}{=}\PYG{n}{vcs}\PYG{o}{.}\PYG{n}{init}\PYG{p}{(}\PYG{p}{)} \PYG{c+c1}{\PYGZsh{} Make a Canvas object to work with}
\PYG{g+gp}{\PYGZgt{}\PYGZgt{}\PYGZgt{} }\PYG{n}{ex}\PYG{o}{=}\PYG{n}{a}\PYG{o}{.}\PYG{n}{getvector}\PYG{p}{(}\PYG{p}{)} \PYG{c+c1}{\PYGZsh{} Get default vector}
\PYG{g+gp}{\PYGZgt{}\PYGZgt{}\PYGZgt{} }\PYG{n}{ex}\PYG{o}{.}\PYG{n}{script}\PYG{p}{(}\PYG{l+s+s1}{\PYGZsq{}}\PYG{l+s+s1}{filename.py}\PYG{l+s+s1}{\PYGZsq{}}\PYG{p}{)} \PYG{c+c1}{\PYGZsh{} Append to a Python script named \PYGZsq{}filename.py\PYGZsq{}}
\PYG{g+gp}{\PYGZgt{}\PYGZgt{}\PYGZgt{} }\PYG{n}{ex}\PYG{o}{.}\PYG{n}{script}\PYG{p}{(}\PYG{l+s+s1}{\PYGZsq{}}\PYG{l+s+s1}{filename}\PYG{l+s+s1}{\PYGZsq{}}\PYG{p}{,}\PYG{l+s+s1}{\PYGZsq{}}\PYG{l+s+s1}{w}\PYG{l+s+s1}{\PYGZsq{}}\PYG{p}{)} \PYG{c+c1}{\PYGZsh{} Create or overwrite a JSON file \PYGZsq{}filename.json\PYGZsq{}.}
\end{Verbatim}

\item[{Parameters}] \leavevmode\begin{itemize}
\item {} 
\textbf{\texttt{script\_filename}} (\href{https://docs.python.org/2/library/functions.html\#str}{\emph{\texttt{str}}}) -- Output name of the script file. If no extension is specified, a .json object is created.

\item {} 
\textbf{\texttt{mode}} (\href{https://docs.python.org/2/library/functions.html\#str}{\emph{\texttt{str}}}) -- Either `w' for replace, or `a' for append. Defaults to `a', if not specified.

\end{itemize}

\end{description}\end{quote}

\end{fulllineitems}


\end{fulllineitems}



\section{Templating}
\label{vcs/template/templ::doc}\label{vcs/template/templ:templating}
Templates define the layout of your visualization. The Template object is used to lay out labels, lines, tick marks, etc.; it uses the P* objects to do so.


\subsection{template}
\label{vcs/template/template:module-vcs.template}\label{vcs/template/template::doc}\label{vcs/template/template:template}\index{vcs.template (module)}
\# Template (P) module
\index{P (class in vcs.template)}

\begin{fulllineitems}
\phantomsection\label{vcs/template/template:vcs.template.P}\pysiglinewithargsret{\sphinxstrong{class }\sphinxcode{vcs.template.}\sphinxbfcode{P}}{\emph{Pic\_name=None}, \emph{Pic\_name\_src='default'}}{}
The template primary method (P) determines the location of each picture
segment, the space to be allocated to it, and related properties relevant
to its display.
\begin{quote}


\begin{fulllineitems}
\pysigline{\sphinxbfcode{Useful~Functions:}}~
\begin{Verbatim}[commandchars=\\\{\}]
\PYG{c+c1}{\PYGZsh{} Show predefined templates}
\PYG{n}{a}\PYG{o}{.}\PYG{n}{show}\PYG{p}{(}\PYG{l+s+s1}{\PYGZsq{}}\PYG{l+s+s1}{template}\PYG{l+s+s1}{\PYGZsq{}}\PYG{p}{)}
\PYG{c+c1}{\PYGZsh{} Show predefined text table methods}
\PYG{n}{a}\PYG{o}{.}\PYG{n}{show}\PYG{p}{(}\PYG{l+s+s1}{\PYGZsq{}}\PYG{l+s+s1}{texttable}\PYG{l+s+s1}{\PYGZsq{}}\PYG{p}{)}
\PYG{c+c1}{\PYGZsh{} Show predefined text orientation methods}
\PYG{n}{a}\PYG{o}{.}\PYG{n}{show}\PYG{p}{(}\PYG{l+s+s1}{\PYGZsq{}}\PYG{l+s+s1}{textorientation}\PYG{l+s+s1}{\PYGZsq{}}\PYG{p}{)}
\PYG{c+c1}{\PYGZsh{} Show predefined line methods}
\PYG{n}{a}\PYG{o}{.}\PYG{n}{show}\PYG{p}{(}\PYG{l+s+s1}{\PYGZsq{}}\PYG{l+s+s1}{line}\PYG{l+s+s1}{\PYGZsq{}}\PYG{p}{)}
\PYG{c+c1}{\PYGZsh{} Show templates as a Python list}
\PYG{n}{a}\PYG{o}{.}\PYG{n}{listelements}\PYG{p}{(}\PYG{l+s+s1}{\PYGZsq{}}\PYG{l+s+s1}{template}\PYG{l+s+s1}{\PYGZsq{}}\PYG{p}{)}
\PYG{c+c1}{\PYGZsh{} Updates the VCS Canvas at user\PYGZsq{}s request}
\PYG{n}{a}\PYG{o}{.}\PYG{n}{update}\PYG{p}{(}\PYG{p}{)}
\end{Verbatim}

\end{fulllineitems}

\end{quote}


\begin{fulllineitems}
\pysigline{\sphinxbfcode{Make~a~Canvas~object~to~work~with:}}~
\begin{Verbatim}[commandchars=\\\{\}]
\PYG{c+c1}{\PYGZsh{} VCS Canvas constructor}
\PYG{n}{a}\PYG{o}{=}\PYG{n}{vcs}\PYG{o}{.}\PYG{n}{init}\PYG{p}{(}\PYG{p}{)}
\end{Verbatim}

\end{fulllineitems}



\begin{fulllineitems}
\pysigline{\sphinxbfcode{Create~a~new~instance~of~template:}}~
\begin{Verbatim}[commandchars=\\\{\}]
\PYG{c+c1}{\PYGZsh{} Two ways to create a templates:}
\PYG{c+c1}{\PYGZsh{} Copies content of \PYGZsq{}hovmuller\PYGZsq{} to \PYGZsq{}new\PYGZsq{}}
\PYG{n}{temp}\PYG{o}{=}\PYG{n}{a}\PYG{o}{.}\PYG{n}{createtemplate}\PYG{p}{(}\PYG{l+s+s1}{\PYGZsq{}}\PYG{l+s+s1}{new}\PYG{l+s+s1}{\PYGZsq{}}\PYG{p}{,}\PYG{l+s+s1}{\PYGZsq{}}\PYG{l+s+s1}{hovmuller}\PYG{l+s+s1}{\PYGZsq{}}\PYG{p}{)}
\PYG{c+c1}{\PYGZsh{} Copies content of \PYGZsq{}default\PYGZsq{} to \PYGZsq{}new\PYGZsq{}}
\PYG{n}{temp}\PYG{o}{=}\PYG{n}{a}\PYG{o}{.}\PYG{n}{createtemplate}\PYG{p}{(}\PYG{l+s+s1}{\PYGZsq{}}\PYG{l+s+s1}{new}\PYG{l+s+s1}{\PYGZsq{}}\PYG{p}{)}
\end{Verbatim}

\end{fulllineitems}



\begin{fulllineitems}
\pysigline{\sphinxbfcode{Modify~an~existing~template:}}~
\begin{Verbatim}[commandchars=\\\{\}]
\PYG{n}{temp}\PYG{o}{=}\PYG{n}{a}\PYG{o}{.}\PYG{n}{gettemplate}\PYG{p}{(}\PYG{l+s+s1}{\PYGZsq{}}\PYG{l+s+s1}{hovmuller}\PYG{l+s+s1}{\PYGZsq{}}\PYG{p}{)}
\end{Verbatim}

\end{fulllineitems}

\index{blank() (vcs.template.P method)}

\begin{fulllineitems}
\phantomsection\label{vcs/template/template:vcs.template.P.blank}\pysiglinewithargsret{\sphinxbfcode{blank}}{\emph{attribute=None}}{}~\begin{quote}

This function turns off elements of a template object.
\end{quote}
\begin{quote}\begin{description}
\item[{Parameters}] \leavevmode
\textbf{\texttt{attribute}} (\emph{\texttt{None, str, list}}) -- String or list, indicating the elements of a template which should be turned off.
If attribute is left blank, or is None, all elements of the template will be turned off.

\end{description}\end{quote}

\end{fulllineitems}

\index{move() (vcs.template.P method)}

\begin{fulllineitems}
\phantomsection\label{vcs/template/template:vcs.template.P.move}\pysiglinewithargsret{\sphinxbfcode{move}}{\emph{p}, \emph{axis}}{}
Move a template by p\% along the axis `x' or `y'.
Positive values of p mean movement toward right/top
Negative values of p mean movement toward left/bottom
The reference point is t.data.x1/y1
\begin{quote}\begin{description}
\item[{Example}] \leavevmode
\begin{Verbatim}[commandchars=\\\{\}]
\PYG{g+gp}{\PYGZgt{}\PYGZgt{}\PYGZgt{} }\PYG{n}{t} \PYG{o}{=} \PYG{n}{vcs}\PYG{o}{.}\PYG{n}{createtemplate}\PYG{p}{(}\PYG{l+s+s1}{\PYGZsq{}}\PYG{l+s+s1}{example1}\PYG{l+s+s1}{\PYGZsq{}}\PYG{p}{,} \PYG{l+s+s1}{\PYGZsq{}}\PYG{l+s+s1}{default}\PYG{l+s+s1}{\PYGZsq{}}\PYG{p}{)} \PYG{c+c1}{\PYGZsh{} Create template \PYGZsq{}example1\PYGZsq{}, inherits from \PYGZsq{}default\PYGZsq{}}
\PYG{g+gp}{\PYGZgt{}\PYGZgt{}\PYGZgt{} }\PYG{n}{t}\PYG{o}{.}\PYG{n}{move}\PYG{p}{(}\PYG{l+m+mf}{0.2}\PYG{p}{,}\PYG{l+s+s1}{\PYGZsq{}}\PYG{l+s+s1}{x}\PYG{l+s+s1}{\PYGZsq{}}\PYG{p}{)} \PYG{c+c1}{\PYGZsh{} Move everything right by 20\PYGZpc{}}
\PYG{g+gp}{\PYGZgt{}\PYGZgt{}\PYGZgt{} }\PYG{n}{t}\PYG{o}{.}\PYG{n}{move}\PYG{p}{(}\PYG{l+m+mf}{0.2}\PYG{p}{,}\PYG{l+s+s1}{\PYGZsq{}}\PYG{l+s+s1}{y}\PYG{l+s+s1}{\PYGZsq{}}\PYG{p}{)} \PYG{c+c1}{\PYGZsh{} Move everything up by 20\PYGZpc{}}
\end{Verbatim}

\item[{Parameters}] \leavevmode\begin{itemize}
\item {} 
\textbf{\texttt{p}} (\href{https://docs.python.org/2/library/functions.html\#float}{\emph{\texttt{float}}}) -- Float indicating the percentage by which the template should move. i.e. 0.2 = 20\%.

\item {} 
\textbf{\texttt{axis}} (\href{https://docs.python.org/2/library/functions.html\#str}{\emph{\texttt{str}}}) -- One of {[}'x', `y'{]}. The axis along which the template will move.

\end{itemize}

\end{description}\end{quote}

\end{fulllineitems}

\index{moveto() (vcs.template.P method)}

\begin{fulllineitems}
\phantomsection\label{vcs/template/template:vcs.template.P.moveto}\pysiglinewithargsret{\sphinxbfcode{moveto}}{\emph{x}, \emph{y}}{}
Move a template to point (x,y), adjusting all attributes so data.x1 = x, and data.y1 = y.
\begin{quote}\begin{description}
\item[{Example}] \leavevmode
\begin{Verbatim}[commandchars=\\\{\}]
\PYG{g+gp}{\PYGZgt{}\PYGZgt{}\PYGZgt{} }\PYG{n}{t} \PYG{o}{=} \PYG{n}{vcs}\PYG{o}{.}\PYG{n}{createtemplate}\PYG{p}{(}\PYG{l+s+s1}{\PYGZsq{}}\PYG{l+s+s1}{example1}\PYG{l+s+s1}{\PYGZsq{}}\PYG{p}{,} \PYG{l+s+s1}{\PYGZsq{}}\PYG{l+s+s1}{default}\PYG{l+s+s1}{\PYGZsq{}}\PYG{p}{)} \PYG{c+c1}{\PYGZsh{} Create template \PYGZsq{}example1\PYGZsq{}, inherits from \PYGZsq{}default\PYGZsq{}}
\PYG{g+gp}{\PYGZgt{}\PYGZgt{}\PYGZgt{} }\PYG{n}{t}\PYG{o}{.}\PYG{n}{moveto}\PYG{p}{(}\PYG{l+m+mf}{0.2}\PYG{p}{,} \PYG{l+m+mf}{0.2}\PYG{p}{)} \PYG{c+c1}{\PYGZsh{} Move everything so that data.x1= 0.2 and data.y1= 0.2}
\end{Verbatim}

\item[{Parameters}] \leavevmode\begin{itemize}
\item {} 
\textbf{\texttt{x}} (\href{https://docs.python.org/2/library/functions.html\#float}{\emph{\texttt{float}}}) -- Float representing the new coordinate of the template's data.x1 attribute.

\item {} 
\textbf{\texttt{y}} (\href{https://docs.python.org/2/library/functions.html\#float}{\emph{\texttt{float}}}) -- Float representing the new coordinate of the template's data.y1 attribute.

\end{itemize}

\end{description}\end{quote}

\end{fulllineitems}

\index{scale() (vcs.template.P method)}

\begin{fulllineitems}
\phantomsection\label{vcs/template/template:vcs.template.P.scale}\pysiglinewithargsret{\sphinxbfcode{scale}}{\emph{scale}, \emph{axis='xy'}, \emph{font=-1}}{}
Scale a template along the axis `x' or `y' by scale
Positive values of scale mean increase
Negative values of scale mean decrease
The reference point is t.data.x1/y1
\begin{quote}\begin{description}
\item[{Example}] \leavevmode
\begin{Verbatim}[commandchars=\\\{\}]
\PYG{g+gp}{\PYGZgt{}\PYGZgt{}\PYGZgt{} }\PYG{n}{t} \PYG{o}{=} \PYG{n}{vcs}\PYG{o}{.}\PYG{n}{createtemplate}\PYG{p}{(}\PYG{l+s+s1}{\PYGZsq{}}\PYG{l+s+s1}{example1}\PYG{l+s+s1}{\PYGZsq{}}\PYG{p}{,} \PYG{l+s+s1}{\PYGZsq{}}\PYG{l+s+s1}{default}\PYG{l+s+s1}{\PYGZsq{}}\PYG{p}{)} \PYG{c+c1}{\PYGZsh{} Create template \PYGZsq{}example1\PYGZsq{}, inherits from \PYGZsq{}default\PYGZsq{}}
\PYG{g+gp}{\PYGZgt{}\PYGZgt{}\PYGZgt{} }\PYG{n}{t}\PYG{o}{.}\PYG{n}{scale}\PYG{p}{(}\PYG{l+m+mf}{0.5}\PYG{p}{)} \PYG{c+c1}{\PYGZsh{} Halves the template size}
\PYG{g+gp}{\PYGZgt{}\PYGZgt{}\PYGZgt{} }\PYG{n}{t}\PYG{o}{.}\PYG{n}{scale}\PYG{p}{(}\PYG{l+m+mf}{1.2}\PYG{p}{)} \PYG{c+c1}{\PYGZsh{} Upsize everything to 20\PYGZpc{} more than the original size}
\PYG{g+gp}{\PYGZgt{}\PYGZgt{}\PYGZgt{} }\PYG{n}{t}\PYG{o}{.}\PYG{n}{scale}\PYG{p}{(}\PYG{l+m+mi}{2}\PYG{p}{,}\PYG{l+s+s1}{\PYGZsq{}}\PYG{l+s+s1}{x}\PYG{l+s+s1}{\PYGZsq{}}\PYG{p}{)} \PYG{c+c1}{\PYGZsh{} Double the x axis}
\end{Verbatim}

\item[{Parameters}] \leavevmode\begin{itemize}
\item {} 
\textbf{\texttt{scale}} (\href{https://docs.python.org/2/library/functions.html\#float}{\emph{\texttt{float}}}) -- Float representing the factor by which to scale the template.

\item {} 
\textbf{\texttt{axis}} (\href{https://docs.python.org/2/library/functions.html\#str}{\emph{\texttt{str}}}) -- One of {[}'x', `y', `xy'{]}. Represents the axis/axes along which the template should be scaled.

\item {} 
\textbf{\texttt{font}} (\href{https://docs.python.org/2/library/functions.html\#int}{\emph{\texttt{int}}}) -- Integer flag indicating what should be done with the template's fonts. One of {[}-1, 0, 1{]}.
0: means do not scale the fonts. 1: means scale the fonts.
-1: means do not scale the fonts unless axis='xy'

\end{itemize}

\end{description}\end{quote}

\end{fulllineitems}

\index{script() (vcs.template.P method)}

\begin{fulllineitems}
\phantomsection\label{vcs/template/template:vcs.template.P.script}\pysiglinewithargsret{\sphinxbfcode{script}}{\emph{script\_filename=None}, \emph{mode=None}}{}
Saves out a copy of the template graphics method in JSON, or Python format to a designated file.
\begin{quote}

\begin{notice}{note}{Note:}
If the the filename has a `.py' at the end, it will produce a
Python script. If no extension is given, then by default a
.json file containing a JSON serialization of the object's
data will be produced.
\end{notice}

\begin{notice}{warning}{Warning:}
VCS Scripts Deprecated.
SCR script files are no longer generated by this function.
\end{notice}
\end{quote}
\begin{quote}\begin{description}
\item[{Example}] \leavevmode
\begin{Verbatim}[commandchars=\\\{\}]
\PYG{g+gp}{\PYGZgt{}\PYGZgt{}\PYGZgt{} }\PYG{n}{a}\PYG{o}{=}\PYG{n}{vcs}\PYG{o}{.}\PYG{n}{init}\PYG{p}{(}\PYG{p}{)} \PYG{c+c1}{\PYGZsh{} Make a Canvas object to work with}
\PYG{g+gp}{\PYGZgt{}\PYGZgt{}\PYGZgt{} }\PYG{n}{ex}\PYG{o}{=}\PYG{n}{a}\PYG{o}{.}\PYG{n}{gettemplate}\PYG{p}{(}\PYG{p}{)} \PYG{c+c1}{\PYGZsh{} Get default template}
\PYG{g+gp}{\PYGZgt{}\PYGZgt{}\PYGZgt{} }\PYG{n}{ex}\PYG{o}{.}\PYG{n}{script}\PYG{p}{(}\PYG{l+s+s1}{\PYGZsq{}}\PYG{l+s+s1}{filename.py}\PYG{l+s+s1}{\PYGZsq{}}\PYG{p}{)} \PYG{c+c1}{\PYGZsh{} Append to a Python script named \PYGZsq{}filename.py\PYGZsq{}}
\PYG{g+gp}{\PYGZgt{}\PYGZgt{}\PYGZgt{} }\PYG{n}{ex}\PYG{o}{.}\PYG{n}{script}\PYG{p}{(}\PYG{l+s+s1}{\PYGZsq{}}\PYG{l+s+s1}{filename}\PYG{l+s+s1}{\PYGZsq{}}\PYG{p}{,}\PYG{l+s+s1}{\PYGZsq{}}\PYG{l+s+s1}{w}\PYG{l+s+s1}{\PYGZsq{}}\PYG{p}{)} \PYG{c+c1}{\PYGZsh{} Create or overwrite a JSON file \PYGZsq{}filename.json\PYGZsq{}.}
\end{Verbatim}

\item[{Parameters}] \leavevmode\begin{itemize}
\item {} 
\textbf{\texttt{script\_filename}} (\href{https://docs.python.org/2/library/functions.html\#str}{\emph{\texttt{str}}}) -- Output name of the script file. If no extension is specified, a .json object is created.

\item {} 
\textbf{\texttt{mode}} (\href{https://docs.python.org/2/library/functions.html\#str}{\emph{\texttt{str}}}) -- Either `w' for replace, or `a' for append. Defaults to `a', if not specified.

\end{itemize}

\end{description}\end{quote}

\end{fulllineitems}


\end{fulllineitems}

\index{epsilon\_gte() (in module vcs.template)}

\begin{fulllineitems}
\phantomsection\label{vcs/template/template:vcs.template.epsilon_gte}\pysiglinewithargsret{\sphinxcode{vcs.template.}\sphinxbfcode{epsilon\_gte}}{\emph{a}, \emph{b}}{}
a \textgreater{}= b, using floating point epsilon value.

\end{fulllineitems}

\index{epsilon\_lte() (in module vcs.template)}

\begin{fulllineitems}
\phantomsection\label{vcs/template/template:vcs.template.epsilon_lte}\pysiglinewithargsret{\sphinxcode{vcs.template.}\sphinxbfcode{epsilon\_lte}}{\emph{a}, \emph{b}}{}
a \textless{}= b, using floating point epsilon value.

\end{fulllineitems}



\subsection{Pboxeslines}
\label{vcs/template/Pboxeslines:module-vcs.Pboxeslines}\label{vcs/template/Pboxeslines::doc}\label{vcs/template/Pboxeslines:pboxeslines}\index{vcs.Pboxeslines (module)}
\# Template Boxes and Lines (Pbl) module
\index{Pbl (class in vcs.Pboxeslines)}

\begin{fulllineitems}
\phantomsection\label{vcs/template/Pboxeslines:vcs.Pboxeslines.Pbl}\pysiglinewithargsret{\sphinxstrong{class }\sphinxcode{vcs.Pboxeslines.}\sphinxbfcode{Pbl}}{\emph{member}}{}~\begin{quote}

The Template text object allows the manipulation of line type, width, and color index.

This class is used to define a line table entry used in VCS, or it
can be used to change some or all of the line attributes in an
existing line table entry.
\begin{quote}\begin{description}
\item[{Example}] \leavevmode
\end{description}\end{quote}
\end{quote}

\begin{Verbatim}[commandchars=\\\{\}]
\PYG{c+c1}{\PYGZsh{} Basic Usage Overview:}

\PYG{n}{a}\PYG{o}{=}\PYG{n}{vcs}\PYG{o}{.}\PYG{n}{init}\PYG{p}{(}\PYG{p}{)}
\PYG{c+c1}{\PYGZsh{} Show predefined line objects}
\PYG{n}{a}\PYG{o}{.}\PYG{n}{show}\PYG{p}{(}\PYG{l+s+s1}{\PYGZsq{}}\PYG{l+s+s1}{line}\PYG{l+s+s1}{\PYGZsq{}}\PYG{p}{)}
\PYG{c+c1}{\PYGZsh{} Updates the VCS Canvas at user\PYGZsq{}s request}
\PYG{n}{a}\PYG{o}{.}\PYG{n}{update}\PYG{p}{(}\PYG{p}{)}

\PYG{c+c1}{\PYGZsh{}For mode:}
\PYG{c+c1}{\PYGZsh{}   If 1, then automatic update.}
\PYG{c+c1}{\PYGZsh{}   If 0,use update function to update VCS canvas}
\PYG{n}{a}\PYG{o}{.}\PYG{n}{mode}\PYG{o}{=}\PYG{l+m+mi}{1}

\PYG{c+c1}{\PYGZsh{}To Create a new instance of line use:}
\PYG{c+c1}{\PYGZsh{}    Copies content of \PYGZsq{}red\PYGZsq{} to \PYGZsq{}new\PYGZsq{}}
\PYG{n}{ln}\PYG{o}{=}\PYG{n}{a}\PYG{o}{.}\PYG{n}{createline}\PYG{p}{(}\PYG{l+s+s1}{\PYGZsq{}}\PYG{l+s+s1}{new}\PYG{l+s+s1}{\PYGZsq{}}\PYG{p}{,}\PYG{l+s+s1}{\PYGZsq{}}\PYG{l+s+s1}{red}\PYG{l+s+s1}{\PYGZsq{}}\PYG{p}{)}
\PYG{c+c1}{\PYGZsh{}    Copies content of \PYGZsq{}default\PYGZsq{} to \PYGZsq{}new\PYGZsq{}}
\PYG{n}{ln}\PYG{o}{=}\PYG{n}{a}\PYG{o}{.}\PYG{n}{createline}\PYG{p}{(}\PYG{l+s+s1}{\PYGZsq{}}\PYG{l+s+s1}{new}\PYG{l+s+s1}{\PYGZsq{}}\PYG{p}{)}

\PYG{c+c1}{\PYGZsh{}To Modify an existing line use:}
\PYG{n}{ln}\PYG{o}{=}\PYG{n}{a}\PYG{o}{.}\PYG{n}{getline}\PYG{p}{(}\PYG{l+s+s1}{\PYGZsq{}}\PYG{l+s+s1}{red}\PYG{l+s+s1}{\PYGZsq{}}\PYG{p}{)}

\PYG{c+c1}{\PYGZsh{} Will list all the line attribute values}
\PYG{n}{ln}\PYG{o}{.}\PYG{n}{list}\PYG{p}{(}\PYG{p}{)}
\PYG{c+c1}{\PYGZsh{} Range from 1 to 256}
\PYG{n}{ln}\PYG{o}{.}\PYG{n}{color}\PYG{o}{=}\PYG{l+m+mi}{100}
\PYG{c+c1}{\PYGZsh{} Range from 1 to 300}
\PYG{n}{ln}\PYG{o}{.}\PYG{n}{width}\PYG{o}{=}\PYG{l+m+mi}{100}

 \PYG{c+c1}{\PYGZsh{}Specify the line type:}
 \PYG{c+c1}{\PYGZsh{} Same as ln.type=0}
 \PYG{n}{ln}\PYG{o}{.}\PYG{n}{type}\PYG{o}{=}\PYG{l+s+s1}{\PYGZsq{}}\PYG{l+s+s1}{solid}\PYG{l+s+s1}{\PYGZsq{}}
 \PYG{c+c1}{\PYGZsh{} Same as ln.type=1}
 \PYG{n}{ln}\PYG{o}{.}\PYG{n}{type}\PYG{o}{=}\PYG{l+s+s1}{\PYGZsq{}}\PYG{l+s+s1}{dash}\PYG{l+s+s1}{\PYGZsq{}}
 \PYG{c+c1}{\PYGZsh{} Same as ln.type=2}
 \PYG{n}{ln}\PYG{o}{.}\PYG{n}{type}\PYG{o}{=}\PYG{l+s+s1}{\PYGZsq{}}\PYG{l+s+s1}{dot}\PYG{l+s+s1}{\PYGZsq{}}
 \PYG{c+c1}{\PYGZsh{} Same as ln.type=3}
 \PYG{n}{ln}\PYG{o}{.}\PYG{n}{type}\PYG{o}{=}\PYG{l+s+s1}{\PYGZsq{}}\PYG{l+s+s1}{dash\PYGZhy{}dot}\PYG{l+s+s1}{\PYGZsq{}}
 \PYG{c+c1}{\PYGZsh{} Same as ln.type=4}
 \PYG{n}{ln}\PYG{o}{.}\PYG{n}{type}\PYG{o}{=}\PYG{l+s+s1}{\PYGZsq{}}\PYG{l+s+s1}{long\PYGZhy{}dash}\PYG{l+s+s1}{\PYGZsq{}}
\end{Verbatim}

\end{fulllineitems}



\subsection{Pdata}
\label{vcs/template/Pdata::doc}\label{vcs/template/Pdata:module-vcs.Pdata}\label{vcs/template/Pdata:pdata}\index{vcs.Pdata (module)}
\# Template Data Space (Pds) module
\index{Pds (class in vcs.Pdata)}

\begin{fulllineitems}
\phantomsection\label{vcs/template/Pdata:vcs.Pdata.Pds}\pysiglinewithargsret{\sphinxstrong{class }\sphinxcode{vcs.Pdata.}\sphinxbfcode{Pds}}{\emph{member}}{}
The Template text object allows the manipulation of line type, width, and color index.

This class is used to define an line table entry used in VCS, or it
can be used to change some or all of the line attributes in an
existing line table entry.


\begin{fulllineitems}
\pysigline{\sphinxbfcode{Useful~Functions:}}~
\begin{Verbatim}[commandchars=\\\{\}]
\PYG{c+c1}{\PYGZsh{} VCS Canvas Constructor}
\PYG{n}{a}\PYG{o}{=}\PYG{n}{vcs}\PYG{o}{.}\PYG{n}{init}\PYG{p}{(}\PYG{p}{)}
\PYG{c+c1}{\PYGZsh{} Show predefined line objects}
\PYG{n}{a}\PYG{o}{.}\PYG{n}{show}\PYG{p}{(}\PYG{l+s+s1}{\PYGZsq{}}\PYG{l+s+s1}{line}\PYG{l+s+s1}{\PYGZsq{}}\PYG{p}{)}
\PYG{c+c1}{\PYGZsh{} Updates the VCS Canvas at user\PYGZsq{}s request}
\PYG{n}{a}\PYG{o}{.}\PYG{n}{update}\PYG{p}{(}\PYG{p}{)}
\end{Verbatim}

\end{fulllineitems}



\begin{fulllineitems}
\pysigline{\sphinxbfcode{Make~a~Canvas~object~to~work~with:}}~
\begin{Verbatim}[commandchars=\\\{\}]
\PYG{n}{a}\PYG{o}{=}\PYG{n}{vcs}\PYG{o}{.}\PYG{n}{init}\PYG{p}{(}\PYG{p}{)}
\end{Verbatim}

\end{fulllineitems}



\begin{fulllineitems}
\pysigline{\sphinxbfcode{Create~a~new~instance~of~line:}}~
\begin{Verbatim}[commandchars=\\\{\}]
\PYG{c+c1}{\PYGZsh{} Copies content of \PYGZsq{}red\PYGZsq{} to \PYGZsq{}new\PYGZsq{}}
\PYG{n}{ln}\PYG{o}{=}\PYG{n}{a}\PYG{o}{.}\PYG{n}{createline}\PYG{p}{(}\PYG{l+s+s1}{\PYGZsq{}}\PYG{l+s+s1}{new}\PYG{l+s+s1}{\PYGZsq{}}\PYG{p}{,}\PYG{l+s+s1}{\PYGZsq{}}\PYG{l+s+s1}{red}\PYG{l+s+s1}{\PYGZsq{}}\PYG{p}{)}
\PYG{c+c1}{\PYGZsh{} Copies content of \PYGZsq{}default\PYGZsq{} to \PYGZsq{}new\PYGZsq{}}
\PYG{n}{ln}\PYG{o}{=}\PYG{n}{a}\PYG{o}{.}\PYG{n}{createline}\PYG{p}{(}\PYG{l+s+s1}{\PYGZsq{}}\PYG{l+s+s1}{new}\PYG{l+s+s1}{\PYGZsq{}}\PYG{p}{)}
\end{Verbatim}

\end{fulllineitems}



\begin{fulllineitems}
\pysigline{\sphinxbfcode{Modify~an~existing~line:}}~
\begin{Verbatim}[commandchars=\\\{\}]
\PYG{c+c1}{\PYGZsh{} Get a copy of \PYGZsq{}red\PYGZsq{} line}
\PYG{n}{ln}\PYG{o}{=}\PYG{n}{a}\PYG{o}{.}\PYG{n}{getline}\PYG{p}{(}\PYG{l+s+s1}{\PYGZsq{}}\PYG{l+s+s1}{red}\PYG{l+s+s1}{\PYGZsq{}}\PYG{p}{)}
\end{Verbatim}

\end{fulllineitems}



\begin{fulllineitems}
\pysigline{\sphinxbfcode{Overview~of~line~attributes:}}~\begin{itemize}
\item {} 
Listing line attributes:
\begin{quote}

\begin{Verbatim}[commandchars=\\\{\}]
\PYG{c+c1}{\PYGZsh{} Will list all the line attribute values}
\PYG{n}{ln}\PYG{o}{.}\PYG{n}{list}\PYG{p}{(}\PYG{p}{)}
\PYG{c+c1}{\PYGZsh{} Range from 1 to 256}
\PYG{n}{ln}\PYG{o}{.}\PYG{n}{color}\PYG{o}{=}\PYG{l+m+mi}{100}
\PYG{c+c1}{\PYGZsh{} Range from 1 to 300}
\PYG{n}{ln}\PYG{o}{.}\PYG{n}{width}\PYG{o}{=}\PYG{l+m+mi}{100}
\end{Verbatim}
\end{quote}

\item {} 
Specifying the line type:
\begin{quote}

\begin{Verbatim}[commandchars=\\\{\}]
\PYG{c+c1}{\PYGZsh{} Same as ln.type=0}
\PYG{n}{ln}\PYG{o}{.}\PYG{n}{type}\PYG{o}{=}\PYG{l+s+s1}{\PYGZsq{}}\PYG{l+s+s1}{solid}\PYG{l+s+s1}{\PYGZsq{}}
\PYG{c+c1}{\PYGZsh{} Same as ln.type=1}
\PYG{n}{ln}\PYG{o}{.}\PYG{n}{type}\PYG{o}{=}\PYG{l+s+s1}{\PYGZsq{}}\PYG{l+s+s1}{dash}\PYG{l+s+s1}{\PYGZsq{}}
\PYG{c+c1}{\PYGZsh{} Same as ln.type=2}
\PYG{n}{ln}\PYG{o}{.}\PYG{n}{type}\PYG{o}{=}\PYG{l+s+s1}{\PYGZsq{}}\PYG{l+s+s1}{dot}\PYG{l+s+s1}{\PYGZsq{}}
\PYG{c+c1}{\PYGZsh{} Same as ln.type=3}
\PYG{n}{ln}\PYG{o}{.}\PYG{n}{type}\PYG{o}{=}\PYG{l+s+s1}{\PYGZsq{}}\PYG{l+s+s1}{dash\PYGZhy{}dot}\PYG{l+s+s1}{\PYGZsq{}}
\PYG{c+c1}{\PYGZsh{} Same as ln.type=4}
\PYG{n}{ln}\PYG{o}{.}\PYG{n}{type}\PYG{o}{=}\PYG{l+s+s1}{\PYGZsq{}}\PYG{l+s+s1}{long\PYGZhy{}dash}\PYG{l+s+s1}{\PYGZsq{}}
\end{Verbatim}
\end{quote}

\end{itemize}

\end{fulllineitems}


\end{fulllineitems}



\subsection{Pformat}
\label{vcs/template/Pformat:module-vcs.Pformat}\label{vcs/template/Pformat::doc}\label{vcs/template/Pformat:pformat}\index{vcs.Pformat (module)}
\# Template Format (Pf) module
\index{Pf (class in vcs.Pformat)}

\begin{fulllineitems}
\phantomsection\label{vcs/template/Pformat:vcs.Pformat.Pf}\pysiglinewithargsret{\sphinxstrong{class }\sphinxcode{vcs.Pformat.}\sphinxbfcode{Pf}}{\emph{member}}{}
The Template text object allows the manipulation of line type, width, and color index.

This class is used to define an line table entry used in VCS, or it
can be used to change some or all of the line attributes in an
existing line table entry.


\begin{fulllineitems}
\pysigline{\sphinxbfcode{Useful~Functions:}}~
\begin{Verbatim}[commandchars=\\\{\}]
\PYG{c+c1}{\PYGZsh{} VCS Canvas Constructor}
\PYG{n}{a}\PYG{o}{=}\PYG{n}{vcs}\PYG{o}{.}\PYG{n}{init}\PYG{p}{(}\PYG{p}{)}
\PYG{c+c1}{\PYGZsh{} Show predefined line objects}
\PYG{n}{a}\PYG{o}{.}\PYG{n}{show}\PYG{p}{(}\PYG{l+s+s1}{\PYGZsq{}}\PYG{l+s+s1}{line}\PYG{l+s+s1}{\PYGZsq{}}\PYG{p}{)}
\PYG{c+c1}{\PYGZsh{} Updates the VCS Canvas at user\PYGZsq{}s request}
\PYG{n}{a}\PYG{o}{.}\PYG{n}{update}\PYG{p}{(}\PYG{p}{)}
\end{Verbatim}

\end{fulllineitems}



\begin{fulllineitems}
\pysigline{\sphinxbfcode{Make~a~Canvas~object~to~work~with:}}~
\begin{Verbatim}[commandchars=\\\{\}]
\PYG{n}{a}\PYG{o}{=}\PYG{n}{vcs}\PYG{o}{.}\PYG{n}{init}\PYG{p}{(}\PYG{p}{)}
\end{Verbatim}

\end{fulllineitems}



\begin{fulllineitems}
\pysigline{\sphinxbfcode{Create~a~new~instance~of~line:}}~
\begin{Verbatim}[commandchars=\\\{\}]
\PYG{c+c1}{\PYGZsh{} Copies content of \PYGZsq{}red\PYGZsq{} to \PYGZsq{}new\PYGZsq{}}
\PYG{n}{ln}\PYG{o}{=}\PYG{n}{a}\PYG{o}{.}\PYG{n}{createline}\PYG{p}{(}\PYG{l+s+s1}{\PYGZsq{}}\PYG{l+s+s1}{new}\PYG{l+s+s1}{\PYGZsq{}}\PYG{p}{,}\PYG{l+s+s1}{\PYGZsq{}}\PYG{l+s+s1}{red}\PYG{l+s+s1}{\PYGZsq{}}\PYG{p}{)}
\PYG{c+c1}{\PYGZsh{} Copies content of \PYGZsq{}default\PYGZsq{} to \PYGZsq{}new\PYGZsq{}}
\PYG{n}{ln}\PYG{o}{=}\PYG{n}{a}\PYG{o}{.}\PYG{n}{createline}\PYG{p}{(}\PYG{l+s+s1}{\PYGZsq{}}\PYG{l+s+s1}{new}\PYG{l+s+s1}{\PYGZsq{}}\PYG{p}{)}
\end{Verbatim}

\end{fulllineitems}



\begin{fulllineitems}
\pysigline{\sphinxbfcode{Modify~an~existing~line:}}~
\begin{Verbatim}[commandchars=\\\{\}]
\PYG{n}{ln}\PYG{o}{=}\PYG{n}{a}\PYG{o}{.}\PYG{n}{getline}\PYG{p}{(}\PYG{l+s+s1}{\PYGZsq{}}\PYG{l+s+s1}{red}\PYG{l+s+s1}{\PYGZsq{}}\PYG{p}{)}
\end{Verbatim}

\end{fulllineitems}



\begin{fulllineitems}
\pysigline{\sphinxbfcode{Overview~of~line~attributes:}}~\begin{itemize}
\item {} 
Listing line attributes:
\begin{quote}

\begin{Verbatim}[commandchars=\\\{\}]
\PYG{c+c1}{\PYGZsh{} Will list all the line attribute values}
\PYG{n}{ln}\PYG{o}{.}\PYG{n}{list}\PYG{p}{(}\PYG{p}{)}
\PYG{c+c1}{\PYGZsh{} Range from 1 to 256}
\PYG{n}{ln}\PYG{o}{.}\PYG{n}{color}\PYG{o}{=}\PYG{l+m+mi}{100}
\PYG{c+c1}{\PYGZsh{} Range from 1 to 300}
\PYG{n}{ln}\PYG{o}{.}\PYG{n}{width}\PYG{o}{=}\PYG{l+m+mi}{100}
\end{Verbatim}
\end{quote}

\item {} 
Specifying the line type:
\begin{quote}

\begin{Verbatim}[commandchars=\\\{\}]
\PYG{c+c1}{\PYGZsh{} Same as ln.type=0}
\PYG{n}{ln}\PYG{o}{.}\PYG{n}{type}\PYG{o}{=}\PYG{l+s+s1}{\PYGZsq{}}\PYG{l+s+s1}{solid}\PYG{l+s+s1}{\PYGZsq{}}
\PYG{c+c1}{\PYGZsh{} Same as ln.type=1}
\PYG{n}{ln}\PYG{o}{.}\PYG{n}{type}\PYG{o}{=}\PYG{l+s+s1}{\PYGZsq{}}\PYG{l+s+s1}{dash}\PYG{l+s+s1}{\PYGZsq{}}
\PYG{c+c1}{\PYGZsh{} Same as ln.type=2}
\PYG{n}{ln}\PYG{o}{.}\PYG{n}{type}\PYG{o}{=}\PYG{l+s+s1}{\PYGZsq{}}\PYG{l+s+s1}{dot}\PYG{l+s+s1}{\PYGZsq{}}
\PYG{c+c1}{\PYGZsh{} Same as ln.type=3}
\PYG{n}{ln}\PYG{o}{.}\PYG{n}{type}\PYG{o}{=}\PYG{l+s+s1}{\PYGZsq{}}\PYG{l+s+s1}{dash\PYGZhy{}dot}\PYG{l+s+s1}{\PYGZsq{}}
\PYG{c+c1}{\PYGZsh{} Same as ln.type=4}
\PYG{n}{ln}\PYG{o}{.}\PYG{n}{type}\PYG{o}{=}\PYG{l+s+s1}{\PYGZsq{}}\PYG{l+s+s1}{long\PYGZhy{}dash}\PYG{l+s+s1}{\PYGZsq{}}
\end{Verbatim}
\end{quote}

\end{itemize}

\end{fulllineitems}


\end{fulllineitems}



\subsection{Plegend}
\label{vcs/template/Plegend:module-vcs.Plegend}\label{vcs/template/Plegend:plegend}\label{vcs/template/Plegend::doc}\index{vcs.Plegend (module)}
\# Template Legend Space (Pls) module
\index{Pls (class in vcs.Plegend)}

\begin{fulllineitems}
\phantomsection\label{vcs/template/Plegend:vcs.Plegend.Pls}\pysiglinewithargsret{\sphinxstrong{class }\sphinxcode{vcs.Plegend.}\sphinxbfcode{Pls}}{\emph{member}}{}
The Template text object allows the manipulation of line type, width, and color index.

This class is used to define an line table entry used in VCS, or it
can be used to change some or all of the line attributes in an
existing line table entry.


\begin{fulllineitems}
\pysigline{\sphinxbfcode{Useful~Functions:}}~
\begin{Verbatim}[commandchars=\\\{\}]
\PYG{c+c1}{\PYGZsh{} VCS Canvas Constructor}
\PYG{n}{a}\PYG{o}{=}\PYG{n}{vcs}\PYG{o}{.}\PYG{n}{init}\PYG{p}{(}\PYG{p}{)}
\PYG{c+c1}{\PYGZsh{} Show predefined line objects}
\PYG{n}{a}\PYG{o}{.}\PYG{n}{show}\PYG{p}{(}\PYG{l+s+s1}{\PYGZsq{}}\PYG{l+s+s1}{line}\PYG{l+s+s1}{\PYGZsq{}}\PYG{p}{)}
\PYG{c+c1}{\PYGZsh{} Updates the VCS Canvas at user\PYGZsq{}s request}
\PYG{n}{a}\PYG{o}{.}\PYG{n}{update}\PYG{p}{(}\PYG{p}{)}
\end{Verbatim}

\end{fulllineitems}



\begin{fulllineitems}
\pysigline{\sphinxbfcode{Make~a~Canvas~object~to~work~with:}}~
\begin{Verbatim}[commandchars=\\\{\}]
\PYG{n}{a}\PYG{o}{=}\PYG{n}{vcs}\PYG{o}{.}\PYG{n}{init}\PYG{p}{(}\PYG{p}{)}
\end{Verbatim}

\end{fulllineitems}



\begin{fulllineitems}
\pysigline{\sphinxbfcode{Create~a~new~instance~of~line:}}~
\begin{Verbatim}[commandchars=\\\{\}]
\PYG{c+c1}{\PYGZsh{} Copies content of \PYGZsq{}red\PYGZsq{} to \PYGZsq{}new\PYGZsq{}}
\PYG{n}{ln}\PYG{o}{=}\PYG{n}{a}\PYG{o}{.}\PYG{n}{createline}\PYG{p}{(}\PYG{l+s+s1}{\PYGZsq{}}\PYG{l+s+s1}{new}\PYG{l+s+s1}{\PYGZsq{}}\PYG{p}{,}\PYG{l+s+s1}{\PYGZsq{}}\PYG{l+s+s1}{red}\PYG{l+s+s1}{\PYGZsq{}}\PYG{p}{)}
\PYG{c+c1}{\PYGZsh{} Copies content of \PYGZsq{}default\PYGZsq{} to \PYGZsq{}new\PYGZsq{}}
\PYG{n}{ln}\PYG{o}{=}\PYG{n}{a}\PYG{o}{.}\PYG{n}{createline}\PYG{p}{(}\PYG{l+s+s1}{\PYGZsq{}}\PYG{l+s+s1}{new}\PYG{l+s+s1}{\PYGZsq{}}\PYG{p}{)}
\end{Verbatim}

\end{fulllineitems}



\begin{fulllineitems}
\pysigline{\sphinxbfcode{Modify~an~existing~line:}}~
\begin{Verbatim}[commandchars=\\\{\}]
\PYG{n}{ln}\PYG{o}{=}\PYG{n}{a}\PYG{o}{.}\PYG{n}{getline}\PYG{p}{(}\PYG{l+s+s1}{\PYGZsq{}}\PYG{l+s+s1}{red}\PYG{l+s+s1}{\PYGZsq{}}\PYG{p}{)}
\end{Verbatim}

\end{fulllineitems}



\begin{fulllineitems}
\pysigline{\sphinxbfcode{Overview~of~line~attributes:}}~\begin{itemize}
\item {} 
Listing line attributes:
\begin{quote}

\begin{Verbatim}[commandchars=\\\{\}]
\PYG{c+c1}{\PYGZsh{} Will list all the line attribute values}
\PYG{n}{ln}\PYG{o}{.}\PYG{n}{list}\PYG{p}{(}\PYG{p}{)}
\PYG{c+c1}{\PYGZsh{} Range from 1 to 256}
\PYG{n}{ln}\PYG{o}{.}\PYG{n}{color}\PYG{o}{=}\PYG{l+m+mi}{100}
\PYG{c+c1}{\PYGZsh{} Range from 1 to 300}
\PYG{n}{ln}\PYG{o}{.}\PYG{n}{width}\PYG{o}{=}\PYG{l+m+mi}{100}
\end{Verbatim}
\end{quote}

\item {} 
Specifying the line type:
\begin{quote}

\begin{Verbatim}[commandchars=\\\{\}]
\PYG{c+c1}{\PYGZsh{} Same as ln.type=0}
\PYG{n}{ln}\PYG{o}{.}\PYG{n}{type}\PYG{o}{=}\PYG{l+s+s1}{\PYGZsq{}}\PYG{l+s+s1}{solid}\PYG{l+s+s1}{\PYGZsq{}}
\PYG{c+c1}{\PYGZsh{} Same as ln.type=1}
\PYG{n}{ln}\PYG{o}{.}\PYG{n}{type}\PYG{o}{=}\PYG{l+s+s1}{\PYGZsq{}}\PYG{l+s+s1}{dash}\PYG{l+s+s1}{\PYGZsq{}}
\PYG{c+c1}{\PYGZsh{} Same as ln.type=2}
\PYG{n}{ln}\PYG{o}{.}\PYG{n}{type}\PYG{o}{=}\PYG{l+s+s1}{\PYGZsq{}}\PYG{l+s+s1}{dot}\PYG{l+s+s1}{\PYGZsq{}}
\PYG{c+c1}{\PYGZsh{} Same as ln.type=3}
\PYG{n}{ln}\PYG{o}{.}\PYG{n}{type}\PYG{o}{=}\PYG{l+s+s1}{\PYGZsq{}}\PYG{l+s+s1}{dash\PYGZhy{}dot}\PYG{l+s+s1}{\PYGZsq{}}
\PYG{c+c1}{\PYGZsh{} Same as ln.type=4}
\PYG{n}{ln}\PYG{o}{.}\PYG{n}{type}\PYG{o}{=}\PYG{l+s+s1}{\PYGZsq{}}\PYG{l+s+s1}{long\PYGZhy{}dash}\PYG{l+s+s1}{\PYGZsq{}}
\end{Verbatim}
\end{quote}

\end{itemize}

\end{fulllineitems}


\end{fulllineitems}



\subsection{Ptext}
\label{vcs/template/Ptext:module-vcs.Ptext}\label{vcs/template/Ptext:ptext}\label{vcs/template/Ptext::doc}\index{vcs.Ptext (module)}
\# Template Text (Pt) module
\index{Pt (class in vcs.Ptext)}

\begin{fulllineitems}
\phantomsection\label{vcs/template/Ptext:vcs.Ptext.Pt}\pysiglinewithargsret{\sphinxstrong{class }\sphinxcode{vcs.Ptext.}\sphinxbfcode{Pt}}{\emph{member}}{}
The Template text object allows the manipulation of line type, width, and color index.

This class is used to define an line table entry used in VCS, or it
can be used to change some or all of the line attributes in an
existing line table entry.


\begin{fulllineitems}
\pysigline{\sphinxbfcode{Useful~Functions:}}~
\begin{Verbatim}[commandchars=\\\{\}]
\PYG{c+c1}{\PYGZsh{} VCS Canvas Constructor}
\PYG{n}{a}\PYG{o}{=}\PYG{n}{vcs}\PYG{o}{.}\PYG{n}{init}\PYG{p}{(}\PYG{p}{)}
\PYG{c+c1}{\PYGZsh{} Show predefined line objects}
\PYG{n}{a}\PYG{o}{.}\PYG{n}{show}\PYG{p}{(}\PYG{l+s+s1}{\PYGZsq{}}\PYG{l+s+s1}{line}\PYG{l+s+s1}{\PYGZsq{}}\PYG{p}{)}
\PYG{c+c1}{\PYGZsh{} Updates the VCS Canvas at user\PYGZsq{}s request}
\PYG{n}{a}\PYG{o}{.}\PYG{n}{update}\PYG{p}{(}\PYG{p}{)}
\end{Verbatim}

\end{fulllineitems}



\begin{fulllineitems}
\pysigline{\sphinxbfcode{Make~a~Canvas~object~to~work~with:}}~
\begin{Verbatim}[commandchars=\\\{\}]
\PYG{n}{a}\PYG{o}{=}\PYG{n}{vcs}\PYG{o}{.}\PYG{n}{init}\PYG{p}{(}\PYG{p}{)}
\end{Verbatim}

\end{fulllineitems}



\begin{fulllineitems}
\pysigline{\sphinxbfcode{Create~a~new~instance~of~line:}}~
\begin{Verbatim}[commandchars=\\\{\}]
\PYG{c+c1}{\PYGZsh{} Copies content of \PYGZsq{}red\PYGZsq{} to \PYGZsq{}new\PYGZsq{}}
\PYG{n}{ln}\PYG{o}{=}\PYG{n}{a}\PYG{o}{.}\PYG{n}{createline}\PYG{p}{(}\PYG{l+s+s1}{\PYGZsq{}}\PYG{l+s+s1}{new}\PYG{l+s+s1}{\PYGZsq{}}\PYG{p}{,}\PYG{l+s+s1}{\PYGZsq{}}\PYG{l+s+s1}{red}\PYG{l+s+s1}{\PYGZsq{}}\PYG{p}{)}
\PYG{c+c1}{\PYGZsh{} Copies content of \PYGZsq{}default\PYGZsq{} to \PYGZsq{}new\PYGZsq{}}
\PYG{n}{ln}\PYG{o}{=}\PYG{n}{a}\PYG{o}{.}\PYG{n}{createline}\PYG{p}{(}\PYG{l+s+s1}{\PYGZsq{}}\PYG{l+s+s1}{new}\PYG{l+s+s1}{\PYGZsq{}}\PYG{p}{)}
\end{Verbatim}

\end{fulllineitems}



\begin{fulllineitems}
\pysigline{\sphinxbfcode{Modify~an~existing~line:}}~
\begin{Verbatim}[commandchars=\\\{\}]
\PYG{n}{ln}\PYG{o}{=}\PYG{n}{a}\PYG{o}{.}\PYG{n}{getline}\PYG{p}{(}\PYG{l+s+s1}{\PYGZsq{}}\PYG{l+s+s1}{red}\PYG{l+s+s1}{\PYGZsq{}}\PYG{p}{)}
\end{Verbatim}

\end{fulllineitems}



\begin{fulllineitems}
\pysigline{\sphinxbfcode{Overview~of~line~attributes:}}~\begin{itemize}
\item {} 
Listing line attributes:
\begin{quote}

\begin{Verbatim}[commandchars=\\\{\}]
\PYG{c+c1}{\PYGZsh{} Will list all the line attribute values}
\PYG{n}{ln}\PYG{o}{.}\PYG{n}{list}\PYG{p}{(}\PYG{p}{)}
\PYG{c+c1}{\PYGZsh{} Range from 1 to 256}
\PYG{n}{ln}\PYG{o}{.}\PYG{n}{color}\PYG{o}{=}\PYG{l+m+mi}{100}
\PYG{c+c1}{\PYGZsh{} Range from 1 to 300}
\PYG{n}{ln}\PYG{o}{.}\PYG{n}{width}\PYG{o}{=}\PYG{l+m+mi}{100}
\end{Verbatim}
\end{quote}

\item {} 
Specifying the line type:
\begin{quote}

\begin{Verbatim}[commandchars=\\\{\}]
\PYG{c+c1}{\PYGZsh{} Same as ln.type=0}
\PYG{n}{ln}\PYG{o}{.}\PYG{n}{type}\PYG{o}{=}\PYG{l+s+s1}{\PYGZsq{}}\PYG{l+s+s1}{solid}\PYG{l+s+s1}{\PYGZsq{}}
\PYG{c+c1}{\PYGZsh{} Same as ln.type=1}
\PYG{n}{ln}\PYG{o}{.}\PYG{n}{type}\PYG{o}{=}\PYG{l+s+s1}{\PYGZsq{}}\PYG{l+s+s1}{dash}\PYG{l+s+s1}{\PYGZsq{}}
\PYG{c+c1}{\PYGZsh{} Same as ln.type=2}
\PYG{n}{ln}\PYG{o}{.}\PYG{n}{type}\PYG{o}{=}\PYG{l+s+s1}{\PYGZsq{}}\PYG{l+s+s1}{dot}\PYG{l+s+s1}{\PYGZsq{}}
\PYG{c+c1}{\PYGZsh{} Same as ln.type=3}
\PYG{n}{ln}\PYG{o}{.}\PYG{n}{type}\PYG{o}{=}\PYG{l+s+s1}{\PYGZsq{}}\PYG{l+s+s1}{dash\PYGZhy{}dot}\PYG{l+s+s1}{\PYGZsq{}}
\PYG{c+c1}{\PYGZsh{} Same as ln.type=4}
\PYG{n}{ln}\PYG{o}{.}\PYG{n}{type}\PYG{o}{=}\PYG{l+s+s1}{\PYGZsq{}}\PYG{l+s+s1}{long\PYGZhy{}dash}\PYG{l+s+s1}{\PYGZsq{}}
\end{Verbatim}
\end{quote}

\end{itemize}

\end{fulllineitems}


\end{fulllineitems}



\subsection{Pxlabels}
\label{vcs/template/Pxlabels::doc}\label{vcs/template/Pxlabels:module-vcs.Pxlabels}\label{vcs/template/Pxlabels:pxlabels}\index{vcs.Pxlabels (module)}
\# Template X - Labels (Pxl) module
\index{Pxl (class in vcs.Pxlabels)}

\begin{fulllineitems}
\phantomsection\label{vcs/template/Pxlabels:vcs.Pxlabels.Pxl}\pysiglinewithargsret{\sphinxstrong{class }\sphinxcode{vcs.Pxlabels.}\sphinxbfcode{Pxl}}{\emph{member}}{}
The Template text object allows the manipulation of line type, width, and color index.

This class is used to define an line table entry used in VCS, or it
can be used to change some or all of the line attributes in an
existing line table entry.


\begin{fulllineitems}
\pysigline{\sphinxbfcode{Useful~Functions:}}~
\begin{Verbatim}[commandchars=\\\{\}]
\PYG{c+c1}{\PYGZsh{} VCS Canvas Constructor}
\PYG{n}{a}\PYG{o}{=}\PYG{n}{vcs}\PYG{o}{.}\PYG{n}{init}\PYG{p}{(}\PYG{p}{)}
\PYG{c+c1}{\PYGZsh{} Show predefined line objects}
\PYG{n}{a}\PYG{o}{.}\PYG{n}{show}\PYG{p}{(}\PYG{l+s+s1}{\PYGZsq{}}\PYG{l+s+s1}{line}\PYG{l+s+s1}{\PYGZsq{}}\PYG{p}{)}
\PYG{c+c1}{\PYGZsh{} Updates the VCS Canvas at user\PYGZsq{}s request}
\PYG{n}{a}\PYG{o}{.}\PYG{n}{update}\PYG{p}{(}\PYG{p}{)}
\end{Verbatim}

\end{fulllineitems}



\begin{fulllineitems}
\pysigline{\sphinxbfcode{Make~a~Canvas~object~to~work~with:}}~
\begin{Verbatim}[commandchars=\\\{\}]
\PYG{n}{a}\PYG{o}{=}\PYG{n}{vcs}\PYG{o}{.}\PYG{n}{init}\PYG{p}{(}\PYG{p}{)}
\end{Verbatim}

\end{fulllineitems}



\begin{fulllineitems}
\pysigline{\sphinxbfcode{Create~a~new~instance~of~line:}}~
\begin{Verbatim}[commandchars=\\\{\}]
\PYG{c+c1}{\PYGZsh{} Copies content of \PYGZsq{}red\PYGZsq{} to \PYGZsq{}new\PYGZsq{}}
\PYG{n}{ln}\PYG{o}{=}\PYG{n}{a}\PYG{o}{.}\PYG{n}{createline}\PYG{p}{(}\PYG{l+s+s1}{\PYGZsq{}}\PYG{l+s+s1}{new}\PYG{l+s+s1}{\PYGZsq{}}\PYG{p}{,}\PYG{l+s+s1}{\PYGZsq{}}\PYG{l+s+s1}{red}\PYG{l+s+s1}{\PYGZsq{}}\PYG{p}{)}
\PYG{c+c1}{\PYGZsh{} Copies content of \PYGZsq{}default\PYGZsq{} to \PYGZsq{}new\PYGZsq{}}
\PYG{n}{ln}\PYG{o}{=}\PYG{n}{a}\PYG{o}{.}\PYG{n}{createline}\PYG{p}{(}\PYG{l+s+s1}{\PYGZsq{}}\PYG{l+s+s1}{new}\PYG{l+s+s1}{\PYGZsq{}}\PYG{p}{)}
\end{Verbatim}

\end{fulllineitems}



\begin{fulllineitems}
\pysigline{\sphinxbfcode{Modify~an~existing~line:}}~
\begin{Verbatim}[commandchars=\\\{\}]
\PYG{c+c1}{\PYGZsh{} Get a copy of \PYGZsq{}red\PYGZsq{} line}
\PYG{n}{ln}\PYG{o}{=}\PYG{n}{a}\PYG{o}{.}\PYG{n}{getline}\PYG{p}{(}\PYG{l+s+s1}{\PYGZsq{}}\PYG{l+s+s1}{red}\PYG{l+s+s1}{\PYGZsq{}}\PYG{p}{)}
\end{Verbatim}

\end{fulllineitems}



\begin{fulllineitems}
\pysigline{\sphinxbfcode{Overview~of~line~attributes:}}~\begin{itemize}
\item {} 
Listing line attributes:
\begin{quote}

\begin{Verbatim}[commandchars=\\\{\}]
\PYG{c+c1}{\PYGZsh{} Will list all the line attribute values}
\PYG{n}{ln}\PYG{o}{.}\PYG{n}{list}\PYG{p}{(}\PYG{p}{)}
\PYG{c+c1}{\PYGZsh{} Range from 1 to 256}
\PYG{n}{ln}\PYG{o}{.}\PYG{n}{color}\PYG{o}{=}\PYG{l+m+mi}{100}
\PYG{c+c1}{\PYGZsh{} Range from 1 to 300}
\PYG{n}{ln}\PYG{o}{.}\PYG{n}{width}\PYG{o}{=}\PYG{l+m+mi}{100}
\end{Verbatim}
\end{quote}

\item {} 
Specifying the line type:
\begin{quote}

\begin{Verbatim}[commandchars=\\\{\}]
\PYG{c+c1}{\PYGZsh{} Same as ln.type=0}
\PYG{n}{ln}\PYG{o}{.}\PYG{n}{type}\PYG{o}{=}\PYG{l+s+s1}{\PYGZsq{}}\PYG{l+s+s1}{solid}\PYG{l+s+s1}{\PYGZsq{}}
\PYG{c+c1}{\PYGZsh{} Same as ln.type=1}
\PYG{n}{ln}\PYG{o}{.}\PYG{n}{type}\PYG{o}{=}\PYG{l+s+s1}{\PYGZsq{}}\PYG{l+s+s1}{dash}\PYG{l+s+s1}{\PYGZsq{}}
\PYG{c+c1}{\PYGZsh{} Same as ln.type=2}
\PYG{n}{ln}\PYG{o}{.}\PYG{n}{type}\PYG{o}{=}\PYG{l+s+s1}{\PYGZsq{}}\PYG{l+s+s1}{dot}\PYG{l+s+s1}{\PYGZsq{}}
\PYG{c+c1}{\PYGZsh{} Same as ln.type=3}
\PYG{n}{ln}\PYG{o}{.}\PYG{n}{type}\PYG{o}{=}\PYG{l+s+s1}{\PYGZsq{}}\PYG{l+s+s1}{dash\PYGZhy{}dot}\PYG{l+s+s1}{\PYGZsq{}}
\PYG{c+c1}{\PYGZsh{} Same as ln.type=4}
\PYG{n}{ln}\PYG{o}{.}\PYG{n}{type}\PYG{o}{=}\PYG{l+s+s1}{\PYGZsq{}}\PYG{l+s+s1}{long\PYGZhy{}dash}\PYG{l+s+s1}{\PYGZsq{}}
\end{Verbatim}
\end{quote}

\end{itemize}

\end{fulllineitems}


\end{fulllineitems}



\subsection{Pxtickmarks}
\label{vcs/template/Pxtickmarks:module-vcs.Pxtickmarks}\label{vcs/template/Pxtickmarks:pxtickmarks}\label{vcs/template/Pxtickmarks::doc}\index{vcs.Pxtickmarks (module)}
\# Template X - Tick Marks (Pxt) module
\index{Pxt (class in vcs.Pxtickmarks)}

\begin{fulllineitems}
\phantomsection\label{vcs/template/Pxtickmarks:vcs.Pxtickmarks.Pxt}\pysiglinewithargsret{\sphinxstrong{class }\sphinxcode{vcs.Pxtickmarks.}\sphinxbfcode{Pxt}}{\emph{member}}{}
The Template text object allows the manipulation of line type, width, and color index.

This class is used to define an line table entry used in VCS, or it
can be used to change some or all of the line attributes in an
existing line table entry.


\begin{fulllineitems}
\pysigline{\sphinxbfcode{Useful~Functions:}}~
\begin{Verbatim}[commandchars=\\\{\}]
\PYG{c+c1}{\PYGZsh{} VCS Canvas Constructor}
\PYG{n}{a}\PYG{o}{=}\PYG{n}{vcs}\PYG{o}{.}\PYG{n}{init}\PYG{p}{(}\PYG{p}{)}
\PYG{c+c1}{\PYGZsh{} Show predefined line objects}
\PYG{n}{a}\PYG{o}{.}\PYG{n}{show}\PYG{p}{(}\PYG{l+s+s1}{\PYGZsq{}}\PYG{l+s+s1}{line}\PYG{l+s+s1}{\PYGZsq{}}\PYG{p}{)}
\PYG{c+c1}{\PYGZsh{} Updates the VCS Canvas at user\PYGZsq{}s request}
\PYG{n}{a}\PYG{o}{.}\PYG{n}{update}\PYG{p}{(}\PYG{p}{)}
\end{Verbatim}

\end{fulllineitems}



\begin{fulllineitems}
\pysigline{\sphinxbfcode{Make~a~Canvas~object~to~work~with:}}~
\begin{Verbatim}[commandchars=\\\{\}]
\PYG{n}{a}\PYG{o}{=}\PYG{n}{vcs}\PYG{o}{.}\PYG{n}{init}\PYG{p}{(}\PYG{p}{)}
\end{Verbatim}

\end{fulllineitems}



\begin{fulllineitems}
\pysigline{\sphinxbfcode{Create~a~new~instance~of~line:}}~
\begin{Verbatim}[commandchars=\\\{\}]
\PYG{c+c1}{\PYGZsh{} Copies content of \PYGZsq{}red\PYGZsq{} to \PYGZsq{}new\PYGZsq{}}
\PYG{n}{ln}\PYG{o}{=}\PYG{n}{a}\PYG{o}{.}\PYG{n}{createline}\PYG{p}{(}\PYG{l+s+s1}{\PYGZsq{}}\PYG{l+s+s1}{new}\PYG{l+s+s1}{\PYGZsq{}}\PYG{p}{,}\PYG{l+s+s1}{\PYGZsq{}}\PYG{l+s+s1}{red}\PYG{l+s+s1}{\PYGZsq{}}\PYG{p}{)}
\PYG{c+c1}{\PYGZsh{} Copies content of \PYGZsq{}default\PYGZsq{} to \PYGZsq{}new\PYGZsq{}}
\PYG{n}{ln}\PYG{o}{=}\PYG{n}{a}\PYG{o}{.}\PYG{n}{createline}\PYG{p}{(}\PYG{l+s+s1}{\PYGZsq{}}\PYG{l+s+s1}{new}\PYG{l+s+s1}{\PYGZsq{}}\PYG{p}{)}
\end{Verbatim}

\end{fulllineitems}



\begin{fulllineitems}
\pysigline{\sphinxbfcode{Modify~an~existing~line:}}~
\begin{Verbatim}[commandchars=\\\{\}]
\PYG{c+c1}{\PYGZsh{} Get a copy of \PYGZsq{}red\PYGZsq{} line}
\PYG{n}{ln}\PYG{o}{=}\PYG{n}{a}\PYG{o}{.}\PYG{n}{getline}\PYG{p}{(}\PYG{l+s+s1}{\PYGZsq{}}\PYG{l+s+s1}{red}\PYG{l+s+s1}{\PYGZsq{}}\PYG{p}{)}
\end{Verbatim}

\end{fulllineitems}



\begin{fulllineitems}
\pysigline{\sphinxbfcode{Overview~of~line~attributes:}}~\begin{itemize}
\item {} 
Listing line attributes:
\begin{quote}

\begin{Verbatim}[commandchars=\\\{\}]
\PYG{c+c1}{\PYGZsh{} Will list all the line attribute values}
\PYG{n}{ln}\PYG{o}{.}\PYG{n}{list}\PYG{p}{(}\PYG{p}{)}
\PYG{c+c1}{\PYGZsh{} Range from 1 to 256}
\PYG{n}{ln}\PYG{o}{.}\PYG{n}{color}\PYG{o}{=}\PYG{l+m+mi}{100}
\PYG{c+c1}{\PYGZsh{} Range from 1 to 300}
\PYG{n}{ln}\PYG{o}{.}\PYG{n}{width}\PYG{o}{=}\PYG{l+m+mi}{100}
\end{Verbatim}
\end{quote}

\item {} 
Specifying the line type:
\begin{quote}

\begin{Verbatim}[commandchars=\\\{\}]
\PYG{c+c1}{\PYGZsh{} Same as ln.type=0}
\PYG{n}{ln}\PYG{o}{.}\PYG{n}{type}\PYG{o}{=}\PYG{l+s+s1}{\PYGZsq{}}\PYG{l+s+s1}{solid}\PYG{l+s+s1}{\PYGZsq{}}
\PYG{c+c1}{\PYGZsh{} Same as ln.type=1}
\PYG{n}{ln}\PYG{o}{.}\PYG{n}{type}\PYG{o}{=}\PYG{l+s+s1}{\PYGZsq{}}\PYG{l+s+s1}{dash}\PYG{l+s+s1}{\PYGZsq{}}
\PYG{c+c1}{\PYGZsh{} Same as ln.type=2}
\PYG{n}{ln}\PYG{o}{.}\PYG{n}{type}\PYG{o}{=}\PYG{l+s+s1}{\PYGZsq{}}\PYG{l+s+s1}{dot}\PYG{l+s+s1}{\PYGZsq{}}
\PYG{c+c1}{\PYGZsh{} Same as ln.type=3}
\PYG{n}{ln}\PYG{o}{.}\PYG{n}{type}\PYG{o}{=}\PYG{l+s+s1}{\PYGZsq{}}\PYG{l+s+s1}{dash\PYGZhy{}dot}\PYG{l+s+s1}{\PYGZsq{}}
\PYG{c+c1}{\PYGZsh{} Same as ln.type=4}
\PYG{n}{ln}\PYG{o}{.}\PYG{n}{type}\PYG{o}{=}\PYG{l+s+s1}{\PYGZsq{}}\PYG{l+s+s1}{long\PYGZhy{}dash}\PYG{l+s+s1}{\PYGZsq{}}
\end{Verbatim}
\end{quote}

\end{itemize}

\end{fulllineitems}


\end{fulllineitems}



\subsection{Pylabels}
\label{vcs/template/Pylabels:module-vcs.Pylabels}\label{vcs/template/Pylabels:pylabels}\label{vcs/template/Pylabels::doc}\index{vcs.Pylabels (module)}
\# Template Y - Labels (Pyl) module
\index{Pyl (class in vcs.Pylabels)}

\begin{fulllineitems}
\phantomsection\label{vcs/template/Pylabels:vcs.Pylabels.Pyl}\pysiglinewithargsret{\sphinxstrong{class }\sphinxcode{vcs.Pylabels.}\sphinxbfcode{Pyl}}{\emph{member}}{}
The Template text object allows the manipulation of line type, width, and color index.

This class is used to define an line table entry used in VCS, or it
can be used to change some or all of the line attributes in an
existing line table entry.


\begin{fulllineitems}
\pysigline{\sphinxbfcode{Useful~Functions:}}~
\begin{Verbatim}[commandchars=\\\{\}]
\PYG{c+c1}{\PYGZsh{} VCS Canvas Constructor}
\PYG{n}{a}\PYG{o}{=}\PYG{n}{vcs}\PYG{o}{.}\PYG{n}{init}\PYG{p}{(}\PYG{p}{)}
\PYG{c+c1}{\PYGZsh{} Show predefined line objects}
\PYG{n}{a}\PYG{o}{.}\PYG{n}{show}\PYG{p}{(}\PYG{l+s+s1}{\PYGZsq{}}\PYG{l+s+s1}{line}\PYG{l+s+s1}{\PYGZsq{}}\PYG{p}{)}
\PYG{c+c1}{\PYGZsh{} Updates the VCS Canvas at user\PYGZsq{}s request}
\PYG{n}{a}\PYG{o}{.}\PYG{n}{update}\PYG{p}{(}\PYG{p}{)}
\end{Verbatim}

\end{fulllineitems}



\begin{fulllineitems}
\pysigline{\sphinxbfcode{Make~a~Canvas~object~to~work~with:}}~
\begin{Verbatim}[commandchars=\\\{\}]
\PYG{n}{a}\PYG{o}{=}\PYG{n}{vcs}\PYG{o}{.}\PYG{n}{init}\PYG{p}{(}\PYG{p}{)}
\end{Verbatim}

\end{fulllineitems}



\begin{fulllineitems}
\pysigline{\sphinxbfcode{Create~a~new~instance~of~line:}}~
\begin{Verbatim}[commandchars=\\\{\}]
\PYG{c+c1}{\PYGZsh{} Copies content of \PYGZsq{}red\PYGZsq{} to \PYGZsq{}new\PYGZsq{}}
\PYG{n}{ln}\PYG{o}{=}\PYG{n}{a}\PYG{o}{.}\PYG{n}{createline}\PYG{p}{(}\PYG{l+s+s1}{\PYGZsq{}}\PYG{l+s+s1}{new}\PYG{l+s+s1}{\PYGZsq{}}\PYG{p}{,}\PYG{l+s+s1}{\PYGZsq{}}\PYG{l+s+s1}{red}\PYG{l+s+s1}{\PYGZsq{}}\PYG{p}{)}
\PYG{c+c1}{\PYGZsh{} Copies content of \PYGZsq{}default\PYGZsq{} to \PYGZsq{}new\PYGZsq{}}
\PYG{n}{ln}\PYG{o}{=}\PYG{n}{a}\PYG{o}{.}\PYG{n}{createline}\PYG{p}{(}\PYG{l+s+s1}{\PYGZsq{}}\PYG{l+s+s1}{new}\PYG{l+s+s1}{\PYGZsq{}}\PYG{p}{)}
\end{Verbatim}

\end{fulllineitems}



\begin{fulllineitems}
\pysigline{\sphinxbfcode{Modify~an~existing~line:}}~
\begin{Verbatim}[commandchars=\\\{\}]
\PYG{c+c1}{\PYGZsh{} Get a copy of \PYGZsq{}red\PYGZsq{} line}
\PYG{n}{ln}\PYG{o}{=}\PYG{n}{a}\PYG{o}{.}\PYG{n}{getline}\PYG{p}{(}\PYG{l+s+s1}{\PYGZsq{}}\PYG{l+s+s1}{red}\PYG{l+s+s1}{\PYGZsq{}}\PYG{p}{)}
\end{Verbatim}

\end{fulllineitems}



\begin{fulllineitems}
\pysigline{\sphinxbfcode{Overview~of~line~attributes:}}~\begin{itemize}
\item {} 
Listing line attributes:
\begin{quote}

\begin{Verbatim}[commandchars=\\\{\}]
\PYG{c+c1}{\PYGZsh{} Will list all the line attribute values}
\PYG{n}{ln}\PYG{o}{.}\PYG{n}{list}\PYG{p}{(}\PYG{p}{)}
\PYG{c+c1}{\PYGZsh{} Range from 1 to 256}
\PYG{n}{ln}\PYG{o}{.}\PYG{n}{color}\PYG{o}{=}\PYG{l+m+mi}{100}
\PYG{c+c1}{\PYGZsh{} Range from 1 to 300}
\PYG{n}{ln}\PYG{o}{.}\PYG{n}{width}\PYG{o}{=}\PYG{l+m+mi}{100}
\end{Verbatim}
\end{quote}

\item {} 
Specifying the line type:
\begin{quote}

\begin{Verbatim}[commandchars=\\\{\}]
\PYG{c+c1}{\PYGZsh{} Same as ln.type=0}
\PYG{n}{ln}\PYG{o}{.}\PYG{n}{type}\PYG{o}{=}\PYG{l+s+s1}{\PYGZsq{}}\PYG{l+s+s1}{solid}\PYG{l+s+s1}{\PYGZsq{}}
\PYG{c+c1}{\PYGZsh{} Same as ln.type=1}
\PYG{n}{ln}\PYG{o}{.}\PYG{n}{type}\PYG{o}{=}\PYG{l+s+s1}{\PYGZsq{}}\PYG{l+s+s1}{dash}\PYG{l+s+s1}{\PYGZsq{}}
\PYG{c+c1}{\PYGZsh{} Same as ln.type=2}
\PYG{n}{ln}\PYG{o}{.}\PYG{n}{type}\PYG{o}{=}\PYG{l+s+s1}{\PYGZsq{}}\PYG{l+s+s1}{dot}\PYG{l+s+s1}{\PYGZsq{}}
\PYG{c+c1}{\PYGZsh{} Same as ln.type=3}
\PYG{n}{ln}\PYG{o}{.}\PYG{n}{type}\PYG{o}{=}\PYG{l+s+s1}{\PYGZsq{}}\PYG{l+s+s1}{dash\PYGZhy{}dot}\PYG{l+s+s1}{\PYGZsq{}}
\PYG{c+c1}{\PYGZsh{} Same as ln.type=4}
\PYG{n}{ln}\PYG{o}{.}\PYG{n}{type}\PYG{o}{=}\PYG{l+s+s1}{\PYGZsq{}}\PYG{l+s+s1}{long\PYGZhy{}dash}\PYG{l+s+s1}{\PYGZsq{}}
\end{Verbatim}
\end{quote}

\end{itemize}

\end{fulllineitems}


\end{fulllineitems}



\subsection{Pytickmarks}
\label{vcs/template/Pytickmarks:pytickmarks}\label{vcs/template/Pytickmarks::doc}\label{vcs/template/Pytickmarks:module-vcs.Pytickmarks}\index{vcs.Pytickmarks (module)}
\# Template Y - Tick Marks (Pyt) module
\index{Pyt (class in vcs.Pytickmarks)}

\begin{fulllineitems}
\phantomsection\label{vcs/template/Pytickmarks:vcs.Pytickmarks.Pyt}\pysiglinewithargsret{\sphinxstrong{class }\sphinxcode{vcs.Pytickmarks.}\sphinxbfcode{Pyt}}{\emph{member}}{}
The Template text object allows the manipulation of line type, width, and color index.

This class is used to define an line table entry used in VCS, or it
can be used to change some or all of the line attributes in an
existing line table entry.


\begin{fulllineitems}
\pysigline{\sphinxbfcode{Useful~Functions:}}~
\begin{Verbatim}[commandchars=\\\{\}]
\PYG{c+c1}{\PYGZsh{} VCS Canvas Constructor}
\PYG{n}{a}\PYG{o}{=}\PYG{n}{vcs}\PYG{o}{.}\PYG{n}{init}\PYG{p}{(}\PYG{p}{)}
\PYG{c+c1}{\PYGZsh{} Show predefined line objects}
\PYG{n}{a}\PYG{o}{.}\PYG{n}{show}\PYG{p}{(}\PYG{l+s+s1}{\PYGZsq{}}\PYG{l+s+s1}{line}\PYG{l+s+s1}{\PYGZsq{}}\PYG{p}{)}
\PYG{c+c1}{\PYGZsh{} Updates the VCS Canvas at user\PYGZsq{}s request}
\PYG{n}{a}\PYG{o}{.}\PYG{n}{update}\PYG{p}{(}\PYG{p}{)}
\end{Verbatim}

\end{fulllineitems}



\begin{fulllineitems}
\pysigline{\sphinxbfcode{Make~a~Canvas~object~to~work~with:}}~
\begin{Verbatim}[commandchars=\\\{\}]
\PYG{n}{a}\PYG{o}{=}\PYG{n}{vcs}\PYG{o}{.}\PYG{n}{init}\PYG{p}{(}\PYG{p}{)}
\end{Verbatim}

\end{fulllineitems}



\begin{fulllineitems}
\pysigline{\sphinxbfcode{Create~a~new~instance~of~line:}}~
\begin{Verbatim}[commandchars=\\\{\}]
\PYG{c+c1}{\PYGZsh{} Copies content of \PYGZsq{}red\PYGZsq{} to \PYGZsq{}new\PYGZsq{}}
\PYG{n}{ln}\PYG{o}{=}\PYG{n}{a}\PYG{o}{.}\PYG{n}{createline}\PYG{p}{(}\PYG{l+s+s1}{\PYGZsq{}}\PYG{l+s+s1}{new}\PYG{l+s+s1}{\PYGZsq{}}\PYG{p}{,}\PYG{l+s+s1}{\PYGZsq{}}\PYG{l+s+s1}{red}\PYG{l+s+s1}{\PYGZsq{}}\PYG{p}{)}
\PYG{c+c1}{\PYGZsh{} Copies content of \PYGZsq{}default\PYGZsq{} to \PYGZsq{}new\PYGZsq{}}
\PYG{n}{ln}\PYG{o}{=}\PYG{n}{a}\PYG{o}{.}\PYG{n}{createline}\PYG{p}{(}\PYG{l+s+s1}{\PYGZsq{}}\PYG{l+s+s1}{new}\PYG{l+s+s1}{\PYGZsq{}}\PYG{p}{)}
\end{Verbatim}

\end{fulllineitems}



\begin{fulllineitems}
\pysigline{\sphinxbfcode{Modify~an~existing~line:}}~
\begin{Verbatim}[commandchars=\\\{\}]
\PYG{c+c1}{\PYGZsh{} Get a copy of \PYGZsq{}red\PYGZsq{} line}
\PYG{n}{ln}\PYG{o}{=}\PYG{n}{a}\PYG{o}{.}\PYG{n}{getline}\PYG{p}{(}\PYG{l+s+s1}{\PYGZsq{}}\PYG{l+s+s1}{red}\PYG{l+s+s1}{\PYGZsq{}}\PYG{p}{)}
\end{Verbatim}

\end{fulllineitems}



\begin{fulllineitems}
\pysigline{\sphinxbfcode{Overview~of~line~attributes:}}~\begin{itemize}
\item {} 
Listing line attributes:
\begin{quote}

\begin{Verbatim}[commandchars=\\\{\}]
\PYG{c+c1}{\PYGZsh{} Will list all the line attribute values}
\PYG{n}{ln}\PYG{o}{.}\PYG{n}{list}\PYG{p}{(}\PYG{p}{)}
\PYG{c+c1}{\PYGZsh{} Range from 1 to 256}
\PYG{n}{ln}\PYG{o}{.}\PYG{n}{color}\PYG{o}{=}\PYG{l+m+mi}{100}
\PYG{c+c1}{\PYGZsh{} Range from 1 to 300}
\PYG{n}{ln}\PYG{o}{.}\PYG{n}{width}\PYG{o}{=}\PYG{l+m+mi}{100}
\end{Verbatim}
\end{quote}

\item {} 
Specifying the line type:
\begin{quote}

\begin{Verbatim}[commandchars=\\\{\}]
\PYG{c+c1}{\PYGZsh{} Same as ln.type=0}
\PYG{n}{ln}\PYG{o}{.}\PYG{n}{type}\PYG{o}{=}\PYG{l+s+s1}{\PYGZsq{}}\PYG{l+s+s1}{solid}\PYG{l+s+s1}{\PYGZsq{}}
\PYG{c+c1}{\PYGZsh{} Same as ln.type=1}
\PYG{n}{ln}\PYG{o}{.}\PYG{n}{type}\PYG{o}{=}\PYG{l+s+s1}{\PYGZsq{}}\PYG{l+s+s1}{dash}\PYG{l+s+s1}{\PYGZsq{}}
\PYG{c+c1}{\PYGZsh{} Same as ln.type=2}
\PYG{n}{ln}\PYG{o}{.}\PYG{n}{type}\PYG{o}{=}\PYG{l+s+s1}{\PYGZsq{}}\PYG{l+s+s1}{dot}\PYG{l+s+s1}{\PYGZsq{}}
\PYG{c+c1}{\PYGZsh{} Same as ln.type=3}
\PYG{n}{ln}\PYG{o}{.}\PYG{n}{type}\PYG{o}{=}\PYG{l+s+s1}{\PYGZsq{}}\PYG{l+s+s1}{dash\PYGZhy{}dot}\PYG{l+s+s1}{\PYGZsq{}}
\PYG{c+c1}{\PYGZsh{} Same as ln.type=4}
\PYG{n}{ln}\PYG{o}{.}\PYG{n}{type}\PYG{o}{=}\PYG{l+s+s1}{\PYGZsq{}}\PYG{l+s+s1}{long\PYGZhy{}dash}\PYG{l+s+s1}{\PYGZsq{}}
\end{Verbatim}
\end{quote}

\end{itemize}

\end{fulllineitems}


\end{fulllineitems}



\section{Secondary Graphics Methods}
\label{vcs/secondary/gms::doc}\label{vcs/secondary/gms:secondary-graphics-methods}
Secondary graphics methods define primitives that can be used to create arbitrary shapes, lines, glyphs, and labels on your visualization.


\subsection{fillarea}
\label{vcs/secondary/fillarea:module-vcs.fillarea}\label{vcs/secondary/fillarea:fillarea}\label{vcs/secondary/fillarea::doc}\index{vcs.fillarea (module)}
\# Fillarea (Tf) module
\index{Tf (class in vcs.fillarea)}

\begin{fulllineitems}
\phantomsection\label{vcs/secondary/fillarea:vcs.fillarea.Tf}\pysiglinewithargsret{\sphinxstrong{class }\sphinxcode{vcs.fillarea.}\sphinxbfcode{Tf}}{\emph{Tf\_name=None}, \emph{Tf\_name\_src='default'}}{}
The Fillarea class object allows the user to edit fillarea attributes, including
fillarea interior style, style index, and color index.

This class is used to define an fillarea table entry used in VCS, or it
can be used to change some or all of the fillarea attributes in an
existing fillarea table entry.


\begin{fulllineitems}
\pysigline{\sphinxbfcode{Useful~Functions:}}~
\begin{Verbatim}[commandchars=\\\{\}]
\PYG{c+c1}{\PYGZsh{} VCS Canvas Constructor}
\PYG{n}{a}\PYG{o}{=}\PYG{n}{vcs}\PYG{o}{.}\PYG{n}{init}\PYG{p}{(}\PYG{p}{)}
\PYG{c+c1}{\PYGZsh{} Show predefined fillarea objects}
\PYG{n}{a}\PYG{o}{.}\PYG{n}{show}\PYG{p}{(}\PYG{l+s+s1}{\PYGZsq{}}\PYG{l+s+s1}{fillarea}\PYG{l+s+s1}{\PYGZsq{}}\PYG{p}{)}
\PYG{c+c1}{\PYGZsh{} Updates the VCS Canvas at user\PYGZsq{}s request}
\PYG{n}{a}\PYG{o}{.}\PYG{n}{update}\PYG{p}{(}\PYG{p}{)}
\end{Verbatim}

\end{fulllineitems}



\begin{fulllineitems}
\pysigline{\sphinxbfcode{Create~a~fillarea~object:}}~
\begin{Verbatim}[commandchars=\\\{\}]
\PYG{c+c1}{\PYGZsh{}Create a VCS Canvas object}
\PYG{n}{a}\PYG{o}{=}\PYG{n}{vcs}\PYG{o}{.}\PYG{n}{init}\PYG{p}{(}\PYG{p}{)}

\PYG{c+c1}{\PYGZsh{} Two ways to create a fillarea:}

\PYG{c+c1}{\PYGZsh{} Copies content of \PYGZsq{}def37\PYGZsq{} to \PYGZsq{}new\PYGZsq{}ea:}
\PYG{n}{fa}\PYG{o}{=}\PYG{n}{a}\PYG{o}{.}\PYG{n}{createfillarea}\PYG{p}{(}\PYG{l+s+s1}{\PYGZsq{}}\PYG{l+s+s1}{new}\PYG{l+s+s1}{\PYGZsq{}}\PYG{p}{,}\PYG{l+s+s1}{\PYGZsq{}}\PYG{l+s+s1}{def37}\PYG{l+s+s1}{\PYGZsq{}}\PYG{p}{)}
\PYG{c+c1}{\PYGZsh{} Copies content of \PYGZsq{}default\PYGZsq{} to \PYGZsq{}new\PYGZsq{}}
\PYG{n}{fa}\PYG{o}{=}\PYG{n}{a}\PYG{o}{.}\PYG{n}{createfillarea}\PYG{p}{(}\PYG{l+s+s1}{\PYGZsq{}}\PYG{l+s+s1}{new}\PYG{l+s+s1}{\PYGZsq{}}\PYG{p}{)}
\end{Verbatim}

\end{fulllineitems}



\begin{fulllineitems}
\pysigline{\sphinxbfcode{Modify~an~existing~fillarea:}}~
\begin{Verbatim}[commandchars=\\\{\}]
\PYG{n}{fa}\PYG{o}{=}\PYG{n}{a}\PYG{o}{.}\PYG{n}{getfillarea}\PYG{p}{(}\PYG{l+s+s1}{\PYGZsq{}}\PYG{l+s+s1}{red}\PYG{l+s+s1}{\PYGZsq{}}\PYG{p}{)}
\end{Verbatim}

\end{fulllineitems}

\begin{itemize}
\item {} 
Overview of fillarea attributes:
\begin{itemize}
\item {} 
List all the fillarea attribute values
\begin{quote}

\begin{Verbatim}[commandchars=\\\{\}]
\PYG{n}{fa}\PYG{o}{.}\PYG{n}{list}\PYG{p}{(}\PYG{p}{)}
\end{Verbatim}
\end{quote}

\item {} 
There are three possibilities for setting the isofill style:
\begin{quote}

\begin{Verbatim}[commandchars=\\\{\}]
\PYG{n}{fa}\PYG{o}{.}\PYG{n}{style} \PYG{o}{=} \PYG{l+s+s1}{\PYGZsq{}}\PYG{l+s+s1}{solid}\PYG{l+s+s1}{\PYGZsq{}}
\PYG{n}{fa}\PYG{o}{.}\PYG{n}{style} \PYG{o}{=} \PYG{l+s+s1}{\PYGZsq{}}\PYG{l+s+s1}{hatch}\PYG{l+s+s1}{\PYGZsq{}}
\PYG{n}{fa}\PYG{o}{.}\PYG{n}{style} \PYG{o}{=} \PYG{l+s+s1}{\PYGZsq{}}\PYG{l+s+s1}{pattern}\PYG{l+s+s1}{\PYGZsq{}}
\end{Verbatim}
\end{quote}

\item {} 
Setting index, color, opacity:
\begin{quote}

\begin{Verbatim}[commandchars=\\\{\}]
\PYG{c+c1}{\PYGZsh{} Range from 1 to 20}
\PYG{n}{fa}\PYG{o}{.}\PYG{n}{index}\PYG{o}{=}\PYG{l+m+mi}{1}
\PYG{c+c1}{\PYGZsh{} Range from 1 to 256}
\PYG{n}{fa}\PYG{o}{.}\PYG{n}{color}\PYG{o}{=}\PYG{l+m+mi}{100}
\PYG{c+c1}{\PYGZsh{} Range from 0 to 100}
\PYG{n}{fa}\PYG{o}{.}\PYG{n}{opacity}\PYG{o}{=}\PYG{l+m+mi}{100}
\end{Verbatim}
\end{quote}

\item {} 
Setting the graphics priority viewport, worldcoordinate:
\begin{quote}

\begin{Verbatim}[commandchars=\\\{\}]
\PYG{n}{fa}\PYG{o}{.}\PYG{n}{priority}\PYG{o}{=}\PYG{l+m+mi}{1}
\PYG{c+c1}{\PYGZsh{} FloatType [0,1]x[0,1]}
\PYG{n}{fa}\PYG{o}{.}\PYG{n}{viewport}\PYG{o}{=}\PYG{p}{[}\PYG{l+m+mi}{0}\PYG{p}{,} \PYG{l+m+mf}{1.0}\PYG{p}{,} \PYG{l+m+mi}{0}\PYG{p}{,}\PYG{l+m+mf}{1.0}\PYG{p}{]}
\PYG{c+c1}{\PYGZsh{} FloatType [\PYGZsh{},\PYGZsh{}]x[\PYGZsh{},\PYGZsh{}]}
\PYG{n}{fa}\PYG{o}{.}\PYG{n}{worldcoordinate}\PYG{o}{=}\PYG{p}{[}\PYG{l+m+mi}{0}\PYG{p}{,}\PYG{l+m+mf}{1.0}\PYG{p}{,}\PYG{l+m+mi}{0}\PYG{p}{,}\PYG{l+m+mf}{1.0}\PYG{p}{]}
\end{Verbatim}
\end{quote}

\item {} 
Setting x and y values:
\begin{quote}

\begin{Verbatim}[commandchars=\\\{\}]
\PYG{c+c1}{\PYGZsh{}List of FloatTypes}
\PYG{n}{fa}\PYG{o}{.}\PYG{n}{x}\PYG{o}{=}\PYG{p}{[}\PYG{p}{[}\PYG{l+m+mi}{0}\PYG{p}{,}\PYG{o}{.}\PYG{l+m+mi}{1}\PYG{p}{,}\PYG{o}{.}\PYG{l+m+mi}{2}\PYG{p}{]}\PYG{p}{,} \PYG{p}{[}\PYG{o}{.}\PYG{l+m+mi}{3}\PYG{p}{,}\PYG{o}{.}\PYG{l+m+mi}{4}\PYG{p}{,}\PYG{o}{.}\PYG{l+m+mi}{5}\PYG{p}{]}\PYG{p}{]}
\PYG{c+c1}{\PYGZsh{} List of FloatTypes}
\PYG{n}{fa}\PYG{o}{.}\PYG{n}{y}\PYG{o}{=}\PYG{p}{[}\PYG{p}{[}\PYG{o}{.}\PYG{l+m+mi}{5}\PYG{p}{,}\PYG{o}{.}\PYG{l+m+mi}{4}\PYG{p}{,}\PYG{o}{.}\PYG{l+m+mi}{3}\PYG{p}{]}\PYG{p}{,} \PYG{p}{[}\PYG{o}{.}\PYG{l+m+mi}{2}\PYG{p}{,}\PYG{o}{.}\PYG{l+m+mi}{1}\PYG{p}{,}\PYG{l+m+mi}{0}\PYG{p}{]}\PYG{p}{]}
\end{Verbatim}
\end{quote}

\end{itemize}

\end{itemize}
\index{script() (vcs.fillarea.Tf method)}

\begin{fulllineitems}
\phantomsection\label{vcs/secondary/fillarea:vcs.fillarea.Tf.script}\pysiglinewithargsret{\sphinxbfcode{script}}{\emph{script\_filename=None}, \emph{mode=None}}{}
Saves out a copy of the fillarea secondary method in JSON, or Python format to a designated file.
\begin{quote}

\begin{notice}{note}{Note:}
If the the filename has a `.py' at the end, it will produce a
Python script. If no extension is given, then by default a
.json file containing a JSON serialization of the object's
data will be produced.
\end{notice}

\begin{notice}{warning}{Warning:}
VCS Scripts Deprecated.
SCR script files are no longer generated by this function.
\end{notice}
\end{quote}
\begin{quote}\begin{description}
\item[{Example}] \leavevmode
\begin{Verbatim}[commandchars=\\\{\}]
\PYG{g+gp}{\PYGZgt{}\PYGZgt{}\PYGZgt{} }\PYG{n}{a}\PYG{o}{=}\PYG{n}{vcs}\PYG{o}{.}\PYG{n}{init}\PYG{p}{(}\PYG{p}{)} \PYG{c+c1}{\PYGZsh{} Make a Canvas object to work with}
\PYG{g+gp}{\PYGZgt{}\PYGZgt{}\PYGZgt{} }\PYG{n}{ex}\PYG{o}{=}\PYG{n}{a}\PYG{o}{.}\PYG{n}{getfillarea}\PYG{p}{(}\PYG{p}{)} \PYG{c+c1}{\PYGZsh{} Get default fillarea}
\PYG{g+gp}{\PYGZgt{}\PYGZgt{}\PYGZgt{} }\PYG{n}{ex}\PYG{o}{.}\PYG{n}{script}\PYG{p}{(}\PYG{l+s+s1}{\PYGZsq{}}\PYG{l+s+s1}{filename.py}\PYG{l+s+s1}{\PYGZsq{}}\PYG{p}{)} \PYG{c+c1}{\PYGZsh{} Append to a Python script named \PYGZsq{}filename.py\PYGZsq{}}
\PYG{g+gp}{\PYGZgt{}\PYGZgt{}\PYGZgt{} }\PYG{n}{ex}\PYG{o}{.}\PYG{n}{script}\PYG{p}{(}\PYG{l+s+s1}{\PYGZsq{}}\PYG{l+s+s1}{filename}\PYG{l+s+s1}{\PYGZsq{}}\PYG{p}{,}\PYG{l+s+s1}{\PYGZsq{}}\PYG{l+s+s1}{w}\PYG{l+s+s1}{\PYGZsq{}}\PYG{p}{)} \PYG{c+c1}{\PYGZsh{} Create or overwrite a JSON file \PYGZsq{}filename.json\PYGZsq{}.}
\end{Verbatim}

\item[{Parameters}] \leavevmode\begin{itemize}
\item {} 
\textbf{\texttt{script\_filename}} (\href{https://docs.python.org/2/library/functions.html\#str}{\emph{\texttt{str}}}) -- Output name of the script file. If no extension is specified, a .json object is created.

\item {} 
\textbf{\texttt{mode}} (\href{https://docs.python.org/2/library/functions.html\#str}{\emph{\texttt{str}}}) -- Either `w' for replace, or `a' for append. Defaults to `a', if not specified.

\end{itemize}

\end{description}\end{quote}

\end{fulllineitems}


\end{fulllineitems}



\subsection{line}
\label{vcs/secondary/line:module-vcs.line}\label{vcs/secondary/line::doc}\label{vcs/secondary/line:line}\index{vcs.line (module)}
\# Line (Tl) module
\index{Tl (class in vcs.line)}

\begin{fulllineitems}
\phantomsection\label{vcs/secondary/line:vcs.line.Tl}\pysiglinewithargsret{\sphinxstrong{class }\sphinxcode{vcs.line.}\sphinxbfcode{Tl}}{\emph{Tl\_name}, \emph{Tl\_name\_src='default'}}{}
The Line object allows the manipulation of line type, width, color index,
view port, world coordinates, and (x,y) points.

This class is used to define an line table entry used in VCS, or it
can be used to change some or all of the line attributes in an
existing line table entry.


\begin{fulllineitems}
\pysigline{\sphinxbfcode{Useful~Functions:}}~
\begin{Verbatim}[commandchars=\\\{\}]
\PYG{c+c1}{\PYGZsh{} VCS Canvas Constructor}
\PYG{n}{a}\PYG{o}{=}\PYG{n}{vcs}\PYG{o}{.}\PYG{n}{init}\PYG{p}{(}\PYG{p}{)}
\PYG{c+c1}{\PYGZsh{} Show predefined line objects}
\PYG{n}{a}\PYG{o}{.}\PYG{n}{show}\PYG{p}{(}\PYG{l+s+s1}{\PYGZsq{}}\PYG{l+s+s1}{line}\PYG{l+s+s1}{\PYGZsq{}}\PYG{p}{)}
\PYG{c+c1}{\PYGZsh{} Will list all the line attribute values}
\PYG{n}{ln}\PYG{o}{.}\PYG{n}{list}\PYG{p}{(}\PYG{p}{)}
\PYG{c+c1}{\PYGZsh{} Updates the VCS Canvas at user\PYGZsq{}s request}
\PYG{n}{a}\PYG{o}{.}\PYG{n}{update}\PYG{p}{(}\PYG{p}{)}
\end{Verbatim}

\end{fulllineitems}



\begin{fulllineitems}
\pysigline{\sphinxbfcode{Create~a~new~instance~of~line:}}~
\begin{Verbatim}[commandchars=\\\{\}]
\PYG{c+c1}{\PYGZsh{} Copies content of \PYGZsq{}red\PYGZsq{} to \PYGZsq{}new\PYGZsq{}}
\PYG{n}{ln}\PYG{o}{=}\PYG{n}{a}\PYG{o}{.}\PYG{n}{createline}\PYG{p}{(}\PYG{l+s+s1}{\PYGZsq{}}\PYG{l+s+s1}{new}\PYG{l+s+s1}{\PYGZsq{}}\PYG{p}{,}\PYG{l+s+s1}{\PYGZsq{}}\PYG{l+s+s1}{red}\PYG{l+s+s1}{\PYGZsq{}}\PYG{p}{)}
\PYG{c+c1}{\PYGZsh{} Copies content of \PYGZsq{}default\PYGZsq{} to \PYGZsq{}new\PYGZsq{}}
\PYG{n}{ln}\PYG{o}{=}\PYG{n}{a}\PYG{o}{.}\PYG{n}{createline}\PYG{p}{(}\PYG{l+s+s1}{\PYGZsq{}}\PYG{l+s+s1}{new}\PYG{l+s+s1}{\PYGZsq{}}\PYG{p}{)}
\end{Verbatim}

\end{fulllineitems}



\begin{fulllineitems}
\pysigline{\sphinxbfcode{Modify~an~existing~line:}}~\begin{itemize}
\item {} 
Get a line object `ln' to manipulate:
\begin{quote}

\begin{Verbatim}[commandchars=\\\{\}]
\PYG{n}{ln}\PYG{o}{=}\PYG{n}{a}\PYG{o}{.}\PYG{n}{getline}\PYG{p}{(}\PYG{l+s+s1}{\PYGZsq{}}\PYG{l+s+s1}{red}\PYG{l+s+s1}{\PYGZsq{}}\PYG{p}{)}
\end{Verbatim}
\end{quote}

\item {} 
Set line color:
\begin{quote}

\begin{Verbatim}[commandchars=\\\{\}]
\PYG{c+c1}{\PYGZsh{} Range from 1 to 256}
\PYG{n}{ln}\PYG{o}{.}\PYG{n}{color}\PYG{o}{=}\PYG{l+m+mi}{100}
\end{Verbatim}
\end{quote}

\item {} 
Set line width:
\begin{quote}

\begin{Verbatim}[commandchars=\\\{\}]
\PYG{c+c1}{\PYGZsh{} Range from 1 to 300}
\PYG{n}{ln}\PYG{o}{.}\PYG{n}{width}\PYG{o}{=}\PYG{l+m+mi}{100}
\end{Verbatim}
\end{quote}

\item {} 
Specify the line type:
\begin{quote}

\begin{Verbatim}[commandchars=\\\{\}]
\PYG{c+c1}{\PYGZsh{} Same as ln.type=0}
 \PYG{n}{ln}\PYG{o}{.}\PYG{n}{type}\PYG{o}{=}\PYG{l+s+s1}{\PYGZsq{}}\PYG{l+s+s1}{solid}\PYG{l+s+s1}{\PYGZsq{}}
 \PYG{c+c1}{\PYGZsh{} Same as ln.type=1}
 \PYG{n}{ln}\PYG{o}{.}\PYG{n}{type}\PYG{o}{=}\PYG{l+s+s1}{\PYGZsq{}}\PYG{l+s+s1}{dash}\PYG{l+s+s1}{\PYGZsq{}}
 \PYG{c+c1}{\PYGZsh{} Same as ln.type=2}
 \PYG{n}{ln}\PYG{o}{.}\PYG{n}{type}\PYG{o}{=}\PYG{l+s+s1}{\PYGZsq{}}\PYG{l+s+s1}{dot}\PYG{l+s+s1}{\PYGZsq{}}
 \PYG{c+c1}{\PYGZsh{} Same as ln.type=3}
 \PYG{n}{ln}\PYG{o}{.}\PYG{n}{type}\PYG{o}{=}\PYG{l+s+s1}{\PYGZsq{}}\PYG{l+s+s1}{dash\PYGZhy{}dot}\PYG{l+s+s1}{\PYGZsq{}}
 \PYG{c+c1}{\PYGZsh{} Same as ln.type=4}
 \PYG{n}{ln}\PYG{o}{.}\PYG{n}{type}\PYG{o}{=}\PYG{l+s+s1}{\PYGZsq{}}\PYG{l+s+s1}{long\PYGZhy{}dash}\PYG{l+s+s1}{\PYGZsq{}}
\end{Verbatim}
\end{quote}

\item {} 
Set the graphics priority on the canvas:
\begin{quote}

\begin{Verbatim}[commandchars=\\\{\}]
\PYG{n}{ln}\PYG{o}{.}\PYG{n}{priority}\PYG{o}{=}\PYG{l+m+mi}{1}
\PYG{c+c1}{\PYGZsh{} FloatType [0,1]x[0,1]}
\PYG{n}{ln}\PYG{o}{.}\PYG{n}{viewport}\PYG{o}{=}\PYG{p}{[}\PYG{l+m+mi}{0}\PYG{p}{,} \PYG{l+m+mf}{1.0}\PYG{p}{,} \PYG{l+m+mi}{0}\PYG{p}{,}\PYG{l+m+mf}{1.0}\PYG{p}{]}
\PYG{c+c1}{\PYGZsh{} FloatType [\PYGZsh{},\PYGZsh{}]x[\PYGZsh{},\PYGZsh{}]}
\PYG{n}{ln}\PYG{o}{.}\PYG{n}{worldcoordinate}\PYG{o}{=}\PYG{p}{[}\PYG{l+m+mi}{0}\PYG{p}{,}\PYG{l+m+mf}{1.0}\PYG{p}{,}\PYG{l+m+mi}{0}\PYG{p}{,}\PYG{l+m+mf}{1.0}\PYG{p}{]}
\end{Verbatim}
\end{quote}

\item {} 
Set line x and y values:
\begin{quote}

\begin{Verbatim}[commandchars=\\\{\}]
\PYG{c+c1}{\PYGZsh{} List of FloatTypes}
\PYG{n}{ln}\PYG{o}{.}\PYG{n}{x}\PYG{o}{=}\PYG{p}{[}\PYG{p}{[}\PYG{l+m+mi}{0}\PYG{p}{,}\PYG{o}{.}\PYG{l+m+mi}{1}\PYG{p}{,}\PYG{o}{.}\PYG{l+m+mi}{2}\PYG{p}{]}\PYG{p}{,} \PYG{p}{[}\PYG{o}{.}\PYG{l+m+mi}{3}\PYG{p}{,}\PYG{o}{.}\PYG{l+m+mi}{4}\PYG{p}{,}\PYG{o}{.}\PYG{l+m+mi}{5}\PYG{p}{]}\PYG{p}{]}
\PYG{c+c1}{\PYGZsh{} List of FloatTypes}
\PYG{n}{ln}\PYG{o}{.}\PYG{n}{y}\PYG{o}{=}\PYG{p}{[}\PYG{p}{[}\PYG{o}{.}\PYG{l+m+mi}{5}\PYG{p}{,}\PYG{o}{.}\PYG{l+m+mi}{4}\PYG{p}{,}\PYG{o}{.}\PYG{l+m+mi}{3}\PYG{p}{]}\PYG{p}{,} \PYG{p}{[}\PYG{o}{.}\PYG{l+m+mi}{2}\PYG{p}{,}\PYG{o}{.}\PYG{l+m+mi}{1}\PYG{p}{,}\PYG{l+m+mi}{0}\PYG{p}{]}\PYG{p}{]}
\end{Verbatim}
\end{quote}

\end{itemize}

\end{fulllineitems}

\index{script() (vcs.line.Tl method)}

\begin{fulllineitems}
\phantomsection\label{vcs/secondary/line:vcs.line.Tl.script}\pysiglinewithargsret{\sphinxbfcode{script}}{\emph{script\_filename=None}, \emph{mode=None}}{}
Saves out a copy of the line secondary method in JSON, or Python format to a designated file.
\begin{quote}

\begin{notice}{note}{Note:}
If the the filename has a `.py' at the end, it will produce a
Python script. If no extension is given, then by default a
.json file containing a JSON serialization of the object's
data will be produced.
\end{notice}

\begin{notice}{warning}{Warning:}
VCS Scripts Deprecated.
SCR script files are no longer generated by this function.
\end{notice}
\end{quote}
\begin{quote}\begin{description}
\item[{Example}] \leavevmode
\begin{Verbatim}[commandchars=\\\{\}]
\PYG{g+gp}{\PYGZgt{}\PYGZgt{}\PYGZgt{} }\PYG{n}{a}\PYG{o}{=}\PYG{n}{vcs}\PYG{o}{.}\PYG{n}{init}\PYG{p}{(}\PYG{p}{)} \PYG{c+c1}{\PYGZsh{} Make a Canvas object to work with}
\PYG{g+gp}{\PYGZgt{}\PYGZgt{}\PYGZgt{} }\PYG{n}{ex}\PYG{o}{=}\PYG{n}{a}\PYG{o}{.}\PYG{n}{getline}\PYG{p}{(}\PYG{p}{)} \PYG{c+c1}{\PYGZsh{} Get default line}
\PYG{g+gp}{\PYGZgt{}\PYGZgt{}\PYGZgt{} }\PYG{n}{ex}\PYG{o}{.}\PYG{n}{script}\PYG{p}{(}\PYG{l+s+s1}{\PYGZsq{}}\PYG{l+s+s1}{filename.py}\PYG{l+s+s1}{\PYGZsq{}}\PYG{p}{)} \PYG{c+c1}{\PYGZsh{} Append to a Python script named \PYGZsq{}filename.py\PYGZsq{}}
\PYG{g+gp}{\PYGZgt{}\PYGZgt{}\PYGZgt{} }\PYG{n}{ex}\PYG{o}{.}\PYG{n}{script}\PYG{p}{(}\PYG{l+s+s1}{\PYGZsq{}}\PYG{l+s+s1}{filename}\PYG{l+s+s1}{\PYGZsq{}}\PYG{p}{,}\PYG{l+s+s1}{\PYGZsq{}}\PYG{l+s+s1}{w}\PYG{l+s+s1}{\PYGZsq{}}\PYG{p}{)} \PYG{c+c1}{\PYGZsh{} Create or overwrite a JSON file \PYGZsq{}filename.json\PYGZsq{}.}
\end{Verbatim}

\item[{Parameters}] \leavevmode\begin{itemize}
\item {} 
\textbf{\texttt{script\_filename}} (\href{https://docs.python.org/2/library/functions.html\#str}{\emph{\texttt{str}}}) -- Output name of the script file. If no extension is specified, a .json object is created.

\item {} 
\textbf{\texttt{mode}} (\href{https://docs.python.org/2/library/functions.html\#str}{\emph{\texttt{str}}}) -- Either `w' for replace, or `a' for append. Defaults to `a', if not specified.

\end{itemize}

\end{description}\end{quote}

\end{fulllineitems}


\end{fulllineitems}



\subsection{marker}
\label{vcs/secondary/marker:marker}\label{vcs/secondary/marker:module-vcs.marker}\label{vcs/secondary/marker::doc}\index{vcs.marker (module)}\index{Tm (class in vcs.marker)}

\begin{fulllineitems}
\phantomsection\label{vcs/secondary/marker:vcs.marker.Tm}\pysiglinewithargsret{\sphinxstrong{class }\sphinxcode{vcs.marker.}\sphinxbfcode{Tm}}{\emph{Tm\_name}, \emph{Tm\_name\_src='default'}}{}
The Marker object allows the manipulation of marker type, size, and color index.

This class is used to define an marker table entry used in VCS, or it
can be used to change some or all of the marker attributes in an
existing marker table entry.


\begin{fulllineitems}
\pysigline{\sphinxbfcode{Useful~Functions:}}~
\begin{Verbatim}[commandchars=\\\{\}]
\PYG{c+c1}{\PYGZsh{} VCS Canvas Constructor}
\PYG{n}{a}\PYG{o}{=}\PYG{n}{vcs}\PYG{o}{.}\PYG{n}{init}\PYG{p}{(}\PYG{p}{)}
\PYG{c+c1}{\PYGZsh{} Show predefined marker objects}
\PYG{n}{a}\PYG{o}{.}\PYG{n}{show}\PYG{p}{(}\PYG{l+s+s1}{\PYGZsq{}}\PYG{l+s+s1}{marker}\PYG{l+s+s1}{\PYGZsq{}}\PYG{p}{)}
\PYG{c+c1}{\PYGZsh{} Updates the VCS Canvas at user\PYGZsq{}s request}
\PYG{n}{a}\PYG{o}{.}\PYG{n}{update}\PYG{p}{(}\PYG{p}{)}
\PYG{n}{a}\PYG{o}{=}\PYG{n}{vcs}\PYG{o}{.}\PYG{n}{init}\PYG{p}{(}\PYG{p}{)}
\end{Verbatim}

\end{fulllineitems}



\begin{fulllineitems}
\pysigline{\sphinxbfcode{Create~a~new~instance~of~marker:}}~
\begin{Verbatim}[commandchars=\\\{\}]
\PYG{c+c1}{\PYGZsh{} Copies content of \PYGZsq{}red\PYGZsq{} to \PYGZsq{}new\PYGZsq{}}
\PYG{n}{mk}\PYG{o}{=}\PYG{n}{a}\PYG{o}{.}\PYG{n}{createmarker}\PYG{p}{(}\PYG{l+s+s1}{\PYGZsq{}}\PYG{l+s+s1}{new}\PYG{l+s+s1}{\PYGZsq{}}\PYG{p}{,}\PYG{l+s+s1}{\PYGZsq{}}\PYG{l+s+s1}{red}\PYG{l+s+s1}{\PYGZsq{}}\PYG{p}{)}
\PYG{c+c1}{\PYGZsh{} Copies content of \PYGZsq{}default\PYGZsq{} to \PYGZsq{}new\PYGZsq{}}
\PYG{n}{mk}\PYG{o}{=}\PYG{n}{a}\PYG{o}{.}\PYG{n}{createmarker}\PYG{p}{(}\PYG{l+s+s1}{\PYGZsq{}}\PYG{l+s+s1}{new}\PYG{l+s+s1}{\PYGZsq{}}\PYG{p}{)}
\end{Verbatim}

\end{fulllineitems}



\begin{fulllineitems}
\pysigline{\sphinxbfcode{Modify~an~existing~marker:}}~
\begin{Verbatim}[commandchars=\\\{\}]
\PYG{n}{mk}\PYG{o}{=}\PYG{n}{a}\PYG{o}{.}\PYG{n}{getmarker}\PYG{p}{(}\PYG{l+s+s1}{\PYGZsq{}}\PYG{l+s+s1}{red}\PYG{l+s+s1}{\PYGZsq{}}\PYG{p}{)}
\end{Verbatim}

\end{fulllineitems}



\begin{fulllineitems}
\pysigline{\sphinxbfcode{Overview~of~marker~attributes:}}~\begin{itemize}
\item {} 
List all the marker attribute values:
\begin{quote}

\begin{Verbatim}[commandchars=\\\{\}]
\PYG{n}{mk}\PYG{o}{.}\PYG{n}{list}\PYG{p}{(}\PYG{p}{)}
\PYG{c+c1}{\PYGZsh{} Range from 1 to 256}
\PYG{n}{mk}\PYG{o}{.}\PYG{n}{color}\PYG{o}{=}\PYG{l+m+mi}{100}
\PYG{c+c1}{\PYGZsh{} Range from 1 to 300}
\PYG{n}{mk}\PYG{o}{.}\PYG{n}{size}\PYG{o}{=}\PYG{l+m+mi}{100}
\end{Verbatim}
\end{quote}

\item {} 
Specify the marker type:
\begin{quote}

\begin{Verbatim}[commandchars=\\\{\}]
\PYG{c+c1}{\PYGZsh{} Same as mk.type=1}
\PYG{n}{mk}\PYG{o}{.}\PYG{n}{type}\PYG{o}{=}\PYG{l+s+s1}{\PYGZsq{}}\PYG{l+s+s1}{dot}\PYG{l+s+s1}{\PYGZsq{}}
\PYG{c+c1}{\PYGZsh{} Same as mk.type=2}
\PYG{n}{mk}\PYG{o}{.}\PYG{n}{type}\PYG{o}{=}\PYG{l+s+s1}{\PYGZsq{}}\PYG{l+s+s1}{plus}\PYG{l+s+s1}{\PYGZsq{}}
\PYG{c+c1}{\PYGZsh{} Same as mk.type=3}
\PYG{n}{mk}\PYG{o}{.}\PYG{n}{type}\PYG{o}{=}\PYG{l+s+s1}{\PYGZsq{}}\PYG{l+s+s1}{star}\PYG{l+s+s1}{\PYGZsq{}}
\PYG{c+c1}{\PYGZsh{} Same as mk.type=4}
\PYG{n}{mk}\PYG{o}{.}\PYG{n}{type}\PYG{o}{=}\PYG{l+s+s1}{\PYGZsq{}}\PYG{l+s+s1}{circle}\PYG{l+s+s1}{\PYGZsq{}}
\PYG{c+c1}{\PYGZsh{} Same as mk.type=5}
\PYG{n}{mk}\PYG{o}{.}\PYG{n}{type}\PYG{o}{=}\PYG{l+s+s1}{\PYGZsq{}}\PYG{l+s+s1}{cross}\PYG{l+s+s1}{\PYGZsq{}}
\PYG{c+c1}{\PYGZsh{} Same as mk.type=6}
\PYG{n}{mk}\PYG{o}{.}\PYG{n}{type}\PYG{o}{=}\PYG{l+s+s1}{\PYGZsq{}}\PYG{l+s+s1}{diamond}\PYG{l+s+s1}{\PYGZsq{}}
\PYG{c+c1}{\PYGZsh{} Same as mk.type=7}
\PYG{n}{mk}\PYG{o}{.}\PYG{n}{type}\PYG{o}{=}\PYG{l+s+s1}{\PYGZsq{}}\PYG{l+s+s1}{triangle\PYGZus{}up}\PYG{l+s+s1}{\PYGZsq{}}
\PYG{c+c1}{\PYGZsh{} Same as mk.type=8}
\PYG{n}{mk}\PYG{o}{.}\PYG{n}{type}\PYG{o}{=}\PYG{l+s+s1}{\PYGZsq{}}\PYG{l+s+s1}{triangle\PYGZus{}down}\PYG{l+s+s1}{\PYGZsq{}}
\PYG{c+c1}{\PYGZsh{} Same as mk.type=9}
\PYG{n}{mk}\PYG{o}{.}\PYG{n}{type}\PYG{o}{=}\PYG{l+s+s1}{\PYGZsq{}}\PYG{l+s+s1}{triangle\PYGZus{}left}\PYG{l+s+s1}{\PYGZsq{}}
\PYG{c+c1}{\PYGZsh{} Same as mk.type=10}
\PYG{n}{mk}\PYG{o}{.}\PYG{n}{type}\PYG{o}{=}\PYG{l+s+s1}{\PYGZsq{}}\PYG{l+s+s1}{triangle\PYGZus{}right}\PYG{l+s+s1}{\PYGZsq{}}
\PYG{c+c1}{\PYGZsh{} Same as mk.type=11}
\PYG{n}{mk}\PYG{o}{.}\PYG{n}{type}\PYG{o}{=}\PYG{l+s+s1}{\PYGZsq{}}\PYG{l+s+s1}{square}\PYG{l+s+s1}{\PYGZsq{}}
\PYG{c+c1}{\PYGZsh{} Same as mk.type=12}
\PYG{n}{mk}\PYG{o}{.}\PYG{n}{type}\PYG{o}{=}\PYG{l+s+s1}{\PYGZsq{}}\PYG{l+s+s1}{diamond\PYGZus{}fill}\PYG{l+s+s1}{\PYGZsq{}}
\PYG{c+c1}{\PYGZsh{} Same as mk.type=13}
\PYG{n}{mk}\PYG{o}{.}\PYG{n}{type}\PYG{o}{=}\PYG{l+s+s1}{\PYGZsq{}}\PYG{l+s+s1}{triangle\PYGZus{}up\PYGZus{}fill}\PYG{l+s+s1}{\PYGZsq{}}
\PYG{c+c1}{\PYGZsh{} Same as mk.type=14}
\PYG{n}{mk}\PYG{o}{.}\PYG{n}{type}\PYG{o}{=}\PYG{l+s+s1}{\PYGZsq{}}\PYG{l+s+s1}{triangle\PYGZus{}down\PYGZus{}fill}\PYG{l+s+s1}{\PYGZsq{}}
\PYG{c+c1}{\PYGZsh{} Same as mk.type=15}
\PYG{n}{mk}\PYG{o}{.}\PYG{n}{type}\PYG{o}{=}\PYG{l+s+s1}{\PYGZsq{}}\PYG{l+s+s1}{triangle\PYGZus{}left\PYGZus{}fill}\PYG{l+s+s1}{\PYGZsq{}}
\PYG{c+c1}{\PYGZsh{} Same as mk.type=16}
\PYG{n}{mk}\PYG{o}{.}\PYG{n}{type}\PYG{o}{=}\PYG{l+s+s1}{\PYGZsq{}}\PYG{l+s+s1}{triangle\PYGZus{}right\PYGZus{}fill}\PYG{l+s+s1}{\PYGZsq{}}
\PYG{c+c1}{\PYGZsh{} Same as mk.type=17}
\PYG{n}{mk}\PYG{o}{.}\PYG{n}{type}\PYG{o}{=}\PYG{l+s+s1}{\PYGZsq{}}\PYG{l+s+s1}{square\PYGZus{}fill}\PYG{l+s+s1}{\PYGZsq{}}
\end{Verbatim}
\end{quote}

\item {} 
Set the graphics priority on the canvas
\begin{quote}

\begin{Verbatim}[commandchars=\\\{\}]
\PYG{n}{mk}\PYG{o}{.}\PYG{n}{priority}\PYG{o}{=}\PYG{l+m+mi}{1}
\PYG{c+c1}{\PYGZsh{} FloatType [0,1]x[0,1]}
\PYG{n}{mk}\PYG{o}{.}\PYG{n}{viewport}\PYG{o}{=}\PYG{p}{[}\PYG{l+m+mi}{0}\PYG{p}{,} \PYG{l+m+mf}{1.0}\PYG{p}{,} \PYG{l+m+mi}{0}\PYG{p}{,}\PYG{l+m+mf}{1.0}\PYG{p}{]}
\PYG{c+c1}{\PYGZsh{} FloatType [\PYGZsh{},\PYGZsh{}]x[\PYGZsh{},\PYGZsh{}]}
\PYG{n}{mk}\PYG{o}{.}\PYG{n}{worldcoordinate}\PYG{o}{=}\PYG{p}{[}\PYG{l+m+mi}{0}\PYG{p}{,}\PYG{l+m+mf}{1.0}\PYG{p}{,}\PYG{l+m+mi}{0}\PYG{p}{,}\PYG{l+m+mf}{1.0}\PYG{p}{]}
\end{Verbatim}
\end{quote}

\item {} 
Example x and y coordinates:
\begin{quote}

\begin{Verbatim}[commandchars=\\\{\}]
\PYG{c+c1}{\PYGZsh{} List of FloatTypes}
\PYG{n}{mk}\PYG{o}{.}\PYG{n}{x}\PYG{o}{=}\PYG{p}{[}\PYG{p}{[}\PYG{l+m+mi}{0}\PYG{p}{,}\PYG{o}{.}\PYG{l+m+mi}{1}\PYG{p}{,}\PYG{o}{.}\PYG{l+m+mi}{2}\PYG{p}{]}\PYG{p}{,} \PYG{p}{[}\PYG{o}{.}\PYG{l+m+mi}{3}\PYG{p}{,}\PYG{o}{.}\PYG{l+m+mi}{4}\PYG{p}{,}\PYG{o}{.}\PYG{l+m+mi}{5}\PYG{p}{]}\PYG{p}{]}
\PYG{c+c1}{\PYGZsh{} List of FloatTypes}
\PYG{n}{mk}\PYG{o}{.}\PYG{n}{y}\PYG{o}{=}\PYG{p}{[}\PYG{p}{[}\PYG{o}{.}\PYG{l+m+mi}{5}\PYG{p}{,}\PYG{o}{.}\PYG{l+m+mi}{4}\PYG{p}{,}\PYG{o}{.}\PYG{l+m+mi}{3}\PYG{p}{]}\PYG{p}{,} \PYG{p}{[}\PYG{o}{.}\PYG{l+m+mi}{2}\PYG{p}{,}\PYG{o}{.}\PYG{l+m+mi}{1}\PYG{p}{,}\PYG{l+m+mi}{0}\PYG{p}{]}\PYG{p}{]}
\end{Verbatim}
\end{quote}

\end{itemize}

\end{fulllineitems}

\index{script() (vcs.marker.Tm method)}

\begin{fulllineitems}
\phantomsection\label{vcs/secondary/marker:vcs.marker.Tm.script}\pysiglinewithargsret{\sphinxbfcode{script}}{\emph{script\_filename=None}, \emph{mode=None}}{}
Saves out a copy of the marker secondary method in JSON, or Python format to a designated file.
\begin{quote}

\begin{notice}{note}{Note:}
If the the filename has a `.py' at the end, it will produce a
Python script. If no extension is given, then by default a
.json file containing a JSON serialization of the object's
data will be produced.
\end{notice}

\begin{notice}{warning}{Warning:}
VCS Scripts Deprecated.
SCR script files are no longer generated by this function.
\end{notice}
\end{quote}
\begin{quote}\begin{description}
\item[{Example}] \leavevmode
\begin{Verbatim}[commandchars=\\\{\}]
\PYG{g+gp}{\PYGZgt{}\PYGZgt{}\PYGZgt{} }\PYG{n}{a}\PYG{o}{=}\PYG{n}{vcs}\PYG{o}{.}\PYG{n}{init}\PYG{p}{(}\PYG{p}{)} \PYG{c+c1}{\PYGZsh{} Make a Canvas object to work with}
\PYG{g+gp}{\PYGZgt{}\PYGZgt{}\PYGZgt{} }\PYG{n}{ex}\PYG{o}{=}\PYG{n}{a}\PYG{o}{.}\PYG{n}{getmarker}\PYG{p}{(}\PYG{p}{)} \PYG{c+c1}{\PYGZsh{} Get default marker}
\PYG{g+gp}{\PYGZgt{}\PYGZgt{}\PYGZgt{} }\PYG{n}{ex}\PYG{o}{.}\PYG{n}{script}\PYG{p}{(}\PYG{l+s+s1}{\PYGZsq{}}\PYG{l+s+s1}{filename.py}\PYG{l+s+s1}{\PYGZsq{}}\PYG{p}{)} \PYG{c+c1}{\PYGZsh{} Append to a Python script named \PYGZsq{}filename.py\PYGZsq{}}
\PYG{g+gp}{\PYGZgt{}\PYGZgt{}\PYGZgt{} }\PYG{n}{ex}\PYG{o}{.}\PYG{n}{script}\PYG{p}{(}\PYG{l+s+s1}{\PYGZsq{}}\PYG{l+s+s1}{filename}\PYG{l+s+s1}{\PYGZsq{}}\PYG{p}{,}\PYG{l+s+s1}{\PYGZsq{}}\PYG{l+s+s1}{w}\PYG{l+s+s1}{\PYGZsq{}}\PYG{p}{)} \PYG{c+c1}{\PYGZsh{} Create or overwrite a JSON file \PYGZsq{}filename.json\PYGZsq{}.}
\end{Verbatim}

\item[{Parameters}] \leavevmode\begin{itemize}
\item {} 
\textbf{\texttt{script\_filename}} (\href{https://docs.python.org/2/library/functions.html\#str}{\emph{\texttt{str}}}) -- Output name of the script file. If no extension is specified, a .json object is created.

\item {} 
\textbf{\texttt{mode}} (\href{https://docs.python.org/2/library/functions.html\#str}{\emph{\texttt{str}}}) -- Either `w' for replace, or `a' for append. Defaults to `a', if not specified.

\end{itemize}

\end{description}\end{quote}

\end{fulllineitems}


\end{fulllineitems}



\subsection{textcombined}
\label{vcs/secondary/textcombined:module-vcs.textcombined}\label{vcs/secondary/textcombined:textcombined}\label{vcs/secondary/textcombined::doc}\index{vcs.textcombined (module)}
\# Text Combined (Tc) module
\index{Tc (class in vcs.textcombined)}

\begin{fulllineitems}
\phantomsection\label{vcs/secondary/textcombined:vcs.textcombined.Tc}\pysiglinewithargsret{\sphinxstrong{class }\sphinxcode{vcs.textcombined.}\sphinxbfcode{Tc}}{\emph{Tt\_name=None}, \emph{Tt\_name\_src='default'}, \emph{To\_name=None}, \emph{To\_name\_src='default'}}{}
The (Tc) Text Combined class will combine a text table class and a text orientation
class together. From combining the two classess, the user will be able to set
attributes for both classes (i.e., define the font, spacing, expansion, color
index, height, angle, path, vertical alignment, and horizontal alignment).

This class is used to define and list a combined text table and text orientation
entry used in VCS.


\begin{fulllineitems}
\pysigline{\sphinxbfcode{Useful~Functions:}}~
\begin{Verbatim}[commandchars=\\\{\}]
\PYG{c+c1}{\PYGZsh{} Constructor}
 \PYG{n}{a}\PYG{o}{=}\PYG{n}{vcs}\PYG{o}{.}\PYG{n}{init}\PYG{p}{(}\PYG{p}{)}
 \PYG{c+c1}{\PYGZsh{} Show predefined text table objects}
 \PYG{n}{a}\PYG{o}{.}\PYG{n}{show}\PYG{p}{(}\PYG{l+s+s1}{\PYGZsq{}}\PYG{l+s+s1}{texttable}\PYG{l+s+s1}{\PYGZsq{}}\PYG{p}{)}
 \PYG{c+c1}{\PYGZsh{} Show predefined text orientation objects}
 \PYG{n}{a}\PYG{o}{.}\PYG{n}{show}\PYG{p}{(}\PYG{l+s+s1}{\PYGZsq{}}\PYG{l+s+s1}{textorientation}\PYG{l+s+s1}{\PYGZsq{}}\PYG{p}{)}
 \PYG{c+c1}{\PYGZsh{} Updates the VCS Canvas at user\PYGZsq{}s request}
 \PYG{n}{a}\PYG{o}{.}\PYG{n}{update}\PYG{p}{(}\PYG{p}{)}
\end{Verbatim}

\end{fulllineitems}



\begin{fulllineitems}
\pysigline{\sphinxbfcode{Make~a~Canvas~object~to~work~with:}}~
\begin{Verbatim}[commandchars=\\\{\}]
\PYG{n}{a}\PYG{o}{=}\PYG{n}{vcs}\PYG{o}{.}\PYG{n}{init}\PYG{p}{(}\PYG{p}{)}
\end{Verbatim}

\end{fulllineitems}



\begin{fulllineitems}
\pysigline{\sphinxbfcode{Create~a~new~instance~of~text~table:}}~
\begin{Verbatim}[commandchars=\\\{\}]
\PYG{c+c1}{\PYGZsh{} Copies content of \PYGZsq{}std\PYGZsq{} to \PYGZsq{}new\PYGZus{}tt\PYGZsq{} and \PYGZsq{}7left\PYGZsq{} to \PYGZsq{}new\PYGZus{}to\PYGZsq{}}
\PYG{n}{tc}\PYG{o}{=}\PYG{n}{a}\PYG{o}{.}\PYG{n}{createtextcombined}\PYG{p}{(}\PYG{l+s+s1}{\PYGZsq{}}\PYG{l+s+s1}{new\PYGZus{}tt}\PYG{l+s+s1}{\PYGZsq{}}\PYG{p}{,}\PYG{l+s+s1}{\PYGZsq{}}\PYG{l+s+s1}{std}\PYG{l+s+s1}{\PYGZsq{}}\PYG{p}{,}\PYG{l+s+s1}{\PYGZsq{}}\PYG{l+s+s1}{new\PYGZus{}to}\PYG{l+s+s1}{\PYGZsq{}}\PYG{p}{,}\PYG{l+s+s1}{\PYGZsq{}}\PYG{l+s+s1}{7left}\PYG{l+s+s1}{\PYGZsq{}}\PYG{p}{)}
\end{Verbatim}

\end{fulllineitems}



\begin{fulllineitems}
\pysigline{\sphinxbfcode{Modify~an~existing~texttable:}}~
\begin{Verbatim}[commandchars=\\\{\}]
\PYG{n}{tc}\PYG{o}{=}\PYG{n}{a}\PYG{o}{.}\PYG{n}{gettextcombined}\PYG{p}{(}\PYG{l+s+s1}{\PYGZsq{}}\PYG{l+s+s1}{std}\PYG{l+s+s1}{\PYGZsq{}}\PYG{p}{,}\PYG{l+s+s1}{\PYGZsq{}}\PYG{l+s+s1}{7left}\PYG{l+s+s1}{\PYGZsq{}}\PYG{p}{)}
\end{Verbatim}

\end{fulllineitems}



\begin{fulllineitems}
\pysigline{\sphinxbfcode{Overview~of~textcombined~attributes:}}~
\begin{notice}{note}{Note:}
Textcombined attributes are a combination of texttable and textorientation attributes
\end{notice}
\begin{itemize}
\item {} 
Listing the attributes:
\begin{quote}

\begin{Verbatim}[commandchars=\\\{\}]
\PYG{c+c1}{\PYGZsh{} Will list all the textcombined attribute values}
\PYG{n}{tc}\PYG{o}{.}\PYG{n}{list}\PYG{p}{(}\PYG{p}{)}
\end{Verbatim}
\end{quote}

\item {} 
Specify the text font type:
\begin{quote}

\begin{Verbatim}[commandchars=\\\{\}]
\PYG{c+c1}{\PYGZsh{} The font value must be in the range 1 to 9}
\PYG{n}{tc}\PYG{o}{.}\PYG{n}{font}\PYG{o}{=}\PYG{l+m+mi}{1}
\end{Verbatim}
\end{quote}

\item {} 
Specify the text spacing:
\begin{quote}

\begin{Verbatim}[commandchars=\\\{\}]
\PYG{c+c1}{\PYGZsh{} The spacing value must be in the range \PYGZhy{}50 to 50}
\PYG{n}{tc}\PYG{o}{.}\PYG{n}{spacing}\PYG{o}{=}\PYG{l+m+mi}{2}
\end{Verbatim}
\end{quote}

\item {} 
Specify the text expansion:
\begin{quote}

\begin{Verbatim}[commandchars=\\\{\}]
\PYG{c+c1}{\PYGZsh{} The expansion value ranges from 50 to 150}
\PYG{n}{tc}\PYG{o}{.}\PYG{n}{expansion}\PYG{o}{=}\PYG{l+m+mi}{100}
\end{Verbatim}
\end{quote}

\item {} 
Specify the text color:
\begin{quote}

\begin{Verbatim}[commandchars=\\\{\}]
\PYG{c+c1}{\PYGZsh{} The text color value ranges from 1 to 257}
\PYG{n}{tc}\PYG{o}{.}\PYG{n}{color}\PYG{o}{=}\PYG{l+m+mi}{241}
\end{Verbatim}
\end{quote}

\item {} 
Specify the graphics text priority on the VCS Canvas:
\begin{quote}

\begin{Verbatim}[commandchars=\\\{\}]
\PYG{n}{tt}\PYG{o}{.}\PYG{n}{priority} \PYG{o}{=} \PYG{l+m+mi}{1}
\end{Verbatim}
\end{quote}

\item {} 
Specify the viewport and world coordinate:
\begin{quote}

\begin{Verbatim}[commandchars=\\\{\}]

\end{Verbatim}

\# FloatType {[}0,1{]}x{[}0,1{]}
tt.viewport={[}0, 1.0, 0,1.0{]}
\# FloatType {[}\#,\#{]}x{[}\#,\#{]}
tt.worldcoordinate={[}0,1.0,0,1.0{]}
\end{quote}

\item {} 
Specify the location of the text:
\begin{quote}

\begin{Verbatim}[commandchars=\\\{\}]
\PYG{c+c1}{\PYGZsh{} List of FloatTypes}
\PYG{n}{tt}\PYG{o}{.}\PYG{n}{x}\PYG{o}{=}\PYG{p}{[}\PYG{p}{[}\PYG{l+m+mi}{0}\PYG{p}{,}\PYG{o}{.}\PYG{l+m+mi}{1}\PYG{p}{,}\PYG{o}{.}\PYG{l+m+mi}{2}\PYG{p}{]}\PYG{p}{,} \PYG{p}{[}\PYG{o}{.}\PYG{l+m+mi}{3}\PYG{p}{,}\PYG{o}{.}\PYG{l+m+mi}{4}\PYG{p}{,}\PYG{o}{.}\PYG{l+m+mi}{5}\PYG{p}{]}\PYG{p}{]}
\PYG{c+c1}{\PYGZsh{} List of FloatTypes}
\PYG{n}{tt}\PYG{o}{.}\PYG{n}{y}\PYG{o}{=}\PYG{p}{[}\PYG{p}{[}\PYG{o}{.}\PYG{l+m+mi}{5}\PYG{p}{,}\PYG{o}{.}\PYG{l+m+mi}{4}\PYG{p}{,}\PYG{o}{.}\PYG{l+m+mi}{3}\PYG{p}{]}\PYG{p}{,} \PYG{p}{[}\PYG{o}{.}\PYG{l+m+mi}{2}\PYG{p}{,}\PYG{o}{.}\PYG{l+m+mi}{1}\PYG{p}{,}\PYG{l+m+mi}{0}\PYG{p}{]}\PYG{p}{]}
\end{Verbatim}
\end{quote}

\item {} 
Specify the text height:
\begin{quote}

\begin{Verbatim}[commandchars=\\\{\}]
\PYG{c+c1}{\PYGZsh{} The height value must be an integer}
\PYG{n}{tc}\PYG{o}{.}\PYG{n}{height}\PYG{o}{=}\PYG{l+m+mi}{20}
\end{Verbatim}
\end{quote}

\item {} 
Specify the text angle:
\begin{quote}

\begin{Verbatim}[commandchars=\\\{\}]
\PYG{c+c1}{\PYGZsh{} The angle value ranges from 0 to 360}
\PYG{n}{tc}\PYG{o}{.}\PYG{n}{angle}\PYG{o}{=}\PYG{l+m+mi}{0}
\end{Verbatim}
\end{quote}

\item {} 
Specify the text path:
\begin{quote}

\begin{Verbatim}[commandchars=\\\{\}]
\PYG{c+c1}{\PYGZsh{} Same as tc.path=0}
\PYG{n}{tc}\PYG{o}{.}\PYG{n}{path}\PYG{o}{=}\PYG{l+s+s1}{\PYGZsq{}}\PYG{l+s+s1}{right}\PYG{l+s+s1}{\PYGZsq{}}
\PYG{c+c1}{\PYGZsh{} Same as tc.path=1}
\PYG{n}{tc}\PYG{o}{.}\PYG{n}{path}\PYG{o}{=}\PYG{l+s+s1}{\PYGZsq{}}\PYG{l+s+s1}{left}\PYG{l+s+s1}{\PYGZsq{}}
\PYG{c+c1}{\PYGZsh{} Same as tc.path=2}
\PYG{n}{tc}\PYG{o}{.}\PYG{n}{path}\PYG{o}{=}\PYG{l+s+s1}{\PYGZsq{}}\PYG{l+s+s1}{up}\PYG{l+s+s1}{\PYGZsq{}}
\PYG{c+c1}{\PYGZsh{} Same as tc.path=3}
\PYG{n}{tc}\PYG{o}{.}\PYG{n}{path}\PYG{o}{=}\PYG{l+s+s1}{\PYGZsq{}}\PYG{l+s+s1}{down}\PYG{l+s+s1}{\PYGZsq{}}
\end{Verbatim}
\end{quote}

\item {} 
Specify the text horizontal alignment:
\begin{quote}

\begin{Verbatim}[commandchars=\\\{\}]
\PYG{c+c1}{\PYGZsh{} Same as tc.halign=0}
\PYG{n}{tc}\PYG{o}{.}\PYG{n}{halign}\PYG{o}{=}\PYG{l+s+s1}{\PYGZsq{}}\PYG{l+s+s1}{right}\PYG{l+s+s1}{\PYGZsq{}}
\PYG{c+c1}{\PYGZsh{} Same as tc.halign=1}
\PYG{n}{tc}\PYG{o}{.}\PYG{n}{halign}\PYG{o}{=}\PYG{l+s+s1}{\PYGZsq{}}\PYG{l+s+s1}{center}\PYG{l+s+s1}{\PYGZsq{}}
\PYG{c+c1}{\PYGZsh{} Same as tc.halign=2}
\PYG{n}{tc}\PYG{o}{.}\PYG{n}{halign}\PYG{o}{=}\PYG{l+s+s1}{\PYGZsq{}}\PYG{l+s+s1}{right}\PYG{l+s+s1}{\PYGZsq{}}
\end{Verbatim}
\end{quote}

\item {} 
Specify the text vertical alignment:
\begin{quote}

\begin{Verbatim}[commandchars=\\\{\}]
\PYG{c+c1}{\PYGZsh{} Same as tcvalign=0}
\PYG{n}{tc}\PYG{o}{.}\PYG{n}{valign}\PYG{o}{=}\PYG{l+s+s1}{\PYGZsq{}}\PYG{l+s+s1}{tcp}\PYG{l+s+s1}{\PYGZsq{}}
\PYG{c+c1}{\PYGZsh{} Same as tcvalign=1}
\PYG{n}{tc}\PYG{o}{.}\PYG{n}{valign}\PYG{o}{=}\PYG{l+s+s1}{\PYGZsq{}}\PYG{l+s+s1}{cap}\PYG{l+s+s1}{\PYGZsq{}}
\PYG{c+c1}{\PYGZsh{} Same as tcvalign=2}
\PYG{n}{tc}\PYG{o}{.}\PYG{n}{valign}\PYG{o}{=}\PYG{l+s+s1}{\PYGZsq{}}\PYG{l+s+s1}{half}\PYG{l+s+s1}{\PYGZsq{}}
\PYG{c+c1}{\PYGZsh{} Same as tcvalign=3}
\PYG{n}{tc}\PYG{o}{.}\PYG{n}{valign}\PYG{o}{=}\PYG{l+s+s1}{\PYGZsq{}}\PYG{l+s+s1}{base}\PYG{l+s+s1}{\PYGZsq{}}
\PYG{c+c1}{\PYGZsh{} Same as tcvalign=4}
\PYG{n}{tc}\PYG{o}{.}\PYG{n}{valign}\PYG{o}{=}\PYG{l+s+s1}{\PYGZsq{}}\PYG{l+s+s1}{bottom}\PYG{l+s+s1}{\PYGZsq{}}
\end{Verbatim}
\end{quote}

\end{itemize}

\end{fulllineitems}

\index{script() (vcs.textcombined.Tc method)}

\begin{fulllineitems}
\phantomsection\label{vcs/secondary/textcombined:vcs.textcombined.Tc.script}\pysiglinewithargsret{\sphinxbfcode{script}}{\emph{script\_filename=None}, \emph{mode=None}}{}
Saves out a copy of the text table and text orientation secondary method in JSON, or Python format to a designated file.
\begin{quote}

\begin{notice}{note}{Note:}
If the the filename has a `.py' at the end, it will produce a
Python script. If no extension is given, then by default a
.json file containing a JSON serialization of the object's
data will be produced.
\end{notice}

\begin{notice}{warning}{Warning:}
VCS Scripts Deprecated.
SCR script files are no longer generated by this function.
\end{notice}
\end{quote}
\begin{quote}\begin{description}
\item[{Example}] \leavevmode
\begin{Verbatim}[commandchars=\\\{\}]
\PYG{g+gp}{\PYGZgt{}\PYGZgt{}\PYGZgt{} }\PYG{n}{a}\PYG{o}{=}\PYG{n}{vcs}\PYG{o}{.}\PYG{n}{init}\PYG{p}{(}\PYG{p}{)} \PYG{c+c1}{\PYGZsh{} Make a Canvas object to work with}
\PYG{g+gp}{\PYGZgt{}\PYGZgt{}\PYGZgt{} }\PYG{n}{a}\PYG{o}{.}\PYG{n}{createtextcombined}\PYG{p}{(}\PYG{l+s+s1}{\PYGZsq{}}\PYG{l+s+s1}{EXAMPLE\PYGZus{}tt}\PYG{l+s+s1}{\PYGZsq{}}\PYG{p}{,} \PYG{l+s+s1}{\PYGZsq{}}\PYG{l+s+s1}{qa}\PYG{l+s+s1}{\PYGZsq{}}\PYG{p}{,} \PYG{l+s+s1}{\PYGZsq{}}\PYG{l+s+s1}{EXAMPLE\PYGZus{}tto}\PYG{l+s+s1}{\PYGZsq{}}\PYG{p}{,} \PYG{l+s+s1}{\PYGZsq{}}\PYG{l+s+s1}{7left}\PYG{l+s+s1}{\PYGZsq{}}\PYG{p}{)} \PYG{c+c1}{\PYGZsh{} Create \PYGZsq{}EXAMPLE\PYGZus{}tt\PYGZsq{} and \PYGZsq{}EXAMPLE\PYGZus{}tto\PYGZsq{}}
\PYG{g+go}{\PYGZlt{}vcs.textcombined.Tc ...\PYGZgt{}}
\PYG{g+gp}{\PYGZgt{}\PYGZgt{}\PYGZgt{} }\PYG{n}{ex}\PYG{o}{=}\PYG{n}{a}\PYG{o}{.}\PYG{n}{gettextcombined}\PYG{p}{(}\PYG{l+s+s1}{\PYGZsq{}}\PYG{l+s+s1}{EXAMPLE\PYGZus{}tt}\PYG{l+s+s1}{\PYGZsq{}}\PYG{p}{,} \PYG{l+s+s1}{\PYGZsq{}}\PYG{l+s+s1}{EXAMPLE\PYGZus{}tto}\PYG{l+s+s1}{\PYGZsq{}}\PYG{p}{)} \PYG{c+c1}{\PYGZsh{} Get default textcombined}
\PYG{g+gp}{\PYGZgt{}\PYGZgt{}\PYGZgt{} }\PYG{n}{ex}\PYG{o}{.}\PYG{n}{script}\PYG{p}{(}\PYG{l+s+s1}{\PYGZsq{}}\PYG{l+s+s1}{filename.py}\PYG{l+s+s1}{\PYGZsq{}}\PYG{p}{)} \PYG{c+c1}{\PYGZsh{} Append to a Python script named \PYGZsq{}filename.py\PYGZsq{}}
\PYG{g+gp}{\PYGZgt{}\PYGZgt{}\PYGZgt{} }\PYG{n}{ex}\PYG{o}{.}\PYG{n}{script}\PYG{p}{(}\PYG{l+s+s1}{\PYGZsq{}}\PYG{l+s+s1}{filename}\PYG{l+s+s1}{\PYGZsq{}}\PYG{p}{,}\PYG{l+s+s1}{\PYGZsq{}}\PYG{l+s+s1}{w}\PYG{l+s+s1}{\PYGZsq{}}\PYG{p}{)} \PYG{c+c1}{\PYGZsh{} Create or overwrite a JSON file \PYGZsq{}filename.json\PYGZsq{}.}
\end{Verbatim}

\item[{Parameters}] \leavevmode\begin{itemize}
\item {} 
\textbf{\texttt{script\_filename}} (\href{https://docs.python.org/2/library/functions.html\#str}{\emph{\texttt{str}}}) -- Output name of the script file. If no extension is specified, a .json object is created.

\item {} 
\textbf{\texttt{mode}} (\href{https://docs.python.org/2/library/functions.html\#str}{\emph{\texttt{str}}}) -- Either `w' for replace, or `a' for append. Defaults to `a', if not specified.

\end{itemize}

\end{description}\end{quote}

\end{fulllineitems}


\end{fulllineitems}



\subsection{textorientation}
\label{vcs/secondary/textorientation:module-vcs.textorientation}\label{vcs/secondary/textorientation::doc}\label{vcs/secondary/textorientation:textorientation}\index{vcs.textorientation (module)}
\# Text Orientation (To) module
\index{To (class in vcs.textorientation)}

\begin{fulllineitems}
\phantomsection\label{vcs/secondary/textorientation:vcs.textorientation.To}\pysiglinewithargsret{\sphinxstrong{class }\sphinxcode{vcs.textorientation.}\sphinxbfcode{To}}{\emph{To\_name}, \emph{To\_name\_src='default'}}{}
The (To) Text Orientation lists text attribute set names that define the font, spacing,
expansion, and color index.

This class is used to define an text orientation table entry used in VCS, or it
can be used to change some or all of the text orientation attributes in an
existing text orientation table entry.


\begin{fulllineitems}
\pysigline{\sphinxbfcode{Useful~Functions:}}~
\begin{Verbatim}[commandchars=\\\{\}]
\PYG{c+c1}{\PYGZsh{} VCS Canvas Constructor}
\PYG{n}{a}\PYG{o}{=}\PYG{n}{vcs}\PYG{o}{.}\PYG{n}{init}\PYG{p}{(}\PYG{p}{)}
\PYG{c+c1}{\PYGZsh{} Show predefined text orientation objects}
\PYG{n}{a}\PYG{o}{.}\PYG{n}{show}\PYG{p}{(}\PYG{l+s+s1}{\PYGZsq{}}\PYG{l+s+s1}{textorientation}\PYG{l+s+s1}{\PYGZsq{}}\PYG{p}{)}
\PYG{c+c1}{\PYGZsh{} Updates the VCS Canvas at user\PYGZsq{}s request}
\PYG{n}{a}\PYG{o}{.}\PYG{n}{update}\PYG{p}{(}\PYG{p}{)}
\end{Verbatim}

\end{fulllineitems}



\begin{fulllineitems}
\pysigline{\sphinxbfcode{Make~a~canvas~object~to~work~with:}}~
\begin{Verbatim}[commandchars=\\\{\}]
\PYG{n}{a}\PYG{o}{=}\PYG{n}{vcs}\PYG{o}{.}\PYG{n}{init}\PYG{p}{(}\PYG{p}{)}
\end{Verbatim}

\end{fulllineitems}



\begin{fulllineitems}
\pysigline{\sphinxbfcode{Create~a~new~instance~of~text~orientation:}}~
\begin{Verbatim}[commandchars=\\\{\}]
\PYG{c+c1}{\PYGZsh{} Copies content of \PYGZsq{}7left\PYGZsq{} to \PYGZsq{}new\PYGZsq{}}
\PYG{n}{to}\PYG{o}{=}\PYG{n}{a}\PYG{o}{.}\PYG{n}{createtextorientation}\PYG{p}{(}\PYG{l+s+s1}{\PYGZsq{}}\PYG{l+s+s1}{new}\PYG{l+s+s1}{\PYGZsq{}}\PYG{p}{,}\PYG{l+s+s1}{\PYGZsq{}}\PYG{l+s+s1}{7left}\PYG{l+s+s1}{\PYGZsq{}}\PYG{p}{)}
\PYG{c+c1}{\PYGZsh{} Copies content of \PYGZsq{}default\PYGZsq{} to \PYGZsq{}new\PYGZsq{}}
\PYG{n}{to}\PYG{o}{=}\PYG{n}{a}\PYG{o}{.}\PYG{n}{createtextorientation}\PYG{p}{(}\PYG{l+s+s1}{\PYGZsq{}}\PYG{l+s+s1}{new}\PYG{l+s+s1}{\PYGZsq{}}\PYG{p}{)}
\end{Verbatim}

\end{fulllineitems}



\begin{fulllineitems}
\pysigline{\sphinxbfcode{Modify~an~existing~textorientation:}}~
\begin{Verbatim}[commandchars=\\\{\}]
\PYG{n}{to}\PYG{o}{=}\PYG{n}{a}\PYG{o}{.}\PYG{n}{gettextorientation}\PYG{p}{(}\PYG{l+s+s1}{\PYGZsq{}}\PYG{l+s+s1}{7left}\PYG{l+s+s1}{\PYGZsq{}}\PYG{p}{)}
\end{Verbatim}

\end{fulllineitems}



\begin{fulllineitems}
\pysigline{\sphinxbfcode{Overview~of~textorientation~attributes:}}~\begin{itemize}
\item {} 
Listing the attributes:
\begin{quote}

\begin{Verbatim}[commandchars=\\\{\}]
\PYG{c+c1}{\PYGZsh{} Will list all the textorientation attribute values}
\PYG{n}{to}\PYG{o}{.}\PYG{n}{list}\PYG{p}{(}\PYG{p}{)}
\end{Verbatim}
\end{quote}

\item {} 
Specify the text height:
\begin{quote}

\begin{Verbatim}[commandchars=\\\{\}]
\PYG{c+c1}{\PYGZsh{} The height value must be an integer}
\PYG{n}{to}\PYG{o}{.}\PYG{n}{height}\PYG{o}{=}\PYG{l+m+mi}{20}
\end{Verbatim}
\end{quote}

\item {} 
Specify the text angle:
\begin{quote}

\begin{Verbatim}[commandchars=\\\{\}]
\PYG{c+c1}{\PYGZsh{} The angle value must be in the range 0 to 360}
\PYG{n}{to}\PYG{o}{.}\PYG{n}{angle}\PYG{o}{=}\PYG{l+m+mi}{0}
\end{Verbatim}
\end{quote}

\item {} 
Specify the text path:
\begin{quote}

\begin{Verbatim}[commandchars=\\\{\}]
\PYG{c+c1}{\PYGZsh{} Same as to.path=0}
\PYG{n}{to}\PYG{o}{.}\PYG{n}{path}\PYG{o}{=}\PYG{l+s+s1}{\PYGZsq{}}\PYG{l+s+s1}{right}\PYG{l+s+s1}{\PYGZsq{}}
\PYG{c+c1}{\PYGZsh{} Same as to.path=1}
\PYG{n}{to}\PYG{o}{.}\PYG{n}{path}\PYG{o}{=}\PYG{l+s+s1}{\PYGZsq{}}\PYG{l+s+s1}{left}\PYG{l+s+s1}{\PYGZsq{}}
\PYG{c+c1}{\PYGZsh{} Same as to.path=2}
\PYG{n}{to}\PYG{o}{.}\PYG{n}{path}\PYG{o}{=}\PYG{l+s+s1}{\PYGZsq{}}\PYG{l+s+s1}{up}\PYG{l+s+s1}{\PYGZsq{}}
\PYG{c+c1}{\PYGZsh{} Same as to.path=3}
\PYG{n}{to}\PYG{o}{.}\PYG{n}{path}\PYG{o}{=}\PYG{l+s+s1}{\PYGZsq{}}\PYG{l+s+s1}{down}\PYG{l+s+s1}{\PYGZsq{}}
\end{Verbatim}
\end{quote}

\item {} 
Specify the text horizontal alignment:
\begin{quote}

\begin{Verbatim}[commandchars=\\\{\}]
\PYG{c+c1}{\PYGZsh{} Same as to.halign=0}
\PYG{n}{to}\PYG{o}{.}\PYG{n}{halign}\PYG{o}{=}\PYG{l+s+s1}{\PYGZsq{}}\PYG{l+s+s1}{right}\PYG{l+s+s1}{\PYGZsq{}}
\PYG{c+c1}{\PYGZsh{} Same as to.halign=1}
\PYG{n}{to}\PYG{o}{.}\PYG{n}{halign}\PYG{o}{=}\PYG{l+s+s1}{\PYGZsq{}}\PYG{l+s+s1}{center}\PYG{l+s+s1}{\PYGZsq{}}
\PYG{c+c1}{\PYGZsh{} Same as to.halign=2}
\PYG{n}{to}\PYG{o}{.}\PYG{n}{halign}\PYG{o}{=}\PYG{l+s+s1}{\PYGZsq{}}\PYG{l+s+s1}{right}\PYG{l+s+s1}{\PYGZsq{}}
\end{Verbatim}
\end{quote}

\item {} 
Specify the text vertical alignment:
\begin{quote}

\begin{Verbatim}[commandchars=\\\{\}]
\PYG{c+c1}{\PYGZsh{} Same as tovalign=0}
\PYG{n}{to}\PYG{o}{.}\PYG{n}{valign}\PYG{o}{=}\PYG{l+s+s1}{\PYGZsq{}}\PYG{l+s+s1}{top}\PYG{l+s+s1}{\PYGZsq{}}
\PYG{c+c1}{\PYGZsh{} Same as tovalign=1}
\PYG{n}{to}\PYG{o}{.}\PYG{n}{valign}\PYG{o}{=}\PYG{l+s+s1}{\PYGZsq{}}\PYG{l+s+s1}{cap}\PYG{l+s+s1}{\PYGZsq{}}
\PYG{c+c1}{\PYGZsh{} Same as tovalign=2}
\PYG{n}{to}\PYG{o}{.}\PYG{n}{valign}\PYG{o}{=}\PYG{l+s+s1}{\PYGZsq{}}\PYG{l+s+s1}{half}\PYG{l+s+s1}{\PYGZsq{}}
\PYG{c+c1}{\PYGZsh{} Same as tovalign=3}
\PYG{n}{to}\PYG{o}{.}\PYG{n}{valign}\PYG{o}{=}\PYG{l+s+s1}{\PYGZsq{}}\PYG{l+s+s1}{base}\PYG{l+s+s1}{\PYGZsq{}}
\PYG{c+c1}{\PYGZsh{} Same as tovalign=4}
\PYG{n}{to}\PYG{o}{.}\PYG{n}{valign}\PYG{o}{=}\PYG{l+s+s1}{\PYGZsq{}}\PYG{l+s+s1}{bottom}\PYG{l+s+s1}{\PYGZsq{}}
\end{Verbatim}
\end{quote}

\end{itemize}

\end{fulllineitems}

\index{script() (vcs.textorientation.To method)}

\begin{fulllineitems}
\phantomsection\label{vcs/secondary/textorientation:vcs.textorientation.To.script}\pysiglinewithargsret{\sphinxbfcode{script}}{\emph{script\_filename=None}, \emph{mode=None}}{}
Saves out a copy of the textorientation secondary method in JSON, or Python format to a designated file.
\begin{quote}

\begin{notice}{note}{Note:}
If the the filename has a `.py' at the end, it will produce a
Python script. If no extension is given, then by default a
.json file containing a JSON serialization of the object's
data will be produced.
\end{notice}

\begin{notice}{warning}{Warning:}
VCS Scripts Deprecated.
SCR script files are no longer generated by this function.
\end{notice}
\end{quote}
\begin{quote}\begin{description}
\item[{Example}] \leavevmode
\begin{Verbatim}[commandchars=\\\{\}]
\PYG{g+gp}{\PYGZgt{}\PYGZgt{}\PYGZgt{} }\PYG{n}{a}\PYG{o}{=}\PYG{n}{vcs}\PYG{o}{.}\PYG{n}{init}\PYG{p}{(}\PYG{p}{)} \PYG{c+c1}{\PYGZsh{} Make a Canvas object to work with}
\PYG{g+gp}{\PYGZgt{}\PYGZgt{}\PYGZgt{} }\PYG{n}{ex}\PYG{o}{=}\PYG{n}{a}\PYG{o}{.}\PYG{n}{gettextorientation}\PYG{p}{(}\PYG{p}{)} \PYG{c+c1}{\PYGZsh{} Get default textorientation}
\PYG{g+gp}{\PYGZgt{}\PYGZgt{}\PYGZgt{} }\PYG{n}{ex}\PYG{o}{.}\PYG{n}{script}\PYG{p}{(}\PYG{l+s+s1}{\PYGZsq{}}\PYG{l+s+s1}{filename.py}\PYG{l+s+s1}{\PYGZsq{}}\PYG{p}{)} \PYG{c+c1}{\PYGZsh{} Append to a Python script named \PYGZsq{}filename.py\PYGZsq{}}
\PYG{g+gp}{\PYGZgt{}\PYGZgt{}\PYGZgt{} }\PYG{n}{ex}\PYG{o}{.}\PYG{n}{script}\PYG{p}{(}\PYG{l+s+s1}{\PYGZsq{}}\PYG{l+s+s1}{filename}\PYG{l+s+s1}{\PYGZsq{}}\PYG{p}{,}\PYG{l+s+s1}{\PYGZsq{}}\PYG{l+s+s1}{w}\PYG{l+s+s1}{\PYGZsq{}}\PYG{p}{)} \PYG{c+c1}{\PYGZsh{} Create or overwrite a JSON file \PYGZsq{}filename.json\PYGZsq{}.}
\end{Verbatim}

\item[{Parameters}] \leavevmode\begin{itemize}
\item {} 
\textbf{\texttt{script\_filename}} (\href{https://docs.python.org/2/library/functions.html\#str}{\emph{\texttt{str}}}) -- Output name of the script file. If no extension is specified, a .json object is created.

\item {} 
\textbf{\texttt{mode}} (\href{https://docs.python.org/2/library/functions.html\#str}{\emph{\texttt{str}}}) -- Either `w' for replace, or `a' for append. Defaults to `a', if not specified.

\end{itemize}

\end{description}\end{quote}

\end{fulllineitems}


\end{fulllineitems}



\subsection{texttable}
\label{vcs/secondary/texttable:module-vcs.texttable}\label{vcs/secondary/texttable::doc}\label{vcs/secondary/texttable:texttable}\index{vcs.texttable (module)}
\# Text Table (Tt) module
\index{Tt (class in vcs.texttable)}

\begin{fulllineitems}
\phantomsection\label{vcs/secondary/texttable:vcs.texttable.Tt}\pysiglinewithargsret{\sphinxstrong{class }\sphinxcode{vcs.texttable.}\sphinxbfcode{Tt}}{\emph{Tt\_name=None}, \emph{Tt\_name\_src='default'}}{}
The (Tt) Text Table lists text attribute set names that define the font, spacing,
expansion, and color index.

This class is used to define an text table table entry used in VCS, or it
can be used to change some or all of the text table attributes in an
existing text table table entry.


\begin{fulllineitems}
\pysigline{\sphinxbfcode{Useful~Functions:}}~
\begin{Verbatim}[commandchars=\\\{\}]
\PYG{c+c1}{\PYGZsh{} VCS Canvas Constructor}
\PYG{n}{a}\PYG{o}{=}\PYG{n}{vcs}\PYG{o}{.}\PYG{n}{init}\PYG{p}{(}\PYG{p}{)}
\PYG{c+c1}{\PYGZsh{} Show predefined text table objects}
\PYG{n}{a}\PYG{o}{.}\PYG{n}{show}\PYG{p}{(}\PYG{l+s+s1}{\PYGZsq{}}\PYG{l+s+s1}{texttable}\PYG{l+s+s1}{\PYGZsq{}}\PYG{p}{)}
\PYG{c+c1}{\PYGZsh{} Updates the VCS Canvas at user\PYGZsq{}s request}
\PYG{n}{a}\PYG{o}{.}\PYG{n}{update}\PYG{p}{(}\PYG{p}{)}
\end{Verbatim}

\end{fulllineitems}



\begin{fulllineitems}
\pysigline{\sphinxbfcode{Make~a~Canvas~object~to~work~with:}}~
\begin{Verbatim}[commandchars=\\\{\}]
\PYG{n}{a}\PYG{o}{=}\PYG{n}{vcs}\PYG{o}{.}\PYG{n}{init}\PYG{p}{(}\PYG{p}{)}
\end{Verbatim}

\end{fulllineitems}



\begin{fulllineitems}
\pysigline{\sphinxbfcode{Create~a~new~instance~of~text~table:}}~
\begin{Verbatim}[commandchars=\\\{\}]
\PYG{c+c1}{\PYGZsh{} Copies content of \PYGZsq{}std\PYGZsq{} to \PYGZsq{}new\PYGZsq{}}
\PYG{n}{tt}\PYG{o}{=}\PYG{n}{a}\PYG{o}{.}\PYG{n}{createtexttable}\PYG{p}{(}\PYG{l+s+s1}{\PYGZsq{}}\PYG{l+s+s1}{new}\PYG{l+s+s1}{\PYGZsq{}}\PYG{p}{,}\PYG{l+s+s1}{\PYGZsq{}}\PYG{l+s+s1}{std}\PYG{l+s+s1}{\PYGZsq{}}\PYG{p}{)}
\PYG{c+c1}{\PYGZsh{} Copies content of \PYGZsq{}default\PYGZsq{} to \PYGZsq{}new\PYGZsq{}}
\PYG{n}{tt}\PYG{o}{=}\PYG{n}{a}\PYG{o}{.}\PYG{n}{createtexttable}\PYG{p}{(}\PYG{l+s+s1}{\PYGZsq{}}\PYG{l+s+s1}{new}\PYG{l+s+s1}{\PYGZsq{}}\PYG{p}{)}
\end{Verbatim}

\end{fulllineitems}



\begin{fulllineitems}
\pysigline{\sphinxbfcode{Modify~an~existing~texttable:}}~
\begin{Verbatim}[commandchars=\\\{\}]
\PYG{n}{tt}\PYG{o}{=}\PYG{n}{a}\PYG{o}{.}\PYG{n}{gettexttable}\PYG{p}{(}\PYG{l+s+s1}{\PYGZsq{}}\PYG{l+s+s1}{std}\PYG{l+s+s1}{\PYGZsq{}}\PYG{p}{)}
\end{Verbatim}

\end{fulllineitems}



\begin{fulllineitems}
\pysigline{\sphinxbfcode{Overview~of~texttable~attributes:}}~\begin{itemize}
\item {} 
Listing attributes:
\begin{quote}

\begin{Verbatim}[commandchars=\\\{\}]
\PYG{c+c1}{\PYGZsh{} Will list all the texttable attribute values}
\PYG{n}{tt}\PYG{o}{.}\PYG{n}{list}\PYG{p}{(}\PYG{p}{)}
\end{Verbatim}
\end{quote}

\item {} 
Specify the text font type:
\begin{quote}

\begin{Verbatim}[commandchars=\\\{\}]
\PYG{c+c1}{\PYGZsh{} The font value must be in the range 1 to 9}
\PYG{n}{tt}\PYG{o}{.}\PYG{n}{font}\PYG{o}{=}\PYG{l+m+mi}{1}
\end{Verbatim}
\end{quote}

\item {} 
Specify the text spacing:
\begin{quote}

\begin{Verbatim}[commandchars=\\\{\}]
\PYG{c+c1}{\PYGZsh{} The spacing value must be in the range \PYGZhy{}50 to 50}
\PYG{n}{tt}\PYG{o}{.}\PYG{n}{spacing}\PYG{o}{=}\PYG{l+m+mi}{2}
\end{Verbatim}
\end{quote}

\item {} 
Specify the text expansion:
\begin{quote}

\begin{Verbatim}[commandchars=\\\{\}]
\PYG{c+c1}{\PYGZsh{} The expansion value must be in the range 50 to 150}
\PYG{n}{tt}\PYG{o}{.}\PYG{n}{expansion}\PYG{o}{=}\PYG{l+m+mi}{100}
\end{Verbatim}
\end{quote}

\item {} 
Specify the text color:
\begin{quote}

\begin{Verbatim}[commandchars=\\\{\}]

\end{Verbatim}

\# The text color attribute value must be in the range 1 to 257
tt.color=241
\end{quote}

\item {} 
Specify the text background color and opacity:
\begin{quote}

\begin{Verbatim}[commandchars=\\\{\}]
\PYG{c+c1}{\PYGZsh{} The text backgroundcolor attribute value must be in the range 1 to 257}
\PYG{n}{tt}\PYG{o}{.}\PYG{n}{backgroundcolor}\PYG{o}{=}\PYG{l+m+mi}{241}
\PYG{c+c1}{\PYGZsh{} The text backgroundopacity attribute value must be in the range 0 to 100}
\PYG{n}{tt}\PYG{o}{.}\PYG{n}{backgroundopacity}\PYG{o}{=}\PYG{l+m+mi}{0}
\PYG{c+c1}{\PYGZsh{} Set the graphics priority on the canvas}
\PYG{n}{tt}\PYG{o}{.}\PYG{n}{priority}\PYG{o}{=}\PYG{l+m+mi}{1}
\PYG{c+c1}{\PYGZsh{} FloatType [0,1]x[0,1]}
\PYG{n}{tt}\PYG{o}{.}\PYG{n}{viewport}\PYG{o}{=}\PYG{p}{[}\PYG{l+m+mi}{0}\PYG{p}{,} \PYG{l+m+mf}{1.0}\PYG{p}{,} \PYG{l+m+mi}{0}\PYG{p}{,}\PYG{l+m+mf}{1.0}\PYG{p}{]}
\PYG{c+c1}{\PYGZsh{} FloatType [\PYGZsh{},\PYGZsh{}]x[\PYGZsh{},\PYGZsh{}]}
\PYG{n}{tt}\PYG{o}{.}\PYG{n}{worldcoordinate}\PYG{o}{=}\PYG{p}{[}\PYG{l+m+mi}{0}\PYG{p}{,}\PYG{l+m+mf}{1.0}\PYG{p}{,}\PYG{l+m+mi}{0}\PYG{p}{,}\PYG{l+m+mf}{1.0}\PYG{p}{]}
\PYG{c+c1}{\PYGZsh{} List of FloatTypes}
\PYG{n}{tt}\PYG{o}{.}\PYG{n}{x}\PYG{o}{=}\PYG{p}{[}\PYG{p}{[}\PYG{l+m+mi}{0}\PYG{p}{,}\PYG{o}{.}\PYG{l+m+mi}{1}\PYG{p}{,}\PYG{o}{.}\PYG{l+m+mi}{2}\PYG{p}{]}\PYG{p}{,} \PYG{p}{[}\PYG{o}{.}\PYG{l+m+mi}{3}\PYG{p}{,}\PYG{o}{.}\PYG{l+m+mi}{4}\PYG{p}{,}\PYG{o}{.}\PYG{l+m+mi}{5}\PYG{p}{]}\PYG{p}{]}
\PYG{c+c1}{\PYGZsh{} List of FloatTypes}
\PYG{n}{tt}\PYG{o}{.}\PYG{n}{y}\PYG{o}{=}\PYG{p}{[}\PYG{p}{[}\PYG{o}{.}\PYG{l+m+mi}{5}\PYG{p}{,}\PYG{o}{.}\PYG{l+m+mi}{4}\PYG{p}{,}\PYG{o}{.}\PYG{l+m+mi}{3}\PYG{p}{]}\PYG{p}{,} \PYG{p}{[}\PYG{o}{.}\PYG{l+m+mi}{2}\PYG{p}{,}\PYG{o}{.}\PYG{l+m+mi}{1}\PYG{p}{,}\PYG{l+m+mi}{0}\PYG{p}{]}\PYG{p}{]}
\end{Verbatim}
\end{quote}

\end{itemize}

\end{fulllineitems}

\index{script() (vcs.texttable.Tt method)}

\begin{fulllineitems}
\phantomsection\label{vcs/secondary/texttable:vcs.texttable.Tt.script}\pysiglinewithargsret{\sphinxbfcode{script}}{\emph{script\_filename=None}, \emph{mode=None}}{}
Saves out a copy of the texttable secondary method in JSON, or Python format to a designated file.
\begin{quote}

\begin{notice}{note}{Note:}
If the the filename has a `.py' at the end, it will produce a
Python script. If no extension is given, then by default a
.json file containing a JSON serialization of the object's
data will be produced.
\end{notice}

\begin{notice}{warning}{Warning:}
VCS Scripts Deprecated.
SCR script files are no longer generated by this function.
\end{notice}
\end{quote}
\begin{quote}\begin{description}
\item[{Example}] \leavevmode
\begin{Verbatim}[commandchars=\\\{\}]
\PYG{g+gp}{\PYGZgt{}\PYGZgt{}\PYGZgt{} }\PYG{n}{a}\PYG{o}{=}\PYG{n}{vcs}\PYG{o}{.}\PYG{n}{init}\PYG{p}{(}\PYG{p}{)} \PYG{c+c1}{\PYGZsh{} Make a Canvas object to work with}
\PYG{g+gp}{\PYGZgt{}\PYGZgt{}\PYGZgt{} }\PYG{n}{ex}\PYG{o}{=}\PYG{n}{a}\PYG{o}{.}\PYG{n}{gettexttable}\PYG{p}{(}\PYG{p}{)} \PYG{c+c1}{\PYGZsh{} Get default texttable}
\PYG{g+gp}{\PYGZgt{}\PYGZgt{}\PYGZgt{} }\PYG{n}{ex}\PYG{o}{.}\PYG{n}{script}\PYG{p}{(}\PYG{l+s+s1}{\PYGZsq{}}\PYG{l+s+s1}{filename.py}\PYG{l+s+s1}{\PYGZsq{}}\PYG{p}{)} \PYG{c+c1}{\PYGZsh{} Append to a Python script named \PYGZsq{}filename.py\PYGZsq{}}
\PYG{g+gp}{\PYGZgt{}\PYGZgt{}\PYGZgt{} }\PYG{n}{ex}\PYG{o}{.}\PYG{n}{script}\PYG{p}{(}\PYG{l+s+s1}{\PYGZsq{}}\PYG{l+s+s1}{filename}\PYG{l+s+s1}{\PYGZsq{}}\PYG{p}{,}\PYG{l+s+s1}{\PYGZsq{}}\PYG{l+s+s1}{w}\PYG{l+s+s1}{\PYGZsq{}}\PYG{p}{)} \PYG{c+c1}{\PYGZsh{} Create or overwrite a JSON file \PYGZsq{}filename.json\PYGZsq{}.}
\end{Verbatim}

\item[{Parameters}] \leavevmode\begin{itemize}
\item {} 
\textbf{\texttt{script\_filename}} (\href{https://docs.python.org/2/library/functions.html\#str}{\emph{\texttt{str}}}) -- Output name of the script file. If no extension is specified, a .json object is created.

\item {} 
\textbf{\texttt{mode}} (\href{https://docs.python.org/2/library/functions.html\#str}{\emph{\texttt{str}}}) -- Either `w' for replace, or `a' for append. Defaults to `a', if not specified.

\end{itemize}

\end{description}\end{quote}

\end{fulllineitems}


\end{fulllineitems}



\section{Miscellaneous Modules}
\label{vcs/misc/misc::doc}\label{vcs/misc/misc:miscellaneous-modules}
These are a variety of modules from VCS that help out with useful functionality.


\subsection{animate\_helper}
\label{vcs/misc/animate_helper:animate-helper}\label{vcs/misc/animate_helper::doc}\label{vcs/misc/animate_helper:module-vcs.animate_helper}\index{vcs.animate\_helper (module)}\index{animate\_obj\_old (class in vcs.animate\_helper)}

\begin{fulllineitems}
\phantomsection\label{vcs/misc/animate_helper:vcs.animate_helper.animate_obj_old}\pysiglinewithargsret{\sphinxstrong{class }\sphinxcode{vcs.animate\_helper.}\sphinxbfcode{animate\_obj\_old}}{\emph{vcs\_self}}{}
Animate the contents of the VCS Canvas. The animation can also be controlled from
the animation GUI. (See VCDAT for more details.)

See the {\color{red}\bfseries{}{}`animation GUI documenation{}`\_}
.. \_animation GUI documenation: \url{http://www-pcmdi.llnl.gov/software/vcs}
\begin{quote}\begin{description}
\item[{Example}] \leavevmode
\begin{Verbatim}[commandchars=\\\{\}]
\PYG{g+gp}{\PYGZgt{}\PYGZgt{}\PYGZgt{} }\PYG{n}{a}\PYG{o}{=}\PYG{n}{vcs}\PYG{o}{.}\PYG{n}{init}\PYG{p}{(}\PYG{p}{)}
\PYG{g+gp}{\PYGZgt{}\PYGZgt{}\PYGZgt{} }\PYG{n}{a}\PYG{o}{.}\PYG{n}{plot}\PYG{p}{(}\PYG{n}{array}\PYG{p}{,}\PYG{l+s+s1}{\PYGZsq{}}\PYG{l+s+s1}{default}\PYG{l+s+s1}{\PYGZsq{}}\PYG{p}{,}\PYG{l+s+s1}{\PYGZsq{}}\PYG{l+s+s1}{isofill}\PYG{l+s+s1}{\PYGZsq{}}\PYG{p}{,}\PYG{l+s+s1}{\PYGZsq{}}\PYG{l+s+s1}{quick}\PYG{l+s+s1}{\PYGZsq{}}\PYG{p}{)}
\PYG{g+gp}{\PYGZgt{}\PYGZgt{}\PYGZgt{} }\PYG{n}{a}\PYG{o}{.}\PYG{n}{animate}\PYG{p}{(}\PYG{p}{)}
\end{Verbatim}

\end{description}\end{quote}

\end{fulllineitems}



\subsection{projection}
\label{vcs/misc/projection::doc}\label{vcs/misc/projection:module-vcs.projection}\label{vcs/misc/projection:projection}\index{vcs.projection (module)}
\# Projection (Proj) module
\index{Proj (class in vcs.projection)}

\begin{fulllineitems}
\phantomsection\label{vcs/misc/projection:vcs.projection.Proj}\pysiglinewithargsret{\sphinxstrong{class }\sphinxcode{vcs.projection.}\sphinxbfcode{Proj}}{\emph{Proj\_name=None}, \emph{Proj\_name\_src='default'}}{}
The projection secondary method (Proj) is used when plotting 2D data, and define
how to project from lon/lat coord to another mapping system (lambert, mercator, mollweide, etc...)

This class is used to define a projection table entry used in VCS, or it
can be used to change some or all of the attributes in an existing
projection table entry.


\begin{fulllineitems}
\pysigline{\sphinxbfcode{Projection~Transformation~Package~Projection~Parameters}}~
\begin{longtable}{|l|l|l|l|l|l|l|l|l|l|}
\hline
\textsf{\relax \unskip}\relax &\multicolumn{9}{l|}{\relax \textsf{\relax 
Array Element
\unskip}\relax \unskip}\relax \\
\hline\textsf{\relax 
\emph{Code \& Projection Id}
\unskip}\relax &\textsf{\relax 
1
\unskip}\relax &\textsf{\relax 
2
\unskip}\relax &\textsf{\relax 
3
\unskip}\relax &\textsf{\relax 
4
\unskip}\relax &\textsf{\relax 
5
\unskip}\relax &\textsf{\relax 
6
\unskip}\relax &\textsf{\relax 
7
\unskip}\relax &\textsf{\relax 
8
\unskip}\relax &\textsf{\relax 
9
\unskip}\relax \\
\hline\endfirsthead

\multicolumn{10}{c}%
{{\tablecontinued{\tablename\ \thetable{} -- continued from previous page}}} \\
\hline
\textsf{\relax \unskip}\relax &\multicolumn{9}{l|}{\relax \textsf{\relax 
Array Element
\unskip}\relax \unskip}\relax \\
\hline\textsf{\relax 
\emph{Code \& Projection Id}
\unskip}\relax &\textsf{\relax 
1
\unskip}\relax &\textsf{\relax 
2
\unskip}\relax &\textsf{\relax 
3
\unskip}\relax &\textsf{\relax 
4
\unskip}\relax &\textsf{\relax 
5
\unskip}\relax &\textsf{\relax 
6
\unskip}\relax &\textsf{\relax 
7
\unskip}\relax &\textsf{\relax 
8
\unskip}\relax &\textsf{\relax 
9
\unskip}\relax \\
\hline\endhead

\hline \multicolumn{10}{|r|}{{\tablecontinued{Continued on next page}}} \\ \hline
\endfoot

\endlastfoot


0 Geographic
&&&&&&&&&\\
\hline
1 U T M
&
Lon/Z
&
Lat/Z
&&&&&&&\\
\hline
2 State Plane
&&&&&&&&&\\
\hline
3 Albers Equal Area
&
SMajor
&
SMinor
&
STDPR1
&
STDPR2
&
CentMer
&
OriginLat
&
FE
&
FN
&\\
\hline
4 Lambert Conformal C
&
SMajor
&
SMinor
&
STDPR1
&
STDPR2
&
CentMer
&
OriginLat
&
FE
&
FN
&\\
\hline
5 Mercator
&
SMajor
&
SMinor
&&&
CentMer
&
TrueScale
&
FE
&
FN
&\\
\hline
6 Polar Stereographic
&
SMajor
&
SMinor
&&&
LongPol
&
TrueScale
&
FE
&
FN
&\\
\hline
7 Polyconic
&
SMajor
&
SMinor
&&&
CentMer
&
OriginLat
&
FE
&
FN
&\\
\hline
8 Equid. Conic A
&
SMajor
&
SMinor
&
STDPAR
&&
CentMer
&
OriginLat
&
FE
&
FN
&
zero
\\
\hline
Equid. Conic B
&
SMajor
&
SMinor
&
STDPR1
&
STDPR2
&
CentMer
&
OriginLat
&
FE
&
FN
&
one
\\
\hline
9 Transverse Mercator
&
SMajor
&
SMinor
&
Factor
&&
CentMer
&
OriginLat
&
FE
&
FN
&\\
\hline
10 Stereographic
&
Sphere
&&&&
CentLon
&
CenterLat
&
FE
&
FN
&\\
\hline
11 Lambert Azimuthal
&
Sphere
&&&&
CentLon
&
CenterLat
&
FE
&
FN
&\\
\hline
12 Azimuthal
&
Sphere
&&&&
CentLon
&
CenterLat
&
FE
&
FN
&\\
\hline
13 Gnomonic
&
Sphere
&&&&
CentLon
&
CenterLat
&
FE
&
FN
&\\
\hline
14 Orthographic
&
Sphere
&&&&
CentLon
&
CenterLat
&
FE
&
FN
&\\
\hline
15 Gen. Vert. Near Per
&
Sphere
&&
Height
&&
CentLon
&
CenterLat
&
FE
&
FN
&\\
\hline
16 Sinusoidal
&
Sphere
&&&&
CentMer
&&
FE
&
FN
&\\
\hline
17 Equirectangular
&
Sphere
&&&&
CentMer
&
TrueScale
&
FE
&
FN
&\\
\hline
18 Miller Cylindrical
&
Sphere
&&&&
CentMer
&&
FE
&
FN
&\\
\hline
19 Van der Grinten
&
Sphere
&&&&
CentMer
&
OriginLat
&
FE
&
FN
&\\
\hline
20 Hotin Oblique Merc A
&
SMajor
&
SMinor
&
Factor
&&&
OriginLat
&
FE
&
FN
&
Long1
\\
\hline
Hotin Oblique Merc B
&
SMajor
&
SMinor
&
Factor
&
AziAng
&
AzmthPt
&
OriginLat
&
FE
&
FN
&\\
\hline
21 Robinson
&
Sphere
&&&&
CentMer
&&
FE
&
FN
&\\
\hline
22 Space Oblique Merc A
&
SMajor
&
SMinor
&&
IncAng
&
AscLong
&&
FE
&
FN
&
PSRev
\\
\hline
Space Oblique Merc B
&
SMajor
&
SMinor
&
Satnum
&
Path
&&&
FE
&
FN
&\\
\hline
23 Alaska Conformal
&
SMajor
&
SMinor
&&&&&
FE
&
FN
&\\
\hline
24 Interrupted Goode
&
Sphere
&&&&&&&&\\
\hline
25 Mollweide
&
Sphere
&&&&
CentMer
&&
FE
&
FN
&\\
\hline
26 Interrupt Mollweide
&
Sphere
&&&&&&&&\\
\hline
27 Hammer
&
Sphere
&&&&
CentMer
&&
FE
&
FN
&\\
\hline
28 Wagner IV
&
Sphere
&&&&
CentMer
&&
FE
&
FN
&\\
\hline
29 Wagner VII
&
Sphere
&&&&
CentMer
&&
FE
&
FN
&\\
\hline
30 Oblated Equal Area
&
Sphere
&&
Shapem
&
Shapen
&
CentLon
&
CenterLat
&
FE
&
FN
&
Angle
\\
\hline\end{longtable}


\noindent\begin{tabulary}{\linewidth}{|L|L|L|L|L|}
\hline
\textsf{\relax \unskip}\relax &\multicolumn{4}{l|}{\relax \textsf{\relax 
Array Element
\unskip}\relax \unskip}\relax \\
\hline\textsf{\relax 
Code \& Projection Id
\unskip}\relax &\textsf{\relax 
10
\unskip}\relax &\textsf{\relax 
11
\unskip}\relax &\textsf{\relax 
12
\unskip}\relax &\textsf{\relax 
13
\unskip}\relax \\
\hline
20 Hotin Oblique Merc A
&
Lat1
&
Long2
&
Lat2
&
zero
\\
\hline
Hotin Oblique Merc B
&&&&
one
\\
\hline
22 Space Oblique Merc A
&
LRat
&
PFlag
&&
zero
\\
\hline
Space Oblique Merc B
&&&&
one
\\
\hline\end{tabulary}

\begin{quote}

\begin{notice}{note}{Note:}
All other projections are blank (containing 0) for elements 10-13
\end{notice}
\end{quote}
\begin{description}
\item[{Lon/Z\index{Lon/Z|textbf}}] \leavevmode\phantomsection\label{vcs/misc/projection:term-lon-z}
Longitude of any point in the UTM zone or zero.  If zero,
a zone code must be specified.

\item[{Lat/Z\index{Lat/Z|textbf}}] \leavevmode\phantomsection\label{vcs/misc/projection:term-lat-z}
Latitude of any point in the UTM zone or zero.  If zero, a
zone code must be specified.

\item[{SMajor\index{SMajor|textbf}}] \leavevmode\phantomsection\label{vcs/misc/projection:term-smajor}
Semi-major axis of ellipsoid.  If zero, Clarke 1866 in meters
is assumed.

\item[{SMinor\index{SMinor|textbf}}] \leavevmode\phantomsection\label{vcs/misc/projection:term-sminor}
Eccentricity squared of the ellipsoid if less than zero,
if zero, a spherical form is assumed, or if greater than
zero, the semi-minor axis of ellipsoid.

\item[{Sphere\index{Sphere|textbf}}] \leavevmode\phantomsection\label{vcs/misc/projection:term-sphere}
Radius of reference sphere.  If zero, 6370997 meters is used.

\item[{STDPAR\index{STDPAR|textbf}}] \leavevmode\phantomsection\label{vcs/misc/projection:term-stdpar}
Latitude of the standard parallel

\item[{STDPR1\index{STDPR1|textbf}}] \leavevmode\phantomsection\label{vcs/misc/projection:term-stdpr1}
Latitude of the first standard parallel

\item[{STDPR2\index{STDPR2|textbf}}] \leavevmode\phantomsection\label{vcs/misc/projection:term-stdpr2}
Latitude of the second standard parallel

\item[{CentMer\index{CentMer|textbf}}] \leavevmode\phantomsection\label{vcs/misc/projection:term-centmer}
Longitude of the central meridian

\item[{OriginLat\index{OriginLat|textbf}}] \leavevmode\phantomsection\label{vcs/misc/projection:term-originlat}
Latitude of the projection origin

\item[{FE\index{FE|textbf}}] \leavevmode\phantomsection\label{vcs/misc/projection:term-fe}
False easting in the same units as the semi-major axis

\item[{FN\index{FN|textbf}}] \leavevmode\phantomsection\label{vcs/misc/projection:term-fn}
False northing in the same units as the semi-major axis

\item[{TrueScale\index{TrueScale|textbf}}] \leavevmode\phantomsection\label{vcs/misc/projection:term-truescale}
Latitude of true scale

\item[{LongPol\index{LongPol|textbf}}] \leavevmode\phantomsection\label{vcs/misc/projection:term-longpol}
Longitude down below pole of map

\item[{Factor\index{Factor|textbf}}] \leavevmode\phantomsection\label{vcs/misc/projection:term-factor}
Scale factor at central meridian (Transverse Mercator) or
center of projection (Hotine Oblique Mercator)

\item[{CentLon\index{CentLon|textbf}}] \leavevmode\phantomsection\label{vcs/misc/projection:term-centlon}
Longitude of center of projection

\item[{CenterLat\index{CenterLat|textbf}}] \leavevmode\phantomsection\label{vcs/misc/projection:term-centerlat}
Latitude of center of projection

\item[{Height\index{Height|textbf}}] \leavevmode\phantomsection\label{vcs/misc/projection:term-height}
Height of perspective point

\item[{Long1\index{Long1|textbf}}] \leavevmode\phantomsection\label{vcs/misc/projection:term-long1}
Longitude of first point on center line (Hotine Oblique
Mercator, format A)

\item[{Long2\index{Long2|textbf}}] \leavevmode\phantomsection\label{vcs/misc/projection:term-long2}
Longitude of second point on center line (Hotine Oblique
Mercator, format A)

\item[{Lat1\index{Lat1|textbf}}] \leavevmode\phantomsection\label{vcs/misc/projection:term-lat1}
Latitude of first point on center line (Hotine Oblique
Mercator, format A)

\item[{Lat2\index{Lat2|textbf}}] \leavevmode\phantomsection\label{vcs/misc/projection:term-lat2}
Latitude of second point on center line (Hotine Oblique
Mercator, format A)

\item[{AziAng\index{AziAng|textbf}}] \leavevmode\phantomsection\label{vcs/misc/projection:term-aziang}
Azimuth angle east of north of center line (Hotine Oblique
Mercator, format B)

\item[{AzmthPt\index{AzmthPt|textbf}}] \leavevmode\phantomsection\label{vcs/misc/projection:term-azmthpt}
Longitude of point on central meridian where azimuth occurs
(Hotine Oblique Mercator, format B)

\item[{IncAng\index{IncAng|textbf}}] \leavevmode\phantomsection\label{vcs/misc/projection:term-incang}
Inclination of orbit at ascending node, counter-clockwise
from equator (SOM, format A)

\item[{AscLong\index{AscLong|textbf}}] \leavevmode\phantomsection\label{vcs/misc/projection:term-asclong}
Longitude of ascending orbit at equator (SOM, format A)

\item[{PSRev\index{PSRev|textbf}}] \leavevmode\phantomsection\label{vcs/misc/projection:term-psrev}
Period of satellite revolution in minutes (SOM, format A)

\item[{LRat\index{LRat|textbf}}] \leavevmode\phantomsection\label{vcs/misc/projection:term-lrat}
Landsat ratio to compensate for confusion at northern end
of orbit (SOM, format A -- use 0.5201613)

\item[{PFlag\index{PFlag|textbf}}] \leavevmode\phantomsection\label{vcs/misc/projection:term-pflag}
End of path flag for Landsat:  0 = start of path,
1 = end of path (SOM, format A)

\item[{Satnum\index{Satnum|textbf}}] \leavevmode\phantomsection\label{vcs/misc/projection:term-satnum}
Landsat Satellite Number (SOM, format B)

\item[{Path\index{Path|textbf}}] \leavevmode\phantomsection\label{vcs/misc/projection:term-path}
Landsat Path Number (Use WRS-1 for Landsat 1, 2 and 3 and
WRS-2 for Landsat 4, 5 and 6.)  (SOM, format B)

\item[{Shapem\index{Shapem|textbf}}] \leavevmode\phantomsection\label{vcs/misc/projection:term-shapem}
Oblated Equal Area oval shape parameter m

\item[{Shapen\index{Shapen|textbf}}] \leavevmode\phantomsection\label{vcs/misc/projection:term-shapen}
Oblated Equal Area oval shape parameter n

\item[{Angle\index{Angle|textbf}}] \leavevmode\phantomsection\label{vcs/misc/projection:term-angle}
Oblated Equal Area oval rotation angle

\end{description}


\begin{fulllineitems}
\pysigline{\sphinxbfcode{Array~Elements:}}~\begin{itemize}
\item {} 
Array elements 14 and 15 are set to zero

\item {} 
All array elements with blank fields are set to zero

\item {} \begin{description}
\item[{All angles (latitudes, longitudes, azimuths, etc.) are entered in packed}] \leavevmode
degrees/ minutes/ seconds (DDDMMMSSS.SS) format

\end{description}

\end{itemize}

\end{fulllineitems}



\begin{fulllineitems}
\pysigline{\sphinxbfcode{Space~Oblique~Mercator~A~projection:}}~\begin{itemize}
\item {} \begin{description}
\item[{A portion of Landsat rows 1 and 2 may also be seen as parts of rows}] \leavevmode
246 or 247.  To place these locations at rows 246 or 247, set the end of
path flag (parameter 11) to 1--end of path.  This flag defaults to zero.

\end{description}

\item {} \begin{description}
\item[{When Landsat-1,2,3 orbits are being used, use the following values}] \leavevmode
for the specified parameters:
\begin{itemize}
\item {} 
Parameter 4   099005031.2

\item {} 
Parameter 5   128.87 degrees - (360/251 * path number) in packed DMS format

\item {} 
Parameter 9   103.2669323

\item {} 
Parameter 10  0.5201613

\end{itemize}

\end{description}

\item {} \begin{description}
\item[{When Landsat-4,5 orbits are being used, use the following values}] \leavevmode
for the specified parameters:
\begin{itemize}
\item {} 
Parameter 4   098012000.0

\item {} 
Parameter 5   129.30 degrees - (360/233 * path number) in packed DMS format

\item {} 
Parameter 9   98.884119

\item {} 
Parameter 10  0.5201613

\end{itemize}

\end{description}

\end{itemize}

\end{fulllineitems}


\begin{notice}{note}{Note:}
In vcs angles can be entered either in DDDMMMSSS or regular angle format.
\end{notice}

\end{fulllineitems}



\begin{fulllineitems}
\pysigline{\sphinxbfcode{Useful~Functions:}}~
\begin{Verbatim}[commandchars=\\\{\}]
\PYG{c+c1}{\PYGZsh{} VCS Canvas Constructor}
\PYG{n}{a}\PYG{o}{=}\PYG{n}{vcs}\PYG{o}{.}\PYG{n}{init}\PYG{p}{(}\PYG{p}{)}
\PYG{c+c1}{\PYGZsh{} Show predefined projection secondary methods}
\PYG{n}{a}\PYG{o}{.}\PYG{n}{show}\PYG{p}{(}\PYG{l+s+s1}{\PYGZsq{}}\PYG{l+s+s1}{projection}\PYG{l+s+s1}{\PYGZsq{}}\PYG{p}{)}
\end{Verbatim}

\end{fulllineitems}



\begin{fulllineitems}
\pysigline{\sphinxbfcode{Create~a~Canvas~object~to~work~with:}}~
\begin{Verbatim}[commandchars=\\\{\}]
\PYG{n}{a}\PYG{o}{=}\PYG{n}{vcs}\PYG{o}{.}\PYG{n}{init}\PYG{p}{(}\PYG{p}{)}
\end{Verbatim}

\end{fulllineitems}



\begin{fulllineitems}
\pysigline{\sphinxbfcode{Create~a~new~instance~of~projection:}}~
\begin{Verbatim}[commandchars=\\\{\}]
\PYG{c+c1}{\PYGZsh{} Copies content of \PYGZsq{}quick\PYGZsq{} to \PYGZsq{}new\PYGZsq{}}
\PYG{n}{p}\PYG{o}{=}\PYG{n}{a}\PYG{o}{.}\PYG{n}{createprojection}\PYG{p}{(}\PYG{l+s+s1}{\PYGZsq{}}\PYG{l+s+s1}{new}\PYG{l+s+s1}{\PYGZsq{}}\PYG{p}{,}\PYG{l+s+s1}{\PYGZsq{}}\PYG{l+s+s1}{quick}\PYG{l+s+s1}{\PYGZsq{}}\PYG{p}{)}
\PYG{c+c1}{\PYGZsh{} Copies content of \PYGZsq{}default\PYGZsq{} to \PYGZsq{}new\PYGZsq{}}
\PYG{n}{p}\PYG{o}{=}\PYG{n}{a}\PYG{o}{.}\PYG{n}{createprojection}\PYG{p}{(}\PYG{l+s+s1}{\PYGZsq{}}\PYG{l+s+s1}{new}\PYG{l+s+s1}{\PYGZsq{}}\PYG{p}{)}
\end{Verbatim}

\end{fulllineitems}



\begin{fulllineitems}
\pysigline{\sphinxbfcode{Modify~an~existing~projection:}}~
\begin{Verbatim}[commandchars=\\\{\}]
\PYG{n}{p}\PYG{o}{=}\PYG{n}{a}\PYG{o}{.}\PYG{n}{getprojection}\PYG{p}{(}\PYG{l+s+s1}{\PYGZsq{}}\PYG{l+s+s1}{lambert}\PYG{l+s+s1}{\PYGZsq{}}\PYG{p}{)}
\PYG{c+c1}{\PYGZsh{} List all the projection attribute values}
\PYG{n}{p}\PYG{o}{.}\PYG{n}{list}\PYG{p}{(}\PYG{p}{)}
\PYG{n}{p}\PYG{o}{.}\PYG{n}{type}\PYG{o}{=}\PYG{l+s+s1}{\PYGZsq{}}\PYG{l+s+s1}{lambert}\PYG{l+s+s1}{\PYGZsq{}}
\PYG{c+c1}{\PYGZsh{} Fill a list with projection parameter values}
\PYG{n}{params}\PYG{o}{=} \PYG{p}{[}\PYG{p}{]}
\PYG{k}{for} \PYG{n}{\PYGZus{}} \PYG{o+ow}{in} \PYG{n+nb}{range}\PYG{p}{(}\PYG{l+m+mi}{0}\PYG{p}{,}\PYG{l+m+mi}{14}\PYG{p}{)}\PYG{p}{:}
   \PYG{n}{params}\PYG{o}{.}\PYG{n}{append}\PYG{p}{(}\PYG{l+m+mf}{1.e20}\PYG{p}{)}
\PYG{c+c1}{\PYGZsh{} params now a list with 1.e20, 15 times}
\PYG{n}{p}\PYG{o}{.}\PYG{n}{parameters}\PYG{o}{=} \PYG{n}{params}
\PYG{n}{iso}\PYG{o}{=}\PYG{n}{x}\PYG{o}{.}\PYG{n}{createisoline}\PYG{p}{(}\PYG{l+s+s1}{\PYGZsq{}}\PYG{l+s+s1}{new}\PYG{l+s+s1}{\PYGZsq{}}\PYG{p}{)}
\PYG{n}{iso}\PYG{o}{.}\PYG{n}{projection}\PYG{o}{=}\PYG{n}{p}
\PYG{c+c1}{\PYGZsh{} or}
\PYG{n}{iso}\PYG{o}{.}\PYG{n}{projection}\PYG{o}{=}\PYG{l+s+s1}{\PYGZsq{}}\PYG{l+s+s1}{lambert}\PYG{l+s+s1}{\PYGZsq{}}
\end{Verbatim}

\end{fulllineitems}

\index{script() (vcs.projection.Proj method)}

\begin{fulllineitems}
\phantomsection\label{vcs/misc/projection:vcs.projection.Proj.script}\pysiglinewithargsret{\sphinxbfcode{script}}{\emph{script\_filename=None}, \emph{mode=None}}{}
Saves out a copy of the projection graphics method in JSON, or Python format to a designated file.
\begin{quote}

\begin{notice}{note}{Note:}
If the the filename has a `.py' at the end, it will produce a
Python script. If no extension is given, then by default a
.json file containing a JSON serialization of the object's
data will be produced.
\end{notice}

\begin{notice}{warning}{Warning:}
VCS Scripts Deprecated.
SCR script files are no longer generated by this function.
\end{notice}
\end{quote}
\begin{quote}\begin{description}
\item[{Example}] \leavevmode
\begin{Verbatim}[commandchars=\\\{\}]
\PYG{g+gp}{\PYGZgt{}\PYGZgt{}\PYGZgt{} }\PYG{n}{a}\PYG{o}{=}\PYG{n}{vcs}\PYG{o}{.}\PYG{n}{init}\PYG{p}{(}\PYG{p}{)} \PYG{c+c1}{\PYGZsh{} Make a Canvas object to work with}
\PYG{g+gp}{\PYGZgt{}\PYGZgt{}\PYGZgt{} }\PYG{n}{ex}\PYG{o}{=}\PYG{n}{a}\PYG{o}{.}\PYG{n}{getprojection}\PYG{p}{(}\PYG{p}{)} \PYG{c+c1}{\PYGZsh{} Get default projection}
\PYG{g+gp}{\PYGZgt{}\PYGZgt{}\PYGZgt{} }\PYG{n}{ex}\PYG{o}{.}\PYG{n}{script}\PYG{p}{(}\PYG{l+s+s1}{\PYGZsq{}}\PYG{l+s+s1}{filename.py}\PYG{l+s+s1}{\PYGZsq{}}\PYG{p}{)} \PYG{c+c1}{\PYGZsh{} Append to a Python script named \PYGZsq{}filename.py\PYGZsq{}}
\PYG{g+gp}{\PYGZgt{}\PYGZgt{}\PYGZgt{} }\PYG{n}{ex}\PYG{o}{.}\PYG{n}{script}\PYG{p}{(}\PYG{l+s+s1}{\PYGZsq{}}\PYG{l+s+s1}{filename}\PYG{l+s+s1}{\PYGZsq{}}\PYG{p}{,}\PYG{l+s+s1}{\PYGZsq{}}\PYG{l+s+s1}{w}\PYG{l+s+s1}{\PYGZsq{}}\PYG{p}{)} \PYG{c+c1}{\PYGZsh{} Create or overwrite a JSON file \PYGZsq{}filename.json\PYGZsq{}.}
\end{Verbatim}

\item[{Parameters}] \leavevmode\begin{itemize}
\item {} 
\textbf{\texttt{script\_filename}} (\href{https://docs.python.org/2/library/functions.html\#str}{\emph{\texttt{str}}}) -- Output name of the script file. If no extension is specified, a .json object is created.

\item {} 
\textbf{\texttt{mode}} (\href{https://docs.python.org/2/library/functions.html\#str}{\emph{\texttt{str}}}) -- Either `w' for replace, or `a' for append. Defaults to `a', if not specified.

\end{itemize}

\end{description}\end{quote}

\end{fulllineitems}


\end{fulllineitems}



\subsection{colormap}
\label{vcs/misc/colormap:module-vcs.colormap}\label{vcs/misc/colormap::doc}\label{vcs/misc/colormap:colormap}\index{vcs.colormap (module)}
\# Colormap (Cp) module
\index{Cp (class in vcs.colormap)}

\begin{fulllineitems}
\phantomsection\label{vcs/misc/colormap:vcs.colormap.Cp}\pysiglinewithargsret{\sphinxstrong{class }\sphinxcode{vcs.colormap.}\sphinxbfcode{Cp}}{\emph{Cp\_name}, \emph{Cp\_name\_src='default'}}{}
The Colormap object allows the manipulation of the colormap index R,G,B values.

This class is used to define a colormap table entry used in VCS, or it
can be used to change some or all of the colormap R,G,B attributes in an
existing colormap table entry.


\begin{fulllineitems}
\pysigline{\sphinxbfcode{Some~Useful~Functions:}}~
\begin{Verbatim}[commandchars=\\\{\}]
\PYG{c+c1}{\PYGZsh{} Constructor}
\PYG{n}{a}\PYG{o}{=}\PYG{n}{vcs}\PYG{o}{.}\PYG{n}{init}\PYG{p}{(}\PYG{p}{)}
\PYG{c+c1}{\PYGZsh{} Show predefined colormap objects}
\PYG{n}{a}\PYG{o}{.}\PYG{n}{show}\PYG{p}{(}\PYG{l+s+s1}{\PYGZsq{}}\PYG{l+s+s1}{colormap}\PYG{l+s+s1}{\PYGZsq{}}\PYG{p}{)}
\PYG{c+c1}{\PYGZsh{} Updates the VCS Canvas at user\PYGZsq{}s request}
\PYG{n}{a}\PYG{o}{.}\PYG{n}{update}\PYG{p}{(}\PYG{p}{)}
\PYG{c+c1}{\PYGZsh{} If mode=1, automatic update}
\PYG{n}{a}\PYG{o}{.}\PYG{n}{mode}\PYG{o}{=}\PYG{l+m+mi}{1}
\PYG{c+c1}{\PYGZsh{}If mode=0, use update function to update the VCS Canvas.}
\PYG{n}{a}\PYG{o}{.}\PYG{n}{mode}\PYG{o}{=}\PYG{l+m+mi}{0}
\end{Verbatim}

\end{fulllineitems}



\begin{fulllineitems}
\pysigline{\sphinxbfcode{General~use~of~a~colormap:}}~
\begin{Verbatim}[commandchars=\\\{\}]
\PYG{c+c1}{\PYGZsh{} Create a VCS Canvas object}
\PYG{n}{a}\PYG{o}{=}\PYG{n}{vcs}\PYG{o}{.}\PYG{n}{init}\PYG{p}{(}\PYG{p}{)}
\PYG{c+c1}{\PYGZsh{}To Create a new instance of colormap use:}
\PYG{c+c1}{\PYGZsh{} Copies content of \PYGZsq{}red\PYGZsq{} to \PYGZsq{}new\PYGZsq{}}
\PYG{n}{cp}\PYG{o}{=}\PYG{n}{a}\PYG{o}{.}\PYG{n}{createcolormap}\PYG{p}{(}\PYG{l+s+s1}{\PYGZsq{}}\PYG{l+s+s1}{new}\PYG{l+s+s1}{\PYGZsq{}}\PYG{p}{,}\PYG{l+s+s1}{\PYGZsq{}}\PYG{l+s+s1}{quick}\PYG{l+s+s1}{\PYGZsq{}}\PYG{p}{)}
\PYG{c+c1}{\PYGZsh{} Copies content of \PYGZsq{}default\PYGZsq{} to \PYGZsq{}new\PYGZsq{}}
\PYG{n}{cp}\PYG{o}{=}\PYG{n}{a}\PYG{o}{.}\PYG{n}{createcolormap}\PYG{p}{(}\PYG{l+s+s1}{\PYGZsq{}}\PYG{l+s+s1}{new}\PYG{l+s+s1}{\PYGZsq{}}\PYG{p}{)}
\end{Verbatim}

\end{fulllineitems}



\begin{fulllineitems}
\pysigline{\sphinxbfcode{Modifying~an~existing~colormap:}}~
\begin{Verbatim}[commandchars=\\\{\}]
\PYG{n}{cp}\PYG{o}{=}\PYG{n}{a}\PYG{o}{.}\PYG{n}{getcolormap}\PYG{p}{(}\PYG{l+s+s1}{\PYGZsq{}}\PYG{l+s+s1}{quick}\PYG{l+s+s1}{\PYGZsq{}}\PYG{p}{)}
\end{Verbatim}

\end{fulllineitems}



\begin{fulllineitems}
\pysigline{\sphinxbfcode{Overview~of~colormap~attributes:}}~\begin{itemize}
\item {} 
List all the colormap indices and R,G,B attribute values
\begin{quote}

\begin{Verbatim}[commandchars=\\\{\}]
\PYG{n}{cp}\PYG{o}{.}\PYG{n}{list}\PYG{p}{(}\PYG{p}{)}
\end{Verbatim}
\end{quote}

\item {} 
Setting colormap attribute values:
\begin{quote}

\begin{Verbatim}[commandchars=\\\{\}]
\PYG{c+c1}{\PYGZsh{} Index, R, G, B}
\PYG{n}{cp}\PYG{o}{.}\PYG{n}{color}\PYG{o}{=}\PYG{l+m+mi}{16}\PYG{p}{,}\PYG{l+m+mi}{100}\PYG{p}{,}\PYG{l+m+mi}{0}\PYG{p}{,}\PYG{l+m+mi}{0}
\PYG{c+c1}{\PYGZsh{} Index range from 0 to 255, but can only modify from 0 to 239}
\PYG{n}{cp}\PYG{o}{.}\PYG{n}{color}\PYG{o}{=}\PYG{l+m+mi}{16}\PYG{p}{,}\PYG{l+m+mi}{0}\PYG{p}{,}\PYG{l+m+mi}{100}\PYG{p}{,}\PYG{l+m+mi}{0}
\PYG{c+c1}{\PYGZsh{} R, G, B values range from 0 to 100, where 0 is low intensity and 100 is highest intensity}
\PYG{n}{cp}\PYG{o}{.}\PYG{n}{color}\PYG{o}{=}\PYG{l+m+mi}{17}\PYG{p}{,}\PYG{l+m+mi}{0}\PYG{p}{,}\PYG{l+m+mi}{0}\PYG{p}{,}\PYG{l+m+mi}{100}
\end{Verbatim}
\end{quote}

\end{itemize}

\end{fulllineitems}

\index{getcolorcell() (vcs.colormap.Cp method)}

\begin{fulllineitems}
\phantomsection\label{vcs/misc/colormap:vcs.colormap.Cp.getcolorcell}\pysiglinewithargsret{\sphinxbfcode{getcolorcell}}{\emph{index}}{}
Gets the R,G,B,A values of a colorcell.
\begin{quote}\begin{description}
\item[{Example}] \leavevmode
\begin{Verbatim}[commandchars=\\\{\}]
\PYG{g+gp}{\PYGZgt{}\PYGZgt{}\PYGZgt{} }\PYG{n}{a}\PYG{o}{=}\PYG{n}{vcs}\PYG{o}{.}\PYG{n}{init}\PYG{p}{(}\PYG{p}{)} \PYG{c+c1}{\PYGZsh{} Create a vcs Canvas}
\PYG{g+gp}{\PYGZgt{}\PYGZgt{}\PYGZgt{} }\PYG{n}{cmap} \PYG{o}{=} \PYG{n}{a}\PYG{o}{.}\PYG{n}{createcolormap}\PYG{p}{(}\PYG{l+s+s1}{\PYGZsq{}}\PYG{l+s+s1}{example}\PYG{l+s+s1}{\PYGZsq{}}\PYG{p}{,} \PYG{l+s+s1}{\PYGZsq{}}\PYG{l+s+s1}{default}\PYG{l+s+s1}{\PYGZsq{}}\PYG{p}{)} \PYG{c+c1}{\PYGZsh{} Create a colormap}
\PYG{g+gp}{\PYGZgt{}\PYGZgt{}\PYGZgt{} }\PYG{n}{cmap}\PYG{o}{.}\PYG{n}{getcolorcell}\PYG{p}{(}\PYG{l+m+mi}{1}\PYG{p}{)} \PYG{c+c1}{\PYGZsh{} Get RGBA values}
\PYG{g+go}{[0, 0, 0, 100.0]}
\end{Verbatim}

\item[{Parameters}] \leavevmode
\textbf{\texttt{index}} (\href{https://docs.python.org/2/library/functions.html\#int}{\emph{\texttt{int}}}) -- Index of a cell in the colormap. Must be an integer from 0-255.

\item[{Returns}] \leavevmode
A list containing the red, green, blue, and alpha values (in that order), of the colorcell at the given index.

\item[{Return type}] \leavevmode
{\hyperref[vcs/graphics/boxfill:vcs.boxfill.Gfb.list]{\sphinxcrossref{list}}}

\end{description}\end{quote}

\end{fulllineitems}

\index{script() (vcs.colormap.Cp method)}

\begin{fulllineitems}
\phantomsection\label{vcs/misc/colormap:vcs.colormap.Cp.script}\pysiglinewithargsret{\sphinxbfcode{script}}{\emph{script\_filename=None}, \emph{mode=None}}{}
Saves out a copy of the colormap secondary method in JSON, or Python format to a designated file.
\begin{quote}

\begin{notice}{note}{Note:}
If the the filename has a `.py' at the end, it will produce a
Python script. If no extension is given, then by default a
.json file containing a JSON serialization of the object's
data will be produced.
\end{notice}

\begin{notice}{warning}{Warning:}
VCS Scripts Deprecated.
SCR script files are no longer generated by this function.
\end{notice}
\end{quote}
\begin{quote}\begin{description}
\item[{Example}] \leavevmode
\begin{Verbatim}[commandchars=\\\{\}]
\PYG{g+gp}{\PYGZgt{}\PYGZgt{}\PYGZgt{} }\PYG{n}{a}\PYG{o}{=}\PYG{n}{vcs}\PYG{o}{.}\PYG{n}{init}\PYG{p}{(}\PYG{p}{)} \PYG{c+c1}{\PYGZsh{} Make a Canvas object to work with}
\PYG{g+gp}{\PYGZgt{}\PYGZgt{}\PYGZgt{} }\PYG{n}{ex}\PYG{o}{=}\PYG{n}{a}\PYG{o}{.}\PYG{n}{getcolormap}\PYG{p}{(}\PYG{p}{)} \PYG{c+c1}{\PYGZsh{} Get default colormap}
\PYG{g+gp}{\PYGZgt{}\PYGZgt{}\PYGZgt{} }\PYG{n}{ex}\PYG{o}{.}\PYG{n}{script}\PYG{p}{(}\PYG{l+s+s1}{\PYGZsq{}}\PYG{l+s+s1}{filename.py}\PYG{l+s+s1}{\PYGZsq{}}\PYG{p}{)} \PYG{c+c1}{\PYGZsh{} Append to a Python script named \PYGZsq{}filename.py\PYGZsq{}}
\PYG{g+gp}{\PYGZgt{}\PYGZgt{}\PYGZgt{} }\PYG{n}{ex}\PYG{o}{.}\PYG{n}{script}\PYG{p}{(}\PYG{l+s+s1}{\PYGZsq{}}\PYG{l+s+s1}{filename}\PYG{l+s+s1}{\PYGZsq{}}\PYG{p}{,}\PYG{l+s+s1}{\PYGZsq{}}\PYG{l+s+s1}{w}\PYG{l+s+s1}{\PYGZsq{}}\PYG{p}{)} \PYG{c+c1}{\PYGZsh{} Create or overwrite a JSON file \PYGZsq{}filename.json\PYGZsq{}.}
\end{Verbatim}

\item[{Parameters}] \leavevmode\begin{itemize}
\item {} 
\textbf{\texttt{script\_filename}} (\href{https://docs.python.org/2/library/functions.html\#str}{\emph{\texttt{str}}}) -- Output name of the script file. If no extension is specified, a .json object is created.

\item {} 
\textbf{\texttt{mode}} (\href{https://docs.python.org/2/library/functions.html\#str}{\emph{\texttt{str}}}) -- Either `w' for replace, or `a' for append. Defaults to `a', if not specified.

\end{itemize}

\end{description}\end{quote}

\end{fulllineitems}

\index{setcolorcell() (vcs.colormap.Cp method)}

\begin{fulllineitems}
\phantomsection\label{vcs/misc/colormap:vcs.colormap.Cp.setcolorcell}\pysiglinewithargsret{\sphinxbfcode{setcolorcell}}{\emph{index}, \emph{red}, \emph{green}, \emph{blue}, \emph{alpha=100.0}}{}
Sets the R,G,B,A values of a colorcell
\begin{quote}\begin{description}
\item[{Example}] \leavevmode
\begin{Verbatim}[commandchars=\\\{\}]
\PYG{g+gp}{\PYGZgt{}\PYGZgt{}\PYGZgt{} }\PYG{n}{a} \PYG{o}{=} \PYG{n}{vcs}\PYG{o}{.}\PYG{n}{init}\PYG{p}{(}\PYG{p}{)} \PYG{c+c1}{\PYGZsh{} Create a vcs Canvas}
\PYG{g+gp}{\PYGZgt{}\PYGZgt{}\PYGZgt{} }\PYG{n}{cmap} \PYG{o}{=} \PYG{n}{a}\PYG{o}{.}\PYG{n}{createcolormap}\PYG{p}{(}\PYG{l+s+s1}{\PYGZsq{}}\PYG{l+s+s1}{example}\PYG{l+s+s1}{\PYGZsq{}}\PYG{p}{,} \PYG{l+s+s1}{\PYGZsq{}}\PYG{l+s+s1}{default}\PYG{l+s+s1}{\PYGZsq{}}\PYG{p}{)} \PYG{c+c1}{\PYGZsh{} Create a colormap}
\PYG{g+gp}{\PYGZgt{}\PYGZgt{}\PYGZgt{} }\PYG{n}{cmap}\PYG{o}{.}\PYG{n}{setcolorcell}\PYG{p}{(}\PYG{l+m+mi}{40}\PYG{p}{,}\PYG{l+m+mi}{80}\PYG{p}{,}\PYG{l+m+mi}{95}\PYG{p}{,}\PYG{l+m+mf}{1.0}\PYG{p}{)} \PYG{c+c1}{\PYGZsh{} Set RGBA values}
\end{Verbatim}

\item[{Parameters}] \leavevmode\begin{itemize}
\item {} 
\textbf{\texttt{index}} (\href{https://docs.python.org/2/library/functions.html\#int}{\emph{\texttt{int}}}) -- Integer from 0-255.

\item {} 
\textbf{\texttt{red}} (\href{https://docs.python.org/2/library/functions.html\#int}{\emph{\texttt{int}}}) -- Integer from 0-255 representing the concentration of red in the colorcell.

\item {} 
\textbf{\texttt{green}} (\href{https://docs.python.org/2/library/functions.html\#int}{\emph{\texttt{int}}}) -- Integer from 0-255 representing the concentration of green in the colorcell.

\item {} 
\textbf{\texttt{blue}} (\href{https://docs.python.org/2/library/functions.html\#int}{\emph{\texttt{int}}}) -- Integer from 0-255 representing the concentration of blue in the colorcell.

\item {} 
\textbf{\texttt{alpha}} (\href{https://docs.python.org/2/library/functions.html\#float}{\emph{\texttt{float}}}) -- Float representing the percentage of opacity in the colorcell.

\end{itemize}

\end{description}\end{quote}

\end{fulllineitems}


\end{fulllineitems}



\subsection{colors}
\label{vcs/misc/colors:module-vcs.colors}\label{vcs/misc/colors::doc}\label{vcs/misc/colors:colors}\index{vcs.colors (module)}\index{matplotlib2vcs() (in module vcs.colors)}

\begin{fulllineitems}
\phantomsection\label{vcs/misc/colors:vcs.colors.matplotlib2vcs}\pysiglinewithargsret{\sphinxcode{vcs.colors.}\sphinxbfcode{matplotlib2vcs}}{\emph{cmap}, \emph{vcs\_name=None}}{}
Convert a matplotlib colormap to a vcs colormap
Input can be either the actual matplotlib colormap or its name
Optional second argument: vcs\_name, name of the resulting vcs colormap
\begin{quote}\begin{description}
\item[{Parameters}] \leavevmode\begin{itemize}
\item {} 
\textbf{\texttt{cmap}} (\emph{\texttt{str , matplotlib colormap}}) -- A matplotlib colormap or string name of a matplotlib colormap

\item {} 
\textbf{\texttt{vcs\_name}} (\href{https://docs.python.org/2/library/functions.html\#str}{\emph{\texttt{str}}}) -- String to set the name of the generated VCS colormap

\end{itemize}

\item[{Returns}] \leavevmode
A VCS colormap object

\item[{Return type}] \leavevmode
{\hyperref[vcs/misc/colormap:vcs.colormap.Cp]{\sphinxcrossref{vcs.colormap.Cp}}}

\end{description}\end{quote}

\end{fulllineitems}



\subsection{displayplot}
\label{vcs/misc/displayplot:module-vcs.displayplot}\label{vcs/misc/displayplot:displayplot}\label{vcs/misc/displayplot::doc}\index{vcs.displayplot (module)}
\# Display Plot (Dp) module
\index{Dp (class in vcs.displayplot)}

\begin{fulllineitems}
\phantomsection\label{vcs/misc/displayplot:vcs.displayplot.Dp}\pysiglinewithargsret{\sphinxstrong{class }\sphinxcode{vcs.displayplot.}\sphinxbfcode{Dp}}{\emph{Dp\_name}, \emph{Dp\_name\_src='default'}, \emph{parent=None}}{}
The Display plot object allows the manipulation of the plot name, off,
priority, template, graphics type, graphics name, and data array(s).

This class is used to define a display plot table entry used in VCS, or it
can be used to change some or all of the display plot attributes in an
existing display plot table entry.


\begin{fulllineitems}
\pysigline{\sphinxbfcode{Useful~Functions:}}~
\begin{Verbatim}[commandchars=\\\{\}]
\PYG{c+c1}{\PYGZsh{} Canvas constructor}
\PYG{n}{a}\PYG{o}{=}\PYG{n}{vcs}\PYG{o}{.}\PYG{n}{init}\PYG{p}{(}\PYG{p}{)}
\PYG{c+c1}{\PYGZsh{} Show display plot objects}
\PYG{n}{a}\PYG{o}{.}\PYG{n}{show}\PYG{p}{(}\PYG{l+s+s1}{\PYGZsq{}}\PYG{l+s+s1}{plot}\PYG{l+s+s1}{\PYGZsq{}}\PYG{p}{)}
\PYG{c+c1}{\PYGZsh{} Updates the VCS Canvas at user\PYGZsq{}s request}
\PYG{n}{a}\PYG{o}{.}\PYG{n}{update}\PYG{p}{(}\PYG{p}{)}
\end{Verbatim}

\end{fulllineitems}



\begin{fulllineitems}
\pysigline{\sphinxbfcode{General~display~plot~usage:}}~
\begin{Verbatim}[commandchars=\\\{\}]
\PYG{c+c1}{\PYGZsh{}Create a VCS Canvas object}
\PYG{n}{a}\PYG{o}{=}\PYG{n}{vcs}\PYG{o}{.}\PYG{n}{init}\PYG{p}{(}\PYG{p}{)}
\PYG{c+c1}{\PYGZsh{}To Create a new instance of plot:}
\PYG{c+c1}{\PYGZsh{} Create a plot object}
\PYG{n}{p1}\PYG{o}{=}\PYG{n}{a}\PYG{o}{.}\PYG{n}{plot}\PYG{p}{(}\PYG{n}{s}\PYG{p}{)}
\PYG{c+c1}{\PYGZsh{}To Modify an existing plot in use:}
\PYG{n}{p1}\PYG{o}{=}\PYG{n}{a}\PYG{o}{.}\PYG{n}{getplot}\PYG{p}{(}\PYG{l+s+s1}{\PYGZsq{}}\PYG{l+s+s1}{dpy\PYGZus{}plot\PYGZus{}1}\PYG{l+s+s1}{\PYGZsq{}}\PYG{p}{)}
\end{Verbatim}

\end{fulllineitems}



\begin{fulllineitems}
\pysigline{\sphinxbfcode{Display~plot~object~attributes:}}~
\begin{Verbatim}[commandchars=\\\{\}]
\PYG{c+c1}{\PYGZsh{} Will list all the display plot attributes}
\PYG{n}{p1}\PYG{o}{.}\PYG{n}{list}\PYG{p}{(}\PYG{p}{)}
\PYG{c+c1}{\PYGZsh{} \PYGZdq{}On\PYGZdq{} or \PYGZdq{}Off\PYGZdq{} status, 1=on, 0=off}
\PYG{n}{p1}\PYG{o}{.}\PYG{n}{off}\PYG{o}{=}\PYG{l+m+mi}{1}
\PYG{c+c1}{\PYGZsh{} Priority to place plot in front of other objects}
\PYG{n}{p1}\PYG{o}{.}\PYG{n}{priority}\PYG{o}{=}\PYG{l+m+mi}{1}
\PYG{c+c1}{\PYGZsh{} Name of template object}
\PYG{n}{p1}\PYG{o}{.}\PYG{n}{template}\PYG{o}{=}\PYG{l+s+s1}{\PYGZsq{}}\PYG{l+s+s1}{quick}\PYG{l+s+s1}{\PYGZsq{}}
\PYG{c+c1}{\PYGZsh{} Graphics method type}
\PYG{n}{p1}\PYG{o}{.}\PYG{n}{g\PYGZus{}type}\PYG{o}{=}\PYG{l+s+s1}{\PYGZsq{}}\PYG{l+s+s1}{boxfill}\PYG{l+s+s1}{\PYGZsq{}}
\PYG{c+c1}{\PYGZsh{} Graphics method name}
\PYG{n}{p1}\PYG{o}{.}\PYG{n}{g\PYGZus{}name}\PYG{o}{=}\PYG{l+s+s1}{\PYGZsq{}}\PYG{l+s+s1}{quick}\PYG{l+s+s1}{\PYGZsq{}}
\PYG{c+c1}{\PYGZsh{} List of all the array names}
\PYG{n}{p1}\PYG{o}{.}\PYG{n}{array}\PYG{o}{=}\PYG{p}{[}\PYG{l+s+s1}{\PYGZsq{}}\PYG{l+s+s1}{a1}\PYG{l+s+s1}{\PYGZsq{}}\PYG{p}{]}
\end{Verbatim}

\end{fulllineitems}

\index{backend (vcs.displayplot.Dp attribute)}

\begin{fulllineitems}
\phantomsection\label{vcs/misc/displayplot:vcs.displayplot.Dp.backend}\pysigline{\sphinxbfcode{backend}}
dictionary of things the backend wants to be able to reuse

\end{fulllineitems}


\end{fulllineitems}



\subsection{error}
\label{vcs/misc/error:module-vcs.error}\label{vcs/misc/error::doc}\label{vcs/misc/error:error}\index{vcs.error (module)}
Error object for vcs module, vcsError


\subsection{manageElements}
\label{vcs/misc/manageElements:manageelements}\label{vcs/misc/manageElements:module-vcs.manageElements}\label{vcs/misc/manageElements::doc}\index{vcs.manageElements (module)}\index{check\_name\_source() (in module vcs.manageElements)}

\begin{fulllineitems}
\phantomsection\label{vcs/misc/manageElements:vcs.manageElements.check_name_source}\pysiglinewithargsret{\sphinxcode{vcs.manageElements.}\sphinxbfcode{check\_name\_source}}{\emph{name}, \emph{source}, \emph{typ}}{}
make sure it is a unique name for this type or generates a name for user

\end{fulllineitems}

\index{create3d\_dual\_scalar() (in module vcs.manageElements)}

\begin{fulllineitems}
\phantomsection\label{vcs/misc/manageElements:vcs.manageElements.create3d_dual_scalar}\pysiglinewithargsret{\sphinxcode{vcs.manageElements.}\sphinxbfcode{create3d\_dual\_scalar}}{\emph{name=None}, \emph{source='default'}}{}
Create a new dv3d graphics method given the the name and the existing
dv3d graphics method to copy the attributes from. If no existing
dv3d graphics method is given, then the default dv3d graphics method will be used as the graphics method
to which the attributes will be copied from.

\begin{notice}{note}{Note:}
If the name provided already exists, then an error will be returned. graphics method
names must be unique.
\end{notice}
\begin{quote}\begin{description}
\item[{Example}] \leavevmode
\begin{Verbatim}[commandchars=\\\{\}]
\PYG{g+gp}{\PYGZgt{}\PYGZgt{}\PYGZgt{} }\PYG{n}{vcs}\PYG{o}{.}\PYG{n}{show}\PYG{p}{(}\PYG{l+s+s1}{\PYGZsq{}}\PYG{l+s+s1}{3d\PYGZus{}dual\PYGZus{}scalar}\PYG{l+s+s1}{\PYGZsq{}}\PYG{p}{)} \PYG{c+c1}{\PYGZsh{} show all available 3d\PYGZus{}dual\PYGZus{}scalar}
\PYG{g+go}{*******************3d\PYGZus{}dual\PYGZus{}scalar Names List**********************}
\PYG{g+gp}{...}
\PYG{g+go}{*******************End 3d\PYGZus{}dual\PYGZus{}scalar Names List**********************}
\PYG{g+gp}{\PYGZgt{}\PYGZgt{}\PYGZgt{} }\PYG{n}{ex}\PYG{o}{=}\PYG{n}{vcs}\PYG{o}{.}\PYG{n}{create3d\PYGZus{}dual\PYGZus{}scalar}\PYG{p}{(}\PYG{l+s+s1}{\PYGZsq{}}\PYG{l+s+s1}{3d\PYGZus{}dual\PYGZus{}scalar\PYGZus{}ex1}\PYG{l+s+s1}{\PYGZsq{}}\PYG{p}{)} \PYG{c+c1}{\PYGZsh{} Create 3d\PYGZus{}dual\PYGZus{}scalar \PYGZsq{}3d\PYGZus{}dual\PYGZus{}scalar\PYGZus{}ex1\PYGZsq{} that inherits from \PYGZsq{}default\PYGZsq{}}
\PYG{g+gp}{\PYGZgt{}\PYGZgt{}\PYGZgt{} }\PYG{n}{vcs}\PYG{o}{.}\PYG{n}{listelements}\PYG{p}{(}\PYG{l+s+s1}{\PYGZsq{}}\PYG{l+s+s1}{3d\PYGZus{}dual\PYGZus{}scalar}\PYG{l+s+s1}{\PYGZsq{}}\PYG{p}{)} \PYG{c+c1}{\PYGZsh{} should now contain the \PYGZsq{}3d\PYGZus{}dual\PYGZus{}scalar\PYGZus{}ex1\PYGZsq{} 3d\PYGZus{}dual\PYGZus{}scalar}
\PYG{g+go}{[...\PYGZsq{}3d\PYGZus{}dual\PYGZus{}scalar\PYGZus{}ex1\PYGZsq{}...]}
\end{Verbatim}

\item[{Parameters}] \leavevmode\begin{itemize}
\item {} 
\textbf{\texttt{name}} (\href{https://docs.python.org/2/library/functions.html\#str}{\emph{\texttt{str}}}) -- The name of the created object

\item {} 
\textbf{\texttt{source}} (\emph{\texttt{a 3d\_dual\_scalar or a string name of a 3d\_dual\_scalar}}) -- The object to inherit from

\end{itemize}

\item[{Returns}] \leavevmode
A 3d\_dual\_scalar graphics method object

\item[{Return type}] \leavevmode
vcs.dv3d.Gf3DDualScalar

\end{description}\end{quote}

\end{fulllineitems}

\index{create3d\_scalar() (in module vcs.manageElements)}

\begin{fulllineitems}
\phantomsection\label{vcs/misc/manageElements:vcs.manageElements.create3d_scalar}\pysiglinewithargsret{\sphinxcode{vcs.manageElements.}\sphinxbfcode{create3d\_scalar}}{\emph{name=None}, \emph{source='default'}}{}
Create a new dv3d graphics method given the the name and the existing
dv3d graphics method to copy the attributes from. If no existing
dv3d graphics method is given, then the default dv3d graphics method will be used as the graphics method
to which the attributes will be copied from.

\begin{notice}{note}{Note:}
If the name provided already exists, then an error will be returned. graphics method
names must be unique.
\end{notice}
\begin{quote}\begin{description}
\item[{Example}] \leavevmode
\begin{Verbatim}[commandchars=\\\{\}]
\PYG{g+gp}{\PYGZgt{}\PYGZgt{}\PYGZgt{} }\PYG{n}{vcs}\PYG{o}{.}\PYG{n}{show}\PYG{p}{(}\PYG{l+s+s1}{\PYGZsq{}}\PYG{l+s+s1}{3d\PYGZus{}scalar}\PYG{l+s+s1}{\PYGZsq{}}\PYG{p}{)} \PYG{c+c1}{\PYGZsh{} show all available 3d\PYGZus{}scalar}
\PYG{g+go}{*******************3d\PYGZus{}scalar Names List**********************}
\PYG{g+gp}{...}
\PYG{g+go}{*******************End 3d\PYGZus{}scalar Names List**********************}
\PYG{g+gp}{\PYGZgt{}\PYGZgt{}\PYGZgt{} }\PYG{n}{ex}\PYG{o}{=}\PYG{n}{vcs}\PYG{o}{.}\PYG{n}{create3d\PYGZus{}scalar}\PYG{p}{(}\PYG{l+s+s1}{\PYGZsq{}}\PYG{l+s+s1}{3d\PYGZus{}scalar\PYGZus{}ex1}\PYG{l+s+s1}{\PYGZsq{}}\PYG{p}{)} \PYG{c+c1}{\PYGZsh{} Create 3d\PYGZus{}scalar \PYGZsq{}3d\PYGZus{}scalar\PYGZus{}ex1\PYGZsq{} that inherits from \PYGZsq{}default\PYGZsq{}}
\PYG{g+gp}{\PYGZgt{}\PYGZgt{}\PYGZgt{} }\PYG{n}{vcs}\PYG{o}{.}\PYG{n}{listelements}\PYG{p}{(}\PYG{l+s+s1}{\PYGZsq{}}\PYG{l+s+s1}{3d\PYGZus{}scalar}\PYG{l+s+s1}{\PYGZsq{}}\PYG{p}{)} \PYG{c+c1}{\PYGZsh{} should now contain the \PYGZsq{}3d\PYGZus{}scalar\PYGZus{}ex1\PYGZsq{} 3d\PYGZus{}scalar}
\PYG{g+go}{[...\PYGZsq{}3d\PYGZus{}scalar\PYGZus{}ex1\PYGZsq{}...]}
\end{Verbatim}

\item[{Parameters}] \leavevmode\begin{itemize}
\item {} 
\textbf{\texttt{name}} (\href{https://docs.python.org/2/library/functions.html\#str}{\emph{\texttt{str}}}) -- The name of the created object

\item {} 
\textbf{\texttt{source}} (\emph{\texttt{a 3d\_scalar or a string name of a 3d\_scalar}}) -- The object to inherit from

\end{itemize}

\item[{Returns}] \leavevmode
A 3d\_scalar graphics method object

\item[{Return type}] \leavevmode
vcs.dv3d.Gf3Dscalar

\end{description}\end{quote}

\end{fulllineitems}

\index{create3d\_vector() (in module vcs.manageElements)}

\begin{fulllineitems}
\phantomsection\label{vcs/misc/manageElements:vcs.manageElements.create3d_vector}\pysiglinewithargsret{\sphinxcode{vcs.manageElements.}\sphinxbfcode{create3d\_vector}}{\emph{name=None}, \emph{source='default'}}{}
Create a new dv3d graphics method given the the name and the existing
dv3d graphics method to copy the attributes from. If no existing
dv3d graphics method is given, then the default dv3d graphics method will be used as the graphics method
to which the attributes will be copied from.

\begin{notice}{note}{Note:}
If the name provided already exists, then an error will be returned. graphics method
names must be unique.
\end{notice}
\begin{quote}\begin{description}
\item[{Example}] \leavevmode
\begin{Verbatim}[commandchars=\\\{\}]
\PYG{g+gp}{\PYGZgt{}\PYGZgt{}\PYGZgt{} }\PYG{n}{vcs}\PYG{o}{.}\PYG{n}{show}\PYG{p}{(}\PYG{l+s+s1}{\PYGZsq{}}\PYG{l+s+s1}{3d\PYGZus{}vector}\PYG{l+s+s1}{\PYGZsq{}}\PYG{p}{)} \PYG{c+c1}{\PYGZsh{} show all available 3d\PYGZus{}vector}
\PYG{g+go}{*******************3d\PYGZus{}vector Names List**********************}
\PYG{g+gp}{...}
\PYG{g+go}{*******************End 3d\PYGZus{}vector Names List**********************}
\PYG{g+gp}{\PYGZgt{}\PYGZgt{}\PYGZgt{} }\PYG{n}{ex}\PYG{o}{=}\PYG{n}{vcs}\PYG{o}{.}\PYG{n}{create3d\PYGZus{}vector}\PYG{p}{(}\PYG{l+s+s1}{\PYGZsq{}}\PYG{l+s+s1}{3d\PYGZus{}vector\PYGZus{}ex1}\PYG{l+s+s1}{\PYGZsq{}}\PYG{p}{)} \PYG{c+c1}{\PYGZsh{} Create 3d\PYGZus{}vector \PYGZsq{}3d\PYGZus{}vector\PYGZus{}ex1\PYGZsq{} that inherits from \PYGZsq{}default\PYGZsq{}}
\PYG{g+gp}{\PYGZgt{}\PYGZgt{}\PYGZgt{} }\PYG{n}{vcs}\PYG{o}{.}\PYG{n}{listelements}\PYG{p}{(}\PYG{l+s+s1}{\PYGZsq{}}\PYG{l+s+s1}{3d\PYGZus{}vector}\PYG{l+s+s1}{\PYGZsq{}}\PYG{p}{)} \PYG{c+c1}{\PYGZsh{} should now contain the \PYGZsq{}3d\PYGZus{}vector\PYGZus{}ex1\PYGZsq{} 3d\PYGZus{}vector}
\PYG{g+go}{[...\PYGZsq{}3d\PYGZus{}vector\PYGZus{}ex1\PYGZsq{}...]}
\end{Verbatim}

\item[{Parameters}] \leavevmode\begin{itemize}
\item {} 
\textbf{\texttt{name}} (\href{https://docs.python.org/2/library/functions.html\#str}{\emph{\texttt{str}}}) -- The name of the created object

\item {} 
\textbf{\texttt{source}} (\emph{\texttt{a 3d\_vector or a string name of a 3d\_vector}}) -- The object to inherit from

\end{itemize}

\item[{Returns}] \leavevmode
A 3d\_vector graphics method object

\item[{Return type}] \leavevmode
vcs.dv3d.Gf3Dvector

\end{description}\end{quote}

\end{fulllineitems}

\index{createboxfill() (in module vcs.manageElements)}

\begin{fulllineitems}
\phantomsection\label{vcs/misc/manageElements:vcs.manageElements.createboxfill}\pysiglinewithargsret{\sphinxcode{vcs.manageElements.}\sphinxbfcode{createboxfill}}{\emph{name=None}, \emph{source='default'}}{}
Create a new boxfill graphics method given the the name and the existing
boxfill graphics method to copy the attributes from. If no existing
boxfill graphics method is given, then the default boxfill graphics method will be used as the graphics method
to which the attributes will be copied from.

\begin{notice}{note}{Note:}
If the name provided already exists, then an error will be returned. graphics method
names must be unique.
\end{notice}
\begin{quote}\begin{description}
\item[{Example}] \leavevmode
\begin{Verbatim}[commandchars=\\\{\}]
\PYG{g+gp}{\PYGZgt{}\PYGZgt{}\PYGZgt{} }\PYG{n}{vcs}\PYG{o}{.}\PYG{n}{show}\PYG{p}{(}\PYG{l+s+s1}{\PYGZsq{}}\PYG{l+s+s1}{boxfill}\PYG{l+s+s1}{\PYGZsq{}}\PYG{p}{)} \PYG{c+c1}{\PYGZsh{} show all available boxfill}
\PYG{g+go}{*******************Boxfill Names List**********************}
\PYG{g+gp}{...}
\PYG{g+go}{*******************End Boxfill Names List**********************}
\PYG{g+gp}{\PYGZgt{}\PYGZgt{}\PYGZgt{} }\PYG{n}{ex}\PYG{o}{=}\PYG{n}{vcs}\PYG{o}{.}\PYG{n}{createboxfill}\PYG{p}{(}\PYG{l+s+s1}{\PYGZsq{}}\PYG{l+s+s1}{boxfill\PYGZus{}ex1}\PYG{l+s+s1}{\PYGZsq{}}\PYG{p}{)} \PYG{c+c1}{\PYGZsh{} Create boxfill \PYGZsq{}boxfill\PYGZus{}ex1\PYGZsq{} that inherits from \PYGZsq{}default\PYGZsq{}}
\PYG{g+gp}{\PYGZgt{}\PYGZgt{}\PYGZgt{} }\PYG{n}{vcs}\PYG{o}{.}\PYG{n}{listelements}\PYG{p}{(}\PYG{l+s+s1}{\PYGZsq{}}\PYG{l+s+s1}{boxfill}\PYG{l+s+s1}{\PYGZsq{}}\PYG{p}{)} \PYG{c+c1}{\PYGZsh{} should now contain the \PYGZsq{}boxfill\PYGZus{}ex1\PYGZsq{} boxfill}
\PYG{g+go}{[...\PYGZsq{}boxfill\PYGZus{}ex1\PYGZsq{}...]}
\PYG{g+gp}{\PYGZgt{}\PYGZgt{}\PYGZgt{} }\PYG{n}{ex2}\PYG{o}{=}\PYG{n}{vcs}\PYG{o}{.}\PYG{n}{createboxfill}\PYG{p}{(}\PYG{l+s+s1}{\PYGZsq{}}\PYG{l+s+s1}{boxfill\PYGZus{}ex2}\PYG{l+s+s1}{\PYGZsq{}}\PYG{p}{,}\PYG{l+s+s1}{\PYGZsq{}}\PYG{l+s+s1}{polar}\PYG{l+s+s1}{\PYGZsq{}}\PYG{p}{)} \PYG{c+c1}{\PYGZsh{} create \PYGZsq{}boxfill\PYGZus{}ex2\PYGZsq{} from \PYGZsq{}polar\PYGZsq{} template}
\PYG{g+gp}{\PYGZgt{}\PYGZgt{}\PYGZgt{} }\PYG{n}{vcs}\PYG{o}{.}\PYG{n}{listelements}\PYG{p}{(}\PYG{l+s+s1}{\PYGZsq{}}\PYG{l+s+s1}{boxfill}\PYG{l+s+s1}{\PYGZsq{}}\PYG{p}{)} \PYG{c+c1}{\PYGZsh{} should now contain the \PYGZsq{}boxfill\PYGZus{}ex2\PYGZsq{} boxfill}
\PYG{g+go}{[...\PYGZsq{}boxfill\PYGZus{}ex2\PYGZsq{}...]}
\end{Verbatim}

\item[{Parameters}] \leavevmode\begin{itemize}
\item {} 
\textbf{\texttt{name}} (\href{https://docs.python.org/2/library/functions.html\#str}{\emph{\texttt{str}}}) -- The name of the created object

\item {} 
\textbf{\texttt{source}} (\emph{\texttt{a boxfill or a string name of a boxfill}}) -- The object to inherit from

\item {} 
\textbf{\texttt{xaxis}} (\emph{\texttt{cdms2.axis.TransientAxis}}) -- Axis object to replace the slab -1 dim axis

\item {} 
\textbf{\texttt{yaxis}} (\emph{\texttt{cdms2.axis.TransientAxis}}) -- Axis object to replace the slab -2 dim axis, only if slab has more than 1D

\item {} 
\textbf{\texttt{zaxis}} (\emph{\texttt{cdms2.axis.TransientAxis}}) -- Axis object to replace the slab -3 dim axis, only if slab has more than 2D

\item {} 
\textbf{\texttt{taxis}} (\emph{\texttt{cdms2.axis.TransientAxis}}) -- Axis object to replace the slab -4 dim axis, only if slab has more than 3D

\item {} 
\textbf{\texttt{waxis}} (\emph{\texttt{cdms2.axis.TransientAxis}}) -- Axis object to replace the slab -5 dim axis, only if slab has more than 4D

\item {} 
\textbf{\texttt{xrev}} (\href{https://docs.python.org/2/library/functions.html\#bool}{\emph{\texttt{bool}}}) -- reverse x axis

\item {} 
\textbf{\texttt{yrev}} (\href{https://docs.python.org/2/library/functions.html\#bool}{\emph{\texttt{bool}}}) -- reverse y axis, only if slab has more than 1D

\item {} 
\textbf{\texttt{xarray}} (\href{https://docs.python.org/2/library/array.html\#module-array}{\emph{\texttt{array}}}) -- Values to use instead of x axis

\item {} 
\textbf{\texttt{yarray}} (\href{https://docs.python.org/2/library/array.html\#module-array}{\emph{\texttt{array}}}) -- Values to use instead of y axis, only if var has more than 1D

\item {} 
\textbf{\texttt{zarray}} (\href{https://docs.python.org/2/library/array.html\#module-array}{\emph{\texttt{array}}}) -- Values to use instead of z axis, only if var has more than 2D

\item {} 
\textbf{\texttt{tarray}} (\href{https://docs.python.org/2/library/array.html\#module-array}{\emph{\texttt{array}}}) -- Values to use instead of t axis, only if var has more than 3D

\item {} 
\textbf{\texttt{warray}} (\href{https://docs.python.org/2/library/array.html\#module-array}{\emph{\texttt{array}}}) -- Values to use instead of w axis, only if var has more than 4D

\item {} 
\textbf{\texttt{continents}} (\href{https://docs.python.org/2/library/functions.html\#int}{\emph{\texttt{int}}}) -- continents type number

\item {} 
\textbf{\texttt{name}} -- replaces variable name on plot

\item {} 
\textbf{\texttt{time}} (\emph{\texttt{A cdtime object}}) -- replaces time name on plot

\item {} 
\textbf{\texttt{units}} (\href{https://docs.python.org/2/library/functions.html\#str}{\emph{\texttt{str}}}) -- replaces units value on plot

\item {} 
\textbf{\texttt{ymd}} (\href{https://docs.python.org/2/library/functions.html\#str}{\emph{\texttt{str}}}) -- replaces year/month/day on plot

\item {} 
\textbf{\texttt{hms}} (\href{https://docs.python.org/2/library/functions.html\#str}{\emph{\texttt{str}}}) -- replaces hh/mm/ss on plot

\item {} 
\textbf{\texttt{file\_comment}} (\href{https://docs.python.org/2/library/functions.html\#str}{\emph{\texttt{str}}}) -- replaces file\_comment on plot

\item {} 
\textbf{\texttt{xbounds}} (\href{https://docs.python.org/2/library/array.html\#module-array}{\emph{\texttt{array}}}) -- Values to use instead of x axis bounds values

\item {} 
\textbf{\texttt{ybounds}} (\href{https://docs.python.org/2/library/array.html\#module-array}{\emph{\texttt{array}}}) -- Values to use instead of y axis bounds values (if exist)

\item {} 
\textbf{\texttt{xname}} (\href{https://docs.python.org/2/library/functions.html\#str}{\emph{\texttt{str}}}) -- replace xaxis name on plot

\item {} 
\textbf{\texttt{yname}} (\href{https://docs.python.org/2/library/functions.html\#str}{\emph{\texttt{str}}}) -- replace yaxis name on plot (if exists)

\item {} 
\textbf{\texttt{zname}} (\href{https://docs.python.org/2/library/functions.html\#str}{\emph{\texttt{str}}}) -- replace zaxis name on plot (if exists)

\item {} 
\textbf{\texttt{tname}} (\href{https://docs.python.org/2/library/functions.html\#str}{\emph{\texttt{str}}}) -- replace taxis name on plot (if exists)

\item {} 
\textbf{\texttt{wname}} (\href{https://docs.python.org/2/library/functions.html\#str}{\emph{\texttt{str}}}) -- replace waxis name on plot (if exists)

\item {} 
\textbf{\texttt{xunits}} (\href{https://docs.python.org/2/library/functions.html\#str}{\emph{\texttt{str}}}) -- replace xaxis units on plot

\item {} 
\textbf{\texttt{yunits}} (\href{https://docs.python.org/2/library/functions.html\#str}{\emph{\texttt{str}}}) -- replace yaxis units on plot (if exists)

\item {} 
\textbf{\texttt{zunits}} (\href{https://docs.python.org/2/library/functions.html\#str}{\emph{\texttt{str}}}) -- replace zaxis units on plot (if exists)

\item {} 
\textbf{\texttt{tunits}} (\href{https://docs.python.org/2/library/functions.html\#str}{\emph{\texttt{str}}}) -- replace taxis units on plot (if exists)

\item {} 
\textbf{\texttt{wunits}} (\href{https://docs.python.org/2/library/functions.html\#str}{\emph{\texttt{str}}}) -- replace waxis units on plot (if exists)

\item {} 
\textbf{\texttt{xweights}} (\href{https://docs.python.org/2/library/array.html\#module-array}{\emph{\texttt{array}}}) -- replace xaxis weights used for computing mean

\item {} 
\textbf{\texttt{yweights}} (\href{https://docs.python.org/2/library/array.html\#module-array}{\emph{\texttt{array}}}) -- replace xaxis weights used for computing mean

\item {} 
\textbf{\texttt{comment1}} (\href{https://docs.python.org/2/library/functions.html\#str}{\emph{\texttt{str}}}) -- replaces comment1 on plot

\item {} 
\textbf{\texttt{comment2}} (\href{https://docs.python.org/2/library/functions.html\#str}{\emph{\texttt{str}}}) -- replaces comment2 on plot

\item {} 
\textbf{\texttt{comment3}} (\href{https://docs.python.org/2/library/functions.html\#str}{\emph{\texttt{str}}}) -- replaces comment3 on plot

\item {} 
\textbf{\texttt{comment4}} (\href{https://docs.python.org/2/library/functions.html\#str}{\emph{\texttt{str}}}) -- replaces comment4 on plot

\item {} 
\textbf{\texttt{long\_name}} (\href{https://docs.python.org/2/library/functions.html\#str}{\emph{\texttt{str}}}) -- replaces long\_name on plot

\item {} 
\textbf{\texttt{grid}} (\emph{\texttt{cdms2.grid.TransientRectGrid}}) -- replaces array grid (if exists)

\item {} 
\textbf{\texttt{bg}} (\emph{\texttt{bool/int}}) -- plots in background mode

\item {} 
\textbf{\texttt{ratio}} (\index{xmtics1 (in module vcs.manageElements)}\index{xmtics2 (in module vcs.manageElements)}\index{ymtics1 (in module vcs.manageElements)}\index{ymtics2 (in module vcs.manageElements)}\index{xticlabels1 (in module vcs.manageElements)}\index{xticlabels2 (in module vcs.manageElements)}\index{yticlabels1 (in module vcs.manageElements)}\index{yticlabels2 (in module vcs.manageElements)}\index{projection (in module vcs.manageElements)}\index{datawc\_x1 (in module vcs.manageElements)}\index{datawc\_x2 (in module vcs.manageElements)}\index{datawc\_y1 (in module vcs.manageElements)}\index{datawc\_y2 (in module vcs.manageElements)}\index{datawc\_timeunits (in module vcs.manageElements)}\index{datawc\_calendar (in module vcs.manageElements)}) -- sets the y/x ratio ,if passed as a string with `t' at the end, will aslo moves the ticks

\item {} 
\textbf{\texttt{xaxisconvert}} (\href{https://docs.python.org/2/library/functions.html\#str}{\emph{\texttt{str}}}) -- (Ex: `linear') converting xaxis linear/log/log10/ln/exp/area\_wt

\item {} 
\textbf{\texttt{yaxisconvert}} (\href{https://docs.python.org/2/library/functions.html\#str}{\emph{\texttt{str}}}) -- (Ex: `linear') converting yaxis linear/log/log10/ln/exp/area\_wt

\item {} 
\textbf{\texttt{new\_GM\_name}} (\href{https://docs.python.org/2/library/functions.html\#str}{\emph{\texttt{str}}}) -- (Ex: `my\_awesome\_gm') name of the new graphics method object. If no name is given, then one will be created for use.

\item {} 
\textbf{\texttt{source\_GM\_name}} -- (Ex: `default') copy the contents of the source object to the newly created one. If no name is given, then the `default' graphics methond contents is copied over to the new object.

\end{itemize}

\item[{Returns}] \leavevmode
A boxfill graphics method object

\item[{Return type}] \leavevmode
{\hyperref[vcs/graphics/boxfill:vcs.boxfill.Gfb]{\sphinxcrossref{vcs.boxfill.Gfb}}}

\end{description}\end{quote}

\end{fulllineitems}

\index{createcolormap() (in module vcs.manageElements)}

\begin{fulllineitems}
\phantomsection\label{vcs/misc/manageElements:vcs.manageElements.createcolormap}\pysiglinewithargsret{\sphinxcode{vcs.manageElements.}\sphinxbfcode{createcolormap}}{\emph{Cp\_name=None}, \emph{Cp\_name\_src='default'}}{}
Create a new colormap secondary method given the the name and the existing
colormap secondary method to copy the attributes from. If no existing
colormap secondary method is given, then the default colormap secondary method will be used as the graphics method
to which the attributes will be copied from.

\begin{notice}{note}{Note:}
If the name provided already exists, then an error will be returned. secondary method
names must be unique.
\end{notice}
\begin{quote}\begin{description}
\item[{Example}] \leavevmode
\begin{Verbatim}[commandchars=\\\{\}]
\PYG{g+gp}{\PYGZgt{}\PYGZgt{}\PYGZgt{} }\PYG{n}{vcs}\PYG{o}{.}\PYG{n}{show}\PYG{p}{(}\PYG{l+s+s1}{\PYGZsq{}}\PYG{l+s+s1}{colormap}\PYG{l+s+s1}{\PYGZsq{}}\PYG{p}{)} \PYG{c+c1}{\PYGZsh{} show all available colormap}
\PYG{g+go}{*******************Colormap Names List**********************}
\PYG{g+gp}{...}
\PYG{g+go}{*******************End Colormap Names List**********************}
\PYG{g+gp}{\PYGZgt{}\PYGZgt{}\PYGZgt{} }\PYG{n}{ex}\PYG{o}{=}\PYG{n}{vcs}\PYG{o}{.}\PYG{n}{createcolormap}\PYG{p}{(}\PYG{l+s+s1}{\PYGZsq{}}\PYG{l+s+s1}{colormap\PYGZus{}ex1}\PYG{l+s+s1}{\PYGZsq{}}\PYG{p}{)} \PYG{c+c1}{\PYGZsh{} Create colormap \PYGZsq{}colormap\PYGZus{}ex1\PYGZsq{} that inherits from \PYGZsq{}default\PYGZsq{}}
\PYG{g+gp}{\PYGZgt{}\PYGZgt{}\PYGZgt{} }\PYG{n}{vcs}\PYG{o}{.}\PYG{n}{listelements}\PYG{p}{(}\PYG{l+s+s1}{\PYGZsq{}}\PYG{l+s+s1}{colormap}\PYG{l+s+s1}{\PYGZsq{}}\PYG{p}{)} \PYG{c+c1}{\PYGZsh{} should now contain the \PYGZsq{}colormap\PYGZus{}ex1\PYGZsq{} colormap}
\PYG{g+go}{[...\PYGZsq{}colormap\PYGZus{}ex1\PYGZsq{}...]}
\PYG{g+gp}{\PYGZgt{}\PYGZgt{}\PYGZgt{} }\PYG{n}{ex2}\PYG{o}{=}\PYG{n}{vcs}\PYG{o}{.}\PYG{n}{createcolormap}\PYG{p}{(}\PYG{l+s+s1}{\PYGZsq{}}\PYG{l+s+s1}{colormap\PYGZus{}ex2}\PYG{l+s+s1}{\PYGZsq{}}\PYG{p}{,}\PYG{l+s+s1}{\PYGZsq{}}\PYG{l+s+s1}{rainbow}\PYG{l+s+s1}{\PYGZsq{}}\PYG{p}{)} \PYG{c+c1}{\PYGZsh{} create \PYGZsq{}colormap\PYGZus{}ex2\PYGZsq{} from \PYGZsq{}rainbow\PYGZsq{} template}
\PYG{g+gp}{\PYGZgt{}\PYGZgt{}\PYGZgt{} }\PYG{n}{vcs}\PYG{o}{.}\PYG{n}{listelements}\PYG{p}{(}\PYG{l+s+s1}{\PYGZsq{}}\PYG{l+s+s1}{colormap}\PYG{l+s+s1}{\PYGZsq{}}\PYG{p}{)} \PYG{c+c1}{\PYGZsh{} should now contain the \PYGZsq{}colormap\PYGZus{}ex2\PYGZsq{} colormap}
\PYG{g+go}{[...\PYGZsq{}colormap\PYGZus{}ex2\PYGZsq{}...]}
\end{Verbatim}

\item[{Parameters}] \leavevmode\begin{itemize}
\item {} 
\textbf{\texttt{Cp\_name}} (\href{https://docs.python.org/2/library/functions.html\#str}{\emph{\texttt{str}}}) -- The name of the created object

\item {} 
\textbf{\texttt{Cp\_name\_src}} (\emph{\texttt{a colormap or a string name of a colormap}}) -- The object to inherit

\end{itemize}

\item[{Returns}] \leavevmode
A VCS colormap object

\item[{Return type}] \leavevmode
{\hyperref[vcs/misc/colormap:vcs.colormap.Cp]{\sphinxcrossref{vcs.colormap.Cp}}}

\end{description}\end{quote}

\end{fulllineitems}

\index{createfillarea() (in module vcs.manageElements)}

\begin{fulllineitems}
\phantomsection\label{vcs/misc/manageElements:vcs.manageElements.createfillarea}\pysiglinewithargsret{\sphinxcode{vcs.manageElements.}\sphinxbfcode{createfillarea}}{\emph{name=None}, \emph{source='default'}, \emph{style=None}, \emph{index=None}, \emph{color=None}, \emph{priority=None}, \emph{viewport=None}, \emph{worldcoordinate=None}, \emph{x=None}, \emph{y=None}}{}
Create a new fillarea secondary method given the the name and the existing
fillarea secondary method to copy the attributes from. If no existing
fillarea secondary method is given, then the default fillarea secondary method will be used as the graphics method
to which the attributes will be copied from.

\begin{notice}{note}{Note:}
If the name provided already exists, then an error will be returned. secondary method
names must be unique.
\end{notice}
\begin{quote}\begin{description}
\item[{Example}] \leavevmode
\begin{Verbatim}[commandchars=\\\{\}]
\PYG{g+gp}{\PYGZgt{}\PYGZgt{}\PYGZgt{} }\PYG{n}{vcs}\PYG{o}{.}\PYG{n}{show}\PYG{p}{(}\PYG{l+s+s1}{\PYGZsq{}}\PYG{l+s+s1}{fillarea}\PYG{l+s+s1}{\PYGZsq{}}\PYG{p}{)} \PYG{c+c1}{\PYGZsh{} show all available fillarea}
\PYG{g+go}{*******************Fillarea Names List**********************}
\PYG{g+gp}{...}
\PYG{g+go}{*******************End Fillarea Names List**********************}
\PYG{g+gp}{\PYGZgt{}\PYGZgt{}\PYGZgt{} }\PYG{n}{ex}\PYG{o}{=}\PYG{n}{vcs}\PYG{o}{.}\PYG{n}{createfillarea}\PYG{p}{(}\PYG{l+s+s1}{\PYGZsq{}}\PYG{l+s+s1}{fillarea\PYGZus{}ex1}\PYG{l+s+s1}{\PYGZsq{}}\PYG{p}{)} \PYG{c+c1}{\PYGZsh{} Create fillarea \PYGZsq{}fillarea\PYGZus{}ex1\PYGZsq{} that inherits from \PYGZsq{}default\PYGZsq{}}
\PYG{g+gp}{\PYGZgt{}\PYGZgt{}\PYGZgt{} }\PYG{n}{vcs}\PYG{o}{.}\PYG{n}{listelements}\PYG{p}{(}\PYG{l+s+s1}{\PYGZsq{}}\PYG{l+s+s1}{fillarea}\PYG{l+s+s1}{\PYGZsq{}}\PYG{p}{)} \PYG{c+c1}{\PYGZsh{} should now contain the \PYGZsq{}fillarea\PYGZus{}ex1\PYGZsq{} fillarea}
\PYG{g+go}{[...\PYGZsq{}fillarea\PYGZus{}ex1\PYGZsq{}...]}
\end{Verbatim}

\item[{Parameters}] \leavevmode\begin{itemize}
\item {} 
\textbf{\texttt{name}} (\href{https://docs.python.org/2/library/functions.html\#str}{\emph{\texttt{str}}}) -- Name of created object

\item {} 
\textbf{\texttt{source}} (\href{https://docs.python.org/2/library/functions.html\#str}{\emph{\texttt{str}}}) -- a fillarea, or string name of a fillarea

\item {} 
\textbf{\texttt{style}} (\href{https://docs.python.org/2/library/functions.html\#str}{\emph{\texttt{str}}}) -- One of ``hatch'', ``solid'', or ``pattern''.

\item {} 
\textbf{\texttt{index}} -- Specifies which \href{http://uvcdat.llnl.gov/gallery/fullsize/pattern\_chart.png}{pattern} to fill with.

\end{itemize}

\end{description}\end{quote}

Accepts ints from 1-20.
\begin{quote}\begin{description}
\item[{Parameters}] \leavevmode
\textbf{\texttt{color}} -- A color name from the \href{https://en.wikipedia.org/wiki/X11\_color\_names}{X11 Color Names list},

\end{description}\end{quote}

or an integer value from 0-255, or an RGB/RGBA tuple/list (e.g. (0,100,0), (100,100,0,50))
\begin{quote}\begin{description}
\item[{Parameters}] \leavevmode\begin{itemize}
\item {} 
\textbf{\texttt{priority}} (\href{https://docs.python.org/2/library/functions.html\#int}{\emph{\texttt{int}}}) -- The layer on which the fillarea will be drawn.

\item {} 
\textbf{\texttt{viewport}} (\emph{\texttt{list of floats}}) -- 4 floats between 0 and 1. These specify the area that the X/Y values are mapped to inside of the canvas

\item {} 
\textbf{\texttt{worldcoordinate}} (\emph{\texttt{list of floats}}) -- List of 4 floats (xmin, xmax, ymin, ymax)

\item {} 
\textbf{\texttt{x}} (\emph{\texttt{list of floats}}) -- List of lists of x coordinates. Values must be between worldcoordinate{[}0{]} and worldcoordinate{[}1{]}.

\item {} 
\textbf{\texttt{y}} (\emph{\texttt{list of floats}}) -- List of lists of y coordinates. Values must be between worldcoordinate{[}2{]} and worldcoordinate{[}3{]}.

\end{itemize}

\item[{Returns}] \leavevmode
A fillarea object

\item[{Return type}] \leavevmode
{\hyperref[vcs/secondary/fillarea:vcs.fillarea.Tf]{\sphinxcrossref{vcs.fillarea.Tf}}}

\end{description}\end{quote}

\end{fulllineitems}

\index{createisofill() (in module vcs.manageElements)}

\begin{fulllineitems}
\phantomsection\label{vcs/misc/manageElements:vcs.manageElements.createisofill}\pysiglinewithargsret{\sphinxcode{vcs.manageElements.}\sphinxbfcode{createisofill}}{\emph{name=None}, \emph{source='default'}}{}
Create a new isofill graphics method given the the name and the existing
isofill graphics method to copy the attributes from. If no existing
isofill graphics method is given, then the default isofill graphics method will be used as the graphics method
to which the attributes will be copied from.

\begin{notice}{note}{Note:}
If the name provided already exists, then an error will be returned. graphics method
names must be unique.
\end{notice}
\begin{quote}\begin{description}
\item[{Example}] \leavevmode
\begin{Verbatim}[commandchars=\\\{\}]
\PYG{g+gp}{\PYGZgt{}\PYGZgt{}\PYGZgt{} }\PYG{n}{vcs}\PYG{o}{.}\PYG{n}{show}\PYG{p}{(}\PYG{l+s+s1}{\PYGZsq{}}\PYG{l+s+s1}{isofill}\PYG{l+s+s1}{\PYGZsq{}}\PYG{p}{)} \PYG{c+c1}{\PYGZsh{} show all available isofill}
\PYG{g+go}{*******************Isofill Names List**********************}
\PYG{g+gp}{...}
\PYG{g+go}{*******************End Isofill Names List**********************}
\PYG{g+gp}{\PYGZgt{}\PYGZgt{}\PYGZgt{} }\PYG{n}{ex}\PYG{o}{=}\PYG{n}{vcs}\PYG{o}{.}\PYG{n}{createisofill}\PYG{p}{(}\PYG{l+s+s1}{\PYGZsq{}}\PYG{l+s+s1}{isofill\PYGZus{}ex1}\PYG{l+s+s1}{\PYGZsq{}}\PYG{p}{)} \PYG{c+c1}{\PYGZsh{} Create isofill \PYGZsq{}isofill\PYGZus{}ex1\PYGZsq{} that inherits from \PYGZsq{}default\PYGZsq{}}
\PYG{g+gp}{\PYGZgt{}\PYGZgt{}\PYGZgt{} }\PYG{n}{vcs}\PYG{o}{.}\PYG{n}{listelements}\PYG{p}{(}\PYG{l+s+s1}{\PYGZsq{}}\PYG{l+s+s1}{isofill}\PYG{l+s+s1}{\PYGZsq{}}\PYG{p}{)} \PYG{c+c1}{\PYGZsh{} should now contain the \PYGZsq{}isofill\PYGZus{}ex1\PYGZsq{} isofill}
\PYG{g+go}{[...\PYGZsq{}isofill\PYGZus{}ex1\PYGZsq{}...]}
\PYG{g+gp}{\PYGZgt{}\PYGZgt{}\PYGZgt{} }\PYG{n}{ex2}\PYG{o}{=}\PYG{n}{vcs}\PYG{o}{.}\PYG{n}{createisofill}\PYG{p}{(}\PYG{l+s+s1}{\PYGZsq{}}\PYG{l+s+s1}{isofill\PYGZus{}ex2}\PYG{l+s+s1}{\PYGZsq{}}\PYG{p}{,}\PYG{l+s+s1}{\PYGZsq{}}\PYG{l+s+s1}{polar}\PYG{l+s+s1}{\PYGZsq{}}\PYG{p}{)} \PYG{c+c1}{\PYGZsh{} create \PYGZsq{}isofill\PYGZus{}ex2\PYGZsq{} from \PYGZsq{}polar\PYGZsq{} template}
\PYG{g+gp}{\PYGZgt{}\PYGZgt{}\PYGZgt{} }\PYG{n}{vcs}\PYG{o}{.}\PYG{n}{listelements}\PYG{p}{(}\PYG{l+s+s1}{\PYGZsq{}}\PYG{l+s+s1}{isofill}\PYG{l+s+s1}{\PYGZsq{}}\PYG{p}{)} \PYG{c+c1}{\PYGZsh{} should now contain the \PYGZsq{}isofill\PYGZus{}ex2\PYGZsq{} isofill}
\PYG{g+go}{[...\PYGZsq{}isofill\PYGZus{}ex2\PYGZsq{}...]}
\end{Verbatim}

\item[{Parameters}] \leavevmode\begin{itemize}
\item {} 
\textbf{\texttt{name}} (\href{https://docs.python.org/2/library/functions.html\#str}{\emph{\texttt{str}}}) -- The name of the created object

\item {} 
\textbf{\texttt{source}} (\emph{\texttt{an isofill object, or string name of an isofill object}}) -- The object to inherit from

\item {} 
\textbf{\texttt{xaxis}} (\emph{\texttt{cdms2.axis.TransientAxis}}) -- Axis object to replace the slab -1 dim axis

\item {} 
\textbf{\texttt{yaxis}} (\emph{\texttt{cdms2.axis.TransientAxis}}) -- Axis object to replace the slab -2 dim axis, only if slab has more than 1D

\item {} 
\textbf{\texttt{zaxis}} (\emph{\texttt{cdms2.axis.TransientAxis}}) -- Axis object to replace the slab -3 dim axis, only if slab has more than 2D

\item {} 
\textbf{\texttt{taxis}} (\emph{\texttt{cdms2.axis.TransientAxis}}) -- Axis object to replace the slab -4 dim axis, only if slab has more than 3D

\item {} 
\textbf{\texttt{waxis}} (\emph{\texttt{cdms2.axis.TransientAxis}}) -- Axis object to replace the slab -5 dim axis, only if slab has more than 4D

\item {} 
\textbf{\texttt{xrev}} (\href{https://docs.python.org/2/library/functions.html\#bool}{\emph{\texttt{bool}}}) -- reverse x axis

\item {} 
\textbf{\texttt{yrev}} (\href{https://docs.python.org/2/library/functions.html\#bool}{\emph{\texttt{bool}}}) -- reverse y axis, only if slab has more than 1D

\item {} 
\textbf{\texttt{xarray}} (\href{https://docs.python.org/2/library/array.html\#module-array}{\emph{\texttt{array}}}) -- Values to use instead of x axis

\item {} 
\textbf{\texttt{yarray}} (\href{https://docs.python.org/2/library/array.html\#module-array}{\emph{\texttt{array}}}) -- Values to use instead of y axis, only if var has more than 1D

\item {} 
\textbf{\texttt{zarray}} (\href{https://docs.python.org/2/library/array.html\#module-array}{\emph{\texttt{array}}}) -- Values to use instead of z axis, only if var has more than 2D

\item {} 
\textbf{\texttt{tarray}} (\href{https://docs.python.org/2/library/array.html\#module-array}{\emph{\texttt{array}}}) -- Values to use instead of t axis, only if var has more than 3D

\item {} 
\textbf{\texttt{warray}} (\href{https://docs.python.org/2/library/array.html\#module-array}{\emph{\texttt{array}}}) -- Values to use instead of w axis, only if var has more than 4D

\item {} 
\textbf{\texttt{continents}} (\href{https://docs.python.org/2/library/functions.html\#int}{\emph{\texttt{int}}}) -- continents type number

\item {} 
\textbf{\texttt{name}} -- replaces variable name on plot

\item {} 
\textbf{\texttt{time}} (\emph{\texttt{A cdtime object}}) -- replaces time name on plot

\item {} 
\textbf{\texttt{units}} (\href{https://docs.python.org/2/library/functions.html\#str}{\emph{\texttt{str}}}) -- replaces units value on plot

\item {} 
\textbf{\texttt{ymd}} (\href{https://docs.python.org/2/library/functions.html\#str}{\emph{\texttt{str}}}) -- replaces year/month/day on plot

\item {} 
\textbf{\texttt{hms}} (\href{https://docs.python.org/2/library/functions.html\#str}{\emph{\texttt{str}}}) -- replaces hh/mm/ss on plot

\item {} 
\textbf{\texttt{file\_comment}} (\href{https://docs.python.org/2/library/functions.html\#str}{\emph{\texttt{str}}}) -- replaces file\_comment on plot

\item {} 
\textbf{\texttt{xbounds}} (\href{https://docs.python.org/2/library/array.html\#module-array}{\emph{\texttt{array}}}) -- Values to use instead of x axis bounds values

\item {} 
\textbf{\texttt{ybounds}} (\href{https://docs.python.org/2/library/array.html\#module-array}{\emph{\texttt{array}}}) -- Values to use instead of y axis bounds values (if exist)

\item {} 
\textbf{\texttt{xname}} (\href{https://docs.python.org/2/library/functions.html\#str}{\emph{\texttt{str}}}) -- replace xaxis name on plot

\item {} 
\textbf{\texttt{yname}} (\href{https://docs.python.org/2/library/functions.html\#str}{\emph{\texttt{str}}}) -- replace yaxis name on plot (if exists)

\item {} 
\textbf{\texttt{zname}} (\href{https://docs.python.org/2/library/functions.html\#str}{\emph{\texttt{str}}}) -- replace zaxis name on plot (if exists)

\item {} 
\textbf{\texttt{tname}} (\href{https://docs.python.org/2/library/functions.html\#str}{\emph{\texttt{str}}}) -- replace taxis name on plot (if exists)

\item {} 
\textbf{\texttt{wname}} (\href{https://docs.python.org/2/library/functions.html\#str}{\emph{\texttt{str}}}) -- replace waxis name on plot (if exists)

\item {} 
\textbf{\texttt{xunits}} (\href{https://docs.python.org/2/library/functions.html\#str}{\emph{\texttt{str}}}) -- replace xaxis units on plot

\item {} 
\textbf{\texttt{yunits}} (\href{https://docs.python.org/2/library/functions.html\#str}{\emph{\texttt{str}}}) -- replace yaxis units on plot (if exists)

\item {} 
\textbf{\texttt{zunits}} (\href{https://docs.python.org/2/library/functions.html\#str}{\emph{\texttt{str}}}) -- replace zaxis units on plot (if exists)

\item {} 
\textbf{\texttt{tunits}} (\href{https://docs.python.org/2/library/functions.html\#str}{\emph{\texttt{str}}}) -- replace taxis units on plot (if exists)

\item {} 
\textbf{\texttt{wunits}} (\href{https://docs.python.org/2/library/functions.html\#str}{\emph{\texttt{str}}}) -- replace waxis units on plot (if exists)

\item {} 
\textbf{\texttt{xweights}} (\href{https://docs.python.org/2/library/array.html\#module-array}{\emph{\texttt{array}}}) -- replace xaxis weights used for computing mean

\item {} 
\textbf{\texttt{yweights}} (\href{https://docs.python.org/2/library/array.html\#module-array}{\emph{\texttt{array}}}) -- replace xaxis weights used for computing mean

\item {} 
\textbf{\texttt{comment1}} (\href{https://docs.python.org/2/library/functions.html\#str}{\emph{\texttt{str}}}) -- replaces comment1 on plot

\item {} 
\textbf{\texttt{comment2}} (\href{https://docs.python.org/2/library/functions.html\#str}{\emph{\texttt{str}}}) -- replaces comment2 on plot

\item {} 
\textbf{\texttt{comment3}} (\href{https://docs.python.org/2/library/functions.html\#str}{\emph{\texttt{str}}}) -- replaces comment3 on plot

\item {} 
\textbf{\texttt{comment4}} (\href{https://docs.python.org/2/library/functions.html\#str}{\emph{\texttt{str}}}) -- replaces comment4 on plot

\item {} 
\textbf{\texttt{long\_name}} (\href{https://docs.python.org/2/library/functions.html\#str}{\emph{\texttt{str}}}) -- replaces long\_name on plot

\item {} 
\textbf{\texttt{grid}} (\emph{\texttt{cdms2.grid.TransientRectGrid}}) -- replaces array grid (if exists)

\item {} 
\textbf{\texttt{bg}} (\emph{\texttt{bool/int}}) -- plots in background mode

\item {} 
\textbf{\texttt{ratio}} (\index{xmtics1 (in module vcs.manageElements)}\index{xmtics2 (in module vcs.manageElements)}\index{ymtics1 (in module vcs.manageElements)}\index{ymtics2 (in module vcs.manageElements)}\index{xticlabels1 (in module vcs.manageElements)}\index{xticlabels2 (in module vcs.manageElements)}\index{yticlabels1 (in module vcs.manageElements)}\index{yticlabels2 (in module vcs.manageElements)}\index{projection (in module vcs.manageElements)}\index{datawc\_x1 (in module vcs.manageElements)}\index{datawc\_x2 (in module vcs.manageElements)}\index{datawc\_y1 (in module vcs.manageElements)}\index{datawc\_y2 (in module vcs.manageElements)}\index{datawc\_timeunits (in module vcs.manageElements)}\index{datawc\_calendar (in module vcs.manageElements)}) -- sets the y/x ratio ,if passed as a string with `t' at the end, will aslo moves the ticks

\item {} 
\textbf{\texttt{xaxisconvert}} (\href{https://docs.python.org/2/library/functions.html\#str}{\emph{\texttt{str}}}) -- (Ex: `linear') converting xaxis linear/log/log10/ln/exp/area\_wt

\item {} 
\textbf{\texttt{yaxisconvert}} (\href{https://docs.python.org/2/library/functions.html\#str}{\emph{\texttt{str}}}) -- (Ex: `linear') converting yaxis linear/log/log10/ln/exp/area\_wt

\item {} 
\textbf{\texttt{new\_GM\_name}} (\href{https://docs.python.org/2/library/functions.html\#str}{\emph{\texttt{str}}}) -- (Ex: `my\_awesome\_gm') name of the new graphics method object. If no name is given, then one will be created for use.

\item {} 
\textbf{\texttt{source\_GM\_name}} -- (Ex: `default') copy the contents of the source object to the newly created one. If no name is given, then the `default' graphics methond contents is copied over to the new object.

\end{itemize}

\item[{Returns}] \leavevmode
An isofill graphics method

\item[{Return type}] \leavevmode
{\hyperref[vcs/graphics/isofill:vcs.isofill.Gfi]{\sphinxcrossref{vcs.isofill.Gfi}}}

\end{description}\end{quote}

\end{fulllineitems}

\index{createisoline() (in module vcs.manageElements)}

\begin{fulllineitems}
\phantomsection\label{vcs/misc/manageElements:vcs.manageElements.createisoline}\pysiglinewithargsret{\sphinxcode{vcs.manageElements.}\sphinxbfcode{createisoline}}{\emph{name=None}, \emph{source='default'}}{}
Create a new isoline graphics method given the the name and the existing
isoline graphics method to copy the attributes from. If no existing
isoline graphics method is given, then the default isoline graphics method will be used as the graphics method
to which the attributes will be copied from.

\begin{notice}{note}{Note:}
If the name provided already exists, then an error will be returned. graphics method
names must be unique.
\end{notice}
\begin{quote}\begin{description}
\item[{Example}] \leavevmode
\begin{Verbatim}[commandchars=\\\{\}]
\PYG{g+gp}{\PYGZgt{}\PYGZgt{}\PYGZgt{} }\PYG{n}{vcs}\PYG{o}{.}\PYG{n}{show}\PYG{p}{(}\PYG{l+s+s1}{\PYGZsq{}}\PYG{l+s+s1}{isoline}\PYG{l+s+s1}{\PYGZsq{}}\PYG{p}{)} \PYG{c+c1}{\PYGZsh{} show all available isoline}
\PYG{g+go}{*******************Isoline Names List**********************}
\PYG{g+gp}{...}
\PYG{g+go}{*******************End Isoline Names List**********************}
\PYG{g+gp}{\PYGZgt{}\PYGZgt{}\PYGZgt{} }\PYG{n}{ex}\PYG{o}{=}\PYG{n}{vcs}\PYG{o}{.}\PYG{n}{createisoline}\PYG{p}{(}\PYG{l+s+s1}{\PYGZsq{}}\PYG{l+s+s1}{isoline\PYGZus{}ex1}\PYG{l+s+s1}{\PYGZsq{}}\PYG{p}{)} \PYG{c+c1}{\PYGZsh{} Create isoline \PYGZsq{}isoline\PYGZus{}ex1\PYGZsq{} that inherits from \PYGZsq{}default\PYGZsq{}}
\PYG{g+gp}{\PYGZgt{}\PYGZgt{}\PYGZgt{} }\PYG{n}{vcs}\PYG{o}{.}\PYG{n}{listelements}\PYG{p}{(}\PYG{l+s+s1}{\PYGZsq{}}\PYG{l+s+s1}{isoline}\PYG{l+s+s1}{\PYGZsq{}}\PYG{p}{)} \PYG{c+c1}{\PYGZsh{} should now contain the \PYGZsq{}isoline\PYGZus{}ex1\PYGZsq{} isoline}
\PYG{g+go}{[...\PYGZsq{}isoline\PYGZus{}ex1\PYGZsq{}...]}
\PYG{g+gp}{\PYGZgt{}\PYGZgt{}\PYGZgt{} }\PYG{n}{ex2}\PYG{o}{=}\PYG{n}{vcs}\PYG{o}{.}\PYG{n}{createisoline}\PYG{p}{(}\PYG{l+s+s1}{\PYGZsq{}}\PYG{l+s+s1}{isoline\PYGZus{}ex2}\PYG{l+s+s1}{\PYGZsq{}}\PYG{p}{,}\PYG{l+s+s1}{\PYGZsq{}}\PYG{l+s+s1}{polar}\PYG{l+s+s1}{\PYGZsq{}}\PYG{p}{)} \PYG{c+c1}{\PYGZsh{} create \PYGZsq{}isoline\PYGZus{}ex2\PYGZsq{} from \PYGZsq{}polar\PYGZsq{} template}
\PYG{g+gp}{\PYGZgt{}\PYGZgt{}\PYGZgt{} }\PYG{n}{vcs}\PYG{o}{.}\PYG{n}{listelements}\PYG{p}{(}\PYG{l+s+s1}{\PYGZsq{}}\PYG{l+s+s1}{isoline}\PYG{l+s+s1}{\PYGZsq{}}\PYG{p}{)} \PYG{c+c1}{\PYGZsh{} should now contain the \PYGZsq{}isoline\PYGZus{}ex2\PYGZsq{} isoline}
\PYG{g+go}{[...\PYGZsq{}isoline\PYGZus{}ex2\PYGZsq{}...]}
\end{Verbatim}

\item[{Parameters}] \leavevmode\begin{itemize}
\item {} 
\textbf{\texttt{name}} (\href{https://docs.python.org/2/library/functions.html\#str}{\emph{\texttt{str}}}) -- The name of the created object

\item {} 
\textbf{\texttt{source}} (\emph{\texttt{an isoline object, or string name of an isoline object}}) -- The object to inherit from

\item {} 
\textbf{\texttt{xaxis}} (\emph{\texttt{cdms2.axis.TransientAxis}}) -- Axis object to replace the slab -1 dim axis

\item {} 
\textbf{\texttt{yaxis}} (\emph{\texttt{cdms2.axis.TransientAxis}}) -- Axis object to replace the slab -2 dim axis, only if slab has more than 1D

\item {} 
\textbf{\texttt{zaxis}} (\emph{\texttt{cdms2.axis.TransientAxis}}) -- Axis object to replace the slab -3 dim axis, only if slab has more than 2D

\item {} 
\textbf{\texttt{taxis}} (\emph{\texttt{cdms2.axis.TransientAxis}}) -- Axis object to replace the slab -4 dim axis, only if slab has more than 3D

\item {} 
\textbf{\texttt{waxis}} (\emph{\texttt{cdms2.axis.TransientAxis}}) -- Axis object to replace the slab -5 dim axis, only if slab has more than 4D

\item {} 
\textbf{\texttt{xrev}} (\href{https://docs.python.org/2/library/functions.html\#bool}{\emph{\texttt{bool}}}) -- reverse x axis

\item {} 
\textbf{\texttt{yrev}} (\href{https://docs.python.org/2/library/functions.html\#bool}{\emph{\texttt{bool}}}) -- reverse y axis, only if slab has more than 1D

\item {} 
\textbf{\texttt{xarray}} (\href{https://docs.python.org/2/library/array.html\#module-array}{\emph{\texttt{array}}}) -- Values to use instead of x axis

\item {} 
\textbf{\texttt{yarray}} (\href{https://docs.python.org/2/library/array.html\#module-array}{\emph{\texttt{array}}}) -- Values to use instead of y axis, only if var has more than 1D

\item {} 
\textbf{\texttt{zarray}} (\href{https://docs.python.org/2/library/array.html\#module-array}{\emph{\texttt{array}}}) -- Values to use instead of z axis, only if var has more than 2D

\item {} 
\textbf{\texttt{tarray}} (\href{https://docs.python.org/2/library/array.html\#module-array}{\emph{\texttt{array}}}) -- Values to use instead of t axis, only if var has more than 3D

\item {} 
\textbf{\texttt{warray}} (\href{https://docs.python.org/2/library/array.html\#module-array}{\emph{\texttt{array}}}) -- Values to use instead of w axis, only if var has more than 4D

\item {} 
\textbf{\texttt{continents}} (\href{https://docs.python.org/2/library/functions.html\#int}{\emph{\texttt{int}}}) -- continents type number

\item {} 
\textbf{\texttt{name}} -- replaces variable name on plot

\item {} 
\textbf{\texttt{time}} (\emph{\texttt{A cdtime object}}) -- replaces time name on plot

\item {} 
\textbf{\texttt{units}} (\href{https://docs.python.org/2/library/functions.html\#str}{\emph{\texttt{str}}}) -- replaces units value on plot

\item {} 
\textbf{\texttt{ymd}} (\href{https://docs.python.org/2/library/functions.html\#str}{\emph{\texttt{str}}}) -- replaces year/month/day on plot

\item {} 
\textbf{\texttt{hms}} (\href{https://docs.python.org/2/library/functions.html\#str}{\emph{\texttt{str}}}) -- replaces hh/mm/ss on plot

\item {} 
\textbf{\texttt{file\_comment}} (\href{https://docs.python.org/2/library/functions.html\#str}{\emph{\texttt{str}}}) -- replaces file\_comment on plot

\item {} 
\textbf{\texttt{xbounds}} (\href{https://docs.python.org/2/library/array.html\#module-array}{\emph{\texttt{array}}}) -- Values to use instead of x axis bounds values

\item {} 
\textbf{\texttt{ybounds}} (\href{https://docs.python.org/2/library/array.html\#module-array}{\emph{\texttt{array}}}) -- Values to use instead of y axis bounds values (if exist)

\item {} 
\textbf{\texttt{xname}} (\href{https://docs.python.org/2/library/functions.html\#str}{\emph{\texttt{str}}}) -- replace xaxis name on plot

\item {} 
\textbf{\texttt{yname}} (\href{https://docs.python.org/2/library/functions.html\#str}{\emph{\texttt{str}}}) -- replace yaxis name on plot (if exists)

\item {} 
\textbf{\texttt{zname}} (\href{https://docs.python.org/2/library/functions.html\#str}{\emph{\texttt{str}}}) -- replace zaxis name on plot (if exists)

\item {} 
\textbf{\texttt{tname}} (\href{https://docs.python.org/2/library/functions.html\#str}{\emph{\texttt{str}}}) -- replace taxis name on plot (if exists)

\item {} 
\textbf{\texttt{wname}} (\href{https://docs.python.org/2/library/functions.html\#str}{\emph{\texttt{str}}}) -- replace waxis name on plot (if exists)

\item {} 
\textbf{\texttt{xunits}} (\href{https://docs.python.org/2/library/functions.html\#str}{\emph{\texttt{str}}}) -- replace xaxis units on plot

\item {} 
\textbf{\texttt{yunits}} (\href{https://docs.python.org/2/library/functions.html\#str}{\emph{\texttt{str}}}) -- replace yaxis units on plot (if exists)

\item {} 
\textbf{\texttt{zunits}} (\href{https://docs.python.org/2/library/functions.html\#str}{\emph{\texttt{str}}}) -- replace zaxis units on plot (if exists)

\item {} 
\textbf{\texttt{tunits}} (\href{https://docs.python.org/2/library/functions.html\#str}{\emph{\texttt{str}}}) -- replace taxis units on plot (if exists)

\item {} 
\textbf{\texttt{wunits}} (\href{https://docs.python.org/2/library/functions.html\#str}{\emph{\texttt{str}}}) -- replace waxis units on plot (if exists)

\item {} 
\textbf{\texttt{xweights}} (\href{https://docs.python.org/2/library/array.html\#module-array}{\emph{\texttt{array}}}) -- replace xaxis weights used for computing mean

\item {} 
\textbf{\texttt{yweights}} (\href{https://docs.python.org/2/library/array.html\#module-array}{\emph{\texttt{array}}}) -- replace xaxis weights used for computing mean

\item {} 
\textbf{\texttt{comment1}} (\href{https://docs.python.org/2/library/functions.html\#str}{\emph{\texttt{str}}}) -- replaces comment1 on plot

\item {} 
\textbf{\texttt{comment2}} (\href{https://docs.python.org/2/library/functions.html\#str}{\emph{\texttt{str}}}) -- replaces comment2 on plot

\item {} 
\textbf{\texttt{comment3}} (\href{https://docs.python.org/2/library/functions.html\#str}{\emph{\texttt{str}}}) -- replaces comment3 on plot

\item {} 
\textbf{\texttt{comment4}} (\href{https://docs.python.org/2/library/functions.html\#str}{\emph{\texttt{str}}}) -- replaces comment4 on plot

\item {} 
\textbf{\texttt{long\_name}} (\href{https://docs.python.org/2/library/functions.html\#str}{\emph{\texttt{str}}}) -- replaces long\_name on plot

\item {} 
\textbf{\texttt{grid}} (\emph{\texttt{cdms2.grid.TransientRectGrid}}) -- replaces array grid (if exists)

\item {} 
\textbf{\texttt{bg}} (\emph{\texttt{bool/int}}) -- plots in background mode

\item {} 
\textbf{\texttt{ratio}} (\index{xmtics1 (in module vcs.manageElements)}\index{xmtics2 (in module vcs.manageElements)}\index{ymtics1 (in module vcs.manageElements)}\index{ymtics2 (in module vcs.manageElements)}\index{xticlabels1 (in module vcs.manageElements)}\index{xticlabels2 (in module vcs.manageElements)}\index{yticlabels1 (in module vcs.manageElements)}\index{yticlabels2 (in module vcs.manageElements)}\index{projection (in module vcs.manageElements)}\index{datawc\_x1 (in module vcs.manageElements)}\index{datawc\_x2 (in module vcs.manageElements)}\index{datawc\_y1 (in module vcs.manageElements)}\index{datawc\_y2 (in module vcs.manageElements)}\index{datawc\_timeunits (in module vcs.manageElements)}\index{datawc\_calendar (in module vcs.manageElements)}) -- sets the y/x ratio ,if passed as a string with `t' at the end, will aslo moves the ticks

\item {} 
\textbf{\texttt{xaxisconvert}} (\href{https://docs.python.org/2/library/functions.html\#str}{\emph{\texttt{str}}}) -- (Ex: `linear') converting xaxis linear/log/log10/ln/exp/area\_wt

\item {} 
\textbf{\texttt{yaxisconvert}} (\href{https://docs.python.org/2/library/functions.html\#str}{\emph{\texttt{str}}}) -- (Ex: `linear') converting yaxis linear/log/log10/ln/exp/area\_wt

\item {} 
\textbf{\texttt{new\_GM\_name}} (\href{https://docs.python.org/2/library/functions.html\#str}{\emph{\texttt{str}}}) -- (Ex: `my\_awesome\_gm') name of the new graphics method object. If no name is given, then one will be created for use.

\item {} 
\textbf{\texttt{source\_GM\_name}} -- (Ex: `default') copy the contents of the source object to the newly created one. If no name is given, then the `default' graphics methond contents is copied over to the new object.

\end{itemize}

\item[{Returns}] \leavevmode
An isoline graphics method object

\item[{Return type}] \leavevmode
{\hyperref[vcs/graphics/isoline:vcs.isoline.Gi]{\sphinxcrossref{vcs.isoline.Gi}}}

\end{description}\end{quote}

\end{fulllineitems}

\index{createline() (in module vcs.manageElements)}

\begin{fulllineitems}
\phantomsection\label{vcs/misc/manageElements:vcs.manageElements.createline}\pysiglinewithargsret{\sphinxcode{vcs.manageElements.}\sphinxbfcode{createline}}{\emph{name=None}, \emph{source='default'}, \emph{ltype=None}, \emph{width=None}, \emph{color=None}, \emph{priority=None}, \emph{viewport=None}, \emph{worldcoordinate=None}, \emph{x=None}, \emph{y=None}, \emph{projection=None}}{}
Create a new line secondary method given the the name and the existing
line secondary method to copy the attributes from. If no existing
line secondary method is given, then the default line secondary method will be used as the graphics method
to which the attributes will be copied from.

\begin{notice}{note}{Note:}
If the name provided already exists, then an error will be returned. secondary method
names must be unique.
\end{notice}
\begin{quote}\begin{description}
\item[{Example}] \leavevmode
\begin{Verbatim}[commandchars=\\\{\}]
\PYG{g+gp}{\PYGZgt{}\PYGZgt{}\PYGZgt{} }\PYG{n}{vcs}\PYG{o}{.}\PYG{n}{show}\PYG{p}{(}\PYG{l+s+s1}{\PYGZsq{}}\PYG{l+s+s1}{line}\PYG{l+s+s1}{\PYGZsq{}}\PYG{p}{)} \PYG{c+c1}{\PYGZsh{} show all available line}
\PYG{g+go}{*******************Line Names List**********************}
\PYG{g+gp}{...}
\PYG{g+go}{*******************End Line Names List**********************}
\PYG{g+gp}{\PYGZgt{}\PYGZgt{}\PYGZgt{} }\PYG{n}{ex}\PYG{o}{=}\PYG{n}{vcs}\PYG{o}{.}\PYG{n}{createline}\PYG{p}{(}\PYG{l+s+s1}{\PYGZsq{}}\PYG{l+s+s1}{line\PYGZus{}ex1}\PYG{l+s+s1}{\PYGZsq{}}\PYG{p}{)} \PYG{c+c1}{\PYGZsh{} Create line \PYGZsq{}line\PYGZus{}ex1\PYGZsq{} that inherits from \PYGZsq{}default\PYGZsq{}}
\PYG{g+gp}{\PYGZgt{}\PYGZgt{}\PYGZgt{} }\PYG{n}{vcs}\PYG{o}{.}\PYG{n}{listelements}\PYG{p}{(}\PYG{l+s+s1}{\PYGZsq{}}\PYG{l+s+s1}{line}\PYG{l+s+s1}{\PYGZsq{}}\PYG{p}{)} \PYG{c+c1}{\PYGZsh{} should now contain the \PYGZsq{}line\PYGZus{}ex1\PYGZsq{} line}
\PYG{g+go}{[...\PYGZsq{}line\PYGZus{}ex1\PYGZsq{}...]}
\PYG{g+gp}{\PYGZgt{}\PYGZgt{}\PYGZgt{} }\PYG{n}{ex2}\PYG{o}{=}\PYG{n}{vcs}\PYG{o}{.}\PYG{n}{createline}\PYG{p}{(}\PYG{l+s+s1}{\PYGZsq{}}\PYG{l+s+s1}{line\PYGZus{}ex2}\PYG{l+s+s1}{\PYGZsq{}}\PYG{p}{,}\PYG{l+s+s1}{\PYGZsq{}}\PYG{l+s+s1}{red}\PYG{l+s+s1}{\PYGZsq{}}\PYG{p}{)} \PYG{c+c1}{\PYGZsh{} create \PYGZsq{}line\PYGZus{}ex2\PYGZsq{} from \PYGZsq{}red\PYGZsq{} template}
\PYG{g+gp}{\PYGZgt{}\PYGZgt{}\PYGZgt{} }\PYG{n}{vcs}\PYG{o}{.}\PYG{n}{listelements}\PYG{p}{(}\PYG{l+s+s1}{\PYGZsq{}}\PYG{l+s+s1}{line}\PYG{l+s+s1}{\PYGZsq{}}\PYG{p}{)} \PYG{c+c1}{\PYGZsh{} should now contain the \PYGZsq{}line\PYGZus{}ex2\PYGZsq{} line}
\PYG{g+go}{[...\PYGZsq{}line\PYGZus{}ex2\PYGZsq{}...]}
\end{Verbatim}

\item[{Parameters}] \leavevmode\begin{itemize}
\item {} 
\textbf{\texttt{name}} (\href{https://docs.python.org/2/library/functions.html\#str}{\emph{\texttt{str}}}) -- Name of created object

\item {} 
\textbf{\texttt{source}} (\href{https://docs.python.org/2/library/functions.html\#str}{\emph{\texttt{str}}}) -- a line, or string name of a line

\item {} 
\textbf{\texttt{ltype}} (\href{https://docs.python.org/2/library/functions.html\#str}{\emph{\texttt{str}}}) -- One of ``dash'', ``dash-dot'', ``solid'', ``dot'', or ``long-dash''.

\item {} 
\textbf{\texttt{width}} (\href{https://docs.python.org/2/library/functions.html\#int}{\emph{\texttt{int}}}) -- Thickness of the line to be created

\item {} 
\textbf{\texttt{color}} (\emph{\texttt{str or int}}) -- 
A color name from the \href{https://en.wikipedia.org/wiki/X11\_color\_names}{X11 Color Names list},
or an integer value from 0-255, or an RGB/RGBA tuple/list (e.g. (0,100,0), (100,100,0,50))


\item {} 
\textbf{\texttt{priority}} (\href{https://docs.python.org/2/library/functions.html\#int}{\emph{\texttt{int}}}) -- The layer on which the line will be drawn.

\item {} 
\textbf{\texttt{viewport}} (\emph{\texttt{list of floats}}) -- 4 floats between 0 and 1. These specify the area that the X/Y values are mapped to inside of the canvas

\item {} 
\textbf{\texttt{worldcoordinate}} (\emph{\texttt{list of floats}}) -- List of 4 floats (xmin, xmax, ymin, ymax)

\item {} 
\textbf{\texttt{x}} (\emph{\texttt{list of floats}}) -- List of lists of x coordinates. Values must be between worldcoordinate{[}0{]} and worldcoordinate{[}1{]}.

\item {} 
\textbf{\texttt{y}} (\emph{\texttt{list of floats}}) -- List of lists of y coordinates. Values must be between worldcoordinate{[}2{]} and worldcoordinate{[}3{]}.

\item {} 
\textbf{\texttt{projection}} (\emph{\texttt{str or projection object}}) -- Specify a geographic projection used to convert x/y from spherical coordinates into 2D coordinates.

\end{itemize}

\item[{Returns}] \leavevmode
A VCS line secondary method object

\item[{Return type}] \leavevmode
{\hyperref[vcs/secondary/line:vcs.line.Tl]{\sphinxcrossref{vcs.line.Tl}}}

\end{description}\end{quote}

\end{fulllineitems}

\index{createmarker() (in module vcs.manageElements)}

\begin{fulllineitems}
\phantomsection\label{vcs/misc/manageElements:vcs.manageElements.createmarker}\pysiglinewithargsret{\sphinxcode{vcs.manageElements.}\sphinxbfcode{createmarker}}{\emph{name=None}, \emph{source='default'}, \emph{mtype=None}, \emph{size=None}, \emph{color=None}, \emph{priority=None}, \emph{viewport=None}, \emph{worldcoordinate=None}, \emph{x=None}, \emph{y=None}, \emph{projection=None}}{}
Create a new marker secondary method given the the name and the existing
marker secondary method to copy the attributes from. If no existing
marker secondary method is given, then the default marker secondary method will be used as the graphics method
to which the attributes will be copied from.

\begin{notice}{note}{Note:}
If the name provided already exists, then an error will be returned. secondary method
names must be unique.
\end{notice}
\begin{quote}\begin{description}
\item[{Example}] \leavevmode
\begin{Verbatim}[commandchars=\\\{\}]
\PYG{g+gp}{\PYGZgt{}\PYGZgt{}\PYGZgt{} }\PYG{n}{vcs}\PYG{o}{.}\PYG{n}{show}\PYG{p}{(}\PYG{l+s+s1}{\PYGZsq{}}\PYG{l+s+s1}{marker}\PYG{l+s+s1}{\PYGZsq{}}\PYG{p}{)} \PYG{c+c1}{\PYGZsh{} show all available marker}
\PYG{g+go}{*******************Marker Names List**********************}
\PYG{g+gp}{...}
\PYG{g+go}{*******************End Marker Names List**********************}
\PYG{g+gp}{\PYGZgt{}\PYGZgt{}\PYGZgt{} }\PYG{n}{ex}\PYG{o}{=}\PYG{n}{vcs}\PYG{o}{.}\PYG{n}{createmarker}\PYG{p}{(}\PYG{l+s+s1}{\PYGZsq{}}\PYG{l+s+s1}{marker\PYGZus{}ex1}\PYG{l+s+s1}{\PYGZsq{}}\PYG{p}{)} \PYG{c+c1}{\PYGZsh{} Create marker \PYGZsq{}marker\PYGZus{}ex1\PYGZsq{} that inherits from \PYGZsq{}default\PYGZsq{}}
\PYG{g+gp}{\PYGZgt{}\PYGZgt{}\PYGZgt{} }\PYG{n}{vcs}\PYG{o}{.}\PYG{n}{listelements}\PYG{p}{(}\PYG{l+s+s1}{\PYGZsq{}}\PYG{l+s+s1}{marker}\PYG{l+s+s1}{\PYGZsq{}}\PYG{p}{)} \PYG{c+c1}{\PYGZsh{} should now contain the \PYGZsq{}marker\PYGZus{}ex1\PYGZsq{} marker}
\PYG{g+go}{[...\PYGZsq{}marker\PYGZus{}ex1\PYGZsq{}...]}
\PYG{g+gp}{\PYGZgt{}\PYGZgt{}\PYGZgt{} }\PYG{n}{ex2}\PYG{o}{=}\PYG{n}{vcs}\PYG{o}{.}\PYG{n}{createmarker}\PYG{p}{(}\PYG{l+s+s1}{\PYGZsq{}}\PYG{l+s+s1}{marker\PYGZus{}ex2}\PYG{l+s+s1}{\PYGZsq{}}\PYG{p}{,}\PYG{l+s+s1}{\PYGZsq{}}\PYG{l+s+s1}{red}\PYG{l+s+s1}{\PYGZsq{}}\PYG{p}{)} \PYG{c+c1}{\PYGZsh{} create \PYGZsq{}marker\PYGZus{}ex2\PYGZsq{} from \PYGZsq{}red\PYGZsq{} template}
\PYG{g+gp}{\PYGZgt{}\PYGZgt{}\PYGZgt{} }\PYG{n}{vcs}\PYG{o}{.}\PYG{n}{listelements}\PYG{p}{(}\PYG{l+s+s1}{\PYGZsq{}}\PYG{l+s+s1}{marker}\PYG{l+s+s1}{\PYGZsq{}}\PYG{p}{)} \PYG{c+c1}{\PYGZsh{} should now contain the \PYGZsq{}marker\PYGZus{}ex2\PYGZsq{} marker}
\PYG{g+go}{[...\PYGZsq{}marker\PYGZus{}ex2\PYGZsq{}...]}
\end{Verbatim}

\item[{Parameters}] \leavevmode\begin{itemize}
\item {} 
\textbf{\texttt{name}} (\href{https://docs.python.org/2/library/functions.html\#str}{\emph{\texttt{str}}}) -- Name of created object

\item {} 
\textbf{\texttt{source}} (\href{https://docs.python.org/2/library/functions.html\#str}{\emph{\texttt{str}}}) -- A marker, or string name of a marker

\item {} 
\textbf{\texttt{mtype}} (\href{https://docs.python.org/2/library/functions.html\#str}{\emph{\texttt{str}}}) -- Specifies the type of marker, i.e. ``dot'', ``circle''

\item {} 
\textbf{\texttt{size}} (\href{https://docs.python.org/2/library/functions.html\#int}{\emph{\texttt{int}}}) -- 

\item {} 
\textbf{\texttt{color}} (\emph{\texttt{str or int}}) -- 
A color name from the \href{https://en.wikipedia.org/wiki/X11\_color\_names}{X11 Color Names list},
or an integer value from 0-255, or an RGB/RGBA tuple/list (e.g. (0,100,0), (100,100,0,50))


\item {} 
\textbf{\texttt{priority}} (\href{https://docs.python.org/2/library/functions.html\#int}{\emph{\texttt{int}}}) -- The layer on which the marker will be drawn.

\item {} 
\textbf{\texttt{viewport}} (\emph{\texttt{list of floats}}) -- 4 floats between 0 and 1. These specify the area that the X/Y values are mapped to inside of the canvas

\item {} 
\textbf{\texttt{worldcoordinate}} (\emph{\texttt{list of floats}}) -- List of 4 floats (xmin, xmax, ymin, ymax)

\item {} 
\textbf{\texttt{x}} (\emph{\texttt{list of floats}}) -- List of lists of x coordinates. Values must be between worldcoordinate{[}0{]} and worldcoordinate{[}1{]}.

\item {} 
\textbf{\texttt{y}} (\emph{\texttt{list of floats}}) -- List of lists of y coordinates. Values must be between worldcoordinate{[}2{]} and worldcoordinate{[}3{]}.

\end{itemize}

\item[{Returns}] \leavevmode
A secondary marker method

\item[{Return type}] \leavevmode
{\hyperref[vcs/secondary/marker:vcs.marker.Tm]{\sphinxcrossref{vcs.marker.Tm}}}

\end{description}\end{quote}

\end{fulllineitems}

\index{createmeshfill() (in module vcs.manageElements)}

\begin{fulllineitems}
\phantomsection\label{vcs/misc/manageElements:vcs.manageElements.createmeshfill}\pysiglinewithargsret{\sphinxcode{vcs.manageElements.}\sphinxbfcode{createmeshfill}}{\emph{name=None}, \emph{source='default'}}{}
Create a new meshfill graphics method given the the name and the existing
meshfill graphics method to copy the attributes from. If no existing
meshfill graphics method is given, then the default meshfill graphics method will be used as the graphics method
to which the attributes will be copied from.

\begin{notice}{note}{Note:}
If the name provided already exists, then an error will be returned. graphics method
names must be unique.
\end{notice}
\begin{quote}\begin{description}
\item[{Example}] \leavevmode
\begin{Verbatim}[commandchars=\\\{\}]
\PYG{g+gp}{\PYGZgt{}\PYGZgt{}\PYGZgt{} }\PYG{n}{vcs}\PYG{o}{.}\PYG{n}{show}\PYG{p}{(}\PYG{l+s+s1}{\PYGZsq{}}\PYG{l+s+s1}{meshfill}\PYG{l+s+s1}{\PYGZsq{}}\PYG{p}{)} \PYG{c+c1}{\PYGZsh{} show all available meshfill}
\PYG{g+go}{*******************Meshfill Names List**********************}
\PYG{g+gp}{...}
\PYG{g+go}{*******************End Meshfill Names List**********************}
\PYG{g+gp}{\PYGZgt{}\PYGZgt{}\PYGZgt{} }\PYG{n}{ex}\PYG{o}{=}\PYG{n}{vcs}\PYG{o}{.}\PYG{n}{createmeshfill}\PYG{p}{(}\PYG{l+s+s1}{\PYGZsq{}}\PYG{l+s+s1}{meshfill\PYGZus{}ex1}\PYG{l+s+s1}{\PYGZsq{}}\PYG{p}{)} \PYG{c+c1}{\PYGZsh{} Create meshfill \PYGZsq{}meshfill\PYGZus{}ex1\PYGZsq{} that inherits from \PYGZsq{}default\PYGZsq{}}
\PYG{g+gp}{\PYGZgt{}\PYGZgt{}\PYGZgt{} }\PYG{n}{vcs}\PYG{o}{.}\PYG{n}{listelements}\PYG{p}{(}\PYG{l+s+s1}{\PYGZsq{}}\PYG{l+s+s1}{meshfill}\PYG{l+s+s1}{\PYGZsq{}}\PYG{p}{)} \PYG{c+c1}{\PYGZsh{} should now contain the \PYGZsq{}meshfill\PYGZus{}ex1\PYGZsq{} meshfill}
\PYG{g+go}{[...\PYGZsq{}meshfill\PYGZus{}ex1\PYGZsq{}...]}
\PYG{g+gp}{\PYGZgt{}\PYGZgt{}\PYGZgt{} }\PYG{n}{ex2}\PYG{o}{=}\PYG{n}{vcs}\PYG{o}{.}\PYG{n}{createmeshfill}\PYG{p}{(}\PYG{l+s+s1}{\PYGZsq{}}\PYG{l+s+s1}{meshfill\PYGZus{}ex2}\PYG{l+s+s1}{\PYGZsq{}}\PYG{p}{,}\PYG{l+s+s1}{\PYGZsq{}}\PYG{l+s+s1}{a\PYGZus{}polar\PYGZus{}meshfill}\PYG{l+s+s1}{\PYGZsq{}}\PYG{p}{)} \PYG{c+c1}{\PYGZsh{} create \PYGZsq{}meshfill\PYGZus{}ex2\PYGZsq{} from \PYGZsq{}a\PYGZus{}polar\PYGZus{}meshfill\PYGZsq{} template}
\PYG{g+gp}{\PYGZgt{}\PYGZgt{}\PYGZgt{} }\PYG{n}{vcs}\PYG{o}{.}\PYG{n}{listelements}\PYG{p}{(}\PYG{l+s+s1}{\PYGZsq{}}\PYG{l+s+s1}{meshfill}\PYG{l+s+s1}{\PYGZsq{}}\PYG{p}{)} \PYG{c+c1}{\PYGZsh{} should now contain the \PYGZsq{}meshfill\PYGZus{}ex2\PYGZsq{} meshfill}
\PYG{g+go}{[...\PYGZsq{}meshfill\PYGZus{}ex2\PYGZsq{}...]}
\end{Verbatim}

\item[{Parameters}] \leavevmode\begin{itemize}
\item {} 
\textbf{\texttt{name}} (\href{https://docs.python.org/2/library/functions.html\#str}{\emph{\texttt{str}}}) -- The name of the created object

\item {} 
\textbf{\texttt{source}} (\emph{\texttt{a meshfill or a string name of a meshfill}}) -- The object to inherit from

\end{itemize}

\item[{Returns}] \leavevmode
A meshfill graphics method object

\item[{Return type}] \leavevmode
{\hyperref[vcs/graphics/meshfill:vcs.meshfill.Gfm]{\sphinxcrossref{vcs.meshfill.Gfm}}}

\end{description}\end{quote}

\end{fulllineitems}

\index{createprojection() (in module vcs.manageElements)}

\begin{fulllineitems}
\phantomsection\label{vcs/misc/manageElements:vcs.manageElements.createprojection}\pysiglinewithargsret{\sphinxcode{vcs.manageElements.}\sphinxbfcode{createprojection}}{\emph{name=None}, \emph{source='default'}}{}
Create a new projection graphics method given the the name and the existing
projection graphics method to copy the attributes from. If no existing
projection graphics method is given, then the default projection graphics method will be used as the graphics method
to which the attributes will be copied from.

\begin{notice}{note}{Note:}
If the name provided already exists, then an error will be returned. graphics method
names must be unique.
\end{notice}
\begin{quote}\begin{description}
\item[{Example}] \leavevmode
\begin{Verbatim}[commandchars=\\\{\}]
\PYG{g+gp}{\PYGZgt{}\PYGZgt{}\PYGZgt{} }\PYG{n}{vcs}\PYG{o}{.}\PYG{n}{show}\PYG{p}{(}\PYG{l+s+s1}{\PYGZsq{}}\PYG{l+s+s1}{projection}\PYG{l+s+s1}{\PYGZsq{}}\PYG{p}{)} \PYG{c+c1}{\PYGZsh{} show all available projection}
\PYG{g+go}{*******************Projection Names List**********************}
\PYG{g+gp}{...}
\PYG{g+go}{*******************End Projection Names List**********************}
\PYG{g+gp}{\PYGZgt{}\PYGZgt{}\PYGZgt{} }\PYG{n}{ex}\PYG{o}{=}\PYG{n}{vcs}\PYG{o}{.}\PYG{n}{createprojection}\PYG{p}{(}\PYG{l+s+s1}{\PYGZsq{}}\PYG{l+s+s1}{projection\PYGZus{}ex1}\PYG{l+s+s1}{\PYGZsq{}}\PYG{p}{)} \PYG{c+c1}{\PYGZsh{} Create projection \PYGZsq{}projection\PYGZus{}ex1\PYGZsq{} that inherits from \PYGZsq{}default\PYGZsq{}}
\PYG{g+gp}{\PYGZgt{}\PYGZgt{}\PYGZgt{} }\PYG{n}{vcs}\PYG{o}{.}\PYG{n}{listelements}\PYG{p}{(}\PYG{l+s+s1}{\PYGZsq{}}\PYG{l+s+s1}{projection}\PYG{l+s+s1}{\PYGZsq{}}\PYG{p}{)} \PYG{c+c1}{\PYGZsh{} should now contain the \PYGZsq{}projection\PYGZus{}ex1\PYGZsq{} projection}
\PYG{g+go}{[...\PYGZsq{}projection\PYGZus{}ex1\PYGZsq{}...]}
\PYG{g+gp}{\PYGZgt{}\PYGZgt{}\PYGZgt{} }\PYG{n}{ex2}\PYG{o}{=}\PYG{n}{vcs}\PYG{o}{.}\PYG{n}{createprojection}\PYG{p}{(}\PYG{l+s+s1}{\PYGZsq{}}\PYG{l+s+s1}{projection\PYGZus{}ex2}\PYG{l+s+s1}{\PYGZsq{}}\PYG{p}{,}\PYG{l+s+s1}{\PYGZsq{}}\PYG{l+s+s1}{polar}\PYG{l+s+s1}{\PYGZsq{}}\PYG{p}{)} \PYG{c+c1}{\PYGZsh{} create \PYGZsq{}projection\PYGZus{}ex2\PYGZsq{} from \PYGZsq{}polar\PYGZsq{} template}
\PYG{g+gp}{\PYGZgt{}\PYGZgt{}\PYGZgt{} }\PYG{n}{vcs}\PYG{o}{.}\PYG{n}{listelements}\PYG{p}{(}\PYG{l+s+s1}{\PYGZsq{}}\PYG{l+s+s1}{projection}\PYG{l+s+s1}{\PYGZsq{}}\PYG{p}{)} \PYG{c+c1}{\PYGZsh{} should now contain the \PYGZsq{}projection\PYGZus{}ex2\PYGZsq{} projection}
\PYG{g+go}{[...\PYGZsq{}projection\PYGZus{}ex2\PYGZsq{}...]}
\end{Verbatim}

\item[{Parameters}] \leavevmode\begin{itemize}
\item {} 
\textbf{\texttt{name}} (\href{https://docs.python.org/2/library/functions.html\#str}{\emph{\texttt{str}}}) -- The name of the created object

\item {} 
\textbf{\texttt{source}} (\emph{\texttt{a projection or a string name of a projection}}) -- The object to inherit from

\end{itemize}

\item[{Returns}] \leavevmode
A projection graphics method object

\item[{Return type}] \leavevmode
{\hyperref[vcs/misc/projection:vcs.projection.Proj]{\sphinxcrossref{vcs.projection.Proj}}}

\end{description}\end{quote}

\end{fulllineitems}

\index{createscatter() (in module vcs.manageElements)}

\begin{fulllineitems}
\phantomsection\label{vcs/misc/manageElements:vcs.manageElements.createscatter}\pysiglinewithargsret{\sphinxcode{vcs.manageElements.}\sphinxbfcode{createscatter}}{\emph{name=None}, \emph{source='default'}}{}
Create a new scatter graphics method given the the name and the existing
scatter graphics method to copy the attributes from. If no existing
scatter graphics method is given, then the default scatter graphics method will be used as the graphics method
to which the attributes will be copied from.

\begin{notice}{note}{Note:}
If the name provided already exists, then an error will be returned. graphics method
names must be unique.
\end{notice}
\begin{quote}\begin{description}
\item[{Example}] \leavevmode
\begin{Verbatim}[commandchars=\\\{\}]
\PYG{g+gp}{\PYGZgt{}\PYGZgt{}\PYGZgt{} }\PYG{n}{vcs}\PYG{o}{.}\PYG{n}{show}\PYG{p}{(}\PYG{l+s+s1}{\PYGZsq{}}\PYG{l+s+s1}{scatter}\PYG{l+s+s1}{\PYGZsq{}}\PYG{p}{)} \PYG{c+c1}{\PYGZsh{} show all available scatter}
\PYG{g+go}{*******************Scatter Names List**********************}
\PYG{g+gp}{...}
\PYG{g+go}{*******************End Scatter Names List**********************}
\PYG{g+gp}{\PYGZgt{}\PYGZgt{}\PYGZgt{} }\PYG{n}{ex}\PYG{o}{=}\PYG{n}{vcs}\PYG{o}{.}\PYG{n}{createscatter}\PYG{p}{(}\PYG{l+s+s1}{\PYGZsq{}}\PYG{l+s+s1}{scatter\PYGZus{}ex1}\PYG{l+s+s1}{\PYGZsq{}}\PYG{p}{)} \PYG{c+c1}{\PYGZsh{} Create scatter \PYGZsq{}scatter\PYGZus{}ex1\PYGZsq{} that inherits from \PYGZsq{}default\PYGZsq{}}
\PYG{g+gp}{\PYGZgt{}\PYGZgt{}\PYGZgt{} }\PYG{n}{vcs}\PYG{o}{.}\PYG{n}{listelements}\PYG{p}{(}\PYG{l+s+s1}{\PYGZsq{}}\PYG{l+s+s1}{scatter}\PYG{l+s+s1}{\PYGZsq{}}\PYG{p}{)} \PYG{c+c1}{\PYGZsh{} should now contain the \PYGZsq{}scatter\PYGZus{}ex1\PYGZsq{} scatter}
\PYG{g+go}{[...\PYGZsq{}scatter\PYGZus{}ex1\PYGZsq{}...]}
\end{Verbatim}

\item[{Parameters}] \leavevmode\begin{itemize}
\item {} 
\textbf{\texttt{name}} (\href{https://docs.python.org/2/library/functions.html\#str}{\emph{\texttt{str}}}) -- The name of the created object

\item {} 
\textbf{\texttt{source}} (\emph{\texttt{a scatter or a string name of a scatter}}) -- The object to inherit from

\item {} 
\textbf{\texttt{xaxis}} (\emph{\texttt{cdms2.axis.TransientAxis}}) -- Axis object to replace the slab -1 dim axis

\item {} 
\textbf{\texttt{yaxis}} (\emph{\texttt{cdms2.axis.TransientAxis}}) -- Axis object to replace the slab -2 dim axis, only if slab has more than 1D

\item {} 
\textbf{\texttt{zaxis}} (\emph{\texttt{cdms2.axis.TransientAxis}}) -- Axis object to replace the slab -3 dim axis, only if slab has more than 2D

\item {} 
\textbf{\texttt{taxis}} (\emph{\texttt{cdms2.axis.TransientAxis}}) -- Axis object to replace the slab -4 dim axis, only if slab has more than 3D

\item {} 
\textbf{\texttt{waxis}} (\emph{\texttt{cdms2.axis.TransientAxis}}) -- Axis object to replace the slab -5 dim axis, only if slab has more than 4D

\item {} 
\textbf{\texttt{xrev}} (\href{https://docs.python.org/2/library/functions.html\#bool}{\emph{\texttt{bool}}}) -- reverse x axis

\item {} 
\textbf{\texttt{yrev}} (\href{https://docs.python.org/2/library/functions.html\#bool}{\emph{\texttt{bool}}}) -- reverse y axis, only if slab has more than 1D

\item {} 
\textbf{\texttt{xarray}} (\href{https://docs.python.org/2/library/array.html\#module-array}{\emph{\texttt{array}}}) -- Values to use instead of x axis

\item {} 
\textbf{\texttt{yarray}} (\href{https://docs.python.org/2/library/array.html\#module-array}{\emph{\texttt{array}}}) -- Values to use instead of y axis, only if var has more than 1D

\item {} 
\textbf{\texttt{zarray}} (\href{https://docs.python.org/2/library/array.html\#module-array}{\emph{\texttt{array}}}) -- Values to use instead of z axis, only if var has more than 2D

\item {} 
\textbf{\texttt{tarray}} (\href{https://docs.python.org/2/library/array.html\#module-array}{\emph{\texttt{array}}}) -- Values to use instead of t axis, only if var has more than 3D

\item {} 
\textbf{\texttt{warray}} (\href{https://docs.python.org/2/library/array.html\#module-array}{\emph{\texttt{array}}}) -- Values to use instead of w axis, only if var has more than 4D

\item {} 
\textbf{\texttt{continents}} (\href{https://docs.python.org/2/library/functions.html\#int}{\emph{\texttt{int}}}) -- continents type number

\item {} 
\textbf{\texttt{name}} -- replaces variable name on plot

\item {} 
\textbf{\texttt{time}} (\emph{\texttt{A cdtime object}}) -- replaces time name on plot

\item {} 
\textbf{\texttt{units}} (\href{https://docs.python.org/2/library/functions.html\#str}{\emph{\texttt{str}}}) -- replaces units value on plot

\item {} 
\textbf{\texttt{ymd}} (\href{https://docs.python.org/2/library/functions.html\#str}{\emph{\texttt{str}}}) -- replaces year/month/day on plot

\item {} 
\textbf{\texttt{hms}} (\href{https://docs.python.org/2/library/functions.html\#str}{\emph{\texttt{str}}}) -- replaces hh/mm/ss on plot

\item {} 
\textbf{\texttt{file\_comment}} (\href{https://docs.python.org/2/library/functions.html\#str}{\emph{\texttt{str}}}) -- replaces file\_comment on plot

\item {} 
\textbf{\texttt{xbounds}} (\href{https://docs.python.org/2/library/array.html\#module-array}{\emph{\texttt{array}}}) -- Values to use instead of x axis bounds values

\item {} 
\textbf{\texttt{ybounds}} (\href{https://docs.python.org/2/library/array.html\#module-array}{\emph{\texttt{array}}}) -- Values to use instead of y axis bounds values (if exist)

\item {} 
\textbf{\texttt{xname}} (\href{https://docs.python.org/2/library/functions.html\#str}{\emph{\texttt{str}}}) -- replace xaxis name on plot

\item {} 
\textbf{\texttt{yname}} (\href{https://docs.python.org/2/library/functions.html\#str}{\emph{\texttt{str}}}) -- replace yaxis name on plot (if exists)

\item {} 
\textbf{\texttt{zname}} (\href{https://docs.python.org/2/library/functions.html\#str}{\emph{\texttt{str}}}) -- replace zaxis name on plot (if exists)

\item {} 
\textbf{\texttt{tname}} (\href{https://docs.python.org/2/library/functions.html\#str}{\emph{\texttt{str}}}) -- replace taxis name on plot (if exists)

\item {} 
\textbf{\texttt{wname}} (\href{https://docs.python.org/2/library/functions.html\#str}{\emph{\texttt{str}}}) -- replace waxis name on plot (if exists)

\item {} 
\textbf{\texttt{xunits}} (\href{https://docs.python.org/2/library/functions.html\#str}{\emph{\texttt{str}}}) -- replace xaxis units on plot

\item {} 
\textbf{\texttt{yunits}} (\href{https://docs.python.org/2/library/functions.html\#str}{\emph{\texttt{str}}}) -- replace yaxis units on plot (if exists)

\item {} 
\textbf{\texttt{zunits}} (\href{https://docs.python.org/2/library/functions.html\#str}{\emph{\texttt{str}}}) -- replace zaxis units on plot (if exists)

\item {} 
\textbf{\texttt{tunits}} (\href{https://docs.python.org/2/library/functions.html\#str}{\emph{\texttt{str}}}) -- replace taxis units on plot (if exists)

\item {} 
\textbf{\texttt{wunits}} (\href{https://docs.python.org/2/library/functions.html\#str}{\emph{\texttt{str}}}) -- replace waxis units on plot (if exists)

\item {} 
\textbf{\texttt{xweights}} (\href{https://docs.python.org/2/library/array.html\#module-array}{\emph{\texttt{array}}}) -- replace xaxis weights used for computing mean

\item {} 
\textbf{\texttt{yweights}} (\href{https://docs.python.org/2/library/array.html\#module-array}{\emph{\texttt{array}}}) -- replace xaxis weights used for computing mean

\item {} 
\textbf{\texttt{comment1}} (\href{https://docs.python.org/2/library/functions.html\#str}{\emph{\texttt{str}}}) -- replaces comment1 on plot

\item {} 
\textbf{\texttt{comment2}} (\href{https://docs.python.org/2/library/functions.html\#str}{\emph{\texttt{str}}}) -- replaces comment2 on plot

\item {} 
\textbf{\texttt{comment3}} (\href{https://docs.python.org/2/library/functions.html\#str}{\emph{\texttt{str}}}) -- replaces comment3 on plot

\item {} 
\textbf{\texttt{comment4}} (\href{https://docs.python.org/2/library/functions.html\#str}{\emph{\texttt{str}}}) -- replaces comment4 on plot

\item {} 
\textbf{\texttt{long\_name}} (\href{https://docs.python.org/2/library/functions.html\#str}{\emph{\texttt{str}}}) -- replaces long\_name on plot

\item {} 
\textbf{\texttt{grid}} (\emph{\texttt{cdms2.grid.TransientRectGrid}}) -- replaces array grid (if exists)

\item {} 
\textbf{\texttt{bg}} (\emph{\texttt{bool/int}}) -- plots in background mode

\item {} 
\textbf{\texttt{ratio}} (\index{xmtics1 (in module vcs.manageElements)}\index{xmtics2 (in module vcs.manageElements)}\index{ymtics1 (in module vcs.manageElements)}\index{ymtics2 (in module vcs.manageElements)}\index{xticlabels1 (in module vcs.manageElements)}\index{xticlabels2 (in module vcs.manageElements)}\index{yticlabels1 (in module vcs.manageElements)}\index{yticlabels2 (in module vcs.manageElements)}\index{projection (in module vcs.manageElements)}\index{datawc\_x1 (in module vcs.manageElements)}\index{datawc\_x2 (in module vcs.manageElements)}\index{datawc\_y1 (in module vcs.manageElements)}\index{datawc\_y2 (in module vcs.manageElements)}\index{datawc\_timeunits (in module vcs.manageElements)}\index{datawc\_calendar (in module vcs.manageElements)}) -- sets the y/x ratio ,if passed as a string with `t' at the end, will aslo moves the ticks

\item {} 
\textbf{\texttt{xaxisconvert}} (\href{https://docs.python.org/2/library/functions.html\#str}{\emph{\texttt{str}}}) -- (Ex: `linear') converting xaxis linear/log/log10/ln/exp/area\_wt

\item {} 
\textbf{\texttt{yaxisconvert}} (\href{https://docs.python.org/2/library/functions.html\#str}{\emph{\texttt{str}}}) -- (Ex: `linear') converting yaxis linear/log/log10/ln/exp/area\_wt

\item {} 
\textbf{\texttt{new\_GM\_name}} (\href{https://docs.python.org/2/library/functions.html\#str}{\emph{\texttt{str}}}) -- (Ex: `my\_awesome\_gm') name of the new graphics method object. If no name is given, then one will be created for use.

\item {} 
\textbf{\texttt{source\_GM\_name}} -- (Ex: `default') copy the contents of the source object to the newly created one. If no name is given, then the `default' graphics methond contents is copied over to the new object.

\end{itemize}

\item[{Returns}] \leavevmode
A scatter graphics method

\item[{Return type}] \leavevmode
{\hyperref[vcs/graphics/unified1D:vcs.unified1D.G1d]{\sphinxcrossref{vcs.unified1D.G1d}}}

\end{description}\end{quote}

\end{fulllineitems}

\index{createtaylordiagram() (in module vcs.manageElements)}

\begin{fulllineitems}
\phantomsection\label{vcs/misc/manageElements:vcs.manageElements.createtaylordiagram}\pysiglinewithargsret{\sphinxcode{vcs.manageElements.}\sphinxbfcode{createtaylordiagram}}{\emph{name=None}, \emph{source='default'}}{}
Create a new taylordiagram graphics method given the the name and the existing
taylordiagram graphics method to copy the attributes from. If no existing
taylordiagram graphics method is given, then the default taylordiagram graphics method will be used as the graphics method
to which the attributes will be copied from.

\begin{notice}{note}{Note:}
If the name provided already exists, then an error will be returned. graphics method
names must be unique.
\end{notice}
\begin{quote}\begin{description}
\item[{Example}] \leavevmode
\begin{Verbatim}[commandchars=\\\{\}]
\PYG{g+gp}{\PYGZgt{}\PYGZgt{}\PYGZgt{} }\PYG{n}{vcs}\PYG{o}{.}\PYG{n}{show}\PYG{p}{(}\PYG{l+s+s1}{\PYGZsq{}}\PYG{l+s+s1}{taylordiagram}\PYG{l+s+s1}{\PYGZsq{}}\PYG{p}{)} \PYG{c+c1}{\PYGZsh{} show all available taylordiagram}
\PYG{g+go}{*******************Taylordiagram Names List**********************}
\PYG{g+gp}{...}
\PYG{g+go}{*******************End Taylordiagram Names List**********************}
\PYG{g+gp}{\PYGZgt{}\PYGZgt{}\PYGZgt{} }\PYG{n}{ex}\PYG{o}{=}\PYG{n}{vcs}\PYG{o}{.}\PYG{n}{createtaylordiagram}\PYG{p}{(}\PYG{l+s+s1}{\PYGZsq{}}\PYG{l+s+s1}{taylordiagram\PYGZus{}ex1}\PYG{l+s+s1}{\PYGZsq{}}\PYG{p}{)} \PYG{c+c1}{\PYGZsh{} Create taylordiagram \PYGZsq{}taylordiagram\PYGZus{}ex1\PYGZsq{} that inherits from \PYGZsq{}default\PYGZsq{}}
\PYG{g+gp}{\PYGZgt{}\PYGZgt{}\PYGZgt{} }\PYG{n}{vcs}\PYG{o}{.}\PYG{n}{listelements}\PYG{p}{(}\PYG{l+s+s1}{\PYGZsq{}}\PYG{l+s+s1}{taylordiagram}\PYG{l+s+s1}{\PYGZsq{}}\PYG{p}{)} \PYG{c+c1}{\PYGZsh{} should now contain the \PYGZsq{}taylordiagram\PYGZus{}ex1\PYGZsq{} taylordiagram}
\PYG{g+go}{[...\PYGZsq{}taylordiagram\PYGZus{}ex1\PYGZsq{}...]}
\end{Verbatim}

\item[{Parameters}] \leavevmode\begin{itemize}
\item {} 
\textbf{\texttt{name}} (\href{https://docs.python.org/2/library/functions.html\#str}{\emph{\texttt{str}}}) -- The name of the created object

\item {} 
\textbf{\texttt{source}} (\emph{\texttt{a taylordiagram or a string name of a}}) -- The object to inherit from

\end{itemize}

\item[{Returns}] \leavevmode
A taylordiagram graphics method object

\item[{Return type}] \leavevmode
{\hyperref[vcs/graphics/taylor:vcs.taylor.Gtd]{\sphinxcrossref{vcs.taylor.Gtd}}}

\end{description}\end{quote}

\end{fulllineitems}

\index{createtemplate() (in module vcs.manageElements)}

\begin{fulllineitems}
\phantomsection\label{vcs/misc/manageElements:vcs.manageElements.createtemplate}\pysiglinewithargsret{\sphinxcode{vcs.manageElements.}\sphinxbfcode{createtemplate}}{\emph{name=None}, \emph{source='default'}}{}
Create a new template graphics method given the the name and the existing
template graphics method to copy the attributes from. If no existing
template graphics method is given, then the default template graphics method will be used as the graphics method
to which the attributes will be copied from.

\begin{notice}{note}{Note:}
If the name provided already exists, then an error will be returned. graphics method
names must be unique.
\end{notice}
\begin{quote}\begin{description}
\item[{Example}] \leavevmode
\begin{Verbatim}[commandchars=\\\{\}]
\PYG{g+gp}{\PYGZgt{}\PYGZgt{}\PYGZgt{} }\PYG{n}{vcs}\PYG{o}{.}\PYG{n}{show}\PYG{p}{(}\PYG{l+s+s1}{\PYGZsq{}}\PYG{l+s+s1}{template}\PYG{l+s+s1}{\PYGZsq{}}\PYG{p}{)} \PYG{c+c1}{\PYGZsh{} show all available template}
\PYG{g+go}{*******************Template Names List**********************}
\PYG{g+gp}{...}
\PYG{g+go}{*******************End Template Names List**********************}
\PYG{g+gp}{\PYGZgt{}\PYGZgt{}\PYGZgt{} }\PYG{n}{ex}\PYG{o}{=}\PYG{n}{vcs}\PYG{o}{.}\PYG{n}{createtemplate}\PYG{p}{(}\PYG{l+s+s1}{\PYGZsq{}}\PYG{l+s+s1}{template\PYGZus{}ex1}\PYG{l+s+s1}{\PYGZsq{}}\PYG{p}{)} \PYG{c+c1}{\PYGZsh{} Create template \PYGZsq{}template\PYGZus{}ex1\PYGZsq{} that inherits from \PYGZsq{}default\PYGZsq{}}
\PYG{g+gp}{\PYGZgt{}\PYGZgt{}\PYGZgt{} }\PYG{n}{vcs}\PYG{o}{.}\PYG{n}{listelements}\PYG{p}{(}\PYG{l+s+s1}{\PYGZsq{}}\PYG{l+s+s1}{template}\PYG{l+s+s1}{\PYGZsq{}}\PYG{p}{)} \PYG{c+c1}{\PYGZsh{} should now contain the \PYGZsq{}template\PYGZus{}ex1\PYGZsq{} template}
\PYG{g+go}{[...\PYGZsq{}template\PYGZus{}ex1\PYGZsq{}...]}
\PYG{g+gp}{\PYGZgt{}\PYGZgt{}\PYGZgt{} }\PYG{n}{ex2}\PYG{o}{=}\PYG{n}{vcs}\PYG{o}{.}\PYG{n}{createtemplate}\PYG{p}{(}\PYG{l+s+s1}{\PYGZsq{}}\PYG{l+s+s1}{template\PYGZus{}ex2}\PYG{l+s+s1}{\PYGZsq{}}\PYG{p}{,}\PYG{l+s+s1}{\PYGZsq{}}\PYG{l+s+s1}{polar}\PYG{l+s+s1}{\PYGZsq{}}\PYG{p}{)} \PYG{c+c1}{\PYGZsh{} create \PYGZsq{}template\PYGZus{}ex2\PYGZsq{} from \PYGZsq{}polar\PYGZsq{} template}
\PYG{g+gp}{\PYGZgt{}\PYGZgt{}\PYGZgt{} }\PYG{n}{vcs}\PYG{o}{.}\PYG{n}{listelements}\PYG{p}{(}\PYG{l+s+s1}{\PYGZsq{}}\PYG{l+s+s1}{template}\PYG{l+s+s1}{\PYGZsq{}}\PYG{p}{)} \PYG{c+c1}{\PYGZsh{} should now contain the \PYGZsq{}template\PYGZus{}ex2\PYGZsq{} template}
\PYG{g+go}{[...\PYGZsq{}template\PYGZus{}ex2\PYGZsq{}...]}
\end{Verbatim}

\item[{Parameters}] \leavevmode\begin{itemize}
\item {} 
\textbf{\texttt{name}} (\href{https://docs.python.org/2/library/functions.html\#str}{\emph{\texttt{str}}}) -- The name of the created object

\item {} 
\textbf{\texttt{source}} (\emph{\texttt{a template or a string name of a template}}) -- The object to inherit from

\end{itemize}

\item[{Returns}] \leavevmode
A template

\item[{Return type}] \leavevmode
{\hyperref[vcs/template/template:vcs.template.P]{\sphinxcrossref{vcs.template.P}}}

\end{description}\end{quote}

\end{fulllineitems}

\index{createtext() (in module vcs.manageElements)}

\begin{fulllineitems}
\phantomsection\label{vcs/misc/manageElements:vcs.manageElements.createtext}\pysiglinewithargsret{\sphinxcode{vcs.manageElements.}\sphinxbfcode{createtext}}{\emph{Tt\_name=None}, \emph{Tt\_source='default'}, \emph{To\_name=None}, \emph{To\_source='default'}, \emph{font=None}, \emph{spacing=None}, \emph{expansion=None}, \emph{color=None}, \emph{priority=None}, \emph{viewport=None}, \emph{worldcoordinate=None}, \emph{x=None}, \emph{y=None}, \emph{height=None}, \emph{angle=None}, \emph{path=None}, \emph{halign=None}, \emph{valign=None}, \emph{projection=None}}{}
Create a new textcombined secondary method given the the name and the existing
textcombined secondary method to copy the attributes from. If no existing
textcombined secondary method is given, then the default textcombined secondary method will be used as the graphics method
to which the attributes will be copied from.

\begin{notice}{note}{Note:}
If the name provided already exists, then an error will be returned. secondary method
names must be unique.
\end{notice}
\begin{quote}\begin{description}
\item[{Example}] \leavevmode
\begin{Verbatim}[commandchars=\\\{\}]
\PYG{g+gp}{\PYGZgt{}\PYGZgt{}\PYGZgt{} }\PYG{n}{vcs}\PYG{o}{.}\PYG{n}{show}\PYG{p}{(}\PYG{l+s+s1}{\PYGZsq{}}\PYG{l+s+s1}{textcombined}\PYG{l+s+s1}{\PYGZsq{}}\PYG{p}{)} \PYG{c+c1}{\PYGZsh{} show all available textcombined}
\PYG{g+go}{*******************Textcombined Names List**********************}
\PYG{g+gp}{...}
\PYG{g+go}{*******************End Textcombined Names List**********************}
\PYG{g+gp}{\PYGZgt{}\PYGZgt{}\PYGZgt{} }\PYG{n}{a}\PYG{o}{.}\PYG{n}{createtextcombined}\PYG{p}{(}\PYG{l+s+s1}{\PYGZsq{}}\PYG{l+s+s1}{EXAMPLE\PYGZus{}tt}\PYG{l+s+s1}{\PYGZsq{}}\PYG{p}{,} \PYG{l+s+s1}{\PYGZsq{}}\PYG{l+s+s1}{qa}\PYG{l+s+s1}{\PYGZsq{}}\PYG{p}{,} \PYG{l+s+s1}{\PYGZsq{}}\PYG{l+s+s1}{EXAMPLE\PYGZus{}tto}\PYG{l+s+s1}{\PYGZsq{}}\PYG{p}{,} \PYG{l+s+s1}{\PYGZsq{}}\PYG{l+s+s1}{7left}\PYG{l+s+s1}{\PYGZsq{}}\PYG{p}{)} \PYG{c+c1}{\PYGZsh{} Create \PYGZsq{}EXAMPLE\PYGZus{}tt\PYGZsq{} and \PYGZsq{}EXAMPLE\PYGZus{}tto\PYGZsq{}}
\PYG{g+go}{\PYGZlt{}vcs.textcombined.Tc ...\PYGZgt{}}
\PYG{g+gp}{\PYGZgt{}\PYGZgt{}\PYGZgt{} }\PYG{n}{vcs}\PYG{o}{.}\PYG{n}{listelements}\PYG{p}{(}\PYG{l+s+s1}{\PYGZsq{}}\PYG{l+s+s1}{textcombined}\PYG{l+s+s1}{\PYGZsq{}}\PYG{p}{)} \PYG{c+c1}{\PYGZsh{} should now contain the \PYGZsq{}qa\PYGZus{}tt:::left\PYGZus{}tto\PYGZsq{} textcombined}
\PYG{g+go}{[...\PYGZsq{}qa\PYGZus{}tt:::left\PYGZus{}tto\PYGZsq{}...]}
\end{Verbatim}

\item[{Parameters}] \leavevmode\begin{itemize}
\item {} 
\textbf{\texttt{Tt\_name}} (\href{https://docs.python.org/2/library/functions.html\#str}{\emph{\texttt{str}}}) -- Name of created object

\item {} 
\textbf{\texttt{Tt\_source}} (\emph{\texttt{str or vcs.texttable.Tt}}) -- Texttable object to inherit from. Can be a texttable, or a string name of a texttable.

\item {} 
\textbf{\texttt{To\_name}} (\href{https://docs.python.org/2/library/functions.html\#str}{\emph{\texttt{str}}}) -- Name of the textcombined's text orientation  (to be created)

\item {} 
\textbf{\texttt{To\_source}} (\emph{\texttt{str or vcs.textorientation.To}}) -- Name of the textorientation to inherit. Can be a textorientation, or a string name of a textorientation.

\item {} 
\textbf{\texttt{font}} (\emph{\texttt{int or str}}) -- Which font to use (index or name).

\item {} 
\textbf{\texttt{spacing}} (\emph{\texttt{DEPRECATED}}) -- DEPRECATED

\item {} 
\textbf{\texttt{expansion}} (\emph{\texttt{DEPRECATED}}) -- DEPRECATED

\item {} 
\textbf{\texttt{color}} (\emph{\texttt{str or int}}) -- 
A color name from the \href{https://en.wikipedia.org/wiki/X11\_color\_names}{X11 Color Names list},
or an integer value from 0-255, or an RGB/RGBA tuple/list (e.g. (0,100,0), (100,100,0,50))


\item {} 
\textbf{\texttt{priority}} (\href{https://docs.python.org/2/library/functions.html\#int}{\emph{\texttt{int}}}) -- The layer on which the object will be drawn.

\item {} 
\textbf{\texttt{viewport}} (\emph{\texttt{list of floats}}) -- 4 floats between 0 and 1. These specify the area that the X/Y values are mapped to inside of the canvas

\item {} 
\textbf{\texttt{worldcoordinate}} (\emph{\texttt{list of floats}}) -- List of 4 floats (xmin, xmax, ymin, ymax)

\item {} 
\textbf{\texttt{x}} (\emph{\texttt{list of floats}}) -- List of lists of x coordinates. Values must be between worldcoordinate{[}0{]} and worldcoordinate{[}1{]}.

\item {} 
\textbf{\texttt{y}} (\emph{\texttt{list of floats}}) -- List of lists of y coordinates. Values must be between worldcoordinate{[}2{]} and worldcoordinate{[}3{]}.

\item {} 
\textbf{\texttt{height}} (\href{https://docs.python.org/2/library/functions.html\#int}{\emph{\texttt{int}}}) -- Size of the font

\item {} 
\textbf{\texttt{angle}} (\href{https://docs.python.org/2/library/functions.html\#int}{\emph{\texttt{int}}}) -- Angle of the text, in degrees

\item {} 
\textbf{\texttt{path}} (\emph{\texttt{DEPRECATED}}) -- DEPRECATED

\item {} 
\textbf{\texttt{halign}} (\href{https://docs.python.org/2/library/functions.html\#str}{\emph{\texttt{str}}}) -- Horizontal alignment of the text. One of {[}''left'', ``center'', ``right''{]}.

\item {} 
\textbf{\texttt{valign}} (\href{https://docs.python.org/2/library/functions.html\#str}{\emph{\texttt{str}}}) -- Vertical alignment of the text. One of {[}''top'', ``center'', ``botom''{]}.

\item {} 
\textbf{\texttt{projection}} (\emph{\texttt{str or projection object}}) -- Specify a geographic projection used to convert x/y from spherical coordinates into 2D coordinates.

\end{itemize}

\item[{Returns}] \leavevmode
A VCS text object

\item[{Return type}] \leavevmode
{\hyperref[vcs/secondary/textcombined:vcs.textcombined.Tc]{\sphinxcrossref{vcs.textcombined.Tc}}}

\end{description}\end{quote}

\end{fulllineitems}

\index{createtextcombined() (in module vcs.manageElements)}

\begin{fulllineitems}
\phantomsection\label{vcs/misc/manageElements:vcs.manageElements.createtextcombined}\pysiglinewithargsret{\sphinxcode{vcs.manageElements.}\sphinxbfcode{createtextcombined}}{\emph{Tt\_name=None}, \emph{Tt\_source='default'}, \emph{To\_name=None}, \emph{To\_source='default'}, \emph{font=None}, \emph{spacing=None}, \emph{expansion=None}, \emph{color=None}, \emph{priority=None}, \emph{viewport=None}, \emph{worldcoordinate=None}, \emph{x=None}, \emph{y=None}, \emph{height=None}, \emph{angle=None}, \emph{path=None}, \emph{halign=None}, \emph{valign=None}, \emph{projection=None}}{}
Create a new textcombined secondary method given the the name and the existing
textcombined secondary method to copy the attributes from. If no existing
textcombined secondary method is given, then the default textcombined secondary method will be used as the graphics method
to which the attributes will be copied from.

\begin{notice}{note}{Note:}
If the name provided already exists, then an error will be returned. secondary method
names must be unique.
\end{notice}
\begin{quote}\begin{description}
\item[{Example}] \leavevmode
\begin{Verbatim}[commandchars=\\\{\}]
\PYG{g+gp}{\PYGZgt{}\PYGZgt{}\PYGZgt{} }\PYG{n}{vcs}\PYG{o}{.}\PYG{n}{show}\PYG{p}{(}\PYG{l+s+s1}{\PYGZsq{}}\PYG{l+s+s1}{textcombined}\PYG{l+s+s1}{\PYGZsq{}}\PYG{p}{)} \PYG{c+c1}{\PYGZsh{} show all available textcombined}
\PYG{g+go}{*******************Textcombined Names List**********************}
\PYG{g+gp}{...}
\PYG{g+go}{*******************End Textcombined Names List**********************}
\PYG{g+gp}{\PYGZgt{}\PYGZgt{}\PYGZgt{} }\PYG{n}{a}\PYG{o}{.}\PYG{n}{createtextcombined}\PYG{p}{(}\PYG{l+s+s1}{\PYGZsq{}}\PYG{l+s+s1}{EXAMPLE\PYGZus{}tt}\PYG{l+s+s1}{\PYGZsq{}}\PYG{p}{,} \PYG{l+s+s1}{\PYGZsq{}}\PYG{l+s+s1}{qa}\PYG{l+s+s1}{\PYGZsq{}}\PYG{p}{,} \PYG{l+s+s1}{\PYGZsq{}}\PYG{l+s+s1}{EXAMPLE\PYGZus{}tto}\PYG{l+s+s1}{\PYGZsq{}}\PYG{p}{,} \PYG{l+s+s1}{\PYGZsq{}}\PYG{l+s+s1}{7left}\PYG{l+s+s1}{\PYGZsq{}}\PYG{p}{)} \PYG{c+c1}{\PYGZsh{} Create \PYGZsq{}EXAMPLE\PYGZus{}tt\PYGZsq{} and \PYGZsq{}EXAMPLE\PYGZus{}tto\PYGZsq{}}
\PYG{g+go}{\PYGZlt{}vcs.textcombined.Tc ...\PYGZgt{}}
\PYG{g+gp}{\PYGZgt{}\PYGZgt{}\PYGZgt{} }\PYG{n}{vcs}\PYG{o}{.}\PYG{n}{listelements}\PYG{p}{(}\PYG{l+s+s1}{\PYGZsq{}}\PYG{l+s+s1}{textcombined}\PYG{l+s+s1}{\PYGZsq{}}\PYG{p}{)} \PYG{c+c1}{\PYGZsh{} should now contain the \PYGZsq{}qa\PYGZus{}tt:::left\PYGZus{}tto\PYGZsq{} textcombined}
\PYG{g+go}{[...\PYGZsq{}qa\PYGZus{}tt:::left\PYGZus{}tto\PYGZsq{}...]}
\end{Verbatim}

\item[{Parameters}] \leavevmode\begin{itemize}
\item {} 
\textbf{\texttt{Tt\_name}} (\href{https://docs.python.org/2/library/functions.html\#str}{\emph{\texttt{str}}}) -- Name of created object

\item {} 
\textbf{\texttt{Tt\_source}} (\emph{\texttt{str or vcs.texttable.Tt}}) -- Texttable object to inherit from. Can be a texttable, or a string name of a texttable.

\item {} 
\textbf{\texttt{To\_name}} (\href{https://docs.python.org/2/library/functions.html\#str}{\emph{\texttt{str}}}) -- Name of the textcombined's text orientation  (to be created)

\item {} 
\textbf{\texttt{To\_source}} (\emph{\texttt{str or vcs.textorientation.To}}) -- Name of the textorientation to inherit. Can be a textorientation, or a string name of a textorientation.

\item {} 
\textbf{\texttt{font}} (\emph{\texttt{int or str}}) -- Which font to use (index or name).

\item {} 
\textbf{\texttt{spacing}} (\emph{\texttt{DEPRECATED}}) -- DEPRECATED

\item {} 
\textbf{\texttt{expansion}} (\emph{\texttt{DEPRECATED}}) -- DEPRECATED

\item {} 
\textbf{\texttt{color}} (\emph{\texttt{str or int}}) -- 
A color name from the \href{https://en.wikipedia.org/wiki/X11\_color\_names}{X11 Color Names list},
or an integer value from 0-255, or an RGB/RGBA tuple/list (e.g. (0,100,0), (100,100,0,50))


\item {} 
\textbf{\texttt{priority}} (\href{https://docs.python.org/2/library/functions.html\#int}{\emph{\texttt{int}}}) -- The layer on which the object will be drawn.

\item {} 
\textbf{\texttt{viewport}} (\emph{\texttt{list of floats}}) -- 4 floats between 0 and 1. These specify the area that the X/Y values are mapped to inside of the canvas

\item {} 
\textbf{\texttt{worldcoordinate}} (\emph{\texttt{list of floats}}) -- List of 4 floats (xmin, xmax, ymin, ymax)

\item {} 
\textbf{\texttt{x}} (\emph{\texttt{list of floats}}) -- List of lists of x coordinates. Values must be between worldcoordinate{[}0{]} and worldcoordinate{[}1{]}.

\item {} 
\textbf{\texttt{y}} (\emph{\texttt{list of floats}}) -- List of lists of y coordinates. Values must be between worldcoordinate{[}2{]} and worldcoordinate{[}3{]}.

\item {} 
\textbf{\texttt{height}} (\href{https://docs.python.org/2/library/functions.html\#int}{\emph{\texttt{int}}}) -- Size of the font

\item {} 
\textbf{\texttt{angle}} (\href{https://docs.python.org/2/library/functions.html\#int}{\emph{\texttt{int}}}) -- Angle of the text, in degrees

\item {} 
\textbf{\texttt{path}} (\emph{\texttt{DEPRECATED}}) -- DEPRECATED

\item {} 
\textbf{\texttt{halign}} (\href{https://docs.python.org/2/library/functions.html\#str}{\emph{\texttt{str}}}) -- Horizontal alignment of the text. One of {[}''left'', ``center'', ``right''{]}.

\item {} 
\textbf{\texttt{valign}} (\href{https://docs.python.org/2/library/functions.html\#str}{\emph{\texttt{str}}}) -- Vertical alignment of the text. One of {[}''top'', ``center'', ``botom''{]}.

\item {} 
\textbf{\texttt{projection}} (\emph{\texttt{str or projection object}}) -- Specify a geographic projection used to convert x/y from spherical coordinates into 2D coordinates.

\end{itemize}

\item[{Returns}] \leavevmode
A VCS text object

\item[{Return type}] \leavevmode
{\hyperref[vcs/secondary/textcombined:vcs.textcombined.Tc]{\sphinxcrossref{vcs.textcombined.Tc}}}

\end{description}\end{quote}

\end{fulllineitems}

\index{createtextorientation() (in module vcs.manageElements)}

\begin{fulllineitems}
\phantomsection\label{vcs/misc/manageElements:vcs.manageElements.createtextorientation}\pysiglinewithargsret{\sphinxcode{vcs.manageElements.}\sphinxbfcode{createtextorientation}}{\emph{name=None}, \emph{source='default'}}{}
Create a new textorientation secondary method given the the name and the existing
textorientation secondary method to copy the attributes from. If no existing
textorientation secondary method is given, then the default textorientation secondary method will be used as the graphics method
to which the attributes will be copied from.

\begin{notice}{note}{Note:}
If the name provided already exists, then an error will be returned. secondary method
names must be unique.
\end{notice}
\begin{quote}\begin{description}
\item[{Example}] \leavevmode
\begin{Verbatim}[commandchars=\\\{\}]
\PYG{g+gp}{\PYGZgt{}\PYGZgt{}\PYGZgt{} }\PYG{n}{vcs}\PYG{o}{.}\PYG{n}{show}\PYG{p}{(}\PYG{l+s+s1}{\PYGZsq{}}\PYG{l+s+s1}{textorientation}\PYG{l+s+s1}{\PYGZsq{}}\PYG{p}{)} \PYG{c+c1}{\PYGZsh{} show all available textorientation}
\PYG{g+go}{*******************Textorientation Names List**********************}
\PYG{g+gp}{...}
\PYG{g+go}{*******************End Textorientation Names List**********************}
\PYG{g+gp}{\PYGZgt{}\PYGZgt{}\PYGZgt{} }\PYG{n}{ex}\PYG{o}{=}\PYG{n}{vcs}\PYG{o}{.}\PYG{n}{createtextorientation}\PYG{p}{(}\PYG{l+s+s1}{\PYGZsq{}}\PYG{l+s+s1}{textorientation\PYGZus{}ex1}\PYG{l+s+s1}{\PYGZsq{}}\PYG{p}{)} \PYG{c+c1}{\PYGZsh{} Create textorientation \PYGZsq{}textorientation\PYGZus{}ex1\PYGZsq{} that inherits from \PYGZsq{}default\PYGZsq{}}
\PYG{g+gp}{\PYGZgt{}\PYGZgt{}\PYGZgt{} }\PYG{n}{vcs}\PYG{o}{.}\PYG{n}{listelements}\PYG{p}{(}\PYG{l+s+s1}{\PYGZsq{}}\PYG{l+s+s1}{textorientation}\PYG{l+s+s1}{\PYGZsq{}}\PYG{p}{)} \PYG{c+c1}{\PYGZsh{} should now contain the \PYGZsq{}textorientation\PYGZus{}ex1\PYGZsq{} textorientation}
\PYG{g+go}{[...\PYGZsq{}textorientation\PYGZus{}ex1\PYGZsq{}...]}
\PYG{g+gp}{\PYGZgt{}\PYGZgt{}\PYGZgt{} }\PYG{n}{ex2}\PYG{o}{=}\PYG{n}{vcs}\PYG{o}{.}\PYG{n}{createtextorientation}\PYG{p}{(}\PYG{l+s+s1}{\PYGZsq{}}\PYG{l+s+s1}{textorientation\PYGZus{}ex2}\PYG{l+s+s1}{\PYGZsq{}}\PYG{p}{,}\PYG{l+s+s1}{\PYGZsq{}}\PYG{l+s+s1}{bigger}\PYG{l+s+s1}{\PYGZsq{}}\PYG{p}{)} \PYG{c+c1}{\PYGZsh{} create \PYGZsq{}textorientation\PYGZus{}ex2\PYGZsq{} from \PYGZsq{}bigger\PYGZsq{} template}
\PYG{g+gp}{\PYGZgt{}\PYGZgt{}\PYGZgt{} }\PYG{n}{vcs}\PYG{o}{.}\PYG{n}{listelements}\PYG{p}{(}\PYG{l+s+s1}{\PYGZsq{}}\PYG{l+s+s1}{textorientation}\PYG{l+s+s1}{\PYGZsq{}}\PYG{p}{)} \PYG{c+c1}{\PYGZsh{} should now contain the \PYGZsq{}textorientation\PYGZus{}ex2\PYGZsq{} textorientation}
\PYG{g+go}{[...\PYGZsq{}textorientation\PYGZus{}ex2\PYGZsq{}...]}
\end{Verbatim}

\item[{Parameters}] \leavevmode\begin{itemize}
\item {} 
\textbf{\texttt{name}} (\href{https://docs.python.org/2/library/functions.html\#str}{\emph{\texttt{str}}}) -- The name of the created object

\item {} 
\textbf{\texttt{source}} (\emph{\texttt{a textorientation or a string name of a textorientation}}) -- The object to inherit from

\end{itemize}

\item[{Returns}] \leavevmode
A textorientation secondary method

\item[{Return type}] \leavevmode
{\hyperref[vcs/secondary/textorientation:vcs.textorientation.To]{\sphinxcrossref{vcs.textorientation.To}}}

\end{description}\end{quote}

\end{fulllineitems}

\index{createtexttable() (in module vcs.manageElements)}

\begin{fulllineitems}
\phantomsection\label{vcs/misc/manageElements:vcs.manageElements.createtexttable}\pysiglinewithargsret{\sphinxcode{vcs.manageElements.}\sphinxbfcode{createtexttable}}{\emph{name=None}, \emph{source='default'}, \emph{font=None}, \emph{spacing=None}, \emph{expansion=None}, \emph{color=None}, \emph{priority=None}, \emph{viewport=None}, \emph{worldcoordinate=None}, \emph{x=None}, \emph{y=None}}{}
Create a new texttable secondary method given the the name and the existing
texttable secondary method to copy the attributes from. If no existing
texttable secondary method is given, then the default texttable secondary method will be used as the graphics method
to which the attributes will be copied from.

\begin{notice}{note}{Note:}
If the name provided already exists, then an error will be returned. secondary method
names must be unique.
\end{notice}
\begin{quote}\begin{description}
\item[{Example}] \leavevmode
\begin{Verbatim}[commandchars=\\\{\}]
\PYG{g+gp}{\PYGZgt{}\PYGZgt{}\PYGZgt{} }\PYG{n}{vcs}\PYG{o}{.}\PYG{n}{show}\PYG{p}{(}\PYG{l+s+s1}{\PYGZsq{}}\PYG{l+s+s1}{texttable}\PYG{l+s+s1}{\PYGZsq{}}\PYG{p}{)} \PYG{c+c1}{\PYGZsh{} show all available texttable}
\PYG{g+go}{*******************Texttable Names List**********************}
\PYG{g+gp}{...}
\PYG{g+go}{*******************End Texttable Names List**********************}
\PYG{g+gp}{\PYGZgt{}\PYGZgt{}\PYGZgt{} }\PYG{n}{ex}\PYG{o}{=}\PYG{n}{vcs}\PYG{o}{.}\PYG{n}{createtexttable}\PYG{p}{(}\PYG{l+s+s1}{\PYGZsq{}}\PYG{l+s+s1}{texttable\PYGZus{}ex1}\PYG{l+s+s1}{\PYGZsq{}}\PYG{p}{)} \PYG{c+c1}{\PYGZsh{} Create texttable \PYGZsq{}texttable\PYGZus{}ex1\PYGZsq{} that inherits from \PYGZsq{}default\PYGZsq{}}
\PYG{g+gp}{\PYGZgt{}\PYGZgt{}\PYGZgt{} }\PYG{n}{vcs}\PYG{o}{.}\PYG{n}{listelements}\PYG{p}{(}\PYG{l+s+s1}{\PYGZsq{}}\PYG{l+s+s1}{texttable}\PYG{l+s+s1}{\PYGZsq{}}\PYG{p}{)} \PYG{c+c1}{\PYGZsh{} should now contain the \PYGZsq{}texttable\PYGZus{}ex1\PYGZsq{} texttable}
\PYG{g+go}{[...\PYGZsq{}texttable\PYGZus{}ex1\PYGZsq{}...]}
\PYG{g+gp}{\PYGZgt{}\PYGZgt{}\PYGZgt{} }\PYG{n}{ex2}\PYG{o}{=}\PYG{n}{vcs}\PYG{o}{.}\PYG{n}{createtexttable}\PYG{p}{(}\PYG{l+s+s1}{\PYGZsq{}}\PYG{l+s+s1}{texttable\PYGZus{}ex2}\PYG{l+s+s1}{\PYGZsq{}}\PYG{p}{,}\PYG{l+s+s1}{\PYGZsq{}}\PYG{l+s+s1}{bigger}\PYG{l+s+s1}{\PYGZsq{}}\PYG{p}{)} \PYG{c+c1}{\PYGZsh{} create \PYGZsq{}texttable\PYGZus{}ex2\PYGZsq{} from \PYGZsq{}bigger\PYGZsq{} template}
\PYG{g+gp}{\PYGZgt{}\PYGZgt{}\PYGZgt{} }\PYG{n}{vcs}\PYG{o}{.}\PYG{n}{listelements}\PYG{p}{(}\PYG{l+s+s1}{\PYGZsq{}}\PYG{l+s+s1}{texttable}\PYG{l+s+s1}{\PYGZsq{}}\PYG{p}{)} \PYG{c+c1}{\PYGZsh{} should now contain the \PYGZsq{}texttable\PYGZus{}ex2\PYGZsq{} texttable}
\PYG{g+go}{[...\PYGZsq{}texttable\PYGZus{}ex2\PYGZsq{}...]}
\end{Verbatim}

\item[{Parameters}] \leavevmode\begin{itemize}
\item {} 
\textbf{\texttt{name}} (\href{https://docs.python.org/2/library/functions.html\#str}{\emph{\texttt{str}}}) -- Name of created object

\item {} 
\textbf{\texttt{source}} (\href{https://docs.python.org/2/library/functions.html\#str}{\emph{\texttt{str}}}) -- a texttable, or string name of a texttable

\item {} 
\textbf{\texttt{font}} (\emph{\texttt{int or string}}) -- Which font to use (index or name).

\item {} 
\textbf{\texttt{expansion}} (\emph{\texttt{DEPRECATED}}) -- DEPRECATED

\item {} 
\textbf{\texttt{color}} (\emph{\texttt{str or int}}) -- 
A color name from the \href{https://en.wikipedia.org/wiki/X11\_color\_names}{X11 Color Names list},
or an integer value from 0-255, or an RGB/RGBA tuple/list (e.g. (0,100,0), (100,100,0,50))


\item {} 
\textbf{\texttt{priority}} (\href{https://docs.python.org/2/library/functions.html\#int}{\emph{\texttt{int}}}) -- The layer on which the texttable will be drawn.

\item {} 
\textbf{\texttt{viewport}} (\emph{\texttt{list of floats}}) -- 4 floats between 0 and 1. These specify the area that the X/Y values are mapped to inside of the canvas

\item {} 
\textbf{\texttt{worldcoordinate}} (\emph{\texttt{list of floats}}) -- List of 4 floats (xmin, xmax, ymin, ymax)

\item {} 
\textbf{\texttt{x}} (\emph{\texttt{list of floats}}) -- List of lists of x coordinates. Values must be between worldcoordinate{[}0{]} and worldcoordinate{[}1{]}.

\item {} 
\textbf{\texttt{y}} (\emph{\texttt{list of floats}}) -- List of lists of y coordinates. Values must be between worldcoordinate{[}2{]} and worldcoordinate{[}3{]}.

\end{itemize}

\item[{Returns}] \leavevmode
A texttable graphics method object

\item[{Return type}] \leavevmode
{\hyperref[vcs/secondary/texttable:vcs.texttable.Tt]{\sphinxcrossref{vcs.texttable.Tt}}}

\end{description}\end{quote}

\end{fulllineitems}

\index{createvector() (in module vcs.manageElements)}

\begin{fulllineitems}
\phantomsection\label{vcs/misc/manageElements:vcs.manageElements.createvector}\pysiglinewithargsret{\sphinxcode{vcs.manageElements.}\sphinxbfcode{createvector}}{\emph{name=None}, \emph{source='default'}}{}
Create a new vector graphics method given the the name and the existing
vector graphics method to copy the attributes from. If no existing
vector graphics method is given, then the default vector graphics method will be used as the graphics method
to which the attributes will be copied from.

\begin{notice}{note}{Note:}
If the name provided already exists, then an error will be returned. graphics method
names must be unique.
\end{notice}
\begin{quote}\begin{description}
\item[{Example}] \leavevmode
\begin{Verbatim}[commandchars=\\\{\}]
\PYG{g+gp}{\PYGZgt{}\PYGZgt{}\PYGZgt{} }\PYG{n}{vcs}\PYG{o}{.}\PYG{n}{show}\PYG{p}{(}\PYG{l+s+s1}{\PYGZsq{}}\PYG{l+s+s1}{vector}\PYG{l+s+s1}{\PYGZsq{}}\PYG{p}{)} \PYG{c+c1}{\PYGZsh{} show all available vector}
\PYG{g+go}{*******************Vector Names List**********************}
\PYG{g+gp}{...}
\PYG{g+go}{*******************End Vector Names List**********************}
\PYG{g+gp}{\PYGZgt{}\PYGZgt{}\PYGZgt{} }\PYG{n}{ex}\PYG{o}{=}\PYG{n}{vcs}\PYG{o}{.}\PYG{n}{createvector}\PYG{p}{(}\PYG{l+s+s1}{\PYGZsq{}}\PYG{l+s+s1}{vector\PYGZus{}ex1}\PYG{l+s+s1}{\PYGZsq{}}\PYG{p}{)} \PYG{c+c1}{\PYGZsh{} Create vector \PYGZsq{}vector\PYGZus{}ex1\PYGZsq{} that inherits from \PYGZsq{}default\PYGZsq{}}
\PYG{g+gp}{\PYGZgt{}\PYGZgt{}\PYGZgt{} }\PYG{n}{vcs}\PYG{o}{.}\PYG{n}{listelements}\PYG{p}{(}\PYG{l+s+s1}{\PYGZsq{}}\PYG{l+s+s1}{vector}\PYG{l+s+s1}{\PYGZsq{}}\PYG{p}{)} \PYG{c+c1}{\PYGZsh{} should now contain the \PYGZsq{}vector\PYGZus{}ex1\PYGZsq{} vector}
\PYG{g+go}{[...\PYGZsq{}vector\PYGZus{}ex1\PYGZsq{}...]}
\end{Verbatim}

\item[{Parameters}] \leavevmode\begin{itemize}
\item {} 
\textbf{\texttt{name}} (\href{https://docs.python.org/2/library/functions.html\#str}{\emph{\texttt{str}}}) -- The name of the created object

\item {} 
\textbf{\texttt{source}} (\emph{\texttt{a vector or a string name of a vector}}) -- The object to inherit from

\end{itemize}

\item[{Returns}] \leavevmode
A vector graphics method object

\item[{Return type}] \leavevmode
{\hyperref[vcs/graphics/vector:vcs.vector.Gv]{\sphinxcrossref{vcs.vector.Gv}}}

\end{description}\end{quote}

\end{fulllineitems}

\index{createxvsy() (in module vcs.manageElements)}

\begin{fulllineitems}
\phantomsection\label{vcs/misc/manageElements:vcs.manageElements.createxvsy}\pysiglinewithargsret{\sphinxcode{vcs.manageElements.}\sphinxbfcode{createxvsy}}{\emph{name=None}, \emph{source='default'}}{}
Create a new xvsy graphics method given the the name and the existing
xvsy graphics method to copy the attributes from. If no existing
xvsy graphics method is given, then the default xvsy graphics method will be used as the graphics method
to which the attributes will be copied from.

\begin{notice}{note}{Note:}
If the name provided already exists, then an error will be returned. graphics method
names must be unique.
\end{notice}
\begin{quote}\begin{description}
\item[{Example}] \leavevmode
\begin{Verbatim}[commandchars=\\\{\}]
\PYG{g+gp}{\PYGZgt{}\PYGZgt{}\PYGZgt{} }\PYG{n}{vcs}\PYG{o}{.}\PYG{n}{show}\PYG{p}{(}\PYG{l+s+s1}{\PYGZsq{}}\PYG{l+s+s1}{xvsy}\PYG{l+s+s1}{\PYGZsq{}}\PYG{p}{)} \PYG{c+c1}{\PYGZsh{} show all available xvsy}
\PYG{g+go}{*******************Xvsy Names List**********************}
\PYG{g+gp}{...}
\PYG{g+go}{*******************End Xvsy Names List**********************}
\PYG{g+gp}{\PYGZgt{}\PYGZgt{}\PYGZgt{} }\PYG{n}{ex}\PYG{o}{=}\PYG{n}{vcs}\PYG{o}{.}\PYG{n}{createxvsy}\PYG{p}{(}\PYG{l+s+s1}{\PYGZsq{}}\PYG{l+s+s1}{xvsy\PYGZus{}ex1}\PYG{l+s+s1}{\PYGZsq{}}\PYG{p}{)} \PYG{c+c1}{\PYGZsh{} Create xvsy \PYGZsq{}xvsy\PYGZus{}ex1\PYGZsq{} that inherits from \PYGZsq{}default\PYGZsq{}}
\PYG{g+gp}{\PYGZgt{}\PYGZgt{}\PYGZgt{} }\PYG{n}{vcs}\PYG{o}{.}\PYG{n}{listelements}\PYG{p}{(}\PYG{l+s+s1}{\PYGZsq{}}\PYG{l+s+s1}{xvsy}\PYG{l+s+s1}{\PYGZsq{}}\PYG{p}{)} \PYG{c+c1}{\PYGZsh{} should now contain the \PYGZsq{}xvsy\PYGZus{}ex1\PYGZsq{} xvsy}
\PYG{g+go}{[...\PYGZsq{}xvsy\PYGZus{}ex1\PYGZsq{}...]}
\end{Verbatim}

\item[{Parameters}] \leavevmode\begin{itemize}
\item {} 
\textbf{\texttt{name}} (\href{https://docs.python.org/2/library/functions.html\#str}{\emph{\texttt{str}}}) -- The name of the created object

\item {} 
\textbf{\texttt{source}} (\emph{\texttt{a xvsy or a string name of a xvsy}}) -- The object to inherit from

\item {} 
\textbf{\texttt{xaxis}} (\emph{\texttt{cdms2.axis.TransientAxis}}) -- Axis object to replace the slab -1 dim axis

\item {} 
\textbf{\texttt{yaxis}} (\emph{\texttt{cdms2.axis.TransientAxis}}) -- Axis object to replace the slab -2 dim axis, only if slab has more than 1D

\item {} 
\textbf{\texttt{zaxis}} (\emph{\texttt{cdms2.axis.TransientAxis}}) -- Axis object to replace the slab -3 dim axis, only if slab has more than 2D

\item {} 
\textbf{\texttt{taxis}} (\emph{\texttt{cdms2.axis.TransientAxis}}) -- Axis object to replace the slab -4 dim axis, only if slab has more than 3D

\item {} 
\textbf{\texttt{waxis}} (\emph{\texttt{cdms2.axis.TransientAxis}}) -- Axis object to replace the slab -5 dim axis, only if slab has more than 4D

\item {} 
\textbf{\texttt{xrev}} (\href{https://docs.python.org/2/library/functions.html\#bool}{\emph{\texttt{bool}}}) -- reverse x axis

\item {} 
\textbf{\texttt{yrev}} (\href{https://docs.python.org/2/library/functions.html\#bool}{\emph{\texttt{bool}}}) -- reverse y axis, only if slab has more than 1D

\item {} 
\textbf{\texttt{xarray}} (\href{https://docs.python.org/2/library/array.html\#module-array}{\emph{\texttt{array}}}) -- Values to use instead of x axis

\item {} 
\textbf{\texttt{yarray}} (\href{https://docs.python.org/2/library/array.html\#module-array}{\emph{\texttt{array}}}) -- Values to use instead of y axis, only if var has more than 1D

\item {} 
\textbf{\texttt{zarray}} (\href{https://docs.python.org/2/library/array.html\#module-array}{\emph{\texttt{array}}}) -- Values to use instead of z axis, only if var has more than 2D

\item {} 
\textbf{\texttt{tarray}} (\href{https://docs.python.org/2/library/array.html\#module-array}{\emph{\texttt{array}}}) -- Values to use instead of t axis, only if var has more than 3D

\item {} 
\textbf{\texttt{warray}} (\href{https://docs.python.org/2/library/array.html\#module-array}{\emph{\texttt{array}}}) -- Values to use instead of w axis, only if var has more than 4D

\item {} 
\textbf{\texttt{continents}} (\href{https://docs.python.org/2/library/functions.html\#int}{\emph{\texttt{int}}}) -- continents type number

\item {} 
\textbf{\texttt{name}} -- replaces variable name on plot

\item {} 
\textbf{\texttt{time}} (\emph{\texttt{A cdtime object}}) -- replaces time name on plot

\item {} 
\textbf{\texttt{units}} (\href{https://docs.python.org/2/library/functions.html\#str}{\emph{\texttt{str}}}) -- replaces units value on plot

\item {} 
\textbf{\texttt{ymd}} (\href{https://docs.python.org/2/library/functions.html\#str}{\emph{\texttt{str}}}) -- replaces year/month/day on plot

\item {} 
\textbf{\texttt{hms}} (\href{https://docs.python.org/2/library/functions.html\#str}{\emph{\texttt{str}}}) -- replaces hh/mm/ss on plot

\item {} 
\textbf{\texttt{file\_comment}} (\href{https://docs.python.org/2/library/functions.html\#str}{\emph{\texttt{str}}}) -- replaces file\_comment on plot

\item {} 
\textbf{\texttt{xbounds}} (\href{https://docs.python.org/2/library/array.html\#module-array}{\emph{\texttt{array}}}) -- Values to use instead of x axis bounds values

\item {} 
\textbf{\texttt{ybounds}} (\href{https://docs.python.org/2/library/array.html\#module-array}{\emph{\texttt{array}}}) -- Values to use instead of y axis bounds values (if exist)

\item {} 
\textbf{\texttt{xname}} (\href{https://docs.python.org/2/library/functions.html\#str}{\emph{\texttt{str}}}) -- replace xaxis name on plot

\item {} 
\textbf{\texttt{yname}} (\href{https://docs.python.org/2/library/functions.html\#str}{\emph{\texttt{str}}}) -- replace yaxis name on plot (if exists)

\item {} 
\textbf{\texttt{zname}} (\href{https://docs.python.org/2/library/functions.html\#str}{\emph{\texttt{str}}}) -- replace zaxis name on plot (if exists)

\item {} 
\textbf{\texttt{tname}} (\href{https://docs.python.org/2/library/functions.html\#str}{\emph{\texttt{str}}}) -- replace taxis name on plot (if exists)

\item {} 
\textbf{\texttt{wname}} (\href{https://docs.python.org/2/library/functions.html\#str}{\emph{\texttt{str}}}) -- replace waxis name on plot (if exists)

\item {} 
\textbf{\texttt{xunits}} (\href{https://docs.python.org/2/library/functions.html\#str}{\emph{\texttt{str}}}) -- replace xaxis units on plot

\item {} 
\textbf{\texttt{yunits}} (\href{https://docs.python.org/2/library/functions.html\#str}{\emph{\texttt{str}}}) -- replace yaxis units on plot (if exists)

\item {} 
\textbf{\texttt{zunits}} (\href{https://docs.python.org/2/library/functions.html\#str}{\emph{\texttt{str}}}) -- replace zaxis units on plot (if exists)

\item {} 
\textbf{\texttt{tunits}} (\href{https://docs.python.org/2/library/functions.html\#str}{\emph{\texttt{str}}}) -- replace taxis units on plot (if exists)

\item {} 
\textbf{\texttt{wunits}} (\href{https://docs.python.org/2/library/functions.html\#str}{\emph{\texttt{str}}}) -- replace waxis units on plot (if exists)

\item {} 
\textbf{\texttt{xweights}} (\href{https://docs.python.org/2/library/array.html\#module-array}{\emph{\texttt{array}}}) -- replace xaxis weights used for computing mean

\item {} 
\textbf{\texttt{yweights}} (\href{https://docs.python.org/2/library/array.html\#module-array}{\emph{\texttt{array}}}) -- replace xaxis weights used for computing mean

\item {} 
\textbf{\texttt{comment1}} (\href{https://docs.python.org/2/library/functions.html\#str}{\emph{\texttt{str}}}) -- replaces comment1 on plot

\item {} 
\textbf{\texttt{comment2}} (\href{https://docs.python.org/2/library/functions.html\#str}{\emph{\texttt{str}}}) -- replaces comment2 on plot

\item {} 
\textbf{\texttt{comment3}} (\href{https://docs.python.org/2/library/functions.html\#str}{\emph{\texttt{str}}}) -- replaces comment3 on plot

\item {} 
\textbf{\texttt{comment4}} (\href{https://docs.python.org/2/library/functions.html\#str}{\emph{\texttt{str}}}) -- replaces comment4 on plot

\item {} 
\textbf{\texttt{long\_name}} (\href{https://docs.python.org/2/library/functions.html\#str}{\emph{\texttt{str}}}) -- replaces long\_name on plot

\item {} 
\textbf{\texttt{grid}} (\emph{\texttt{cdms2.grid.TransientRectGrid}}) -- replaces array grid (if exists)

\item {} 
\textbf{\texttt{bg}} (\emph{\texttt{bool/int}}) -- plots in background mode

\item {} 
\textbf{\texttt{ratio}} (\index{xmtics1 (in module vcs.manageElements)}\index{xmtics2 (in module vcs.manageElements)}\index{ymtics1 (in module vcs.manageElements)}\index{ymtics2 (in module vcs.manageElements)}\index{xticlabels1 (in module vcs.manageElements)}\index{xticlabels2 (in module vcs.manageElements)}\index{yticlabels1 (in module vcs.manageElements)}\index{yticlabels2 (in module vcs.manageElements)}\index{projection (in module vcs.manageElements)}\index{datawc\_x1 (in module vcs.manageElements)}\index{datawc\_x2 (in module vcs.manageElements)}\index{datawc\_y1 (in module vcs.manageElements)}\index{datawc\_y2 (in module vcs.manageElements)}\index{datawc\_timeunits (in module vcs.manageElements)}\index{datawc\_calendar (in module vcs.manageElements)}) -- sets the y/x ratio ,if passed as a string with `t' at the end, will aslo moves the ticks

\item {} 
\textbf{\texttt{xaxisconvert}} (\href{https://docs.python.org/2/library/functions.html\#str}{\emph{\texttt{str}}}) -- (Ex: `linear') converting xaxis linear/log/log10/ln/exp/area\_wt

\item {} 
\textbf{\texttt{yaxisconvert}} (\href{https://docs.python.org/2/library/functions.html\#str}{\emph{\texttt{str}}}) -- (Ex: `linear') converting yaxis linear/log/log10/ln/exp/area\_wt

\item {} 
\textbf{\texttt{new\_GM\_name}} (\href{https://docs.python.org/2/library/functions.html\#str}{\emph{\texttt{str}}}) -- (Ex: `my\_awesome\_gm') name of the new graphics method object. If no name is given, then one will be created for use.

\item {} 
\textbf{\texttt{source\_GM\_name}} -- (Ex: `default') copy the contents of the source object to the newly created one. If no name is given, then the `default' graphics methond contents is copied over to the new object.

\end{itemize}

\item[{Returns}] \leavevmode
A XvsY graphics method object

\item[{Return type}] \leavevmode
{\hyperref[vcs/graphics/unified1D:vcs.unified1D.G1d]{\sphinxcrossref{vcs.unified1D.G1d}}}

\end{description}\end{quote}

\end{fulllineitems}

\index{createxyvsy() (in module vcs.manageElements)}

\begin{fulllineitems}
\phantomsection\label{vcs/misc/manageElements:vcs.manageElements.createxyvsy}\pysiglinewithargsret{\sphinxcode{vcs.manageElements.}\sphinxbfcode{createxyvsy}}{\emph{name=None}, \emph{source='default'}}{}
Create a new xyvsy graphics method given the the name and the existing
xyvsy graphics method to copy the attributes from. If no existing
xyvsy graphics method is given, then the default xyvsy graphics method will be used as the graphics method
to which the attributes will be copied from.

\begin{notice}{note}{Note:}
If the name provided already exists, then an error will be returned. graphics method
names must be unique.
\end{notice}
\begin{quote}\begin{description}
\item[{Example}] \leavevmode
\begin{Verbatim}[commandchars=\\\{\}]
\PYG{g+gp}{\PYGZgt{}\PYGZgt{}\PYGZgt{} }\PYG{n}{vcs}\PYG{o}{.}\PYG{n}{show}\PYG{p}{(}\PYG{l+s+s1}{\PYGZsq{}}\PYG{l+s+s1}{xyvsy}\PYG{l+s+s1}{\PYGZsq{}}\PYG{p}{)} \PYG{c+c1}{\PYGZsh{} show all available xyvsy}
\PYG{g+go}{*******************Xyvsy Names List**********************}
\PYG{g+gp}{...}
\PYG{g+go}{*******************End Xyvsy Names List**********************}
\PYG{g+gp}{\PYGZgt{}\PYGZgt{}\PYGZgt{} }\PYG{n}{ex}\PYG{o}{=}\PYG{n}{vcs}\PYG{o}{.}\PYG{n}{createxyvsy}\PYG{p}{(}\PYG{l+s+s1}{\PYGZsq{}}\PYG{l+s+s1}{xyvsy\PYGZus{}ex1}\PYG{l+s+s1}{\PYGZsq{}}\PYG{p}{)} \PYG{c+c1}{\PYGZsh{} Create xyvsy \PYGZsq{}xyvsy\PYGZus{}ex1\PYGZsq{} that inherits from \PYGZsq{}default\PYGZsq{}}
\PYG{g+gp}{\PYGZgt{}\PYGZgt{}\PYGZgt{} }\PYG{n}{vcs}\PYG{o}{.}\PYG{n}{listelements}\PYG{p}{(}\PYG{l+s+s1}{\PYGZsq{}}\PYG{l+s+s1}{xyvsy}\PYG{l+s+s1}{\PYGZsq{}}\PYG{p}{)} \PYG{c+c1}{\PYGZsh{} should now contain the \PYGZsq{}xyvsy\PYGZus{}ex1\PYGZsq{} xyvsy}
\PYG{g+go}{[...\PYGZsq{}xyvsy\PYGZus{}ex1\PYGZsq{}...]}
\end{Verbatim}

\item[{Parameters}] \leavevmode\begin{itemize}
\item {} 
\textbf{\texttt{name}} (\href{https://docs.python.org/2/library/functions.html\#str}{\emph{\texttt{str}}}) -- The name of the created object

\item {} 
\textbf{\texttt{source}} (\emph{\texttt{a xyvsy or a string name of a xyvsy}}) -- The object to inherit from

\item {} 
\textbf{\texttt{xaxis}} (\emph{\texttt{cdms2.axis.TransientAxis}}) -- Axis object to replace the slab -1 dim axis

\item {} 
\textbf{\texttt{yaxis}} (\emph{\texttt{cdms2.axis.TransientAxis}}) -- Axis object to replace the slab -2 dim axis, only if slab has more than 1D

\item {} 
\textbf{\texttt{zaxis}} (\emph{\texttt{cdms2.axis.TransientAxis}}) -- Axis object to replace the slab -3 dim axis, only if slab has more than 2D

\item {} 
\textbf{\texttt{taxis}} (\emph{\texttt{cdms2.axis.TransientAxis}}) -- Axis object to replace the slab -4 dim axis, only if slab has more than 3D

\item {} 
\textbf{\texttt{waxis}} (\emph{\texttt{cdms2.axis.TransientAxis}}) -- Axis object to replace the slab -5 dim axis, only if slab has more than 4D

\item {} 
\textbf{\texttt{xrev}} (\href{https://docs.python.org/2/library/functions.html\#bool}{\emph{\texttt{bool}}}) -- reverse x axis

\item {} 
\textbf{\texttt{yrev}} (\href{https://docs.python.org/2/library/functions.html\#bool}{\emph{\texttt{bool}}}) -- reverse y axis, only if slab has more than 1D

\item {} 
\textbf{\texttt{xarray}} (\href{https://docs.python.org/2/library/array.html\#module-array}{\emph{\texttt{array}}}) -- Values to use instead of x axis

\item {} 
\textbf{\texttt{yarray}} (\href{https://docs.python.org/2/library/array.html\#module-array}{\emph{\texttt{array}}}) -- Values to use instead of y axis, only if var has more than 1D

\item {} 
\textbf{\texttt{zarray}} (\href{https://docs.python.org/2/library/array.html\#module-array}{\emph{\texttt{array}}}) -- Values to use instead of z axis, only if var has more than 2D

\item {} 
\textbf{\texttt{tarray}} (\href{https://docs.python.org/2/library/array.html\#module-array}{\emph{\texttt{array}}}) -- Values to use instead of t axis, only if var has more than 3D

\item {} 
\textbf{\texttt{warray}} (\href{https://docs.python.org/2/library/array.html\#module-array}{\emph{\texttt{array}}}) -- Values to use instead of w axis, only if var has more than 4D

\item {} 
\textbf{\texttt{continents}} (\href{https://docs.python.org/2/library/functions.html\#int}{\emph{\texttt{int}}}) -- continents type number

\item {} 
\textbf{\texttt{name}} -- replaces variable name on plot

\item {} 
\textbf{\texttt{time}} (\emph{\texttt{A cdtime object}}) -- replaces time name on plot

\item {} 
\textbf{\texttt{units}} (\href{https://docs.python.org/2/library/functions.html\#str}{\emph{\texttt{str}}}) -- replaces units value on plot

\item {} 
\textbf{\texttt{ymd}} (\href{https://docs.python.org/2/library/functions.html\#str}{\emph{\texttt{str}}}) -- replaces year/month/day on plot

\item {} 
\textbf{\texttt{hms}} (\href{https://docs.python.org/2/library/functions.html\#str}{\emph{\texttt{str}}}) -- replaces hh/mm/ss on plot

\item {} 
\textbf{\texttt{file\_comment}} (\href{https://docs.python.org/2/library/functions.html\#str}{\emph{\texttt{str}}}) -- replaces file\_comment on plot

\item {} 
\textbf{\texttt{xbounds}} (\href{https://docs.python.org/2/library/array.html\#module-array}{\emph{\texttt{array}}}) -- Values to use instead of x axis bounds values

\item {} 
\textbf{\texttt{ybounds}} (\href{https://docs.python.org/2/library/array.html\#module-array}{\emph{\texttt{array}}}) -- Values to use instead of y axis bounds values (if exist)

\item {} 
\textbf{\texttt{xname}} (\href{https://docs.python.org/2/library/functions.html\#str}{\emph{\texttt{str}}}) -- replace xaxis name on plot

\item {} 
\textbf{\texttt{yname}} (\href{https://docs.python.org/2/library/functions.html\#str}{\emph{\texttt{str}}}) -- replace yaxis name on plot (if exists)

\item {} 
\textbf{\texttt{zname}} (\href{https://docs.python.org/2/library/functions.html\#str}{\emph{\texttt{str}}}) -- replace zaxis name on plot (if exists)

\item {} 
\textbf{\texttt{tname}} (\href{https://docs.python.org/2/library/functions.html\#str}{\emph{\texttt{str}}}) -- replace taxis name on plot (if exists)

\item {} 
\textbf{\texttt{wname}} (\href{https://docs.python.org/2/library/functions.html\#str}{\emph{\texttt{str}}}) -- replace waxis name on plot (if exists)

\item {} 
\textbf{\texttt{xunits}} (\href{https://docs.python.org/2/library/functions.html\#str}{\emph{\texttt{str}}}) -- replace xaxis units on plot

\item {} 
\textbf{\texttt{yunits}} (\href{https://docs.python.org/2/library/functions.html\#str}{\emph{\texttt{str}}}) -- replace yaxis units on plot (if exists)

\item {} 
\textbf{\texttt{zunits}} (\href{https://docs.python.org/2/library/functions.html\#str}{\emph{\texttt{str}}}) -- replace zaxis units on plot (if exists)

\item {} 
\textbf{\texttt{tunits}} (\href{https://docs.python.org/2/library/functions.html\#str}{\emph{\texttt{str}}}) -- replace taxis units on plot (if exists)

\item {} 
\textbf{\texttt{wunits}} (\href{https://docs.python.org/2/library/functions.html\#str}{\emph{\texttt{str}}}) -- replace waxis units on plot (if exists)

\item {} 
\textbf{\texttt{xweights}} (\href{https://docs.python.org/2/library/array.html\#module-array}{\emph{\texttt{array}}}) -- replace xaxis weights used for computing mean

\item {} 
\textbf{\texttt{yweights}} (\href{https://docs.python.org/2/library/array.html\#module-array}{\emph{\texttt{array}}}) -- replace xaxis weights used for computing mean

\item {} 
\textbf{\texttt{comment1}} (\href{https://docs.python.org/2/library/functions.html\#str}{\emph{\texttt{str}}}) -- replaces comment1 on plot

\item {} 
\textbf{\texttt{comment2}} (\href{https://docs.python.org/2/library/functions.html\#str}{\emph{\texttt{str}}}) -- replaces comment2 on plot

\item {} 
\textbf{\texttt{comment3}} (\href{https://docs.python.org/2/library/functions.html\#str}{\emph{\texttt{str}}}) -- replaces comment3 on plot

\item {} 
\textbf{\texttt{comment4}} (\href{https://docs.python.org/2/library/functions.html\#str}{\emph{\texttt{str}}}) -- replaces comment4 on plot

\item {} 
\textbf{\texttt{long\_name}} (\href{https://docs.python.org/2/library/functions.html\#str}{\emph{\texttt{str}}}) -- replaces long\_name on plot

\item {} 
\textbf{\texttt{grid}} (\emph{\texttt{cdms2.grid.TransientRectGrid}}) -- replaces array grid (if exists)

\item {} 
\textbf{\texttt{bg}} (\emph{\texttt{bool/int}}) -- plots in background mode

\item {} 
\textbf{\texttt{ratio}} (\index{xmtics1 (in module vcs.manageElements)}\index{xmtics2 (in module vcs.manageElements)}\index{ymtics1 (in module vcs.manageElements)}\index{ymtics2 (in module vcs.manageElements)}\index{xticlabels1 (in module vcs.manageElements)}\index{xticlabels2 (in module vcs.manageElements)}\index{yticlabels1 (in module vcs.manageElements)}\index{yticlabels2 (in module vcs.manageElements)}\index{projection (in module vcs.manageElements)}\index{datawc\_x1 (in module vcs.manageElements)}\index{datawc\_x2 (in module vcs.manageElements)}\index{datawc\_y1 (in module vcs.manageElements)}\index{datawc\_y2 (in module vcs.manageElements)}\index{datawc\_timeunits (in module vcs.manageElements)}\index{datawc\_calendar (in module vcs.manageElements)}) -- sets the y/x ratio ,if passed as a string with `t' at the end, will aslo moves the ticks

\item {} 
\textbf{\texttt{xaxisconvert}} (\href{https://docs.python.org/2/library/functions.html\#str}{\emph{\texttt{str}}}) -- (Ex: `linear') converting xaxis linear/log/log10/ln/exp/area\_wt

\item {} 
\textbf{\texttt{yaxisconvert}} (\href{https://docs.python.org/2/library/functions.html\#str}{\emph{\texttt{str}}}) -- (Ex: `linear') converting yaxis linear/log/log10/ln/exp/area\_wt

\item {} 
\textbf{\texttt{new\_GM\_name}} (\href{https://docs.python.org/2/library/functions.html\#str}{\emph{\texttt{str}}}) -- (Ex: `my\_awesome\_gm') name of the new graphics method object. If no name is given, then one will be created for use.

\item {} 
\textbf{\texttt{source\_GM\_name}} -- (Ex: `default') copy the contents of the source object to the newly created one. If no name is given, then the `default' graphics methond contents is copied over to the new object.

\end{itemize}

\item[{Returns}] \leavevmode
A XYvsY graphics method object

\item[{Return type}] \leavevmode
{\hyperref[vcs/graphics/unified1D:vcs.unified1D.G1d]{\sphinxcrossref{vcs.unified1D.G1d}}}

\end{description}\end{quote}

\end{fulllineitems}

\index{createyxvsx() (in module vcs.manageElements)}

\begin{fulllineitems}
\phantomsection\label{vcs/misc/manageElements:vcs.manageElements.createyxvsx}\pysiglinewithargsret{\sphinxcode{vcs.manageElements.}\sphinxbfcode{createyxvsx}}{\emph{name=None}, \emph{source='default'}}{}
Create a new yxvsx graphics method given the the name and the existing
yxvsx graphics method to copy the attributes from. If no existing
yxvsx graphics method is given, then the default yxvsx graphics method will be used as the graphics method
to which the attributes will be copied from.

\begin{notice}{note}{Note:}
If the name provided already exists, then an error will be returned. graphics method
names must be unique.
\end{notice}
\begin{quote}\begin{description}
\item[{Example}] \leavevmode
\begin{Verbatim}[commandchars=\\\{\}]
\PYG{g+gp}{\PYGZgt{}\PYGZgt{}\PYGZgt{} }\PYG{n}{vcs}\PYG{o}{.}\PYG{n}{show}\PYG{p}{(}\PYG{l+s+s1}{\PYGZsq{}}\PYG{l+s+s1}{yxvsx}\PYG{l+s+s1}{\PYGZsq{}}\PYG{p}{)} \PYG{c+c1}{\PYGZsh{} show all available yxvsx}
\PYG{g+go}{*******************Yxvsx Names List**********************}
\PYG{g+gp}{...}
\PYG{g+go}{*******************End Yxvsx Names List**********************}
\PYG{g+gp}{\PYGZgt{}\PYGZgt{}\PYGZgt{} }\PYG{n}{ex}\PYG{o}{=}\PYG{n}{vcs}\PYG{o}{.}\PYG{n}{createyxvsx}\PYG{p}{(}\PYG{l+s+s1}{\PYGZsq{}}\PYG{l+s+s1}{yxvsx\PYGZus{}ex1}\PYG{l+s+s1}{\PYGZsq{}}\PYG{p}{)} \PYG{c+c1}{\PYGZsh{} Create yxvsx \PYGZsq{}yxvsx\PYGZus{}ex1\PYGZsq{} that inherits from \PYGZsq{}default\PYGZsq{}}
\PYG{g+gp}{\PYGZgt{}\PYGZgt{}\PYGZgt{} }\PYG{n}{vcs}\PYG{o}{.}\PYG{n}{listelements}\PYG{p}{(}\PYG{l+s+s1}{\PYGZsq{}}\PYG{l+s+s1}{yxvsx}\PYG{l+s+s1}{\PYGZsq{}}\PYG{p}{)} \PYG{c+c1}{\PYGZsh{} should now contain the \PYGZsq{}yxvsx\PYGZus{}ex1\PYGZsq{} yxvsx}
\PYG{g+go}{[...\PYGZsq{}yxvsx\PYGZus{}ex1\PYGZsq{}...]}
\end{Verbatim}

\item[{Parameters}] \leavevmode\begin{itemize}
\item {} 
\textbf{\texttt{name}} (\href{https://docs.python.org/2/library/functions.html\#str}{\emph{\texttt{str}}}) -- The name of the created object

\item {} 
\textbf{\texttt{source}} (\emph{\texttt{a yxvsy or a string name of a yxvsy}}) -- The object to inherit from

\item {} 
\textbf{\texttt{xaxis}} (\emph{\texttt{cdms2.axis.TransientAxis}}) -- Axis object to replace the slab -1 dim axis

\item {} 
\textbf{\texttt{yaxis}} (\emph{\texttt{cdms2.axis.TransientAxis}}) -- Axis object to replace the slab -2 dim axis, only if slab has more than 1D

\item {} 
\textbf{\texttt{zaxis}} (\emph{\texttt{cdms2.axis.TransientAxis}}) -- Axis object to replace the slab -3 dim axis, only if slab has more than 2D

\item {} 
\textbf{\texttt{taxis}} (\emph{\texttt{cdms2.axis.TransientAxis}}) -- Axis object to replace the slab -4 dim axis, only if slab has more than 3D

\item {} 
\textbf{\texttt{waxis}} (\emph{\texttt{cdms2.axis.TransientAxis}}) -- Axis object to replace the slab -5 dim axis, only if slab has more than 4D

\item {} 
\textbf{\texttt{xrev}} (\href{https://docs.python.org/2/library/functions.html\#bool}{\emph{\texttt{bool}}}) -- reverse x axis

\item {} 
\textbf{\texttt{yrev}} (\href{https://docs.python.org/2/library/functions.html\#bool}{\emph{\texttt{bool}}}) -- reverse y axis, only if slab has more than 1D

\item {} 
\textbf{\texttt{xarray}} (\href{https://docs.python.org/2/library/array.html\#module-array}{\emph{\texttt{array}}}) -- Values to use instead of x axis

\item {} 
\textbf{\texttt{yarray}} (\href{https://docs.python.org/2/library/array.html\#module-array}{\emph{\texttt{array}}}) -- Values to use instead of y axis, only if var has more than 1D

\item {} 
\textbf{\texttt{zarray}} (\href{https://docs.python.org/2/library/array.html\#module-array}{\emph{\texttt{array}}}) -- Values to use instead of z axis, only if var has more than 2D

\item {} 
\textbf{\texttt{tarray}} (\href{https://docs.python.org/2/library/array.html\#module-array}{\emph{\texttt{array}}}) -- Values to use instead of t axis, only if var has more than 3D

\item {} 
\textbf{\texttt{warray}} (\href{https://docs.python.org/2/library/array.html\#module-array}{\emph{\texttt{array}}}) -- Values to use instead of w axis, only if var has more than 4D

\item {} 
\textbf{\texttt{continents}} (\href{https://docs.python.org/2/library/functions.html\#int}{\emph{\texttt{int}}}) -- continents type number

\item {} 
\textbf{\texttt{name}} -- replaces variable name on plot

\item {} 
\textbf{\texttt{time}} (\emph{\texttt{A cdtime object}}) -- replaces time name on plot

\item {} 
\textbf{\texttt{units}} (\href{https://docs.python.org/2/library/functions.html\#str}{\emph{\texttt{str}}}) -- replaces units value on plot

\item {} 
\textbf{\texttt{ymd}} (\href{https://docs.python.org/2/library/functions.html\#str}{\emph{\texttt{str}}}) -- replaces year/month/day on plot

\item {} 
\textbf{\texttt{hms}} (\href{https://docs.python.org/2/library/functions.html\#str}{\emph{\texttt{str}}}) -- replaces hh/mm/ss on plot

\item {} 
\textbf{\texttt{file\_comment}} (\href{https://docs.python.org/2/library/functions.html\#str}{\emph{\texttt{str}}}) -- replaces file\_comment on plot

\item {} 
\textbf{\texttt{xbounds}} (\href{https://docs.python.org/2/library/array.html\#module-array}{\emph{\texttt{array}}}) -- Values to use instead of x axis bounds values

\item {} 
\textbf{\texttt{ybounds}} (\href{https://docs.python.org/2/library/array.html\#module-array}{\emph{\texttt{array}}}) -- Values to use instead of y axis bounds values (if exist)

\item {} 
\textbf{\texttt{xname}} (\href{https://docs.python.org/2/library/functions.html\#str}{\emph{\texttt{str}}}) -- replace xaxis name on plot

\item {} 
\textbf{\texttt{yname}} (\href{https://docs.python.org/2/library/functions.html\#str}{\emph{\texttt{str}}}) -- replace yaxis name on plot (if exists)

\item {} 
\textbf{\texttt{zname}} (\href{https://docs.python.org/2/library/functions.html\#str}{\emph{\texttt{str}}}) -- replace zaxis name on plot (if exists)

\item {} 
\textbf{\texttt{tname}} (\href{https://docs.python.org/2/library/functions.html\#str}{\emph{\texttt{str}}}) -- replace taxis name on plot (if exists)

\item {} 
\textbf{\texttt{wname}} (\href{https://docs.python.org/2/library/functions.html\#str}{\emph{\texttt{str}}}) -- replace waxis name on plot (if exists)

\item {} 
\textbf{\texttt{xunits}} (\href{https://docs.python.org/2/library/functions.html\#str}{\emph{\texttt{str}}}) -- replace xaxis units on plot

\item {} 
\textbf{\texttt{yunits}} (\href{https://docs.python.org/2/library/functions.html\#str}{\emph{\texttt{str}}}) -- replace yaxis units on plot (if exists)

\item {} 
\textbf{\texttt{zunits}} (\href{https://docs.python.org/2/library/functions.html\#str}{\emph{\texttt{str}}}) -- replace zaxis units on plot (if exists)

\item {} 
\textbf{\texttt{tunits}} (\href{https://docs.python.org/2/library/functions.html\#str}{\emph{\texttt{str}}}) -- replace taxis units on plot (if exists)

\item {} 
\textbf{\texttt{wunits}} (\href{https://docs.python.org/2/library/functions.html\#str}{\emph{\texttt{str}}}) -- replace waxis units on plot (if exists)

\item {} 
\textbf{\texttt{xweights}} (\href{https://docs.python.org/2/library/array.html\#module-array}{\emph{\texttt{array}}}) -- replace xaxis weights used for computing mean

\item {} 
\textbf{\texttt{yweights}} (\href{https://docs.python.org/2/library/array.html\#module-array}{\emph{\texttt{array}}}) -- replace xaxis weights used for computing mean

\item {} 
\textbf{\texttt{comment1}} (\href{https://docs.python.org/2/library/functions.html\#str}{\emph{\texttt{str}}}) -- replaces comment1 on plot

\item {} 
\textbf{\texttt{comment2}} (\href{https://docs.python.org/2/library/functions.html\#str}{\emph{\texttt{str}}}) -- replaces comment2 on plot

\item {} 
\textbf{\texttt{comment3}} (\href{https://docs.python.org/2/library/functions.html\#str}{\emph{\texttt{str}}}) -- replaces comment3 on plot

\item {} 
\textbf{\texttt{comment4}} (\href{https://docs.python.org/2/library/functions.html\#str}{\emph{\texttt{str}}}) -- replaces comment4 on plot

\item {} 
\textbf{\texttt{long\_name}} (\href{https://docs.python.org/2/library/functions.html\#str}{\emph{\texttt{str}}}) -- replaces long\_name on plot

\item {} 
\textbf{\texttt{grid}} (\emph{\texttt{cdms2.grid.TransientRectGrid}}) -- replaces array grid (if exists)

\item {} 
\textbf{\texttt{bg}} (\emph{\texttt{bool/int}}) -- plots in background mode

\item {} 
\textbf{\texttt{ratio}} (\index{xmtics1 (in module vcs.manageElements)}\index{xmtics2 (in module vcs.manageElements)}\index{ymtics1 (in module vcs.manageElements)}\index{ymtics2 (in module vcs.manageElements)}\index{xticlabels1 (in module vcs.manageElements)}\index{xticlabels2 (in module vcs.manageElements)}\index{yticlabels1 (in module vcs.manageElements)}\index{yticlabels2 (in module vcs.manageElements)}\index{projection (in module vcs.manageElements)}\index{datawc\_x1 (in module vcs.manageElements)}\index{datawc\_x2 (in module vcs.manageElements)}\index{datawc\_y1 (in module vcs.manageElements)}\index{datawc\_y2 (in module vcs.manageElements)}\index{datawc\_timeunits (in module vcs.manageElements)}\index{datawc\_calendar (in module vcs.manageElements)}) -- sets the y/x ratio ,if passed as a string with `t' at the end, will aslo moves the ticks

\item {} 
\textbf{\texttt{xaxisconvert}} (\href{https://docs.python.org/2/library/functions.html\#str}{\emph{\texttt{str}}}) -- (Ex: `linear') converting xaxis linear/log/log10/ln/exp/area\_wt

\item {} 
\textbf{\texttt{yaxisconvert}} (\href{https://docs.python.org/2/library/functions.html\#str}{\emph{\texttt{str}}}) -- (Ex: `linear') converting yaxis linear/log/log10/ln/exp/area\_wt

\item {} 
\textbf{\texttt{new\_GM\_name}} (\href{https://docs.python.org/2/library/functions.html\#str}{\emph{\texttt{str}}}) -- (Ex: `my\_awesome\_gm') name of the new graphics method object. If no name is given, then one will be created for use.

\item {} 
\textbf{\texttt{source\_GM\_name}} -- (Ex: `default') copy the contents of the source object to the newly created one. If no name is given, then the `default' graphics methond contents is copied over to the new object.

\end{itemize}

\item[{Returns}] \leavevmode
A YXvsX graphics method object

\item[{Return type}] \leavevmode
{\hyperref[vcs/graphics/unified1D:vcs.unified1D.G1d]{\sphinxcrossref{vcs.unified1D.G1d}}}

\end{description}\end{quote}

\end{fulllineitems}

\index{get3d\_dual\_scalar() (in module vcs.manageElements)}

\begin{fulllineitems}
\phantomsection\label{vcs/misc/manageElements:vcs.manageElements.get3d_dual_scalar}\pysiglinewithargsret{\sphinxcode{vcs.manageElements.}\sphinxbfcode{get3d\_dual\_scalar}}{\emph{Gfdv3d\_name\_src='default'}}{}
VCS contains a list of graphics methods. This function will create a
dv3d class object from an existing VCS dv3d graphics method. If
no dv3d name is given, then dv3d `default' will be used.

\begin{notice}{note}{Note:}
VCS does not allow the modification of `default' attribute sets.
However, a `default' attribute set that has been copied under a
different name can be modified. (See the {\hyperref[vcs/misc/manageElements:vcs.manageElements.create3d_dual_scalar]{\sphinxcrossref{\sphinxcode{vcs.manageElements.create3d\_dual\_scalar()}}}} function.)
\end{notice}
\begin{quote}\begin{description}
\item[{Example}] \leavevmode
\begin{Verbatim}[commandchars=\\\{\}]
\PYG{g+gp}{\PYGZgt{}\PYGZgt{}\PYGZgt{} }\PYG{n}{a}\PYG{o}{=}\PYG{n}{vcs}\PYG{o}{.}\PYG{n}{init}\PYG{p}{(}\PYG{p}{)}
\PYG{g+gp}{\PYGZgt{}\PYGZgt{}\PYGZgt{} }\PYG{n}{vcs}\PYG{o}{.}\PYG{n}{listelements}\PYG{p}{(}\PYG{l+s+s1}{\PYGZsq{}}\PYG{l+s+s1}{3d\PYGZus{}dual\PYGZus{}scalar}\PYG{l+s+s1}{\PYGZsq{}}\PYG{p}{)} \PYG{c+c1}{\PYGZsh{} Show all the existing 3d\PYGZus{}dual\PYGZus{}scalar graphics methods}
\PYG{g+go}{[...]}
\PYG{g+gp}{\PYGZgt{}\PYGZgt{}\PYGZgt{} }\PYG{n}{ex}\PYG{o}{=}\PYG{n}{vcs}\PYG{o}{.}\PYG{n}{get3d\PYGZus{}dual\PYGZus{}scalar}\PYG{p}{(}\PYG{p}{)}  \PYG{c+c1}{\PYGZsh{} instance of \PYGZsq{}default\PYGZsq{} 3d\PYGZus{}dual\PYGZus{}scalar graphics method}
\PYG{g+gp}{\PYGZgt{}\PYGZgt{}\PYGZgt{} }\PYG{k+kn}{import} \PYG{n+nn}{cdms2} \PYG{c+c1}{\PYGZsh{} Need cdms2 to create a slab}
\PYG{g+gp}{\PYGZgt{}\PYGZgt{}\PYGZgt{} }\PYG{n}{f} \PYG{o}{=} \PYG{n}{cdms2}\PYG{o}{.}\PYG{n}{open}\PYG{p}{(}\PYG{n}{vcs}\PYG{o}{.}\PYG{n}{sample\PYGZus{}data}\PYG{o}{+}\PYG{l+s+s1}{\PYGZsq{}}\PYG{l+s+s1}{/clt.nc}\PYG{l+s+s1}{\PYGZsq{}}\PYG{p}{)} \PYG{c+c1}{\PYGZsh{} use cdms2 to open a data file}
\PYG{g+gp}{\PYGZgt{}\PYGZgt{}\PYGZgt{} }\PYG{n}{slab1} \PYG{o}{=} \PYG{n}{f}\PYG{p}{(}\PYG{l+s+s1}{\PYGZsq{}}\PYG{l+s+s1}{u}\PYG{l+s+s1}{\PYGZsq{}}\PYG{p}{)} \PYG{c+c1}{\PYGZsh{} use the data file to create a cdms2 slab}
\PYG{g+gp}{\PYGZgt{}\PYGZgt{}\PYGZgt{} }\PYG{n}{slab2} \PYG{o}{=} \PYG{n}{f}\PYG{p}{(}\PYG{l+s+s1}{\PYGZsq{}}\PYG{l+s+s1}{v}\PYG{l+s+s1}{\PYGZsq{}}\PYG{p}{)} \PYG{c+c1}{\PYGZsh{} need 2 slabs, so get another}
\PYG{g+gp}{\PYGZgt{}\PYGZgt{}\PYGZgt{} }\PYG{n}{a}\PYG{o}{.}\PYG{n}{plot}\PYG{p}{(}\PYG{n}{ex}\PYG{p}{,} \PYG{n}{slab1}\PYG{p}{,} \PYG{n}{slab2}\PYG{p}{)} \PYG{c+c1}{\PYGZsh{} plot using specified 3d\PYGZus{}dual\PYGZus{}scalar object}
\PYG{g+go}{\PYGZlt{}vcs.displayplot.Dp ...\PYGZgt{}}
\end{Verbatim}

\item[{Parameters}] \leavevmode
\textbf{\texttt{Gfdv3d\_name\_src}} (\href{https://docs.python.org/2/library/functions.html\#str}{\emph{\texttt{str}}}) -- String name of an existing 3d\_dual\_scalar VCS object

\item[{Returns}] \leavevmode
A pre-existing 3d\_dual\_scalar VCS object

\item[{Return type}] \leavevmode
vcs.dv3d.Gf3DDualScalar

\end{description}\end{quote}

\end{fulllineitems}

\index{get3d\_scalar() (in module vcs.manageElements)}

\begin{fulllineitems}
\phantomsection\label{vcs/misc/manageElements:vcs.manageElements.get3d_scalar}\pysiglinewithargsret{\sphinxcode{vcs.manageElements.}\sphinxbfcode{get3d\_scalar}}{\emph{Gfdv3d\_name\_src='default'}}{}
VCS contains a list of graphics methods. This function will create a
dv3d class object from an existing VCS dv3d graphics method. If
no dv3d name is given, then dv3d `default' will be used.

\begin{notice}{note}{Note:}
VCS does not allow the modification of `default' attribute sets.
However, a `default' attribute set that has been copied under a
different name can be modified. (See the {\hyperref[vcs/misc/manageElements:vcs.manageElements.create3d_scalar]{\sphinxcrossref{\sphinxcode{vcs.manageElements.create3d\_scalar()}}}} function.)
\end{notice}
\begin{quote}\begin{description}
\item[{Example}] \leavevmode
\begin{Verbatim}[commandchars=\\\{\}]
\PYG{g+gp}{\PYGZgt{}\PYGZgt{}\PYGZgt{} }\PYG{n}{a}\PYG{o}{=}\PYG{n}{vcs}\PYG{o}{.}\PYG{n}{init}\PYG{p}{(}\PYG{p}{)}
\PYG{g+gp}{\PYGZgt{}\PYGZgt{}\PYGZgt{} }\PYG{n}{vcs}\PYG{o}{.}\PYG{n}{listelements}\PYG{p}{(}\PYG{l+s+s1}{\PYGZsq{}}\PYG{l+s+s1}{3d\PYGZus{}scalar}\PYG{l+s+s1}{\PYGZsq{}}\PYG{p}{)} \PYG{c+c1}{\PYGZsh{} Show all the existing 3d\PYGZus{}scalar graphics methods}
\PYG{g+go}{[...]}
\PYG{g+gp}{\PYGZgt{}\PYGZgt{}\PYGZgt{} }\PYG{n}{ex}\PYG{o}{=}\PYG{n}{vcs}\PYG{o}{.}\PYG{n}{get3d\PYGZus{}scalar}\PYG{p}{(}\PYG{p}{)}  \PYG{c+c1}{\PYGZsh{} instance of \PYGZsq{}default\PYGZsq{} 3d\PYGZus{}scalar graphics method}
\PYG{g+gp}{\PYGZgt{}\PYGZgt{}\PYGZgt{} }\PYG{k+kn}{import} \PYG{n+nn}{cdms2} \PYG{c+c1}{\PYGZsh{} Need cdms2 to create a slab}
\PYG{g+gp}{\PYGZgt{}\PYGZgt{}\PYGZgt{} }\PYG{n}{f} \PYG{o}{=} \PYG{n}{cdms2}\PYG{o}{.}\PYG{n}{open}\PYG{p}{(}\PYG{n}{vcs}\PYG{o}{.}\PYG{n}{sample\PYGZus{}data}\PYG{o}{+}\PYG{l+s+s1}{\PYGZsq{}}\PYG{l+s+s1}{/clt.nc}\PYG{l+s+s1}{\PYGZsq{}}\PYG{p}{)} \PYG{c+c1}{\PYGZsh{} use cdms2 to open a data file}
\PYG{g+gp}{\PYGZgt{}\PYGZgt{}\PYGZgt{} }\PYG{n}{slab1} \PYG{o}{=} \PYG{n}{f}\PYG{p}{(}\PYG{l+s+s1}{\PYGZsq{}}\PYG{l+s+s1}{u}\PYG{l+s+s1}{\PYGZsq{}}\PYG{p}{)} \PYG{c+c1}{\PYGZsh{} use the data file to create a cdms2 slab}
\PYG{g+gp}{\PYGZgt{}\PYGZgt{}\PYGZgt{} }\PYG{n}{a}\PYG{o}{.}\PYG{n}{plot}\PYG{p}{(}\PYG{n}{ex}\PYG{p}{,} \PYG{n}{slab1}\PYG{p}{)} \PYG{c+c1}{\PYGZsh{} plot using specified 3d\PYGZus{}scalar object}
\PYG{g+go}{\PYGZlt{}vcs.displayplot.Dp ...\PYGZgt{}}
\end{Verbatim}

\item[{Parameters}] \leavevmode
\textbf{\texttt{Gfdv3d\_name\_src}} (\href{https://docs.python.org/2/library/functions.html\#str}{\emph{\texttt{str}}}) -- String name of an existing 3d\_scalar VCS object.

\item[{Returns}] \leavevmode
A pre-existing 3d\_scalar VCS object

\item[{Return type}] \leavevmode
vcs.dv3d.Gf3Dscalar

\end{description}\end{quote}

\end{fulllineitems}

\index{get3d\_vector() (in module vcs.manageElements)}

\begin{fulllineitems}
\phantomsection\label{vcs/misc/manageElements:vcs.manageElements.get3d_vector}\pysiglinewithargsret{\sphinxcode{vcs.manageElements.}\sphinxbfcode{get3d\_vector}}{\emph{Gfdv3d\_name\_src='default'}}{}
VCS contains a list of graphics methods. This function will create a
dv3d class object from an existing VCS dv3d graphics method. If
no dv3d name is given, then dv3d `default' will be used.

\begin{notice}{note}{Note:}
VCS does not allow the modification of `default' attribute sets.
However, a `default' attribute set that has been copied under a
different name can be modified. (See the {\hyperref[vcs/misc/manageElements:vcs.manageElements.create3d_vector]{\sphinxcrossref{\sphinxcode{vcs.manageElements.create3d\_vector()}}}} function.)
\end{notice}
\begin{quote}\begin{description}
\item[{Example}] \leavevmode
\begin{Verbatim}[commandchars=\\\{\}]
\PYG{g+gp}{\PYGZgt{}\PYGZgt{}\PYGZgt{} }\PYG{n}{a}\PYG{o}{=}\PYG{n}{vcs}\PYG{o}{.}\PYG{n}{init}\PYG{p}{(}\PYG{p}{)}
\PYG{g+gp}{\PYGZgt{}\PYGZgt{}\PYGZgt{} }\PYG{n}{vcs}\PYG{o}{.}\PYG{n}{listelements}\PYG{p}{(}\PYG{l+s+s1}{\PYGZsq{}}\PYG{l+s+s1}{3d\PYGZus{}vector}\PYG{l+s+s1}{\PYGZsq{}}\PYG{p}{)} \PYG{c+c1}{\PYGZsh{} Show all the existing 3d\PYGZus{}vector graphics methods}
\PYG{g+go}{[...]}
\PYG{g+gp}{\PYGZgt{}\PYGZgt{}\PYGZgt{} }\PYG{n}{ex}\PYG{o}{=}\PYG{n}{vcs}\PYG{o}{.}\PYG{n}{get3d\PYGZus{}vector}\PYG{p}{(}\PYG{p}{)}  \PYG{c+c1}{\PYGZsh{} instance of \PYGZsq{}default\PYGZsq{} 3d\PYGZus{}vector graphics method}
\PYG{g+gp}{\PYGZgt{}\PYGZgt{}\PYGZgt{} }\PYG{k+kn}{import} \PYG{n+nn}{cdms2} \PYG{c+c1}{\PYGZsh{} Need cdms2 to create a slab}
\PYG{g+gp}{\PYGZgt{}\PYGZgt{}\PYGZgt{} }\PYG{n}{f} \PYG{o}{=} \PYG{n}{cdms2}\PYG{o}{.}\PYG{n}{open}\PYG{p}{(}\PYG{n}{vcs}\PYG{o}{.}\PYG{n}{sample\PYGZus{}data}\PYG{o}{+}\PYG{l+s+s1}{\PYGZsq{}}\PYG{l+s+s1}{/clt.nc}\PYG{l+s+s1}{\PYGZsq{}}\PYG{p}{)} \PYG{c+c1}{\PYGZsh{} use cdms2 to open a data file}
\PYG{g+gp}{\PYGZgt{}\PYGZgt{}\PYGZgt{} }\PYG{n}{slab1} \PYG{o}{=} \PYG{n}{f}\PYG{p}{(}\PYG{l+s+s1}{\PYGZsq{}}\PYG{l+s+s1}{u}\PYG{l+s+s1}{\PYGZsq{}}\PYG{p}{)} \PYG{c+c1}{\PYGZsh{} use the data file to create a cdms2 slab}
\PYG{g+gp}{\PYGZgt{}\PYGZgt{}\PYGZgt{} }\PYG{n}{slab2} \PYG{o}{=} \PYG{n}{f}\PYG{p}{(}\PYG{l+s+s1}{\PYGZsq{}}\PYG{l+s+s1}{v}\PYG{l+s+s1}{\PYGZsq{}}\PYG{p}{)} \PYG{c+c1}{\PYGZsh{} need 2 slabs, so get another}
\PYG{g+gp}{\PYGZgt{}\PYGZgt{}\PYGZgt{} }\PYG{n}{a}\PYG{o}{.}\PYG{n}{plot}\PYG{p}{(}\PYG{n}{ex}\PYG{p}{,} \PYG{n}{slab1}\PYG{p}{,} \PYG{n}{slab2}\PYG{p}{)} \PYG{c+c1}{\PYGZsh{} plot using specified 3d\PYGZus{}vector object}
\PYG{g+go}{\PYGZlt{}vcs.displayplot.Dp ...\PYGZgt{}}
\end{Verbatim}

\item[{Parameters}] \leavevmode
\textbf{\texttt{Gfdv3d\_name\_src}} (\href{https://docs.python.org/2/library/functions.html\#str}{\emph{\texttt{str}}}) -- String name of an existing 3d\_vector VCS object

\item[{Returns}] \leavevmode
A pre-existing 3d\_vector VCS object

\item[{Return type}] \leavevmode
vcs.dv3d.Gf3Dvector

\end{description}\end{quote}

\end{fulllineitems}

\index{getboxfill() (in module vcs.manageElements)}

\begin{fulllineitems}
\phantomsection\label{vcs/misc/manageElements:vcs.manageElements.getboxfill}\pysiglinewithargsret{\sphinxcode{vcs.manageElements.}\sphinxbfcode{getboxfill}}{\emph{Gfb\_name\_src='default'}}{}
VCS contains a list of graphics methods. This function will create a
boxfill class object from an existing VCS boxfill graphics method. If
no boxfill name is given, then boxfill `default' will be used.

\begin{notice}{note}{Note:}
VCS does not allow the modification of `default' attribute sets.
However, a `default' attribute set that has been copied under a
different name can be modified. (See the {\hyperref[vcs/misc/manageElements:vcs.manageElements.createboxfill]{\sphinxcrossref{\sphinxcode{vcs.manageElements.createboxfill()}}}} function.)
\end{notice}
\begin{quote}\begin{description}
\item[{Example}] \leavevmode
\begin{Verbatim}[commandchars=\\\{\}]
\PYG{g+gp}{\PYGZgt{}\PYGZgt{}\PYGZgt{} }\PYG{n}{a}\PYG{o}{=}\PYG{n}{vcs}\PYG{o}{.}\PYG{n}{init}\PYG{p}{(}\PYG{p}{)}
\PYG{g+gp}{\PYGZgt{}\PYGZgt{}\PYGZgt{} }\PYG{n}{vcs}\PYG{o}{.}\PYG{n}{listelements}\PYG{p}{(}\PYG{l+s+s1}{\PYGZsq{}}\PYG{l+s+s1}{boxfill}\PYG{l+s+s1}{\PYGZsq{}}\PYG{p}{)} \PYG{c+c1}{\PYGZsh{} Show all the existing boxfill graphics methods}
\PYG{g+go}{[...]}
\PYG{g+gp}{\PYGZgt{}\PYGZgt{}\PYGZgt{} }\PYG{n}{ex}\PYG{o}{=}\PYG{n}{vcs}\PYG{o}{.}\PYG{n}{getboxfill}\PYG{p}{(}\PYG{p}{)}  \PYG{c+c1}{\PYGZsh{} instance of \PYGZsq{}default\PYGZsq{} boxfill graphics method}
\PYG{g+gp}{\PYGZgt{}\PYGZgt{}\PYGZgt{} }\PYG{k+kn}{import} \PYG{n+nn}{cdms2} \PYG{c+c1}{\PYGZsh{} Need cdms2 to create a slab}
\PYG{g+gp}{\PYGZgt{}\PYGZgt{}\PYGZgt{} }\PYG{n}{f} \PYG{o}{=} \PYG{n}{cdms2}\PYG{o}{.}\PYG{n}{open}\PYG{p}{(}\PYG{n}{vcs}\PYG{o}{.}\PYG{n}{sample\PYGZus{}data}\PYG{o}{+}\PYG{l+s+s1}{\PYGZsq{}}\PYG{l+s+s1}{/clt.nc}\PYG{l+s+s1}{\PYGZsq{}}\PYG{p}{)} \PYG{c+c1}{\PYGZsh{} use cdms2 to open a data file}
\PYG{g+gp}{\PYGZgt{}\PYGZgt{}\PYGZgt{} }\PYG{n}{slab1} \PYG{o}{=} \PYG{n}{f}\PYG{p}{(}\PYG{l+s+s1}{\PYGZsq{}}\PYG{l+s+s1}{u}\PYG{l+s+s1}{\PYGZsq{}}\PYG{p}{)} \PYG{c+c1}{\PYGZsh{} use the data file to create a cdms2 slab}
\PYG{g+gp}{\PYGZgt{}\PYGZgt{}\PYGZgt{} }\PYG{n}{a}\PYG{o}{.}\PYG{n}{boxfill}\PYG{p}{(}\PYG{n}{ex}\PYG{p}{,} \PYG{n}{slab1}\PYG{p}{)} \PYG{c+c1}{\PYGZsh{} plot using specified boxfill object}
\PYG{g+go}{\PYGZlt{}vcs.displayplot.Dp ...\PYGZgt{}}
\PYG{g+gp}{\PYGZgt{}\PYGZgt{}\PYGZgt{} }\PYG{n}{ex2}\PYG{o}{=}\PYG{n}{vcs}\PYG{o}{.}\PYG{n}{getboxfill}\PYG{p}{(}\PYG{l+s+s1}{\PYGZsq{}}\PYG{l+s+s1}{polar}\PYG{l+s+s1}{\PYGZsq{}}\PYG{p}{)}  \PYG{c+c1}{\PYGZsh{} instance of \PYGZsq{}polar\PYGZsq{} boxfill graphics method}
\PYG{g+gp}{\PYGZgt{}\PYGZgt{}\PYGZgt{} }\PYG{n}{a}\PYG{o}{.}\PYG{n}{boxfill}\PYG{p}{(}\PYG{n}{ex2}\PYG{p}{,} \PYG{n}{slab1}\PYG{p}{)} \PYG{c+c1}{\PYGZsh{} plot using specified boxfill object}
\PYG{g+go}{\PYGZlt{}vcs.displayplot.Dp ...\PYGZgt{}}
\end{Verbatim}

\item[{Parameters}] \leavevmode\begin{itemize}
\item {} 
\textbf{\texttt{Gfb\_name\_src}} (\href{https://docs.python.org/2/library/functions.html\#str}{\emph{\texttt{str}}}) -- String name of an existing boxfill VCS object

\item {} 
\textbf{\texttt{xaxis}} (\emph{\texttt{cdms2.axis.TransientAxis}}) -- Axis object to replace the slab -1 dim axis

\item {} 
\textbf{\texttt{yaxis}} (\emph{\texttt{cdms2.axis.TransientAxis}}) -- Axis object to replace the slab -2 dim axis, only if slab has more than 1D

\item {} 
\textbf{\texttt{zaxis}} (\emph{\texttt{cdms2.axis.TransientAxis}}) -- Axis object to replace the slab -3 dim axis, only if slab has more than 2D

\item {} 
\textbf{\texttt{taxis}} (\emph{\texttt{cdms2.axis.TransientAxis}}) -- Axis object to replace the slab -4 dim axis, only if slab has more than 3D

\item {} 
\textbf{\texttt{waxis}} (\emph{\texttt{cdms2.axis.TransientAxis}}) -- Axis object to replace the slab -5 dim axis, only if slab has more than 4D

\item {} 
\textbf{\texttt{xrev}} (\href{https://docs.python.org/2/library/functions.html\#bool}{\emph{\texttt{bool}}}) -- reverse x axis

\item {} 
\textbf{\texttt{yrev}} (\href{https://docs.python.org/2/library/functions.html\#bool}{\emph{\texttt{bool}}}) -- reverse y axis, only if slab has more than 1D

\item {} 
\textbf{\texttt{xarray}} (\href{https://docs.python.org/2/library/array.html\#module-array}{\emph{\texttt{array}}}) -- Values to use instead of x axis

\item {} 
\textbf{\texttt{yarray}} (\href{https://docs.python.org/2/library/array.html\#module-array}{\emph{\texttt{array}}}) -- Values to use instead of y axis, only if var has more than 1D

\item {} 
\textbf{\texttt{zarray}} (\href{https://docs.python.org/2/library/array.html\#module-array}{\emph{\texttt{array}}}) -- Values to use instead of z axis, only if var has more than 2D

\item {} 
\textbf{\texttt{tarray}} (\href{https://docs.python.org/2/library/array.html\#module-array}{\emph{\texttt{array}}}) -- Values to use instead of t axis, only if var has more than 3D

\item {} 
\textbf{\texttt{warray}} (\href{https://docs.python.org/2/library/array.html\#module-array}{\emph{\texttt{array}}}) -- Values to use instead of w axis, only if var has more than 4D

\item {} 
\textbf{\texttt{continents}} (\href{https://docs.python.org/2/library/functions.html\#int}{\emph{\texttt{int}}}) -- continents type number

\item {} 
\textbf{\texttt{name}} (\href{https://docs.python.org/2/library/functions.html\#str}{\emph{\texttt{str}}}) -- replaces variable name on plot

\item {} 
\textbf{\texttt{time}} (\emph{\texttt{A cdtime object}}) -- replaces time name on plot

\item {} 
\textbf{\texttt{units}} (\href{https://docs.python.org/2/library/functions.html\#str}{\emph{\texttt{str}}}) -- replaces units value on plot

\item {} 
\textbf{\texttt{ymd}} (\href{https://docs.python.org/2/library/functions.html\#str}{\emph{\texttt{str}}}) -- replaces year/month/day on plot

\item {} 
\textbf{\texttt{hms}} (\href{https://docs.python.org/2/library/functions.html\#str}{\emph{\texttt{str}}}) -- replaces hh/mm/ss on plot

\item {} 
\textbf{\texttt{file\_comment}} (\href{https://docs.python.org/2/library/functions.html\#str}{\emph{\texttt{str}}}) -- replaces file\_comment on plot

\item {} 
\textbf{\texttt{xbounds}} (\href{https://docs.python.org/2/library/array.html\#module-array}{\emph{\texttt{array}}}) -- Values to use instead of x axis bounds values

\item {} 
\textbf{\texttt{ybounds}} (\href{https://docs.python.org/2/library/array.html\#module-array}{\emph{\texttt{array}}}) -- Values to use instead of y axis bounds values (if exist)

\item {} 
\textbf{\texttt{xname}} (\href{https://docs.python.org/2/library/functions.html\#str}{\emph{\texttt{str}}}) -- replace xaxis name on plot

\item {} 
\textbf{\texttt{yname}} (\href{https://docs.python.org/2/library/functions.html\#str}{\emph{\texttt{str}}}) -- replace yaxis name on plot (if exists)

\item {} 
\textbf{\texttt{zname}} (\href{https://docs.python.org/2/library/functions.html\#str}{\emph{\texttt{str}}}) -- replace zaxis name on plot (if exists)

\item {} 
\textbf{\texttt{tname}} (\href{https://docs.python.org/2/library/functions.html\#str}{\emph{\texttt{str}}}) -- replace taxis name on plot (if exists)

\item {} 
\textbf{\texttt{wname}} (\href{https://docs.python.org/2/library/functions.html\#str}{\emph{\texttt{str}}}) -- replace waxis name on plot (if exists)

\item {} 
\textbf{\texttt{xunits}} (\href{https://docs.python.org/2/library/functions.html\#str}{\emph{\texttt{str}}}) -- replace xaxis units on plot

\item {} 
\textbf{\texttt{yunits}} (\href{https://docs.python.org/2/library/functions.html\#str}{\emph{\texttt{str}}}) -- replace yaxis units on plot (if exists)

\item {} 
\textbf{\texttt{zunits}} (\href{https://docs.python.org/2/library/functions.html\#str}{\emph{\texttt{str}}}) -- replace zaxis units on plot (if exists)

\item {} 
\textbf{\texttt{tunits}} (\href{https://docs.python.org/2/library/functions.html\#str}{\emph{\texttt{str}}}) -- replace taxis units on plot (if exists)

\item {} 
\textbf{\texttt{wunits}} (\href{https://docs.python.org/2/library/functions.html\#str}{\emph{\texttt{str}}}) -- replace waxis units on plot (if exists)

\item {} 
\textbf{\texttt{xweights}} (\href{https://docs.python.org/2/library/array.html\#module-array}{\emph{\texttt{array}}}) -- replace xaxis weights used for computing mean

\item {} 
\textbf{\texttt{yweights}} (\href{https://docs.python.org/2/library/array.html\#module-array}{\emph{\texttt{array}}}) -- replace xaxis weights used for computing mean

\item {} 
\textbf{\texttt{comment1}} (\href{https://docs.python.org/2/library/functions.html\#str}{\emph{\texttt{str}}}) -- replaces comment1 on plot

\item {} 
\textbf{\texttt{comment2}} (\href{https://docs.python.org/2/library/functions.html\#str}{\emph{\texttt{str}}}) -- replaces comment2 on plot

\item {} 
\textbf{\texttt{comment3}} (\href{https://docs.python.org/2/library/functions.html\#str}{\emph{\texttt{str}}}) -- replaces comment3 on plot

\item {} 
\textbf{\texttt{comment4}} (\href{https://docs.python.org/2/library/functions.html\#str}{\emph{\texttt{str}}}) -- replaces comment4 on plot

\item {} 
\textbf{\texttt{long\_name}} (\href{https://docs.python.org/2/library/functions.html\#str}{\emph{\texttt{str}}}) -- replaces long\_name on plot

\item {} 
\textbf{\texttt{grid}} (\emph{\texttt{cdms2.grid.TransientRectGrid}}) -- replaces array grid (if exists)

\item {} 
\textbf{\texttt{bg}} (\emph{\texttt{bool/int}}) -- plots in background mode

\item {} 
\textbf{\texttt{ratio}} (\index{xmtics1 (in module vcs.manageElements)}\index{xmtics2 (in module vcs.manageElements)}\index{ymtics1 (in module vcs.manageElements)}\index{ymtics2 (in module vcs.manageElements)}\index{xticlabels1 (in module vcs.manageElements)}\index{xticlabels2 (in module vcs.manageElements)}\index{yticlabels1 (in module vcs.manageElements)}\index{yticlabels2 (in module vcs.manageElements)}\index{projection (in module vcs.manageElements)}\index{datawc\_x1 (in module vcs.manageElements)}\index{datawc\_x2 (in module vcs.manageElements)}\index{datawc\_y1 (in module vcs.manageElements)}\index{datawc\_y2 (in module vcs.manageElements)}\index{datawc\_timeunits (in module vcs.manageElements)}\index{datawc\_calendar (in module vcs.manageElements)}) -- sets the y/x ratio ,if passed as a string with `t' at the end, will aslo moves the ticks

\item {} 
\textbf{\texttt{xaxisconvert}} (\href{https://docs.python.org/2/library/functions.html\#str}{\emph{\texttt{str}}}) -- (Ex: `linear') converting xaxis linear/log/log10/ln/exp/area\_wt

\item {} 
\textbf{\texttt{yaxisconvert}} (\href{https://docs.python.org/2/library/functions.html\#str}{\emph{\texttt{str}}}) -- (Ex: `linear') converting yaxis linear/log/log10/ln/exp/area\_wt

\item {} 
\textbf{\texttt{GM\_name}} -- (Ex: `default') retrieve the graphics method object of the given name. If no name is given, then retrieve the `default' graphics method.

\end{itemize}

\item[{Returns}] \leavevmode
A pre-existing boxfill graphics method

\item[{Return type}] \leavevmode
{\hyperref[vcs/graphics/boxfill:vcs.boxfill.Gfb]{\sphinxcrossref{vcs.boxfill.Gfb}}}

\end{description}\end{quote}

\end{fulllineitems}

\index{getcolormap() (in module vcs.manageElements)}

\begin{fulllineitems}
\phantomsection\label{vcs/misc/manageElements:vcs.manageElements.getcolormap}\pysiglinewithargsret{\sphinxcode{vcs.manageElements.}\sphinxbfcode{getcolormap}}{\emph{Cp\_name\_src='default'}}{}
VCS contains a list of secondary methods. This function will create a
colormap class object from an existing VCS colormap secondary method. If
no colormap name is given, then colormap `default' will be used.

\begin{notice}{note}{Note:}
VCS does not allow the modification of `default' attribute sets.
However, a `default' attribute set that has been copied under a
different name can be modified. (See the {\hyperref[vcs/misc/manageElements:vcs.manageElements.createcolormap]{\sphinxcrossref{\sphinxcode{vcs.manageElements.createcolormap()}}}} function.)
\end{notice}
\begin{quote}\begin{description}
\item[{Example}] \leavevmode
\begin{Verbatim}[commandchars=\\\{\}]
\PYG{g+gp}{\PYGZgt{}\PYGZgt{}\PYGZgt{} }\PYG{n}{a}\PYG{o}{=}\PYG{n}{vcs}\PYG{o}{.}\PYG{n}{init}\PYG{p}{(}\PYG{p}{)}
\PYG{g+gp}{\PYGZgt{}\PYGZgt{}\PYGZgt{} }\PYG{n}{vcs}\PYG{o}{.}\PYG{n}{listelements}\PYG{p}{(}\PYG{l+s+s1}{\PYGZsq{}}\PYG{l+s+s1}{colormap}\PYG{l+s+s1}{\PYGZsq{}}\PYG{p}{)} \PYG{c+c1}{\PYGZsh{} Show all the existing colormap secondary methods}
\PYG{g+go}{[...]}
\PYG{g+gp}{\PYGZgt{}\PYGZgt{}\PYGZgt{} }\PYG{n}{ex}\PYG{o}{=}\PYG{n}{vcs}\PYG{o}{.}\PYG{n}{getcolormap}\PYG{p}{(}\PYG{p}{)}  \PYG{c+c1}{\PYGZsh{} instance of \PYGZsq{}default\PYGZsq{} colormap secondary method}
\PYG{g+gp}{\PYGZgt{}\PYGZgt{}\PYGZgt{} }\PYG{n}{ex2}\PYG{o}{=}\PYG{n}{vcs}\PYG{o}{.}\PYG{n}{getcolormap}\PYG{p}{(}\PYG{l+s+s1}{\PYGZsq{}}\PYG{l+s+s1}{rainbow}\PYG{l+s+s1}{\PYGZsq{}}\PYG{p}{)}  \PYG{c+c1}{\PYGZsh{} instance of \PYGZsq{}rainbow\PYGZsq{} colormap secondary method}
\end{Verbatim}

\item[{Parameters}] \leavevmode
\textbf{\texttt{Cp\_name\_src}} (\href{https://docs.python.org/2/library/functions.html\#str}{\emph{\texttt{str}}}) -- String name of an existing colormap VCS object

\item[{Returns}] \leavevmode
A pre-existing VCS colormap object

\item[{Return type}] \leavevmode
{\hyperref[vcs/misc/colormap:vcs.colormap.Cp]{\sphinxcrossref{vcs.colormap.Cp}}}

\end{description}\end{quote}

\end{fulllineitems}

\index{getfillarea() (in module vcs.manageElements)}

\begin{fulllineitems}
\phantomsection\label{vcs/misc/manageElements:vcs.manageElements.getfillarea}\pysiglinewithargsret{\sphinxcode{vcs.manageElements.}\sphinxbfcode{getfillarea}}{\emph{name='default'}, \emph{style=None}, \emph{index=None}, \emph{color=None}, \emph{priority=None}, \emph{viewport=None}, \emph{worldcoordinate=None}, \emph{x=None}, \emph{y=None}}{}
VCS contains a list of secondary methods. This function will create a
fillarea class object from an existing VCS fillarea secondary method. If
no fillarea name is given, then fillarea `default' will be used.

\begin{notice}{note}{Note:}
VCS does not allow the modification of `default' attribute sets.
However, a `default' attribute set that has been copied under a
different name can be modified. (See the {\hyperref[vcs/misc/manageElements:vcs.manageElements.createfillarea]{\sphinxcrossref{\sphinxcode{vcs.manageElements.createfillarea()}}}} function.)
\end{notice}
\begin{quote}\begin{description}
\item[{Example}] \leavevmode
\begin{Verbatim}[commandchars=\\\{\}]
\PYG{g+gp}{\PYGZgt{}\PYGZgt{}\PYGZgt{} }\PYG{n}{a}\PYG{o}{=}\PYG{n}{vcs}\PYG{o}{.}\PYG{n}{init}\PYG{p}{(}\PYG{p}{)}
\PYG{g+gp}{\PYGZgt{}\PYGZgt{}\PYGZgt{} }\PYG{n}{vcs}\PYG{o}{.}\PYG{n}{listelements}\PYG{p}{(}\PYG{l+s+s1}{\PYGZsq{}}\PYG{l+s+s1}{fillarea}\PYG{l+s+s1}{\PYGZsq{}}\PYG{p}{)} \PYG{c+c1}{\PYGZsh{} Show all the existing fillarea secondary methods}
\PYG{g+go}{[...]}
\PYG{g+gp}{\PYGZgt{}\PYGZgt{}\PYGZgt{} }\PYG{n}{ex}\PYG{o}{=}\PYG{n}{vcs}\PYG{o}{.}\PYG{n}{getfillarea}\PYG{p}{(}\PYG{p}{)}  \PYG{c+c1}{\PYGZsh{} instance of \PYGZsq{}default\PYGZsq{} fillarea secondary method}
\PYG{g+gp}{\PYGZgt{}\PYGZgt{}\PYGZgt{} }\PYG{n}{a}\PYG{o}{.}\PYG{n}{fillarea}\PYG{p}{(}\PYG{n}{ex}\PYG{p}{)} \PYG{c+c1}{\PYGZsh{} plot using specified fillarea object}
\PYG{g+go}{\PYGZlt{}vcs.displayplot.Dp ...\PYGZgt{}}
\end{Verbatim}

\item[{Parameters}] \leavevmode\begin{itemize}
\item {} 
\textbf{\texttt{name}} (\href{https://docs.python.org/2/library/functions.html\#str}{\emph{\texttt{str}}}) -- String name of an existing fillarea VCS object

\item {} 
\textbf{\texttt{style}} (\href{https://docs.python.org/2/library/functions.html\#str}{\emph{\texttt{str}}}) -- One of ``hatch'', ``solid'', or ``pattern''.

\item {} 
\textbf{\texttt{index}} (\href{https://docs.python.org/2/library/functions.html\#int}{\emph{\texttt{int}}}) -- 
Specifies which \href{http://uvcdat.llnl.gov/gallery/fullsize/pattern\_chart.png}{pattern} to fill with.
Accepts ints from 1-20.


\item {} 
\textbf{\texttt{color}} (\emph{\texttt{str or int}}) -- 
A color name from the \href{https://en.wikipedia.org/wiki/X11\_color\_names}{X11 Color Names list},
or an integer value from 0-255, or an RGB/RGBA tuple/list (e.g. (0,100,0), (100,100,0,50))


\item {} 
\textbf{\texttt{priority}} (\href{https://docs.python.org/2/library/functions.html\#int}{\emph{\texttt{int}}}) -- The layer on which the texttable will be drawn.

\item {} 
\textbf{\texttt{viewport}} (\emph{\texttt{list of floats}}) -- 4 floats between 0 and 1. These specify the area that the X/Y values are mapped to inside of the canvas

\item {} 
\textbf{\texttt{worldcoordinate}} (\emph{\texttt{list of floats}}) -- List of 4 floats (xmin, xmax, ymin, ymax)

\item {} 
\textbf{\texttt{x}} (\emph{\texttt{list of floats}}) -- List of lists of x coordinates. Values must be between worldcoordinate{[}0{]} and worldcoordinate{[}1{]}.

\item {} 
\textbf{\texttt{y}} (\emph{\texttt{list of floats}}) -- List of lists of y coordinates. Values must be between worldcoordinate{[}2{]} and worldcoordinate{[}3{]}.

\end{itemize}

\item[{Returns}] \leavevmode
A fillarea secondary object

\item[{Return type}] \leavevmode
{\hyperref[vcs/secondary/fillarea:vcs.fillarea.Tf]{\sphinxcrossref{vcs.fillarea.Tf}}}

\end{description}\end{quote}

\end{fulllineitems}

\index{getisofill() (in module vcs.manageElements)}

\begin{fulllineitems}
\phantomsection\label{vcs/misc/manageElements:vcs.manageElements.getisofill}\pysiglinewithargsret{\sphinxcode{vcs.manageElements.}\sphinxbfcode{getisofill}}{\emph{Gfi\_name\_src='default'}}{}
VCS contains a list of graphics methods. This function will create a
isofill class object from an existing VCS isofill graphics method. If
no isofill name is given, then isofill `default' will be used.

\begin{notice}{note}{Note:}
VCS does not allow the modification of `default' attribute sets.
However, a `default' attribute set that has been copied under a
different name can be modified. (See the {\hyperref[vcs/misc/manageElements:vcs.manageElements.createisofill]{\sphinxcrossref{\sphinxcode{vcs.manageElements.createisofill()}}}} function.)
\end{notice}
\begin{quote}\begin{description}
\item[{Example}] \leavevmode
\begin{Verbatim}[commandchars=\\\{\}]
\PYG{g+gp}{\PYGZgt{}\PYGZgt{}\PYGZgt{} }\PYG{n}{a}\PYG{o}{=}\PYG{n}{vcs}\PYG{o}{.}\PYG{n}{init}\PYG{p}{(}\PYG{p}{)}
\PYG{g+gp}{\PYGZgt{}\PYGZgt{}\PYGZgt{} }\PYG{n}{vcs}\PYG{o}{.}\PYG{n}{listelements}\PYG{p}{(}\PYG{l+s+s1}{\PYGZsq{}}\PYG{l+s+s1}{isofill}\PYG{l+s+s1}{\PYGZsq{}}\PYG{p}{)} \PYG{c+c1}{\PYGZsh{} Show all the existing isofill graphics methods}
\PYG{g+go}{[...]}
\PYG{g+gp}{\PYGZgt{}\PYGZgt{}\PYGZgt{} }\PYG{n}{ex}\PYG{o}{=}\PYG{n}{vcs}\PYG{o}{.}\PYG{n}{getisofill}\PYG{p}{(}\PYG{p}{)}  \PYG{c+c1}{\PYGZsh{} instance of \PYGZsq{}default\PYGZsq{} isofill graphics method}
\PYG{g+gp}{\PYGZgt{}\PYGZgt{}\PYGZgt{} }\PYG{k+kn}{import} \PYG{n+nn}{cdms2} \PYG{c+c1}{\PYGZsh{} Need cdms2 to create a slab}
\PYG{g+gp}{\PYGZgt{}\PYGZgt{}\PYGZgt{} }\PYG{n}{f} \PYG{o}{=} \PYG{n}{cdms2}\PYG{o}{.}\PYG{n}{open}\PYG{p}{(}\PYG{n}{vcs}\PYG{o}{.}\PYG{n}{sample\PYGZus{}data}\PYG{o}{+}\PYG{l+s+s1}{\PYGZsq{}}\PYG{l+s+s1}{/clt.nc}\PYG{l+s+s1}{\PYGZsq{}}\PYG{p}{)} \PYG{c+c1}{\PYGZsh{} use cdms2 to open a data file}
\PYG{g+gp}{\PYGZgt{}\PYGZgt{}\PYGZgt{} }\PYG{n}{slab1} \PYG{o}{=} \PYG{n}{f}\PYG{p}{(}\PYG{l+s+s1}{\PYGZsq{}}\PYG{l+s+s1}{u}\PYG{l+s+s1}{\PYGZsq{}}\PYG{p}{)} \PYG{c+c1}{\PYGZsh{} use the data file to create a cdms2 slab}
\PYG{g+gp}{\PYGZgt{}\PYGZgt{}\PYGZgt{} }\PYG{n}{a}\PYG{o}{.}\PYG{n}{isofill}\PYG{p}{(}\PYG{n}{ex}\PYG{p}{,} \PYG{n}{slab1}\PYG{p}{)} \PYG{c+c1}{\PYGZsh{} plot using specified isofill object}
\PYG{g+go}{\PYGZlt{}vcs.displayplot.Dp ...\PYGZgt{}}
\PYG{g+gp}{\PYGZgt{}\PYGZgt{}\PYGZgt{} }\PYG{n}{ex2}\PYG{o}{=}\PYG{n}{vcs}\PYG{o}{.}\PYG{n}{getisofill}\PYG{p}{(}\PYG{l+s+s1}{\PYGZsq{}}\PYG{l+s+s1}{polar}\PYG{l+s+s1}{\PYGZsq{}}\PYG{p}{)}  \PYG{c+c1}{\PYGZsh{} instance of \PYGZsq{}polar\PYGZsq{} isofill graphics method}
\PYG{g+gp}{\PYGZgt{}\PYGZgt{}\PYGZgt{} }\PYG{n}{a}\PYG{o}{.}\PYG{n}{isofill}\PYG{p}{(}\PYG{n}{ex2}\PYG{p}{,} \PYG{n}{slab1}\PYG{p}{)} \PYG{c+c1}{\PYGZsh{} plot using specified isofill object}
\PYG{g+go}{\PYGZlt{}vcs.displayplot.Dp ...\PYGZgt{}}
\end{Verbatim}

\item[{Parameters}] \leavevmode\begin{itemize}
\item {} 
\textbf{\texttt{Gfi\_name\_src}} (\href{https://docs.python.org/2/library/functions.html\#str}{\emph{\texttt{str}}}) -- String name of an existing isofill VCS object

\item {} 
\textbf{\texttt{xaxis}} (\emph{\texttt{cdms2.axis.TransientAxis}}) -- Axis object to replace the slab -1 dim axis

\item {} 
\textbf{\texttt{yaxis}} (\emph{\texttt{cdms2.axis.TransientAxis}}) -- Axis object to replace the slab -2 dim axis, only if slab has more than 1D

\item {} 
\textbf{\texttt{zaxis}} (\emph{\texttt{cdms2.axis.TransientAxis}}) -- Axis object to replace the slab -3 dim axis, only if slab has more than 2D

\item {} 
\textbf{\texttt{taxis}} (\emph{\texttt{cdms2.axis.TransientAxis}}) -- Axis object to replace the slab -4 dim axis, only if slab has more than 3D

\item {} 
\textbf{\texttt{waxis}} (\emph{\texttt{cdms2.axis.TransientAxis}}) -- Axis object to replace the slab -5 dim axis, only if slab has more than 4D

\item {} 
\textbf{\texttt{xrev}} (\href{https://docs.python.org/2/library/functions.html\#bool}{\emph{\texttt{bool}}}) -- reverse x axis

\item {} 
\textbf{\texttt{yrev}} (\href{https://docs.python.org/2/library/functions.html\#bool}{\emph{\texttt{bool}}}) -- reverse y axis, only if slab has more than 1D

\item {} 
\textbf{\texttt{xarray}} (\href{https://docs.python.org/2/library/array.html\#module-array}{\emph{\texttt{array}}}) -- Values to use instead of x axis

\item {} 
\textbf{\texttt{yarray}} (\href{https://docs.python.org/2/library/array.html\#module-array}{\emph{\texttt{array}}}) -- Values to use instead of y axis, only if var has more than 1D

\item {} 
\textbf{\texttt{zarray}} (\href{https://docs.python.org/2/library/array.html\#module-array}{\emph{\texttt{array}}}) -- Values to use instead of z axis, only if var has more than 2D

\item {} 
\textbf{\texttt{tarray}} (\href{https://docs.python.org/2/library/array.html\#module-array}{\emph{\texttt{array}}}) -- Values to use instead of t axis, only if var has more than 3D

\item {} 
\textbf{\texttt{warray}} (\href{https://docs.python.org/2/library/array.html\#module-array}{\emph{\texttt{array}}}) -- Values to use instead of w axis, only if var has more than 4D

\item {} 
\textbf{\texttt{continents}} (\href{https://docs.python.org/2/library/functions.html\#int}{\emph{\texttt{int}}}) -- continents type number

\item {} 
\textbf{\texttt{name}} (\href{https://docs.python.org/2/library/functions.html\#str}{\emph{\texttt{str}}}) -- replaces variable name on plot

\item {} 
\textbf{\texttt{time}} (\emph{\texttt{A cdtime object}}) -- replaces time name on plot

\item {} 
\textbf{\texttt{units}} (\href{https://docs.python.org/2/library/functions.html\#str}{\emph{\texttt{str}}}) -- replaces units value on plot

\item {} 
\textbf{\texttt{ymd}} (\href{https://docs.python.org/2/library/functions.html\#str}{\emph{\texttt{str}}}) -- replaces year/month/day on plot

\item {} 
\textbf{\texttt{hms}} (\href{https://docs.python.org/2/library/functions.html\#str}{\emph{\texttt{str}}}) -- replaces hh/mm/ss on plot

\item {} 
\textbf{\texttt{file\_comment}} (\href{https://docs.python.org/2/library/functions.html\#str}{\emph{\texttt{str}}}) -- replaces file\_comment on plot

\item {} 
\textbf{\texttt{xbounds}} (\href{https://docs.python.org/2/library/array.html\#module-array}{\emph{\texttt{array}}}) -- Values to use instead of x axis bounds values

\item {} 
\textbf{\texttt{ybounds}} (\href{https://docs.python.org/2/library/array.html\#module-array}{\emph{\texttt{array}}}) -- Values to use instead of y axis bounds values (if exist)

\item {} 
\textbf{\texttt{xname}} (\href{https://docs.python.org/2/library/functions.html\#str}{\emph{\texttt{str}}}) -- replace xaxis name on plot

\item {} 
\textbf{\texttt{yname}} (\href{https://docs.python.org/2/library/functions.html\#str}{\emph{\texttt{str}}}) -- replace yaxis name on plot (if exists)

\item {} 
\textbf{\texttt{zname}} (\href{https://docs.python.org/2/library/functions.html\#str}{\emph{\texttt{str}}}) -- replace zaxis name on plot (if exists)

\item {} 
\textbf{\texttt{tname}} (\href{https://docs.python.org/2/library/functions.html\#str}{\emph{\texttt{str}}}) -- replace taxis name on plot (if exists)

\item {} 
\textbf{\texttt{wname}} (\href{https://docs.python.org/2/library/functions.html\#str}{\emph{\texttt{str}}}) -- replace waxis name on plot (if exists)

\item {} 
\textbf{\texttt{xunits}} (\href{https://docs.python.org/2/library/functions.html\#str}{\emph{\texttt{str}}}) -- replace xaxis units on plot

\item {} 
\textbf{\texttt{yunits}} (\href{https://docs.python.org/2/library/functions.html\#str}{\emph{\texttt{str}}}) -- replace yaxis units on plot (if exists)

\item {} 
\textbf{\texttt{zunits}} (\href{https://docs.python.org/2/library/functions.html\#str}{\emph{\texttt{str}}}) -- replace zaxis units on plot (if exists)

\item {} 
\textbf{\texttt{tunits}} (\href{https://docs.python.org/2/library/functions.html\#str}{\emph{\texttt{str}}}) -- replace taxis units on plot (if exists)

\item {} 
\textbf{\texttt{wunits}} (\href{https://docs.python.org/2/library/functions.html\#str}{\emph{\texttt{str}}}) -- replace waxis units on plot (if exists)

\item {} 
\textbf{\texttt{xweights}} (\href{https://docs.python.org/2/library/array.html\#module-array}{\emph{\texttt{array}}}) -- replace xaxis weights used for computing mean

\item {} 
\textbf{\texttt{yweights}} (\href{https://docs.python.org/2/library/array.html\#module-array}{\emph{\texttt{array}}}) -- replace xaxis weights used for computing mean

\item {} 
\textbf{\texttt{comment1}} (\href{https://docs.python.org/2/library/functions.html\#str}{\emph{\texttt{str}}}) -- replaces comment1 on plot

\item {} 
\textbf{\texttt{comment2}} (\href{https://docs.python.org/2/library/functions.html\#str}{\emph{\texttt{str}}}) -- replaces comment2 on plot

\item {} 
\textbf{\texttt{comment3}} (\href{https://docs.python.org/2/library/functions.html\#str}{\emph{\texttt{str}}}) -- replaces comment3 on plot

\item {} 
\textbf{\texttt{comment4}} (\href{https://docs.python.org/2/library/functions.html\#str}{\emph{\texttt{str}}}) -- replaces comment4 on plot

\item {} 
\textbf{\texttt{long\_name}} (\href{https://docs.python.org/2/library/functions.html\#str}{\emph{\texttt{str}}}) -- replaces long\_name on plot

\item {} 
\textbf{\texttt{grid}} (\emph{\texttt{cdms2.grid.TransientRectGrid}}) -- replaces array grid (if exists)

\item {} 
\textbf{\texttt{bg}} (\emph{\texttt{bool/int}}) -- plots in background mode

\item {} 
\textbf{\texttt{ratio}} (\index{xmtics1 (in module vcs.manageElements)}\index{xmtics2 (in module vcs.manageElements)}\index{ymtics1 (in module vcs.manageElements)}\index{ymtics2 (in module vcs.manageElements)}\index{xticlabels1 (in module vcs.manageElements)}\index{xticlabels2 (in module vcs.manageElements)}\index{yticlabels1 (in module vcs.manageElements)}\index{yticlabels2 (in module vcs.manageElements)}\index{projection (in module vcs.manageElements)}\index{datawc\_x1 (in module vcs.manageElements)}\index{datawc\_x2 (in module vcs.manageElements)}\index{datawc\_y1 (in module vcs.manageElements)}\index{datawc\_y2 (in module vcs.manageElements)}\index{datawc\_timeunits (in module vcs.manageElements)}\index{datawc\_calendar (in module vcs.manageElements)}) -- sets the y/x ratio ,if passed as a string with `t' at the end, will aslo moves the ticks

\item {} 
\textbf{\texttt{xaxisconvert}} (\href{https://docs.python.org/2/library/functions.html\#str}{\emph{\texttt{str}}}) -- (Ex: `linear') converting xaxis linear/log/log10/ln/exp/area\_wt

\item {} 
\textbf{\texttt{yaxisconvert}} (\href{https://docs.python.org/2/library/functions.html\#str}{\emph{\texttt{str}}}) -- (Ex: `linear') converting yaxis linear/log/log10/ln/exp/area\_wt

\item {} 
\textbf{\texttt{GM\_name}} -- (Ex: `default') retrieve the graphics method object of the given name. If no name is given, then retrieve the `default' graphics method.

\end{itemize}

\item[{Returns}] \leavevmode
The specified isofill VCS object

\item[{Return type}] \leavevmode
{\hyperref[vcs/graphics/isofill:vcs.isofill.Gfi]{\sphinxcrossref{vcs.isofill.Gfi}}}

\end{description}\end{quote}

\end{fulllineitems}

\index{getisoline() (in module vcs.manageElements)}

\begin{fulllineitems}
\phantomsection\label{vcs/misc/manageElements:vcs.manageElements.getisoline}\pysiglinewithargsret{\sphinxcode{vcs.manageElements.}\sphinxbfcode{getisoline}}{\emph{Gi\_name\_src='default'}}{}
VCS contains a list of graphics methods. This function will create a
isoline class object from an existing VCS isoline graphics method. If
no isoline name is given, then isoline `default' will be used.

\begin{notice}{note}{Note:}
VCS does not allow the modification of `default' attribute sets.
However, a `default' attribute set that has been copied under a
different name can be modified. (See the {\hyperref[vcs/misc/manageElements:vcs.manageElements.createisoline]{\sphinxcrossref{\sphinxcode{vcs.manageElements.createisoline()}}}} function.)
\end{notice}
\begin{quote}\begin{description}
\item[{Example}] \leavevmode
\begin{Verbatim}[commandchars=\\\{\}]
\PYG{g+gp}{\PYGZgt{}\PYGZgt{}\PYGZgt{} }\PYG{n}{a}\PYG{o}{=}\PYG{n}{vcs}\PYG{o}{.}\PYG{n}{init}\PYG{p}{(}\PYG{p}{)}
\PYG{g+gp}{\PYGZgt{}\PYGZgt{}\PYGZgt{} }\PYG{n}{vcs}\PYG{o}{.}\PYG{n}{listelements}\PYG{p}{(}\PYG{l+s+s1}{\PYGZsq{}}\PYG{l+s+s1}{isoline}\PYG{l+s+s1}{\PYGZsq{}}\PYG{p}{)} \PYG{c+c1}{\PYGZsh{} Show all the existing isoline graphics methods}
\PYG{g+go}{[...]}
\PYG{g+gp}{\PYGZgt{}\PYGZgt{}\PYGZgt{} }\PYG{n}{ex}\PYG{o}{=}\PYG{n}{vcs}\PYG{o}{.}\PYG{n}{getisoline}\PYG{p}{(}\PYG{p}{)}  \PYG{c+c1}{\PYGZsh{} instance of \PYGZsq{}default\PYGZsq{} isoline graphics method}
\PYG{g+gp}{\PYGZgt{}\PYGZgt{}\PYGZgt{} }\PYG{k+kn}{import} \PYG{n+nn}{cdms2} \PYG{c+c1}{\PYGZsh{} Need cdms2 to create a slab}
\PYG{g+gp}{\PYGZgt{}\PYGZgt{}\PYGZgt{} }\PYG{n}{f} \PYG{o}{=} \PYG{n}{cdms2}\PYG{o}{.}\PYG{n}{open}\PYG{p}{(}\PYG{n}{vcs}\PYG{o}{.}\PYG{n}{sample\PYGZus{}data}\PYG{o}{+}\PYG{l+s+s1}{\PYGZsq{}}\PYG{l+s+s1}{/clt.nc}\PYG{l+s+s1}{\PYGZsq{}}\PYG{p}{)} \PYG{c+c1}{\PYGZsh{} use cdms2 to open a data file}
\PYG{g+gp}{\PYGZgt{}\PYGZgt{}\PYGZgt{} }\PYG{n}{slab1} \PYG{o}{=} \PYG{n}{f}\PYG{p}{(}\PYG{l+s+s1}{\PYGZsq{}}\PYG{l+s+s1}{u}\PYG{l+s+s1}{\PYGZsq{}}\PYG{p}{)} \PYG{c+c1}{\PYGZsh{} use the data file to create a cdms2 slab}
\PYG{g+gp}{\PYGZgt{}\PYGZgt{}\PYGZgt{} }\PYG{n}{a}\PYG{o}{.}\PYG{n}{isoline}\PYG{p}{(}\PYG{n}{ex}\PYG{p}{,} \PYG{n}{slab1}\PYG{p}{)} \PYG{c+c1}{\PYGZsh{} plot using specified isoline object}
\PYG{g+go}{\PYGZlt{}vcs.displayplot.Dp ...\PYGZgt{}}
\PYG{g+gp}{\PYGZgt{}\PYGZgt{}\PYGZgt{} }\PYG{n}{ex2}\PYG{o}{=}\PYG{n}{vcs}\PYG{o}{.}\PYG{n}{getisoline}\PYG{p}{(}\PYG{l+s+s1}{\PYGZsq{}}\PYG{l+s+s1}{polar}\PYG{l+s+s1}{\PYGZsq{}}\PYG{p}{)}  \PYG{c+c1}{\PYGZsh{} instance of \PYGZsq{}polar\PYGZsq{} isoline graphics method}
\PYG{g+gp}{\PYGZgt{}\PYGZgt{}\PYGZgt{} }\PYG{n}{a}\PYG{o}{.}\PYG{n}{isoline}\PYG{p}{(}\PYG{n}{ex2}\PYG{p}{,} \PYG{n}{slab1}\PYG{p}{)} \PYG{c+c1}{\PYGZsh{} plot using specified isoline object}
\PYG{g+go}{\PYGZlt{}vcs.displayplot.Dp ...\PYGZgt{}}
\end{Verbatim}

\item[{Parameters}] \leavevmode\begin{itemize}
\item {} 
\textbf{\texttt{Gi\_name\_src}} (\href{https://docs.python.org/2/library/functions.html\#str}{\emph{\texttt{str}}}) -- String name of an existing isoline VCS object

\item {} 
\textbf{\texttt{xaxis}} (\emph{\texttt{cdms2.axis.TransientAxis}}) -- Axis object to replace the slab -1 dim axis

\item {} 
\textbf{\texttt{yaxis}} (\emph{\texttt{cdms2.axis.TransientAxis}}) -- Axis object to replace the slab -2 dim axis, only if slab has more than 1D

\item {} 
\textbf{\texttt{zaxis}} (\emph{\texttt{cdms2.axis.TransientAxis}}) -- Axis object to replace the slab -3 dim axis, only if slab has more than 2D

\item {} 
\textbf{\texttt{taxis}} (\emph{\texttt{cdms2.axis.TransientAxis}}) -- Axis object to replace the slab -4 dim axis, only if slab has more than 3D

\item {} 
\textbf{\texttt{waxis}} (\emph{\texttt{cdms2.axis.TransientAxis}}) -- Axis object to replace the slab -5 dim axis, only if slab has more than 4D

\item {} 
\textbf{\texttt{xrev}} (\href{https://docs.python.org/2/library/functions.html\#bool}{\emph{\texttt{bool}}}) -- reverse x axis

\item {} 
\textbf{\texttt{yrev}} (\href{https://docs.python.org/2/library/functions.html\#bool}{\emph{\texttt{bool}}}) -- reverse y axis, only if slab has more than 1D

\item {} 
\textbf{\texttt{xarray}} (\href{https://docs.python.org/2/library/array.html\#module-array}{\emph{\texttt{array}}}) -- Values to use instead of x axis

\item {} 
\textbf{\texttt{yarray}} (\href{https://docs.python.org/2/library/array.html\#module-array}{\emph{\texttt{array}}}) -- Values to use instead of y axis, only if var has more than 1D

\item {} 
\textbf{\texttt{zarray}} (\href{https://docs.python.org/2/library/array.html\#module-array}{\emph{\texttt{array}}}) -- Values to use instead of z axis, only if var has more than 2D

\item {} 
\textbf{\texttt{tarray}} (\href{https://docs.python.org/2/library/array.html\#module-array}{\emph{\texttt{array}}}) -- Values to use instead of t axis, only if var has more than 3D

\item {} 
\textbf{\texttt{warray}} (\href{https://docs.python.org/2/library/array.html\#module-array}{\emph{\texttt{array}}}) -- Values to use instead of w axis, only if var has more than 4D

\item {} 
\textbf{\texttt{continents}} (\href{https://docs.python.org/2/library/functions.html\#int}{\emph{\texttt{int}}}) -- continents type number

\item {} 
\textbf{\texttt{name}} (\href{https://docs.python.org/2/library/functions.html\#str}{\emph{\texttt{str}}}) -- replaces variable name on plot

\item {} 
\textbf{\texttt{time}} (\emph{\texttt{A cdtime object}}) -- replaces time name on plot

\item {} 
\textbf{\texttt{units}} (\href{https://docs.python.org/2/library/functions.html\#str}{\emph{\texttt{str}}}) -- replaces units value on plot

\item {} 
\textbf{\texttt{ymd}} (\href{https://docs.python.org/2/library/functions.html\#str}{\emph{\texttt{str}}}) -- replaces year/month/day on plot

\item {} 
\textbf{\texttt{hms}} (\href{https://docs.python.org/2/library/functions.html\#str}{\emph{\texttt{str}}}) -- replaces hh/mm/ss on plot

\item {} 
\textbf{\texttt{file\_comment}} (\href{https://docs.python.org/2/library/functions.html\#str}{\emph{\texttt{str}}}) -- replaces file\_comment on plot

\item {} 
\textbf{\texttt{xbounds}} (\href{https://docs.python.org/2/library/array.html\#module-array}{\emph{\texttt{array}}}) -- Values to use instead of x axis bounds values

\item {} 
\textbf{\texttt{ybounds}} (\href{https://docs.python.org/2/library/array.html\#module-array}{\emph{\texttt{array}}}) -- Values to use instead of y axis bounds values (if exist)

\item {} 
\textbf{\texttt{xname}} (\href{https://docs.python.org/2/library/functions.html\#str}{\emph{\texttt{str}}}) -- replace xaxis name on plot

\item {} 
\textbf{\texttt{yname}} (\href{https://docs.python.org/2/library/functions.html\#str}{\emph{\texttt{str}}}) -- replace yaxis name on plot (if exists)

\item {} 
\textbf{\texttt{zname}} (\href{https://docs.python.org/2/library/functions.html\#str}{\emph{\texttt{str}}}) -- replace zaxis name on plot (if exists)

\item {} 
\textbf{\texttt{tname}} (\href{https://docs.python.org/2/library/functions.html\#str}{\emph{\texttt{str}}}) -- replace taxis name on plot (if exists)

\item {} 
\textbf{\texttt{wname}} (\href{https://docs.python.org/2/library/functions.html\#str}{\emph{\texttt{str}}}) -- replace waxis name on plot (if exists)

\item {} 
\textbf{\texttt{xunits}} (\href{https://docs.python.org/2/library/functions.html\#str}{\emph{\texttt{str}}}) -- replace xaxis units on plot

\item {} 
\textbf{\texttt{yunits}} (\href{https://docs.python.org/2/library/functions.html\#str}{\emph{\texttt{str}}}) -- replace yaxis units on plot (if exists)

\item {} 
\textbf{\texttt{zunits}} (\href{https://docs.python.org/2/library/functions.html\#str}{\emph{\texttt{str}}}) -- replace zaxis units on plot (if exists)

\item {} 
\textbf{\texttt{tunits}} (\href{https://docs.python.org/2/library/functions.html\#str}{\emph{\texttt{str}}}) -- replace taxis units on plot (if exists)

\item {} 
\textbf{\texttt{wunits}} (\href{https://docs.python.org/2/library/functions.html\#str}{\emph{\texttt{str}}}) -- replace waxis units on plot (if exists)

\item {} 
\textbf{\texttt{xweights}} (\href{https://docs.python.org/2/library/array.html\#module-array}{\emph{\texttt{array}}}) -- replace xaxis weights used for computing mean

\item {} 
\textbf{\texttt{yweights}} (\href{https://docs.python.org/2/library/array.html\#module-array}{\emph{\texttt{array}}}) -- replace xaxis weights used for computing mean

\item {} 
\textbf{\texttt{comment1}} (\href{https://docs.python.org/2/library/functions.html\#str}{\emph{\texttt{str}}}) -- replaces comment1 on plot

\item {} 
\textbf{\texttt{comment2}} (\href{https://docs.python.org/2/library/functions.html\#str}{\emph{\texttt{str}}}) -- replaces comment2 on plot

\item {} 
\textbf{\texttt{comment3}} (\href{https://docs.python.org/2/library/functions.html\#str}{\emph{\texttt{str}}}) -- replaces comment3 on plot

\item {} 
\textbf{\texttt{comment4}} (\href{https://docs.python.org/2/library/functions.html\#str}{\emph{\texttt{str}}}) -- replaces comment4 on plot

\item {} 
\textbf{\texttt{long\_name}} (\href{https://docs.python.org/2/library/functions.html\#str}{\emph{\texttt{str}}}) -- replaces long\_name on plot

\item {} 
\textbf{\texttt{grid}} (\emph{\texttt{cdms2.grid.TransientRectGrid}}) -- replaces array grid (if exists)

\item {} 
\textbf{\texttt{bg}} (\emph{\texttt{bool/int}}) -- plots in background mode

\item {} 
\textbf{\texttt{ratio}} (\index{xmtics1 (in module vcs.manageElements)}\index{xmtics2 (in module vcs.manageElements)}\index{ymtics1 (in module vcs.manageElements)}\index{ymtics2 (in module vcs.manageElements)}\index{xticlabels1 (in module vcs.manageElements)}\index{xticlabels2 (in module vcs.manageElements)}\index{yticlabels1 (in module vcs.manageElements)}\index{yticlabels2 (in module vcs.manageElements)}\index{projection (in module vcs.manageElements)}\index{datawc\_x1 (in module vcs.manageElements)}\index{datawc\_x2 (in module vcs.manageElements)}\index{datawc\_y1 (in module vcs.manageElements)}\index{datawc\_y2 (in module vcs.manageElements)}\index{datawc\_timeunits (in module vcs.manageElements)}\index{datawc\_calendar (in module vcs.manageElements)}) -- sets the y/x ratio ,if passed as a string with `t' at the end, will aslo moves the ticks

\item {} 
\textbf{\texttt{xaxisconvert}} (\href{https://docs.python.org/2/library/functions.html\#str}{\emph{\texttt{str}}}) -- (Ex: `linear') converting xaxis linear/log/log10/ln/exp/area\_wt

\item {} 
\textbf{\texttt{yaxisconvert}} (\href{https://docs.python.org/2/library/functions.html\#str}{\emph{\texttt{str}}}) -- (Ex: `linear') converting yaxis linear/log/log10/ln/exp/area\_wt

\item {} 
\textbf{\texttt{GM\_name}} -- (Ex: `default') retrieve the graphics method object of the given name. If no name is given, then retrieve the `default' graphics method.

\end{itemize}

\item[{Returns}] \leavevmode
The requested isoline VCS object

\item[{Return type}] \leavevmode
{\hyperref[vcs/graphics/isoline:vcs.isoline.Gi]{\sphinxcrossref{vcs.isoline.Gi}}}

\end{description}\end{quote}

\end{fulllineitems}

\index{getline() (in module vcs.manageElements)}

\begin{fulllineitems}
\phantomsection\label{vcs/misc/manageElements:vcs.manageElements.getline}\pysiglinewithargsret{\sphinxcode{vcs.manageElements.}\sphinxbfcode{getline}}{\emph{name='default'}, \emph{ltype=None}, \emph{width=None}, \emph{color=None}, \emph{priority=None}, \emph{viewport=None}, \emph{worldcoordinate=None}, \emph{x=None}, \emph{y=None}}{}
VCS contains a list of secondary methods. This function will create a
line class object from an existing VCS line secondary method. If
no line name is given, then line `default' will be used.

\begin{notice}{note}{Note:}
VCS does not allow the modification of `default' attribute sets.
However, a `default' attribute set that has been copied under a
different name can be modified. (See the {\hyperref[vcs/misc/manageElements:vcs.manageElements.createline]{\sphinxcrossref{\sphinxcode{vcs.manageElements.createline()}}}} function.)
\end{notice}
\begin{quote}\begin{description}
\item[{Example}] \leavevmode
\begin{Verbatim}[commandchars=\\\{\}]
\PYG{g+gp}{\PYGZgt{}\PYGZgt{}\PYGZgt{} }\PYG{n}{a}\PYG{o}{=}\PYG{n}{vcs}\PYG{o}{.}\PYG{n}{init}\PYG{p}{(}\PYG{p}{)}
\PYG{g+gp}{\PYGZgt{}\PYGZgt{}\PYGZgt{} }\PYG{n}{vcs}\PYG{o}{.}\PYG{n}{listelements}\PYG{p}{(}\PYG{l+s+s1}{\PYGZsq{}}\PYG{l+s+s1}{line}\PYG{l+s+s1}{\PYGZsq{}}\PYG{p}{)} \PYG{c+c1}{\PYGZsh{} Show all the existing line secondary methods}
\PYG{g+go}{[...]}
\PYG{g+gp}{\PYGZgt{}\PYGZgt{}\PYGZgt{} }\PYG{n}{ex}\PYG{o}{=}\PYG{n}{vcs}\PYG{o}{.}\PYG{n}{getline}\PYG{p}{(}\PYG{p}{)}  \PYG{c+c1}{\PYGZsh{} instance of \PYGZsq{}default\PYGZsq{} line secondary method}
\PYG{g+gp}{\PYGZgt{}\PYGZgt{}\PYGZgt{} }\PYG{n}{a}\PYG{o}{.}\PYG{n}{line}\PYG{p}{(}\PYG{n}{ex}\PYG{p}{)} \PYG{c+c1}{\PYGZsh{} plot using specified line object}
\PYG{g+go}{\PYGZlt{}vcs.displayplot.Dp ...\PYGZgt{}}
\PYG{g+gp}{\PYGZgt{}\PYGZgt{}\PYGZgt{} }\PYG{n}{ex2}\PYG{o}{=}\PYG{n}{vcs}\PYG{o}{.}\PYG{n}{getline}\PYG{p}{(}\PYG{l+s+s1}{\PYGZsq{}}\PYG{l+s+s1}{red}\PYG{l+s+s1}{\PYGZsq{}}\PYG{p}{)}  \PYG{c+c1}{\PYGZsh{} instance of \PYGZsq{}red\PYGZsq{} line secondary method}
\PYG{g+gp}{\PYGZgt{}\PYGZgt{}\PYGZgt{} }\PYG{n}{a}\PYG{o}{.}\PYG{n}{line}\PYG{p}{(}\PYG{n}{ex2}\PYG{p}{)} \PYG{c+c1}{\PYGZsh{} plot using specified line object}
\PYG{g+go}{\PYGZlt{}vcs.displayplot.Dp ...\PYGZgt{}}
\end{Verbatim}

\item[{Parameters}] \leavevmode\begin{itemize}
\item {} 
\textbf{\texttt{name}} (\href{https://docs.python.org/2/library/functions.html\#str}{\emph{\texttt{str}}}) -- Name of created object

\item {} 
\textbf{\texttt{ltype}} (\href{https://docs.python.org/2/library/functions.html\#str}{\emph{\texttt{str}}}) -- One of ``dash'', ``dash-dot'', ``solid'', ``dot'', or ``long-dash''.

\item {} 
\textbf{\texttt{width}} (\href{https://docs.python.org/2/library/functions.html\#int}{\emph{\texttt{int}}}) -- Thickness of the line to be created

\item {} 
\textbf{\texttt{color}} (\emph{\texttt{str or int}}) -- 
A color name from the \href{https://en.wikipedia.org/wiki/X11\_color\_names}{X11 Color Names list},
or an integer value from 0-255, or an RGB/RGBA tuple/list (e.g. (0,100,0), (100,100,0,50))


\item {} 
\textbf{\texttt{priority}} (\href{https://docs.python.org/2/library/functions.html\#int}{\emph{\texttt{int}}}) -- The layer on which the marker will be drawn.

\item {} 
\textbf{\texttt{viewport}} (\emph{\texttt{list of floats}}) -- 4 floats between 0 and 1. These specify the area that the X/Y values are mapped to inside of the canvas

\item {} 
\textbf{\texttt{worldcoordinate}} (\emph{\texttt{list of floats}}) -- List of 4 floats (xmin, xmax, ymin, ymax)

\item {} 
\textbf{\texttt{x}} (\emph{\texttt{list of floats}}) -- List of lists of x coordinates. Values must be between worldcoordinate{[}0{]} and worldcoordinate{[}1{]}.

\item {} 
\textbf{\texttt{y}} (\emph{\texttt{list of floats}}) -- List of lists of y coordinates. Values must be between worldcoordinate{[}2{]} and worldcoordinate{[}3{]}.

\end{itemize}

\item[{Returns}] \leavevmode
A VCS line object

\item[{Return type}] \leavevmode
{\hyperref[vcs/secondary/line:vcs.line.Tl]{\sphinxcrossref{vcs.line.Tl}}}

\end{description}\end{quote}

\end{fulllineitems}

\index{getmarker() (in module vcs.manageElements)}

\begin{fulllineitems}
\phantomsection\label{vcs/misc/manageElements:vcs.manageElements.getmarker}\pysiglinewithargsret{\sphinxcode{vcs.manageElements.}\sphinxbfcode{getmarker}}{\emph{name='default'}, \emph{mtype=None}, \emph{size=None}, \emph{color=None}, \emph{priority=None}, \emph{viewport=None}, \emph{worldcoordinate=None}, \emph{x=None}, \emph{y=None}}{}
VCS contains a list of secondary methods. This function will create a
marker class object from an existing VCS marker secondary method. If
no marker name is given, then marker `default' will be used.

\begin{notice}{note}{Note:}
VCS does not allow the modification of `default' attribute sets.
However, a `default' attribute set that has been copied under a
different name can be modified. (See the {\hyperref[vcs/misc/manageElements:vcs.manageElements.createmarker]{\sphinxcrossref{\sphinxcode{vcs.manageElements.createmarker()}}}} function.)
\end{notice}
\begin{quote}\begin{description}
\item[{Example}] \leavevmode
\begin{Verbatim}[commandchars=\\\{\}]
\PYG{g+gp}{\PYGZgt{}\PYGZgt{}\PYGZgt{} }\PYG{n}{a}\PYG{o}{=}\PYG{n}{vcs}\PYG{o}{.}\PYG{n}{init}\PYG{p}{(}\PYG{p}{)}
\PYG{g+gp}{\PYGZgt{}\PYGZgt{}\PYGZgt{} }\PYG{n}{vcs}\PYG{o}{.}\PYG{n}{listelements}\PYG{p}{(}\PYG{l+s+s1}{\PYGZsq{}}\PYG{l+s+s1}{marker}\PYG{l+s+s1}{\PYGZsq{}}\PYG{p}{)} \PYG{c+c1}{\PYGZsh{} Show all the existing marker secondary methods}
\PYG{g+go}{[...]}
\PYG{g+gp}{\PYGZgt{}\PYGZgt{}\PYGZgt{} }\PYG{n}{ex}\PYG{o}{=}\PYG{n}{vcs}\PYG{o}{.}\PYG{n}{getmarker}\PYG{p}{(}\PYG{p}{)}  \PYG{c+c1}{\PYGZsh{} instance of \PYGZsq{}default\PYGZsq{} marker secondary method}
\PYG{g+gp}{\PYGZgt{}\PYGZgt{}\PYGZgt{} }\PYG{n}{a}\PYG{o}{.}\PYG{n}{marker}\PYG{p}{(}\PYG{n}{ex}\PYG{p}{)} \PYG{c+c1}{\PYGZsh{} plot using specified marker object}
\PYG{g+go}{\PYGZlt{}vcs.displayplot.Dp ...\PYGZgt{}}
\PYG{g+gp}{\PYGZgt{}\PYGZgt{}\PYGZgt{} }\PYG{n}{ex2}\PYG{o}{=}\PYG{n}{vcs}\PYG{o}{.}\PYG{n}{getmarker}\PYG{p}{(}\PYG{l+s+s1}{\PYGZsq{}}\PYG{l+s+s1}{red}\PYG{l+s+s1}{\PYGZsq{}}\PYG{p}{)}  \PYG{c+c1}{\PYGZsh{} instance of \PYGZsq{}red\PYGZsq{} marker secondary method}
\PYG{g+gp}{\PYGZgt{}\PYGZgt{}\PYGZgt{} }\PYG{n}{a}\PYG{o}{.}\PYG{n}{marker}\PYG{p}{(}\PYG{n}{ex2}\PYG{p}{)} \PYG{c+c1}{\PYGZsh{} plot using specified marker object}
\PYG{g+go}{\PYGZlt{}vcs.displayplot.Dp ...\PYGZgt{}}
\end{Verbatim}

\item[{Parameters}] \leavevmode\begin{itemize}
\item {} 
\textbf{\texttt{name}} (\href{https://docs.python.org/2/library/functions.html\#str}{\emph{\texttt{str}}}) -- Name of created object

\item {} 
\textbf{\texttt{source}} (\href{https://docs.python.org/2/library/functions.html\#str}{\emph{\texttt{str}}}) -- A marker, or string name of a marker

\item {} 
\textbf{\texttt{mtype}} (\href{https://docs.python.org/2/library/functions.html\#str}{\emph{\texttt{str}}}) -- Specifies the type of marker, i.e. ``dot'', ``circle''

\item {} 
\textbf{\texttt{size}} (\href{https://docs.python.org/2/library/functions.html\#int}{\emph{\texttt{int}}}) -- Size of the marker

\item {} 
\textbf{\texttt{color}} (\emph{\texttt{str or int}}) -- 
A color name from the \href{https://en.wikipedia.org/wiki/X11\_color\_names}{X11 Color Names list},
or an integer value from 0-255, or an RGB/RGBA tuple/list (e.g. (0,100,0), (100,100,0,50))


\item {} 
\textbf{\texttt{priority}} (\href{https://docs.python.org/2/library/functions.html\#int}{\emph{\texttt{int}}}) -- The layer on which the marker will be drawn.

\item {} 
\textbf{\texttt{viewport}} (\emph{\texttt{list of floats}}) -- 4 floats between 0 and 1. These specify the area that the X/Y values are mapped to inside of the canvas

\item {} 
\textbf{\texttt{worldcoordinate}} (\emph{\texttt{list of floats}}) -- List of 4 floats (xmin, xmax, ymin, ymax)

\item {} 
\textbf{\texttt{x}} (\emph{\texttt{list of floats}}) -- List of lists of x coordinates. Values must be between worldcoordinate{[}0{]} and worldcoordinate{[}1{]}.

\item {} 
\textbf{\texttt{y}} (\emph{\texttt{list of floats}}) -- List of lists of y coordinates. Values must be between worldcoordinate{[}2{]} and worldcoordinate{[}3{]}.

\end{itemize}

\item[{Returns}] \leavevmode
A marker graphics method object

\item[{Return type}] \leavevmode
{\hyperref[vcs/secondary/marker:vcs.marker.Tm]{\sphinxcrossref{vcs.marker.Tm}}}

\end{description}\end{quote}

\end{fulllineitems}

\index{getmeshfill() (in module vcs.manageElements)}

\begin{fulllineitems}
\phantomsection\label{vcs/misc/manageElements:vcs.manageElements.getmeshfill}\pysiglinewithargsret{\sphinxcode{vcs.manageElements.}\sphinxbfcode{getmeshfill}}{\emph{Gfm\_name\_src='default'}}{}
VCS contains a list of graphics methods. This function will create a
meshfill class object from an existing VCS meshfill graphics method. If
no meshfill name is given, then meshfill `default' will be used.

\begin{notice}{note}{Note:}
VCS does not allow the modification of `default' attribute sets.
However, a `default' attribute set that has been copied under a
different name can be modified. (See the {\hyperref[vcs/misc/manageElements:vcs.manageElements.createmeshfill]{\sphinxcrossref{\sphinxcode{vcs.manageElements.createmeshfill()}}}} function.)
\end{notice}
\begin{quote}\begin{description}
\item[{Example}] \leavevmode
\begin{Verbatim}[commandchars=\\\{\}]
\PYG{g+gp}{\PYGZgt{}\PYGZgt{}\PYGZgt{} }\PYG{n}{a}\PYG{o}{=}\PYG{n}{vcs}\PYG{o}{.}\PYG{n}{init}\PYG{p}{(}\PYG{p}{)}
\PYG{g+gp}{\PYGZgt{}\PYGZgt{}\PYGZgt{} }\PYG{n}{vcs}\PYG{o}{.}\PYG{n}{listelements}\PYG{p}{(}\PYG{l+s+s1}{\PYGZsq{}}\PYG{l+s+s1}{meshfill}\PYG{l+s+s1}{\PYGZsq{}}\PYG{p}{)} \PYG{c+c1}{\PYGZsh{} Show all the existing meshfill graphics methods}
\PYG{g+go}{[...]}
\PYG{g+gp}{\PYGZgt{}\PYGZgt{}\PYGZgt{} }\PYG{n}{ex}\PYG{o}{=}\PYG{n}{vcs}\PYG{o}{.}\PYG{n}{getmeshfill}\PYG{p}{(}\PYG{p}{)}  \PYG{c+c1}{\PYGZsh{} instance of \PYGZsq{}default\PYGZsq{} meshfill graphics method}
\PYG{g+gp}{\PYGZgt{}\PYGZgt{}\PYGZgt{} }\PYG{k+kn}{import} \PYG{n+nn}{cdms2} \PYG{c+c1}{\PYGZsh{} Need cdms2 to create a slab}
\PYG{g+gp}{\PYGZgt{}\PYGZgt{}\PYGZgt{} }\PYG{n}{f} \PYG{o}{=} \PYG{n}{cdms2}\PYG{o}{.}\PYG{n}{open}\PYG{p}{(}\PYG{n}{vcs}\PYG{o}{.}\PYG{n}{sample\PYGZus{}data}\PYG{o}{+}\PYG{l+s+s1}{\PYGZsq{}}\PYG{l+s+s1}{/clt.nc}\PYG{l+s+s1}{\PYGZsq{}}\PYG{p}{)} \PYG{c+c1}{\PYGZsh{} use cdms2 to open a data file}
\PYG{g+gp}{\PYGZgt{}\PYGZgt{}\PYGZgt{} }\PYG{n}{slab1} \PYG{o}{=} \PYG{n}{f}\PYG{p}{(}\PYG{l+s+s1}{\PYGZsq{}}\PYG{l+s+s1}{u}\PYG{l+s+s1}{\PYGZsq{}}\PYG{p}{)} \PYG{c+c1}{\PYGZsh{} use the data file to create a cdms2 slab}
\PYG{g+gp}{\PYGZgt{}\PYGZgt{}\PYGZgt{} }\PYG{n}{a}\PYG{o}{.}\PYG{n}{meshfill}\PYG{p}{(}\PYG{n}{ex}\PYG{p}{,} \PYG{n}{slab1}\PYG{p}{)} \PYG{c+c1}{\PYGZsh{} plot using specified meshfill object}
\PYG{g+go}{\PYGZlt{}vcs.displayplot.Dp ...\PYGZgt{}}
\PYG{g+gp}{\PYGZgt{}\PYGZgt{}\PYGZgt{} }\PYG{n}{ex2}\PYG{o}{=}\PYG{n}{vcs}\PYG{o}{.}\PYG{n}{getmeshfill}\PYG{p}{(}\PYG{l+s+s1}{\PYGZsq{}}\PYG{l+s+s1}{a\PYGZus{}polar\PYGZus{}meshfill}\PYG{l+s+s1}{\PYGZsq{}}\PYG{p}{)}  \PYG{c+c1}{\PYGZsh{} instance of \PYGZsq{}a\PYGZus{}polar\PYGZus{}meshfill\PYGZsq{} meshfill graphics method}
\PYG{g+gp}{\PYGZgt{}\PYGZgt{}\PYGZgt{} }\PYG{n}{a}\PYG{o}{.}\PYG{n}{meshfill}\PYG{p}{(}\PYG{n}{ex2}\PYG{p}{,} \PYG{n}{slab1}\PYG{p}{)} \PYG{c+c1}{\PYGZsh{} plot using specified meshfill object}
\PYG{g+go}{\PYGZlt{}vcs.displayplot.Dp ...\PYGZgt{}}
\end{Verbatim}

\item[{Parameters}] \leavevmode
\textbf{\texttt{Gfm\_name\_src}} (\href{https://docs.python.org/2/library/functions.html\#str}{\emph{\texttt{str}}}) -- String name of an existing meshfill VCS object

\item[{Returns}] \leavevmode
A meshfill VCS object

\item[{Return type}] \leavevmode
{\hyperref[vcs/graphics/meshfill:vcs.meshfill.Gfm]{\sphinxcrossref{vcs.meshfill.Gfm}}}

\end{description}\end{quote}

\end{fulllineitems}

\index{getprojection() (in module vcs.manageElements)}

\begin{fulllineitems}
\phantomsection\label{vcs/misc/manageElements:vcs.manageElements.getprojection}\pysiglinewithargsret{\sphinxcode{vcs.manageElements.}\sphinxbfcode{getprojection}}{\emph{Proj\_name\_src='default'}}{}
VCS contains a list of graphics methods. This function will create a
projection class object from an existing VCS projection graphics method. If
no projection name is given, then projection `default' will be used.

\begin{notice}{note}{Note:}
VCS does not allow the modification of `default' attribute sets.
However, a `default' attribute set that has been copied under a
different name can be modified. (See the {\hyperref[vcs/misc/manageElements:vcs.manageElements.createprojection]{\sphinxcrossref{\sphinxcode{vcs.manageElements.createprojection()}}}} function.)
\end{notice}
\begin{quote}\begin{description}
\item[{Example}] \leavevmode
\begin{Verbatim}[commandchars=\\\{\}]
\PYG{g+gp}{\PYGZgt{}\PYGZgt{}\PYGZgt{} }\PYG{n}{a}\PYG{o}{=}\PYG{n}{vcs}\PYG{o}{.}\PYG{n}{init}\PYG{p}{(}\PYG{p}{)}
\PYG{g+gp}{\PYGZgt{}\PYGZgt{}\PYGZgt{} }\PYG{n}{vcs}\PYG{o}{.}\PYG{n}{listelements}\PYG{p}{(}\PYG{l+s+s1}{\PYGZsq{}}\PYG{l+s+s1}{projection}\PYG{l+s+s1}{\PYGZsq{}}\PYG{p}{)} \PYG{c+c1}{\PYGZsh{} Show all the existing projection graphics methods}
\PYG{g+go}{[...]}
\PYG{g+gp}{\PYGZgt{}\PYGZgt{}\PYGZgt{} }\PYG{n}{ex}\PYG{o}{=}\PYG{n}{vcs}\PYG{o}{.}\PYG{n}{getprojection}\PYG{p}{(}\PYG{p}{)}  \PYG{c+c1}{\PYGZsh{} instance of \PYGZsq{}default\PYGZsq{} projection graphics method}
\PYG{g+gp}{\PYGZgt{}\PYGZgt{}\PYGZgt{} }\PYG{k+kn}{import} \PYG{n+nn}{cdms2} \PYG{c+c1}{\PYGZsh{} Need cdms2 to create a slab}
\PYG{g+gp}{\PYGZgt{}\PYGZgt{}\PYGZgt{} }\PYG{n}{f} \PYG{o}{=} \PYG{n}{cdms2}\PYG{o}{.}\PYG{n}{open}\PYG{p}{(}\PYG{n}{vcs}\PYG{o}{.}\PYG{n}{sample\PYGZus{}data}\PYG{o}{+}\PYG{l+s+s1}{\PYGZsq{}}\PYG{l+s+s1}{/clt.nc}\PYG{l+s+s1}{\PYGZsq{}}\PYG{p}{)} \PYG{c+c1}{\PYGZsh{} use cdms2 to open a data file}
\PYG{g+gp}{\PYGZgt{}\PYGZgt{}\PYGZgt{} }\PYG{n}{slab1} \PYG{o}{=} \PYG{n}{f}\PYG{p}{(}\PYG{l+s+s1}{\PYGZsq{}}\PYG{l+s+s1}{u}\PYG{l+s+s1}{\PYGZsq{}}\PYG{p}{)} \PYG{c+c1}{\PYGZsh{} use the data file to create a cdms2 slab}
\PYG{g+gp}{\PYGZgt{}\PYGZgt{}\PYGZgt{} }\PYG{n}{a}\PYG{o}{.}\PYG{n}{plot}\PYG{p}{(}\PYG{n}{ex}\PYG{p}{,} \PYG{n}{slab1}\PYG{p}{)} \PYG{c+c1}{\PYGZsh{} plot using specified projection object}
\PYG{g+go}{\PYGZlt{}vcs.displayplot.Dp ...\PYGZgt{}}
\PYG{g+gp}{\PYGZgt{}\PYGZgt{}\PYGZgt{} }\PYG{n}{ex2}\PYG{o}{=}\PYG{n}{vcs}\PYG{o}{.}\PYG{n}{getprojection}\PYG{p}{(}\PYG{l+s+s1}{\PYGZsq{}}\PYG{l+s+s1}{polar}\PYG{l+s+s1}{\PYGZsq{}}\PYG{p}{)}  \PYG{c+c1}{\PYGZsh{} instance of \PYGZsq{}polar\PYGZsq{} projection graphics method}
\PYG{g+gp}{\PYGZgt{}\PYGZgt{}\PYGZgt{} }\PYG{n}{a}\PYG{o}{.}\PYG{n}{plot}\PYG{p}{(}\PYG{n}{ex2}\PYG{p}{,} \PYG{n}{slab1}\PYG{p}{)} \PYG{c+c1}{\PYGZsh{} plot using specified projection object}
\PYG{g+go}{\PYGZlt{}vcs.displayplot.Dp ...\PYGZgt{}}
\end{Verbatim}

\item[{Parameters}] \leavevmode
\textbf{\texttt{Proj\_name\_src}} (\href{https://docs.python.org/2/library/functions.html\#str}{\emph{\texttt{str}}}) -- String name of an existing VCS projection object

\item[{Returns}] \leavevmode
A VCS projection object

\item[{Return type}] \leavevmode
{\hyperref[vcs/misc/projection:vcs.projection.Proj]{\sphinxcrossref{vcs.projection.Proj}}}

\end{description}\end{quote}

\end{fulllineitems}

\index{getscatter() (in module vcs.manageElements)}

\begin{fulllineitems}
\phantomsection\label{vcs/misc/manageElements:vcs.manageElements.getscatter}\pysiglinewithargsret{\sphinxcode{vcs.manageElements.}\sphinxbfcode{getscatter}}{\emph{GSp\_name\_src='default'}}{}
VCS contains a list of graphics methods. This function will create a
scatter class object from an existing VCS scatter graphics method. If
no scatter name is given, then scatter `'{\color{red}\bfseries{}default\_scatter\_}`' will be used.

\begin{notice}{note}{Note:}
VCS does not allow the modification of `default' attribute sets.
However, a `default' attribute set that has been copied under a
different name can be modified. (See the {\hyperref[vcs/misc/manageElements:vcs.manageElements.createscatter]{\sphinxcrossref{\sphinxcode{vcs.manageElements.createscatter()}}}} function.)
\end{notice}
\begin{quote}\begin{description}
\item[{Example}] \leavevmode
\begin{Verbatim}[commandchars=\\\{\}]
\PYG{g+gp}{\PYGZgt{}\PYGZgt{}\PYGZgt{} }\PYG{n}{a}\PYG{o}{=}\PYG{n}{vcs}\PYG{o}{.}\PYG{n}{init}\PYG{p}{(}\PYG{p}{)}
\PYG{g+gp}{\PYGZgt{}\PYGZgt{}\PYGZgt{} }\PYG{n}{vcs}\PYG{o}{.}\PYG{n}{listelements}\PYG{p}{(}\PYG{l+s+s1}{\PYGZsq{}}\PYG{l+s+s1}{scatter}\PYG{l+s+s1}{\PYGZsq{}}\PYG{p}{)} \PYG{c+c1}{\PYGZsh{} Show all the existing scatter graphics methods}
\PYG{g+go}{[...]}
\PYG{g+gp}{\PYGZgt{}\PYGZgt{}\PYGZgt{} }\PYG{n}{ex}\PYG{o}{=}\PYG{n}{vcs}\PYG{o}{.}\PYG{n}{getscatter}\PYG{p}{(}\PYG{l+s+s1}{\PYGZsq{}}\PYG{l+s+s1}{default\PYGZus{}scatter\PYGZus{}}\PYG{l+s+s1}{\PYGZsq{}}\PYG{p}{)}  \PYG{c+c1}{\PYGZsh{} instance of \PYGZsq{}\PYGZsq{}default\PYGZus{}scatter\PYGZus{}\PYGZsq{}\PYGZsq{} scatter graphics method}
\PYG{g+gp}{\PYGZgt{}\PYGZgt{}\PYGZgt{} }\PYG{k+kn}{import} \PYG{n+nn}{cdms2} \PYG{c+c1}{\PYGZsh{} Need cdms2 to create a slab}
\PYG{g+gp}{\PYGZgt{}\PYGZgt{}\PYGZgt{} }\PYG{n}{f} \PYG{o}{=} \PYG{n}{cdms2}\PYG{o}{.}\PYG{n}{open}\PYG{p}{(}\PYG{n}{vcs}\PYG{o}{.}\PYG{n}{sample\PYGZus{}data}\PYG{o}{+}\PYG{l+s+s1}{\PYGZsq{}}\PYG{l+s+s1}{/clt.nc}\PYG{l+s+s1}{\PYGZsq{}}\PYG{p}{)} \PYG{c+c1}{\PYGZsh{} use cdms2 to open a data file}
\PYG{g+gp}{\PYGZgt{}\PYGZgt{}\PYGZgt{} }\PYG{n}{slab1} \PYG{o}{=} \PYG{n}{f}\PYG{p}{(}\PYG{l+s+s1}{\PYGZsq{}}\PYG{l+s+s1}{u}\PYG{l+s+s1}{\PYGZsq{}}\PYG{p}{)} \PYG{c+c1}{\PYGZsh{} use the data file to create a cdms2 slab}
\PYG{g+gp}{\PYGZgt{}\PYGZgt{}\PYGZgt{} }\PYG{n}{slab2} \PYG{o}{=} \PYG{n}{f}\PYG{p}{(}\PYG{l+s+s1}{\PYGZsq{}}\PYG{l+s+s1}{v}\PYG{l+s+s1}{\PYGZsq{}}\PYG{p}{)} \PYG{c+c1}{\PYGZsh{} need 2 slabs, so get another}
\PYG{g+gp}{\PYGZgt{}\PYGZgt{}\PYGZgt{} }\PYG{n}{a}\PYG{o}{.}\PYG{n}{scatter}\PYG{p}{(}\PYG{n}{ex}\PYG{p}{,} \PYG{n}{slab1}\PYG{p}{,} \PYG{n}{slab2}\PYG{p}{)} \PYG{c+c1}{\PYGZsh{} plot using specified scatter object}
\PYG{g+go}{\PYGZlt{}vcs.displayplot.Dp ...\PYGZgt{}}
\end{Verbatim}

\item[{Parameters}] \leavevmode\begin{itemize}
\item {} 
\textbf{\texttt{GSp\_name\_src}} (\href{https://docs.python.org/2/library/functions.html\#str}{\emph{\texttt{str}}}) -- String name of an existing scatter VCS object.

\item {} 
\textbf{\texttt{xaxis}} (\emph{\texttt{cdms2.axis.TransientAxis}}) -- Axis object to replace the slab -1 dim axis

\item {} 
\textbf{\texttt{yaxis}} (\emph{\texttt{cdms2.axis.TransientAxis}}) -- Axis object to replace the slab -2 dim axis, only if slab has more than 1D

\item {} 
\textbf{\texttt{zaxis}} (\emph{\texttt{cdms2.axis.TransientAxis}}) -- Axis object to replace the slab -3 dim axis, only if slab has more than 2D

\item {} 
\textbf{\texttt{taxis}} (\emph{\texttt{cdms2.axis.TransientAxis}}) -- Axis object to replace the slab -4 dim axis, only if slab has more than 3D

\item {} 
\textbf{\texttt{waxis}} (\emph{\texttt{cdms2.axis.TransientAxis}}) -- Axis object to replace the slab -5 dim axis, only if slab has more than 4D

\item {} 
\textbf{\texttt{xrev}} (\href{https://docs.python.org/2/library/functions.html\#bool}{\emph{\texttt{bool}}}) -- reverse x axis

\item {} 
\textbf{\texttt{yrev}} (\href{https://docs.python.org/2/library/functions.html\#bool}{\emph{\texttt{bool}}}) -- reverse y axis, only if slab has more than 1D

\item {} 
\textbf{\texttt{xarray}} (\href{https://docs.python.org/2/library/array.html\#module-array}{\emph{\texttt{array}}}) -- Values to use instead of x axis

\item {} 
\textbf{\texttt{yarray}} (\href{https://docs.python.org/2/library/array.html\#module-array}{\emph{\texttt{array}}}) -- Values to use instead of y axis, only if var has more than 1D

\item {} 
\textbf{\texttt{zarray}} (\href{https://docs.python.org/2/library/array.html\#module-array}{\emph{\texttt{array}}}) -- Values to use instead of z axis, only if var has more than 2D

\item {} 
\textbf{\texttt{tarray}} (\href{https://docs.python.org/2/library/array.html\#module-array}{\emph{\texttt{array}}}) -- Values to use instead of t axis, only if var has more than 3D

\item {} 
\textbf{\texttt{warray}} (\href{https://docs.python.org/2/library/array.html\#module-array}{\emph{\texttt{array}}}) -- Values to use instead of w axis, only if var has more than 4D

\item {} 
\textbf{\texttt{continents}} (\href{https://docs.python.org/2/library/functions.html\#int}{\emph{\texttt{int}}}) -- continents type number

\item {} 
\textbf{\texttt{name}} (\href{https://docs.python.org/2/library/functions.html\#str}{\emph{\texttt{str}}}) -- replaces variable name on plot

\item {} 
\textbf{\texttt{time}} (\emph{\texttt{A cdtime object}}) -- replaces time name on plot

\item {} 
\textbf{\texttt{units}} (\href{https://docs.python.org/2/library/functions.html\#str}{\emph{\texttt{str}}}) -- replaces units value on plot

\item {} 
\textbf{\texttt{ymd}} (\href{https://docs.python.org/2/library/functions.html\#str}{\emph{\texttt{str}}}) -- replaces year/month/day on plot

\item {} 
\textbf{\texttt{hms}} (\href{https://docs.python.org/2/library/functions.html\#str}{\emph{\texttt{str}}}) -- replaces hh/mm/ss on plot

\item {} 
\textbf{\texttt{file\_comment}} (\href{https://docs.python.org/2/library/functions.html\#str}{\emph{\texttt{str}}}) -- replaces file\_comment on plot

\item {} 
\textbf{\texttt{xbounds}} (\href{https://docs.python.org/2/library/array.html\#module-array}{\emph{\texttt{array}}}) -- Values to use instead of x axis bounds values

\item {} 
\textbf{\texttt{ybounds}} (\href{https://docs.python.org/2/library/array.html\#module-array}{\emph{\texttt{array}}}) -- Values to use instead of y axis bounds values (if exist)

\item {} 
\textbf{\texttt{xname}} (\href{https://docs.python.org/2/library/functions.html\#str}{\emph{\texttt{str}}}) -- replace xaxis name on plot

\item {} 
\textbf{\texttt{yname}} (\href{https://docs.python.org/2/library/functions.html\#str}{\emph{\texttt{str}}}) -- replace yaxis name on plot (if exists)

\item {} 
\textbf{\texttt{zname}} (\href{https://docs.python.org/2/library/functions.html\#str}{\emph{\texttt{str}}}) -- replace zaxis name on plot (if exists)

\item {} 
\textbf{\texttt{tname}} (\href{https://docs.python.org/2/library/functions.html\#str}{\emph{\texttt{str}}}) -- replace taxis name on plot (if exists)

\item {} 
\textbf{\texttt{wname}} (\href{https://docs.python.org/2/library/functions.html\#str}{\emph{\texttt{str}}}) -- replace waxis name on plot (if exists)

\item {} 
\textbf{\texttt{xunits}} (\href{https://docs.python.org/2/library/functions.html\#str}{\emph{\texttt{str}}}) -- replace xaxis units on plot

\item {} 
\textbf{\texttt{yunits}} (\href{https://docs.python.org/2/library/functions.html\#str}{\emph{\texttt{str}}}) -- replace yaxis units on plot (if exists)

\item {} 
\textbf{\texttt{zunits}} (\href{https://docs.python.org/2/library/functions.html\#str}{\emph{\texttt{str}}}) -- replace zaxis units on plot (if exists)

\item {} 
\textbf{\texttt{tunits}} (\href{https://docs.python.org/2/library/functions.html\#str}{\emph{\texttt{str}}}) -- replace taxis units on plot (if exists)

\item {} 
\textbf{\texttt{wunits}} (\href{https://docs.python.org/2/library/functions.html\#str}{\emph{\texttt{str}}}) -- replace waxis units on plot (if exists)

\item {} 
\textbf{\texttt{xweights}} (\href{https://docs.python.org/2/library/array.html\#module-array}{\emph{\texttt{array}}}) -- replace xaxis weights used for computing mean

\item {} 
\textbf{\texttt{yweights}} (\href{https://docs.python.org/2/library/array.html\#module-array}{\emph{\texttt{array}}}) -- replace xaxis weights used for computing mean

\item {} 
\textbf{\texttt{comment1}} (\href{https://docs.python.org/2/library/functions.html\#str}{\emph{\texttt{str}}}) -- replaces comment1 on plot

\item {} 
\textbf{\texttt{comment2}} (\href{https://docs.python.org/2/library/functions.html\#str}{\emph{\texttt{str}}}) -- replaces comment2 on plot

\item {} 
\textbf{\texttt{comment3}} (\href{https://docs.python.org/2/library/functions.html\#str}{\emph{\texttt{str}}}) -- replaces comment3 on plot

\item {} 
\textbf{\texttt{comment4}} (\href{https://docs.python.org/2/library/functions.html\#str}{\emph{\texttt{str}}}) -- replaces comment4 on plot

\item {} 
\textbf{\texttt{long\_name}} (\href{https://docs.python.org/2/library/functions.html\#str}{\emph{\texttt{str}}}) -- replaces long\_name on plot

\item {} 
\textbf{\texttt{grid}} (\emph{\texttt{cdms2.grid.TransientRectGrid}}) -- replaces array grid (if exists)

\item {} 
\textbf{\texttt{bg}} (\emph{\texttt{bool/int}}) -- plots in background mode

\item {} 
\textbf{\texttt{ratio}} (\index{xmtics1 (in module vcs.manageElements)}\index{xmtics2 (in module vcs.manageElements)}\index{ymtics1 (in module vcs.manageElements)}\index{ymtics2 (in module vcs.manageElements)}\index{xticlabels1 (in module vcs.manageElements)}\index{xticlabels2 (in module vcs.manageElements)}\index{yticlabels1 (in module vcs.manageElements)}\index{yticlabels2 (in module vcs.manageElements)}\index{projection (in module vcs.manageElements)}\index{datawc\_x1 (in module vcs.manageElements)}\index{datawc\_x2 (in module vcs.manageElements)}\index{datawc\_y1 (in module vcs.manageElements)}\index{datawc\_y2 (in module vcs.manageElements)}\index{datawc\_timeunits (in module vcs.manageElements)}\index{datawc\_calendar (in module vcs.manageElements)}) -- sets the y/x ratio ,if passed as a string with `t' at the end, will aslo moves the ticks

\item {} 
\textbf{\texttt{xaxisconvert}} (\href{https://docs.python.org/2/library/functions.html\#str}{\emph{\texttt{str}}}) -- (Ex: `linear') converting xaxis linear/log/log10/ln/exp/area\_wt

\item {} 
\textbf{\texttt{yaxisconvert}} (\href{https://docs.python.org/2/library/functions.html\#str}{\emph{\texttt{str}}}) -- (Ex: `linear') converting yaxis linear/log/log10/ln/exp/area\_wt

\item {} 
\textbf{\texttt{GM\_name}} -- (Ex: `default') retrieve the graphics method object of the given name. If no name is given, then retrieve the `default' graphics method.

\end{itemize}

\item[{Returns}] \leavevmode
A scatter graphics method object

\item[{Return type}] \leavevmode
{\hyperref[vcs/graphics/unified1D:vcs.unified1D.G1d]{\sphinxcrossref{vcs.unified1D.G1d}}}

\end{description}\end{quote}

\end{fulllineitems}

\index{gettaylordiagram() (in module vcs.manageElements)}

\begin{fulllineitems}
\phantomsection\label{vcs/misc/manageElements:vcs.manageElements.gettaylordiagram}\pysiglinewithargsret{\sphinxcode{vcs.manageElements.}\sphinxbfcode{gettaylordiagram}}{\emph{Gtd\_name\_src='default'}}{}
VCS contains a list of graphics methods. This function will create a
taylordiagram class object from an existing VCS taylordiagram graphics method. If
no taylordiagram name is given, then taylordiagram `default' will be used.

\begin{notice}{note}{Note:}
VCS does not allow the modification of `default' attribute sets.
However, a `default' attribute set that has been copied under a
different name can be modified. (See the {\hyperref[vcs/misc/manageElements:vcs.manageElements.createtaylordiagram]{\sphinxcrossref{\sphinxcode{vcs.manageElements.createtaylordiagram()}}}} function.)
\end{notice}
\begin{quote}\begin{description}
\item[{Example}] \leavevmode
\begin{Verbatim}[commandchars=\\\{\}]
\PYG{g+gp}{\PYGZgt{}\PYGZgt{}\PYGZgt{} }\PYG{n}{a}\PYG{o}{=}\PYG{n}{vcs}\PYG{o}{.}\PYG{n}{init}\PYG{p}{(}\PYG{p}{)}
\PYG{g+gp}{\PYGZgt{}\PYGZgt{}\PYGZgt{} }\PYG{n}{vcs}\PYG{o}{.}\PYG{n}{listelements}\PYG{p}{(}\PYG{l+s+s1}{\PYGZsq{}}\PYG{l+s+s1}{taylordiagram}\PYG{l+s+s1}{\PYGZsq{}}\PYG{p}{)} \PYG{c+c1}{\PYGZsh{} Show all the existing taylordiagram graphics methods}
\PYG{g+go}{[...]}
\PYG{g+gp}{\PYGZgt{}\PYGZgt{}\PYGZgt{} }\PYG{n}{ex}\PYG{o}{=}\PYG{n}{vcs}\PYG{o}{.}\PYG{n}{gettaylordiagram}\PYG{p}{(}\PYG{p}{)}  \PYG{c+c1}{\PYGZsh{} instance of \PYGZsq{}default\PYGZsq{} taylordiagram graphics method}
\PYG{g+gp}{\PYGZgt{}\PYGZgt{}\PYGZgt{} }\PYG{k+kn}{import} \PYG{n+nn}{cdms2} \PYG{c+c1}{\PYGZsh{} Need cdms2 to create a slab}
\PYG{g+gp}{\PYGZgt{}\PYGZgt{}\PYGZgt{} }\PYG{n}{f} \PYG{o}{=} \PYG{n}{cdms2}\PYG{o}{.}\PYG{n}{open}\PYG{p}{(}\PYG{n}{vcs}\PYG{o}{.}\PYG{n}{sample\PYGZus{}data}\PYG{o}{+}\PYG{l+s+s1}{\PYGZsq{}}\PYG{l+s+s1}{/clt.nc}\PYG{l+s+s1}{\PYGZsq{}}\PYG{p}{)} \PYG{c+c1}{\PYGZsh{} use cdms2 to open a data file}
\PYG{g+gp}{\PYGZgt{}\PYGZgt{}\PYGZgt{} }\PYG{n}{slab1} \PYG{o}{=} \PYG{n}{f}\PYG{p}{(}\PYG{l+s+s1}{\PYGZsq{}}\PYG{l+s+s1}{u}\PYG{l+s+s1}{\PYGZsq{}}\PYG{p}{)} \PYG{c+c1}{\PYGZsh{} use the data file to create a cdms2 slab}
\PYG{g+gp}{\PYGZgt{}\PYGZgt{}\PYGZgt{} }\PYG{n}{a}\PYG{o}{.}\PYG{n}{taylordiagram}\PYG{p}{(}\PYG{n}{ex}\PYG{p}{,} \PYG{n}{slab1}\PYG{p}{)} \PYG{c+c1}{\PYGZsh{} plot using specified taylordiagram object}
\PYG{g+go}{\PYGZlt{}vcs.displayplot.Dp ...\PYGZgt{}}
\end{Verbatim}

\item[{Parameters}] \leavevmode
\textbf{\texttt{Gtd\_name\_src}} (\href{https://docs.python.org/2/library/functions.html\#str}{\emph{\texttt{str}}}) -- String name of an existing taylordiagram VCS object

\item[{Returns}] \leavevmode
A taylordiagram VCS object

\item[{Return type}] \leavevmode
{\hyperref[vcs/graphics/taylor:vcs.taylor.Gtd]{\sphinxcrossref{vcs.taylor.Gtd}}}

\end{description}\end{quote}

\end{fulllineitems}

\index{gettemplate() (in module vcs.manageElements)}

\begin{fulllineitems}
\phantomsection\label{vcs/misc/manageElements:vcs.manageElements.gettemplate}\pysiglinewithargsret{\sphinxcode{vcs.manageElements.}\sphinxbfcode{gettemplate}}{\emph{Pt\_name\_src='default'}}{}
VCS contains a list of graphics methods. This function will create a
template class object from an existing VCS template graphics method. If
no template name is given, then template `default' will be used.

\begin{notice}{note}{Note:}
VCS does not allow the modification of `default' attribute sets.
However, a `default' attribute set that has been copied under a
different name can be modified. (See the {\hyperref[vcs/misc/manageElements:vcs.manageElements.createtemplate]{\sphinxcrossref{\sphinxcode{vcs.manageElements.createtemplate()}}}} function.)
\end{notice}
\begin{quote}\begin{description}
\item[{Example}] \leavevmode
\begin{Verbatim}[commandchars=\\\{\}]
\PYG{g+gp}{\PYGZgt{}\PYGZgt{}\PYGZgt{} }\PYG{n}{a}\PYG{o}{=}\PYG{n}{vcs}\PYG{o}{.}\PYG{n}{init}\PYG{p}{(}\PYG{p}{)}
\PYG{g+gp}{\PYGZgt{}\PYGZgt{}\PYGZgt{} }\PYG{n}{vcs}\PYG{o}{.}\PYG{n}{listelements}\PYG{p}{(}\PYG{l+s+s1}{\PYGZsq{}}\PYG{l+s+s1}{template}\PYG{l+s+s1}{\PYGZsq{}}\PYG{p}{)} \PYG{c+c1}{\PYGZsh{} Show all the existing template graphics methods}
\PYG{g+go}{[...]}
\PYG{g+gp}{\PYGZgt{}\PYGZgt{}\PYGZgt{} }\PYG{n}{ex}\PYG{o}{=}\PYG{n}{vcs}\PYG{o}{.}\PYG{n}{gettemplate}\PYG{p}{(}\PYG{p}{)}  \PYG{c+c1}{\PYGZsh{} instance of \PYGZsq{}default\PYGZsq{} template graphics method}
\PYG{g+gp}{\PYGZgt{}\PYGZgt{}\PYGZgt{} }\PYG{k+kn}{import} \PYG{n+nn}{cdms2} \PYG{c+c1}{\PYGZsh{} Need cdms2 to create a slab}
\PYG{g+gp}{\PYGZgt{}\PYGZgt{}\PYGZgt{} }\PYG{n}{f} \PYG{o}{=} \PYG{n}{cdms2}\PYG{o}{.}\PYG{n}{open}\PYG{p}{(}\PYG{n}{vcs}\PYG{o}{.}\PYG{n}{sample\PYGZus{}data}\PYG{o}{+}\PYG{l+s+s1}{\PYGZsq{}}\PYG{l+s+s1}{/clt.nc}\PYG{l+s+s1}{\PYGZsq{}}\PYG{p}{)} \PYG{c+c1}{\PYGZsh{} use cdms2 to open a data file}
\PYG{g+gp}{\PYGZgt{}\PYGZgt{}\PYGZgt{} }\PYG{n}{slab1} \PYG{o}{=} \PYG{n}{f}\PYG{p}{(}\PYG{l+s+s1}{\PYGZsq{}}\PYG{l+s+s1}{u}\PYG{l+s+s1}{\PYGZsq{}}\PYG{p}{)} \PYG{c+c1}{\PYGZsh{} use the data file to create a cdms2 slab}
\PYG{g+gp}{\PYGZgt{}\PYGZgt{}\PYGZgt{} }\PYG{n}{a}\PYG{o}{.}\PYG{n}{plot}\PYG{p}{(}\PYG{n}{ex}\PYG{p}{,} \PYG{n}{slab1}\PYG{p}{)} \PYG{c+c1}{\PYGZsh{} plot using specified template object}
\PYG{g+go}{\PYGZlt{}vcs.displayplot.Dp ...\PYGZgt{}}
\PYG{g+gp}{\PYGZgt{}\PYGZgt{}\PYGZgt{} }\PYG{n}{ex2}\PYG{o}{=}\PYG{n}{vcs}\PYG{o}{.}\PYG{n}{gettemplate}\PYG{p}{(}\PYG{l+s+s1}{\PYGZsq{}}\PYG{l+s+s1}{polar}\PYG{l+s+s1}{\PYGZsq{}}\PYG{p}{)}  \PYG{c+c1}{\PYGZsh{} instance of \PYGZsq{}polar\PYGZsq{} template graphics method}
\PYG{g+gp}{\PYGZgt{}\PYGZgt{}\PYGZgt{} }\PYG{n}{a}\PYG{o}{.}\PYG{n}{plot}\PYG{p}{(}\PYG{n}{ex2}\PYG{p}{,} \PYG{n}{slab1}\PYG{p}{)} \PYG{c+c1}{\PYGZsh{} plot using specified template object}
\PYG{g+go}{\PYGZlt{}vcs.displayplot.Dp ...\PYGZgt{}}
\end{Verbatim}

\item[{Parameters}] \leavevmode
\textbf{\texttt{Pt\_name\_src}} -- String name of an existing template VCS object

\item[{Returns}] \leavevmode
A VCS template object

\item[{Return type}] \leavevmode
{\hyperref[vcs/template/template:vcs.template.P]{\sphinxcrossref{vcs.template.P}}}

\end{description}\end{quote}

\end{fulllineitems}

\index{gettext() (in module vcs.manageElements)}

\begin{fulllineitems}
\phantomsection\label{vcs/misc/manageElements:vcs.manageElements.gettext}\pysiglinewithargsret{\sphinxcode{vcs.manageElements.}\sphinxbfcode{gettext}}{\emph{Tt\_name\_src='default'}, \emph{To\_name\_src=None}, \emph{string=None}, \emph{font=None}, \emph{spacing=None}, \emph{expansion=None}, \emph{color=None}, \emph{priority=None}, \emph{viewport=None}, \emph{worldcoordinate=None}, \emph{x=None}, \emph{y=None}, \emph{height=None}, \emph{angle=None}, \emph{path=None}, \emph{halign=None}, \emph{valign=None}}{}
VCS contains a list of secondary methods. This function will create a
textcombined class object from an existing VCS textcombined secondary method. If
no textcombined name is given, then textcombined `EXAMPLE\_tt:::EXAMPLE\_tto' will be used.

\begin{notice}{note}{Note:}
VCS does not allow the modification of `default' attribute sets.
However, a `default' attribute set that has been copied under a
different name can be modified. (See the {\hyperref[vcs/misc/manageElements:vcs.manageElements.createtextcombined]{\sphinxcrossref{\sphinxcode{vcs.manageElements.createtextcombined()}}}} function.)
\end{notice}
\begin{quote}\begin{description}
\item[{Example}] \leavevmode
\begin{Verbatim}[commandchars=\\\{\}]
\PYG{g+gp}{\PYGZgt{}\PYGZgt{}\PYGZgt{} }\PYG{n}{a}\PYG{o}{=}\PYG{n}{vcs}\PYG{o}{.}\PYG{n}{init}\PYG{p}{(}\PYG{p}{)}
\PYG{g+gp}{\PYGZgt{}\PYGZgt{}\PYGZgt{} }\PYG{n}{vcs}\PYG{o}{.}\PYG{n}{listelements}\PYG{p}{(}\PYG{l+s+s1}{\PYGZsq{}}\PYG{l+s+s1}{textcombined}\PYG{l+s+s1}{\PYGZsq{}}\PYG{p}{)} \PYG{c+c1}{\PYGZsh{} Show all the existing textcombined secondary methods}
\PYG{g+go}{[...]}
\PYG{g+gp}{\PYGZgt{}\PYGZgt{}\PYGZgt{} }\PYG{n}{a}\PYG{o}{.}\PYG{n}{createtextcombined}\PYG{p}{(}\PYG{l+s+s1}{\PYGZsq{}}\PYG{l+s+s1}{EXAMPLE\PYGZus{}tt}\PYG{l+s+s1}{\PYGZsq{}}\PYG{p}{,} \PYG{l+s+s1}{\PYGZsq{}}\PYG{l+s+s1}{qa}\PYG{l+s+s1}{\PYGZsq{}}\PYG{p}{,} \PYG{l+s+s1}{\PYGZsq{}}\PYG{l+s+s1}{EXAMPLE\PYGZus{}tto}\PYG{l+s+s1}{\PYGZsq{}}\PYG{p}{,} \PYG{l+s+s1}{\PYGZsq{}}\PYG{l+s+s1}{7left}\PYG{l+s+s1}{\PYGZsq{}}\PYG{p}{)} \PYG{c+c1}{\PYGZsh{} Create \PYGZsq{}EXAMPLE\PYGZus{}tt\PYGZsq{} and \PYGZsq{}EXAMPLE\PYGZus{}tto\PYGZsq{}}
\PYG{g+go}{\PYGZlt{}vcs.textcombined.Tc ...\PYGZgt{}}
\PYG{g+gp}{\PYGZgt{}\PYGZgt{}\PYGZgt{} }\PYG{n}{ex}\PYG{o}{=}\PYG{n}{vcs}\PYG{o}{.}\PYG{n}{gettextcombined}\PYG{p}{(}\PYG{l+s+s1}{\PYGZsq{}}\PYG{l+s+s1}{EXAMPLE\PYGZus{}tt}\PYG{l+s+s1}{\PYGZsq{}}\PYG{p}{,} \PYG{l+s+s1}{\PYGZsq{}}\PYG{l+s+s1}{EXAMPLE\PYGZus{}tto}\PYG{l+s+s1}{\PYGZsq{}}\PYG{p}{)}  \PYG{c+c1}{\PYGZsh{} instance of \PYGZsq{}EXAMPLE\PYGZus{}tt:::EXAMPLE\PYGZus{}tto\PYGZsq{} textcombined secondary method}
\PYG{g+gp}{\PYGZgt{}\PYGZgt{}\PYGZgt{} }\PYG{n}{a}\PYG{o}{.}\PYG{n}{textcombined}\PYG{p}{(}\PYG{n}{ex}\PYG{p}{)} \PYG{c+c1}{\PYGZsh{} plot using specified textcombined object}
\PYG{g+go}{\PYGZlt{}vcs.displayplot.Dp ...\PYGZgt{}}
\end{Verbatim}

\item[{Parameters}] \leavevmode\begin{itemize}
\item {} 
\textbf{\texttt{Tt\_name\_src}} (\href{https://docs.python.org/2/library/functions.html\#str}{\emph{\texttt{str}}}) -- Name of created object

\item {} 
\textbf{\texttt{To\_name\_src}} (\href{https://docs.python.org/2/library/functions.html\#str}{\emph{\texttt{str}}}) -- Name of parent textorientation object

\item {} 
\textbf{\texttt{string}} -- Text to render

\item {} 
\textbf{\texttt{string}} -- list of str

\item {} 
\textbf{\texttt{font}} (\emph{\texttt{int or str}}) -- Which font to use (index or name)

\item {} 
\textbf{\texttt{spacing}} (\emph{\texttt{DEPRECATED}}) -- DEPRECATED

\item {} 
\textbf{\texttt{expansion}} (\emph{\texttt{DEPRECATED}}) -- DEPRECATED

\item {} 
\textbf{\texttt{color}} (\emph{\texttt{str or int}}) -- 
A color name from the \href{https://en.wikipedia.org/wiki/X11\_color\_names}{X11 Color Names list},
or an integer value from 0-255, or an RGB/RGBA tuple/list (e.g. (0,100,0), (100,100,0,50))


\item {} 
\textbf{\texttt{priority}} (\href{https://docs.python.org/2/library/functions.html\#int}{\emph{\texttt{int}}}) -- The layer on which the object will be drawn.

\item {} 
\textbf{\texttt{viewport}} (\emph{\texttt{list of floats}}) -- 4 floats between 0 and 1. These specify the area that the X/Y values are mapped to inside of the canvas

\item {} 
\textbf{\texttt{worldcoordinate}} (\emph{\texttt{list of floats}}) -- List of 4 floats (xmin, xmax, ymin, ymax)

\item {} 
\textbf{\texttt{x}} (\emph{\texttt{list of floats}}) -- List of lists of x coordinates. Values must be between worldcoordinate{[}0{]} and worldcoordinate{[}1{]}.

\item {} 
\textbf{\texttt{y}} (\emph{\texttt{list of floats}}) -- List of lists of y coordinates. Values must be between worldcoordinate{[}2{]} and worldcoordinate{[}3{]}.

\item {} 
\textbf{\texttt{height}} (\href{https://docs.python.org/2/library/functions.html\#int}{\emph{\texttt{int}}}) -- Size of the font

\item {} 
\textbf{\texttt{angle}} (\emph{\texttt{list of int}}) -- Angle of the rendered text, in degrees

\item {} 
\textbf{\texttt{path}} (\emph{\texttt{DEPRECATED}}) -- DEPRECATED

\item {} 
\textbf{\texttt{halign}} (\href{https://docs.python.org/2/library/functions.html\#str}{\emph{\texttt{str}}}) -- Horizontal alignment of the text. One of {[}''left'', ``center'', ``right''{]}

\item {} 
\textbf{\texttt{valign}} (\href{https://docs.python.org/2/library/functions.html\#str}{\emph{\texttt{str}}}) -- Vertical alignment of the text. One of {[}''top'', ``center'', ``bottom''{]}

\end{itemize}

\item[{Returns}] \leavevmode
A textcombined object

\item[{Return type}] \leavevmode
{\hyperref[vcs/secondary/textcombined:vcs.textcombined.Tc]{\sphinxcrossref{vcs.textcombined.Tc}}}

\end{description}\end{quote}

\end{fulllineitems}

\index{gettextcombined() (in module vcs.manageElements)}

\begin{fulllineitems}
\phantomsection\label{vcs/misc/manageElements:vcs.manageElements.gettextcombined}\pysiglinewithargsret{\sphinxcode{vcs.manageElements.}\sphinxbfcode{gettextcombined}}{\emph{Tt\_name\_src='default'}, \emph{To\_name\_src=None}, \emph{string=None}, \emph{font=None}, \emph{spacing=None}, \emph{expansion=None}, \emph{color=None}, \emph{priority=None}, \emph{viewport=None}, \emph{worldcoordinate=None}, \emph{x=None}, \emph{y=None}, \emph{height=None}, \emph{angle=None}, \emph{path=None}, \emph{halign=None}, \emph{valign=None}}{}
VCS contains a list of secondary methods. This function will create a
textcombined class object from an existing VCS textcombined secondary method. If
no textcombined name is given, then textcombined `EXAMPLE\_tt:::EXAMPLE\_tto' will be used.

\begin{notice}{note}{Note:}
VCS does not allow the modification of `default' attribute sets.
However, a `default' attribute set that has been copied under a
different name can be modified. (See the {\hyperref[vcs/misc/manageElements:vcs.manageElements.createtextcombined]{\sphinxcrossref{\sphinxcode{vcs.manageElements.createtextcombined()}}}} function.)
\end{notice}
\begin{quote}\begin{description}
\item[{Example}] \leavevmode
\begin{Verbatim}[commandchars=\\\{\}]
\PYG{g+gp}{\PYGZgt{}\PYGZgt{}\PYGZgt{} }\PYG{n}{a}\PYG{o}{=}\PYG{n}{vcs}\PYG{o}{.}\PYG{n}{init}\PYG{p}{(}\PYG{p}{)}
\PYG{g+gp}{\PYGZgt{}\PYGZgt{}\PYGZgt{} }\PYG{n}{vcs}\PYG{o}{.}\PYG{n}{listelements}\PYG{p}{(}\PYG{l+s+s1}{\PYGZsq{}}\PYG{l+s+s1}{textcombined}\PYG{l+s+s1}{\PYGZsq{}}\PYG{p}{)} \PYG{c+c1}{\PYGZsh{} Show all the existing textcombined secondary methods}
\PYG{g+go}{[...]}
\PYG{g+gp}{\PYGZgt{}\PYGZgt{}\PYGZgt{} }\PYG{n}{a}\PYG{o}{.}\PYG{n}{createtextcombined}\PYG{p}{(}\PYG{l+s+s1}{\PYGZsq{}}\PYG{l+s+s1}{EXAMPLE\PYGZus{}tt}\PYG{l+s+s1}{\PYGZsq{}}\PYG{p}{,} \PYG{l+s+s1}{\PYGZsq{}}\PYG{l+s+s1}{qa}\PYG{l+s+s1}{\PYGZsq{}}\PYG{p}{,} \PYG{l+s+s1}{\PYGZsq{}}\PYG{l+s+s1}{EXAMPLE\PYGZus{}tto}\PYG{l+s+s1}{\PYGZsq{}}\PYG{p}{,} \PYG{l+s+s1}{\PYGZsq{}}\PYG{l+s+s1}{7left}\PYG{l+s+s1}{\PYGZsq{}}\PYG{p}{)} \PYG{c+c1}{\PYGZsh{} Create \PYGZsq{}EXAMPLE\PYGZus{}tt\PYGZsq{} and \PYGZsq{}EXAMPLE\PYGZus{}tto\PYGZsq{}}
\PYG{g+go}{\PYGZlt{}vcs.textcombined.Tc ...\PYGZgt{}}
\PYG{g+gp}{\PYGZgt{}\PYGZgt{}\PYGZgt{} }\PYG{n}{ex}\PYG{o}{=}\PYG{n}{vcs}\PYG{o}{.}\PYG{n}{gettextcombined}\PYG{p}{(}\PYG{l+s+s1}{\PYGZsq{}}\PYG{l+s+s1}{EXAMPLE\PYGZus{}tt}\PYG{l+s+s1}{\PYGZsq{}}\PYG{p}{,} \PYG{l+s+s1}{\PYGZsq{}}\PYG{l+s+s1}{EXAMPLE\PYGZus{}tto}\PYG{l+s+s1}{\PYGZsq{}}\PYG{p}{)}  \PYG{c+c1}{\PYGZsh{} instance of \PYGZsq{}EXAMPLE\PYGZus{}tt:::EXAMPLE\PYGZus{}tto\PYGZsq{} textcombined secondary method}
\PYG{g+gp}{\PYGZgt{}\PYGZgt{}\PYGZgt{} }\PYG{n}{a}\PYG{o}{.}\PYG{n}{textcombined}\PYG{p}{(}\PYG{n}{ex}\PYG{p}{)} \PYG{c+c1}{\PYGZsh{} plot using specified textcombined object}
\PYG{g+go}{\PYGZlt{}vcs.displayplot.Dp ...\PYGZgt{}}
\end{Verbatim}

\item[{Parameters}] \leavevmode\begin{itemize}
\item {} 
\textbf{\texttt{Tt\_name\_src}} (\href{https://docs.python.org/2/library/functions.html\#str}{\emph{\texttt{str}}}) -- Name of created object

\item {} 
\textbf{\texttt{To\_name\_src}} (\href{https://docs.python.org/2/library/functions.html\#str}{\emph{\texttt{str}}}) -- Name of parent textorientation object

\item {} 
\textbf{\texttt{string}} -- Text to render

\item {} 
\textbf{\texttt{string}} -- list of str

\item {} 
\textbf{\texttt{font}} (\emph{\texttt{int or str}}) -- Which font to use (index or name)

\item {} 
\textbf{\texttt{spacing}} (\emph{\texttt{DEPRECATED}}) -- DEPRECATED

\item {} 
\textbf{\texttt{expansion}} (\emph{\texttt{DEPRECATED}}) -- DEPRECATED

\item {} 
\textbf{\texttt{color}} (\emph{\texttt{str or int}}) -- 
A color name from the \href{https://en.wikipedia.org/wiki/X11\_color\_names}{X11 Color Names list},
or an integer value from 0-255, or an RGB/RGBA tuple/list (e.g. (0,100,0), (100,100,0,50))


\item {} 
\textbf{\texttt{priority}} (\href{https://docs.python.org/2/library/functions.html\#int}{\emph{\texttt{int}}}) -- The layer on which the object will be drawn.

\item {} 
\textbf{\texttt{viewport}} (\emph{\texttt{list of floats}}) -- 4 floats between 0 and 1. These specify the area that the X/Y values are mapped to inside of the canvas

\item {} 
\textbf{\texttt{worldcoordinate}} (\emph{\texttt{list of floats}}) -- List of 4 floats (xmin, xmax, ymin, ymax)

\item {} 
\textbf{\texttt{x}} (\emph{\texttt{list of floats}}) -- List of lists of x coordinates. Values must be between worldcoordinate{[}0{]} and worldcoordinate{[}1{]}.

\item {} 
\textbf{\texttt{y}} (\emph{\texttt{list of floats}}) -- List of lists of y coordinates. Values must be between worldcoordinate{[}2{]} and worldcoordinate{[}3{]}.

\item {} 
\textbf{\texttt{height}} (\href{https://docs.python.org/2/library/functions.html\#int}{\emph{\texttt{int}}}) -- Size of the font

\item {} 
\textbf{\texttt{angle}} (\emph{\texttt{list of int}}) -- Angle of the rendered text, in degrees

\item {} 
\textbf{\texttt{path}} (\emph{\texttt{DEPRECATED}}) -- DEPRECATED

\item {} 
\textbf{\texttt{halign}} (\href{https://docs.python.org/2/library/functions.html\#str}{\emph{\texttt{str}}}) -- Horizontal alignment of the text. One of {[}''left'', ``center'', ``right''{]}

\item {} 
\textbf{\texttt{valign}} (\href{https://docs.python.org/2/library/functions.html\#str}{\emph{\texttt{str}}}) -- Vertical alignment of the text. One of {[}''top'', ``center'', ``bottom''{]}

\end{itemize}

\item[{Returns}] \leavevmode
A textcombined object

\item[{Return type}] \leavevmode
{\hyperref[vcs/secondary/textcombined:vcs.textcombined.Tc]{\sphinxcrossref{vcs.textcombined.Tc}}}

\end{description}\end{quote}

\end{fulllineitems}

\index{gettextorientation() (in module vcs.manageElements)}

\begin{fulllineitems}
\phantomsection\label{vcs/misc/manageElements:vcs.manageElements.gettextorientation}\pysiglinewithargsret{\sphinxcode{vcs.manageElements.}\sphinxbfcode{gettextorientation}}{\emph{To\_name\_src='default'}}{}
VCS contains a list of secondary methods. This function will create a
textorientation class object from an existing VCS textorientation secondary method. If
no textorientation name is given, then textorientation `default' will be used.

\begin{notice}{note}{Note:}
VCS does not allow the modification of `default' attribute sets.
However, a `default' attribute set that has been copied under a
different name can be modified. (See the {\hyperref[vcs/misc/manageElements:vcs.manageElements.createtextorientation]{\sphinxcrossref{\sphinxcode{vcs.manageElements.createtextorientation()}}}} function.)
\end{notice}
\begin{quote}\begin{description}
\item[{Example}] \leavevmode
\begin{Verbatim}[commandchars=\\\{\}]
\PYG{g+gp}{\PYGZgt{}\PYGZgt{}\PYGZgt{} }\PYG{n}{a}\PYG{o}{=}\PYG{n}{vcs}\PYG{o}{.}\PYG{n}{init}\PYG{p}{(}\PYG{p}{)}
\PYG{g+gp}{\PYGZgt{}\PYGZgt{}\PYGZgt{} }\PYG{n}{vcs}\PYG{o}{.}\PYG{n}{listelements}\PYG{p}{(}\PYG{l+s+s1}{\PYGZsq{}}\PYG{l+s+s1}{textorientation}\PYG{l+s+s1}{\PYGZsq{}}\PYG{p}{)} \PYG{c+c1}{\PYGZsh{} Show all the existing textorientation secondary methods}
\PYG{g+go}{[...]}
\PYG{g+gp}{\PYGZgt{}\PYGZgt{}\PYGZgt{} }\PYG{n}{ex}\PYG{o}{=}\PYG{n}{vcs}\PYG{o}{.}\PYG{n}{gettextorientation}\PYG{p}{(}\PYG{p}{)}  \PYG{c+c1}{\PYGZsh{} instance of \PYGZsq{}default\PYGZsq{} textorientation secondary method}
\PYG{g+gp}{\PYGZgt{}\PYGZgt{}\PYGZgt{} }\PYG{n}{ex2}\PYG{o}{=}\PYG{n}{vcs}\PYG{o}{.}\PYG{n}{gettextorientation}\PYG{p}{(}\PYG{l+s+s1}{\PYGZsq{}}\PYG{l+s+s1}{bigger}\PYG{l+s+s1}{\PYGZsq{}}\PYG{p}{)}  \PYG{c+c1}{\PYGZsh{} instance of \PYGZsq{}bigger\PYGZsq{} textorientation secondary method}
\end{Verbatim}

\item[{Parameters}] \leavevmode
\textbf{\texttt{To\_name\_src}} (\href{https://docs.python.org/2/library/functions.html\#str}{\emph{\texttt{str}}}) -- String name of an existing textorientation VCS object

\item[{Returns}] \leavevmode
A textorientation VCS object

\item[{Return type}] \leavevmode
{\hyperref[vcs/secondary/textorientation:vcs.textorientation.To]{\sphinxcrossref{vcs.textorientation.To}}}

\end{description}\end{quote}

\end{fulllineitems}

\index{gettexttable() (in module vcs.manageElements)}

\begin{fulllineitems}
\phantomsection\label{vcs/misc/manageElements:vcs.manageElements.gettexttable}\pysiglinewithargsret{\sphinxcode{vcs.manageElements.}\sphinxbfcode{gettexttable}}{\emph{name='default'}, \emph{font=None}, \emph{spacing=None}, \emph{expansion=None}, \emph{color=None}, \emph{priority=None}, \emph{viewport=None}, \emph{worldcoordinate=None}, \emph{x=None}, \emph{y=None}}{}
VCS contains a list of secondary methods. This function will create a
texttable class object from an existing VCS texttable secondary method. If
no texttable name is given, then texttable `default' will be used.

\begin{notice}{note}{Note:}
VCS does not allow the modification of `default' attribute sets.
However, a `default' attribute set that has been copied under a
different name can be modified. (See the {\hyperref[vcs/misc/manageElements:vcs.manageElements.createtexttable]{\sphinxcrossref{\sphinxcode{vcs.manageElements.createtexttable()}}}} function.)
\end{notice}
\begin{quote}\begin{description}
\item[{Example}] \leavevmode
\begin{Verbatim}[commandchars=\\\{\}]
\PYG{g+gp}{\PYGZgt{}\PYGZgt{}\PYGZgt{} }\PYG{n}{a}\PYG{o}{=}\PYG{n}{vcs}\PYG{o}{.}\PYG{n}{init}\PYG{p}{(}\PYG{p}{)}
\PYG{g+gp}{\PYGZgt{}\PYGZgt{}\PYGZgt{} }\PYG{n}{vcs}\PYG{o}{.}\PYG{n}{listelements}\PYG{p}{(}\PYG{l+s+s1}{\PYGZsq{}}\PYG{l+s+s1}{texttable}\PYG{l+s+s1}{\PYGZsq{}}\PYG{p}{)} \PYG{c+c1}{\PYGZsh{} Show all the existing texttable secondary methods}
\PYG{g+go}{[...]}
\PYG{g+gp}{\PYGZgt{}\PYGZgt{}\PYGZgt{} }\PYG{n}{ex}\PYG{o}{=}\PYG{n}{vcs}\PYG{o}{.}\PYG{n}{gettexttable}\PYG{p}{(}\PYG{p}{)}  \PYG{c+c1}{\PYGZsh{} instance of \PYGZsq{}default\PYGZsq{} texttable secondary method}
\PYG{g+gp}{\PYGZgt{}\PYGZgt{}\PYGZgt{} }\PYG{n}{ex2}\PYG{o}{=}\PYG{n}{vcs}\PYG{o}{.}\PYG{n}{gettexttable}\PYG{p}{(}\PYG{l+s+s1}{\PYGZsq{}}\PYG{l+s+s1}{bigger}\PYG{l+s+s1}{\PYGZsq{}}\PYG{p}{)}  \PYG{c+c1}{\PYGZsh{} instance of \PYGZsq{}bigger\PYGZsq{} texttable secondary method}
\end{Verbatim}

\item[{Parameters}] \leavevmode\begin{itemize}
\item {} 
\textbf{\texttt{name}} (\href{https://docs.python.org/2/library/functions.html\#str}{\emph{\texttt{str}}}) -- String name of an existing VCS texttable object

\item {} 
\textbf{\texttt{font}} -- 
???


\item {} 
\textbf{\texttt{expansion}} -- 
???


\item {} 
\textbf{\texttt{color}} (\emph{\texttt{str or int}}) -- 
A color name from the \href{https://en.wikipedia.org/wiki/X11\_color\_names}{X11 Color Names list},
or an integer value from 0-255, or an RGB/RGBA tuple/list (e.g. (0,100,0), (100,100,0,50))


\item {} 
\textbf{\texttt{priority}} (\href{https://docs.python.org/2/library/functions.html\#int}{\emph{\texttt{int}}}) -- The layer on which the texttable will be drawn.

\item {} 
\textbf{\texttt{viewport}} (\emph{\texttt{list of floats}}) -- 4 floats between 0 and 1. These specify the area that the X/Y values are mapped to inside of the canvas

\item {} 
\textbf{\texttt{worldcoordinate}} (\emph{\texttt{list of floats}}) -- List of 4 floats (xmin, xmax, ymin, ymax)

\item {} 
\textbf{\texttt{x}} (\emph{\texttt{list of floats}}) -- List of lists of x coordinates. Values must be between worldcoordinate{[}0{]} and worldcoordinate{[}1{]}.

\item {} 
\textbf{\texttt{y}} (\emph{\texttt{list of floats}}) -- List of lists of y coordinates. Values must be between worldcoordinate{[}2{]} and worldcoordinate{[}3{]}.

\end{itemize}

\item[{Returns}] \leavevmode
A texttable graphics method object

\item[{Return type}] \leavevmode
{\hyperref[vcs/secondary/texttable:vcs.texttable.Tt]{\sphinxcrossref{vcs.texttable.Tt}}}

\end{description}\end{quote}

\end{fulllineitems}

\index{getvector() (in module vcs.manageElements)}

\begin{fulllineitems}
\phantomsection\label{vcs/misc/manageElements:vcs.manageElements.getvector}\pysiglinewithargsret{\sphinxcode{vcs.manageElements.}\sphinxbfcode{getvector}}{\emph{Gv\_name\_src='default'}}{}
VCS contains a list of graphics methods. This function will create a
vector class object from an existing VCS vector graphics method. If
no vector name is given, then vector `default' will be used.

\begin{notice}{note}{Note:}
VCS does not allow the modification of `default' attribute sets.
However, a `default' attribute set that has been copied under a
different name can be modified. (See the {\hyperref[vcs/misc/manageElements:vcs.manageElements.createvector]{\sphinxcrossref{\sphinxcode{vcs.manageElements.createvector()}}}} function.)
\end{notice}
\begin{quote}\begin{description}
\item[{Example}] \leavevmode
\begin{Verbatim}[commandchars=\\\{\}]
\PYG{g+gp}{\PYGZgt{}\PYGZgt{}\PYGZgt{} }\PYG{n}{a}\PYG{o}{=}\PYG{n}{vcs}\PYG{o}{.}\PYG{n}{init}\PYG{p}{(}\PYG{p}{)}
\PYG{g+gp}{\PYGZgt{}\PYGZgt{}\PYGZgt{} }\PYG{n}{vcs}\PYG{o}{.}\PYG{n}{listelements}\PYG{p}{(}\PYG{l+s+s1}{\PYGZsq{}}\PYG{l+s+s1}{vector}\PYG{l+s+s1}{\PYGZsq{}}\PYG{p}{)} \PYG{c+c1}{\PYGZsh{} Show all the existing vector graphics methods}
\PYG{g+go}{[...]}
\PYG{g+gp}{\PYGZgt{}\PYGZgt{}\PYGZgt{} }\PYG{n}{ex}\PYG{o}{=}\PYG{n}{vcs}\PYG{o}{.}\PYG{n}{getvector}\PYG{p}{(}\PYG{p}{)}  \PYG{c+c1}{\PYGZsh{} instance of \PYGZsq{}default\PYGZsq{} vector graphics method}
\PYG{g+gp}{\PYGZgt{}\PYGZgt{}\PYGZgt{} }\PYG{k+kn}{import} \PYG{n+nn}{cdms2} \PYG{c+c1}{\PYGZsh{} Need cdms2 to create a slab}
\PYG{g+gp}{\PYGZgt{}\PYGZgt{}\PYGZgt{} }\PYG{n}{f} \PYG{o}{=} \PYG{n}{cdms2}\PYG{o}{.}\PYG{n}{open}\PYG{p}{(}\PYG{n}{vcs}\PYG{o}{.}\PYG{n}{sample\PYGZus{}data}\PYG{o}{+}\PYG{l+s+s1}{\PYGZsq{}}\PYG{l+s+s1}{/clt.nc}\PYG{l+s+s1}{\PYGZsq{}}\PYG{p}{)} \PYG{c+c1}{\PYGZsh{} use cdms2 to open a data file}
\PYG{g+gp}{\PYGZgt{}\PYGZgt{}\PYGZgt{} }\PYG{n}{slab1} \PYG{o}{=} \PYG{n}{f}\PYG{p}{(}\PYG{l+s+s1}{\PYGZsq{}}\PYG{l+s+s1}{u}\PYG{l+s+s1}{\PYGZsq{}}\PYG{p}{)} \PYG{c+c1}{\PYGZsh{} use the data file to create a cdms2 slab}
\PYG{g+gp}{\PYGZgt{}\PYGZgt{}\PYGZgt{} }\PYG{n}{slab2} \PYG{o}{=} \PYG{n}{f}\PYG{p}{(}\PYG{l+s+s1}{\PYGZsq{}}\PYG{l+s+s1}{v}\PYG{l+s+s1}{\PYGZsq{}}\PYG{p}{)} \PYG{c+c1}{\PYGZsh{} need 2 slabs, so get another}
\PYG{g+gp}{\PYGZgt{}\PYGZgt{}\PYGZgt{} }\PYG{n}{a}\PYG{o}{.}\PYG{n}{vector}\PYG{p}{(}\PYG{n}{ex}\PYG{p}{,} \PYG{n}{slab1}\PYG{p}{,} \PYG{n}{slab2}\PYG{p}{)} \PYG{c+c1}{\PYGZsh{} plot using specified vector object}
\PYG{g+go}{\PYGZlt{}vcs.displayplot.Dp ...\PYGZgt{}}
\end{Verbatim}

\item[{Parameters}] \leavevmode
\textbf{\texttt{Gv\_name\_src}} (\href{https://docs.python.org/2/library/functions.html\#str}{\emph{\texttt{str}}}) -- String name of an existing vector VCS object

\item[{Returns}] \leavevmode
A vector graphics method object

\item[{Return type}] \leavevmode
{\hyperref[vcs/graphics/vector:vcs.vector.Gv]{\sphinxcrossref{vcs.vector.Gv}}}

\end{description}\end{quote}

\end{fulllineitems}

\index{getxvsy() (in module vcs.manageElements)}

\begin{fulllineitems}
\phantomsection\label{vcs/misc/manageElements:vcs.manageElements.getxvsy}\pysiglinewithargsret{\sphinxcode{vcs.manageElements.}\sphinxbfcode{getxvsy}}{\emph{GXY\_name\_src='default'}}{}
VCS contains a list of graphics methods. This function will create a
xvsy class object from an existing VCS xvsy graphics method. If
no xvsy name is given, then xvsy `{\color{red}\bfseries{}default\_xvsy\_}` will be used.

\begin{notice}{note}{Note:}
VCS does not allow the modification of `default' attribute sets.
However, a `default' attribute set that has been copied under a
different name can be modified. (See the {\hyperref[vcs/misc/manageElements:vcs.manageElements.createxvsy]{\sphinxcrossref{\sphinxcode{vcs.manageElements.createxvsy()}}}} function.)
\end{notice}
\begin{quote}\begin{description}
\item[{Example}] \leavevmode
\begin{Verbatim}[commandchars=\\\{\}]
\PYG{g+gp}{\PYGZgt{}\PYGZgt{}\PYGZgt{} }\PYG{n}{a}\PYG{o}{=}\PYG{n}{vcs}\PYG{o}{.}\PYG{n}{init}\PYG{p}{(}\PYG{p}{)}
\PYG{g+gp}{\PYGZgt{}\PYGZgt{}\PYGZgt{} }\PYG{n}{vcs}\PYG{o}{.}\PYG{n}{listelements}\PYG{p}{(}\PYG{l+s+s1}{\PYGZsq{}}\PYG{l+s+s1}{xvsy}\PYG{l+s+s1}{\PYGZsq{}}\PYG{p}{)} \PYG{c+c1}{\PYGZsh{} Show all the existing xvsy graphics methods}
\PYG{g+go}{[...]}
\PYG{g+gp}{\PYGZgt{}\PYGZgt{}\PYGZgt{} }\PYG{n}{ex}\PYG{o}{=}\PYG{n}{vcs}\PYG{o}{.}\PYG{n}{getxvsy}\PYG{p}{(}\PYG{p}{)}  \PYG{c+c1}{\PYGZsh{} instance of \PYGZsq{}default\PYGZus{}xvsy\PYGZus{}\PYGZsq{} xvsy graphics method}
\PYG{g+gp}{\PYGZgt{}\PYGZgt{}\PYGZgt{} }\PYG{k+kn}{import} \PYG{n+nn}{cdms2} \PYG{c+c1}{\PYGZsh{} Need cdms2 to create a slab}
\PYG{g+gp}{\PYGZgt{}\PYGZgt{}\PYGZgt{} }\PYG{n}{f} \PYG{o}{=} \PYG{n}{cdms2}\PYG{o}{.}\PYG{n}{open}\PYG{p}{(}\PYG{n}{vcs}\PYG{o}{.}\PYG{n}{sample\PYGZus{}data}\PYG{o}{+}\PYG{l+s+s1}{\PYGZsq{}}\PYG{l+s+s1}{/clt.nc}\PYG{l+s+s1}{\PYGZsq{}}\PYG{p}{)} \PYG{c+c1}{\PYGZsh{} use cdms2 to open a data file}
\PYG{g+gp}{\PYGZgt{}\PYGZgt{}\PYGZgt{} }\PYG{n}{slab1} \PYG{o}{=} \PYG{n}{f}\PYG{p}{(}\PYG{l+s+s1}{\PYGZsq{}}\PYG{l+s+s1}{u}\PYG{l+s+s1}{\PYGZsq{}}\PYG{p}{)} \PYG{c+c1}{\PYGZsh{} use the data file to create a cdms2 slab}
\PYG{g+gp}{\PYGZgt{}\PYGZgt{}\PYGZgt{} }\PYG{n}{slab2} \PYG{o}{=} \PYG{n}{f}\PYG{p}{(}\PYG{l+s+s1}{\PYGZsq{}}\PYG{l+s+s1}{v}\PYG{l+s+s1}{\PYGZsq{}}\PYG{p}{)} \PYG{c+c1}{\PYGZsh{} need 2 slabs, so get another}
\PYG{g+gp}{\PYGZgt{}\PYGZgt{}\PYGZgt{} }\PYG{n}{a}\PYG{o}{.}\PYG{n}{xvsy}\PYG{p}{(}\PYG{n}{ex}\PYG{p}{,} \PYG{n}{slab1}\PYG{p}{,} \PYG{n}{slab2}\PYG{p}{)} \PYG{c+c1}{\PYGZsh{} plot using specified xvsy object}
\PYG{g+go}{\PYGZlt{}vcs.displayplot.Dp ...\PYGZgt{}}
\end{Verbatim}

\item[{Parameters}] \leavevmode\begin{itemize}
\item {} 
\textbf{\texttt{GXY\_name\_src}} (\href{https://docs.python.org/2/library/functions.html\#str}{\emph{\texttt{str}}}) -- String name of a 1d graphics method

\item {} 
\textbf{\texttt{xaxis}} (\emph{\texttt{cdms2.axis.TransientAxis}}) -- Axis object to replace the slab -1 dim axis

\item {} 
\textbf{\texttt{yaxis}} (\emph{\texttt{cdms2.axis.TransientAxis}}) -- Axis object to replace the slab -2 dim axis, only if slab has more than 1D

\item {} 
\textbf{\texttt{zaxis}} (\emph{\texttt{cdms2.axis.TransientAxis}}) -- Axis object to replace the slab -3 dim axis, only if slab has more than 2D

\item {} 
\textbf{\texttt{taxis}} (\emph{\texttt{cdms2.axis.TransientAxis}}) -- Axis object to replace the slab -4 dim axis, only if slab has more than 3D

\item {} 
\textbf{\texttt{waxis}} (\emph{\texttt{cdms2.axis.TransientAxis}}) -- Axis object to replace the slab -5 dim axis, only if slab has more than 4D

\item {} 
\textbf{\texttt{xrev}} (\href{https://docs.python.org/2/library/functions.html\#bool}{\emph{\texttt{bool}}}) -- reverse x axis

\item {} 
\textbf{\texttt{yrev}} (\href{https://docs.python.org/2/library/functions.html\#bool}{\emph{\texttt{bool}}}) -- reverse y axis, only if slab has more than 1D

\item {} 
\textbf{\texttt{xarray}} (\href{https://docs.python.org/2/library/array.html\#module-array}{\emph{\texttt{array}}}) -- Values to use instead of x axis

\item {} 
\textbf{\texttt{yarray}} (\href{https://docs.python.org/2/library/array.html\#module-array}{\emph{\texttt{array}}}) -- Values to use instead of y axis, only if var has more than 1D

\item {} 
\textbf{\texttt{zarray}} (\href{https://docs.python.org/2/library/array.html\#module-array}{\emph{\texttt{array}}}) -- Values to use instead of z axis, only if var has more than 2D

\item {} 
\textbf{\texttt{tarray}} (\href{https://docs.python.org/2/library/array.html\#module-array}{\emph{\texttt{array}}}) -- Values to use instead of t axis, only if var has more than 3D

\item {} 
\textbf{\texttt{warray}} (\href{https://docs.python.org/2/library/array.html\#module-array}{\emph{\texttt{array}}}) -- Values to use instead of w axis, only if var has more than 4D

\item {} 
\textbf{\texttt{continents}} (\href{https://docs.python.org/2/library/functions.html\#int}{\emph{\texttt{int}}}) -- continents type number

\item {} 
\textbf{\texttt{name}} (\href{https://docs.python.org/2/library/functions.html\#str}{\emph{\texttt{str}}}) -- replaces variable name on plot

\item {} 
\textbf{\texttt{time}} (\emph{\texttt{A cdtime object}}) -- replaces time name on plot

\item {} 
\textbf{\texttt{units}} (\href{https://docs.python.org/2/library/functions.html\#str}{\emph{\texttt{str}}}) -- replaces units value on plot

\item {} 
\textbf{\texttt{ymd}} (\href{https://docs.python.org/2/library/functions.html\#str}{\emph{\texttt{str}}}) -- replaces year/month/day on plot

\item {} 
\textbf{\texttt{hms}} (\href{https://docs.python.org/2/library/functions.html\#str}{\emph{\texttt{str}}}) -- replaces hh/mm/ss on plot

\item {} 
\textbf{\texttt{file\_comment}} (\href{https://docs.python.org/2/library/functions.html\#str}{\emph{\texttt{str}}}) -- replaces file\_comment on plot

\item {} 
\textbf{\texttt{xbounds}} (\href{https://docs.python.org/2/library/array.html\#module-array}{\emph{\texttt{array}}}) -- Values to use instead of x axis bounds values

\item {} 
\textbf{\texttt{ybounds}} (\href{https://docs.python.org/2/library/array.html\#module-array}{\emph{\texttt{array}}}) -- Values to use instead of y axis bounds values (if exist)

\item {} 
\textbf{\texttt{xname}} (\href{https://docs.python.org/2/library/functions.html\#str}{\emph{\texttt{str}}}) -- replace xaxis name on plot

\item {} 
\textbf{\texttt{yname}} (\href{https://docs.python.org/2/library/functions.html\#str}{\emph{\texttt{str}}}) -- replace yaxis name on plot (if exists)

\item {} 
\textbf{\texttt{zname}} (\href{https://docs.python.org/2/library/functions.html\#str}{\emph{\texttt{str}}}) -- replace zaxis name on plot (if exists)

\item {} 
\textbf{\texttt{tname}} (\href{https://docs.python.org/2/library/functions.html\#str}{\emph{\texttt{str}}}) -- replace taxis name on plot (if exists)

\item {} 
\textbf{\texttt{wname}} (\href{https://docs.python.org/2/library/functions.html\#str}{\emph{\texttt{str}}}) -- replace waxis name on plot (if exists)

\item {} 
\textbf{\texttt{xunits}} (\href{https://docs.python.org/2/library/functions.html\#str}{\emph{\texttt{str}}}) -- replace xaxis units on plot

\item {} 
\textbf{\texttt{yunits}} (\href{https://docs.python.org/2/library/functions.html\#str}{\emph{\texttt{str}}}) -- replace yaxis units on plot (if exists)

\item {} 
\textbf{\texttt{zunits}} (\href{https://docs.python.org/2/library/functions.html\#str}{\emph{\texttt{str}}}) -- replace zaxis units on plot (if exists)

\item {} 
\textbf{\texttt{tunits}} (\href{https://docs.python.org/2/library/functions.html\#str}{\emph{\texttt{str}}}) -- replace taxis units on plot (if exists)

\item {} 
\textbf{\texttt{wunits}} (\href{https://docs.python.org/2/library/functions.html\#str}{\emph{\texttt{str}}}) -- replace waxis units on plot (if exists)

\item {} 
\textbf{\texttt{xweights}} (\href{https://docs.python.org/2/library/array.html\#module-array}{\emph{\texttt{array}}}) -- replace xaxis weights used for computing mean

\item {} 
\textbf{\texttt{yweights}} (\href{https://docs.python.org/2/library/array.html\#module-array}{\emph{\texttt{array}}}) -- replace xaxis weights used for computing mean

\item {} 
\textbf{\texttt{comment1}} (\href{https://docs.python.org/2/library/functions.html\#str}{\emph{\texttt{str}}}) -- replaces comment1 on plot

\item {} 
\textbf{\texttt{comment2}} (\href{https://docs.python.org/2/library/functions.html\#str}{\emph{\texttt{str}}}) -- replaces comment2 on plot

\item {} 
\textbf{\texttt{comment3}} (\href{https://docs.python.org/2/library/functions.html\#str}{\emph{\texttt{str}}}) -- replaces comment3 on plot

\item {} 
\textbf{\texttt{comment4}} (\href{https://docs.python.org/2/library/functions.html\#str}{\emph{\texttt{str}}}) -- replaces comment4 on plot

\item {} 
\textbf{\texttt{long\_name}} (\href{https://docs.python.org/2/library/functions.html\#str}{\emph{\texttt{str}}}) -- replaces long\_name on plot

\item {} 
\textbf{\texttt{grid}} (\emph{\texttt{cdms2.grid.TransientRectGrid}}) -- replaces array grid (if exists)

\item {} 
\textbf{\texttt{bg}} (\emph{\texttt{bool/int}}) -- plots in background mode

\item {} 
\textbf{\texttt{ratio}} (\index{xmtics1 (in module vcs.manageElements)}\index{xmtics2 (in module vcs.manageElements)}\index{ymtics1 (in module vcs.manageElements)}\index{ymtics2 (in module vcs.manageElements)}\index{xticlabels1 (in module vcs.manageElements)}\index{xticlabels2 (in module vcs.manageElements)}\index{yticlabels1 (in module vcs.manageElements)}\index{yticlabels2 (in module vcs.manageElements)}\index{projection (in module vcs.manageElements)}\index{datawc\_x1 (in module vcs.manageElements)}\index{datawc\_x2 (in module vcs.manageElements)}\index{datawc\_y1 (in module vcs.manageElements)}\index{datawc\_y2 (in module vcs.manageElements)}\index{datawc\_timeunits (in module vcs.manageElements)}\index{datawc\_calendar (in module vcs.manageElements)}) -- sets the y/x ratio ,if passed as a string with `t' at the end, will aslo moves the ticks

\item {} 
\textbf{\texttt{xaxisconvert}} (\href{https://docs.python.org/2/library/functions.html\#str}{\emph{\texttt{str}}}) -- (Ex: `linear') converting xaxis linear/log/log10/ln/exp/area\_wt

\item {} 
\textbf{\texttt{yaxisconvert}} (\href{https://docs.python.org/2/library/functions.html\#str}{\emph{\texttt{str}}}) -- (Ex: `linear') converting yaxis linear/log/log10/ln/exp/area\_wt

\item {} 
\textbf{\texttt{GM\_name}} -- (Ex: `default') retrieve the graphics method object of the given name. If no name is given, then retrieve the `default' graphics method.

\end{itemize}

\item[{Returns}] \leavevmode
A XvsY graphics method object

\item[{Return type}] \leavevmode
{\hyperref[vcs/graphics/unified1D:vcs.unified1D.G1d]{\sphinxcrossref{vcs.unified1D.G1d}}}

\end{description}\end{quote}

\end{fulllineitems}

\index{getxyvsy() (in module vcs.manageElements)}

\begin{fulllineitems}
\phantomsection\label{vcs/misc/manageElements:vcs.manageElements.getxyvsy}\pysiglinewithargsret{\sphinxcode{vcs.manageElements.}\sphinxbfcode{getxyvsy}}{\emph{GXy\_name\_src='default'}}{}
VCS contains a list of graphics methods. This function will create a
xyvsy class object from an existing VCS xyvsy graphics method. If
no xyvsy name is given, then xyvsy `'{\color{red}\bfseries{}default\_xyvsy\_}`' will be used.

\begin{notice}{note}{Note:}
VCS does not allow the modification of `default' attribute sets.
However, a `default' attribute set that has been copied under a
different name can be modified. (See the {\hyperref[vcs/misc/manageElements:vcs.manageElements.createxyvsy]{\sphinxcrossref{\sphinxcode{vcs.manageElements.createxyvsy()}}}} function.)
\end{notice}
\begin{quote}\begin{description}
\item[{Example}] \leavevmode
\begin{Verbatim}[commandchars=\\\{\}]
\PYG{g+gp}{\PYGZgt{}\PYGZgt{}\PYGZgt{} }\PYG{n}{a}\PYG{o}{=}\PYG{n}{vcs}\PYG{o}{.}\PYG{n}{init}\PYG{p}{(}\PYG{p}{)}
\PYG{g+gp}{\PYGZgt{}\PYGZgt{}\PYGZgt{} }\PYG{n}{vcs}\PYG{o}{.}\PYG{n}{listelements}\PYG{p}{(}\PYG{l+s+s1}{\PYGZsq{}}\PYG{l+s+s1}{xyvsy}\PYG{l+s+s1}{\PYGZsq{}}\PYG{p}{)} \PYG{c+c1}{\PYGZsh{} Show all the existing xyvsy graphics methods}
\PYG{g+go}{[...]}
\PYG{g+gp}{\PYGZgt{}\PYGZgt{}\PYGZgt{} }\PYG{n}{ex}\PYG{o}{=}\PYG{n}{vcs}\PYG{o}{.}\PYG{n}{getxyvsy}\PYG{p}{(}\PYG{l+s+s1}{\PYGZsq{}}\PYG{l+s+s1}{default\PYGZus{}xyvsy\PYGZus{}}\PYG{l+s+s1}{\PYGZsq{}}\PYG{p}{)}  \PYG{c+c1}{\PYGZsh{} instance of \PYGZsq{}\PYGZsq{}default\PYGZus{}xyvsy\PYGZus{}\PYGZsq{}\PYGZsq{} xyvsy graphics method}
\PYG{g+gp}{\PYGZgt{}\PYGZgt{}\PYGZgt{} }\PYG{k+kn}{import} \PYG{n+nn}{cdms2} \PYG{c+c1}{\PYGZsh{} Need cdms2 to create a slab}
\PYG{g+gp}{\PYGZgt{}\PYGZgt{}\PYGZgt{} }\PYG{n}{f} \PYG{o}{=} \PYG{n}{cdms2}\PYG{o}{.}\PYG{n}{open}\PYG{p}{(}\PYG{n}{vcs}\PYG{o}{.}\PYG{n}{sample\PYGZus{}data}\PYG{o}{+}\PYG{l+s+s1}{\PYGZsq{}}\PYG{l+s+s1}{/clt.nc}\PYG{l+s+s1}{\PYGZsq{}}\PYG{p}{)} \PYG{c+c1}{\PYGZsh{} use cdms2 to open a data file}
\PYG{g+gp}{\PYGZgt{}\PYGZgt{}\PYGZgt{} }\PYG{n}{slab1} \PYG{o}{=} \PYG{n}{f}\PYG{p}{(}\PYG{l+s+s1}{\PYGZsq{}}\PYG{l+s+s1}{u}\PYG{l+s+s1}{\PYGZsq{}}\PYG{p}{)} \PYG{c+c1}{\PYGZsh{} use the data file to create a cdms2 slab}
\PYG{g+gp}{\PYGZgt{}\PYGZgt{}\PYGZgt{} }\PYG{n}{a}\PYG{o}{.}\PYG{n}{xyvsy}\PYG{p}{(}\PYG{n}{ex}\PYG{p}{,} \PYG{n}{slab1}\PYG{p}{)} \PYG{c+c1}{\PYGZsh{} plot using specified xyvsy object}
\PYG{g+go}{\PYGZlt{}vcs.displayplot.Dp ...\PYGZgt{}}
\end{Verbatim}

\item[{Parameters}] \leavevmode\begin{itemize}
\item {} 
\textbf{\texttt{GXy\_name\_src}} (\href{https://docs.python.org/2/library/functions.html\#str}{\emph{\texttt{str}}}) -- String name of an existing Xyvsy graphics method

\item {} 
\textbf{\texttt{xaxis}} (\emph{\texttt{cdms2.axis.TransientAxis}}) -- Axis object to replace the slab -1 dim axis

\item {} 
\textbf{\texttt{yaxis}} (\emph{\texttt{cdms2.axis.TransientAxis}}) -- Axis object to replace the slab -2 dim axis, only if slab has more than 1D

\item {} 
\textbf{\texttt{zaxis}} (\emph{\texttt{cdms2.axis.TransientAxis}}) -- Axis object to replace the slab -3 dim axis, only if slab has more than 2D

\item {} 
\textbf{\texttt{taxis}} (\emph{\texttt{cdms2.axis.TransientAxis}}) -- Axis object to replace the slab -4 dim axis, only if slab has more than 3D

\item {} 
\textbf{\texttt{waxis}} (\emph{\texttt{cdms2.axis.TransientAxis}}) -- Axis object to replace the slab -5 dim axis, only if slab has more than 4D

\item {} 
\textbf{\texttt{xrev}} (\href{https://docs.python.org/2/library/functions.html\#bool}{\emph{\texttt{bool}}}) -- reverse x axis

\item {} 
\textbf{\texttt{yrev}} (\href{https://docs.python.org/2/library/functions.html\#bool}{\emph{\texttt{bool}}}) -- reverse y axis, only if slab has more than 1D

\item {} 
\textbf{\texttt{xarray}} (\href{https://docs.python.org/2/library/array.html\#module-array}{\emph{\texttt{array}}}) -- Values to use instead of x axis

\item {} 
\textbf{\texttt{yarray}} (\href{https://docs.python.org/2/library/array.html\#module-array}{\emph{\texttt{array}}}) -- Values to use instead of y axis, only if var has more than 1D

\item {} 
\textbf{\texttt{zarray}} (\href{https://docs.python.org/2/library/array.html\#module-array}{\emph{\texttt{array}}}) -- Values to use instead of z axis, only if var has more than 2D

\item {} 
\textbf{\texttt{tarray}} (\href{https://docs.python.org/2/library/array.html\#module-array}{\emph{\texttt{array}}}) -- Values to use instead of t axis, only if var has more than 3D

\item {} 
\textbf{\texttt{warray}} (\href{https://docs.python.org/2/library/array.html\#module-array}{\emph{\texttt{array}}}) -- Values to use instead of w axis, only if var has more than 4D

\item {} 
\textbf{\texttt{continents}} (\href{https://docs.python.org/2/library/functions.html\#int}{\emph{\texttt{int}}}) -- continents type number

\item {} 
\textbf{\texttt{name}} (\href{https://docs.python.org/2/library/functions.html\#str}{\emph{\texttt{str}}}) -- replaces variable name on plot

\item {} 
\textbf{\texttt{time}} (\emph{\texttt{A cdtime object}}) -- replaces time name on plot

\item {} 
\textbf{\texttt{units}} (\href{https://docs.python.org/2/library/functions.html\#str}{\emph{\texttt{str}}}) -- replaces units value on plot

\item {} 
\textbf{\texttt{ymd}} (\href{https://docs.python.org/2/library/functions.html\#str}{\emph{\texttt{str}}}) -- replaces year/month/day on plot

\item {} 
\textbf{\texttt{hms}} (\href{https://docs.python.org/2/library/functions.html\#str}{\emph{\texttt{str}}}) -- replaces hh/mm/ss on plot

\item {} 
\textbf{\texttt{file\_comment}} (\href{https://docs.python.org/2/library/functions.html\#str}{\emph{\texttt{str}}}) -- replaces file\_comment on plot

\item {} 
\textbf{\texttt{xbounds}} (\href{https://docs.python.org/2/library/array.html\#module-array}{\emph{\texttt{array}}}) -- Values to use instead of x axis bounds values

\item {} 
\textbf{\texttt{ybounds}} (\href{https://docs.python.org/2/library/array.html\#module-array}{\emph{\texttt{array}}}) -- Values to use instead of y axis bounds values (if exist)

\item {} 
\textbf{\texttt{xname}} (\href{https://docs.python.org/2/library/functions.html\#str}{\emph{\texttt{str}}}) -- replace xaxis name on plot

\item {} 
\textbf{\texttt{yname}} (\href{https://docs.python.org/2/library/functions.html\#str}{\emph{\texttt{str}}}) -- replace yaxis name on plot (if exists)

\item {} 
\textbf{\texttt{zname}} (\href{https://docs.python.org/2/library/functions.html\#str}{\emph{\texttt{str}}}) -- replace zaxis name on plot (if exists)

\item {} 
\textbf{\texttt{tname}} (\href{https://docs.python.org/2/library/functions.html\#str}{\emph{\texttt{str}}}) -- replace taxis name on plot (if exists)

\item {} 
\textbf{\texttt{wname}} (\href{https://docs.python.org/2/library/functions.html\#str}{\emph{\texttt{str}}}) -- replace waxis name on plot (if exists)

\item {} 
\textbf{\texttt{xunits}} (\href{https://docs.python.org/2/library/functions.html\#str}{\emph{\texttt{str}}}) -- replace xaxis units on plot

\item {} 
\textbf{\texttt{yunits}} (\href{https://docs.python.org/2/library/functions.html\#str}{\emph{\texttt{str}}}) -- replace yaxis units on plot (if exists)

\item {} 
\textbf{\texttt{zunits}} (\href{https://docs.python.org/2/library/functions.html\#str}{\emph{\texttt{str}}}) -- replace zaxis units on plot (if exists)

\item {} 
\textbf{\texttt{tunits}} (\href{https://docs.python.org/2/library/functions.html\#str}{\emph{\texttt{str}}}) -- replace taxis units on plot (if exists)

\item {} 
\textbf{\texttt{wunits}} (\href{https://docs.python.org/2/library/functions.html\#str}{\emph{\texttt{str}}}) -- replace waxis units on plot (if exists)

\item {} 
\textbf{\texttt{xweights}} (\href{https://docs.python.org/2/library/array.html\#module-array}{\emph{\texttt{array}}}) -- replace xaxis weights used for computing mean

\item {} 
\textbf{\texttt{yweights}} (\href{https://docs.python.org/2/library/array.html\#module-array}{\emph{\texttt{array}}}) -- replace xaxis weights used for computing mean

\item {} 
\textbf{\texttt{comment1}} (\href{https://docs.python.org/2/library/functions.html\#str}{\emph{\texttt{str}}}) -- replaces comment1 on plot

\item {} 
\textbf{\texttt{comment2}} (\href{https://docs.python.org/2/library/functions.html\#str}{\emph{\texttt{str}}}) -- replaces comment2 on plot

\item {} 
\textbf{\texttt{comment3}} (\href{https://docs.python.org/2/library/functions.html\#str}{\emph{\texttt{str}}}) -- replaces comment3 on plot

\item {} 
\textbf{\texttt{comment4}} (\href{https://docs.python.org/2/library/functions.html\#str}{\emph{\texttt{str}}}) -- replaces comment4 on plot

\item {} 
\textbf{\texttt{long\_name}} (\href{https://docs.python.org/2/library/functions.html\#str}{\emph{\texttt{str}}}) -- replaces long\_name on plot

\item {} 
\textbf{\texttt{grid}} (\emph{\texttt{cdms2.grid.TransientRectGrid}}) -- replaces array grid (if exists)

\item {} 
\textbf{\texttt{bg}} (\emph{\texttt{bool/int}}) -- plots in background mode

\item {} 
\textbf{\texttt{ratio}} (\index{xmtics1 (in module vcs.manageElements)}\index{xmtics2 (in module vcs.manageElements)}\index{ymtics1 (in module vcs.manageElements)}\index{ymtics2 (in module vcs.manageElements)}\index{xticlabels1 (in module vcs.manageElements)}\index{xticlabels2 (in module vcs.manageElements)}\index{yticlabels1 (in module vcs.manageElements)}\index{yticlabels2 (in module vcs.manageElements)}\index{projection (in module vcs.manageElements)}\index{datawc\_x1 (in module vcs.manageElements)}\index{datawc\_x2 (in module vcs.manageElements)}\index{datawc\_y1 (in module vcs.manageElements)}\index{datawc\_y2 (in module vcs.manageElements)}\index{datawc\_timeunits (in module vcs.manageElements)}\index{datawc\_calendar (in module vcs.manageElements)}) -- sets the y/x ratio ,if passed as a string with `t' at the end, will aslo moves the ticks

\item {} 
\textbf{\texttt{xaxisconvert}} (\href{https://docs.python.org/2/library/functions.html\#str}{\emph{\texttt{str}}}) -- (Ex: `linear') converting xaxis linear/log/log10/ln/exp/area\_wt

\item {} 
\textbf{\texttt{yaxisconvert}} (\href{https://docs.python.org/2/library/functions.html\#str}{\emph{\texttt{str}}}) -- (Ex: `linear') converting yaxis linear/log/log10/ln/exp/area\_wt

\item {} 
\textbf{\texttt{GM\_name}} -- (Ex: `default') retrieve the graphics method object of the given name. If no name is given, then retrieve the `default' graphics method.

\end{itemize}

\item[{Returns}] \leavevmode
An XYvsY graphics method object

\item[{Return type}] \leavevmode
{\hyperref[vcs/graphics/unified1D:vcs.unified1D.G1d]{\sphinxcrossref{vcs.unified1D.G1d}}}

\end{description}\end{quote}

\end{fulllineitems}

\index{getyxvsx() (in module vcs.manageElements)}

\begin{fulllineitems}
\phantomsection\label{vcs/misc/manageElements:vcs.manageElements.getyxvsx}\pysiglinewithargsret{\sphinxcode{vcs.manageElements.}\sphinxbfcode{getyxvsx}}{\emph{GYx\_name\_src='default'}}{}
VCS contains a list of graphics methods. This function will create a
yxvsx class object from an existing VCS yxvsx graphics method. If
no yxvsx name is given, then yxvsx `{\color{red}\bfseries{}default\_yxvsx\_}` will be used.

\begin{notice}{note}{Note:}
VCS does not allow the modification of `default' attribute sets.
However, a `default' attribute set that has been copied under a
different name can be modified. (See the {\hyperref[vcs/misc/manageElements:vcs.manageElements.createyxvsx]{\sphinxcrossref{\sphinxcode{vcs.manageElements.createyxvsx()}}}} function.)
\end{notice}
\begin{quote}\begin{description}
\item[{Example}] \leavevmode
\begin{Verbatim}[commandchars=\\\{\}]
\PYG{g+gp}{\PYGZgt{}\PYGZgt{}\PYGZgt{} }\PYG{n}{a}\PYG{o}{=}\PYG{n}{vcs}\PYG{o}{.}\PYG{n}{init}\PYG{p}{(}\PYG{p}{)}
\PYG{g+gp}{\PYGZgt{}\PYGZgt{}\PYGZgt{} }\PYG{n}{vcs}\PYG{o}{.}\PYG{n}{listelements}\PYG{p}{(}\PYG{l+s+s1}{\PYGZsq{}}\PYG{l+s+s1}{yxvsx}\PYG{l+s+s1}{\PYGZsq{}}\PYG{p}{)} \PYG{c+c1}{\PYGZsh{} Show all the existing yxvsx graphics methods}
\PYG{g+go}{[...]}
\PYG{g+gp}{\PYGZgt{}\PYGZgt{}\PYGZgt{} }\PYG{n}{ex}\PYG{o}{=}\PYG{n}{vcs}\PYG{o}{.}\PYG{n}{getyxvsx}\PYG{p}{(}\PYG{p}{)}  \PYG{c+c1}{\PYGZsh{} instance of \PYGZsq{}default\PYGZus{}yxvsx\PYGZus{}\PYGZsq{} yxvsx graphics method}
\PYG{g+gp}{\PYGZgt{}\PYGZgt{}\PYGZgt{} }\PYG{k+kn}{import} \PYG{n+nn}{cdms2} \PYG{c+c1}{\PYGZsh{} Need cdms2 to create a slab}
\PYG{g+gp}{\PYGZgt{}\PYGZgt{}\PYGZgt{} }\PYG{n}{f} \PYG{o}{=} \PYG{n}{cdms2}\PYG{o}{.}\PYG{n}{open}\PYG{p}{(}\PYG{n}{vcs}\PYG{o}{.}\PYG{n}{sample\PYGZus{}data}\PYG{o}{+}\PYG{l+s+s1}{\PYGZsq{}}\PYG{l+s+s1}{/clt.nc}\PYG{l+s+s1}{\PYGZsq{}}\PYG{p}{)} \PYG{c+c1}{\PYGZsh{} use cdms2 to open a data file}
\PYG{g+gp}{\PYGZgt{}\PYGZgt{}\PYGZgt{} }\PYG{n}{slab1} \PYG{o}{=} \PYG{n}{f}\PYG{p}{(}\PYG{l+s+s1}{\PYGZsq{}}\PYG{l+s+s1}{u}\PYG{l+s+s1}{\PYGZsq{}}\PYG{p}{)} \PYG{c+c1}{\PYGZsh{} use the data file to create a cdms2 slab}
\PYG{g+gp}{\PYGZgt{}\PYGZgt{}\PYGZgt{} }\PYG{n}{a}\PYG{o}{.}\PYG{n}{yxvsx}\PYG{p}{(}\PYG{n}{ex}\PYG{p}{,} \PYG{n}{slab1}\PYG{p}{)} \PYG{c+c1}{\PYGZsh{} plot using specified yxvsx object}
\PYG{g+go}{\PYGZlt{}vcs.displayplot.Dp ...\PYGZgt{}}
\end{Verbatim}

\item[{Parameters}] \leavevmode\begin{itemize}
\item {} 
\textbf{\texttt{GYx\_name\_src}} (\href{https://docs.python.org/2/library/functions.html\#str}{\emph{\texttt{str}}}) -- String name of an existing Yxvsx graphics method

\item {} 
\textbf{\texttt{xaxis}} (\emph{\texttt{cdms2.axis.TransientAxis}}) -- Axis object to replace the slab -1 dim axis

\item {} 
\textbf{\texttt{yaxis}} (\emph{\texttt{cdms2.axis.TransientAxis}}) -- Axis object to replace the slab -2 dim axis, only if slab has more than 1D

\item {} 
\textbf{\texttt{zaxis}} (\emph{\texttt{cdms2.axis.TransientAxis}}) -- Axis object to replace the slab -3 dim axis, only if slab has more than 2D

\item {} 
\textbf{\texttt{taxis}} (\emph{\texttt{cdms2.axis.TransientAxis}}) -- Axis object to replace the slab -4 dim axis, only if slab has more than 3D

\item {} 
\textbf{\texttt{waxis}} (\emph{\texttt{cdms2.axis.TransientAxis}}) -- Axis object to replace the slab -5 dim axis, only if slab has more than 4D

\item {} 
\textbf{\texttt{xrev}} (\href{https://docs.python.org/2/library/functions.html\#bool}{\emph{\texttt{bool}}}) -- reverse x axis

\item {} 
\textbf{\texttt{yrev}} (\href{https://docs.python.org/2/library/functions.html\#bool}{\emph{\texttt{bool}}}) -- reverse y axis, only if slab has more than 1D

\item {} 
\textbf{\texttt{xarray}} (\href{https://docs.python.org/2/library/array.html\#module-array}{\emph{\texttt{array}}}) -- Values to use instead of x axis

\item {} 
\textbf{\texttt{yarray}} (\href{https://docs.python.org/2/library/array.html\#module-array}{\emph{\texttt{array}}}) -- Values to use instead of y axis, only if var has more than 1D

\item {} 
\textbf{\texttt{zarray}} (\href{https://docs.python.org/2/library/array.html\#module-array}{\emph{\texttt{array}}}) -- Values to use instead of z axis, only if var has more than 2D

\item {} 
\textbf{\texttt{tarray}} (\href{https://docs.python.org/2/library/array.html\#module-array}{\emph{\texttt{array}}}) -- Values to use instead of t axis, only if var has more than 3D

\item {} 
\textbf{\texttt{warray}} (\href{https://docs.python.org/2/library/array.html\#module-array}{\emph{\texttt{array}}}) -- Values to use instead of w axis, only if var has more than 4D

\item {} 
\textbf{\texttt{continents}} (\href{https://docs.python.org/2/library/functions.html\#int}{\emph{\texttt{int}}}) -- continents type number

\item {} 
\textbf{\texttt{name}} (\href{https://docs.python.org/2/library/functions.html\#str}{\emph{\texttt{str}}}) -- replaces variable name on plot

\item {} 
\textbf{\texttt{time}} (\emph{\texttt{A cdtime object}}) -- replaces time name on plot

\item {} 
\textbf{\texttt{units}} (\href{https://docs.python.org/2/library/functions.html\#str}{\emph{\texttt{str}}}) -- replaces units value on plot

\item {} 
\textbf{\texttt{ymd}} (\href{https://docs.python.org/2/library/functions.html\#str}{\emph{\texttt{str}}}) -- replaces year/month/day on plot

\item {} 
\textbf{\texttt{hms}} (\href{https://docs.python.org/2/library/functions.html\#str}{\emph{\texttt{str}}}) -- replaces hh/mm/ss on plot

\item {} 
\textbf{\texttt{file\_comment}} (\href{https://docs.python.org/2/library/functions.html\#str}{\emph{\texttt{str}}}) -- replaces file\_comment on plot

\item {} 
\textbf{\texttt{xbounds}} (\href{https://docs.python.org/2/library/array.html\#module-array}{\emph{\texttt{array}}}) -- Values to use instead of x axis bounds values

\item {} 
\textbf{\texttt{ybounds}} (\href{https://docs.python.org/2/library/array.html\#module-array}{\emph{\texttt{array}}}) -- Values to use instead of y axis bounds values (if exist)

\item {} 
\textbf{\texttt{xname}} (\href{https://docs.python.org/2/library/functions.html\#str}{\emph{\texttt{str}}}) -- replace xaxis name on plot

\item {} 
\textbf{\texttt{yname}} (\href{https://docs.python.org/2/library/functions.html\#str}{\emph{\texttt{str}}}) -- replace yaxis name on plot (if exists)

\item {} 
\textbf{\texttt{zname}} (\href{https://docs.python.org/2/library/functions.html\#str}{\emph{\texttt{str}}}) -- replace zaxis name on plot (if exists)

\item {} 
\textbf{\texttt{tname}} (\href{https://docs.python.org/2/library/functions.html\#str}{\emph{\texttt{str}}}) -- replace taxis name on plot (if exists)

\item {} 
\textbf{\texttt{wname}} (\href{https://docs.python.org/2/library/functions.html\#str}{\emph{\texttt{str}}}) -- replace waxis name on plot (if exists)

\item {} 
\textbf{\texttt{xunits}} (\href{https://docs.python.org/2/library/functions.html\#str}{\emph{\texttt{str}}}) -- replace xaxis units on plot

\item {} 
\textbf{\texttt{yunits}} (\href{https://docs.python.org/2/library/functions.html\#str}{\emph{\texttt{str}}}) -- replace yaxis units on plot (if exists)

\item {} 
\textbf{\texttt{zunits}} (\href{https://docs.python.org/2/library/functions.html\#str}{\emph{\texttt{str}}}) -- replace zaxis units on plot (if exists)

\item {} 
\textbf{\texttt{tunits}} (\href{https://docs.python.org/2/library/functions.html\#str}{\emph{\texttt{str}}}) -- replace taxis units on plot (if exists)

\item {} 
\textbf{\texttt{wunits}} (\href{https://docs.python.org/2/library/functions.html\#str}{\emph{\texttt{str}}}) -- replace waxis units on plot (if exists)

\item {} 
\textbf{\texttt{xweights}} (\href{https://docs.python.org/2/library/array.html\#module-array}{\emph{\texttt{array}}}) -- replace xaxis weights used for computing mean

\item {} 
\textbf{\texttt{yweights}} (\href{https://docs.python.org/2/library/array.html\#module-array}{\emph{\texttt{array}}}) -- replace xaxis weights used for computing mean

\item {} 
\textbf{\texttt{comment1}} (\href{https://docs.python.org/2/library/functions.html\#str}{\emph{\texttt{str}}}) -- replaces comment1 on plot

\item {} 
\textbf{\texttt{comment2}} (\href{https://docs.python.org/2/library/functions.html\#str}{\emph{\texttt{str}}}) -- replaces comment2 on plot

\item {} 
\textbf{\texttt{comment3}} (\href{https://docs.python.org/2/library/functions.html\#str}{\emph{\texttt{str}}}) -- replaces comment3 on plot

\item {} 
\textbf{\texttt{comment4}} (\href{https://docs.python.org/2/library/functions.html\#str}{\emph{\texttt{str}}}) -- replaces comment4 on plot

\item {} 
\textbf{\texttt{long\_name}} (\href{https://docs.python.org/2/library/functions.html\#str}{\emph{\texttt{str}}}) -- replaces long\_name on plot

\item {} 
\textbf{\texttt{grid}} (\emph{\texttt{cdms2.grid.TransientRectGrid}}) -- replaces array grid (if exists)

\item {} 
\textbf{\texttt{bg}} (\emph{\texttt{bool/int}}) -- plots in background mode

\item {} 
\textbf{\texttt{ratio}} (\index{xmtics1 (in module vcs.manageElements)}\index{xmtics2 (in module vcs.manageElements)}\index{ymtics1 (in module vcs.manageElements)}\index{ymtics2 (in module vcs.manageElements)}\index{xticlabels1 (in module vcs.manageElements)}\index{xticlabels2 (in module vcs.manageElements)}\index{yticlabels1 (in module vcs.manageElements)}\index{yticlabels2 (in module vcs.manageElements)}\index{projection (in module vcs.manageElements)}\index{datawc\_x1 (in module vcs.manageElements)}\index{datawc\_x2 (in module vcs.manageElements)}\index{datawc\_y1 (in module vcs.manageElements)}\index{datawc\_y2 (in module vcs.manageElements)}\index{datawc\_timeunits (in module vcs.manageElements)}\index{datawc\_calendar (in module vcs.manageElements)}) -- sets the y/x ratio ,if passed as a string with `t' at the end, will aslo moves the ticks

\item {} 
\textbf{\texttt{xaxisconvert}} (\href{https://docs.python.org/2/library/functions.html\#str}{\emph{\texttt{str}}}) -- (Ex: `linear') converting xaxis linear/log/log10/ln/exp/area\_wt

\item {} 
\textbf{\texttt{yaxisconvert}} (\href{https://docs.python.org/2/library/functions.html\#str}{\emph{\texttt{str}}}) -- (Ex: `linear') converting yaxis linear/log/log10/ln/exp/area\_wt

\item {} 
\textbf{\texttt{GM\_name}} -- (Ex: `default') retrieve the graphics method object of the given name. If no name is given, then retrieve the `default' graphics method.

\end{itemize}

\item[{Returns}] \leavevmode
A Yxvsx graphics method object

\item[{Return type}] \leavevmode
{\hyperref[vcs/graphics/unified1D:vcs.unified1D.G1d]{\sphinxcrossref{vcs.unified1D.G1d}}}

\end{description}\end{quote}

\end{fulllineitems}

\index{removeobject() (in module vcs.manageElements)}

\begin{fulllineitems}
\phantomsection\label{vcs/misc/manageElements:vcs.manageElements.removeobject}\pysiglinewithargsret{\sphinxcode{vcs.manageElements.}\sphinxbfcode{removeobject}}{\emph{obj}}{}
The user has the ability to create primary and secondary class
objects. The function allows the user to remove these objects
from the appropriate class list.

Note, To remove the object completely from Python, remember to
use the ``del'' function.

Also note, The user is not allowed to remove a ``default'' class
object.
\begin{quote}\begin{description}
\item[{Example}] \leavevmode
\begin{Verbatim}[commandchars=\\\{\}]
\PYG{g+gp}{\PYGZgt{}\PYGZgt{}\PYGZgt{} }\PYG{n}{a}\PYG{o}{=}\PYG{n}{vcs}\PYG{o}{.}\PYG{n}{init}\PYG{p}{(}\PYG{p}{)}
\PYG{g+gp}{\PYGZgt{}\PYGZgt{}\PYGZgt{} }\PYG{n}{line}\PYG{o}{=}\PYG{n}{a}\PYG{o}{.}\PYG{n}{getline}\PYG{p}{(}\PYG{l+s+s1}{\PYGZsq{}}\PYG{l+s+s1}{red}\PYG{l+s+s1}{\PYGZsq{}}\PYG{p}{)} \PYG{c+c1}{\PYGZsh{} To Modify an existing line object}
\PYG{g+gp}{\PYGZgt{}\PYGZgt{}\PYGZgt{} }\PYG{n}{iso}\PYG{o}{=}\PYG{n}{a}\PYG{o}{.}\PYG{n}{createisoline}\PYG{p}{(}\PYG{l+s+s1}{\PYGZsq{}}\PYG{l+s+s1}{dean}\PYG{l+s+s1}{\PYGZsq{}}\PYG{p}{)} \PYG{c+c1}{\PYGZsh{} Create an instance of an isoline object}
\PYG{g+gp}{\PYGZgt{}\PYGZgt{}\PYGZgt{} }\PYG{n}{a}\PYG{o}{.}\PYG{n}{removeobject}\PYG{p}{(}\PYG{n}{line}\PYG{p}{)} \PYG{c+c1}{\PYGZsh{} Removes line object from VCS list}
\PYG{g+go}{\PYGZsq{}Removed line object red\PYGZsq{}}
\PYG{g+gp}{\PYGZgt{}\PYGZgt{}\PYGZgt{} }\PYG{n}{a}\PYG{o}{.}\PYG{n}{removeobject}\PYG{p}{(}\PYG{n}{iso}\PYG{p}{)} \PYG{c+c1}{\PYGZsh{} Remove isoline object from VCS list}
\PYG{g+go}{\PYGZsq{}Removed isoline object dean\PYGZsq{}}
\end{Verbatim}

\item[{Parameters}] \leavevmode
\textbf{\texttt{obj}} (\emph{\texttt{VCS object}}) -- Any VCS primary or secondary object

\item[{Returns}] \leavevmode
String indicating the specified object was removed

\item[{Return type}] \leavevmode
\href{https://docs.python.org/2/library/functions.html\#str}{str}

\end{description}\end{quote}

\end{fulllineitems}



\subsection{queries}
\label{vcs/misc/queries:module-vcs.queries}\label{vcs/misc/queries::doc}\label{vcs/misc/queries:queries}\index{vcs.queries (module)}\index{graphicsmethodlist() (in module vcs.queries)}

\begin{fulllineitems}
\phantomsection\label{vcs/misc/queries:vcs.queries.graphicsmethodlist}\pysiglinewithargsret{\sphinxcode{vcs.queries.}\sphinxbfcode{graphicsmethodlist}}{}{}
List available graphics methods.
\begin{quote}\begin{description}
\item[{Example}] \leavevmode
\begin{Verbatim}[commandchars=\\\{\}]
\PYG{g+gp}{\PYGZgt{}\PYGZgt{}\PYGZgt{} }\PYG{n}{a}\PYG{o}{=}\PYG{n}{vcs}\PYG{o}{.}\PYG{n}{init}\PYG{p}{(}\PYG{p}{)}
\PYG{g+gp}{\PYGZgt{}\PYGZgt{}\PYGZgt{} }\PYG{n}{vcs}\PYG{o}{.}\PYG{n}{graphicsmethodlist}\PYG{p}{(}\PYG{p}{)} \PYG{c+c1}{\PYGZsh{} Return graphics method list}
\PYG{g+go}{[...]}
\end{Verbatim}

\item[{Returns}] \leavevmode
A list of available grapics methods (i.e., boxfill, isofill, isoline, outfill,
scatter, vector, xvsy, xyvsy, yxvsx, taylordiagram ).

\item[{Return type}] \leavevmode
{\hyperref[vcs/graphics/boxfill:vcs.boxfill.Gfb.list]{\sphinxcrossref{list}}}

\end{description}\end{quote}

\end{fulllineitems}

\index{graphicsmethodtype() (in module vcs.queries)}

\begin{fulllineitems}
\phantomsection\label{vcs/misc/queries:vcs.queries.graphicsmethodtype}\pysiglinewithargsret{\sphinxcode{vcs.queries.}\sphinxbfcode{graphicsmethodtype}}{\emph{gobj}}{}
Check the type of a graphics object.

Returns None if the object is not a graphics method.
\begin{quote}\begin{description}
\item[{Example}] \leavevmode
\begin{Verbatim}[commandchars=\\\{\}]
\PYG{g+gp}{\PYGZgt{}\PYGZgt{}\PYGZgt{} }\PYG{n}{a}\PYG{o}{=}\PYG{n}{vcs}\PYG{o}{.}\PYG{n}{init}\PYG{p}{(}\PYG{p}{)}
\PYG{g+gp}{\PYGZgt{}\PYGZgt{}\PYGZgt{} }\PYG{n}{box}\PYG{o}{=}\PYG{n}{a}\PYG{o}{.}\PYG{n}{getboxfill}\PYG{p}{(}\PYG{p}{)} \PYG{c+c1}{\PYGZsh{} Get default boxfill graphics method}
\PYG{g+gp}{\PYGZgt{}\PYGZgt{}\PYGZgt{} }\PYG{n}{iso}\PYG{o}{=}\PYG{n}{a}\PYG{o}{.}\PYG{n}{getisofill}\PYG{p}{(}\PYG{p}{)} \PYG{c+c1}{\PYGZsh{} Get default isofill graphics method}
\PYG{g+gp}{\PYGZgt{}\PYGZgt{}\PYGZgt{} }\PYG{n}{ln}\PYG{o}{=}\PYG{n}{a}\PYG{o}{.}\PYG{n}{getline}\PYG{p}{(}\PYG{p}{)}\PYG{c+c1}{\PYGZsh{} Get default line element}
\PYG{g+gp}{\PYGZgt{}\PYGZgt{}\PYGZgt{} }\PYG{n}{vcs}\PYG{o}{.}\PYG{n}{graphicsmethodtype}\PYG{p}{(}\PYG{n}{box}\PYG{p}{)}
\PYG{g+go}{\PYGZsq{}boxfill\PYGZsq{}}
\PYG{g+gp}{\PYGZgt{}\PYGZgt{}\PYGZgt{} }\PYG{n}{vcs}\PYG{o}{.}\PYG{n}{graphicsmethodtype}\PYG{p}{(}\PYG{n}{iso}\PYG{p}{)}
\PYG{g+go}{\PYGZsq{}isofill\PYGZsq{}}
\PYG{g+gp}{\PYGZgt{}\PYGZgt{}\PYGZgt{} }\PYG{n}{vcs}\PYG{o}{.}\PYG{n}{graphicsmethodtype}\PYG{p}{(}\PYG{n}{ln}\PYG{p}{)}
\PYG{g+gt}{Traceback (most recent call last):}
\PYG{c}{...}
\PYG{g+gr}{vcsError}: \PYG{n}{The object passed is not a graphics method object.}
\end{Verbatim}
\begin{quote}\begin{description}
\item[{returns}] \leavevmode
If gobj is a graphics method object, returns its type: `boxfill', `isofill', `isoline',
`scatter', `vector', `xvsy', `xyvsy', or `yxvsx', `taylordiagram'.
If gobj is not a graphics method object, raises an exception and prints a vcsError message.

\item[{rtype}] \leavevmode
str or None

\end{description}\end{quote}

\end{description}\end{quote}

\end{fulllineitems}

\index{is1d() (in module vcs.queries)}

\begin{fulllineitems}
\phantomsection\label{vcs/misc/queries:vcs.queries.is1d}\pysiglinewithargsret{\sphinxcode{vcs.queries.}\sphinxbfcode{is1d}}{\emph{obj}}{}
Check to see if this object is a VCS 1d graphics method.
\begin{quote}\begin{description}
\item[{Example}] \leavevmode
\begin{Verbatim}[commandchars=\\\{\}]
\PYG{g+gp}{\PYGZgt{}\PYGZgt{}\PYGZgt{} }\PYG{n}{a}\PYG{o}{=}\PYG{n}{vcs}\PYG{o}{.}\PYG{n}{init}\PYG{p}{(}\PYG{p}{)} \PYG{c+c1}{\PYGZsh{} Make a VCS Canvas object to work with:}
\PYG{g+gp}{\PYGZgt{}\PYGZgt{}\PYGZgt{} }\PYG{n}{a}\PYG{o}{.}\PYG{n}{show}\PYG{p}{(}\PYG{l+s+s1}{\PYGZsq{}}\PYG{l+s+s1}{1d}\PYG{l+s+s1}{\PYGZsq{}}\PYG{p}{)} \PYG{c+c1}{\PYGZsh{} Show all available 1d}
\PYG{g+go}{*******************1d Names List**********************}
\PYG{g+gp}{...}
\PYG{g+go}{*******************End 1d Names List**********************}
\PYG{g+gp}{\PYGZgt{}\PYGZgt{}\PYGZgt{} }\PYG{n}{ex} \PYG{o}{=} \PYG{n}{a}\PYG{o}{.}\PYG{n}{get1d}\PYG{p}{(}\PYG{l+s+s1}{\PYGZsq{}}\PYG{l+s+s1}{default}\PYG{l+s+s1}{\PYGZsq{}}\PYG{p}{)} \PYG{c+c1}{\PYGZsh{} To  test an existing 1d object}
\PYG{g+gp}{\PYGZgt{}\PYGZgt{}\PYGZgt{} }\PYG{n}{vcs}\PYG{o}{.}\PYG{n}{queries}\PYG{o}{.}\PYG{n}{is1d}\PYG{p}{(}\PYG{n}{ex}\PYG{p}{)}
\PYG{g+go}{1}
\end{Verbatim}

\item[{Parameters}] \leavevmode
\textbf{\texttt{obj}} (\emph{\texttt{VCS Object}}) -- A VCS object

\item[{Returns}] \leavevmode
An integer indicating whether the object is a 1d graphics method (1), or not (0).

\item[{Return type}] \leavevmode
\href{https://docs.python.org/2/library/functions.html\#int}{int}

\end{description}\end{quote}

\end{fulllineitems}

\index{is3d\_dual\_scalar() (in module vcs.queries)}

\begin{fulllineitems}
\phantomsection\label{vcs/misc/queries:vcs.queries.is3d_dual_scalar}\pysiglinewithargsret{\sphinxcode{vcs.queries.}\sphinxbfcode{is3d\_dual\_scalar}}{\emph{obj}}{}
Check to see if this object is a VCS 3d\_dual\_scalar graphics method.
\begin{quote}\begin{description}
\item[{Example}] \leavevmode
\begin{Verbatim}[commandchars=\\\{\}]
\PYG{g+gp}{\PYGZgt{}\PYGZgt{}\PYGZgt{} }\PYG{n}{a}\PYG{o}{=}\PYG{n}{vcs}\PYG{o}{.}\PYG{n}{init}\PYG{p}{(}\PYG{p}{)} \PYG{c+c1}{\PYGZsh{} Make a VCS Canvas object to work with:}
\PYG{g+gp}{\PYGZgt{}\PYGZgt{}\PYGZgt{} }\PYG{n}{a}\PYG{o}{.}\PYG{n}{show}\PYG{p}{(}\PYG{l+s+s1}{\PYGZsq{}}\PYG{l+s+s1}{3d\PYGZus{}dual\PYGZus{}scalar}\PYG{l+s+s1}{\PYGZsq{}}\PYG{p}{)} \PYG{c+c1}{\PYGZsh{} Show all available 3d\PYGZus{}dual\PYGZus{}scalar}
\PYG{g+go}{*******************3d\PYGZus{}dual\PYGZus{}scalar Names List**********************}
\PYG{g+gp}{...}
\PYG{g+go}{*******************End 3d\PYGZus{}dual\PYGZus{}scalar Names List**********************}
\PYG{g+gp}{\PYGZgt{}\PYGZgt{}\PYGZgt{} }\PYG{n}{ex} \PYG{o}{=} \PYG{n}{a}\PYG{o}{.}\PYG{n}{get3d\PYGZus{}dual\PYGZus{}scalar}\PYG{p}{(}\PYG{p}{)} \PYG{c+c1}{\PYGZsh{} To  test an existing 3d\PYGZus{}dual\PYGZus{}scalar object}
\PYG{g+gp}{\PYGZgt{}\PYGZgt{}\PYGZgt{} }\PYG{n}{vcs}\PYG{o}{.}\PYG{n}{queries}\PYG{o}{.}\PYG{n}{is3d\PYGZus{}dual\PYGZus{}scalar}\PYG{p}{(}\PYG{n}{ex}\PYG{p}{)}
\PYG{g+go}{1}
\end{Verbatim}

\item[{Parameters}] \leavevmode
\textbf{\texttt{obj}} (\emph{\texttt{VCS Object}}) -- A VCS object

\item[{Returns}] \leavevmode
An integer indicating whether the object is a 3d\_dual\_scalar graphics method (1), or not (0).

\item[{Return type}] \leavevmode
\href{https://docs.python.org/2/library/functions.html\#int}{int}

\end{description}\end{quote}

\end{fulllineitems}

\index{is3d\_scalar() (in module vcs.queries)}

\begin{fulllineitems}
\phantomsection\label{vcs/misc/queries:vcs.queries.is3d_scalar}\pysiglinewithargsret{\sphinxcode{vcs.queries.}\sphinxbfcode{is3d\_scalar}}{\emph{obj}}{}
Check to see if this object is a VCS 3d\_scalar graphics method.
\begin{quote}\begin{description}
\item[{Example}] \leavevmode
\begin{Verbatim}[commandchars=\\\{\}]
\PYG{g+gp}{\PYGZgt{}\PYGZgt{}\PYGZgt{} }\PYG{n}{a}\PYG{o}{=}\PYG{n}{vcs}\PYG{o}{.}\PYG{n}{init}\PYG{p}{(}\PYG{p}{)} \PYG{c+c1}{\PYGZsh{} Make a VCS Canvas object to work with:}
\PYG{g+gp}{\PYGZgt{}\PYGZgt{}\PYGZgt{} }\PYG{n}{a}\PYG{o}{.}\PYG{n}{show}\PYG{p}{(}\PYG{l+s+s1}{\PYGZsq{}}\PYG{l+s+s1}{3d\PYGZus{}scalar}\PYG{l+s+s1}{\PYGZsq{}}\PYG{p}{)} \PYG{c+c1}{\PYGZsh{} Show all available 3d\PYGZus{}scalar}
\PYG{g+go}{*******************3d\PYGZus{}scalar Names List**********************}
\PYG{g+gp}{...}
\PYG{g+go}{*******************End 3d\PYGZus{}scalar Names List**********************}
\PYG{g+gp}{\PYGZgt{}\PYGZgt{}\PYGZgt{} }\PYG{n}{ex} \PYG{o}{=} \PYG{n}{a}\PYG{o}{.}\PYG{n}{get3d\PYGZus{}scalar}\PYG{p}{(}\PYG{p}{)} \PYG{c+c1}{\PYGZsh{} To  test an existing 3d\PYGZus{}scalar object}
\PYG{g+gp}{\PYGZgt{}\PYGZgt{}\PYGZgt{} }\PYG{n}{vcs}\PYG{o}{.}\PYG{n}{queries}\PYG{o}{.}\PYG{n}{is3d\PYGZus{}scalar}\PYG{p}{(}\PYG{n}{ex}\PYG{p}{)}
\PYG{g+go}{1}
\end{Verbatim}

\item[{Parameters}] \leavevmode
\textbf{\texttt{obj}} (\emph{\texttt{VCS Object}}) -- A VCS object

\item[{Returns}] \leavevmode
An integer indicating whether the object is a 3d\_scalar graphics method (1), or not (0).

\item[{Return type}] \leavevmode
\href{https://docs.python.org/2/library/functions.html\#int}{int}

\end{description}\end{quote}

\end{fulllineitems}

\index{is3d\_vector() (in module vcs.queries)}

\begin{fulllineitems}
\phantomsection\label{vcs/misc/queries:vcs.queries.is3d_vector}\pysiglinewithargsret{\sphinxcode{vcs.queries.}\sphinxbfcode{is3d\_vector}}{\emph{obj}}{}
Check to see if this object is a VCS 3d\_vector graphics method.
\begin{quote}\begin{description}
\item[{Example}] \leavevmode
\begin{Verbatim}[commandchars=\\\{\}]
\PYG{g+gp}{\PYGZgt{}\PYGZgt{}\PYGZgt{} }\PYG{n}{a}\PYG{o}{=}\PYG{n}{vcs}\PYG{o}{.}\PYG{n}{init}\PYG{p}{(}\PYG{p}{)} \PYG{c+c1}{\PYGZsh{} Make a VCS Canvas object to work with:}
\PYG{g+gp}{\PYGZgt{}\PYGZgt{}\PYGZgt{} }\PYG{n}{a}\PYG{o}{.}\PYG{n}{show}\PYG{p}{(}\PYG{l+s+s1}{\PYGZsq{}}\PYG{l+s+s1}{3d\PYGZus{}vector}\PYG{l+s+s1}{\PYGZsq{}}\PYG{p}{)} \PYG{c+c1}{\PYGZsh{} Show all available 3d\PYGZus{}vector}
\PYG{g+go}{*******************3d\PYGZus{}vector Names List**********************}
\PYG{g+gp}{...}
\PYG{g+go}{*******************End 3d\PYGZus{}vector Names List**********************}
\PYG{g+gp}{\PYGZgt{}\PYGZgt{}\PYGZgt{} }\PYG{n}{ex} \PYG{o}{=} \PYG{n}{a}\PYG{o}{.}\PYG{n}{get3d\PYGZus{}vector}\PYG{p}{(}\PYG{p}{)} \PYG{c+c1}{\PYGZsh{} To  test an existing 3d\PYGZus{}vector object}
\PYG{g+gp}{\PYGZgt{}\PYGZgt{}\PYGZgt{} }\PYG{n}{vcs}\PYG{o}{.}\PYG{n}{queries}\PYG{o}{.}\PYG{n}{is3d\PYGZus{}vector}\PYG{p}{(}\PYG{n}{ex}\PYG{p}{)}
\PYG{g+go}{1}
\end{Verbatim}

\item[{Parameters}] \leavevmode
\textbf{\texttt{obj}} (\emph{\texttt{VCS Object}}) -- A VCS object

\item[{Returns}] \leavevmode
An integer indicating whether the object is a 3d\_vector graphics method (1), or not (0).

\item[{Return type}] \leavevmode
\href{https://docs.python.org/2/library/functions.html\#int}{int}

\end{description}\end{quote}

\end{fulllineitems}

\index{isboxfill() (in module vcs.queries)}

\begin{fulllineitems}
\phantomsection\label{vcs/misc/queries:vcs.queries.isboxfill}\pysiglinewithargsret{\sphinxcode{vcs.queries.}\sphinxbfcode{isboxfill}}{\emph{obj}}{}
Check to see if this object is a VCS boxfill graphics method.
\begin{quote}\begin{description}
\item[{Example}] \leavevmode
\begin{Verbatim}[commandchars=\\\{\}]
\PYG{g+gp}{\PYGZgt{}\PYGZgt{}\PYGZgt{} }\PYG{n}{a}\PYG{o}{=}\PYG{n}{vcs}\PYG{o}{.}\PYG{n}{init}\PYG{p}{(}\PYG{p}{)} \PYG{c+c1}{\PYGZsh{} Make a VCS Canvas object to work with:}
\PYG{g+gp}{\PYGZgt{}\PYGZgt{}\PYGZgt{} }\PYG{n}{a}\PYG{o}{.}\PYG{n}{show}\PYG{p}{(}\PYG{l+s+s1}{\PYGZsq{}}\PYG{l+s+s1}{boxfill}\PYG{l+s+s1}{\PYGZsq{}}\PYG{p}{)} \PYG{c+c1}{\PYGZsh{} Show all available boxfill}
\PYG{g+go}{*******************Boxfill Names List**********************}
\PYG{g+gp}{...}
\PYG{g+go}{*******************End Boxfill Names List**********************}
\PYG{g+gp}{\PYGZgt{}\PYGZgt{}\PYGZgt{} }\PYG{n}{ex} \PYG{o}{=} \PYG{n}{a}\PYG{o}{.}\PYG{n}{getboxfill}\PYG{p}{(}\PYG{p}{)} \PYG{c+c1}{\PYGZsh{} To  test an existing boxfill object}
\PYG{g+gp}{\PYGZgt{}\PYGZgt{}\PYGZgt{} }\PYG{n}{vcs}\PYG{o}{.}\PYG{n}{queries}\PYG{o}{.}\PYG{n}{isboxfill}\PYG{p}{(}\PYG{n}{ex}\PYG{p}{)}
\PYG{g+go}{1}
\end{Verbatim}

\item[{Parameters}] \leavevmode
\textbf{\texttt{obj}} (\emph{\texttt{VCS Object}}) -- A VCS object

\item[{Returns}] \leavevmode
An integer indicating whether the object is a boxfill graphics method (1), or not (0).

\item[{Return type}] \leavevmode
\href{https://docs.python.org/2/library/functions.html\#int}{int}

\end{description}\end{quote}

\end{fulllineitems}

\index{iscolormap() (in module vcs.queries)}

\begin{fulllineitems}
\phantomsection\label{vcs/misc/queries:vcs.queries.iscolormap}\pysiglinewithargsret{\sphinxcode{vcs.queries.}\sphinxbfcode{iscolormap}}{\emph{obj}}{}
Check to see if this object is a VCS colormap secondary method.
\begin{quote}\begin{description}
\item[{Example}] \leavevmode
\begin{Verbatim}[commandchars=\\\{\}]
\PYG{g+gp}{\PYGZgt{}\PYGZgt{}\PYGZgt{} }\PYG{n}{a}\PYG{o}{=}\PYG{n}{vcs}\PYG{o}{.}\PYG{n}{init}\PYG{p}{(}\PYG{p}{)} \PYG{c+c1}{\PYGZsh{} Make a VCS Canvas object to work with:}
\PYG{g+gp}{\PYGZgt{}\PYGZgt{}\PYGZgt{} }\PYG{n}{a}\PYG{o}{.}\PYG{n}{show}\PYG{p}{(}\PYG{l+s+s1}{\PYGZsq{}}\PYG{l+s+s1}{colormap}\PYG{l+s+s1}{\PYGZsq{}}\PYG{p}{)} \PYG{c+c1}{\PYGZsh{} Show all available colormap}
\PYG{g+go}{*******************Colormap Names List**********************}
\PYG{g+gp}{...}
\PYG{g+go}{*******************End Colormap Names List**********************}
\PYG{g+gp}{\PYGZgt{}\PYGZgt{}\PYGZgt{} }\PYG{n}{ex} \PYG{o}{=} \PYG{n}{a}\PYG{o}{.}\PYG{n}{getcolormap}\PYG{p}{(}\PYG{p}{)} \PYG{c+c1}{\PYGZsh{} To  test an existing colormap object}
\PYG{g+gp}{\PYGZgt{}\PYGZgt{}\PYGZgt{} }\PYG{n}{vcs}\PYG{o}{.}\PYG{n}{queries}\PYG{o}{.}\PYG{n}{iscolormap}\PYG{p}{(}\PYG{n}{ex}\PYG{p}{)}
\PYG{g+go}{1}
\end{Verbatim}

\item[{Parameters}] \leavevmode
\textbf{\texttt{obj}} (\emph{\texttt{VCS Object}}) -- A VCS object

\item[{Returns}] \leavevmode
An integer indicating whether the object is a colormap secondary method (1), or not (0).

\item[{Return type}] \leavevmode
\href{https://docs.python.org/2/library/functions.html\#int}{int}

\end{description}\end{quote}

\end{fulllineitems}

\index{isfillarea() (in module vcs.queries)}

\begin{fulllineitems}
\phantomsection\label{vcs/misc/queries:vcs.queries.isfillarea}\pysiglinewithargsret{\sphinxcode{vcs.queries.}\sphinxbfcode{isfillarea}}{\emph{obj}}{}
Check to see if this object is a VCS fillarea secondary method.
\begin{quote}\begin{description}
\item[{Example}] \leavevmode
\begin{Verbatim}[commandchars=\\\{\}]
\PYG{g+gp}{\PYGZgt{}\PYGZgt{}\PYGZgt{} }\PYG{n}{a}\PYG{o}{=}\PYG{n}{vcs}\PYG{o}{.}\PYG{n}{init}\PYG{p}{(}\PYG{p}{)} \PYG{c+c1}{\PYGZsh{} Make a VCS Canvas object to work with:}
\PYG{g+gp}{\PYGZgt{}\PYGZgt{}\PYGZgt{} }\PYG{n}{a}\PYG{o}{.}\PYG{n}{show}\PYG{p}{(}\PYG{l+s+s1}{\PYGZsq{}}\PYG{l+s+s1}{fillarea}\PYG{l+s+s1}{\PYGZsq{}}\PYG{p}{)} \PYG{c+c1}{\PYGZsh{} Show all available fillarea}
\PYG{g+go}{*******************Fillarea Names List**********************}
\PYG{g+gp}{...}
\PYG{g+go}{*******************End Fillarea Names List**********************}
\PYG{g+gp}{\PYGZgt{}\PYGZgt{}\PYGZgt{} }\PYG{n}{ex} \PYG{o}{=} \PYG{n}{a}\PYG{o}{.}\PYG{n}{getfillarea}\PYG{p}{(}\PYG{p}{)} \PYG{c+c1}{\PYGZsh{} To  test an existing fillarea object}
\PYG{g+gp}{\PYGZgt{}\PYGZgt{}\PYGZgt{} }\PYG{n}{vcs}\PYG{o}{.}\PYG{n}{queries}\PYG{o}{.}\PYG{n}{isfillarea}\PYG{p}{(}\PYG{n}{ex}\PYG{p}{)}
\PYG{g+go}{1}
\end{Verbatim}

\item[{Parameters}] \leavevmode
\textbf{\texttt{obj}} (\emph{\texttt{VCS Object}}) -- A VCS object

\item[{Returns}] \leavevmode
An integer indicating whether the object is a fillarea secondary method (1), or not (0).

\item[{Return type}] \leavevmode
\href{https://docs.python.org/2/library/functions.html\#int}{int}

\end{description}\end{quote}

\end{fulllineitems}

\index{isgraphicsmethod() (in module vcs.queries)}

\begin{fulllineitems}
\phantomsection\label{vcs/misc/queries:vcs.queries.isgraphicsmethod}\pysiglinewithargsret{\sphinxcode{vcs.queries.}\sphinxbfcode{isgraphicsmethod}}{\emph{gobj}}{}
Indicates if the entered argument is one of the following graphics
methods: boxfill, isofill, isoline,
scatter, vector, xvsy, xyvsy, yxvsx.
\begin{quote}\begin{description}
\item[{Example}] \leavevmode
\begin{Verbatim}[commandchars=\\\{\}]
\PYG{g+gp}{\PYGZgt{}\PYGZgt{}\PYGZgt{} }\PYG{n}{a}\PYG{o}{=}\PYG{n}{vcs}\PYG{o}{.}\PYG{n}{init}\PYG{p}{(}\PYG{p}{)}
\PYG{g+gp}{\PYGZgt{}\PYGZgt{}\PYGZgt{} }\PYG{n}{box}\PYG{o}{=}\PYG{n}{a}\PYG{o}{.}\PYG{n}{getboxfill}\PYG{p}{(}\PYG{p}{)} \PYG{c+c1}{\PYGZsh{} get default boxfill object}
\PYG{g+gp}{\PYGZgt{}\PYGZgt{}\PYGZgt{} }\PYG{n}{vcs}\PYG{o}{.}\PYG{n}{isgraphicsmethod}\PYG{p}{(}\PYG{n}{box}\PYG{p}{)}
\PYG{g+go}{1}
\end{Verbatim}

\item[{Parameters}] \leavevmode
\textbf{\texttt{gobj}} (\emph{\texttt{A VCS graphics object}}) -- A graphics object

\item[{Returns}] \leavevmode
Integer reperesenting whether gobj is one of the above graphics methods.
1 indicates true, 0 indicates false.

\item[{Return type}] \leavevmode
\href{https://docs.python.org/2/library/functions.html\#int}{int}

\end{description}\end{quote}

\end{fulllineitems}

\index{isisofill() (in module vcs.queries)}

\begin{fulllineitems}
\phantomsection\label{vcs/misc/queries:vcs.queries.isisofill}\pysiglinewithargsret{\sphinxcode{vcs.queries.}\sphinxbfcode{isisofill}}{\emph{obj}}{}
Check to see if this object is a VCS isofill graphics method.
\begin{quote}\begin{description}
\item[{Example}] \leavevmode
\begin{Verbatim}[commandchars=\\\{\}]
\PYG{g+gp}{\PYGZgt{}\PYGZgt{}\PYGZgt{} }\PYG{n}{a}\PYG{o}{=}\PYG{n}{vcs}\PYG{o}{.}\PYG{n}{init}\PYG{p}{(}\PYG{p}{)} \PYG{c+c1}{\PYGZsh{} Make a VCS Canvas object to work with:}
\PYG{g+gp}{\PYGZgt{}\PYGZgt{}\PYGZgt{} }\PYG{n}{a}\PYG{o}{.}\PYG{n}{show}\PYG{p}{(}\PYG{l+s+s1}{\PYGZsq{}}\PYG{l+s+s1}{isofill}\PYG{l+s+s1}{\PYGZsq{}}\PYG{p}{)} \PYG{c+c1}{\PYGZsh{} Show all available isofill}
\PYG{g+go}{*******************Isofill Names List**********************}
\PYG{g+gp}{...}
\PYG{g+go}{*******************End Isofill Names List**********************}
\PYG{g+gp}{\PYGZgt{}\PYGZgt{}\PYGZgt{} }\PYG{n}{ex} \PYG{o}{=} \PYG{n}{a}\PYG{o}{.}\PYG{n}{getisofill}\PYG{p}{(}\PYG{p}{)} \PYG{c+c1}{\PYGZsh{} To  test an existing isofill object}
\PYG{g+gp}{\PYGZgt{}\PYGZgt{}\PYGZgt{} }\PYG{n}{vcs}\PYG{o}{.}\PYG{n}{queries}\PYG{o}{.}\PYG{n}{isisofill}\PYG{p}{(}\PYG{n}{ex}\PYG{p}{)}
\PYG{g+go}{1}
\end{Verbatim}

\item[{Parameters}] \leavevmode
\textbf{\texttt{obj}} (\emph{\texttt{VCS Object}}) -- A VCS object

\item[{Returns}] \leavevmode
An integer indicating whether the object is a isofill graphics method (1), or not (0).

\item[{Return type}] \leavevmode
\href{https://docs.python.org/2/library/functions.html\#int}{int}

\end{description}\end{quote}

\end{fulllineitems}

\index{isisoline() (in module vcs.queries)}

\begin{fulllineitems}
\phantomsection\label{vcs/misc/queries:vcs.queries.isisoline}\pysiglinewithargsret{\sphinxcode{vcs.queries.}\sphinxbfcode{isisoline}}{\emph{obj}}{}
Check to see if this object is a VCS isoline graphics method.
\begin{quote}\begin{description}
\item[{Example}] \leavevmode
\begin{Verbatim}[commandchars=\\\{\}]
\PYG{g+gp}{\PYGZgt{}\PYGZgt{}\PYGZgt{} }\PYG{n}{a}\PYG{o}{=}\PYG{n}{vcs}\PYG{o}{.}\PYG{n}{init}\PYG{p}{(}\PYG{p}{)} \PYG{c+c1}{\PYGZsh{} Make a VCS Canvas object to work with:}
\PYG{g+gp}{\PYGZgt{}\PYGZgt{}\PYGZgt{} }\PYG{n}{a}\PYG{o}{.}\PYG{n}{show}\PYG{p}{(}\PYG{l+s+s1}{\PYGZsq{}}\PYG{l+s+s1}{isoline}\PYG{l+s+s1}{\PYGZsq{}}\PYG{p}{)} \PYG{c+c1}{\PYGZsh{} Show all available isoline}
\PYG{g+go}{*******************Isoline Names List**********************}
\PYG{g+gp}{...}
\PYG{g+go}{*******************End Isoline Names List**********************}
\PYG{g+gp}{\PYGZgt{}\PYGZgt{}\PYGZgt{} }\PYG{n}{ex} \PYG{o}{=} \PYG{n}{a}\PYG{o}{.}\PYG{n}{getisoline}\PYG{p}{(}\PYG{p}{)} \PYG{c+c1}{\PYGZsh{} To  test an existing isoline object}
\PYG{g+gp}{\PYGZgt{}\PYGZgt{}\PYGZgt{} }\PYG{n}{vcs}\PYG{o}{.}\PYG{n}{queries}\PYG{o}{.}\PYG{n}{isisoline}\PYG{p}{(}\PYG{n}{ex}\PYG{p}{)}
\PYG{g+go}{1}
\end{Verbatim}

\item[{Parameters}] \leavevmode
\textbf{\texttt{obj}} (\emph{\texttt{VCS Object}}) -- A VCS object

\item[{Returns}] \leavevmode
An integer indicating whether the object is a isoline graphics method (1), or not (0).

\item[{Return type}] \leavevmode
\href{https://docs.python.org/2/library/functions.html\#int}{int}

\end{description}\end{quote}

\end{fulllineitems}

\index{isline() (in module vcs.queries)}

\begin{fulllineitems}
\phantomsection\label{vcs/misc/queries:vcs.queries.isline}\pysiglinewithargsret{\sphinxcode{vcs.queries.}\sphinxbfcode{isline}}{\emph{obj}}{}
Check to see if this object is a VCS line secondary method.
\begin{quote}\begin{description}
\item[{Example}] \leavevmode
\begin{Verbatim}[commandchars=\\\{\}]
\PYG{g+gp}{\PYGZgt{}\PYGZgt{}\PYGZgt{} }\PYG{n}{a}\PYG{o}{=}\PYG{n}{vcs}\PYG{o}{.}\PYG{n}{init}\PYG{p}{(}\PYG{p}{)} \PYG{c+c1}{\PYGZsh{} Make a VCS Canvas object to work with:}
\PYG{g+gp}{\PYGZgt{}\PYGZgt{}\PYGZgt{} }\PYG{n}{a}\PYG{o}{.}\PYG{n}{show}\PYG{p}{(}\PYG{l+s+s1}{\PYGZsq{}}\PYG{l+s+s1}{line}\PYG{l+s+s1}{\PYGZsq{}}\PYG{p}{)} \PYG{c+c1}{\PYGZsh{} Show all available line}
\PYG{g+go}{*******************Line Names List**********************}
\PYG{g+gp}{...}
\PYG{g+go}{*******************End Line Names List**********************}
\PYG{g+gp}{\PYGZgt{}\PYGZgt{}\PYGZgt{} }\PYG{n}{ex} \PYG{o}{=} \PYG{n}{a}\PYG{o}{.}\PYG{n}{getline}\PYG{p}{(}\PYG{p}{)} \PYG{c+c1}{\PYGZsh{} To  test an existing line object}
\PYG{g+gp}{\PYGZgt{}\PYGZgt{}\PYGZgt{} }\PYG{n}{vcs}\PYG{o}{.}\PYG{n}{queries}\PYG{o}{.}\PYG{n}{isline}\PYG{p}{(}\PYG{n}{ex}\PYG{p}{)}
\PYG{g+go}{1}
\end{Verbatim}

\item[{Parameters}] \leavevmode
\textbf{\texttt{obj}} (\emph{\texttt{VCS Object}}) -- A VCS object

\item[{Returns}] \leavevmode
An integer indicating whether the object is a line secondary method (1), or not (0).

\item[{Return type}] \leavevmode
\href{https://docs.python.org/2/library/functions.html\#int}{int}

\end{description}\end{quote}

\end{fulllineitems}

\index{ismarker() (in module vcs.queries)}

\begin{fulllineitems}
\phantomsection\label{vcs/misc/queries:vcs.queries.ismarker}\pysiglinewithargsret{\sphinxcode{vcs.queries.}\sphinxbfcode{ismarker}}{\emph{obj}}{}
Check to see if this object is a VCS marker secondary method.
\begin{quote}\begin{description}
\item[{Example}] \leavevmode
\begin{Verbatim}[commandchars=\\\{\}]
\PYG{g+gp}{\PYGZgt{}\PYGZgt{}\PYGZgt{} }\PYG{n}{a}\PYG{o}{=}\PYG{n}{vcs}\PYG{o}{.}\PYG{n}{init}\PYG{p}{(}\PYG{p}{)} \PYG{c+c1}{\PYGZsh{} Make a VCS Canvas object to work with:}
\PYG{g+gp}{\PYGZgt{}\PYGZgt{}\PYGZgt{} }\PYG{n}{a}\PYG{o}{.}\PYG{n}{show}\PYG{p}{(}\PYG{l+s+s1}{\PYGZsq{}}\PYG{l+s+s1}{marker}\PYG{l+s+s1}{\PYGZsq{}}\PYG{p}{)} \PYG{c+c1}{\PYGZsh{} Show all available marker}
\PYG{g+go}{*******************Marker Names List**********************}
\PYG{g+gp}{...}
\PYG{g+go}{*******************End Marker Names List**********************}
\PYG{g+gp}{\PYGZgt{}\PYGZgt{}\PYGZgt{} }\PYG{n}{ex} \PYG{o}{=} \PYG{n}{a}\PYG{o}{.}\PYG{n}{getmarker}\PYG{p}{(}\PYG{p}{)} \PYG{c+c1}{\PYGZsh{} To  test an existing marker object}
\PYG{g+gp}{\PYGZgt{}\PYGZgt{}\PYGZgt{} }\PYG{n}{vcs}\PYG{o}{.}\PYG{n}{queries}\PYG{o}{.}\PYG{n}{ismarker}\PYG{p}{(}\PYG{n}{ex}\PYG{p}{)}
\PYG{g+go}{1}
\end{Verbatim}

\item[{Parameters}] \leavevmode
\textbf{\texttt{obj}} (\emph{\texttt{VCS Object}}) -- A VCS object

\item[{Returns}] \leavevmode
An integer indicating whether the object is a marker secondary method (1), or not (0).

\item[{Return type}] \leavevmode
\href{https://docs.python.org/2/library/functions.html\#int}{int}

\end{description}\end{quote}

\end{fulllineitems}

\index{ismeshfill() (in module vcs.queries)}

\begin{fulllineitems}
\phantomsection\label{vcs/misc/queries:vcs.queries.ismeshfill}\pysiglinewithargsret{\sphinxcode{vcs.queries.}\sphinxbfcode{ismeshfill}}{\emph{obj}}{}
Check to see if this object is a VCS meshfill graphics method.
\begin{quote}\begin{description}
\item[{Example}] \leavevmode
\begin{Verbatim}[commandchars=\\\{\}]
\PYG{g+gp}{\PYGZgt{}\PYGZgt{}\PYGZgt{} }\PYG{n}{a}\PYG{o}{=}\PYG{n}{vcs}\PYG{o}{.}\PYG{n}{init}\PYG{p}{(}\PYG{p}{)} \PYG{c+c1}{\PYGZsh{} Make a VCS Canvas object to work with:}
\PYG{g+gp}{\PYGZgt{}\PYGZgt{}\PYGZgt{} }\PYG{n}{a}\PYG{o}{.}\PYG{n}{show}\PYG{p}{(}\PYG{l+s+s1}{\PYGZsq{}}\PYG{l+s+s1}{meshfill}\PYG{l+s+s1}{\PYGZsq{}}\PYG{p}{)} \PYG{c+c1}{\PYGZsh{} Show all available meshfill}
\PYG{g+go}{*******************Meshfill Names List**********************}
\PYG{g+gp}{...}
\PYG{g+go}{*******************End Meshfill Names List**********************}
\PYG{g+gp}{\PYGZgt{}\PYGZgt{}\PYGZgt{} }\PYG{n}{ex} \PYG{o}{=} \PYG{n}{a}\PYG{o}{.}\PYG{n}{getmeshfill}\PYG{p}{(}\PYG{p}{)} \PYG{c+c1}{\PYGZsh{} To  test an existing meshfill object}
\PYG{g+gp}{\PYGZgt{}\PYGZgt{}\PYGZgt{} }\PYG{n}{vcs}\PYG{o}{.}\PYG{n}{queries}\PYG{o}{.}\PYG{n}{ismeshfill}\PYG{p}{(}\PYG{n}{ex}\PYG{p}{)}
\PYG{g+go}{1}
\end{Verbatim}

\item[{Parameters}] \leavevmode
\textbf{\texttt{obj}} (\emph{\texttt{VCS Object}}) -- A VCS object

\item[{Returns}] \leavevmode
An integer indicating whether the object is a meshfill graphics method (1), or not (0).

\item[{Return type}] \leavevmode
\href{https://docs.python.org/2/library/functions.html\#int}{int}

\end{description}\end{quote}

\end{fulllineitems}

\index{isplot() (in module vcs.queries)}

\begin{fulllineitems}
\phantomsection\label{vcs/misc/queries:vcs.queries.isplot}\pysiglinewithargsret{\sphinxcode{vcs.queries.}\sphinxbfcode{isplot}}{\emph{pobj}}{}
Check to see if this object is a VCS secondary display plot.
\begin{quote}\begin{description}
\item[{Example}] \leavevmode
\begin{Verbatim}[commandchars=\\\{\}]
\PYG{g+gp}{\PYGZgt{}\PYGZgt{}\PYGZgt{} }\PYG{n}{a}\PYG{o}{=}\PYG{n}{vcs}\PYG{o}{.}\PYG{n}{init}\PYG{p}{(}\PYG{p}{)}
\PYG{g+gp}{\PYGZgt{}\PYGZgt{}\PYGZgt{} }\PYG{k+kn}{import} \PYG{n+nn}{cdms2} \PYG{c+c1}{\PYGZsh{} need this to make a slab for a boxfill plot}
\PYG{g+gp}{\PYGZgt{}\PYGZgt{}\PYGZgt{} }\PYG{n}{f} \PYG{o}{=} \PYG{n}{cdms2}\PYG{o}{.}\PYG{n}{open}\PYG{p}{(}\PYG{n}{vcs}\PYG{o}{.}\PYG{n}{sample\PYGZus{}data} \PYG{o}{+} \PYG{l+s+s1}{\PYGZsq{}}\PYG{l+s+s1}{/clt.nc}\PYG{l+s+s1}{\PYGZsq{}}\PYG{p}{)} \PYG{c+c1}{\PYGZsh{} open a variable file}
\PYG{g+gp}{\PYGZgt{}\PYGZgt{}\PYGZgt{} }\PYG{n}{v} \PYG{o}{=} \PYG{n}{f}\PYG{p}{(}\PYG{l+s+s1}{\PYGZsq{}}\PYG{l+s+s1}{v}\PYG{l+s+s1}{\PYGZsq{}}\PYG{p}{)} \PYG{c+c1}{\PYGZsh{} create a slab from the variable file}
\PYG{g+gp}{\PYGZgt{}\PYGZgt{}\PYGZgt{} }\PYG{n}{dsp\PYGZus{}plot}\PYG{o}{=}\PYG{p}{(}\PYG{n}{a}\PYG{o}{.}\PYG{n}{getboxfill}\PYG{p}{(}\PYG{p}{)}\PYG{p}{,} \PYG{n}{v}\PYG{p}{)} \PYG{c+c1}{\PYGZsh{} plot a boxfill. Should return vcs.displayplot.Dp.}
\PYG{g+gp}{\PYGZgt{}\PYGZgt{}\PYGZgt{} }\PYG{n}{vcs}\PYG{o}{.}\PYG{n}{queries}\PYG{o}{.}\PYG{n}{isplot}\PYG{p}{(}\PYG{n}{dsp\PYGZus{}plot}\PYG{p}{)}
\PYG{g+go}{1}
\end{Verbatim}

\item[{Parameters}] \leavevmode
\textbf{\texttt{obj}} (\emph{\texttt{VCS Object}}) -- A VCS object

\item[{Returns}] \leavevmode
An integer indicating whether the object is a display plot (1), or not (0).

\item[{Return type}] \leavevmode
\href{https://docs.python.org/2/library/functions.html\#int}{int}

\end{description}\end{quote}

\end{fulllineitems}

\index{isprojection() (in module vcs.queries)}

\begin{fulllineitems}
\phantomsection\label{vcs/misc/queries:vcs.queries.isprojection}\pysiglinewithargsret{\sphinxcode{vcs.queries.}\sphinxbfcode{isprojection}}{\emph{obj}}{}
Check to see if this object is a VCS projection graphics method.
\begin{quote}\begin{description}
\item[{Example}] \leavevmode
\begin{Verbatim}[commandchars=\\\{\}]
\PYG{g+gp}{\PYGZgt{}\PYGZgt{}\PYGZgt{} }\PYG{n}{a}\PYG{o}{=}\PYG{n}{vcs}\PYG{o}{.}\PYG{n}{init}\PYG{p}{(}\PYG{p}{)} \PYG{c+c1}{\PYGZsh{} Make a VCS Canvas object to work with:}
\PYG{g+gp}{\PYGZgt{}\PYGZgt{}\PYGZgt{} }\PYG{n}{a}\PYG{o}{.}\PYG{n}{show}\PYG{p}{(}\PYG{l+s+s1}{\PYGZsq{}}\PYG{l+s+s1}{projection}\PYG{l+s+s1}{\PYGZsq{}}\PYG{p}{)} \PYG{c+c1}{\PYGZsh{} Show all available projection}
\PYG{g+go}{*******************Projection Names List**********************}
\PYG{g+gp}{...}
\PYG{g+go}{*******************End Projection Names List**********************}
\PYG{g+gp}{\PYGZgt{}\PYGZgt{}\PYGZgt{} }\PYG{n}{ex} \PYG{o}{=} \PYG{n}{a}\PYG{o}{.}\PYG{n}{getprojection}\PYG{p}{(}\PYG{p}{)} \PYG{c+c1}{\PYGZsh{} To  test an existing projection object}
\PYG{g+gp}{\PYGZgt{}\PYGZgt{}\PYGZgt{} }\PYG{n}{vcs}\PYG{o}{.}\PYG{n}{queries}\PYG{o}{.}\PYG{n}{isprojection}\PYG{p}{(}\PYG{n}{ex}\PYG{p}{)}
\PYG{g+go}{1}
\end{Verbatim}

\item[{Parameters}] \leavevmode
\textbf{\texttt{obj}} (\emph{\texttt{VCS Object}}) -- A VCS object

\item[{Returns}] \leavevmode
An integer indicating whether the object is a projection graphics method (1), or not (0).

\item[{Return type}] \leavevmode
\href{https://docs.python.org/2/library/functions.html\#int}{int}

\end{description}\end{quote}

\end{fulllineitems}

\index{isscatter() (in module vcs.queries)}

\begin{fulllineitems}
\phantomsection\label{vcs/misc/queries:vcs.queries.isscatter}\pysiglinewithargsret{\sphinxcode{vcs.queries.}\sphinxbfcode{isscatter}}{\emph{obj}}{}
Check to see if this object is a VCS scatter graphics method.
\begin{quote}\begin{description}
\item[{Example}] \leavevmode
\begin{Verbatim}[commandchars=\\\{\}]
\PYG{g+gp}{\PYGZgt{}\PYGZgt{}\PYGZgt{} }\PYG{n}{a}\PYG{o}{=}\PYG{n}{vcs}\PYG{o}{.}\PYG{n}{init}\PYG{p}{(}\PYG{p}{)} \PYG{c+c1}{\PYGZsh{} Make a VCS Canvas object to work with:}
\PYG{g+gp}{\PYGZgt{}\PYGZgt{}\PYGZgt{} }\PYG{n}{a}\PYG{o}{.}\PYG{n}{show}\PYG{p}{(}\PYG{l+s+s1}{\PYGZsq{}}\PYG{l+s+s1}{scatter}\PYG{l+s+s1}{\PYGZsq{}}\PYG{p}{)} \PYG{c+c1}{\PYGZsh{} Show all available scatter}
\PYG{g+go}{*******************Scatter Names List**********************}
\PYG{g+gp}{...}
\PYG{g+go}{*******************End Scatter Names List**********************}
\PYG{g+gp}{\PYGZgt{}\PYGZgt{}\PYGZgt{} }\PYG{n}{ex} \PYG{o}{=} \PYG{n}{a}\PYG{o}{.}\PYG{n}{getscatter}\PYG{p}{(}\PYG{l+s+s1}{\PYGZsq{}}\PYG{l+s+s1}{default\PYGZus{}scatter\PYGZus{}}\PYG{l+s+s1}{\PYGZsq{}}\PYG{p}{)} \PYG{c+c1}{\PYGZsh{} To  test an existing scatter object}
\PYG{g+gp}{\PYGZgt{}\PYGZgt{}\PYGZgt{} }\PYG{n}{vcs}\PYG{o}{.}\PYG{n}{queries}\PYG{o}{.}\PYG{n}{isscatter}\PYG{p}{(}\PYG{n}{ex}\PYG{p}{)}
\PYG{g+go}{1}
\end{Verbatim}

\item[{Parameters}] \leavevmode
\textbf{\texttt{obj}} (\emph{\texttt{VCS Object}}) -- A VCS object

\item[{Returns}] \leavevmode
An integer indicating whether the object is a scatter graphics method (1), or not (0).

\item[{Return type}] \leavevmode
\href{https://docs.python.org/2/library/functions.html\#int}{int}

\end{description}\end{quote}

\end{fulllineitems}

\index{issecondaryobject() (in module vcs.queries)}

\begin{fulllineitems}
\phantomsection\label{vcs/misc/queries:vcs.queries.issecondaryobject}\pysiglinewithargsret{\sphinxcode{vcs.queries.}\sphinxbfcode{issecondaryobject}}{\emph{sobj}}{}
Check to see if this object is a VCS secondary object
\begin{quote}

\begin{notice}{note}{Note:}
Secondary objects will be one of the following:
1.) colormap: specification of combinations of 256 available
\begin{quote}

colors
\end{quote}

2.) fill area: style, style index, and color index
3.) format: specifications for converting numbers to display
\begin{quote}

strings
\end{quote}

4.) line: line type, width, and color index
5.) list: a sequence of pairs of numerical and character values
6.) marker: marker type, size, and color index
7.) text table: text font type, character spacing, expansion, and
\begin{quote}

color index
\end{quote}
\begin{description}
\item[{8.) text orientation: character height, angle, path, and}] \leavevmode
horizontal/vertical alignment

\end{description}

9.) projections
\end{notice}
\end{quote}
\begin{quote}\begin{description}
\item[{Example}] \leavevmode
\begin{Verbatim}[commandchars=\\\{\}]
\PYG{g+gp}{\PYGZgt{}\PYGZgt{}\PYGZgt{} }\PYG{n}{a}\PYG{o}{=}\PYG{n}{vcs}\PYG{o}{.}\PYG{n}{init}\PYG{p}{(}\PYG{p}{)}
\PYG{g+gp}{\PYGZgt{}\PYGZgt{}\PYGZgt{} }\PYG{n}{a}\PYG{o}{.}\PYG{n}{show}\PYG{p}{(}\PYG{l+s+s1}{\PYGZsq{}}\PYG{l+s+s1}{line}\PYG{l+s+s1}{\PYGZsq{}}\PYG{p}{)} \PYG{c+c1}{\PYGZsh{} Show all available lines}
\PYG{g+go}{*******************Line Names List**********************}
\PYG{g+gp}{...}
\PYG{g+go}{*******************End Line Names List**********************}
\PYG{g+gp}{\PYGZgt{}\PYGZgt{}\PYGZgt{} }\PYG{n}{ex} \PYG{o}{=} \PYG{n}{a}\PYG{o}{.}\PYG{n}{getprojection}\PYG{p}{(}\PYG{l+s+s1}{\PYGZsq{}}\PYG{l+s+s1}{default}\PYG{l+s+s1}{\PYGZsq{}}\PYG{p}{)} \PYG{c+c1}{\PYGZsh{} To test an existing line object}
\PYG{g+gp}{\PYGZgt{}\PYGZgt{}\PYGZgt{} }\PYG{n}{vcs}\PYG{o}{.}\PYG{n}{issecondaryobject}\PYG{p}{(}\PYG{n}{ex}\PYG{p}{)}
\PYG{g+go}{1}
\end{Verbatim}

\item[{Parameters}] \leavevmode
\textbf{\texttt{obj}} (\emph{\texttt{VCS Object}}) -- A VCS object

\item[{Returns}] \leavevmode
An integer indicating whether the object is a projection graphics object (1), or not (0).

\item[{Return type}] \leavevmode
\href{https://docs.python.org/2/library/functions.html\#int}{int}

\end{description}\end{quote}

\end{fulllineitems}

\index{istaylordiagram() (in module vcs.queries)}

\begin{fulllineitems}
\phantomsection\label{vcs/misc/queries:vcs.queries.istaylordiagram}\pysiglinewithargsret{\sphinxcode{vcs.queries.}\sphinxbfcode{istaylordiagram}}{\emph{obj}}{}
Check to see if this object is a VCS taylordiagram graphics method.
\begin{quote}\begin{description}
\item[{Example}] \leavevmode
\begin{Verbatim}[commandchars=\\\{\}]
\PYG{g+gp}{\PYGZgt{}\PYGZgt{}\PYGZgt{} }\PYG{n}{a}\PYG{o}{=}\PYG{n}{vcs}\PYG{o}{.}\PYG{n}{init}\PYG{p}{(}\PYG{p}{)} \PYG{c+c1}{\PYGZsh{} Make a VCS Canvas object to work with:}
\PYG{g+gp}{\PYGZgt{}\PYGZgt{}\PYGZgt{} }\PYG{n}{a}\PYG{o}{.}\PYG{n}{show}\PYG{p}{(}\PYG{l+s+s1}{\PYGZsq{}}\PYG{l+s+s1}{taylordiagram}\PYG{l+s+s1}{\PYGZsq{}}\PYG{p}{)} \PYG{c+c1}{\PYGZsh{} Show all available taylordiagram}
\PYG{g+go}{*******************Taylordiagram Names List**********************}
\PYG{g+gp}{...}
\PYG{g+go}{*******************End Taylordiagram Names List**********************}
\PYG{g+gp}{\PYGZgt{}\PYGZgt{}\PYGZgt{} }\PYG{n}{ex} \PYG{o}{=} \PYG{n}{a}\PYG{o}{.}\PYG{n}{gettaylordiagram}\PYG{p}{(}\PYG{p}{)} \PYG{c+c1}{\PYGZsh{} To  test an existing taylordiagram object}
\PYG{g+gp}{\PYGZgt{}\PYGZgt{}\PYGZgt{} }\PYG{n}{vcs}\PYG{o}{.}\PYG{n}{queries}\PYG{o}{.}\PYG{n}{istaylordiagram}\PYG{p}{(}\PYG{n}{ex}\PYG{p}{)}
\PYG{g+go}{1}
\end{Verbatim}

\item[{Parameters}] \leavevmode
\textbf{\texttt{obj}} (\emph{\texttt{VCS Object}}) -- A VCS object

\item[{Returns}] \leavevmode
An integer indicating whether the object is a taylordiagram graphics method (1), or not (0).

\item[{Return type}] \leavevmode
\href{https://docs.python.org/2/library/functions.html\#int}{int}

\end{description}\end{quote}

\end{fulllineitems}

\index{istemplate() (in module vcs.queries)}

\begin{fulllineitems}
\phantomsection\label{vcs/misc/queries:vcs.queries.istemplate}\pysiglinewithargsret{\sphinxcode{vcs.queries.}\sphinxbfcode{istemplate}}{\emph{gobj}}{}
Check to see if this object is a VCS template graphics method.
\begin{quote}\begin{description}
\item[{Example}] \leavevmode
\begin{Verbatim}[commandchars=\\\{\}]
\PYG{g+gp}{\PYGZgt{}\PYGZgt{}\PYGZgt{} }\PYG{n}{a}\PYG{o}{=}\PYG{n}{vcs}\PYG{o}{.}\PYG{n}{init}\PYG{p}{(}\PYG{p}{)} \PYG{c+c1}{\PYGZsh{} Make a VCS Canvas object to work with:}
\PYG{g+gp}{\PYGZgt{}\PYGZgt{}\PYGZgt{} }\PYG{n}{a}\PYG{o}{.}\PYG{n}{show}\PYG{p}{(}\PYG{l+s+s1}{\PYGZsq{}}\PYG{l+s+s1}{template}\PYG{l+s+s1}{\PYGZsq{}}\PYG{p}{)} \PYG{c+c1}{\PYGZsh{} Show all available template}
\PYG{g+go}{*******************Template Names List**********************}
\PYG{g+gp}{...}
\PYG{g+go}{*******************End Template Names List**********************}
\PYG{g+gp}{\PYGZgt{}\PYGZgt{}\PYGZgt{} }\PYG{n}{ex} \PYG{o}{=} \PYG{n}{a}\PYG{o}{.}\PYG{n}{gettemplate}\PYG{p}{(}\PYG{p}{)} \PYG{c+c1}{\PYGZsh{} To  test an existing template object}
\PYG{g+gp}{\PYGZgt{}\PYGZgt{}\PYGZgt{} }\PYG{n}{vcs}\PYG{o}{.}\PYG{n}{queries}\PYG{o}{.}\PYG{n}{istemplate}\PYG{p}{(}\PYG{n}{ex}\PYG{p}{)}
\PYG{g+go}{1}
\end{Verbatim}

\item[{Parameters}] \leavevmode
\textbf{\texttt{obj}} (\emph{\texttt{VCS Object}}) -- A VCS object

\item[{Returns}] \leavevmode
An integer indicating whether the object is a template graphics method (1), or not (0).

\item[{Return type}] \leavevmode
\href{https://docs.python.org/2/library/functions.html\#int}{int}

\end{description}\end{quote}

\end{fulllineitems}

\index{istext() (in module vcs.queries)}

\begin{fulllineitems}
\phantomsection\label{vcs/misc/queries:vcs.queries.istext}\pysiglinewithargsret{\sphinxcode{vcs.queries.}\sphinxbfcode{istext}}{\emph{obj}}{}
Check to see if this object is a VCS textcombined secondary method.
\begin{quote}\begin{description}
\item[{Example}] \leavevmode
\begin{Verbatim}[commandchars=\\\{\}]
\PYG{g+gp}{\PYGZgt{}\PYGZgt{}\PYGZgt{} }\PYG{n}{a}\PYG{o}{=}\PYG{n}{vcs}\PYG{o}{.}\PYG{n}{init}\PYG{p}{(}\PYG{p}{)} \PYG{c+c1}{\PYGZsh{} Make a VCS Canvas object to work with:}
\PYG{g+gp}{\PYGZgt{}\PYGZgt{}\PYGZgt{} }\PYG{n}{a}\PYG{o}{.}\PYG{n}{createtextcombined}\PYG{p}{(}\PYG{l+s+s1}{\PYGZsq{}}\PYG{l+s+s1}{EXAMPLE\PYGZus{}tt}\PYG{l+s+s1}{\PYGZsq{}}\PYG{p}{,} \PYG{l+s+s1}{\PYGZsq{}}\PYG{l+s+s1}{qa}\PYG{l+s+s1}{\PYGZsq{}}\PYG{p}{,} \PYG{l+s+s1}{\PYGZsq{}}\PYG{l+s+s1}{EXAMPLE\PYGZus{}tto}\PYG{l+s+s1}{\PYGZsq{}}\PYG{p}{,} \PYG{l+s+s1}{\PYGZsq{}}\PYG{l+s+s1}{7left}\PYG{l+s+s1}{\PYGZsq{}}\PYG{p}{)} \PYG{c+c1}{\PYGZsh{} Create \PYGZsq{}EXAMPLE\PYGZus{}tt\PYGZsq{} and \PYGZsq{}EXAMPLE\PYGZus{}tto\PYGZsq{}}
\PYG{g+go}{\PYGZlt{}vcs.textcombined.Tc ...\PYGZgt{}}
\PYG{g+gp}{\PYGZgt{}\PYGZgt{}\PYGZgt{} }\PYG{n}{a}\PYG{o}{.}\PYG{n}{show}\PYG{p}{(}\PYG{l+s+s1}{\PYGZsq{}}\PYG{l+s+s1}{textcombined}\PYG{l+s+s1}{\PYGZsq{}}\PYG{p}{)} \PYG{c+c1}{\PYGZsh{} Show all available textcombined}
\PYG{g+go}{*******************Textcombined Names List**********************}
\PYG{g+gp}{...}
\PYG{g+go}{*******************End Textcombined Names List**********************}
\PYG{g+gp}{\PYGZgt{}\PYGZgt{}\PYGZgt{} }\PYG{n}{ex} \PYG{o}{=} \PYG{n}{a}\PYG{o}{.}\PYG{n}{gettextcombined}\PYG{p}{(}\PYG{l+s+s1}{\PYGZsq{}}\PYG{l+s+s1}{EXAMPLE\PYGZus{}tt}\PYG{l+s+s1}{\PYGZsq{}}\PYG{p}{,} \PYG{l+s+s1}{\PYGZsq{}}\PYG{l+s+s1}{EXAMPLE\PYGZus{}tto}\PYG{l+s+s1}{\PYGZsq{}}\PYG{p}{)} \PYG{c+c1}{\PYGZsh{} To  test an existing textcombined object}
\PYG{g+gp}{\PYGZgt{}\PYGZgt{}\PYGZgt{} }\PYG{n}{vcs}\PYG{o}{.}\PYG{n}{queries}\PYG{o}{.}\PYG{n}{istextcombined}\PYG{p}{(}\PYG{n}{ex}\PYG{p}{)}
\PYG{g+go}{1}
\end{Verbatim}

\item[{Parameters}] \leavevmode
\textbf{\texttt{obj}} (\emph{\texttt{VCS Object}}) -- A VCS object

\item[{Returns}] \leavevmode
An integer indicating whether the object is a textcombined secondary method (1), or not (0).

\item[{Return type}] \leavevmode
\href{https://docs.python.org/2/library/functions.html\#int}{int}

\end{description}\end{quote}

\end{fulllineitems}

\index{istextcombined() (in module vcs.queries)}

\begin{fulllineitems}
\phantomsection\label{vcs/misc/queries:vcs.queries.istextcombined}\pysiglinewithargsret{\sphinxcode{vcs.queries.}\sphinxbfcode{istextcombined}}{\emph{obj}}{}
Check to see if this object is a VCS textcombined secondary method.
\begin{quote}\begin{description}
\item[{Example}] \leavevmode
\begin{Verbatim}[commandchars=\\\{\}]
\PYG{g+gp}{\PYGZgt{}\PYGZgt{}\PYGZgt{} }\PYG{n}{a}\PYG{o}{=}\PYG{n}{vcs}\PYG{o}{.}\PYG{n}{init}\PYG{p}{(}\PYG{p}{)} \PYG{c+c1}{\PYGZsh{} Make a VCS Canvas object to work with:}
\PYG{g+gp}{\PYGZgt{}\PYGZgt{}\PYGZgt{} }\PYG{n}{a}\PYG{o}{.}\PYG{n}{createtextcombined}\PYG{p}{(}\PYG{l+s+s1}{\PYGZsq{}}\PYG{l+s+s1}{EXAMPLE\PYGZus{}tt}\PYG{l+s+s1}{\PYGZsq{}}\PYG{p}{,} \PYG{l+s+s1}{\PYGZsq{}}\PYG{l+s+s1}{qa}\PYG{l+s+s1}{\PYGZsq{}}\PYG{p}{,} \PYG{l+s+s1}{\PYGZsq{}}\PYG{l+s+s1}{EXAMPLE\PYGZus{}tto}\PYG{l+s+s1}{\PYGZsq{}}\PYG{p}{,} \PYG{l+s+s1}{\PYGZsq{}}\PYG{l+s+s1}{7left}\PYG{l+s+s1}{\PYGZsq{}}\PYG{p}{)} \PYG{c+c1}{\PYGZsh{} Create \PYGZsq{}EXAMPLE\PYGZus{}tt\PYGZsq{} and \PYGZsq{}EXAMPLE\PYGZus{}tto\PYGZsq{}}
\PYG{g+go}{\PYGZlt{}vcs.textcombined.Tc ...\PYGZgt{}}
\PYG{g+gp}{\PYGZgt{}\PYGZgt{}\PYGZgt{} }\PYG{n}{a}\PYG{o}{.}\PYG{n}{show}\PYG{p}{(}\PYG{l+s+s1}{\PYGZsq{}}\PYG{l+s+s1}{textcombined}\PYG{l+s+s1}{\PYGZsq{}}\PYG{p}{)} \PYG{c+c1}{\PYGZsh{} Show all available textcombined}
\PYG{g+go}{*******************Textcombined Names List**********************}
\PYG{g+gp}{...}
\PYG{g+go}{*******************End Textcombined Names List**********************}
\PYG{g+gp}{\PYGZgt{}\PYGZgt{}\PYGZgt{} }\PYG{n}{ex} \PYG{o}{=} \PYG{n}{a}\PYG{o}{.}\PYG{n}{gettextcombined}\PYG{p}{(}\PYG{l+s+s1}{\PYGZsq{}}\PYG{l+s+s1}{EXAMPLE\PYGZus{}tt}\PYG{l+s+s1}{\PYGZsq{}}\PYG{p}{,} \PYG{l+s+s1}{\PYGZsq{}}\PYG{l+s+s1}{EXAMPLE\PYGZus{}tto}\PYG{l+s+s1}{\PYGZsq{}}\PYG{p}{)} \PYG{c+c1}{\PYGZsh{} To  test an existing textcombined object}
\PYG{g+gp}{\PYGZgt{}\PYGZgt{}\PYGZgt{} }\PYG{n}{vcs}\PYG{o}{.}\PYG{n}{queries}\PYG{o}{.}\PYG{n}{istextcombined}\PYG{p}{(}\PYG{n}{ex}\PYG{p}{)}
\PYG{g+go}{1}
\end{Verbatim}

\item[{Parameters}] \leavevmode
\textbf{\texttt{obj}} (\emph{\texttt{VCS Object}}) -- A VCS object

\item[{Returns}] \leavevmode
An integer indicating whether the object is a textcombined secondary method (1), or not (0).

\item[{Return type}] \leavevmode
\href{https://docs.python.org/2/library/functions.html\#int}{int}

\end{description}\end{quote}

\end{fulllineitems}

\index{istextorientation() (in module vcs.queries)}

\begin{fulllineitems}
\phantomsection\label{vcs/misc/queries:vcs.queries.istextorientation}\pysiglinewithargsret{\sphinxcode{vcs.queries.}\sphinxbfcode{istextorientation}}{\emph{obj}}{}
Check to see if this object is a VCS textorientation secondary method.
\begin{quote}\begin{description}
\item[{Example}] \leavevmode
\begin{Verbatim}[commandchars=\\\{\}]
\PYG{g+gp}{\PYGZgt{}\PYGZgt{}\PYGZgt{} }\PYG{n}{a}\PYG{o}{=}\PYG{n}{vcs}\PYG{o}{.}\PYG{n}{init}\PYG{p}{(}\PYG{p}{)} \PYG{c+c1}{\PYGZsh{} Make a VCS Canvas object to work with:}
\PYG{g+gp}{\PYGZgt{}\PYGZgt{}\PYGZgt{} }\PYG{n}{a}\PYG{o}{.}\PYG{n}{show}\PYG{p}{(}\PYG{l+s+s1}{\PYGZsq{}}\PYG{l+s+s1}{textorientation}\PYG{l+s+s1}{\PYGZsq{}}\PYG{p}{)} \PYG{c+c1}{\PYGZsh{} Show all available textorientation}
\PYG{g+go}{*******************Textorientation Names List**********************}
\PYG{g+gp}{...}
\PYG{g+go}{*******************End Textorientation Names List**********************}
\PYG{g+gp}{\PYGZgt{}\PYGZgt{}\PYGZgt{} }\PYG{n}{ex} \PYG{o}{=} \PYG{n}{a}\PYG{o}{.}\PYG{n}{gettextorientation}\PYG{p}{(}\PYG{p}{)} \PYG{c+c1}{\PYGZsh{} To  test an existing textorientation object}
\PYG{g+gp}{\PYGZgt{}\PYGZgt{}\PYGZgt{} }\PYG{n}{vcs}\PYG{o}{.}\PYG{n}{queries}\PYG{o}{.}\PYG{n}{istextorientation}\PYG{p}{(}\PYG{n}{ex}\PYG{p}{)}
\PYG{g+go}{1}
\end{Verbatim}

\item[{Parameters}] \leavevmode
\textbf{\texttt{obj}} (\emph{\texttt{VCS Object}}) -- A VCS object

\item[{Returns}] \leavevmode
An integer indicating whether the object is a textorientation secondary method (1), or not (0).

\item[{Return type}] \leavevmode
\href{https://docs.python.org/2/library/functions.html\#int}{int}

\end{description}\end{quote}

\end{fulllineitems}

\index{istexttable() (in module vcs.queries)}

\begin{fulllineitems}
\phantomsection\label{vcs/misc/queries:vcs.queries.istexttable}\pysiglinewithargsret{\sphinxcode{vcs.queries.}\sphinxbfcode{istexttable}}{\emph{obj}}{}
Check to see if this object is a VCS texttable secondary method.
\begin{quote}\begin{description}
\item[{Example}] \leavevmode
\begin{Verbatim}[commandchars=\\\{\}]
\PYG{g+gp}{\PYGZgt{}\PYGZgt{}\PYGZgt{} }\PYG{n}{a}\PYG{o}{=}\PYG{n}{vcs}\PYG{o}{.}\PYG{n}{init}\PYG{p}{(}\PYG{p}{)} \PYG{c+c1}{\PYGZsh{} Make a VCS Canvas object to work with:}
\PYG{g+gp}{\PYGZgt{}\PYGZgt{}\PYGZgt{} }\PYG{n}{a}\PYG{o}{.}\PYG{n}{show}\PYG{p}{(}\PYG{l+s+s1}{\PYGZsq{}}\PYG{l+s+s1}{texttable}\PYG{l+s+s1}{\PYGZsq{}}\PYG{p}{)} \PYG{c+c1}{\PYGZsh{} Show all available texttable}
\PYG{g+go}{*******************Texttable Names List**********************}
\PYG{g+gp}{...}
\PYG{g+go}{*******************End Texttable Names List**********************}
\PYG{g+gp}{\PYGZgt{}\PYGZgt{}\PYGZgt{} }\PYG{n}{ex} \PYG{o}{=} \PYG{n}{a}\PYG{o}{.}\PYG{n}{gettexttable}\PYG{p}{(}\PYG{p}{)} \PYG{c+c1}{\PYGZsh{} To  test an existing texttable object}
\PYG{g+gp}{\PYGZgt{}\PYGZgt{}\PYGZgt{} }\PYG{n}{vcs}\PYG{o}{.}\PYG{n}{queries}\PYG{o}{.}\PYG{n}{istexttable}\PYG{p}{(}\PYG{n}{ex}\PYG{p}{)}
\PYG{g+go}{1}
\end{Verbatim}

\item[{Parameters}] \leavevmode
\textbf{\texttt{obj}} (\emph{\texttt{VCS Object}}) -- A VCS object

\item[{Returns}] \leavevmode
An integer indicating whether the object is a texttable secondary method (1), or not (0).

\item[{Return type}] \leavevmode
\href{https://docs.python.org/2/library/functions.html\#int}{int}

\end{description}\end{quote}

\end{fulllineitems}

\index{isvector() (in module vcs.queries)}

\begin{fulllineitems}
\phantomsection\label{vcs/misc/queries:vcs.queries.isvector}\pysiglinewithargsret{\sphinxcode{vcs.queries.}\sphinxbfcode{isvector}}{\emph{obj}}{}
Check to see if this object is a VCS 1d graphics method.
\begin{quote}\begin{description}
\item[{Example}] \leavevmode
\begin{Verbatim}[commandchars=\\\{\}]
\PYG{g+gp}{\PYGZgt{}\PYGZgt{}\PYGZgt{} }\PYG{n}{a}\PYG{o}{=}\PYG{n}{vcs}\PYG{o}{.}\PYG{n}{init}\PYG{p}{(}\PYG{p}{)} \PYG{c+c1}{\PYGZsh{} Make a VCS Canvas object to work with:}
\PYG{g+gp}{\PYGZgt{}\PYGZgt{}\PYGZgt{} }\PYG{n}{a}\PYG{o}{.}\PYG{n}{show}\PYG{p}{(}\PYG{l+s+s1}{\PYGZsq{}}\PYG{l+s+s1}{1d}\PYG{l+s+s1}{\PYGZsq{}}\PYG{p}{)} \PYG{c+c1}{\PYGZsh{} Show all available 1d}
\PYG{g+go}{*******************1d Names List**********************}
\PYG{g+gp}{...}
\PYG{g+go}{*******************End 1d Names List**********************}
\PYG{g+gp}{\PYGZgt{}\PYGZgt{}\PYGZgt{} }\PYG{n}{ex} \PYG{o}{=} \PYG{n}{a}\PYG{o}{.}\PYG{n}{get1d}\PYG{p}{(}\PYG{l+s+s1}{\PYGZsq{}}\PYG{l+s+s1}{default}\PYG{l+s+s1}{\PYGZsq{}}\PYG{p}{)} \PYG{c+c1}{\PYGZsh{} To  test an existing 1d object}
\PYG{g+gp}{\PYGZgt{}\PYGZgt{}\PYGZgt{} }\PYG{n}{vcs}\PYG{o}{.}\PYG{n}{queries}\PYG{o}{.}\PYG{n}{is1d}\PYG{p}{(}\PYG{n}{ex}\PYG{p}{)}
\PYG{g+go}{1}
\end{Verbatim}

\item[{Parameters}] \leavevmode
\textbf{\texttt{obj}} (\emph{\texttt{VCS Object}}) -- A VCS object

\item[{Returns}] \leavevmode
An integer indicating whether the object is a 1d graphics method (1), or not (0).

\item[{Return type}] \leavevmode
\href{https://docs.python.org/2/library/functions.html\#int}{int}

\end{description}\end{quote}

\end{fulllineitems}

\index{isxvsy() (in module vcs.queries)}

\begin{fulllineitems}
\phantomsection\label{vcs/misc/queries:vcs.queries.isxvsy}\pysiglinewithargsret{\sphinxcode{vcs.queries.}\sphinxbfcode{isxvsy}}{\emph{obj}}{}
Check to see if this object is a VCS xvsy graphics method.
\begin{quote}\begin{description}
\item[{Example}] \leavevmode
\begin{Verbatim}[commandchars=\\\{\}]
\PYG{g+gp}{\PYGZgt{}\PYGZgt{}\PYGZgt{} }\PYG{n}{a}\PYG{o}{=}\PYG{n}{vcs}\PYG{o}{.}\PYG{n}{init}\PYG{p}{(}\PYG{p}{)} \PYG{c+c1}{\PYGZsh{} Make a VCS Canvas object to work with:}
\PYG{g+gp}{\PYGZgt{}\PYGZgt{}\PYGZgt{} }\PYG{n}{a}\PYG{o}{.}\PYG{n}{show}\PYG{p}{(}\PYG{l+s+s1}{\PYGZsq{}}\PYG{l+s+s1}{xvsy}\PYG{l+s+s1}{\PYGZsq{}}\PYG{p}{)} \PYG{c+c1}{\PYGZsh{} Show all available xvsy}
\PYG{g+go}{*******************Xvsy Names List**********************}
\PYG{g+gp}{...}
\PYG{g+go}{*******************End Xvsy Names List**********************}
\PYG{g+gp}{\PYGZgt{}\PYGZgt{}\PYGZgt{} }\PYG{n}{ex} \PYG{o}{=} \PYG{n}{a}\PYG{o}{.}\PYG{n}{getxvsy}\PYG{p}{(}\PYG{p}{)} \PYG{c+c1}{\PYGZsh{} To  test an existing xvsy object}
\PYG{g+gp}{\PYGZgt{}\PYGZgt{}\PYGZgt{} }\PYG{n}{vcs}\PYG{o}{.}\PYG{n}{queries}\PYG{o}{.}\PYG{n}{isxvsy}\PYG{p}{(}\PYG{n}{ex}\PYG{p}{)}
\PYG{g+go}{1}
\end{Verbatim}

\item[{Parameters}] \leavevmode
\textbf{\texttt{obj}} (\emph{\texttt{VCS Object}}) -- A VCS object

\item[{Returns}] \leavevmode
An integer indicating whether the object is a xvsy graphics method (1), or not (0).

\item[{Return type}] \leavevmode
\href{https://docs.python.org/2/library/functions.html\#int}{int}

\end{description}\end{quote}

\end{fulllineitems}

\index{isxyvsy() (in module vcs.queries)}

\begin{fulllineitems}
\phantomsection\label{vcs/misc/queries:vcs.queries.isxyvsy}\pysiglinewithargsret{\sphinxcode{vcs.queries.}\sphinxbfcode{isxyvsy}}{\emph{obj}}{}
Check to see if this object is a VCS xyvsy graphics method.
\begin{quote}\begin{description}
\item[{Example}] \leavevmode
\begin{Verbatim}[commandchars=\\\{\}]
\PYG{g+gp}{\PYGZgt{}\PYGZgt{}\PYGZgt{} }\PYG{n}{a}\PYG{o}{=}\PYG{n}{vcs}\PYG{o}{.}\PYG{n}{init}\PYG{p}{(}\PYG{p}{)} \PYG{c+c1}{\PYGZsh{} Make a VCS Canvas object to work with:}
\PYG{g+gp}{\PYGZgt{}\PYGZgt{}\PYGZgt{} }\PYG{n}{a}\PYG{o}{.}\PYG{n}{show}\PYG{p}{(}\PYG{l+s+s1}{\PYGZsq{}}\PYG{l+s+s1}{xyvsy}\PYG{l+s+s1}{\PYGZsq{}}\PYG{p}{)} \PYG{c+c1}{\PYGZsh{} Show all available xyvsy}
\PYG{g+go}{*******************Xyvsy Names List**********************}
\PYG{g+gp}{...}
\PYG{g+go}{*******************End Xyvsy Names List**********************}
\PYG{g+gp}{\PYGZgt{}\PYGZgt{}\PYGZgt{} }\PYG{n}{ex} \PYG{o}{=} \PYG{n}{a}\PYG{o}{.}\PYG{n}{getxyvsy}\PYG{p}{(}\PYG{l+s+s1}{\PYGZsq{}}\PYG{l+s+s1}{default\PYGZus{}xyvsy\PYGZus{}}\PYG{l+s+s1}{\PYGZsq{}}\PYG{p}{)} \PYG{c+c1}{\PYGZsh{} To  test an existing xyvsy object}
\PYG{g+gp}{\PYGZgt{}\PYGZgt{}\PYGZgt{} }\PYG{n}{vcs}\PYG{o}{.}\PYG{n}{queries}\PYG{o}{.}\PYG{n}{isxyvsy}\PYG{p}{(}\PYG{n}{ex}\PYG{p}{)}
\PYG{g+go}{1}
\end{Verbatim}

\item[{Parameters}] \leavevmode
\textbf{\texttt{obj}} (\emph{\texttt{VCS Object}}) -- A VCS object

\item[{Returns}] \leavevmode
An integer indicating whether the object is a xyvsy graphics method (1), or not (0).

\item[{Return type}] \leavevmode
\href{https://docs.python.org/2/library/functions.html\#int}{int}

\end{description}\end{quote}

\end{fulllineitems}

\index{isyxvsx() (in module vcs.queries)}

\begin{fulllineitems}
\phantomsection\label{vcs/misc/queries:vcs.queries.isyxvsx}\pysiglinewithargsret{\sphinxcode{vcs.queries.}\sphinxbfcode{isyxvsx}}{\emph{obj}}{}
Check to see if this object is a VCS yxvsx graphics method.
\begin{quote}\begin{description}
\item[{Example}] \leavevmode
\begin{Verbatim}[commandchars=\\\{\}]
\PYG{g+gp}{\PYGZgt{}\PYGZgt{}\PYGZgt{} }\PYG{n}{a}\PYG{o}{=}\PYG{n}{vcs}\PYG{o}{.}\PYG{n}{init}\PYG{p}{(}\PYG{p}{)} \PYG{c+c1}{\PYGZsh{} Make a VCS Canvas object to work with:}
\PYG{g+gp}{\PYGZgt{}\PYGZgt{}\PYGZgt{} }\PYG{n}{a}\PYG{o}{.}\PYG{n}{show}\PYG{p}{(}\PYG{l+s+s1}{\PYGZsq{}}\PYG{l+s+s1}{yxvsx}\PYG{l+s+s1}{\PYGZsq{}}\PYG{p}{)} \PYG{c+c1}{\PYGZsh{} Show all available yxvsx}
\PYG{g+go}{*******************Yxvsx Names List**********************}
\PYG{g+gp}{...}
\PYG{g+go}{*******************End Yxvsx Names List**********************}
\PYG{g+gp}{\PYGZgt{}\PYGZgt{}\PYGZgt{} }\PYG{n}{ex} \PYG{o}{=} \PYG{n}{a}\PYG{o}{.}\PYG{n}{getyxvsx}\PYG{p}{(}\PYG{p}{)} \PYG{c+c1}{\PYGZsh{} To  test an existing yxvsx object}
\PYG{g+gp}{\PYGZgt{}\PYGZgt{}\PYGZgt{} }\PYG{n}{vcs}\PYG{o}{.}\PYG{n}{queries}\PYG{o}{.}\PYG{n}{isyxvsx}\PYG{p}{(}\PYG{n}{ex}\PYG{p}{)}
\PYG{g+go}{1}
\end{Verbatim}

\item[{Parameters}] \leavevmode
\textbf{\texttt{obj}} (\emph{\texttt{VCS Object}}) -- A VCS object

\item[{Returns}] \leavevmode
An integer indicating whether the object is a yxvsx graphics method (1), or not (0).

\item[{Return type}] \leavevmode
\href{https://docs.python.org/2/library/functions.html\#int}{int}

\end{description}\end{quote}

\end{fulllineitems}



\subsection{utils}
\label{vcs/misc/utils:module-vcs.utils}\label{vcs/misc/utils::doc}\label{vcs/misc/utils:utils}\index{vcs.utils (module)}\index{generate\_time\_labels() (in module vcs.utils)}

\begin{fulllineitems}
\phantomsection\label{vcs/misc/utils:vcs.utils.generate_time_labels}\pysiglinewithargsret{\sphinxcode{vcs.utils.}\sphinxbfcode{generate\_time\_labels}}{\emph{d1}, \emph{d2}, \emph{units}, \emph{calendar=135441}}{}
Generates a dictionary of time labels for an interval of time,
in a user defined units system.
\begin{quote}\begin{description}
\item[{Example}] \leavevmode
\begin{Verbatim}[commandchars=\\\{\}]
\PYG{c+c1}{\PYGZsh{} Two ways to generate a dictionary of time labels}
\PYG{o}{\PYGZgt{}\PYGZgt{}}\PYG{o}{\PYGZgt{}} \PYG{n}{lbls} \PYG{o}{=} \PYG{n}{generate\PYGZus{}time\PYGZus{}labels}\PYG{p}{(}\PYG{n}{cdtime}\PYG{o}{.}\PYG{n}{reltime}\PYG{p}{(}\PYG{l+m+mi}{0}\PYG{p}{,}\PYG{l+s+s1}{\PYGZsq{}}\PYG{l+s+s1}{months since 2000}\PYG{l+s+s1}{\PYGZsq{}}\PYG{p}{)}\PYG{p}{,}
\PYG{o}{.}\PYG{o}{.}\PYG{o}{.}     \PYG{n}{cdtime}\PYG{o}{.}\PYG{n}{reltime}\PYG{p}{(}\PYG{l+m+mi}{12}\PYG{p}{,}\PYG{l+s+s1}{\PYGZsq{}}\PYG{l+s+s1}{months since 2000}\PYG{l+s+s1}{\PYGZsq{}}\PYG{p}{)}\PYG{p}{,}
\PYG{o}{.}\PYG{o}{.}\PYG{o}{.}     \PYG{l+s+s1}{\PYGZsq{}}\PYG{l+s+s1}{days since 1800}\PYG{l+s+s1}{\PYGZsq{}}\PYG{p}{,}\PYG{p}{)} \PYG{c+c1}{\PYGZsh{} for the year 2000 in units of \PYGZsq{}days since 1800\PYGZsq{}}
\PYG{o}{\PYGZgt{}\PYGZgt{}}\PYG{o}{\PYGZgt{}} \PYG{n}{lbls} \PYG{o}{=} \PYG{n}{generate\PYGZus{}time\PYGZus{}labels}\PYG{p}{(}\PYG{n}{cdtime}\PYG{o}{.}\PYG{n}{reltime}\PYG{p}{(}\PYG{l+m+mi}{0}\PYG{p}{,}\PYG{l+s+s1}{\PYGZsq{}}\PYG{l+s+s1}{months since 2000}\PYG{l+s+s1}{\PYGZsq{}}\PYG{p}{)}\PYG{p}{,}
\PYG{o}{.}\PYG{o}{.}\PYG{o}{.}     \PYG{n}{cdtime}\PYG{o}{.}\PYG{n}{comptime}\PYG{p}{(}\PYG{l+m+mi}{2001}\PYG{p}{)}\PYG{p}{,}
\PYG{o}{.}\PYG{o}{.}\PYG{o}{.}     \PYG{l+s+s1}{\PYGZsq{}}\PYG{l+s+s1}{days since 1800}\PYG{l+s+s1}{\PYGZsq{}}\PYG{p}{,}\PYG{p}{)} \PYG{c+c1}{\PYGZsh{} for the year 2000 in units of \PYGZsq{}days since 1800\PYGZsq{}}
\PYG{o}{\PYGZgt{}\PYGZgt{}}\PYG{o}{\PYGZgt{}} \PYG{n}{lbls} \PYG{o}{=} \PYG{n}{generate\PYGZus{}time\PYGZus{}labels}\PYG{p}{(}\PYG{l+m+mi}{0}\PYG{p}{,} \PYG{l+m+mi}{12}\PYG{p}{,} \PYG{l+s+s1}{\PYGZsq{}}\PYG{l+s+s1}{months since 2000}\PYG{l+s+s1}{\PYGZsq{}}\PYG{p}{,} \PYG{p}{)} \PYG{c+c1}{\PYGZsh{} Generate a dictionary of time labels}
                                                            \PYG{c+c1}{\PYGZsh{} for year 2000, units of \PYGZsq{}months since 2000\PYGZsq{}}
\end{Verbatim}

\item[{Parameters}] \leavevmode\begin{itemize}
\item {} 
\textbf{\texttt{d1}} (\emph{\texttt{cdtime object, int, long, float}}) -- The beginning of the time interval to be labelled. Expects a cdtime object.
Can also take int, long, or float,
which will be used to create a cdtime object with the given units parameter.

\item {} 
\textbf{\texttt{d2}} (\emph{\texttt{cdtime object, int, long, float}}) -- The end of the time interval to be labelled. Expects a cdtime object.
Can also take int, long, or float,
which will be used to create a cdtime object with the given units parameter.

\item {} 
\textbf{\texttt{units}} (\href{https://docs.python.org/2/library/functions.html\#str}{\emph{\texttt{str}}}) -- String with the format `{[}time\_unit{]} since {[}date{]}'.

\item {} 
\textbf{\texttt{calendar}} -- A cdtime calendar,

\end{itemize}

\item[{Returns}] \leavevmode
Dictionary of time labels over the given time interval

\item[{Return type}] \leavevmode
\href{https://docs.python.org/2/library/stdtypes.html\#dict}{dict}

\end{description}\end{quote}

\end{fulllineitems}

\index{getcolorcell() (in module vcs.utils)}

\begin{fulllineitems}
\phantomsection\label{vcs/misc/utils:vcs.utils.getcolorcell}\pysiglinewithargsret{\sphinxcode{vcs.utils.}\sphinxbfcode{getcolorcell}}{\emph{cell}, \emph{obj=None}}{}
Gets the colorcell of the provided object's colormap at the specified cell index.
If no object is provided, or if the provided object has no colormap, the default colormap is used.
\begin{quote}\begin{description}
\item[{Example}] \leavevmode
\begin{Verbatim}[commandchars=\\\{\}]
\PYG{g+gp}{\PYGZgt{}\PYGZgt{}\PYGZgt{} }\PYG{n}{a}\PYG{o}{=}\PYG{n}{vcs}\PYG{o}{.}\PYG{n}{init}\PYG{p}{(}\PYG{p}{)}
\PYG{g+gp}{\PYGZgt{}\PYGZgt{}\PYGZgt{} }\PYG{n}{b}\PYG{o}{=}\PYG{n}{vcs}\PYG{o}{.}\PYG{n}{createboxfill}\PYG{p}{(}\PYG{p}{)}
\PYG{g+gp}{\PYGZgt{}\PYGZgt{}\PYGZgt{} }\PYG{n}{b}\PYG{o}{.}\PYG{n}{colormap}\PYG{o}{=}\PYG{l+s+s1}{\PYGZsq{}}\PYG{l+s+s1}{rainbow}\PYG{l+s+s1}{\PYGZsq{}}
\PYG{g+gp}{\PYGZgt{}\PYGZgt{}\PYGZgt{} }\PYG{n}{a}\PYG{o}{.}\PYG{n}{getcolorcell}\PYG{p}{(}\PYG{l+m+mi}{2}\PYG{p}{,}\PYG{n}{b}\PYG{p}{)}
\PYG{g+go}{[85, 85, 85, 100.0]}
\end{Verbatim}

\item[{Parameters}] \leavevmode\begin{itemize}
\item {} 
\textbf{\texttt{cell}} (\href{https://docs.python.org/2/library/functions.html\#int}{\emph{\texttt{int}}}) -- An integer value indicating the index of the desired colorcell.

\item {} 
\textbf{\texttt{obj}} (\emph{\texttt{Any VCS object capable of containing a colormap}}) -- Optional parameter containing the object to extract a colormap from.

\end{itemize}

\item[{Returns}] \leavevmode
The RGBA values of the colormap at the specified cell index.

\item[{Return type}] \leavevmode
{\hyperref[vcs/graphics/boxfill:vcs.boxfill.Gfb.list]{\sphinxcrossref{list}}}

\end{description}\end{quote}

\end{fulllineitems}

\index{getcolormap() (in module vcs.utils)}

\begin{fulllineitems}
\phantomsection\label{vcs/misc/utils:vcs.utils.getcolormap}\pysiglinewithargsret{\sphinxcode{vcs.utils.}\sphinxbfcode{getcolormap}}{\emph{Cp\_name\_src='default'}}{}
VCS contains a list of secondary methods. This function will create a
colormap class object from an existing VCS colormap secondary method. If
no colormap name is given, then colormap `default' will be used.

\begin{notice}{note}{Note:}
VCS does not allow the modification of `default' attribute sets.
However, a `default' attribute set that has been copied under a
different name can be modified. (See the createcolormap function.)
\end{notice}
\begin{quote}\begin{description}
\item[{Example}] \leavevmode
\begin{Verbatim}[commandchars=\\\{\}]
\PYG{g+gp}{\PYGZgt{}\PYGZgt{}\PYGZgt{} }\PYG{n}{a}\PYG{o}{=}\PYG{n}{vcs}\PYG{o}{.}\PYG{n}{init}\PYG{p}{(}\PYG{p}{)}
\PYG{g+gp}{\PYGZgt{}\PYGZgt{}\PYGZgt{} }\PYG{n}{a}\PYG{o}{.}\PYG{n}{show}\PYG{p}{(}\PYG{l+s+s1}{\PYGZsq{}}\PYG{l+s+s1}{colormap}\PYG{l+s+s1}{\PYGZsq{}}\PYG{p}{)} \PYG{c+c1}{\PYGZsh{} Show all the existing colormap secondary methods}
\PYG{g+go}{*******************Colormap Names List**********************}
\PYG{g+gp}{...}
\PYG{g+go}{*******************End Colormap Names List**********************}
\PYG{g+gp}{\PYGZgt{}\PYGZgt{}\PYGZgt{} }\PYG{n}{cp}\PYG{o}{=}\PYG{n}{a}\PYG{o}{.}\PYG{n}{getcolormap}\PYG{p}{(}\PYG{p}{)} \PYG{c+c1}{\PYGZsh{} cp instance of \PYGZsq{}default\PYGZsq{} colormap secondary method}
\PYG{g+gp}{\PYGZgt{}\PYGZgt{}\PYGZgt{} }\PYG{n}{cp2}\PYG{o}{=}\PYG{n}{a}\PYG{o}{.}\PYG{n}{getcolormap}\PYG{p}{(}\PYG{l+s+s1}{\PYGZsq{}}\PYG{l+s+s1}{rainbow}\PYG{l+s+s1}{\PYGZsq{}}\PYG{p}{)} \PYG{c+c1}{\PYGZsh{} cp2 instance of existing \PYGZsq{}rainbow\PYGZsq{} colormap secondary method}
\end{Verbatim}

\item[{Parameters}] \leavevmode
\textbf{\texttt{Cp\_name\_src}} (\href{https://docs.python.org/2/library/functions.html\#str}{\emph{\texttt{str}}}) -- String name of an existing colormap VCS object

\item[{Returns}] \leavevmode
A pre-existing VCS colormap object

\item[{Return type}] \leavevmode
{\hyperref[vcs/misc/colormap:vcs.colormap.Cp]{\sphinxcrossref{vcs.colormap.Cp}}}

\end{description}\end{quote}

\end{fulllineitems}

\index{getcolors() (in module vcs.utils)}

\begin{fulllineitems}
\phantomsection\label{vcs/misc/utils:vcs.utils.getcolors}\pysiglinewithargsret{\sphinxcode{vcs.utils.}\sphinxbfcode{getcolors}}{\emph{levs, colors={[}16, 17, 18, 19, 20, 21, 22, 23, 24, 25, 26, 27, 28, 29, 30, 31, 32, 33, 34, 35, 36, 37, 38, 39, 40, 41, 42, 43, 44, 45, 46, 47, 48, 49, 50, 51, 52, 53, 54, 55, 56, 57, 58, 59, 60, 61, 62, 63, 64, 65, 66, 67, 68, 69, 70, 71, 72, 73, 74, 75, 76, 77, 78, 79, 80, 81, 82, 83, 84, 85, 86, 87, 88, 89, 90, 91, 92, 93, 94, 95, 96, 97, 98, 99, 100, 101, 102, 103, 104, 105, 106, 107, 108, 109, 110, 111, 112, 113, 114, 115, 116, 117, 118, 119, 120, 121, 122, 123, 124, 125, 126, 127, 128, 129, 130, 131, 132, 133, 134, 135, 136, 137, 138, 139, 140, 141, 142, 143, 144, 145, 146, 147, 148, 149, 150, 151, 152, 153, 154, 155, 156, 157, 158, 159, 160, 161, 162, 163, 164, 165, 166, 167, 168, 169, 170, 171, 172, 173, 174, 175, 176, 177, 178, 179, 180, 181, 182, 183, 184, 185, 186, 187, 188, 189, 190, 191, 192, 193, 194, 195, 196, 197, 198, 199, 200, 201, 202, 203, 204, 205, 206, 207, 208, 209, 210, 211, 212, 213, 214, 215, 216, 217, 218, 219, 220, 221, 222, 223, 224, 225, 226, 227, 228, 229, 230, 231, 232, 233, 234, 235, 236, 237, 238, 239{]}, split=1, white=240}}{}
For isofill/boxfill purposes
Given a list of levels this function returns the colors that would
best spread a list of ``user-defined'' colors (default is 16 to 239,
i.e 224 colors), always using the first and last color.
Optionally the color range can be split into 2 equal domain to
represent \textless{}0 and \textgreater{}0 values.
If the colors are split an interval goes from \textless{}0 to \textgreater{}0
then this is assigned the ``white'' color
\begin{quote}\begin{description}
\item[{Example}] \leavevmode
\begin{Verbatim}[commandchars=\\\{\}]
\PYG{g+gp}{\PYGZgt{}\PYGZgt{}\PYGZgt{} }\PYG{n}{a}\PYG{o}{=}\PYG{p}{[}\PYG{l+m+mf}{0.0}\PYG{p}{,} \PYG{l+m+mf}{2.0}\PYG{p}{,} \PYG{l+m+mf}{4.0}\PYG{p}{,} \PYG{l+m+mf}{6.0}\PYG{p}{,} \PYG{l+m+mf}{8.0}\PYG{p}{,} \PYG{l+m+mf}{10.0}\PYG{p}{,} \PYG{l+m+mf}{12.0}\PYG{p}{,} \PYG{l+m+mf}{14.0}\PYG{p}{,} \PYG{l+m+mf}{16.0}\PYG{p}{,} \PYG{l+m+mf}{18.0}\PYG{p}{,} \PYG{l+m+mf}{20.0}\PYG{p}{]}
\PYG{g+gp}{\PYGZgt{}\PYGZgt{}\PYGZgt{} }\PYG{n}{vcs}\PYG{o}{.}\PYG{n}{getcolors} \PYG{p}{(}\PYG{n}{a}\PYG{p}{)}
\PYG{g+go}{[16, 41, 66, 90, 115, 140, 165, 189, 214, 239]}
\PYG{g+gp}{\PYGZgt{}\PYGZgt{}\PYGZgt{} }\PYG{n}{vcs}\PYG{o}{.}\PYG{n}{getcolors} \PYG{p}{(}\PYG{n}{a}\PYG{p}{,}\PYG{n}{colors}\PYG{o}{=}\PYG{n+nb}{range}\PYG{p}{(}\PYG{l+m+mi}{16}\PYG{p}{,}\PYG{l+m+mi}{200}\PYG{p}{)}\PYG{p}{)}
\PYG{g+go}{[16, 36, 57, 77, 97, 118, 138, 158, 179, 199]}
\PYG{g+gp}{\PYGZgt{}\PYGZgt{}\PYGZgt{} }\PYG{n}{vcs}\PYG{o}{.}\PYG{n}{getcolors}\PYG{p}{(}\PYG{n}{a}\PYG{p}{,}\PYG{n}{colors}\PYG{o}{=}\PYG{p}{[}\PYG{l+m+mi}{16}\PYG{p}{,}\PYG{l+m+mi}{25}\PYG{p}{,}\PYG{l+m+mi}{15}\PYG{p}{,}\PYG{l+m+mi}{56}\PYG{p}{,}\PYG{l+m+mi}{35}\PYG{p}{,}\PYG{l+m+mi}{234}\PYG{p}{,}\PYG{l+m+mi}{12}\PYG{p}{,}\PYG{l+m+mi}{11}\PYG{p}{,}\PYG{l+m+mi}{19}\PYG{p}{,}\PYG{l+m+mi}{32}\PYG{p}{,}\PYG{l+m+mi}{132}\PYG{p}{,}\PYG{l+m+mi}{17}\PYG{p}{]}\PYG{p}{)}
\PYG{g+go}{[16, 25, 15, 35, 234, 12, 11, 32, 132, 17]}
\PYG{g+gp}{\PYGZgt{}\PYGZgt{}\PYGZgt{} }\PYG{n}{a}\PYG{o}{=}\PYG{p}{[}\PYG{o}{\PYGZhy{}}\PYG{l+m+mf}{6.0}\PYG{p}{,} \PYG{o}{\PYGZhy{}}\PYG{l+m+mf}{2.0}\PYG{p}{,} \PYG{l+m+mf}{2.0}\PYG{p}{,} \PYG{l+m+mf}{6.0}\PYG{p}{,} \PYG{l+m+mf}{10.0}\PYG{p}{,} \PYG{l+m+mf}{14.0}\PYG{p}{,} \PYG{l+m+mf}{18.0}\PYG{p}{,} \PYG{l+m+mf}{22.0}\PYG{p}{,} \PYG{l+m+mf}{26.0}\PYG{p}{]}
\PYG{g+gp}{\PYGZgt{}\PYGZgt{}\PYGZgt{} }\PYG{n}{vcs}\PYG{o}{.}\PYG{n}{getcolors} \PYG{p}{(}\PYG{n}{a}\PYG{p}{,}\PYG{n}{white}\PYG{o}{=}\PYG{l+m+mi}{241}\PYG{p}{)}
\PYG{g+go}{[72, 241, 128, 150, 172, 195, 217, 239]}
\PYG{g+gp}{\PYGZgt{}\PYGZgt{}\PYGZgt{} }\PYG{n}{vcs}\PYG{o}{.}\PYG{n}{getcolors} \PYG{p}{(}\PYG{n}{a}\PYG{p}{,}\PYG{n}{white}\PYG{o}{=}\PYG{l+m+mi}{241}\PYG{p}{,}\PYG{n}{split}\PYG{o}{=}\PYG{l+m+mi}{0}\PYG{p}{)}
\PYG{g+go}{[16, 48, 80, 112, 143, 175, 207, 239]}
\end{Verbatim}

\item[{Parameters}] \leavevmode\begin{itemize}
\item {} 
\textbf{\texttt{levs}} (\emph{\texttt{list, tuple}}) -- levels defining the color ranges

\item {} 
\textbf{\texttt{colors}} ({\hyperref[vcs/graphics/boxfill:vcs.boxfill.Gfb.list]{\sphinxcrossref{\emph{\texttt{list}}}}}) -- A list/tuple of the of colors you wish to use

\item {} 
\textbf{\texttt{split}} (\href{https://docs.python.org/2/library/functions.html\#int}{\emph{\texttt{int}}}) -- Integer flag to split colors between two equal domains.
0 : no split
1 : split if the levels go from \textless{}0 to \textgreater{}0
2 : split even if all the values are positive or negative

\item {} 
\textbf{\texttt{white}} (\href{https://docs.python.org/2/library/functions.html\#int}{\emph{\texttt{int}}}) -- If split is on and an interval goes from \textless{}0 to \textgreater{}0 this color number
will be used within this interval (240 is white in the default VCS palette color).
Integer must be between 0 and 255.

\end{itemize}

\item[{Returns}] \leavevmode
List of colors

\item[{Return type}] \leavevmode
{\hyperref[vcs/graphics/boxfill:vcs.boxfill.Gfb.list]{\sphinxcrossref{list}}}

\end{description}\end{quote}

\end{fulllineitems}

\index{getfontname() (in module vcs.utils)}

\begin{fulllineitems}
\phantomsection\label{vcs/misc/utils:vcs.utils.getfontname}\pysiglinewithargsret{\sphinxcode{vcs.utils.}\sphinxbfcode{getfontname}}{\emph{number}}{}
Retrieve a font name for a given font index.
\begin{quote}\begin{description}
\item[{Parameters}] \leavevmode
\textbf{\texttt{number}} (\href{https://docs.python.org/2/library/functions.html\#int}{\emph{\texttt{int}}}) -- Index of the font to get the name of.

\end{description}\end{quote}

\end{fulllineitems}

\index{getfontnumber() (in module vcs.utils)}

\begin{fulllineitems}
\phantomsection\label{vcs/misc/utils:vcs.utils.getfontnumber}\pysiglinewithargsret{\sphinxcode{vcs.utils.}\sphinxbfcode{getfontnumber}}{\emph{name}}{}
Retrieve a font index for a given font name.
\begin{quote}\begin{description}
\item[{Parameters}] \leavevmode
\textbf{\texttt{name}} (\href{https://docs.python.org/2/library/functions.html\#str}{\emph{\texttt{str}}}) -- Name of the font to get the index of.

\end{description}\end{quote}

\end{fulllineitems}

\index{getworldcoordinates() (in module vcs.utils)}

\begin{fulllineitems}
\phantomsection\label{vcs/misc/utils:vcs.utils.getworldcoordinates}\pysiglinewithargsret{\sphinxcode{vcs.utils.}\sphinxbfcode{getworldcoordinates}}{\emph{gm}, \emph{X}, \emph{Y}}{}
Given a graphics method and two axes
figures out correct world coordinates.
\begin{quote}\begin{description}
\item[{Parameters}] \leavevmode\begin{itemize}
\item {} 
\textbf{\texttt{gm}} (\emph{\texttt{graphics method object}}) -- A VCS graphics method object to get worldcoordinates for.

\item {} 
\textbf{\texttt{X}} (\emph{\texttt{cdms2 transient axis}}) -- A cdms2 transient axs

\item {} 
\textbf{\texttt{Y}} (\emph{\texttt{cdms2 transient axis}}) -- A cdms2 transient axs

\end{itemize}

\item[{Returns}] \leavevmode


\item[{Return type}] \leavevmode


\end{description}\end{quote}

\end{fulllineitems}

\index{match\_color() (in module vcs.utils)}

\begin{fulllineitems}
\phantomsection\label{vcs/misc/utils:vcs.utils.match_color}\pysiglinewithargsret{\sphinxcode{vcs.utils.}\sphinxbfcode{match\_color}}{\emph{color}, \emph{colormap=None}}{}
Returns the color in the colormap that is
closest to the required color.
\begin{quote}\begin{description}
\item[{Example}] \leavevmode
\begin{Verbatim}[commandchars=\\\{\}]
\PYG{g+gp}{\PYGZgt{}\PYGZgt{}\PYGZgt{} }\PYG{n}{a}\PYG{o}{=}\PYG{n}{vcs}\PYG{o}{.}\PYG{n}{init}\PYG{p}{(}\PYG{p}{)}
\PYG{g+gp}{\PYGZgt{}\PYGZgt{}\PYGZgt{} }\PYG{n+nb}{print} \PYG{n}{vcs}\PYG{o}{.}\PYG{n}{match\PYGZus{}color}\PYG{p}{(}\PYG{l+s+s1}{\PYGZsq{}}\PYG{l+s+s1}{salmon}\PYG{l+s+s1}{\PYGZsq{}}\PYG{p}{)}
\PYG{g+gp}{\PYGZgt{}\PYGZgt{}\PYGZgt{} }\PYG{n+nb}{print} \PYG{n}{vcs}\PYG{o}{.}\PYG{n}{match\PYGZus{}color}\PYG{p}{(}\PYG{l+s+s1}{\PYGZsq{}}\PYG{l+s+s1}{red}\PYG{l+s+s1}{\PYGZsq{}}\PYG{p}{)}
\PYG{g+gp}{\PYGZgt{}\PYGZgt{}\PYGZgt{} }\PYG{n+nb}{print} \PYG{n}{vcs}\PYG{o}{.}\PYG{n}{match\PYGZus{}color}\PYG{p}{(}\PYG{p}{[}\PYG{l+m+mi}{0}\PYG{p}{,}\PYG{l+m+mi}{0}\PYG{p}{,}\PYG{l+m+mi}{100}\PYG{p}{]}\PYG{p}{,}\PYG{l+s+s1}{\PYGZsq{}}\PYG{l+s+s1}{default}\PYG{l+s+s1}{\PYGZsq{}}\PYG{p}{)} \PYG{c+c1}{\PYGZsh{} closest color from blue}
\end{Verbatim}

\item[{Parameters}] \leavevmode\begin{itemize}
\item {} 
\textbf{\texttt{color}} (\emph{\texttt{str, int}}) -- Either a string name, or a rgb value between 0 and 100.

\item {} 
\textbf{\texttt{colormap}} ({\hyperref[vcs/misc/colormap:vcs.colormap.Cp]{\sphinxcrossref{\emph{\texttt{vcs.colormap.Cp}}}}}) -- A VCS colormap object. If not specified, the default colormap is used.

\end{itemize}

\item[{Returns}] \leavevmode
Integer value representing a matching rgb color

\item[{Return type}] \leavevmode
\href{https://docs.python.org/2/library/functions.html\#int}{int}

\end{description}\end{quote}

\end{fulllineitems}

\index{minmax() (in module vcs.utils)}

\begin{fulllineitems}
\phantomsection\label{vcs/misc/utils:vcs.utils.minmax}\pysiglinewithargsret{\sphinxcode{vcs.utils.}\sphinxbfcode{minmax}}{\emph{*data}}{}
Return the minimum and maximum of a series of array/list/tuples
(or combination of these)
You can combine list/tuples/arrays pretty much any combination is allowed
\begin{quote}\begin{description}
\item[{Example}] \leavevmode
\begin{Verbatim}[commandchars=\\\{\}]
\PYG{g+gp}{\PYGZgt{}\PYGZgt{}\PYGZgt{} }\PYG{n}{s}\PYG{o}{=}\PYG{n+nb}{range}\PYG{p}{(}\PYG{l+m+mi}{7}\PYG{p}{)}
\PYG{g+gp}{\PYGZgt{}\PYGZgt{}\PYGZgt{} }\PYG{n}{vcs}\PYG{o}{.}\PYG{n}{minmax}\PYG{p}{(}\PYG{n}{s}\PYG{p}{)}
\PYG{g+go}{(0.0, 6.0)}
\PYG{g+gp}{\PYGZgt{}\PYGZgt{}\PYGZgt{} }\PYG{n}{vcs}\PYG{o}{.}\PYG{n}{minmax}\PYG{p}{(}\PYG{p}{[}\PYG{n}{s}\PYG{p}{,}\PYG{n}{s}\PYG{p}{]}\PYG{p}{)}
\PYG{g+go}{(0.0, 6.0)}
\PYG{g+gp}{\PYGZgt{}\PYGZgt{}\PYGZgt{} }\PYG{n}{vcs}\PYG{o}{.}\PYG{n}{minmax}\PYG{p}{(}\PYG{p}{[}\PYG{p}{[}\PYG{n}{s}\PYG{p}{,}\PYG{n}{s}\PYG{o}{*}\PYG{l+m+mi}{2}\PYG{p}{]}\PYG{p}{,}\PYG{l+m+mf}{4.}\PYG{p}{,}\PYG{p}{[}\PYG{l+m+mf}{6.}\PYG{p}{,}\PYG{l+m+mf}{7.}\PYG{p}{,}\PYG{n}{s}\PYG{p}{]}\PYG{p}{]}\PYG{p}{,}\PYG{p}{[}\PYG{l+m+mf}{5.}\PYG{p}{,}\PYG{o}{\PYGZhy{}}\PYG{l+m+mf}{7.}\PYG{p}{,}\PYG{l+m+mi}{8}\PYG{p}{,}\PYG{p}{(}\PYG{l+m+mf}{6.}\PYG{p}{,}\PYG{l+m+mf}{1.}\PYG{p}{)}\PYG{p}{]}\PYG{p}{)}
\PYG{g+go}{(\PYGZhy{}7.0, 8.0)}
\end{Verbatim}

\item[{Parameters}] \leavevmode
\textbf{\texttt{data}} ({\hyperref[vcs/graphics/boxfill:vcs.boxfill.Gfb.list]{\sphinxcrossref{\emph{\texttt{list}}}}}) -- A comma-separated list of lists/arrays/tuples

\item[{Returns}] \leavevmode
A tuple in the form (min, max)

\item[{Return type}] \leavevmode
\href{https://docs.python.org/2/library/functions.html\#tuple}{tuple}

\end{description}\end{quote}

\end{fulllineitems}

\index{mkevenlevels() (in module vcs.utils)}

\begin{fulllineitems}
\phantomsection\label{vcs/misc/utils:vcs.utils.mkevenlevels}\pysiglinewithargsret{\sphinxcode{vcs.utils.}\sphinxbfcode{mkevenlevels}}{\emph{n1}, \emph{n2}, \emph{nlev=10}}{}
Return a series of evenly spaced levels going from n1 to n2.
By default 10 intervals will be produced.
\begin{quote}\begin{description}
\item[{Example}] \leavevmode
\begin{Verbatim}[commandchars=\\\{\}]
\PYG{g+gp}{\PYGZgt{}\PYGZgt{}\PYGZgt{} }\PYG{n}{vcs}\PYG{o}{.}\PYG{n}{mkevenlevels}\PYG{p}{(}\PYG{l+m+mi}{0}\PYG{p}{,}\PYG{l+m+mi}{100}\PYG{p}{)}
\PYG{g+go}{[0.0, 10.0, 20.0, 30.0, 40.0, 50.0, 60.0, 70.0, 80.0, 90.0, 100.0]}
\PYG{g+gp}{\PYGZgt{}\PYGZgt{}\PYGZgt{} }\PYG{n}{vcs}\PYG{o}{.}\PYG{n}{mkevenlevels}\PYG{p}{(}\PYG{l+m+mi}{0}\PYG{p}{,}\PYG{l+m+mi}{100}\PYG{p}{,}\PYG{n}{nlev}\PYG{o}{=}\PYG{l+m+mi}{5}\PYG{p}{)}
\PYG{g+go}{[0.0, 20.0, 40.0, 60.0, 80.0, 100.0]}
\PYG{g+gp}{\PYGZgt{}\PYGZgt{}\PYGZgt{} }\PYG{n}{vcs}\PYG{o}{.}\PYG{n}{mkevenlevels}\PYG{p}{(}\PYG{l+m+mi}{100}\PYG{p}{,}\PYG{l+m+mi}{0}\PYG{p}{,}\PYG{n}{nlev}\PYG{o}{=}\PYG{l+m+mi}{5}\PYG{p}{)}
\PYG{g+go}{[100.0, 80.0, 60.0, 40.0, 20.0, 0.0]}
\end{Verbatim}

\item[{Parameters}] \leavevmode\begin{itemize}
\item {} 
\textbf{\texttt{n1}} (\emph{\texttt{int, float}}) -- Beginning of range. Int or float.

\item {} 
\textbf{\texttt{n2}} (\emph{\texttt{int, float}}) -- End of range. Int or float.

\item {} 
\textbf{\texttt{nlev}} (\href{https://docs.python.org/2/library/functions.html\#int}{\emph{\texttt{int}}}) -- Number of levels by which to split the given range.

\end{itemize}

\item[{Returns}] \leavevmode
List of floats, splitting range evenly between n1 and n2

\item[{Return type}] \leavevmode
{\hyperref[vcs/graphics/boxfill:vcs.boxfill.Gfb.list]{\sphinxcrossref{list}}}

\end{description}\end{quote}

\end{fulllineitems}

\index{mklabels() (in module vcs.utils)}

\begin{fulllineitems}
\phantomsection\label{vcs/misc/utils:vcs.utils.mklabels}\pysiglinewithargsret{\sphinxcode{vcs.utils.}\sphinxbfcode{mklabels}}{\emph{vals}, \emph{output='dict'}}{}
This function gets levels and output strings for nice display of the
levels values.
\begin{quote}\begin{description}
\item[{Examples}] \leavevmode
\begin{Verbatim}[commandchars=\\\{\}]
\PYG{g+gp}{\PYGZgt{}\PYGZgt{}\PYGZgt{} }\PYG{n}{a}\PYG{o}{=}\PYG{n}{vcs}\PYG{o}{.}\PYG{n}{mkscale}\PYG{p}{(}\PYG{l+m+mi}{2}\PYG{p}{,}\PYG{l+m+mi}{20}\PYG{p}{,}\PYG{n}{zero}\PYG{o}{=}\PYG{l+m+mi}{2}\PYG{p}{)}
\PYG{g+gp}{\PYGZgt{}\PYGZgt{}\PYGZgt{} }\PYG{n}{vcs}\PYG{o}{.}\PYG{n}{mklabels} \PYG{p}{(}\PYG{n}{a}\PYG{p}{)}
\PYG{g+go}{\PYGZob{}20.0: \PYGZsq{}20\PYGZsq{}, 18.0: \PYGZsq{}18\PYGZsq{}, 16.0: \PYGZsq{}16\PYGZsq{}, 14.0: \PYGZsq{}14\PYGZsq{}, 12.0: \PYGZsq{}12\PYGZsq{},}
\PYG{g+go}{    10.0: \PYGZsq{}10\PYGZsq{}, 8.0: \PYGZsq{}8\PYGZsq{}, 6.0: \PYGZsq{}6\PYGZsq{}, 4.0: \PYGZsq{}4\PYGZsq{}, 2.0: \PYGZsq{}2\PYGZsq{}, 0.0: \PYGZsq{}0\PYGZsq{}\PYGZcb{}}
\PYG{g+gp}{\PYGZgt{}\PYGZgt{}\PYGZgt{} }\PYG{n}{vcs}\PYG{o}{.}\PYG{n}{mklabels} \PYG{p}{(} \PYG{p}{[}\PYG{l+m+mi}{5}\PYG{p}{,}\PYG{o}{.}\PYG{l+m+mi}{005}\PYG{p}{]}\PYG{p}{)}
\PYG{g+go}{\PYGZob{}0.0050000000000000001: \PYGZsq{}0.005\PYGZsq{}, 5.0: \PYGZsq{}5.000\PYGZsq{}\PYGZcb{}}
\PYG{g+gp}{\PYGZgt{}\PYGZgt{}\PYGZgt{} }\PYG{n}{vcs}\PYG{o}{.}\PYG{n}{mklabels} \PYG{p}{(} \PYG{p}{[}\PYG{o}{.}\PYG{l+m+mi}{00002}\PYG{p}{,}\PYG{o}{.}\PYG{l+m+mi}{00005}\PYG{p}{]}\PYG{p}{)}
\PYG{g+go}{\PYGZob{}2.0000000000000002e\PYGZhy{}05: \PYGZsq{}2E\PYGZhy{}5\PYGZsq{}, 5.0000000000000002e\PYGZhy{}05: \PYGZsq{}5E\PYGZhy{}5\PYGZsq{}\PYGZcb{}}
\PYG{g+gp}{\PYGZgt{}\PYGZgt{}\PYGZgt{} }\PYG{n}{vcs}\PYG{o}{.}\PYG{n}{mklabels} \PYG{p}{(} \PYG{p}{[}\PYG{o}{.}\PYG{l+m+mi}{00002}\PYG{p}{,}\PYG{o}{.}\PYG{l+m+mi}{00005}\PYG{p}{]}\PYG{p}{,}\PYG{n}{output}\PYG{o}{=}\PYG{l+s+s1}{\PYGZsq{}}\PYG{l+s+s1}{list}\PYG{l+s+s1}{\PYGZsq{}}\PYG{p}{)}
\PYG{g+go}{[\PYGZsq{}2E\PYGZhy{}5\PYGZsq{}, \PYGZsq{}5E\PYGZhy{}5\PYGZsq{}]}
\end{Verbatim}

\item[{Parameters}] \leavevmode\begin{itemize}
\item {} 
\textbf{\texttt{vals}} (\emph{\texttt{list, tuple}}) -- List or tuple of float values

\item {} 
\textbf{\texttt{output}} (\href{https://docs.python.org/2/library/functions.html\#str}{\emph{\texttt{str}}}) -- Specifies the desired output type. One of {[}'dict', `list'{]}.

\end{itemize}

\item[{Returns}] \leavevmode
Dictionary or list of labels for the given values.

\item[{Return type}] \leavevmode
dict, list

\end{description}\end{quote}

\end{fulllineitems}

\index{mkscale() (in module vcs.utils)}

\begin{fulllineitems}
\phantomsection\label{vcs/misc/utils:vcs.utils.mkscale}\pysiglinewithargsret{\sphinxcode{vcs.utils.}\sphinxbfcode{mkscale}}{\emph{n1}, \emph{n2}, \emph{nc=12}, \emph{zero=1}, \emph{ends=False}}{}
This function return a nice scale given a min and a max

\begin{notice}{warning}{Warning:}
Not all functionality for the `zero' parameter has been implemented.
zero=0 is intended to let the function decide what should be done with zeros,
but it has yet to be defined. Do not use zero=0.
\end{notice}
\begin{quote}\begin{description}
\item[{Examples}] \leavevmode
\begin{Verbatim}[commandchars=\\\{\}]
\PYG{g+gp}{\PYGZgt{}\PYGZgt{}\PYGZgt{} }\PYG{n}{vcs}\PYG{o}{.}\PYG{n}{mkscale}\PYG{p}{(}\PYG{l+m+mi}{0}\PYG{p}{,}\PYG{l+m+mi}{100}\PYG{p}{)}
\PYG{g+go}{[0.0, 10.0, 20.0, 30.0, 40.0, 50.0, 60.0, 70.0, 80.0, 90.0, 100.0]}
\PYG{g+gp}{\PYGZgt{}\PYGZgt{}\PYGZgt{} }\PYG{n}{vcs}\PYG{o}{.}\PYG{n}{mkscale}\PYG{p}{(}\PYG{l+m+mi}{0}\PYG{p}{,}\PYG{l+m+mi}{100}\PYG{p}{,}\PYG{n}{nc}\PYG{o}{=}\PYG{l+m+mi}{5}\PYG{p}{)}
\PYG{g+go}{[0.0, 20.0, 40.0, 60.0, 80.0, 100.0]}
\PYG{g+gp}{\PYGZgt{}\PYGZgt{}\PYGZgt{} }\PYG{n}{vcs}\PYG{o}{.}\PYG{n}{mkscale}\PYG{p}{(}\PYG{o}{\PYGZhy{}}\PYG{l+m+mi}{10}\PYG{p}{,}\PYG{l+m+mi}{100}\PYG{p}{,}\PYG{n}{nc}\PYG{o}{=}\PYG{l+m+mi}{5}\PYG{p}{)}
\PYG{g+go}{[\PYGZhy{}25.0, 0.0, 25.0, 50.0, 75.0, 100.0]}
\PYG{g+gp}{\PYGZgt{}\PYGZgt{}\PYGZgt{} }\PYG{n}{vcs}\PYG{o}{.}\PYG{n}{mkscale}\PYG{p}{(}\PYG{o}{\PYGZhy{}}\PYG{l+m+mi}{10}\PYG{p}{,}\PYG{l+m+mi}{100}\PYG{p}{,}\PYG{n}{nc}\PYG{o}{=}\PYG{l+m+mi}{5}\PYG{p}{,}\PYG{n}{zero}\PYG{o}{=}\PYG{o}{\PYGZhy{}}\PYG{l+m+mi}{1}\PYG{p}{)}
\PYG{g+go}{[\PYGZhy{}20.0, 20.0, 60.0, 100.0]}
\PYG{g+gp}{\PYGZgt{}\PYGZgt{}\PYGZgt{} }\PYG{n}{vcs}\PYG{o}{.}\PYG{n}{mkscale}\PYG{p}{(}\PYG{l+m+mi}{2}\PYG{p}{,}\PYG{l+m+mi}{20}\PYG{p}{)}
\PYG{g+go}{[2.0, 4.0, 6.0, 8.0, 10.0, 12.0, 14.0, 16.0, 18.0, 20.0]}
\PYG{g+gp}{\PYGZgt{}\PYGZgt{}\PYGZgt{} }\PYG{n}{vcs}\PYG{o}{.}\PYG{n}{mkscale}\PYG{p}{(}\PYG{l+m+mi}{2}\PYG{p}{,}\PYG{l+m+mi}{20}\PYG{p}{,}\PYG{n}{zero}\PYG{o}{=}\PYG{l+m+mi}{2}\PYG{p}{)}
\PYG{g+go}{[0.0, 2.0, 4.0, 6.0, 8.0, 10.0, 12.0, 14.0, 16.0, 18.0, 20.0]}
\end{Verbatim}

\item[{Parameters}] \leavevmode\begin{itemize}
\item {} 
\textbf{\texttt{n1}} (\href{https://docs.python.org/2/library/functions.html\#float}{\emph{\texttt{float}}}) -- Minimum number in range.

\item {} 
\textbf{\texttt{n2}} (\href{https://docs.python.org/2/library/functions.html\#float}{\emph{\texttt{float}}}) -- Maximum number in range.

\item {} 
\textbf{\texttt{nc}} (\href{https://docs.python.org/2/library/functions.html\#int}{\emph{\texttt{int}}}) -- Maximum number of intervals

\item {} 
\textbf{\texttt{zero}} (\href{https://docs.python.org/2/library/functions.html\#int}{\emph{\texttt{int}}}) -- 
Integer flag to indicate how zero should be handled. Flags are as follows
-1: zero MUST NOT be a contour
\begin{quote}

0: let the function decide \# NOT IMPLEMENTED
1: zero CAN be a contour  (default)
2: zero MUST be a contour
\end{quote}


\item {} 
\textbf{\texttt{end}} (\href{https://docs.python.org/2/library/functions.html\#bool}{\emph{\texttt{bool}}}) -- Boolean value indicating whether n1 and n2 should be part of the returned labels.
Defaults to False.

\end{itemize}

\item[{Returns}] \leavevmode
List of floats split into nc intervals

\item[{Return type}] \leavevmode
{\hyperref[vcs/graphics/boxfill:vcs.boxfill.Gfb.list]{\sphinxcrossref{list}}}

\end{description}\end{quote}

\end{fulllineitems}

\index{rgba\_color() (in module vcs.utils)}

\begin{fulllineitems}
\phantomsection\label{vcs/misc/utils:vcs.utils.rgba_color}\pysiglinewithargsret{\sphinxcode{vcs.utils.}\sphinxbfcode{rgba\_color}}{\emph{color}, \emph{colormap}}{}
Try all of the various syntaxes of colors and return 0-100 RGBA values.
\begin{quote}\begin{description}
\item[{Example}] \leavevmode
\begin{Verbatim}[commandchars=\\\{\}]
\PYG{g+gp}{\PYGZgt{}\PYGZgt{}\PYGZgt{} }\PYG{n}{cp} \PYG{o}{=} \PYG{n}{vcs}\PYG{o}{.}\PYG{n}{getcolormap}\PYG{p}{(}\PYG{p}{)} \PYG{c+c1}{\PYGZsh{} Get a copy of the default colormap}
\PYG{g+gp}{\PYGZgt{}\PYGZgt{}\PYGZgt{} }\PYG{n}{vcs}\PYG{o}{.}\PYG{n}{rgba\PYGZus{}color}\PYG{p}{(}\PYG{l+s+s1}{\PYGZsq{}}\PYG{l+s+s1}{black}\PYG{l+s+s1}{\PYGZsq{}}\PYG{p}{,} \PYG{n}{cp}\PYG{p}{)} \PYG{c+c1}{\PYGZsh{} Find the rgba equivalent for black}
\PYG{g+go}{[0.0, 0.0, 0.0, 100]}
\end{Verbatim}

\item[{Parameters}] \leavevmode\begin{itemize}
\item {} 
\textbf{\texttt{color}} (\emph{\texttt{int, str}}) -- The color to get the rgba value for. Can be an integer from 0-255, or a string name of a color.

\item {} 
\textbf{\texttt{colormap}} ({\hyperref[vcs/misc/colormap:vcs.colormap.Cp]{\sphinxcrossref{\emph{\texttt{vcs.colormap.Cp}}}}}) -- A VCS colormap

\end{itemize}

\item[{Returns}] \leavevmode
List of 4 floats; the R, G, B, and A values associated with the given color.

\item[{Return type}] \leavevmode
{\hyperref[vcs/graphics/boxfill:vcs.boxfill.Gfb.list]{\sphinxcrossref{list}}}

\end{description}\end{quote}

\end{fulllineitems}

\index{setTicksandLabels() (in module vcs.utils)}

\begin{fulllineitems}
\phantomsection\label{vcs/misc/utils:vcs.utils.setTicksandLabels}\pysiglinewithargsret{\sphinxcode{vcs.utils.}\sphinxbfcode{setTicksandLabels}}{\emph{gm}, \emph{copy\_gm}, \emph{datawc\_x1}, \emph{datawc\_x2}, \emph{datawc\_y1}, \emph{datawc\_y2}, \emph{x=None}, \emph{y=None}}{}
Sets the labels and ticks for a graphics method made in python
\begin{quote}\begin{description}
\item[{Example}] \leavevmode
\item[{Parameters}] \leavevmode\begin{itemize}
\item {} 
\textbf{\texttt{gm}} (\emph{\texttt{VCS graphics method}}) -- A VCS graphics method to alter

\item {} 
\textbf{\texttt{copy\_gm}} (\emph{\texttt{VCS graphics method}}) -- A VCS graphics method object

\item {} 
\textbf{\texttt{datawc\_x1}} (\href{https://docs.python.org/2/library/functions.html\#float}{\emph{\texttt{float}}}) -- Float value to set the graphics method's datawc\_x1 property to.

\item {} 
\textbf{\texttt{datawc\_x2}} (\href{https://docs.python.org/2/library/functions.html\#float}{\emph{\texttt{float}}}) -- Float value to set the graphics method's datawc\_x2 property to.

\item {} 
\textbf{\texttt{datawc\_y1}} (\href{https://docs.python.org/2/library/functions.html\#float}{\emph{\texttt{float}}}) -- Float value to set the graphics method's datawc\_y1 property to.

\item {} 
\textbf{\texttt{datawc\_y2}} (\href{https://docs.python.org/2/library/functions.html\#float}{\emph{\texttt{float}}}) -- Float value to set the graphics method's datawc\_y2 property to.

\item {} 
\textbf{\texttt{x}} (\href{https://docs.python.org/2/library/functions.html\#str}{\emph{\texttt{str}}}) -- If provided, must be the string `longitude'

\item {} 
\textbf{\texttt{y}} (\href{https://docs.python.org/2/library/functions.html\#str}{\emph{\texttt{str}}}) -- If provided, must be the string `latitude'

\end{itemize}

\item[{Returns}] \leavevmode
A VCS graphics method object

\item[{Return type}] \leavevmode
A VCS graphics method object

\end{description}\end{quote}

\end{fulllineitems}

\index{setcolorcell() (in module vcs.utils)}

\begin{fulllineitems}
\phantomsection\label{vcs/misc/utils:vcs.utils.setcolorcell}\pysiglinewithargsret{\sphinxcode{vcs.utils.}\sphinxbfcode{setcolorcell}}{\emph{obj}, \emph{num}, \emph{r}, \emph{g}, \emph{b}, \emph{a=100}}{}
Set a individual color cell in the active colormap. If default is
the active colormap, then return an error string.

\begin{notice}{note}{Note:}
If the the visual display is 16-bit, 24-bit, or 32-bit TrueColor,
then a redrawing
of the VCS Canvas is made every time the color cell is changed.
\end{notice}
\begin{quote}\begin{description}
\item[{Example}] \leavevmode
\begin{Verbatim}[commandchars=\\\{\}]
\PYG{g+gp}{\PYGZgt{}\PYGZgt{}\PYGZgt{} }\PYG{n}{vcs}\PYG{o}{.}\PYG{n}{setcolorcell}\PYG{p}{(}\PYG{l+s+s2}{\PYGZdq{}}\PYG{l+s+s2}{AMIP}\PYG{l+s+s2}{\PYGZdq{}}\PYG{p}{,}\PYG{l+m+mi}{11}\PYG{p}{,}\PYG{l+m+mi}{0}\PYG{p}{,}\PYG{l+m+mi}{0}\PYG{p}{,}\PYG{l+m+mi}{0}\PYG{p}{)}
\PYG{g+gp}{\PYGZgt{}\PYGZgt{}\PYGZgt{} }\PYG{n}{vcs}\PYG{o}{.}\PYG{n}{setcolorcell}\PYG{p}{(}\PYG{l+s+s2}{\PYGZdq{}}\PYG{l+s+s2}{AMIP}\PYG{l+s+s2}{\PYGZdq{}}\PYG{p}{,}\PYG{l+m+mi}{21}\PYG{p}{,}\PYG{l+m+mi}{100}\PYG{p}{,}\PYG{l+m+mi}{0}\PYG{p}{,}\PYG{l+m+mi}{0}\PYG{p}{)}
\PYG{g+gp}{\PYGZgt{}\PYGZgt{}\PYGZgt{} }\PYG{n}{vcs}\PYG{o}{.}\PYG{n}{setcolorcell}\PYG{p}{(}\PYG{l+s+s2}{\PYGZdq{}}\PYG{l+s+s2}{AMIP}\PYG{l+s+s2}{\PYGZdq{}}\PYG{p}{,}\PYG{l+m+mi}{31}\PYG{p}{,}\PYG{l+m+mi}{0}\PYG{p}{,}\PYG{l+m+mi}{100}\PYG{p}{,}\PYG{l+m+mi}{0}\PYG{p}{)}
\PYG{g+gp}{\PYGZgt{}\PYGZgt{}\PYGZgt{} }\PYG{n}{vcs}\PYG{o}{.}\PYG{n}{setcolorcell}\PYG{p}{(}\PYG{l+s+s2}{\PYGZdq{}}\PYG{l+s+s2}{AMIP}\PYG{l+s+s2}{\PYGZdq{}}\PYG{p}{,}\PYG{l+m+mi}{41}\PYG{p}{,}\PYG{l+m+mi}{0}\PYG{p}{,}\PYG{l+m+mi}{0}\PYG{p}{,}\PYG{l+m+mi}{100}\PYG{p}{)}
\PYG{g+gp}{\PYGZgt{}\PYGZgt{}\PYGZgt{} }\PYG{n}{vcs}\PYG{o}{.}\PYG{n}{setcolorcell}\PYG{p}{(}\PYG{l+s+s2}{\PYGZdq{}}\PYG{l+s+s2}{AMIP}\PYG{l+s+s2}{\PYGZdq{}}\PYG{p}{,}\PYG{l+m+mi}{51}\PYG{p}{,}\PYG{l+m+mi}{100}\PYG{p}{,}\PYG{l+m+mi}{100}\PYG{p}{,}\PYG{l+m+mi}{100}\PYG{p}{)}
\PYG{g+gp}{\PYGZgt{}\PYGZgt{}\PYGZgt{} }\PYG{n}{vcs}\PYG{o}{.}\PYG{n}{setcolorcell}\PYG{p}{(}\PYG{l+s+s2}{\PYGZdq{}}\PYG{l+s+s2}{AMIP}\PYG{l+s+s2}{\PYGZdq{}}\PYG{p}{,}\PYG{l+m+mi}{61}\PYG{p}{,}\PYG{l+m+mi}{70}\PYG{p}{,}\PYG{l+m+mi}{70}\PYG{p}{,}\PYG{l+m+mi}{70}\PYG{p}{)}
\end{Verbatim}

\item[{Parameters}] \leavevmode\begin{itemize}
\item {} 
\textbf{\texttt{obj}} (\emph{\texttt{str or VCS object}}) -- String name of a colormap, or a VCS object

\item {} 
\textbf{\texttt{num}} (\href{https://docs.python.org/2/library/functions.html\#int}{\emph{\texttt{int}}}) -- Integer specifying which color cell to change. Must be from 0-239.

\item {} 
\textbf{\texttt{r}} (\href{https://docs.python.org/2/library/functions.html\#int}{\emph{\texttt{int}}}) -- Integer specifying the red value for the colorcell

\item {} 
\textbf{\texttt{g}} (\href{https://docs.python.org/2/library/functions.html\#int}{\emph{\texttt{int}}}) -- Integer specifying the green value for the colorcell

\item {} 
\textbf{\texttt{b}} (\href{https://docs.python.org/2/library/functions.html\#int}{\emph{\texttt{int}}}) -- Integer specifying the blue value for the colorcell

\item {} 
\textbf{\texttt{a}} (\href{https://docs.python.org/2/library/functions.html\#int}{\emph{\texttt{int}}}) -- Integer specifying the opacity value for the colorcell. Must be from 0-100.

\end{itemize}

\end{description}\end{quote}

\end{fulllineitems}

\index{show() (in module vcs.utils)}

\begin{fulllineitems}
\phantomsection\label{vcs/misc/utils:vcs.utils.show}\pysiglinewithargsret{\sphinxcode{vcs.utils.}\sphinxbfcode{show}}{\emph{*args}}{}
Show the list of VCS primary and secondary class objects.
\begin{quote}\begin{description}
\item[{Example}] \leavevmode
\begin{Verbatim}[commandchars=\\\{\}]
\PYG{g+gp}{\PYGZgt{}\PYGZgt{}\PYGZgt{} }\PYG{n}{a}\PYG{o}{=}\PYG{n}{vcs}\PYG{o}{.}\PYG{n}{init}\PYG{p}{(}\PYG{p}{)} \PYG{c+c1}{\PYGZsh{} Create a VCS Canvas instance, named \PYGZsq{}a\PYGZsq{}}
\PYG{g+gp}{\PYGZgt{}\PYGZgt{}\PYGZgt{} }\PYG{n}{a}\PYG{o}{.}\PYG{n}{show}\PYG{p}{(}\PYG{l+s+s1}{\PYGZsq{}}\PYG{l+s+s1}{boxfill}\PYG{l+s+s1}{\PYGZsq{}}\PYG{p}{)} \PYG{c+c1}{\PYGZsh{} List boxfill objects on Canvas \PYGZsq{}a\PYGZsq{}}
\PYG{g+gp}{\PYGZgt{}\PYGZgt{}\PYGZgt{} }\PYG{n}{a}\PYG{o}{.}\PYG{n}{show}\PYG{p}{(}\PYG{l+s+s1}{\PYGZsq{}}\PYG{l+s+s1}{isofill}\PYG{l+s+s1}{\PYGZsq{}}\PYG{p}{)} \PYG{c+c1}{\PYGZsh{} List isofill objects on Canvas \PYGZsq{}a\PYGZsq{}}
\PYG{g+gp}{\PYGZgt{}\PYGZgt{}\PYGZgt{} }\PYG{n}{a}\PYG{o}{.}\PYG{n}{show}\PYG{p}{(}\PYG{l+s+s1}{\PYGZsq{}}\PYG{l+s+s1}{line}\PYG{l+s+s1}{\PYGZsq{}}\PYG{p}{)} \PYG{c+c1}{\PYGZsh{} List line objects on Canvas \PYGZsq{}a\PYGZsq{}}
\PYG{g+gp}{\PYGZgt{}\PYGZgt{}\PYGZgt{} }\PYG{n}{a}\PYG{o}{.}\PYG{n}{show}\PYG{p}{(}\PYG{l+s+s1}{\PYGZsq{}}\PYG{l+s+s1}{marker}\PYG{l+s+s1}{\PYGZsq{}}\PYG{p}{)} \PYG{c+c1}{\PYGZsh{} List marker objects on Canvas \PYGZsq{}a\PYGZsq{}}
\PYG{g+gp}{\PYGZgt{}\PYGZgt{}\PYGZgt{} }\PYG{n}{a}\PYG{o}{.}\PYG{n}{show}\PYG{p}{(}\PYG{l+s+s1}{\PYGZsq{}}\PYG{l+s+s1}{text}\PYG{l+s+s1}{\PYGZsq{}}\PYG{p}{)} \PYG{c+c1}{\PYGZsh{} List text objects on Canvas \PYGZsq{}a\PYGZsq{}}
\end{Verbatim}

\end{description}\end{quote}

\end{fulllineitems}



\subsection{vcshelp}
\label{vcs/misc/vcshelp::doc}\label{vcs/misc/vcshelp:vcshelp}\label{vcs/misc/vcshelp:module-vcs.vcshelp}\index{vcs.vcshelp (module)}
\# VCS help module
\index{objecthelp() (in module vcs.vcshelp)}

\begin{fulllineitems}
\phantomsection\label{vcs/misc/vcshelp:vcs.vcshelp.objecthelp}\pysiglinewithargsret{\sphinxcode{vcs.vcshelp.}\sphinxbfcode{objecthelp}}{\emph{*arg}}{}
Print the documentation of each object in the argument list.
Prints a blank line if no documentation.
\begin{quote}\begin{description}
\item[{Example}] \leavevmode
\begin{Verbatim}[commandchars=\\\{\}]
\PYG{g+gp}{\PYGZgt{}\PYGZgt{}\PYGZgt{} }\PYG{n}{objects} \PYG{o}{=} \PYG{p}{[} \PYG{n}{vcs}\PYG{o}{.}\PYG{n}{get3d\PYGZus{}scalar}\PYG{p}{(}\PYG{p}{)}\PYG{p}{,} \PYG{n}{vcs}\PYG{o}{.}\PYG{n}{getcolormap}\PYG{p}{(}\PYG{p}{)}\PYG{p}{,} \PYG{n}{vcs}\PYG{o}{.}\PYG{n}{getboxfill}\PYG{p}{(}\PYG{p}{)} \PYG{p}{]}
\PYG{g+gp}{\PYGZgt{}\PYGZgt{}\PYGZgt{} }\PYG{k}{for} \PYG{n+nb}{object} \PYG{o+ow}{in} \PYG{n}{objects}\PYG{p}{:}
\PYG{g+gp}{... }    \PYG{n}{vcs}\PYG{o}{.}\PYG{n}{objecthelp}\PYG{p}{(}\PYG{n+nb}{object}\PYG{p}{)}
\end{Verbatim}

\item[{Parameters}] \leavevmode
\textbf{\texttt{arg}} (\emph{\texttt{VCS object, or list of vcs objects}}) -- Instance(s) of VCS object(s) to display the documentation for.
Multiple objects should be comma-delimited.

\end{description}\end{quote}

\end{fulllineitems}



\renewcommand{\indexname}{Python Module Index}
\begin{theindex}
\def\bigletter#1{{\Large\sffamily#1}\nopagebreak\vspace{1mm}}
\bigletter{v}
\item {\texttt{vcs}}, \pageref{vcs/vcs:module-vcs}
\item {\texttt{vcs.animate\_helper}}, \pageref{vcs/misc/animate_helper:module-vcs.animate_helper}
\item {\texttt{vcs.boxfill}}, \pageref{vcs/graphics/boxfill:module-vcs.boxfill}
\item {\texttt{vcs.Canvas}}, \pageref{vcs/Canvas:module-vcs.Canvas}
\item {\texttt{vcs.colormap}}, \pageref{vcs/misc/colormap:module-vcs.colormap}
\item {\texttt{vcs.colors}}, \pageref{vcs/misc/colors:module-vcs.colors}
\item {\texttt{vcs.displayplot}}, \pageref{vcs/misc/displayplot:module-vcs.displayplot}
\item {\texttt{vcs.dv3d}}, \pageref{vcs/graphics/dv3d:module-vcs.dv3d}
\item {\texttt{vcs.error}}, \pageref{vcs/misc/error:module-vcs.error}
\item {\texttt{vcs.fillarea}}, \pageref{vcs/secondary/fillarea:module-vcs.fillarea}
\item {\texttt{vcs.isofill}}, \pageref{vcs/graphics/isofill:module-vcs.isofill}
\item {\texttt{vcs.isoline}}, \pageref{vcs/graphics/isoline:module-vcs.isoline}
\item {\texttt{vcs.line}}, \pageref{vcs/secondary/line:module-vcs.line}
\item {\texttt{vcs.manageElements}}, \pageref{vcs/misc/manageElements:module-vcs.manageElements}
\item {\texttt{vcs.marker}}, \pageref{vcs/secondary/marker:module-vcs.marker}
\item {\texttt{vcs.meshfill}}, \pageref{vcs/graphics/meshfill:module-vcs.meshfill}
\item {\texttt{vcs.Pboxeslines}}, \pageref{vcs/template/Pboxeslines:module-vcs.Pboxeslines}
\item {\texttt{vcs.Pdata}}, \pageref{vcs/template/Pdata:module-vcs.Pdata}
\item {\texttt{vcs.Pformat}}, \pageref{vcs/template/Pformat:module-vcs.Pformat}
\item {\texttt{vcs.Plegend}}, \pageref{vcs/template/Plegend:module-vcs.Plegend}
\item {\texttt{vcs.projection}}, \pageref{vcs/misc/projection:module-vcs.projection}
\item {\texttt{vcs.Ptext}}, \pageref{vcs/template/Ptext:module-vcs.Ptext}
\item {\texttt{vcs.Pxlabels}}, \pageref{vcs/template/Pxlabels:module-vcs.Pxlabels}
\item {\texttt{vcs.Pxtickmarks}}, \pageref{vcs/template/Pxtickmarks:module-vcs.Pxtickmarks}
\item {\texttt{vcs.Pylabels}}, \pageref{vcs/template/Pylabels:module-vcs.Pylabels}
\item {\texttt{vcs.Pytickmarks}}, \pageref{vcs/template/Pytickmarks:module-vcs.Pytickmarks}
\item {\texttt{vcs.queries}}, \pageref{vcs/misc/queries:module-vcs.queries}
\item {\texttt{vcs.taylor}}, \pageref{vcs/graphics/taylor:module-vcs.taylor}
\item {\texttt{vcs.template}}, \pageref{vcs/template/template:module-vcs.template}
\item {\texttt{vcs.textcombined}}, \pageref{vcs/secondary/textcombined:module-vcs.textcombined}
\item {\texttt{vcs.textorientation}}, \pageref{vcs/secondary/textorientation:module-vcs.textorientation}
\item {\texttt{vcs.texttable}}, \pageref{vcs/secondary/texttable:module-vcs.texttable}
\item {\texttt{vcs.unified1D}}, \pageref{vcs/graphics/unified1D:module-vcs.unified1D}
\item {\texttt{vcs.utils}}, \pageref{vcs/misc/utils:module-vcs.utils}
\item {\texttt{vcs.vcshelp}}, \pageref{vcs/misc/vcshelp:module-vcs.vcshelp}
\item {\texttt{vcs.vector}}, \pageref{vcs/graphics/vector:module-vcs.vector}
\end{theindex}

\renewcommand{\indexname}{Index}
\printindex
\end{document}
